\documentclass[eric_thesis.tex]{subfiles}
\begin{document}
\frontmatter
\maketitle

\chapter{Abstract}
Vector-based lexical semantics is a powerful technique that still has many 
undiscovered applications. In this thesis I apply a vector-space 
lexical-semantic model newly developed by Mikolov et. al. trained on 
skip-grams to the lexical hypothesis in personality psychology. The method
produces interpretable dimensions that are consistent across several sets of
descriptive personality words. The dimensions include ones for conflict and
positive and negative evaluation. However they are more descriptive of
word usage semantics than of the characteristics of the thing described and 
thus do not include a recognizable component of the 5 factor model in their 
first 14 dimensions. They do include a component that seems to indicate the 
degree to which the word applies to people which could be useful in identifying
personality words in English.


\mainmatter
\chapter{Introduction}



When the ghost of Jacob Marley confronted Ebenezer Scrooge in ``A Christmas Carol,'' Scrooge tried to justify his former partner by saying ``But you were always a good man of business, Jacob!'' Marley responded with the anguished cry, ``Business! Mankind was my business. \ldots The dealings of my trade were but a drop of water in the comprehensive ocean of my business!'' To those who wish that their dealings this life would benefit not only the easily modeled \textit{homo economicus} but also the rest of our irrational species, it is necessary to understand people. This is the domain of the humanities and social sciences. Among its students are economists, political scientists, linguists, historians, sociologists, psychologists, and anthropologists. Of these, the ones who attempt to understand the mental functions of people are the psychologists.

Finding the invariants in any particular field has been an incredibly successful method of study through the ages. The mental invariants that characterize a person over periods of time are that person's personality. Since the 1980's, many personality psychologists have begun using trait models derived from factor analysis of people's usage of language in describing others. The most famous of these is the Five Factor Model measured by McCrae and Costa's NEO inventory. It measures personality along the five dimensions of Openness to experience, Conscientiousness, Extraversion, Agreeableness, and Neuroticism (which form the mnemonic OCEAN). Personality trait models like the Five Factor Model are used in a wide variety of contexts. They are used in dating sites, career counseling, management, clinical psychology and school adjustment.

These personality factors come out of turning people's descriptions of others into vectors using questionnaire. Each vector dimension corresponds to rating the person on one aspect of personality. People's descriptions on hundreds of adjectives can be predicted well by their ratings on only the 5 OCEAN dimensions. For each adjective, there are 5 values that, when multiplied by the 5 model dimensions, give the prediction of a person's rating for that dimension. These 5 values are a vector. So the 5 factor model turns adjectives into vectors. The components of the vectors are the semantic contributions of each of the model dimensions to that vector's adjective.

In a completely different field, there is another way of turning words into vectors. In the 1990's, techniques like Latent Semantic Analysis (LSA) were developed to turn bodies of text into vectors that in some sense approximated the meaning of words - and do it with very little \textit{a-priori} knowledge. The early techniques only captured a small part of word meaning. Today, the vectors generated by skip-gram models trained on only their own language can be used to translate between different languages with reasonable accuracy given only a very small set of word correspondences. Such demonstrations imply that it is not just the syntactic structure of a language that is being captured but its meaning as well.

Since there is a source for vectors indicating the meaning of different words. It seems reasonable that the most important components of the meaning of personality words would be the factors that make up personality as described by humans. So, I set up an experiment to look at the vectors for personality words in a skip-gram model. On setting it up, I believed that it was likely I would uncover the Five Factor Model (or one of its competitors) in the data. However, the results did not support that hypothesis.

\chapter{Background}

\section{Vector-based lexical semantics}

\subsection{LSA}

Vector-based lexical semantics is the study of algorithms that assign vectors to words in a way that reflects the meaning of those words. The first vector-based lexical semantic algorithm was LSA (Lexical Semantic Analysis), invented in \todo{some-year} by \todo{some-person}. LSA works off of a bag-of-words model of source documents. The training documents are turned into a matrix in which each row corresponds to a document, each column to one word in the vocabulary, and each entry counts the number of occurrences of a particular word in the document. It is important to recognize that a document can be a group of words of any length. Documents could consist of sentences, paragraphs, web-pages, 20 word sliding windows, bible chapters, or any other textual unit that is of interest in the application. This term-document matrix is then decomposed into its principal components and their loadings by SVD (singular value decomposition) and the most significant components are chosen. If the resulting reduced matrices are multiplied together, it produces a smoothed term-document matrix guaranteed to be the closest one can come to the original matrix (in the least-squares sense) with the chosen number of factors.

\todo{include section on how to use LSA for document-query similarity and word-word similarity}

Any modern application of LSA is usually more complicated. For example, the raw counts are usually transformed by the TF/IDF (term-frequency/inverse document frequency) transformation where the counts are replaced by the log of the count in that document divided by the mean count. \todo{ref and check procedure, esp what happens at 0}. This procedure was originally justified on the heuristic grounds that it emphasized words that better distinguished between documents. \todo{ref} Later, a probabilistic justification was discovered. \todo{ref} There are also methods for choosing which words to include in the vocabulary \todo{ref}, modifying the words for better retrieval (stemming) \todo{ref}, choosing the number of dimensions to keep \todo{ref}, and many other refinements to the technique.

LSA came out of the document retrieval field and its first applications were in matching query strings to documents. It was very successful in finding documents that were semantically appropriate but which contained no words in common with the query. For example, with the right training set, "The legislature will meet in Columbus on Thursday for a special session." would be a good match for the query "Ohio capital" because Ohio and capital are both frequently in the same documents with Columbus, legislature, meet, and session.

\subsection{PLSA}

\subsection{LDA}

\subsection{N-word context models}

\subsection{Prototype-based models}

\subsection{Skip-gram Model}

\section{Applications of Vector-based lexical semantics}


\section{Lexical Hypothesis in Personality Psychology}


The lexical hypothesis in psychology can be summarized as: the aspects of personality that people find important will be reflected in people's way of describing others. This has mainly been

\section{Part-of-speech tagging}

\section{Multidimensional Scaling}

\section{PCA and Factor analysis}

\chapter{Related Work}

\section{Skip gram model}

\section{Finding personality dimensions}

\end{document}
