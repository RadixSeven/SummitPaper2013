\appendix
\chapter{Scripts used in preprocessing}
\section{Text of filter\_angle\_tags.pl}
\label{app:filterangletags}
\lstset{language=Perl}
\begin{lstlisting}
#!/usr/bin/perl
use strict;
use warnings;

#######################
# Usage: filter_angle_tags.pl < source > dest
# or:    filter_angle_tags.pl source1 source2 > filtered
#
# Remove stray html, urls and stock symbols to prepare
# for tagging

my $has_bracket_pair = qr!<[A-z/][^<>]*>!;
my $bracket_pair_split = qr!($has_bracket_pair)!;
my $html_string = q(<span |<strong>|</strong>|).
    q(<em>|</em>|<[bauipq]>|</[pqbaui]>|<\s*br\s*/?>|).
    q(<\s*/br\s*>|<p\s*/?>|</?h.>|</?blockquote|).
    q(</?code>|<font[ >]|</font|</?acronym|</?strike|).
    q(</?cite|</?q |</>|</?abbr |</?del |<javascript|).
    q(</?itals>|</?script|</?center>|<SPEAKER );
my $html_regex = qr(${html_string});

sub keep_segment{
    my ($seg) = @_;
    return !($seg =~ m!$has_bracket_pair!) or !(
        $seg =~ m(http://) or
        $seg =~ m(href) or
        $seg =~ m(\.com\W|\.gov\W) or
        $seg =~ m(${html_regex}) or
        $seg =~ m(<[A-z0-9]{0,9}[.=/][A-Z0-9]{0,6}>) or
        $seg =~ m(<CF7[01]>) or
        $seg =~ m(<ID:[A-z0-9]+>)
        );
}

while(<>){
    chomp;
    if (m/$has_bracket_pair/) {
        my @segments = split($bracket_pair_split, $_);
        @segments = grep( &keep_segment($_), 
                          @segments);
        print join(" ", @segments),"\n" 
            if (@segments);
    }else{
        print "$_\n";
    }

}
\end{lstlisting}

\section{Text of tag\_corpus.sh}
\label{app:tagcorpus}
\lstset{language=sh}
\begin{lstlisting}
#!/bin/bash

############
# Script I used to tag the WMT11 corpus - it needed to
# be split because the TreeTagger tokenizer reads
# everything into memory.

# The data file for the cleaned data
WMT11=/mnt/linux_data/wmt11_no_angle_tags.txt

# The prefix used for split files
SPLIT_FILENAME=/mnt/linux_data/wmt11_no_angle_split

# Data file for the tagged data
TAGGED_WMT11=/mnt/linux_data/wmt11_tagged.txt

# Put treetagger on path
PATH="$PATH:/home/eric/SW/TreeTagger/cmd:"\
"/home/eric/SW/TreeTagger/bin:."

# Split into 10,000,000 lines per file (about 1.5GiB
# per)
if [ -f ${SPLIT_FILENAME}aa ]; then 
    echo "Not splitting again. Split already exists"
else
    split -l10000000 "$WMT11" "$SPLIT_FILENAME"
fi

# Tag the split files
if [ -f "$TAGGED_WMT11" ]; then 
    echo "Not tagging again. Tagged file "\
         "already exists"
else
    for i in ${SPLIT_FILENAME}*; do
        echo "Tagging $i"
        tree-tagger-english-utf8 $i | \
            reassemble_tags.pl >> "$TAGGED_WMT11"
    done
fi
\end{lstlisting}

\section{Text of tag\_word\_list.pl}
\label{app:tagwordlist}
\lstset{language=Perl}
\begin{lstlisting}
#!/usr/bin/perl
use strict;
use warnings;
use HotKey;
use URI::Escape;
use File::Fetch;

#
# Reads a word-list from a file (DON'T USE STDIN, it
# mistakes it for keyboard input). For each word,
# displays the word-net entry web-page in firefox and
# then asks if it is a valid as a personality
# adjective. Next it asks if it is valid as a
# personality non-adjective. If it was a valid
# adjective, it writes word_jj to stdout. If it was a
# valid non-adjective, it writes word to stdout.  If
# you type q at any prompt, it exits, writing "THERE
# WAS A PROBLEM" to stdout.
#
# Using the assumption that every word in the file is a
# personality word, does the disambiguation itself if
# the word has only one part of speech. (However, this
# code depends on the format of the html returned and
# so is fragile.)
#
# Prompts are written to stderr (which is fine on
# Linux, but probably a mess under Windows)

sub getynq {
    my $key = readkey();
    while($key =~ m/^[^ynq]$/){
        print STDERR "\nPlease enter y,n, or q.";
        $key = readkey();
    }
    return $key;
}

WORD: while(<>){
    chomp;

    my $query_uri =
        'http://wordnetweb.princeton.edu/perl/webwn?s='.
        uri_escape($_).
        '&sub=Search+WordNet&o2=&o0=1'.
        '&o7=&o5=&o1=1&o6=&o4=&o3=&h=';
    # Try automatic determination - if there is only
    # one level three header (used to contain the part
    # of speech) then if it is "Adjective" the word is
    # a personality adjective. Otherwise, the word is a
    # personality non-adjective. Ignore verb and adverb
    # meanings because I am looking for words about
    # personal qualities. I want "The brave man" or "he
    # had courage" and not "He will brave the elements"
    # or "Home of the brave".
    #
    # With the 2797 word list, I have decided to accept
    # words that are categories - like "the home of the
    # brave."
    my $contents;
    my $fetcher = File::Fetch->new(uri => $query_uri);
    $fetcher->fetch(to => \$contents);

    my @level_3_headers = ($contents =~
                         m:<h3>[^<]+</h3>:gi);

    @level_3_headers = # Remove verbs
        grep{!m:<h3>Verb</h3>:} @level_3_headers; 
    @level_3_headers = # Remove adverbs
        grep{!m:<h3>Adverb</h3>:} @level_3_headers; 

    my $autotag_successful = 0;
    if (@level_3_headers == 1){
        # If the only part of speech is adjective tag
        # as such, otherwise if the only part of speech
        # is noun, tag as non-adjective. If the part of
        # speech is something else (an unknown part of
        # speech or an error message) don't tag and
        # leave it to the humans.
        if($level_3_headers[0] =~ 
              m:<h3>Adjective</h3>:i){
            print STDOUT "${_}_jj\n";
            $autotag_successful = 1;
        }elsif($level_3_headers[0] =~ 
              m:<h3>Noun</h3>:i){
            print STDOUT "${_}\n";
            $autotag_successful = 1;
        }else{
            $autotag_successful = 0;
        }
    }
    if ($autotag_successful){
        # We've tagged the word. Read the next word
        print STDERR "Automatically ",
                     "determined ${_}.\n";
        next WORD;
    }


    # Manual determination - open firefox (ignoring
    # error messages) and ask the user

    system("firefox \"".
           "$query_uri\" > /dev/null 2>&1 &");

    my $had_to_do_with_personality=0;
    print STDERR "\nIs $_ a valid personality ".
        "adjective? (yes/no/quit)";
    my $key = getynq();
    if($key =~ m/y/i){
        print STDOUT "${_}_jj\n";
        $had_to_do_with_personality = 1;
    }elsif($key =~ m/q/i){
        print STDOUT "THERE WAS A PROBLEM\n";
        exit(-1);
    }

    print STDERR "\nIs $_ a valid personality ",
        "quality or type of person ('His ",
        "characteristic $_' or 'The $_ is ",
        "characterized ",
        "by $_-ness')? Only accept for type of ",
        "person if ",
        "the word is not an adjective or a much more ",
        "common usage than the adjective. (braggart, ",
        "for ",
        "example, has an archaic adjectival use but ",
        "is ",
        "much more commonly used as a ",
        "category)(yes/no/quit)"; 
    $key = getynq();
    if($key =~ m/y/i){ 
        print STDOUT "${_}\n";
        $had_to_do_with_personality = 1; 
    }elsif($key =~ m/q/i){ 
        print STDOUT "THERE WAS A PROBLEM\n";
        exit(-1); 
    }

    unless($had_to_do_with_personality){
        print STDERR "\n$_ must have something to ",
             "do with personality. You can't ",
             "answer n on both questions.\n";
        print STDOUT "THERE WAS A PROBLEM\n";
        exit(-1);
    }
} 
\end{lstlisting}


\section{Text of reassemble\_tags.pl}
\label{app:reassembletags}
\lstset{language=Perl}
\begin{lstlisting}
#!/usr/bin/perl
use strict;
use warnings;

################
# Usage: reassemble_tags.pl < tagged.txt
#
# Takes tagged text from treetagger and reassembles it
# into 1 sentence per line (multiple sentence ending
# tags are kept on the same line)

# Tracks whether the previous tag was a sentence-ender
# in order to tell where to end the lines.
my $last_tag_was_sentence=(1==0);

# Combine words and sentences
while(<>){
    my ($word, $tag) = split('\t');
    if(defined($tag)){
        # End the line if transitioning from "SENT" to
        # another tag
        my $tag_is_sentence = $tag eq "SENT";
        if ($last_tag_was_sentence && !$tag_is_sentence){
            print "\n";
        }

        # Output the word with its tag
        if(defined($word)){
            print "${word}_${tag} ";
        }else{
            print "${word}_unknown_tag ";
        }

        # Update last tag state variable
        $last_tag_was_sentence = $tag_is_sentence;
    }
}

# If the last tag in the file was a sentence tag,
# output a newline
if ($last_tag_was_sentence){
    print "\n";
}
\end{lstlisting}

\chapter{Scripts used in analysis}

\section{extract\_vectors.py}
\label{app:extractvectors}
\lstset{language=Python}
\begin{lstlisting}
#!/usr/bin/python
from gensim.models import word2vec
import logging
import os
import csv
import sys


# Check command line arguments and print usage if
# wrong number of arguments
if len(sys.argv) != 3:
    import textwrap
    sys.stderr.write(
        "Usage: %s list_of_terms.txt "+
        "model_file.model\n" % sys.argv[0])
    sys.stderr.write("\n")
    sys.stderr.write(
        textwrap.fill(
            "Each line in list_of_terms.txt is "+
            "treated as a term. For each term, if a "+
            "corresponding term exists in the "+
            "word2vec model (generated by "+
            "word2vec.save) then that term and the "+
            "corresponding vector are printed to "+
            "stdout as a csv file"))
    sys.exit(-1)

# Command line arguments
term_list_filename = sys.argv[1]
input_model_filename = sys.argv[2]

# Set up logging
logging.basicConfig(
    format='%(asctime)s : %(levelname)s : %(message)s', 
    level=logging.INFO)

# Load the model
model = word2vec.Word2Vec.load(input_model_filename)
logging.info("Done loading input model")

# Loop through the terms, outputting appropriate
# vectors
out = csv.writer(sys.stdout)
with open(term_list_filename) as terms:
    for term in terms:
        try:
            term = term.rstrip() # Remove line-ending
            v = model.vocab[term]
            vector = model.syn0[v.index]
            vector_str = ["%.19g" % n for n in vector]
            out.writerow([term] + vector_str)
        except KeyError:
            logging.info('Skipped missing term "%s"' % 
                         term)
\end{lstlisting}


\section{Elbow point algorithms}
\label{app:elbow_point_algorithms}
\lstset{language=Matlab}

\subsection{elbow\_point.m}

\begin{lstlisting}
 function [elbow_index,best_estimate] = ...
      elbow_point(lambdas)
% Returns the index at which the slope of the lambdas
% changes
%
% In using a scree plot to discover the number of 
% principal components, a frequent method is to choose
% the "elbow", the inflection point of the curve. One
% way of formalizing this is approximating the scree 
% plot with two lines. One goes from the  beginning to
% the elbow point and the other to the end. Whichever
% is closest is the best elbow point 
%
% Exhaustive search for the best approximation of
% the function composed of the ordered pairs
% i,lambdas(i) composed of the lines
% (1,lambdas(1))...(i,lambdas(i)) and
% ((i,lambdas(i)...(end, lambdas(end))
%
% Mean absolute value of the difference is used as
% the measure of goodness of fit
%
% best_estimate - the linear fit to the lambdas
% implied by elbow_index

if size(lambdas,1) > 1
    lambdas = lambdas';
end
best = 1;
best_error = inf;
best_estimate = [];
for cur=1:length(lambdas)
    cur_estimate_a = linspace(lambdas(1), ...
         lambdas(cur), cur);
    cur_estimate_b = linspace(lambdas(cur), ...
         lambdas(end), length(lambdas)-cur+1);
    cur_estimate = [cur_estimate_a(1:end-1), ...
         cur_estimate_b];
    cur_error=mean(abs(cur_estimate - lambdas));
    if cur_error < best_error
        best_error = cur_error;
        best = cur;
        best_estimate = cur_estimate;
    end
end

elbow_index = best;

end
\end{lstlisting}

\subsection{flex\_end\_elbow\_point.m}
\begin{lstlisting}
 function [elbow_index,best_estimate] = ...
      flex_end_elbow_point(lambdas)
% Returns the index at which the slope of the lambdas
% changes
%
% In using a scree plot to discover the number of
% principal components, a frequent method is to
% choose the "elbow", the inflection point of the
% curve. One way of formalizing this is approximating
% the scree plot with two lines. One goes from the
% beginning to the elbow point and the other to the
% end. Whichever is closest is the best elbow point.
%
% This code is similar to elbow_point except that the
% height at the ends is adjustable. This frequently
% leads to a more conservative estimate of when the
% slope changes and a closer approximation of the
% lambdas. I think it is more like what humans do than
% offset_elbow_point.
%
% Exhaustive search for the best approximation of the
% function composed of the ordered pairs i,lambdas(i)
% composed of the lines (1,height_1)...(i,lambdas(i))
% and ((i,lambdas(i)...(end, height_end)
%
% Mean squared difference is used as the measure of
% goodness of fit. (It looks more like what a human
% would do than the mean abs used in other
% elbow_point versions.)
%
% best_estimate - the linear fit to the lambdas
% implied by elbow_index

if size(lambdas,1) > 1
    lambdas = lambdas';
end
best = 1;
best_error = inf;
best_estimate = [];
for cur=1:length(lambdas)
    % Since the two lines are independent, we can
    % minimize them in sequence. I minimize the error
    % on the first line, holding the second constant
    % then minimize the second line holding the first
    % at the optimum found in the first pass
    [h1, ~]=fminbnd(@(h) linpredict([h,lambdas(end)]),...
         lambdas(cur), max(lambdas));
    [~, cur_error]=fminbnd(@(h) linpredict([h1,h]),...
         -max(lambdas), lambdas(cur));
    if cur_error < best_error
        best_error = cur_error;
        best = cur;
        best_estimate = cur_estimate;
    end
end

elbow_index = best;

    % Calculate the two-line fit with the elbow_index
    % at cur and the heights at the beginning and end
    % set to heights(1) and heights(2) Also sets
    % cur_estimate and uses lambdas
    function err=linpredict(heights)
        cur_estimate_a = linspace(heights(1),...
             lambdas(cur), cur);
        cur_estimate_b = linspace(lambdas(cur), ...
             heights(2), length(lambdas)-cur+1);
        cur_estimate = [cur_estimate_a(1:end-1),...
             cur_estimate_b];
        err=mean(abs(cur_estimate - lambdas).^2);
    end

end
\end{lstlisting}

\subsection{log\_scree\_elbow.m}
\begin{lstlisting}
function [elbow_index,best_x1, best_x2, predicted] = ...
     log_scree_elbow( lambdas, min_dist )
% Looks for the line in the log(lambdas) that has the
% most inliers, then counts the first outlier as the
% elbow
%
% Scree plots of random stuff look like exponential
% curves. Thus, they are flat on a log scale. The
% dimensions carrying information either have a
% different slope or aren't exponential at all. Thus,
% they will be outliers. On a log plot, the random
% stuff looks like a straight line. (However, if
% the system is rank deficient, the last few principal
% components will sharply trail off and not look like
% a straight line.)
% 
% Let lambdas be the 'latent' return from princomp.
% 
% I fit an exponential to the end-points of all
% sufficiently large sub-intervals of the lambda
% values. I choose the interval whose error looks the
% most like Gaussian noise (that is, has the lowest
% test statistic for a Jarque-Bera). Then I fit an
% exponential to all those points and calculate the
% standard deviation of the errors. Finally, the
% first point with an error less than 3 standard
% deviations, I count as the first random point. All
% points before that are outliers and I count the
% last of the outlier points as my elbow index.
%
% lambdas - a vector of strictly postitive scalars
%     sorted in decreasing order. The 'latent'
%     return from princomp
%
% min_dist - the minimum length interval to be
%     considered as a potential "all-random" interval.
%     The smaller this is, the less power the
%     normality test has. However, it cannot be larger
%     than the number of random values that are
%     decreasing exponentially. The default is
%     length(lambdas)/4

if any(lambdas <= 0)
    error('log_scree_elbow:pos_lambda',...
        ['All lambda values must be strictly '...
        'positive in log_scree_elbow']);
end

if ~exist('min_dist','var')
    min_dist = ceil(length(lambdas)/4);
end

if min_dist < 2
    error('log_scree_elbow:min_dist_too_small',...
        ['min_dist parameter to log_scree_elbow '...
        'must be at least 2.']);
end
        

if length(lambdas)-min_dist < 20 % This is a fudge
                  % factor, but if it is less than 
                  % 20, there are way too few points
    error('log_scree_elbow:not_enough_points',...
        ['There must be at least 20 more lambda '...
        'values than min distance']);
end

if size(lambdas, 1) > 1
    lambdas = lambdas';
end

% Loop two indices x1 and x2 through the lambdas,
% fitting an exponential and finding outliers
n=length(lambdas);
best_x1 = 1;
best_x2 = 1+min_dist;
best_normality_test_stat = inf;
warning('off','stats:jbtest:PTooBig');
warning('off','stats:jbtest:PTooSmall');
for x1=1:n
    y1=lambdas(x1);
    for x2=(x1+min_dist+1):n
        % Fit an exponential to these two points
        y2=lambdas(x2);
        b=(-log(y1)+log(y2))/(-x1+x2);
        a=y1/exp(b*x1);

        % Calculate the errors for that exponential
        predicted=a.*exp(b.*(1:n));
        errors = lambdas-predicted;

        % Calculate the degree to which the errors in
        % that interval match the normal distribution.
        % Ideally, I'd use the Anderson-Darling test
        % statistic because I am particularly
        % interested in excluding outliers from the
        % interval and it is more sensitive to them.
        % However, it is not available in R2012, so
        % I will use the Jauques-Berra test
        is_inlier = (1:n > x1) & (1:n < x2);
        [~,~,jbtest_stat]=jbtest(errors(is_inlier));
        inlier_std = std(errors(is_inlier));
        if isnan(inlier_std)
            fprintf('Nan');
        end
        % If this section looks more normal than any
        % others, call it the best
        if jbtest_stat < best_normality_test_stat
            best_normality_test_stat = jbtest_stat;
            best_x1 = x1;
            best_x2 = x2;
        end
    end
end
warning('on','stats:jbtest:PTooSmall');
warning('on','stats:jbtest:PTooBig');


% Do a least-squares fit on all the inliers
x1 = best_x1;
x2 = best_x2;
y1=lambdas(x1);
y2=lambdas(x2);
b=(-log(y1)+log(y2))/(-x1+x2);
a=y1/exp(b*x1);
is_inlier = (1:n > x1) & (1:n < x2);

xdata = find(is_inlier);
ydata = lambdas(is_inlier);
inlier_params = nlinfit(xdata, ydata, @expfun, [a,b]);

% Calculate the errors for that exponential
predicted=inlier_params(1).*exp(inlier_params(2).*(1:n));
errors = abs(lambdas-predicted);

% Calculate the standard deviation on the points in
% the interval between the two best points
between_std = std(errors(best_x1+1:best_x2-1));

% Outliers are those that are more than 3 std away
is_inlier = errors <= 3*between_std;

% The outlier before the first inlier is our elbow
elbow_index = find(is_inlier,1,'first');
if elbow_index > 1
    elbow_index = elbow_index - 1;
end

% 
% Helper function for fitting
    function predicted = expfun(params, x)
        A = params(1);
        B = params(2);
        predicted = A .* exp(B .* x);
    end

end
\end{lstlisting}

\subsection{offset\_elbow\_point.m}
\begin{lstlisting}
function [elbow_index,best_estimate] = ...
     offset_elbow_point(lambdas)
% Returns the index at which the slope of the lambdas 
% changes
%
% In using a scree plot to discover the number of 
% principal components, a frequent method is to choose
% the "elbow", the inflection point of the curve. One
% way of formalizing this is approximating the scree 
% plot with two lines. One goes from the beginning to
% the elbow point and the other to the end. Whichever
% is closest is the best elbow point.
%
% This is similar to elbow_point except that the
% height at the joining of the two lines is not one
% of the lambdas. This frequently leads to a more
% conservative estimate of when the slope changes and
% a closer approximation of the lambdas.
%
% Exhaustive search for the best approximation of the
% function composed of the ordered pairs i,lambdas(i) 
% composed of the lines (1,lambdas(1))...(i,height)
% and ((i,height...(end, lambdas(end))
%
% Mean absolute value of the difference is used as
% the measure of goodness of fit
%
% best_estimate - the linear fit to the lambdas
%     implied by elbow_index

if size(lambdas,1) > 1
    lambdas = lambdas';
end
best = 1;
best_error = inf;
best_estimate = [];
for cur=1:length(lambdas)
    [~, cur_error]=fminbnd(@linpredict, min(lambdas),...
         max(lambdas));
    if cur_error < best_error
        best_error = cur_error;
        best = cur;
        best_estimate = cur_estimate;
    end
end

elbow_index = best;

    % Calculate the two-line fit with the elbow_index
    % at cur and the height at the elbow set to
    % height. Also sets cur_estimate and uses lambdas
    function err=linpredict(height)
        cur_estimate_a = linspace(lambdas(1), ...
            height, cur);
        cur_estimate_b = linspace(height, ...
            lambdas(end), length(lambdas)-cur+1);
        cur_estimate = [cur_estimate_a(1:end-1), ...
            cur_estimate_b];
        err=mean(abs(cur_estimate - lambdas));
    end

end
\end{lstlisting}

\subsection{scree\_elbow\_using\_robust\_fit.m}
\begin{lstlisting}
function [elbow_index,best_x1, best_x2, ...
     minified_ranks, predicted, ranks] = ...
     scree_elbow_using_robust_fit( lambdas, ...
          error_prctile )
% Looks for the line in the log(lambdas) that has the
% most inliers, then counts the first outlier as the
% elbow
%
% Scree plots of random stuff look like exponential
% curves. Thus, they are flat on a log scale. The
% dimensions carrying information either have a
% different slope or aren't exponential at all. Thus,
% they will be outliers. On a log plot, the random
% stuff looks like a straight line. (However, if the
% system is rank deficient, the last few principal
% components will sharply trail off and not look like
% a straight line.)
% 
% Let lambdas be the 'latent' return from princomp.
% Treat error_prctile like it was a fraction.
% 
% I robustly fit an exponential to the whole curve.
% Then I choose error_prctile of the best errors and
% robustly fit the exponential again to the interval
% containing those points. Finally, I choose the
% interval containing error_prctile of the best points
% again and use that interval as the interval of
% random points. The point before the first point in
% the interval is the elbow index - the first
% non-random lambda.
%
%
%
% INPUT:
%
% lambdas - a vector of strictly postitive scalars 
%     sorted in decreasing order. The 'latent' return
%     from princomp
%
% error_prctile - a scalar from 0..100. The points
%     with an error less than error prctile are used
%     to create the region from which the exponential
%     fit is calculated. Default is 25, so the top 25%
%     of the errors are used. The smaller 
%     error_prctile the smaller the sample used to 
%     estimate the equation governing noise lambdas.
%     So, you'd like error_prctile to be large. 
%     However, if error_prctile is too large then the
%     size of the non-random interval will be
%     overestimated.
%
%
%
% OUTPUT:
%
% elbow_index - the index of the last of the 
%     non-random lambdas
%
% best_x1 - the first index of the random lambda 
%     interval
%
% best_x2 - the last index of the random lambda 
%     interval
%
% minified_ranks - minified_ranks(i)=min(ranks(1:i)).
%     100*minified_ranks(i)/max(ranks(i)) can be
%     thought of a number p such that if 
%     error_prctile > p, then i will be best_x1. If
%     error_prctile were not used twice, this would be
%     the definition. However, there is some minor
%     interaction with error_prctile that makes this
%     only approximate.
%
% predicted - the exponential function that was
%     fitted to the random values
%
% ranks - rank(i) is the rank of the error when
%     estimating lambdas(i) with predicted(i). The
%     smallest errors have the smallest ranks.
%

if any(lambdas <= 0)
    error('scree_elbow_using_robust_fit:pos_lambda',...
        ['All lambda values must be strictly ' ...
        'positive in scree_elbow_using_robust_fit']);
end

if ~exist('error_prctile','var')
    error_prctile = 25;
end

num_points = floor(length(lambdas)*error_prctile/100);

if num_points < 2
    error(['scree_elbow_using_robust_fit:'...
        'num_points_too_small'],...
        ['When error_prctile points are selected '...
        'from lambdas in '...
        'scree_elbow_using_robust_fit there must '...
        'be at least 2.']);
end

if size(lambdas, 1) > 1
    lambdas = lambdas';
end


% Find a starting point for an exponential fit to
% the whole line. Here I do a linear fit to the
% logarithm.
n=length(lambdas);
warning('off','stats:statrobustfit:IterationLimit');
robust_params = robustfit(1:n, log(lambdas));
warning('on','stats:statrobustfit:IterationLimit');

% Next, I use the linear robust fit as the starting
% point to do a nonlinear robust fit on the whole
% line using an exponential function. This should
% be the same, except for the errors being of a more
% appropriate distribution. In particular, I hope that
% the errors in the exponential section are normal
expfun=@(param,x) param(1).*exp(param(2).*x);
fitopts=statset('Robust','on','MaxIter',5000);
starting_pt = [exp(robust_params(1)),...
    robust_params(2)]; % exp(mx+b)=exp(b)*exp(mx)
robust_params=nlinfit(1:n,lambdas(1:n),expfun,...
    starting_pt, fitopts);

% Next, I calculate the inliers. This will help take
% care of the fact that the m-estimators used in
% Matlab's robust fitting still have a zero
% breakdown point.
%
% I find the points with the best error. Then the
% in-lying region is the smallest region that contains
% those points. error_prctile is the fraction of the
% points that are used.
predicted = expfun(robust_params, 1:n);
errors = abs(lambdas - predicted);
max_error = prctile(errors, error_prctile);
best_x1 = find(errors <= max_error, 1, 'first');
best_x2 = find(errors <= max_error, 1, 'last');

% Repeat the estimation in untransformed space using
% only the in-lying points. 
starting_pt = [robust_params(1),robust_params(2)];
robust_params=nlinfit(best_x1:best_x2,...
    lambdas(best_x1:best_x2),expfun,...
    starting_pt, fitopts);
predicted = expfun(robust_params, 1:n);

% Find the in-lying region for the new points. This
% will be the region of random points and the first
% point before it will be our elbow point
errors = abs(lambdas - predicted);
max_error = prctile(errors, error_prctile);
best_x1 = find(errors <= max_error, 1, 'first');
best_x2 = find(errors <= max_error, 1, 'last');

ranks = tiedrank(errors);

% Create a diagnostic based on the ranks of the
% errors. If you divide minified_ranks by 
% length(lambda), you will get the percentile of the
% population you would have needed to choose in order
% for that to be the first in-lying point
minified_ranks = ranks;
for i=2:length(ranks)
    minified_ranks(i)=min(minified_ranks(i-1), ...
        minified_ranks(i)); 
end


if best_x1 > 1
    elbow_index = best_x1 - 1;
else
    elbow_index = 1;
end

end
\end{lstlisting}

\chapter{Ranked word lists}
\label{app:rankedwordlists}
For each list of word vectors, the derived principal components can be used to 
give a numerical score to each word in the list. The following sections list
the most important components derived from each method of analyzing each list.

\section{101 word list}
\subsection{Unnormalized PCA}
\label{app:rankedwordlists:101words:unnormalized}
\begin{longtable}[!htbp]{| rlr@{.}l |}
    \hline
    \textbf{Rank} & \textbf{Word} & \multicolumn{2}{c|}{\textbf{Score}} \\
    \hline
    \endhead
    1 & irritable\_jj & -1 & -5886 \\
    2 & self-pitying\_jj & -1 & -5705 \\
    3 & extroverted\_jj & -1 & -5033 \\
    4 & talkative\_jj & -1 & -4127 \\
    5 & unintelligent\_jj & -1 & -3818 \\
    6 & uncooperative\_jj & -1 & -2884 \\
    7 & rude\_jj & -1 & -2600 \\
    8 & unkind\_jj & -1 & -1717 \\
    9 & bashful\_jj & -1 & -1495 \\
    10 & uncharitable\_jj & -1 & -839 \\
    11 & envious\_jj & -1 & -807 \\
    12 & introspective\_jj & -1 & -655 \\
    13 & selfish\_jj & -1 & -508 \\
    14 & considerate\_jj & -1 & -475 \\
    15 & withdrawn\_jj & -1 & -229 \\
    16 & high-strung\_jj & -1 & -219 \\
    17 & jealous\_jj & 0 & -9988 \\
    18 & insecure\_jj & 0 & -9076 \\
    19 & distrustful\_jj & 0 & -8705 \\
    20 & unsympathetic\_jj & 0 & -8481 \\
    21 & kind\_jj & 0 & -8353 \\
    22 & unemotional\_jj & 0 & -8087 \\
    23 & unsophisticated\_jj & 0 & -6918 \\
    24 & fretful\_jj & 0 & -6844 \\
    25 & unreflective\_jj & 0 & -6163 \\
    26 & moody\_jj & 0 & -6077 \\
    27 & temperamental\_jj & 0 & -5431 \\
    28 & unadventurous\_jj & 0 & -5036 \\
    29 & unimaginative\_jj & 0 & -4083 \\
    30 & agreeable\_jj & 0 & -3564 \\
    61 & harsh\_jj & 0 & 5260 \\
    62 & shallow\_jj & 0 & 5321 \\
    63 & active\_jj & 0 & 5358 \\
    64 & artistic\_jj & 0 & 5453 \\
    65 & inefficient\_jj & 0 & 5701 \\
    66 & haphazard\_jj & 0 & 5841 \\
    67 & cold\_jj & 0 & 5942 \\
    68 & inconsistent\_jj & 0 & 6448 \\
    69 & intellectual\_jj & 0 & 6829 \\
    70 & sloppy\_jj & 0 & 7018 \\
    71 & deep\_jj & 0 & 7727 \\
    72 & cooperative\_jj & 0 & 7826 \\
    73 & bold\_jj & 0 & 8169 \\
    74 & creative\_jj & 0 & 8271 \\
    75 & careful\_jj & 0 & 8584 \\
    76 & warm\_jj & 0 & 8613 \\
    77 & helpful\_jj & 0 & 8729 \\
    78 & neat\_jj & 0 & 8995 \\
    79 & complex\_jj & 0 & 9786 \\
    80 & steady\_jj & 0 & 9945 \\
    81 & organized\_jj & 1 & 125 \\
    82 & practical\_jj & 1 & 305 \\
    83 & vigorous\_jj & 1 & 425 \\
    84 & negligent\_jj & 1 & 606 \\
    85 & simple\_jj & 1 & 1041 \\
    86 & innovative\_jj & 1 & 1617 \\
    87 & systematic\_jj & 1 & 2588 \\
    88 & efficient\_jj & 1 & 3437 \\
    89 & prompt\_jj & 1 & 5582 \\
    90 & thorough\_jj & 1 & 5744 \\
    \hline
    \caption{Scores and rankings for most extreme 30 words in component \#1} \\
\end{longtable}
\begin{longtable}[!htbp]{| rlr@{.}l |}
    \hline
    \textbf{Rank} & \textbf{Word} & \multicolumn{2}{c|}{\textbf{Score}} \\
    \hline
    \endhead
    1 & uncooperative\_jj & -1 & -9476 \\
    2 & negligent\_jj & -1 & -2860 \\
    3 & prompt\_jj & 0 & -9101 \\
    4 & distrustful\_jj & 0 & -7927 \\
    5 & insecure\_jj & 0 & -7198 \\
    6 & inconsistent\_jj & 0 & -7036 \\
    7 & rude\_jj & 0 & -6212 \\
    8 & systematic\_jj & 0 & -6061 \\
    9 & fearful\_jj & 0 & -5728 \\
    10 & impractical\_jj & 0 & -5708 \\
    11 & inefficient\_jj & 0 & -5547 \\
    12 & unsophisticated\_jj & 0 & -5494 \\
    13 & selfish\_jj & 0 & -4869 \\
    14 & unsympathetic\_jj & 0 & -4856 \\
    15 & unintelligent\_jj & 0 & -4842 \\
    16 & careless\_jj & 0 & -4728 \\
    17 & sloppy\_jj & 0 & -4204 \\
    18 & cooperative\_jj & 0 & -3703 \\
    19 & touchy\_jj & 0 & -3605 \\
    20 & withdrawn\_jj & 0 & -3538 \\
    21 & jealous\_jj & 0 & -3474 \\
    22 & irritable\_jj & 0 & -3454 \\
    23 & helpful\_jj & 0 & -3255 \\
    24 & harsh\_jj & 0 & -3029 \\
    25 & undependable\_jj & 0 & -2797 \\
    26 & thorough\_jj & 0 & -2694 \\
    27 & unsystematic\_jj & 0 & -2531 \\
    28 & timid\_jj & 0 & -2525 \\
    29 & unrestrained\_jj & 0 & -2523 \\
    30 & haphazard\_jj & 0 & -2444 \\
    61 & bashful\_jj & 0 & 1837 \\
    62 & intellectual\_jj & 0 & 1839 \\
    63 & self-pitying\_jj & 0 & 1860 \\
    64 & envious\_jj & 0 & 1949 \\
    65 & bright\_jj & 0 & 2087 \\
    66 & temperamental\_jj & 0 & 2311 \\
    67 & philosophical\_jj & 0 & 2331 \\
    68 & unreflective\_jj & 0 & 2349 \\
    69 & reserved\_jj & 0 & 2399 \\
    70 & uncharitable\_jj & 0 & 2433 \\
    71 & daring\_jj & 0 & 2507 \\
    72 & shy\_jj & 0 & 2566 \\
    73 & steady\_jj & 0 & 2581 \\
    74 & kind\_jj & 0 & 2845 \\
    75 & bold\_jj & 0 & 2928 \\
    76 & vigorous\_jj & 0 & 3118 \\
    77 & high-strung\_jj & 0 & 3226 \\
    78 & neat\_jj & 0 & 3237 \\
    79 & moody\_jj & 0 & 3569 \\
    80 & warm\_jj & 0 & 3807 \\
    81 & talkative\_jj & 0 & 4106 \\
    82 & pleasant\_jj & 0 & 4213 \\
    83 & quiet\_jj & 0 & 4237 \\
    84 & extroverted\_jj & 0 & 4315 \\
    85 & imaginative\_jj & 0 & 4469 \\
    86 & artistic\_jj & 0 & 4836 \\
    87 & introspective\_jj & 0 & 6114 \\
    88 & energetic\_jj & 0 & 7997 \\
    89 & considerate\_jj & 0 & 8389 \\
    90 & imperturbable\_jj & 6 & 3861 \\
    \hline
    \caption{Scores and rankings for most extreme 30 words in component \#2} \\
\end{longtable}
\begin{longtable}[!htbp]{| rlr@{.}l |}
    \hline
    \textbf{Rank} & \textbf{Word} & \multicolumn{2}{c|}{\textbf{Score}} \\
    \hline
    \endhead
    1 & imperturbable\_jj & -2 & -9951 \\
    2 & negligent\_jj & -2 & -5643 \\
    3 & uncooperative\_jj & -2 & -2308 \\
    4 & selfish\_jj & -1 & -2090 \\
    5 & careless\_jj & -1 & -1756 \\
    6 & unreflective\_jj & -1 & -120 \\
    7 & self-pitying\_jj & 0 & -9766 \\
    8 & prompt\_jj & 0 & -9704 \\
    9 & sloppy\_jj & 0 & -8880 \\
    10 & inefficient\_jj & 0 & -8721 \\
    11 & unrestrained\_jj & 0 & -8562 \\
    12 & verbal\_jj & 0 & -6895 \\
    13 & uncharitable\_jj & 0 & -6814 \\
    14 & inconsistent\_jj & 0 & -6700 \\
    15 & systematic\_jj & 0 & -6054 \\
    16 & unimaginative\_jj & 0 & -5441 \\
    17 & rude\_jj & 0 & -5359 \\
    18 & haphazard\_jj & 0 & -4472 \\
    19 & unintelligent\_jj & 0 & -4344 \\
    20 & unsophisticated\_jj & 0 & -4287 \\
    21 & unsystematic\_jj & 0 & -4167 \\
    22 & unsympathetic\_jj & 0 & -3971 \\
    23 & unkind\_jj & 0 & -3599 \\
    24 & impractical\_jj & 0 & -3237 \\
    25 & harsh\_jj & 0 & -2279 \\
    26 & fretful\_jj & 0 & -2075 \\
    27 & distrustful\_jj & 0 & -1862 \\
    28 & unexcitable\_jj & 0 & -1623 \\
    29 & unemotional\_jj & 0 & -1477 \\
    30 & intellectual\_jj & 0 & -1445 \\
    61 & artistic\_jj & 0 & 2382 \\
    62 & envious\_jj & 0 & 2579 \\
    63 & cooperative\_jj & 0 & 2840 \\
    64 & careful\_jj & 0 & 3295 \\
    65 & helpful\_jj & 0 & 3528 \\
    66 & practical\_jj & 0 & 3580 \\
    67 & high-strung\_jj & 0 & 3806 \\
    68 & efficient\_jj & 0 & 3825 \\
    69 & moody\_jj & 0 & 4071 \\
    70 & active\_jj & 0 & 4637 \\
    71 & steady\_jj & 0 & 4763 \\
    72 & agreeable\_jj & 0 & 4765 \\
    73 & shy\_jj & 0 & 4805 \\
    74 & cold\_jj & 0 & 5418 \\
    75 & creative\_jj & 0 & 5849 \\
    76 & bashful\_jj & 0 & 5941 \\
    77 & introspective\_jj & 0 & 5995 \\
    78 & neat\_jj & 0 & 6433 \\
    79 & conscientious\_jj & 0 & 6558 \\
    80 & reserved\_jj & 0 & 6789 \\
    81 & quiet\_jj & 0 & 7281 \\
    82 & bright\_jj & 0 & 7479 \\
    83 & generous\_jj & 0 & 7507 \\
    84 & energetic\_jj & 0 & 8527 \\
    85 & extroverted\_jj & 0 & 8611 \\
    86 & talkative\_jj & 1 & 265 \\
    87 & pleasant\_jj & 1 & 328 \\
    88 & kind\_jj & 1 & 4907 \\
    89 & warm\_jj & 1 & 9718 \\
    90 & considerate\_jj & 2 & 836 \\
    \hline
    \caption{Scores and rankings for most extreme 30 words in component \#3} \\
\end{longtable}
\begin{longtable}[!htbp]{| rlr@{.}l |}
    \hline
    \textbf{Rank} & \textbf{Word} & \multicolumn{2}{c|}{\textbf{Score}} \\
    \hline
    \endhead
    1 & considerate\_jj & -3 & -217 \\
    2 & negligent\_jj & -1 & -9858 \\
    3 & kind\_jj & -1 & -3891 \\
    4 & conscientious\_jj & -1 & -2300 \\
    5 & talkative\_jj & -1 & -880 \\
    6 & thorough\_jj & 0 & -9999 \\
    7 & cooperative\_jj & 0 & -9428 \\
    8 & uncooperative\_jj & 0 & -8794 \\
    9 & energetic\_jj & 0 & -8266 \\
    10 & prompt\_jj & 0 & -8010 \\
    11 & efficient\_jj & 0 & -7803 \\
    12 & selfish\_jj & 0 & -5693 \\
    13 & systematic\_jj & 0 & -4453 \\
    14 & generous\_jj & 0 & -4388 \\
    15 & imperturbable\_jj & 0 & -3698 \\
    16 & extroverted\_jj & 0 & -3587 \\
    17 & agreeable\_jj & 0 & -3565 \\
    18 & careful\_jj & 0 & -3526 \\
    19 & innovative\_jj & 0 & -3403 \\
    20 & demanding\_jj & 0 & -3070 \\
    21 & unsympathetic\_jj & 0 & -2928 \\
    22 & helpful\_jj & 0 & -2887 \\
    23 & inefficient\_jj & 0 & -2627 \\
    24 & creative\_jj & 0 & -2600 \\
    25 & assertive\_jj & 0 & -2568 \\
    26 & active\_jj & 0 & -2553 \\
    27 & imaginative\_jj & 0 & -2429 \\
    28 & reserved\_jj & 0 & -2076 \\
    29 & shy\_jj & 0 & -2046 \\
    30 & timid\_jj & 0 & -1717 \\
    61 & unadventurous\_jj & 0 & 2783 \\
    62 & insecure\_jj & 0 & 2977 \\
    63 & envious\_jj & 0 & 2996 \\
    64 & emotional\_jj & 0 & 3125 \\
    65 & bashful\_jj & 0 & 3148 \\
    66 & unexcitable\_jj & 0 & 3217 \\
    67 & unintelligent\_jj & 0 & 3367 \\
    68 & nervous\_jj & 0 & 3540 \\
    69 & simple\_jj & 0 & 3602 \\
    70 & fearful\_jj & 0 & 3618 \\
    71 & unkind\_jj & 0 & 3649 \\
    72 & uncharitable\_jj & 0 & 3759 \\
    73 & moody\_jj & 0 & 3821 \\
    74 & philosophical\_jj & 0 & 3876 \\
    75 & sloppy\_jj & 0 & 4305 \\
    76 & fretful\_jj & 0 & 4827 \\
    77 & touchy\_jj & 0 & 4928 \\
    78 & bright\_jj & 0 & 5164 \\
    79 & undemanding\_jj & 0 & 6033 \\
    80 & verbal\_jj & 0 & 6628 \\
    81 & steady\_jj & 0 & 6721 \\
    82 & self-pitying\_jj & 0 & 7772 \\
    83 & shallow\_jj & 0 & 7965 \\
    84 & irritable\_jj & 0 & 8343 \\
    85 & unreflective\_jj & 0 & 8487 \\
    86 & harsh\_jj & 0 & 9240 \\
    87 & neat\_jj & 0 & 9837 \\
    88 & deep\_jj & 0 & 9968 \\
    89 & warm\_jj & 1 & 3880 \\
    90 & cold\_jj & 1 & 6310 \\
    \hline
    \caption{Scores and rankings for most extreme 30 words in component \#4} \\
\end{longtable}
\begin{longtable}[!htbp]{| rlr@{.}l |}
    \hline
    \textbf{Rank} & \textbf{Word} & \multicolumn{2}{c|}{\textbf{Score}} \\
    \hline
    \endhead
    1 & unreflective\_jj & -1 & -4591 \\
    2 & selfish\_jj & -1 & -3308 \\
    3 & artistic\_jj & -1 & -1915 \\
    4 & neat\_jj & -1 & -1234 \\
    5 & unintelligent\_jj & -1 & -637 \\
    6 & creative\_jj & 0 & -9712 \\
    7 & self-pitying\_jj & 0 & -9420 \\
    8 & imaginative\_jj & 0 & -8228 \\
    9 & intellectual\_jj & 0 & -8197 \\
    10 & careless\_jj & 0 & -6619 \\
    11 & unimaginative\_jj & 0 & -6420 \\
    12 & philosophical\_jj & 0 & -5545 \\
    13 & uncharitable\_jj & 0 & -5346 \\
    14 & unadventurous\_jj & 0 & -5334 \\
    15 & unsophisticated\_jj & 0 & -5119 \\
    16 & innovative\_jj & 0 & -4857 \\
    17 & negligent\_jj & 0 & -4627 \\
    18 & daring\_jj & 0 & -4546 \\
    19 & practical\_jj & 0 & -4160 \\
    20 & jealous\_jj & 0 & -3740 \\
    21 & simple\_jj & 0 & -3198 \\
    22 & bold\_jj & 0 & -3195 \\
    23 & haphazard\_jj & 0 & -3140 \\
    24 & temperamental\_jj & 0 & -2986 \\
    25 & conscientious\_jj & 0 & -2867 \\
    26 & energetic\_jj & 0 & -2759 \\
    27 & envious\_jj & 0 & -2689 \\
    28 & high-strung\_jj & 0 & -2552 \\
    29 & unrestrained\_jj & 0 & -2384 \\
    30 & introspective\_jj & 0 & -2293 \\
    61 & kind\_jj & 0 & 1681 \\
    62 & insecure\_jj & 0 & 1884 \\
    63 & unsympathetic\_jj & 0 & 2386 \\
    64 & distrustful\_jj & 0 & 2712 \\
    65 & careful\_jj & 0 & 2728 \\
    66 & inconsistent\_jj & 0 & 2785 \\
    67 & verbal\_jj & 0 & 2967 \\
    68 & prompt\_jj & 0 & 3188 \\
    69 & reserved\_jj & 0 & 3445 \\
    70 & helpful\_jj & 0 & 3467 \\
    71 & vigorous\_jj & 0 & 3569 \\
    72 & pleasant\_jj & 0 & 3701 \\
    73 & quiet\_jj & 0 & 4316 \\
    74 & sympathetic\_jj & 0 & 4543 \\
    75 & organized\_jj & 0 & 4586 \\
    76 & fearful\_jj & 0 & 4588 \\
    77 & steady\_jj & 0 & 4672 \\
    78 & touchy\_jj & 0 & 4877 \\
    79 & cooperative\_jj & 0 & 5278 \\
    80 & active\_jj & 0 & 5311 \\
    81 & assertive\_jj & 0 & 5616 \\
    82 & talkative\_jj & 0 & 7508 \\
    83 & anxious\_jj & 0 & 7558 \\
    84 & harsh\_jj & 0 & 8646 \\
    85 & nervous\_jj & 0 & 9539 \\
    86 & cold\_jj & 0 & 9686 \\
    87 & imperturbable\_jj & 1 & 2480 \\
    88 & irritable\_jj & 1 & 3042 \\
    89 & warm\_jj & 1 & 6112 \\
    90 & uncooperative\_jj & 3 & 1549 \\
    \hline
    \caption{Scores and rankings for most extreme 30 words in component \#5} \\
\end{longtable}
\begin{longtable}[!htbp]{| rlr@{.}l |}
    \hline
    \textbf{Rank} & \textbf{Word} & \multicolumn{2}{c|}{\textbf{Score}} \\
    \hline
    \endhead
    1 & uncooperative\_jj & -2 & -1704 \\
    2 & efficient\_jj & -1 & -3516 \\
    3 & innovative\_jj & -1 & -475 \\
    4 & complex\_jj & 0 & -8118 \\
    5 & inefficient\_jj & 0 & -7741 \\
    6 & unimaginative\_jj & 0 & -7665 \\
    7 & extroverted\_jj & 0 & -7085 \\
    8 & imaginative\_jj & 0 & -6863 \\
    9 & unsophisticated\_jj & 0 & -6489 \\
    10 & assertive\_jj & 0 & -6212 \\
    11 & impractical\_jj & 0 & -6100 \\
    12 & artistic\_jj & 0 & -5690 \\
    13 & creative\_jj & 0 & -5390 \\
    14 & practical\_jj & 0 & -5124 \\
    15 & undemanding\_jj & 0 & -4817 \\
    16 & introspective\_jj & 0 & -4655 \\
    17 & cooperative\_jj & 0 & -4465 \\
    18 & demanding\_jj & 0 & -4405 \\
    19 & intellectual\_jj & 0 & -4198 \\
    20 & trustful\_jj & 0 & -4082 \\
    21 & daring\_jj & 0 & -3997 \\
    22 & agreeable\_jj & 0 & -3992 \\
    23 & active\_jj & 0 & -3928 \\
    24 & organized\_jj & 0 & -3684 \\
    25 & undependable\_jj & 0 & -3521 \\
    26 & touchy\_jj & 0 & -3513 \\
    27 & unadventurous\_jj & 0 & -3452 \\
    28 & unrestrained\_jj & 0 & -3260 \\
    29 & bold\_jj & 0 & -3157 \\
    30 & philosophical\_jj & 0 & -3135 \\
    61 & timid\_jj & 0 & 1878 \\
    62 & deep\_jj & 0 & 1904 \\
    63 & shallow\_jj & 0 & 2135 \\
    64 & bashful\_jj & 0 & 2232 \\
    65 & withdrawn\_jj & 0 & 2803 \\
    66 & pleasant\_jj & 0 & 2814 \\
    67 & envious\_jj & 0 & 2906 \\
    68 & sloppy\_jj & 0 & 2922 \\
    69 & unkind\_jj & 0 & 3320 \\
    70 & careful\_jj & 0 & 3415 \\
    71 & irritable\_jj & 0 & 3470 \\
    72 & harsh\_jj & 0 & 3987 \\
    73 & fearful\_jj & 0 & 4117 \\
    74 & bright\_jj & 0 & 4959 \\
    75 & quiet\_jj & 0 & 5026 \\
    76 & anxious\_jj & 0 & 5430 \\
    77 & uncharitable\_jj & 0 & 6125 \\
    78 & nervous\_jj & 0 & 6156 \\
    79 & jealous\_jj & 0 & 6194 \\
    80 & rude\_jj & 0 & 6368 \\
    81 & steady\_jj & 0 & 6738 \\
    82 & selfish\_jj & 0 & 7379 \\
    83 & shy\_jj & 0 & 7744 \\
    84 & considerate\_jj & 0 & 8026 \\
    85 & cold\_jj & 0 & 8835 \\
    86 & kind\_jj & 0 & 9197 \\
    87 & warm\_jj & 1 & 4061 \\
    88 & careless\_jj & 1 & 4548 \\
    89 & prompt\_jj & 1 & 7841 \\
    90 & negligent\_jj & 2 & 4228 \\
    \hline
    \caption{Scores and rankings for most extreme 30 words in component \#6} \\
\end{longtable}
\begin{longtable}[!htbp]{| rlr@{.}l |}
    \hline
    \textbf{Rank} & \textbf{Word} & \multicolumn{2}{c|}{\textbf{Score}} \\
    \hline
    \endhead
    1 & prompt\_jj & -3 & -7336 \\
    2 & irritable\_jj & 0 & -9481 \\
    3 & thorough\_jj & 0 & -7985 \\
    4 & unemotional\_jj & 0 & -7628 \\
    5 & vigorous\_jj & 0 & -7150 \\
    6 & assertive\_jj & 0 & -7032 \\
    7 & distrustful\_jj & 0 & -6829 \\
    8 & verbal\_jj & 0 & -6568 \\
    9 & withdrawn\_jj & 0 & -6355 \\
    10 & extroverted\_jj & 0 & -6181 \\
    11 & fearful\_jj & 0 & -4904 \\
    12 & anxious\_jj & 0 & -3823 \\
    13 & introspective\_jj & 0 & -3623 \\
    14 & emotional\_jj & 0 & -3376 \\
    15 & nervous\_jj & 0 & -3356 \\
    16 & philosophical\_jj & 0 & -3229 \\
    17 & self-pitying\_jj & 0 & -3173 \\
    18 & systematic\_jj & 0 & -3160 \\
    19 & demanding\_jj & 0 & -3127 \\
    20 & daring\_jj & 0 & -3070 \\
    21 & conscientious\_jj & 0 & -2914 \\
    22 & unsympathetic\_jj & 0 & -2841 \\
    23 & fretful\_jj & 0 & -2688 \\
    24 & uncharitable\_jj & 0 & -2665 \\
    25 & unrestrained\_jj & 0 & -2608 \\
    26 & sympathetic\_jj & 0 & -2321 \\
    27 & imaginative\_jj & 0 & -2319 \\
    28 & steady\_jj & 0 & -2214 \\
    29 & bashful\_jj & 0 & -2117 \\
    30 & bold\_jj & 0 & -2114 \\
    61 & reserved\_jj & 0 & 1818 \\
    62 & considerate\_jj & 0 & 1885 \\
    63 & creative\_jj & 0 & 1953 \\
    64 & helpful\_jj & 0 & 2340 \\
    65 & simple\_jj & 0 & 2433 \\
    66 & efficient\_jj & 0 & 2481 \\
    67 & unsophisticated\_jj & 0 & 2827 \\
    68 & haphazard\_jj & 0 & 2973 \\
    69 & inconsistent\_jj & 0 & 3129 \\
    70 & sloppy\_jj & 0 & 3143 \\
    71 & generous\_jj & 0 & 3297 \\
    72 & rude\_jj & 0 & 3633 \\
    73 & quiet\_jj & 0 & 4060 \\
    74 & shy\_jj & 0 & 4224 \\
    75 & selfish\_jj & 0 & 4516 \\
    76 & undemanding\_jj & 0 & 4571 \\
    77 & talkative\_jj & 0 & 4743 \\
    78 & unimaginative\_jj & 0 & 5190 \\
    79 & impractical\_jj & 0 & 5365 \\
    80 & kind\_jj & 0 & 5457 \\
    81 & cold\_jj & 0 & 6197 \\
    82 & bright\_jj & 0 & 6826 \\
    83 & inefficient\_jj & 0 & 7130 \\
    84 & careless\_jj & 0 & 7214 \\
    85 & pleasant\_jj & 0 & 7444 \\
    86 & shallow\_jj & 0 & 7541 \\
    87 & neat\_jj & 0 & 9921 \\
    88 & warm\_jj & 1 & 1486 \\
    89 & uncooperative\_jj & 1 & 1548 \\
    90 & negligent\_jj & 1 & 9614 \\
    \hline
    \caption{Scores and rankings for most extreme 30 words in component \#7} \\
\end{longtable}
\begin{longtable}[!htbp]{| rlr@{.}l |}
    \hline
    \textbf{Rank} & \textbf{Word} & \multicolumn{2}{c|}{\textbf{Score}} \\
    \hline
    \endhead
    1 & distrustful\_jj & -1 & -4109 \\
    2 & anxious\_jj & -1 & -2128 \\
    3 & nervous\_jj & -1 & -1637 \\
    4 & fearful\_jj & -1 & -1136 \\
    5 & assertive\_jj & -1 & -227 \\
    6 & irritable\_jj & 0 & -9385 \\
    7 & insecure\_jj & 0 & -8601 \\
    8 & active\_jj & 0 & -8243 \\
    9 & organized\_jj & 0 & -7503 \\
    10 & careful\_jj & 0 & -6499 \\
    11 & fretful\_jj & 0 & -6492 \\
    12 & envious\_jj & 0 & -5960 \\
    13 & jealous\_jj & 0 & -5855 \\
    14 & unsympathetic\_jj & 0 & -5487 \\
    15 & negligent\_jj & 0 & -5292 \\
    16 & shy\_jj & 0 & -4910 \\
    17 & timid\_jj & 0 & -4897 \\
    18 & touchy\_jj & 0 & -4688 \\
    19 & emotional\_jj & 0 & -4247 \\
    20 & conscientious\_jj & 0 & -4195 \\
    21 & helpful\_jj & 0 & -3945 \\
    22 & sympathetic\_jj & 0 & -3637 \\
    23 & intellectual\_jj & 0 & -3490 \\
    24 & demanding\_jj & 0 & -3348 \\
    25 & generous\_jj & 0 & -3275 \\
    26 & uncharitable\_jj & 0 & -3202 \\
    27 & kind\_jj & 0 & -3024 \\
    28 & imperturbable\_jj & 0 & -2464 \\
    29 & inefficient\_jj & 0 & -2174 \\
    30 & undependable\_jj & 0 & -2163 \\
    61 & cold\_jj & 0 & 2187 \\
    62 & bold\_jj & 0 & 2265 \\
    63 & unkind\_jj & 0 & 2290 \\
    64 & practical\_jj & 0 & 2398 \\
    65 & simple\_jj & 0 & 2450 \\
    66 & agreeable\_jj & 0 & 2465 \\
    67 & selfish\_jj & 0 & 2535 \\
    68 & bright\_jj & 0 & 2904 \\
    69 & extroverted\_jj & 0 & 2984 \\
    70 & daring\_jj & 0 & 3012 \\
    71 & moody\_jj & 0 & 3031 \\
    72 & rude\_jj & 0 & 3256 \\
    73 & introspective\_jj & 0 & 3397 \\
    74 & withdrawn\_jj & 0 & 3862 \\
    75 & thorough\_jj & 0 & 3992 \\
    76 & undemanding\_jj & 0 & 4360 \\
    77 & pleasant\_jj & 0 & 4465 \\
    78 & shallow\_jj & 0 & 4517 \\
    79 & talkative\_jj & 0 & 4552 \\
    80 & unreflective\_jj & 0 & 5147 \\
    81 & considerate\_jj & 0 & 6137 \\
    82 & reserved\_jj & 0 & 6150 \\
    83 & neat\_jj & 0 & 6757 \\
    84 & self-pitying\_jj & 0 & 7916 \\
    85 & unintelligent\_jj & 0 & 8484 \\
    86 & high-strung\_jj & 1 & 575 \\
    87 & unemotional\_jj & 1 & 2056 \\
    88 & uncooperative\_jj & 1 & 8397 \\
    89 & warm\_jj & 1 & 8525 \\
    90 & prompt\_jj & 2 & 1005 \\
    \hline
    \caption{Scores and rankings for most extreme 30 words in component \#8} \\
\end{longtable}
\begin{longtable}[!htbp]{| rlr@{.}l |}
    \hline
    \textbf{Rank} & \textbf{Word} & \multicolumn{2}{c|}{\textbf{Score}} \\
    \hline
    \endhead
    1 & negligent\_jj & -2 & -3733 \\
    2 & irritable\_jj & -1 & -1460 \\
    3 & high-strung\_jj & 0 & -9766 \\
    4 & artistic\_jj & 0 & -9648 \\
    5 & verbal\_jj & 0 & -9398 \\
    6 & unemotional\_jj & 0 & -7548 \\
    7 & introspective\_jj & 0 & -7526 \\
    8 & emotional\_jj & 0 & -7367 \\
    9 & extroverted\_jj & 0 & -6964 \\
    10 & intellectual\_jj & 0 & -6672 \\
    11 & creative\_jj & 0 & -6429 \\
    12 & imaginative\_jj & 0 & -5801 \\
    13 & moody\_jj & 0 & -4931 \\
    14 & vigorous\_jj & 0 & -4852 \\
    15 & daring\_jj & 0 & -4807 \\
    16 & fretful\_jj & 0 & -4508 \\
    17 & unreflective\_jj & 0 & -4339 \\
    18 & uncooperative\_jj & 0 & -3931 \\
    19 & energetic\_jj & 0 & -3827 \\
    20 & temperamental\_jj & 0 & -3381 \\
    21 & talkative\_jj & 0 & -3300 \\
    22 & bashful\_jj & 0 & -3290 \\
    23 & demanding\_jj & 0 & -3184 \\
    24 & philosophical\_jj & 0 & -3116 \\
    25 & withdrawn\_jj & 0 & -3070 \\
    26 & innovative\_jj & 0 & -2889 \\
    27 & unadventurous\_jj & 0 & -2738 \\
    28 & reserved\_jj & 0 & -2535 \\
    29 & anxious\_jj & 0 & -2217 \\
    30 & cooperative\_jj & 0 & -1887 \\
    61 & bold\_jj & 0 & 2069 \\
    62 & inconsistent\_jj & 0 & 2278 \\
    63 & neat\_jj & 0 & 2312 \\
    64 & shy\_jj & 0 & 2622 \\
    65 & harsh\_jj & 0 & 2676 \\
    66 & warm\_jj & 0 & 2761 \\
    67 & unkind\_jj & 0 & 3141 \\
    68 & undependable\_jj & 0 & 3160 \\
    69 & jealous\_jj & 0 & 3281 \\
    70 & unsympathetic\_jj & 0 & 3345 \\
    71 & fearful\_jj & 0 & 3685 \\
    72 & timid\_jj & 0 & 3991 \\
    73 & insecure\_jj & 0 & 4270 \\
    74 & helpful\_jj & 0 & 4780 \\
    75 & shallow\_jj & 0 & 4842 \\
    76 & considerate\_jj & 0 & 5057 \\
    77 & unsophisticated\_jj & 0 & 5433 \\
    78 & haphazard\_jj & 0 & 6083 \\
    79 & rude\_jj & 0 & 6116 \\
    80 & efficient\_jj & 0 & 6266 \\
    81 & impractical\_jj & 0 & 6295 \\
    82 & imperturbable\_jj & 0 & 6867 \\
    83 & unimaginative\_jj & 0 & 7372 \\
    84 & generous\_jj & 0 & 7713 \\
    85 & kind\_jj & 0 & 8545 \\
    86 & unintelligent\_jj & 0 & 9773 \\
    87 & distrustful\_jj & 1 & 1078 \\
    88 & prompt\_jj & 1 & 1246 \\
    89 & selfish\_jj & 1 & 4718 \\
    90 & inefficient\_jj & 1 & 6489 \\
    \hline
    \caption{Scores and rankings for most extreme 30 words in component \#9} \\
\end{longtable}
\begin{longtable}[!htbp]{| rlr@{.}l |}
    \hline
    \textbf{Rank} & \textbf{Word} & \multicolumn{2}{c|}{\textbf{Score}} \\
    \hline
    \endhead
    1 & irritable\_jj & -2 & -7227 \\
    2 & efficient\_jj & -1 & -3333 \\
    3 & inefficient\_jj & -1 & -2433 \\
    4 & unimaginative\_jj & 0 & -8547 \\
    5 & withdrawn\_jj & 0 & -7740 \\
    6 & negligent\_jj & 0 & -7106 \\
    7 & unsophisticated\_jj & 0 & -7016 \\
    8 & prompt\_jj & 0 & -6572 \\
    9 & haphazard\_jj & 0 & -6034 \\
    10 & undemanding\_jj & 0 & -5984 \\
    11 & sloppy\_jj & 0 & -5771 \\
    12 & talkative\_jj & 0 & -5467 \\
    13 & innovative\_jj & 0 & -5434 \\
    14 & impractical\_jj & 0 & -5309 \\
    15 & neat\_jj & 0 & -4845 \\
    16 & moody\_jj & 0 & -4765 \\
    17 & bright\_jj & 0 & -4724 \\
    18 & high-strung\_jj & 0 & -4262 \\
    19 & extroverted\_jj & 0 & -3606 \\
    20 & undependable\_jj & 0 & -3563 \\
    21 & insecure\_jj & 0 & -3467 \\
    22 & nervous\_jj & 0 & -3015 \\
    23 & organized\_jj & 0 & -2886 \\
    24 & complex\_jj & 0 & -2571 \\
    25 & unintelligent\_jj & 0 & -1972 \\
    26 & generous\_jj & 0 & -1884 \\
    27 & cold\_jj & 0 & -1693 \\
    28 & imaginative\_jj & 0 & -1522 \\
    29 & demanding\_jj & 0 & -1398 \\
    30 & energetic\_jj & 0 & -1361 \\
    61 & cooperative\_jj & 0 & 1849 \\
    62 & shallow\_jj & 0 & 1898 \\
    63 & agreeable\_jj & 0 & 2178 \\
    64 & sympathetic\_jj & 0 & 2216 \\
    65 & harsh\_jj & 0 & 2298 \\
    66 & warm\_jj & 0 & 2310 \\
    67 & fearful\_jj & 0 & 2366 \\
    68 & unemotional\_jj & 0 & 2640 \\
    69 & thorough\_jj & 0 & 2743 \\
    70 & emotional\_jj & 0 & 2794 \\
    71 & practical\_jj & 0 & 2854 \\
    72 & shy\_jj & 0 & 3498 \\
    73 & reserved\_jj & 0 & 3593 \\
    74 & intellectual\_jj & 0 & 4284 \\
    75 & systematic\_jj & 0 & 4286 \\
    76 & selfish\_jj & 0 & 4356 \\
    77 & quiet\_jj & 0 & 4592 \\
    78 & deep\_jj & 0 & 5323 \\
    79 & unrestrained\_jj & 0 & 5593 \\
    80 & unkind\_jj & 0 & 5805 \\
    81 & jealous\_jj & 0 & 5921 \\
    82 & verbal\_jj & 0 & 6299 \\
    83 & artistic\_jj & 0 & 7088 \\
    84 & unreflective\_jj & 0 & 7269 \\
    85 & careful\_jj & 0 & 7928 \\
    86 & distrustful\_jj & 0 & 9380 \\
    87 & philosophical\_jj & 1 & 201 \\
    88 & uncooperative\_jj & 1 & 316 \\
    89 & conscientious\_jj & 1 & 1750 \\
    90 & touchy\_jj & 1 & 4967 \\
    \hline
    \caption{Scores and rankings for most extreme 30 words in component \#10} \\
\end{longtable}
\begin{longtable}[!htbp]{| rlr@{.}l |}
    \hline
    \textbf{Rank} & \textbf{Word} & \multicolumn{2}{c|}{\textbf{Score}} \\
    \hline
    \endhead
    1 & warm\_jj & -1 & -5897 \\
    2 & selfish\_jj & -1 & -1108 \\
    3 & artistic\_jj & -1 & -891 \\
    4 & intellectual\_jj & 0 & -8987 \\
    5 & irritable\_jj & 0 & -8343 \\
    6 & unreflective\_jj & 0 & -7592 \\
    7 & innovative\_jj & 0 & -7321 \\
    8 & prompt\_jj & 0 & -7232 \\
    9 & creative\_jj & 0 & -7013 \\
    10 & negligent\_jj & 0 & -6541 \\
    11 & agreeable\_jj & 0 & -5960 \\
    12 & cold\_jj & 0 & -5220 \\
    13 & impractical\_jj & 0 & -5034 \\
    14 & conscientious\_jj & 0 & -4910 \\
    15 & active\_jj & 0 & -4660 \\
    16 & distrustful\_jj & 0 & -4441 \\
    17 & organized\_jj & 0 & -4315 \\
    18 & emotional\_jj & 0 & -4189 \\
    19 & trustful\_jj & 0 & -4069 \\
    20 & undependable\_jj & 0 & -3575 \\
    21 & practical\_jj & 0 & -3389 \\
    22 & helpful\_jj & 0 & -3277 \\
    23 & uncharitable\_jj & 0 & -3115 \\
    24 & efficient\_jj & 0 & -3009 \\
    25 & unintelligent\_jj & 0 & -2762 \\
    26 & unsympathetic\_jj & 0 & -2692 \\
    27 & imaginative\_jj & 0 & -2419 \\
    28 & unkind\_jj & 0 & -2241 \\
    29 & imperturbable\_jj & 0 & -2031 \\
    30 & assertive\_jj & 0 & -1922 \\
    61 & reserved\_jj & 0 & 1627 \\
    62 & introspective\_jj & 0 & 1904 \\
    63 & nervous\_jj & 0 & 1934 \\
    64 & withdrawn\_jj & 0 & 2117 \\
    65 & cooperative\_jj & 0 & 2385 \\
    66 & kind\_jj & 0 & 2521 \\
    67 & jealous\_jj & 0 & 2643 \\
    68 & unimaginative\_jj & 0 & 2966 \\
    69 & inconsistent\_jj & 0 & 3057 \\
    70 & unemotional\_jj & 0 & 3322 \\
    71 & touchy\_jj & 0 & 3484 \\
    72 & quiet\_jj & 0 & 4334 \\
    73 & rude\_jj & 0 & 4334 \\
    74 & careless\_jj & 0 & 4661 \\
    75 & timid\_jj & 0 & 5180 \\
    76 & careful\_jj & 0 & 5356 \\
    77 & self-pitying\_jj & 0 & 5910 \\
    78 & unsophisticated\_jj & 0 & 5971 \\
    79 & verbal\_jj & 0 & 6278 \\
    80 & high-strung\_jj & 0 & 6447 \\
    81 & shy\_jj & 0 & 6752 \\
    82 & talkative\_jj & 0 & 6780 \\
    83 & vigorous\_jj & 0 & 6799 \\
    84 & systematic\_jj & 0 & 7107 \\
    85 & bashful\_jj & 0 & 8169 \\
    86 & neat\_jj & 0 & 9127 \\
    87 & haphazard\_jj & 1 & 870 \\
    88 & sloppy\_jj & 1 & 1625 \\
    89 & steady\_jj & 1 & 2213 \\
    90 & thorough\_jj & 1 & 7289 \\
    \hline
    \caption{Scores and rankings for most extreme 30 words in component \#11} \\
\end{longtable}
\begin{longtable}[!htbp]{| rlr@{.}l |}
    \hline
    \textbf{Rank} & \textbf{Word} & \multicolumn{2}{c|}{\textbf{Score}} \\
    \hline
    \endhead
    1 & rude\_jj & -1 & -7913 \\
    2 & helpful\_jj & 0 & -9921 \\
    3 & artistic\_jj & 0 & -8208 \\
    4 & verbal\_jj & 0 & -7793 \\
    5 & careful\_jj & 0 & -7231 \\
    6 & irritable\_jj & 0 & -7075 \\
    7 & selfish\_jj & 0 & -7047 \\
    8 & practical\_jj & 0 & -6840 \\
    9 & creative\_jj & 0 & -6764 \\
    10 & careless\_jj & 0 & -6638 \\
    11 & nervous\_jj & 0 & -6601 \\
    12 & unkind\_jj & 0 & -6388 \\
    13 & uncooperative\_jj & 0 & -6152 \\
    14 & emotional\_jj & 0 & -5762 \\
    15 & touchy\_jj & 0 & -5508 \\
    16 & simple\_jj & 0 & -5419 \\
    17 & shy\_jj & 0 & -5365 \\
    18 & innovative\_jj & 0 & -5078 \\
    19 & uncharitable\_jj & 0 & -4693 \\
    20 & prompt\_jj & 0 & -3625 \\
    21 & pleasant\_jj & 0 & -3594 \\
    22 & imaginative\_jj & 0 & -3427 \\
    23 & intellectual\_jj & 0 & -3371 \\
    24 & bashful\_jj & 0 & -3366 \\
    25 & anxious\_jj & 0 & -3073 \\
    26 & philosophical\_jj & 0 & -2843 \\
    27 & kind\_jj & 0 & -2766 \\
    28 & jealous\_jj & 0 & -2368 \\
    29 & efficient\_jj & 0 & -2296 \\
    30 & impractical\_jj & 0 & -1862 \\
    61 & shallow\_jj & 0 & 1497 \\
    62 & unsystematic\_jj & 0 & 1731 \\
    63 & agreeable\_jj & 0 & 1741 \\
    64 & deep\_jj & 0 & 2700 \\
    65 & steady\_jj & 0 & 3022 \\
    66 & vigorous\_jj & 0 & 3224 \\
    67 & active\_jj & 0 & 3631 \\
    68 & trustful\_jj & 0 & 3697 \\
    69 & cold\_jj & 0 & 3736 \\
    70 & unsympathetic\_jj & 0 & 3832 \\
    71 & thorough\_jj & 0 & 3999 \\
    72 & reserved\_jj & 0 & 4115 \\
    73 & high-strung\_jj & 0 & 4334 \\
    74 & unemotional\_jj & 0 & 4747 \\
    75 & unsophisticated\_jj & 0 & 5042 \\
    76 & fretful\_jj & 0 & 5099 \\
    77 & unrestrained\_jj & 0 & 5200 \\
    78 & undependable\_jj & 0 & 5367 \\
    79 & systematic\_jj & 0 & 5436 \\
    80 & haphazard\_jj & 0 & 5720 \\
    81 & withdrawn\_jj & 0 & 5863 \\
    82 & inefficient\_jj & 0 & 5873 \\
    83 & assertive\_jj & 0 & 6074 \\
    84 & negligent\_jj & 0 & 6624 \\
    85 & self-pitying\_jj & 0 & 7405 \\
    86 & organized\_jj & 0 & 7945 \\
    87 & conscientious\_jj & 1 & 845 \\
    88 & warm\_jj & 1 & 1433 \\
    89 & unreflective\_jj & 1 & 3184 \\
    90 & distrustful\_jj & 1 & 5819 \\
    \hline
    \caption{Scores and rankings for most extreme 30 words in component \#12} \\
\end{longtable}
\begin{longtable}[!htbp]{| rlr@{.}l |}
    \hline
    \textbf{Rank} & \textbf{Word} & \multicolumn{2}{c|}{\textbf{Score}} \\
    \hline
    \endhead
    1 & prompt\_jj & -1 & -3391 \\
    2 & neat\_jj & -1 & -1004 \\
    3 & fretful\_jj & 0 & -8700 \\
    4 & unreflective\_jj & 0 & -7426 \\
    5 & bashful\_jj & 0 & -7316 \\
    6 & unsophisticated\_jj & 0 & -7158 \\
    7 & high-strung\_jj & 0 & -6895 \\
    8 & quiet\_jj & 0 & -6626 \\
    9 & pleasant\_jj & 0 & -5572 \\
    10 & rude\_jj & 0 & -5403 \\
    11 & bright\_jj & 0 & -4945 \\
    12 & conscientious\_jj & 0 & -4823 \\
    13 & uncooperative\_jj & 0 & -4518 \\
    14 & distrustful\_jj & 0 & -4516 \\
    15 & unadventurous\_jj & 0 & -4487 \\
    16 & careless\_jj & 0 & -4390 \\
    17 & negligent\_jj & 0 & -4132 \\
    18 & touchy\_jj & 0 & -3716 \\
    19 & envious\_jj & 0 & -3639 \\
    20 & reserved\_jj & 0 & -3590 \\
    21 & helpful\_jj & 0 & -3576 \\
    22 & efficient\_jj & 0 & -3452 \\
    23 & simple\_jj & 0 & -3303 \\
    24 & agreeable\_jj & 0 & -3193 \\
    25 & impractical\_jj & 0 & -3054 \\
    26 & innovative\_jj & 0 & -2916 \\
    27 & generous\_jj & 0 & -2694 \\
    28 & undemanding\_jj & 0 & -2690 \\
    29 & moody\_jj & 0 & -2661 \\
    30 & anxious\_jj & 0 & -2576 \\
    61 & energetic\_jj & 0 & 1055 \\
    62 & complex\_jj & 0 & 1492 \\
    63 & philosophical\_jj & 0 & 1529 \\
    64 & creative\_jj & 0 & 2389 \\
    65 & self-pitying\_jj & 0 & 2788 \\
    66 & extroverted\_jj & 0 & 3046 \\
    67 & careful\_jj & 0 & 3092 \\
    68 & cooperative\_jj & 0 & 3464 \\
    69 & kind\_jj & 0 & 3629 \\
    70 & unimaginative\_jj & 0 & 3701 \\
    71 & deep\_jj & 0 & 3810 \\
    72 & sloppy\_jj & 0 & 3860 \\
    73 & inconsistent\_jj & 0 & 3924 \\
    74 & cold\_jj & 0 & 4091 \\
    75 & organized\_jj & 0 & 4308 \\
    76 & harsh\_jj & 0 & 4336 \\
    77 & considerate\_jj & 0 & 4507 \\
    78 & vigorous\_jj & 0 & 4600 \\
    79 & intellectual\_jj & 0 & 5055 \\
    80 & emotional\_jj & 0 & 6289 \\
    81 & warm\_jj & 0 & 6305 \\
    82 & haphazard\_jj & 0 & 6520 \\
    83 & shallow\_jj & 0 & 7418 \\
    84 & systematic\_jj & 0 & 8180 \\
    85 & verbal\_jj & 1 & 6 \\
    86 & irritable\_jj & 1 & 473 \\
    87 & thorough\_jj & 1 & 907 \\
    88 & unemotional\_jj & 1 & 1704 \\
    89 & unintelligent\_jj & 1 & 4257 \\
    90 & selfish\_jj & 1 & 5670 \\
    \hline
    \caption{Scores and rankings for most extreme 30 words in component \#13} \\
\end{longtable}
\begin{longtable}[!htbp]{| rlr@{.}l |}
    \hline
    \textbf{Rank} & \textbf{Word} & \multicolumn{2}{c|}{\textbf{Score}} \\
    \hline
    \endhead
    1 & rude\_jj & -1 & -8706 \\
    2 & intellectual\_jj & 0 & -9375 \\
    3 & unsophisticated\_jj & 0 & -9286 \\
    4 & efficient\_jj & 0 & -9185 \\
    5 & verbal\_jj & 0 & -6447 \\
    6 & unintelligent\_jj & 0 & -5991 \\
    7 & self-pitying\_jj & 0 & -5736 \\
    8 & cold\_jj & 0 & -5574 \\
    9 & uncharitable\_jj & 0 & -5504 \\
    10 & generous\_jj & 0 & -5417 \\
    11 & undependable\_jj & 0 & -5342 \\
    12 & envious\_jj & 0 & -4895 \\
    13 & unreflective\_jj & 0 & -4893 \\
    14 & innovative\_jj & 0 & -4069 \\
    15 & considerate\_jj & 0 & -3677 \\
    16 & harsh\_jj & 0 & -3623 \\
    17 & organized\_jj & 0 & -3503 \\
    18 & kind\_jj & 0 & -3470 \\
    19 & thorough\_jj & 0 & -3261 \\
    20 & systematic\_jj & 0 & -3085 \\
    21 & emotional\_jj & 0 & -3032 \\
    22 & warm\_jj & 0 & -2332 \\
    23 & shy\_jj & 0 & -2291 \\
    24 & unemotional\_jj & 0 & -2274 \\
    25 & jealous\_jj & 0 & -1965 \\
    26 & negligent\_jj & 0 & -1957 \\
    27 & anxious\_jj & 0 & -1844 \\
    28 & unexcitable\_jj & 0 & -1825 \\
    29 & inefficient\_jj & 0 & -1774 \\
    30 & daring\_jj & 0 & -1529 \\
    61 & helpful\_jj & 0 & 1310 \\
    62 & demanding\_jj & 0 & 1325 \\
    63 & fretful\_jj & 0 & 1372 \\
    64 & agreeable\_jj & 0 & 1483 \\
    65 & sympathetic\_jj & 0 & 1673 \\
    66 & introspective\_jj & 0 & 1976 \\
    67 & prompt\_jj & 0 & 2045 \\
    68 & uncooperative\_jj & 0 & 2159 \\
    69 & simple\_jj & 0 & 2246 \\
    70 & trustful\_jj & 0 & 2342 \\
    71 & artistic\_jj & 0 & 2610 \\
    72 & bashful\_jj & 0 & 2932 \\
    73 & shallow\_jj & 0 & 3112 \\
    74 & cooperative\_jj & 0 & 3112 \\
    75 & vigorous\_jj & 0 & 3279 \\
    76 & moody\_jj & 0 & 3640 \\
    77 & inconsistent\_jj & 0 & 3687 \\
    78 & philosophical\_jj & 0 & 4201 \\
    79 & high-strung\_jj & 0 & 4516 \\
    80 & conscientious\_jj & 0 & 5387 \\
    81 & reserved\_jj & 0 & 5421 \\
    82 & haphazard\_jj & 0 & 5447 \\
    83 & unimaginative\_jj & 0 & 5630 \\
    84 & timid\_jj & 0 & 6071 \\
    85 & assertive\_jj & 0 & 6231 \\
    86 & careless\_jj & 0 & 7827 \\
    87 & touchy\_jj & 0 & 8475 \\
    88 & irritable\_jj & 1 & 1293 \\
    89 & neat\_jj & 1 & 4408 \\
    90 & selfish\_jj & 1 & 9429 \\
    \hline
    \caption{Scores and rankings for most extreme 30 words in component \#14} \\
\end{longtable}
\begin{longtable}[!htbp]{| rlr@{.}l |}
    \hline
    \textbf{Rank} & \textbf{Word} & \multicolumn{2}{c|}{\textbf{Score}} \\
    \hline
    \endhead
    1 & rude\_jj & -1 & -5669 \\
    2 & unemotional\_jj & -1 & -1299 \\
    3 & assertive\_jj & -1 & -403 \\
    4 & selfish\_jj & 0 & -7746 \\
    5 & high-strung\_jj & 0 & -7587 \\
    6 & bold\_jj & 0 & -7515 \\
    7 & moody\_jj & 0 & -7221 \\
    8 & demanding\_jj & 0 & -7074 \\
    9 & distrustful\_jj & 0 & -6527 \\
    10 & withdrawn\_jj & 0 & -5861 \\
    11 & timid\_jj & 0 & -5546 \\
    12 & reserved\_jj & 0 & -5337 \\
    13 & daring\_jj & 0 & -5322 \\
    14 & bright\_jj & 0 & -4823 \\
    15 & energetic\_jj & 0 & -4583 \\
    16 & introspective\_jj & 0 & -4411 \\
    17 & anxious\_jj & 0 & -3697 \\
    18 & harsh\_jj & 0 & -3428 \\
    19 & negligent\_jj & 0 & -2919 \\
    20 & temperamental\_jj & 0 & -2845 \\
    21 & warm\_jj & 0 & -2653 \\
    22 & self-pitying\_jj & 0 & -2561 \\
    23 & quiet\_jj & 0 & -2516 \\
    24 & complex\_jj & 0 & -2496 \\
    25 & insecure\_jj & 0 & -2402 \\
    26 & cold\_jj & 0 & -2191 \\
    27 & fretful\_jj & 0 & -2016 \\
    28 & inefficient\_jj & 0 & -1983 \\
    29 & impractical\_jj & 0 & -1978 \\
    30 & unsophisticated\_jj & 0 & -1832 \\
    61 & innovative\_jj & 0 & 1829 \\
    62 & prompt\_jj & 0 & 2021 \\
    63 & unexcitable\_jj & 0 & 2119 \\
    64 & undemanding\_jj & 0 & 2240 \\
    65 & touchy\_jj & 0 & 2447 \\
    66 & jealous\_jj & 0 & 2855 \\
    67 & systematic\_jj & 0 & 3088 \\
    68 & pleasant\_jj & 0 & 3125 \\
    69 & unsystematic\_jj & 0 & 3771 \\
    70 & unadventurous\_jj & 0 & 3890 \\
    71 & extroverted\_jj & 0 & 3922 \\
    72 & talkative\_jj & 0 & 4177 \\
    73 & trustful\_jj & 0 & 4374 \\
    74 & helpful\_jj & 0 & 4466 \\
    75 & kind\_jj & 0 & 4582 \\
    76 & thorough\_jj & 0 & 4760 \\
    77 & uncooperative\_jj & 0 & 4807 \\
    78 & bashful\_jj & 0 & 5325 \\
    79 & careless\_jj & 0 & 5383 \\
    80 & neat\_jj & 0 & 5620 \\
    81 & unrestrained\_jj & 0 & 6222 \\
    82 & unintelligent\_jj & 0 & 6250 \\
    83 & considerate\_jj & 0 & 6924 \\
    84 & organized\_jj & 0 & 6966 \\
    85 & envious\_jj & 0 & 7394 \\
    86 & uncharitable\_jj & 0 & 8137 \\
    87 & unkind\_jj & 0 & 9967 \\
    88 & irritable\_jj & 1 & 231 \\
    89 & steady\_jj & 1 & 937 \\
    90 & unreflective\_jj & 1 & 2110 \\
    \hline
    \caption{Scores and rankings for most extreme 30 words in component \#15} \\
\end{longtable}

\subsection{Normalized PCA}
\label{app:rankedwordlists:101words:normalized}
\begin{longtable}[!htbp]{| rlr@{.}l |}
    \hline
    \textbf{Rank} & \textbf{Word} & \multicolumn{2}{c|}{\textbf{Score}} \\
    \hline
    \endhead
    1 & irritable\_jj & -11 & -9901 \\
    2 & uncooperative\_jj & -11 & -1484 \\
    3 & self-pitying\_jj & -11 & -412 \\
    4 & extroverted\_jj & -10 & -7342 \\
    5 & unintelligent\_jj & -10 & -104 \\
    6 & talkative\_jj & -9 & -9943 \\
    7 & rude\_jj & -9 & -2815 \\
    8 & uncharitable\_jj & -8 & -283 \\
    9 & unkind\_jj & -7 & -9745 \\
    10 & withdrawn\_jj & -7 & -8319 \\
    11 & bashful\_jj & -7 & -6757 \\
    12 & selfish\_jj & -7 & -6259 \\
    13 & envious\_jj & -7 & -6000 \\
    14 & introspective\_jj & -7 & -3661 \\
    15 & jealous\_jj & -7 & -733 \\
    16 & high-strung\_jj & -6 & -8977 \\
    17 & considerate\_jj & -6 & -6901 \\
    18 & distrustful\_jj & -6 & -5143 \\
    19 & unsympathetic\_jj & -6 & -4742 \\
    20 & insecure\_jj & -6 & -4229 \\
    21 & unemotional\_jj & -6 & -2868 \\
    22 & kind\_jj & -5 & -4732 \\
    23 & unsophisticated\_jj & -5 & -118 \\
    24 & fretful\_jj & -4 & -7490 \\
    25 & unreflective\_jj & -4 & -4690 \\
    26 & moody\_jj & -4 & -1017 \\
    27 & temperamental\_jj & -3 & -6817 \\
    28 & unadventurous\_jj & -3 & -5340 \\
    29 & unimaginative\_jj & -2 & -7723 \\
    30 & conscientious\_jj & -2 & -4856 \\
    61 & shallow\_jj & 3 & 7977 \\
    62 & active\_jj & 3 & 9366 \\
    63 & inconsistent\_jj & 3 & 9951 \\
    64 & cold\_jj & 4 & 1068 \\
    65 & inefficient\_jj & 4 & 1778 \\
    66 & haphazard\_jj & 4 & 2052 \\
    67 & artistic\_jj & 4 & 3094 \\
    68 & sloppy\_jj & 4 & 5398 \\
    69 & cooperative\_jj & 5 & 1955 \\
    70 & intellectual\_jj & 5 & 3386 \\
    71 & imperturbable\_jj & 5 & 3734 \\
    72 & deep\_jj & 5 & 4724 \\
    73 & bold\_jj & 5 & 6458 \\
    74 & helpful\_jj & 5 & 8953 \\
    75 & creative\_jj & 6 & 1357 \\
    76 & careful\_jj & 6 & 2272 \\
    77 & complex\_jj & 6 & 7324 \\
    78 & negligent\_jj & 6 & 8166 \\
    79 & neat\_jj & 6 & 8351 \\
    80 & warm\_jj & 6 & 8638 \\
    81 & vigorous\_jj & 7 & 2895 \\
    82 & practical\_jj & 7 & 3226 \\
    83 & organized\_jj & 7 & 4149 \\
    84 & steady\_jj & 7 & 5816 \\
    85 & simple\_jj & 7 & 8902 \\
    86 & systematic\_jj & 8 & 3008 \\
    87 & innovative\_jj & 8 & 3934 \\
    88 & efficient\_jj & 9 & 8786 \\
    89 & prompt\_jj & 10 & 4121 \\
    90 & thorough\_jj & 10 & 8417 \\
    \hline
    \caption{Scores and rankings for most extreme 30 words in component \#1} \\
\end{longtable}
\begin{longtable}[!htbp]{| rlr@{.}l |}
    \hline
    \textbf{Rank} & \textbf{Word} & \multicolumn{2}{c|}{\textbf{Score}} \\
    \hline
    \endhead
    1 & uncooperative\_jj & -10 & -6071 \\
    2 & prompt\_jj & -7 & -788 \\
    3 & negligent\_jj & -6 & -7928 \\
    4 & systematic\_jj & -5 & -2588 \\
    5 & distrustful\_jj & -4 & -9348 \\
    6 & inconsistent\_jj & -4 & -8934 \\
    7 & insecure\_jj & -4 & -3282 \\
    8 & fearful\_jj & -4 & -45 \\
    9 & impractical\_jj & -3 & -9667 \\
    10 & inefficient\_jj & -3 & -7114 \\
    11 & unsophisticated\_jj & -3 & -2640 \\
    12 & cooperative\_jj & -3 & -1722 \\
    13 & sloppy\_jj & -3 & -132 \\
    14 & helpful\_jj & -2 & -7725 \\
    15 & rude\_jj & -2 & -7122 \\
    16 & unsympathetic\_jj & -2 & -6072 \\
    17 & unintelligent\_jj & -2 & -5833 \\
    18 & thorough\_jj & -2 & -5661 \\
    19 & organized\_jj & -2 & -3803 \\
    20 & haphazard\_jj & -2 & -2099 \\
    21 & touchy\_jj & -2 & -1550 \\
    22 & harsh\_jj & -2 & -1446 \\
    23 & practical\_jj & -2 & -945 \\
    24 & withdrawn\_jj & -2 & -834 \\
    25 & jealous\_jj & -2 & -204 \\
    26 & complex\_jj & -1 & -9550 \\
    27 & undependable\_jj & -1 & -9464 \\
    28 & unsystematic\_jj & -1 & -9339 \\
    29 & careless\_jj & -1 & -8142 \\
    30 & irritable\_jj & -1 & -7878 \\
    61 & philosophical\_jj & 1 & 103 \\
    62 & warm\_jj & 1 & 633 \\
    63 & neat\_jj & 1 & 653 \\
    64 & bashful\_jj & 1 & 1872 \\
    65 & unadventurous\_jj & 1 & 2145 \\
    66 & agreeable\_jj & 1 & 3032 \\
    67 & bold\_jj & 1 & 3101 \\
    68 & unkind\_jj & 1 & 3124 \\
    69 & reserved\_jj & 1 & 4051 \\
    70 & vigorous\_jj & 1 & 4204 \\
    71 & assertive\_jj & 1 & 4916 \\
    72 & envious\_jj & 1 & 5406 \\
    73 & kind\_jj & 1 & 6729 \\
    74 & shy\_jj & 1 & 7517 \\
    75 & temperamental\_jj & 1 & 9229 \\
    76 & self-pitying\_jj & 2 & 1686 \\
    77 & unreflective\_jj & 2 & 2286 \\
    78 & imaginative\_jj & 2 & 4784 \\
    79 & moody\_jj & 2 & 5054 \\
    80 & artistic\_jj & 2 & 5107 \\
    81 & high-strung\_jj & 2 & 5359 \\
    82 & pleasant\_jj & 2 & 5950 \\
    83 & quiet\_jj & 2 & 6592 \\
    84 & uncharitable\_jj & 2 & 9446 \\
    85 & talkative\_jj & 3 & 3499 \\
    86 & extroverted\_jj & 3 & 5867 \\
    87 & introspective\_jj & 4 & 7056 \\
    88 & energetic\_jj & 5 & 1384 \\
    89 & considerate\_jj & 5 & 5019 \\
    90 & imperturbable\_jj & 48 & 341 \\
    \hline
    \caption{Scores and rankings for most extreme 30 words in component \#2} \\
\end{longtable}
\begin{longtable}[!htbp]{| rlr@{.}l |}
    \hline
    \textbf{Rank} & \textbf{Word} & \multicolumn{2}{c|}{\textbf{Score}} \\
    \hline
    \endhead
    1 & negligent\_jj & -20 & -2453 \\
    2 & uncooperative\_jj & -17 & -1413 \\
    3 & imperturbable\_jj & -16 & -9674 \\
    4 & selfish\_jj & -9 & -3985 \\
    5 & careless\_jj & -9 & -1898 \\
    6 & prompt\_jj & -7 & -9598 \\
    7 & unreflective\_jj & -7 & -9005 \\
    8 & self-pitying\_jj & -6 & -9737 \\
    9 & sloppy\_jj & -6 & -7212 \\
    10 & inefficient\_jj & -6 & -7147 \\
    11 & unrestrained\_jj & -6 & -6224 \\
    12 & inconsistent\_jj & -5 & -4843 \\
    13 & systematic\_jj & -5 & -531 \\
    14 & verbal\_jj & -4 & -4002 \\
    15 & uncharitable\_jj & -4 & -2732 \\
    16 & unimaginative\_jj & -4 & -29 \\
    17 & rude\_jj & -3 & -9503 \\
    18 & unintelligent\_jj & -3 & -8747 \\
    19 & haphazard\_jj & -3 & -7511 \\
    20 & unsophisticated\_jj & -3 & -5659 \\
    21 & unsystematic\_jj & -3 & -974 \\
    22 & impractical\_jj & -2 & -8732 \\
    23 & unsympathetic\_jj & -2 & -7738 \\
    24 & unkind\_jj & -2 & -2395 \\
    25 & harsh\_jj & -1 & -6963 \\
    26 & distrustful\_jj & -1 & -6030 \\
    27 & unemotional\_jj & -1 & -5141 \\
    28 & fretful\_jj & -1 & -1968 \\
    29 & unexcitable\_jj & -1 & -1673 \\
    30 & thorough\_jj & -1 & -845 \\
    61 & irritable\_jj & 2 & 641 \\
    62 & cooperative\_jj & 2 & 860 \\
    63 & envious\_jj & 2 & 2456 \\
    64 & practical\_jj & 2 & 2999 \\
    65 & helpful\_jj & 2 & 3893 \\
    66 & efficient\_jj & 2 & 5559 \\
    67 & careful\_jj & 2 & 6397 \\
    68 & moody\_jj & 3 & 1538 \\
    69 & high-strung\_jj & 3 & 2301 \\
    70 & steady\_jj & 3 & 3619 \\
    71 & cold\_jj & 3 & 4323 \\
    72 & active\_jj & 3 & 4928 \\
    73 & shy\_jj & 3 & 5079 \\
    74 & agreeable\_jj & 3 & 6450 \\
    75 & neat\_jj & 4 & 1616 \\
    76 & creative\_jj & 4 & 3682 \\
    77 & conscientious\_jj & 4 & 5911 \\
    78 & bashful\_jj & 4 & 6060 \\
    79 & introspective\_jj & 4 & 7538 \\
    80 & reserved\_jj & 4 & 7818 \\
    81 & bright\_jj & 4 & 9808 \\
    82 & quiet\_jj & 5 & 6212 \\
    83 & generous\_jj & 5 & 7004 \\
    84 & energetic\_jj & 6 & 4335 \\
    85 & extroverted\_jj & 6 & 6168 \\
    86 & talkative\_jj & 7 & 2828 \\
    87 & pleasant\_jj & 7 & 4310 \\
    88 & kind\_jj & 10 & 5006 \\
    89 & warm\_jj & 13 & 7957 \\
    90 & considerate\_jj & 15 & 944 \\
    \hline
    \caption{Scores and rankings for most extreme 30 words in component \#3} \\
\end{longtable}
\begin{longtable}[!htbp]{| rlr@{.}l |}
    \hline
    \textbf{Rank} & \textbf{Word} & \multicolumn{2}{c|}{\textbf{Score}} \\
    \hline
    \endhead
    1 & cold\_jj & -9 & -6279 \\
    2 & unreflective\_jj & -9 & -2879 \\
    3 & neat\_jj & -8 & -8271 \\
    4 & self-pitying\_jj & -6 & -7402 \\
    5 & deep\_jj & -6 & -6720 \\
    6 & warm\_jj & -6 & -5609 \\
    7 & shallow\_jj & -5 & -4677 \\
    8 & uncharitable\_jj & -4 & -6381 \\
    9 & harsh\_jj & -4 & -5820 \\
    10 & unintelligent\_jj & -4 & -490 \\
    11 & verbal\_jj & -3 & -9706 \\
    12 & steady\_jj & -3 & -9407 \\
    13 & philosophical\_jj & -3 & -9297 \\
    14 & artistic\_jj & -3 & -8398 \\
    15 & fretful\_jj & -3 & -7135 \\
    16 & undemanding\_jj & -3 & -6374 \\
    17 & bright\_jj & -3 & -4522 \\
    18 & irritable\_jj & -3 & -3631 \\
    19 & envious\_jj & -3 & -1743 \\
    20 & jealous\_jj & -3 & -1385 \\
    21 & simple\_jj & -3 & -592 \\
    22 & unkind\_jj & -2 & -8459 \\
    23 & moody\_jj & -2 & -7683 \\
    24 & sloppy\_jj & -2 & -7592 \\
    25 & unexcitable\_jj & -2 & -7158 \\
    26 & unadventurous\_jj & -2 & -6671 \\
    27 & rude\_jj & -2 & -6088 \\
    28 & insecure\_jj & -2 & -5234 \\
    29 & fearful\_jj & -2 & -4787 \\
    30 & emotional\_jj & -2 & -3471 \\
    61 & unemotional\_jj & 1 & 535 \\
    62 & impractical\_jj & 1 & 3297 \\
    63 & timid\_jj & 1 & 4867 \\
    64 & innovative\_jj & 1 & 7083 \\
    65 & sympathetic\_jj & 1 & 8889 \\
    66 & organized\_jj & 1 & 8910 \\
    67 & inefficient\_jj & 2 & 1496 \\
    68 & reserved\_jj & 2 & 2708 \\
    69 & unsympathetic\_jj & 2 & 3847 \\
    70 & demanding\_jj & 2 & 4056 \\
    71 & careful\_jj & 2 & 4679 \\
    72 & helpful\_jj & 2 & 4794 \\
    73 & extroverted\_jj & 2 & 5339 \\
    74 & agreeable\_jj & 2 & 8269 \\
    75 & active\_jj & 2 & 8645 \\
    76 & assertive\_jj & 2 & 9110 \\
    77 & generous\_jj & 3 & 2164 \\
    78 & systematic\_jj & 3 & 2919 \\
    79 & imperturbable\_jj & 3 & 5972 \\
    80 & prompt\_jj & 4 & 2275 \\
    81 & energetic\_jj & 5 & 2893 \\
    82 & efficient\_jj & 5 & 8868 \\
    83 & conscientious\_jj & 7 & 7542 \\
    84 & thorough\_jj & 7 & 9661 \\
    85 & cooperative\_jj & 7 & 9803 \\
    86 & talkative\_jj & 9 & 3140 \\
    87 & kind\_jj & 9 & 5319 \\
    88 & negligent\_jj & 12 & 12 \\
    89 & uncooperative\_jj & 15 & 2763 \\
    90 & considerate\_jj & 20 & 2624 \\
    \hline
    \caption{Scores and rankings for most extreme 30 words in component \#4} \\
\end{longtable}
\begin{longtable}[!htbp]{| rlr@{.}l |}
    \hline
    \textbf{Rank} & \textbf{Word} & \multicolumn{2}{c|}{\textbf{Score}} \\
    \hline
    \endhead
    1 & uncooperative\_jj & -22 & -8612 \\
    2 & warm\_jj & -12 & -1891 \\
    3 & irritable\_jj & -10 & -9639 \\
    4 & cold\_jj & -9 & -3246 \\
    5 & imperturbable\_jj & -7 & -6799 \\
    6 & harsh\_jj & -7 & -2663 \\
    7 & nervous\_jj & -6 & -433 \\
    8 & touchy\_jj & -4 & -6835 \\
    9 & anxious\_jj & -4 & -5700 \\
    10 & organized\_jj & -4 & -935 \\
    11 & active\_jj & -4 & -386 \\
    12 & assertive\_jj & -4 & -268 \\
    13 & fearful\_jj & -3 & -3932 \\
    14 & steady\_jj & -3 & -3552 \\
    15 & verbal\_jj & -3 & -2554 \\
    16 & sympathetic\_jj & -3 & -764 \\
    17 & inconsistent\_jj & -2 & -9076 \\
    18 & vigorous\_jj & -2 & -7055 \\
    19 & cooperative\_jj & -2 & -4749 \\
    20 & deep\_jj & -2 & -3638 \\
    21 & distrustful\_jj & -2 & -2189 \\
    22 & talkative\_jj & -2 & -1508 \\
    23 & helpful\_jj & -2 & -1118 \\
    24 & undemanding\_jj & -2 & -1027 \\
    25 & pleasant\_jj & -2 & -135 \\
    26 & quiet\_jj & -2 & -53 \\
    27 & insecure\_jj & -1 & -9750 \\
    28 & unsympathetic\_jj & -1 & -4645 \\
    29 & complex\_jj & -1 & -4092 \\
    30 & emotional\_jj & -1 & -3008 \\
    61 & bold\_jj & 1 & 9159 \\
    62 & high-strung\_jj & 1 & 9209 \\
    63 & unsophisticated\_jj & 1 & 9470 \\
    64 & haphazard\_jj & 1 & 9613 \\
    65 & practical\_jj & 2 & 316 \\
    66 & innovative\_jj & 2 & 1071 \\
    67 & temperamental\_jj & 2 & 1968 \\
    68 & shy\_jj & 2 & 4022 \\
    69 & unimaginative\_jj & 2 & 4614 \\
    70 & unadventurous\_jj & 2 & 7060 \\
    71 & philosophical\_jj & 2 & 7066 \\
    72 & prompt\_jj & 2 & 8538 \\
    73 & jealous\_jj & 2 & 9643 \\
    74 & kind\_jj & 2 & 9824 \\
    75 & daring\_jj & 3 & 3489 \\
    76 & uncharitable\_jj & 3 & 5387 \\
    77 & energetic\_jj & 3 & 9188 \\
    78 & conscientious\_jj & 4 & 7402 \\
    79 & intellectual\_jj & 4 & 8604 \\
    80 & self-pitying\_jj & 4 & 9173 \\
    81 & imaginative\_jj & 5 & 1623 \\
    82 & unintelligent\_jj & 5 & 5441 \\
    83 & neat\_jj & 5 & 8290 \\
    84 & creative\_jj & 6 & 2261 \\
    85 & artistic\_jj & 6 & 7676 \\
    86 & careless\_jj & 7 & 3657 \\
    87 & unreflective\_jj & 8 & 2122 \\
    88 & considerate\_jj & 9 & 1421 \\
    89 & selfish\_jj & 9 & 9960 \\
    90 & negligent\_jj & 10 & 6421 \\
    \hline
    \caption{Scores and rankings for most extreme 30 words in component \#5} \\
\end{longtable}
\begin{longtable}[!htbp]{| rlr@{.}l |}
    \hline
    \textbf{Rank} & \textbf{Word} & \multicolumn{2}{c|}{\textbf{Score}} \\
    \hline
    \endhead
    1 & uncooperative\_jj & -13 & -2240 \\
    2 & efficient\_jj & -9 & -4857 \\
    3 & innovative\_jj & -8 & -5371 \\
    4 & artistic\_jj & -7 & -621 \\
    5 & unimaginative\_jj & -6 & -8740 \\
    6 & imaginative\_jj & -6 & -7480 \\
    7 & creative\_jj & -5 & -9556 \\
    8 & complex\_jj & -5 & -5965 \\
    9 & inefficient\_jj & -5 & -4223 \\
    10 & unsophisticated\_jj & -5 & -1517 \\
    11 & extroverted\_jj & -4 & -9674 \\
    12 & impractical\_jj & -4 & -9239 \\
    13 & undemanding\_jj & -4 & -8874 \\
    14 & intellectual\_jj & -4 & -6622 \\
    15 & practical\_jj & -4 & -4521 \\
    16 & neat\_jj & -4 & -1263 \\
    17 & introspective\_jj & -4 & -58 \\
    18 & unreflective\_jj & -3 & -8948 \\
    19 & unadventurous\_jj & -3 & -6948 \\
    20 & daring\_jj & -3 & -5413 \\
    21 & unintelligent\_jj & -3 & -3448 \\
    22 & philosophical\_jj & -3 & -3096 \\
    23 & agreeable\_jj & -2 & -8422 \\
    24 & bold\_jj & -2 & -6115 \\
    25 & demanding\_jj & -2 & -5462 \\
    26 & trustful\_jj & -2 & -5354 \\
    27 & high-strung\_jj & -2 & -5019 \\
    28 & undependable\_jj & -2 & -3961 \\
    29 & energetic\_jj & -2 & -1121 \\
    30 & haphazard\_jj & -2 & -1072 \\
    61 & unsympathetic\_jj & 1 & 4958 \\
    62 & envious\_jj & 1 & 6554 \\
    63 & conscientious\_jj & 1 & 6994 \\
    64 & unkind\_jj & 1 & 7481 \\
    65 & pleasant\_jj & 1 & 9188 \\
    66 & withdrawn\_jj & 2 & 1782 \\
    67 & distrustful\_jj & 2 & 2449 \\
    68 & timid\_jj & 2 & 3312 \\
    69 & bright\_jj & 2 & 3507 \\
    70 & imperturbable\_jj & 2 & 5027 \\
    71 & selfish\_jj & 3 & 1607 \\
    72 & rude\_jj & 3 & 8306 \\
    73 & uncharitable\_jj & 3 & 8893 \\
    74 & jealous\_jj & 3 & 9101 \\
    75 & quiet\_jj & 3 & 9799 \\
    76 & careful\_jj & 4 & 1658 \\
    77 & harsh\_jj & 4 & 1809 \\
    78 & considerate\_jj & 5 & 955 \\
    79 & irritable\_jj & 5 & 1872 \\
    80 & fearful\_jj & 5 & 3414 \\
    81 & shy\_jj & 5 & 9268 \\
    82 & steady\_jj & 6 & 773 \\
    83 & anxious\_jj & 6 & 5700 \\
    84 & cold\_jj & 6 & 7646 \\
    85 & nervous\_jj & 7 & 2744 \\
    86 & kind\_jj & 7 & 3470 \\
    87 & careless\_jj & 8 & 7963 \\
    88 & warm\_jj & 9 & 2839 \\
    89 & negligent\_jj & 14 & 4229 \\
    90 & prompt\_jj & 14 & 5246 \\
    \hline
    \caption{Scores and rankings for most extreme 30 words in component \#6} \\
\end{longtable}
\begin{longtable}[!htbp]{| rlr@{.}l |}
    \hline
    \textbf{Rank} & \textbf{Word} & \multicolumn{2}{c|}{\textbf{Score}} \\
    \hline
    \endhead
    1 & distrustful\_jj & -11 & -6201 \\
    2 & assertive\_jj & -10 & -3338 \\
    3 & irritable\_jj & -8 & -3611 \\
    4 & fearful\_jj & -7 & -9849 \\
    5 & anxious\_jj & -7 & -4156 \\
    6 & nervous\_jj & -6 & -6464 \\
    7 & insecure\_jj & -6 & -18 \\
    8 & organized\_jj & -5 & -17 \\
    9 & active\_jj & -4 & -9944 \\
    10 & unsympathetic\_jj & -4 & -6491 \\
    11 & fretful\_jj & -4 & -5660 \\
    12 & conscientious\_jj & -4 & -2058 \\
    13 & emotional\_jj & -4 & -1747 \\
    14 & demanding\_jj & -4 & -777 \\
    15 & vigorous\_jj & -4 & -682 \\
    16 & intellectual\_jj & -3 & -9418 \\
    17 & careful\_jj & -3 & -8258 \\
    18 & sympathetic\_jj & -3 & -1867 \\
    19 & prompt\_jj & -2 & -8041 \\
    20 & envious\_jj & -2 & -6777 \\
    21 & timid\_jj & -2 & -6400 \\
    22 & unrestrained\_jj & -2 & -6288 \\
    23 & uncharitable\_jj & -2 & -3829 \\
    24 & touchy\_jj & -2 & -3586 \\
    25 & undependable\_jj & -2 & -3127 \\
    26 & artistic\_jj & -2 & -1339 \\
    27 & complex\_jj & -1 & -9141 \\
    28 & jealous\_jj & -1 & -8949 \\
    29 & imperturbable\_jj & -1 & -8549 \\
    30 & helpful\_jj & -1 & -4307 \\
    61 & bashful\_jj & 1 & 159 \\
    62 & haphazard\_jj & 1 & 4087 \\
    63 & moody\_jj & 1 & 6629 \\
    64 & unimaginative\_jj & 1 & 6737 \\
    65 & kind\_jj & 1 & 7110 \\
    66 & unreflective\_jj & 1 & 8086 \\
    67 & selfish\_jj & 1 & 9974 \\
    68 & impractical\_jj & 2 & 1152 \\
    69 & simple\_jj & 2 & 2149 \\
    70 & sloppy\_jj & 2 & 9860 \\
    71 & unkind\_jj & 3 & 183 \\
    72 & quiet\_jj & 3 & 1462 \\
    73 & unemotional\_jj & 3 & 7038 \\
    74 & self-pitying\_jj & 3 & 9682 \\
    75 & rude\_jj & 4 & 737 \\
    76 & unintelligent\_jj & 4 & 2260 \\
    77 & considerate\_jj & 4 & 3956 \\
    78 & undemanding\_jj & 4 & 4028 \\
    79 & careless\_jj & 4 & 4936 \\
    80 & reserved\_jj & 4 & 6339 \\
    81 & talkative\_jj & 5 & 2457 \\
    82 & bright\_jj & 5 & 7624 \\
    83 & cold\_jj & 5 & 8623 \\
    84 & shallow\_jj & 6 & 1564 \\
    85 & high-strung\_jj & 6 & 5408 \\
    86 & pleasant\_jj & 6 & 8812 \\
    87 & negligent\_jj & 7 & 9303 \\
    88 & neat\_jj & 8 & 7898 \\
    89 & uncooperative\_jj & 12 & 2259 \\
    90 & warm\_jj & 17 & 8307 \\
    \hline
    \caption{Scores and rankings for most extreme 30 words in component \#7} \\
\end{longtable}
\begin{longtable}[!htbp]{| rlr@{.}l |}
    \hline
    \textbf{Rank} & \textbf{Word} & \multicolumn{2}{c|}{\textbf{Score}} \\
    \hline
    \endhead
    1 & prompt\_jj & -30 & -9518 \\
    2 & unemotional\_jj & -9 & -6086 \\
    3 & thorough\_jj & -6 & -4513 \\
    4 & self-pitying\_jj & -6 & -1468 \\
    5 & withdrawn\_jj & -5 & -6023 \\
    6 & high-strung\_jj & -5 & -1216 \\
    7 & extroverted\_jj & -4 & -8506 \\
    8 & verbal\_jj & -4 & -4773 \\
    9 & introspective\_jj & -3 & -5346 \\
    10 & vigorous\_jj & -3 & -4678 \\
    11 & daring\_jj & -3 & -759 \\
    12 & unintelligent\_jj & -3 & -129 \\
    13 & systematic\_jj & -2 & -1948 \\
    14 & philosophical\_jj & -2 & -1739 \\
    15 & imaginative\_jj & -2 & -1597 \\
    16 & reserved\_jj & -2 & -1108 \\
    17 & bold\_jj & -2 & -217 \\
    18 & bashful\_jj & -1 & -9947 \\
    19 & unreflective\_jj & -1 & -9812 \\
    20 & uncooperative\_jj & -1 & -6388 \\
    21 & moody\_jj & -1 & -5048 \\
    22 & considerate\_jj & -1 & -4679 \\
    23 & practical\_jj & -1 & -4495 \\
    24 & warm\_jj & -1 & -1987 \\
    25 & unrestrained\_jj & 0 & -9697 \\
    26 & irritable\_jj & 0 & -6967 \\
    27 & steady\_jj & 0 & -6747 \\
    28 & artistic\_jj & 0 & -5879 \\
    29 & demanding\_jj & 0 & -5364 \\
    30 & trustful\_jj & 0 & -4962 \\
    61 & quiet\_jj & 1 & 9309 \\
    62 & creative\_jj & 1 & 9694 \\
    63 & fretful\_jj & 1 & 9957 \\
    64 & careful\_jj & 1 & 9964 \\
    65 & pleasant\_jj & 2 & 495 \\
    66 & neat\_jj & 2 & 986 \\
    67 & shallow\_jj & 2 & 1048 \\
    68 & distrustful\_jj & 2 & 1459 \\
    69 & impractical\_jj & 2 & 2222 \\
    70 & timid\_jj & 2 & 2280 \\
    71 & touchy\_jj & 2 & 3591 \\
    72 & fearful\_jj & 2 & 3598 \\
    73 & selfish\_jj & 2 & 4234 \\
    74 & bright\_jj & 2 & 4280 \\
    75 & cold\_jj & 2 & 4441 \\
    76 & envious\_jj & 2 & 5607 \\
    77 & unimaginative\_jj & 2 & 5825 \\
    78 & helpful\_jj & 2 & 6169 \\
    79 & organized\_jj & 2 & 8847 \\
    80 & generous\_jj & 2 & 9104 \\
    81 & insecure\_jj & 3 & 567 \\
    82 & nervous\_jj & 3 & 1501 \\
    83 & anxious\_jj & 3 & 3727 \\
    84 & jealous\_jj & 3 & 7592 \\
    85 & inefficient\_jj & 4 & 647 \\
    86 & kind\_jj & 4 & 1364 \\
    87 & active\_jj & 4 & 3314 \\
    88 & shy\_jj & 4 & 6153 \\
    89 & careless\_jj & 4 & 7547 \\
    90 & negligent\_jj & 14 & 8365 \\
    \hline
    \caption{Scores and rankings for most extreme 30 words in component \#8} \\
\end{longtable}
\begin{longtable}[!htbp]{| rlr@{.}l |}
    \hline
    \textbf{Rank} & \textbf{Word} & \multicolumn{2}{c|}{\textbf{Score}} \\
    \hline
    \endhead
    1 & negligent\_jj & -16 & -3796 \\
    2 & artistic\_jj & -7 & -9021 \\
    3 & verbal\_jj & -7 & -2136 \\
    4 & irritable\_jj & -6 & -5079 \\
    5 & unemotional\_jj & -6 & -4507 \\
    6 & high-strung\_jj & -6 & -1354 \\
    7 & emotional\_jj & -5 & -6099 \\
    8 & intellectual\_jj & -5 & -3362 \\
    9 & introspective\_jj & -4 & -9745 \\
    10 & creative\_jj & -4 & -6833 \\
    11 & unreflective\_jj & -4 & -4872 \\
    12 & extroverted\_jj & -4 & -4172 \\
    13 & uncooperative\_jj & -3 & -5932 \\
    14 & imaginative\_jj & -3 & -5745 \\
    15 & vigorous\_jj & -3 & -5620 \\
    16 & fretful\_jj & -3 & -5278 \\
    17 & philosophical\_jj & -3 & -4886 \\
    18 & daring\_jj & -3 & -3790 \\
    19 & moody\_jj & -3 & -1295 \\
    20 & temperamental\_jj & -2 & -4661 \\
    21 & demanding\_jj & -2 & -4650 \\
    22 & energetic\_jj & -2 & -2503 \\
    23 & bashful\_jj & -2 & -2498 \\
    24 & anxious\_jj & -2 & -1127 \\
    25 & reserved\_jj & -2 & -1124 \\
    26 & conscientious\_jj & -1 & -9160 \\
    27 & cooperative\_jj & -1 & -7770 \\
    28 & thorough\_jj & -1 & -6379 \\
    29 & withdrawn\_jj & -1 & -5270 \\
    30 & active\_jj & -1 & -4534 \\
    61 & warm\_jj & 1 & 2497 \\
    62 & shy\_jj & 1 & 3254 \\
    63 & unkind\_jj & 1 & 5715 \\
    64 & harsh\_jj & 1 & 5879 \\
    65 & jealous\_jj & 1 & 5979 \\
    66 & fearful\_jj & 1 & 8145 \\
    67 & inconsistent\_jj & 1 & 9783 \\
    68 & unsympathetic\_jj & 2 & 980 \\
    69 & undemanding\_jj & 2 & 1111 \\
    70 & undependable\_jj & 2 & 4845 \\
    71 & shallow\_jj & 2 & 7482 \\
    72 & timid\_jj & 2 & 7507 \\
    73 & neat\_jj & 2 & 9822 \\
    74 & helpful\_jj & 3 & 2015 \\
    75 & insecure\_jj & 3 & 5481 \\
    76 & considerate\_jj & 4 & 720 \\
    77 & imperturbable\_jj & 4 & 4206 \\
    78 & rude\_jj & 4 & 5192 \\
    79 & impractical\_jj & 5 & 4250 \\
    80 & unsophisticated\_jj & 5 & 5990 \\
    81 & generous\_jj & 5 & 7471 \\
    82 & haphazard\_jj & 5 & 8811 \\
    83 & kind\_jj & 6 & 2327 \\
    84 & distrustful\_jj & 6 & 6336 \\
    85 & prompt\_jj & 6 & 7112 \\
    86 & unimaginative\_jj & 6 & 7651 \\
    87 & efficient\_jj & 6 & 9913 \\
    88 & unintelligent\_jj & 7 & 1539 \\
    89 & selfish\_jj & 10 & 6413 \\
    90 & inefficient\_jj & 13 & 6169 \\
    \hline
    \caption{Scores and rankings for most extreme 30 words in component \#9} \\
\end{longtable}
\begin{longtable}[!htbp]{| rlr@{.}l |}
    \hline
    \textbf{Rank} & \textbf{Word} & \multicolumn{2}{c|}{\textbf{Score}} \\
    \hline
    \endhead
    1 & irritable\_jj & -21 & -199 \\
    2 & efficient\_jj & -9 & -5285 \\
    3 & negligent\_jj & -7 & -3946 \\
    4 & inefficient\_jj & -7 & -3819 \\
    5 & unimaginative\_jj & -5 & -5352 \\
    6 & withdrawn\_jj & -5 & -3420 \\
    7 & sloppy\_jj & -4 & -9703 \\
    8 & talkative\_jj & -4 & -5994 \\
    9 & high-strung\_jj & -4 & -2956 \\
    10 & undemanding\_jj & -4 & -2202 \\
    11 & haphazard\_jj & -4 & -1543 \\
    12 & extroverted\_jj & -3 & -6683 \\
    13 & moody\_jj & -3 & -6680 \\
    14 & innovative\_jj & -3 & -6253 \\
    15 & unsophisticated\_jj & -3 & -6092 \\
    16 & bright\_jj & -3 & -18 \\
    17 & neat\_jj & -2 & -8292 \\
    18 & prompt\_jj & -2 & -7291 \\
    19 & nervous\_jj & -2 & -6281 \\
    20 & impractical\_jj & -2 & -5004 \\
    21 & insecure\_jj & -2 & -1194 \\
    22 & undependable\_jj & -1 & -8156 \\
    23 & complex\_jj & -1 & -8125 \\
    24 & demanding\_jj & -1 & -6167 \\
    25 & energetic\_jj & -1 & -6116 \\
    26 & imaginative\_jj & -1 & -4701 \\
    27 & organized\_jj & -1 & -4163 \\
    28 & cold\_jj & -1 & -3671 \\
    29 & introspective\_jj & -1 & -889 \\
    30 & self-pitying\_jj & -1 & -701 \\
    61 & emotional\_jj & 1 & 1522 \\
    62 & envious\_jj & 1 & 3782 \\
    63 & sympathetic\_jj & 1 & 5734 \\
    64 & harsh\_jj & 1 & 6236 \\
    65 & kind\_jj & 1 & 8308 \\
    66 & agreeable\_jj & 1 & 8868 \\
    67 & shallow\_jj & 1 & 9062 \\
    68 & unsympathetic\_jj & 1 & 9225 \\
    69 & fearful\_jj & 2 & 498 \\
    70 & shy\_jj & 2 & 648 \\
    71 & practical\_jj & 2 & 5701 \\
    72 & intellectual\_jj & 2 & 5957 \\
    73 & reserved\_jj & 2 & 6241 \\
    74 & verbal\_jj & 2 & 8036 \\
    75 & quiet\_jj & 2 & 9056 \\
    76 & systematic\_jj & 3 & 185 \\
    77 & warm\_jj & 3 & 721 \\
    78 & deep\_jj & 3 & 5082 \\
    79 & unrestrained\_jj & 3 & 7268 \\
    80 & jealous\_jj & 4 & 2951 \\
    81 & unreflective\_jj & 4 & 4459 \\
    82 & unkind\_jj & 4 & 8132 \\
    83 & artistic\_jj & 4 & 8532 \\
    84 & selfish\_jj & 4 & 8549 \\
    85 & careful\_jj & 5 & 3064 \\
    86 & philosophical\_jj & 7 & 1620 \\
    87 & uncooperative\_jj & 8 & 4442 \\
    88 & distrustful\_jj & 8 & 5216 \\
    89 & conscientious\_jj & 9 & 641 \\
    90 & touchy\_jj & 10 & 9759 \\
    \hline
    \caption{Scores and rankings for most extreme 30 words in component \#10} \\
\end{longtable}
\begin{longtable}[!htbp]{| rlr@{.}l |}
    \hline
    \textbf{Rank} & \textbf{Word} & \multicolumn{2}{c|}{\textbf{Score}} \\
    \hline
    \endhead
    1 & warm\_jj & -10 & -4268 \\
    2 & prompt\_jj & -9 & -2473 \\
    3 & artistic\_jj & -7 & -6206 \\
    4 & selfish\_jj & -6 & -5928 \\
    5 & innovative\_jj & -6 & -1251 \\
    6 & irritable\_jj & -6 & -1233 \\
    7 & unreflective\_jj & -5 & -8300 \\
    8 & negligent\_jj & -5 & -3106 \\
    9 & intellectual\_jj & -5 & -3078 \\
    10 & creative\_jj & -4 & -9932 \\
    11 & impractical\_jj & -4 & -7285 \\
    12 & conscientious\_jj & -3 & -9655 \\
    13 & agreeable\_jj & -3 & -7577 \\
    14 & efficient\_jj & -3 & -7573 \\
    15 & helpful\_jj & -3 & -3706 \\
    16 & cold\_jj & -2 & -9412 \\
    17 & active\_jj & -2 & -8945 \\
    18 & uncharitable\_jj & -2 & -8309 \\
    19 & practical\_jj & -2 & -7539 \\
    20 & unkind\_jj & -2 & -4186 \\
    21 & trustful\_jj & -2 & -3854 \\
    22 & undependable\_jj & -2 & -2757 \\
    23 & uncooperative\_jj & -2 & -1559 \\
    24 & unsympathetic\_jj & -2 & -371 \\
    25 & organized\_jj & -1 & -9567 \\
    26 & imaginative\_jj & -1 & -8891 \\
    27 & pleasant\_jj & -1 & -8565 \\
    28 & emotional\_jj & -1 & -8360 \\
    29 & generous\_jj & -1 & -6533 \\
    30 & considerate\_jj & -1 & -4290 \\
    61 & rude\_jj & 0 & 9239 \\
    62 & introspective\_jj & 1 & 1809 \\
    63 & nervous\_jj & 1 & 2342 \\
    64 & careless\_jj & 1 & 2830 \\
    65 & kind\_jj & 1 & 3782 \\
    66 & unimaginative\_jj & 1 & 7843 \\
    67 & shallow\_jj & 1 & 8624 \\
    68 & quiet\_jj & 2 & 976 \\
    69 & touchy\_jj & 2 & 1211 \\
    70 & jealous\_jj & 2 & 5547 \\
    71 & cooperative\_jj & 2 & 6195 \\
    72 & withdrawn\_jj & 2 & 6464 \\
    73 & inconsistent\_jj & 2 & 7978 \\
    74 & timid\_jj & 3 & 4548 \\
    75 & unsophisticated\_jj & 3 & 6126 \\
    76 & careful\_jj & 3 & 8729 \\
    77 & high-strung\_jj & 3 & 9565 \\
    78 & bashful\_jj & 4 & 2188 \\
    79 & talkative\_jj & 4 & 3638 \\
    80 & self-pitying\_jj & 4 & 5048 \\
    81 & shy\_jj & 4 & 6138 \\
    82 & unemotional\_jj & 4 & 8725 \\
    83 & neat\_jj & 4 & 9214 \\
    84 & vigorous\_jj & 5 & 2683 \\
    85 & verbal\_jj & 5 & 6268 \\
    86 & systematic\_jj & 7 & 423 \\
    87 & steady\_jj & 8 & 5647 \\
    88 & sloppy\_jj & 8 & 6132 \\
    89 & haphazard\_jj & 9 & 1708 \\
    90 & thorough\_jj & 14 & 3107 \\
    \hline
    \caption{Scores and rankings for most extreme 30 words in component \#11} \\
\end{longtable}
\begin{longtable}[!htbp]{| rlr@{.}l |}
    \hline
    \textbf{Rank} & \textbf{Word} & \multicolumn{2}{c|}{\textbf{Score}} \\
    \hline
    \endhead
    1 & rude\_jj & -13 & -2543 \\
    2 & helpful\_jj & -6 & -7303 \\
    3 & verbal\_jj & -6 & -6186 \\
    4 & careful\_jj & -6 & -5014 \\
    5 & selfish\_jj & -6 & -3276 \\
    6 & nervous\_jj & -5 & -2083 \\
    7 & artistic\_jj & -4 & -8880 \\
    8 & careless\_jj & -4 & -7318 \\
    9 & touchy\_jj & -4 & -7307 \\
    10 & shy\_jj & -4 & -6795 \\
    11 & creative\_jj & -4 & -4582 \\
    12 & unkind\_jj & -4 & -2865 \\
    13 & practical\_jj & -4 & -1947 \\
    14 & uncooperative\_jj & -4 & -1897 \\
    15 & emotional\_jj & -4 & -1037 \\
    16 & irritable\_jj & -4 & -391 \\
    17 & uncharitable\_jj & -3 & -3340 \\
    18 & simple\_jj & -3 & -2224 \\
    19 & bashful\_jj & -3 & -191 \\
    20 & innovative\_jj & -2 & -8784 \\
    21 & pleasant\_jj & -2 & -6593 \\
    22 & efficient\_jj & -2 & -4235 \\
    23 & anxious\_jj & -2 & -1317 \\
    24 & imaginative\_jj & -1 & -9090 \\
    25 & jealous\_jj & -1 & -8132 \\
    26 & quiet\_jj & -1 & -7468 \\
    27 & prompt\_jj & -1 & -7231 \\
    28 & kind\_jj & -1 & -6500 \\
    29 & daring\_jj & -1 & -6230 \\
    30 & philosophical\_jj & -1 & -5887 \\
    61 & shallow\_jj & 1 & 2798 \\
    62 & unsystematic\_jj & 1 & 4826 \\
    63 & vigorous\_jj & 1 & 5764 \\
    64 & considerate\_jj & 1 & 6857 \\
    65 & agreeable\_jj & 1 & 7036 \\
    66 & deep\_jj & 1 & 9379 \\
    67 & neat\_jj & 2 & 2854 \\
    68 & cold\_jj & 2 & 4532 \\
    69 & unsympathetic\_jj & 2 & 7685 \\
    70 & active\_jj & 2 & 9077 \\
    71 & trustful\_jj & 3 & 150 \\
    72 & reserved\_jj & 3 & 696 \\
    73 & high-strung\_jj & 3 & 1294 \\
    74 & self-pitying\_jj & 3 & 3064 \\
    75 & unemotional\_jj & 3 & 3433 \\
    76 & systematic\_jj & 3 & 3873 \\
    77 & unrestrained\_jj & 3 & 5045 \\
    78 & fretful\_jj & 3 & 7101 \\
    79 & unsophisticated\_jj & 3 & 7742 \\
    80 & haphazard\_jj & 3 & 7987 \\
    81 & inefficient\_jj & 3 & 8787 \\
    82 & assertive\_jj & 4 & 1331 \\
    83 & undependable\_jj & 4 & 2345 \\
    84 & withdrawn\_jj & 4 & 3498 \\
    85 & negligent\_jj & 4 & 8978 \\
    86 & organized\_jj & 6 & 4934 \\
    87 & conscientious\_jj & 7 & 9886 \\
    88 & warm\_jj & 8 & 6743 \\
    89 & unreflective\_jj & 11 & 860 \\
    90 & distrustful\_jj & 13 & 1051 \\
    \hline
    \caption{Scores and rankings for most extreme 30 words in component \#12} \\
\end{longtable}
\begin{longtable}[!htbp]{| rlr@{.}l |}
    \hline
    \textbf{Rank} & \textbf{Word} & \multicolumn{2}{c|}{\textbf{Score}} \\
    \hline
    \endhead
    1 & prompt\_jj & -8 & -5830 \\
    2 & neat\_jj & -8 & -1952 \\
    3 & fretful\_jj & -6 & -4727 \\
    4 & unsophisticated\_jj & -6 & -4613 \\
    5 & quiet\_jj & -6 & -1480 \\
    6 & high-strung\_jj & -6 & -462 \\
    7 & bashful\_jj & -5 & -9540 \\
    8 & rude\_jj & -5 & -6301 \\
    9 & uncooperative\_jj & -4 & -4244 \\
    10 & bright\_jj & -4 & -1374 \\
    11 & careless\_jj & -3 & -7028 \\
    12 & reserved\_jj & -3 & -4555 \\
    13 & pleasant\_jj & -3 & -3550 \\
    14 & distrustful\_jj & -3 & -2660 \\
    15 & moody\_jj & -2 & -8234 \\
    16 & nervous\_jj & -2 & -7549 \\
    17 & anxious\_jj & -2 & -7394 \\
    18 & envious\_jj & -2 & -6618 \\
    19 & unadventurous\_jj & -2 & -4744 \\
    20 & simple\_jj & -2 & -1946 \\
    21 & shy\_jj & -2 & -848 \\
    22 & conscientious\_jj & -2 & -642 \\
    23 & efficient\_jj & -1 & -8736 \\
    24 & temperamental\_jj & -1 & -7804 \\
    25 & bold\_jj & -1 & -7773 \\
    26 & impractical\_jj & -1 & -7499 \\
    27 & negligent\_jj & -1 & -7276 \\
    28 & timid\_jj & -1 & -6266 \\
    29 & touchy\_jj & -1 & -6236 \\
    30 & unreflective\_jj & -1 & -6038 \\
    61 & complex\_jj & 1 & 1048 \\
    62 & undependable\_jj & 1 & 4201 \\
    63 & unsympathetic\_jj & 1 & 5245 \\
    64 & philosophical\_jj & 1 & 7282 \\
    65 & self-pitying\_jj & 1 & 7831 \\
    66 & careful\_jj & 1 & 7902 \\
    67 & unimaginative\_jj & 2 & 3426 \\
    68 & cooperative\_jj & 2 & 3590 \\
    69 & inconsistent\_jj & 2 & 3653 \\
    70 & creative\_jj & 2 & 3823 \\
    71 & kind\_jj & 2 & 5001 \\
    72 & extroverted\_jj & 2 & 6040 \\
    73 & vigorous\_jj & 2 & 6503 \\
    74 & harsh\_jj & 2 & 7084 \\
    75 & haphazard\_jj & 2 & 9923 \\
    76 & deep\_jj & 3 & 3113 \\
    77 & cold\_jj & 3 & 4763 \\
    78 & organized\_jj & 3 & 4961 \\
    79 & intellectual\_jj & 3 & 8466 \\
    80 & considerate\_jj & 4 & 4577 \\
    81 & emotional\_jj & 5 & 469 \\
    82 & shallow\_jj & 5 & 4670 \\
    83 & systematic\_jj & 5 & 5286 \\
    84 & verbal\_jj & 6 & 210 \\
    85 & warm\_jj & 6 & 5616 \\
    86 & thorough\_jj & 6 & 8019 \\
    87 & unemotional\_jj & 7 & 7385 \\
    88 & irritable\_jj & 8 & 6042 \\
    89 & selfish\_jj & 11 & 7755 \\
    90 & unintelligent\_jj & 12 & 2380 \\
    \hline
    \caption{Scores and rankings for most extreme 30 words in component \#13} \\
\end{longtable}
\begin{longtable}[!htbp]{| rlr@{.}l |}
    \hline
    \textbf{Rank} & \textbf{Word} & \multicolumn{2}{c|}{\textbf{Score}} \\
    \hline
    \endhead
    1 & neat\_jj & -13 & -5960 \\
    2 & selfish\_jj & -11 & -6278 \\
    3 & irritable\_jj & -8 & -3178 \\
    4 & uncooperative\_jj & -6 & -8772 \\
    5 & artistic\_jj & -5 & -7434 \\
    6 & steady\_jj & -4 & -8494 \\
    7 & careless\_jj & -4 & -4855 \\
    8 & jealous\_jj & -3 & -6353 \\
    9 & shallow\_jj & -3 & -5422 \\
    10 & moody\_jj & -3 & -3417 \\
    11 & distrustful\_jj & -3 & -1065 \\
    12 & prompt\_jj & -2 & -9445 \\
    13 & nervous\_jj & -2 & -4908 \\
    14 & haphazard\_jj & -2 & -3514 \\
    15 & shy\_jj & -2 & -1951 \\
    16 & cooperative\_jj & -2 & -241 \\
    17 & bashful\_jj & -2 & -10 \\
    18 & fearful\_jj & -1 & -9336 \\
    19 & creative\_jj & -1 & -8451 \\
    20 & withdrawn\_jj & -1 & -8010 \\
    21 & philosophical\_jj & -1 & -5778 \\
    22 & unrestrained\_jj & -1 & -5717 \\
    23 & trustful\_jj & -1 & -4883 \\
    24 & unsystematic\_jj & -1 & -4231 \\
    25 & kind\_jj & -1 & -1446 \\
    26 & deep\_jj & -1 & -1187 \\
    27 & insecure\_jj & -1 & -1029 \\
    28 & inconsistent\_jj & -1 & -735 \\
    29 & envious\_jj & 0 & -8839 \\
    30 & unimaginative\_jj & 0 & -8227 \\
    61 & agreeable\_jj & 1 & 2239 \\
    62 & conscientious\_jj & 1 & 2662 \\
    63 & touchy\_jj & 1 & 4619 \\
    64 & impractical\_jj & 1 & 5251 \\
    65 & inefficient\_jj & 1 & 6411 \\
    66 & emotional\_jj & 1 & 7877 \\
    67 & thorough\_jj & 2 & 879 \\
    68 & sloppy\_jj & 2 & 1909 \\
    69 & intellectual\_jj & 2 & 2016 \\
    70 & careful\_jj & 2 & 2296 \\
    71 & bold\_jj & 2 & 2715 \\
    72 & verbal\_jj & 2 & 3022 \\
    73 & warm\_jj & 2 & 5277 \\
    74 & unemotional\_jj & 2 & 7447 \\
    75 & undependable\_jj & 2 & 8604 \\
    76 & daring\_jj & 2 & 9468 \\
    77 & assertive\_jj & 3 & 119 \\
    78 & unreflective\_jj & 3 & 826 \\
    79 & extroverted\_jj & 3 & 1254 \\
    80 & self-pitying\_jj & 3 & 2156 \\
    81 & negligent\_jj & 3 & 2554 \\
    82 & generous\_jj & 3 & 5536 \\
    83 & unsympathetic\_jj & 3 & 5647 \\
    84 & uncharitable\_jj & 3 & 6933 \\
    85 & unintelligent\_jj & 3 & 8711 \\
    86 & efficient\_jj & 5 & 2551 \\
    87 & harsh\_jj & 5 & 7813 \\
    88 & cold\_jj & 5 & 8931 \\
    89 & unsophisticated\_jj & 6 & 3717 \\
    90 & rude\_jj & 17 & 890 \\
    \hline
    \caption{Scores and rankings for most extreme 30 words in component \#14} \\
\end{longtable}
\begin{longtable}[!htbp]{| rlr@{.}l |}
    \hline
    \textbf{Rank} & \textbf{Word} & \multicolumn{2}{c|}{\textbf{Score}} \\
    \hline
    \endhead
    1 & touchy\_jj & -18 & -6186 \\
    2 & assertive\_jj & -7 & -957 \\
    3 & irritable\_jj & -5 & -128 \\
    4 & philosophical\_jj & -4 & -7569 \\
    5 & reserved\_jj & -4 & -3912 \\
    6 & conscientious\_jj & -4 & -2765 \\
    7 & helpful\_jj & -4 & -1228 \\
    8 & extroverted\_jj & -3 & -4472 \\
    9 & negligent\_jj & -3 & -4367 \\
    10 & careless\_jj & -3 & -3810 \\
    11 & simple\_jj & -3 & -3170 \\
    12 & haphazard\_jj & -3 & -2832 \\
    13 & complex\_jj & -3 & -2581 \\
    14 & neat\_jj & -3 & -1879 \\
    15 & careful\_jj & -3 & -1007 \\
    16 & selfish\_jj & -3 & -468 \\
    17 & high-strung\_jj & -2 & -9985 \\
    18 & unimaginative\_jj & -2 & -9239 \\
    19 & agreeable\_jj & -2 & -8912 \\
    20 & demanding\_jj & -2 & -7895 \\
    21 & timid\_jj & -2 & -6189 \\
    22 & unsympathetic\_jj & -2 & -5768 \\
    23 & trustful\_jj & -2 & -4566 \\
    24 & inconsistent\_jj & -2 & -3767 \\
    25 & practical\_jj & -2 & -3439 \\
    26 & talkative\_jj & -2 & -3304 \\
    27 & impractical\_jj & -2 & -3302 \\
    28 & pleasant\_jj & -2 & -2993 \\
    29 & sympathetic\_jj & -2 & -2169 \\
    30 & cooperative\_jj & -2 & -1041 \\
    61 & insecure\_jj & 1 & 6630 \\
    62 & unexcitable\_jj & 1 & 8795 \\
    63 & withdrawn\_jj & 1 & 8881 \\
    64 & self-pitying\_jj & 2 & 406 \\
    65 & thorough\_jj & 2 & 465 \\
    66 & active\_jj & 2 & 2137 \\
    67 & moody\_jj & 2 & 2866 \\
    68 & bright\_jj & 2 & 5609 \\
    69 & considerate\_jj & 2 & 6747 \\
    70 & undependable\_jj & 2 & 6757 \\
    71 & creative\_jj & 2 & 8618 \\
    72 & kind\_jj & 3 & 176 \\
    73 & rude\_jj & 3 & 508 \\
    74 & verbal\_jj & 3 & 1810 \\
    75 & nervous\_jj & 3 & 3037 \\
    76 & anxious\_jj & 3 & 6301 \\
    77 & organized\_jj & 3 & 6478 \\
    78 & efficient\_jj & 3 & 6969 \\
    79 & systematic\_jj & 3 & 9621 \\
    80 & fearful\_jj & 4 & 1489 \\
    81 & innovative\_jj & 4 & 6163 \\
    82 & intellectual\_jj & 4 & 9001 \\
    83 & artistic\_jj & 5 & 527 \\
    84 & energetic\_jj & 5 & 4648 \\
    85 & unrestrained\_jj & 5 & 7074 \\
    86 & jealous\_jj & 5 & 9534 \\
    87 & envious\_jj & 6 & 992 \\
    88 & steady\_jj & 6 & 2636 \\
    89 & uncooperative\_jj & 6 & 3723 \\
    90 & shy\_jj & 6 & 7243 \\
    \hline
    \caption{Scores and rankings for most extreme 30 words in component \#15} \\
\end{longtable}

\subsection{MDS}
\label{app:rankedwordlists:101words:mds}
\begin{longtable}[!htbp]{| rlr@{.}l |}
    \hline
    \textbf{Rank} & \textbf{Word} & \multicolumn{2}{c|}{\textbf{Score}} \\
    \hline
    \endhead
    1 & envious\_jj & 0 & -3201 \\
    2 & jealous\_jj & 0 & -3133 \\
    3 & self-pitying\_jj & 0 & -2878 \\
    4 & unkind\_jj & 0 & -2816 \\
    5 & bashful\_jj & 0 & -2724 \\
    6 & fretful\_jj & 0 & -2594 \\
    7 & uncharitable\_jj & 0 & -2499 \\
    8 & withdrawn\_jj & 0 & -2498 \\
    9 & unintelligent\_jj & 0 & -2167 \\
    10 & extroverted\_jj & 0 & -2155 \\
    11 & irritable\_jj & 0 & -2139 \\
    12 & rude\_jj & 0 & -2077 \\
    13 & unsympathetic\_jj & 0 & -2076 \\
    14 & unexcitable\_jj & 0 & -2046 \\
    15 & introspective\_jj & 0 & -2022 \\
    16 & insecure\_jj & 0 & -1991 \\
    17 & unadventurous\_jj & 0 & -1965 \\
    18 & temperamental\_jj & 0 & -1902 \\
    19 & talkative\_jj & 0 & -1896 \\
    20 & distrustful\_jj & 0 & -1872 \\
    21 & high-strung\_jj & 0 & -1789 \\
    22 & moody\_jj & 0 & -1658 \\
    23 & selfish\_jj & 0 & -1364 \\
    24 & anxious\_jj & 0 & -1253 \\
    25 & unreflective\_jj & 0 & -1245 \\
    26 & unsophisticated\_jj & 0 & -1193 \\
    27 & undependable\_jj & 0 & -1186 \\
    28 & unemotional\_jj & 0 & -1114 \\
    29 & uncooperative\_jj & 0 & -1102 \\
    30 & fearful\_jj & 0 & -1101 \\
    61 & generous\_jj & 0 & 1059 \\
    62 & inefficient\_jj & 0 & 1082 \\
    63 & artistic\_jj & 0 & 1138 \\
    64 & shallow\_jj & 0 & 1145 \\
    65 & impractical\_jj & 0 & 1163 \\
    66 & warm\_jj & 0 & 1214 \\
    67 & haphazard\_jj & 0 & 1254 \\
    68 & active\_jj & 0 & 1362 \\
    69 & imaginative\_jj & 0 & 1367 \\
    70 & steady\_jj & 0 & 1448 \\
    71 & daring\_jj & 0 & 1471 \\
    72 & inconsistent\_jj & 0 & 1479 \\
    73 & intellectual\_jj & 0 & 1562 \\
    74 & deep\_jj & 0 & 1574 \\
    75 & prompt\_jj & 0 & 1603 \\
    76 & neat\_jj & 0 & 1623 \\
    77 & careful\_jj & 0 & 1981 \\
    78 & organized\_jj & 0 & 2063 \\
    79 & cooperative\_jj & 0 & 2298 \\
    80 & bold\_jj & 0 & 2353 \\
    81 & helpful\_jj & 0 & 2449 \\
    82 & creative\_jj & 0 & 2481 \\
    83 & vigorous\_jj & 0 & 2601 \\
    84 & systematic\_jj & 0 & 2679 \\
    85 & thorough\_jj & 0 & 2825 \\
    86 & efficient\_jj & 0 & 3083 \\
    87 & complex\_jj & 0 & 3156 \\
    88 & simple\_jj & 0 & 3327 \\
    89 & practical\_jj & 0 & 3416 \\
    90 & innovative\_jj & 0 & 3519 \\
    \hline
    \caption{Scores and rankings for most extreme 30 words in component \#1} \\
\end{longtable}
\begin{longtable}[!htbp]{| rlr@{.}l |}
    \hline
    \textbf{Rank} & \textbf{Word} & \multicolumn{2}{c|}{\textbf{Score}} \\
    \hline
    \endhead
    1 & energetic\_jj & 0 & -3131 \\
    2 & considerate\_jj & 0 & -2799 \\
    3 & introspective\_jj & 0 & -2625 \\
    4 & pleasant\_jj & 0 & -2605 \\
    5 & creative\_jj & 0 & -2235 \\
    6 & imaginative\_jj & 0 & -2224 \\
    7 & extroverted\_jj & 0 & -2166 \\
    8 & quiet\_jj & 0 & -2142 \\
    9 & moody\_jj & 0 & -2098 \\
    10 & artistic\_jj & 0 & -2005 \\
    11 & bright\_jj & 0 & -1960 \\
    12 & kind\_jj & 0 & -1948 \\
    13 & talkative\_jj & 0 & -1915 \\
    14 & reserved\_jj & 0 & -1858 \\
    15 & warm\_jj & 0 & -1815 \\
    16 & daring\_jj & 0 & -1773 \\
    17 & bashful\_jj & 0 & -1706 \\
    18 & agreeable\_jj & 0 & -1645 \\
    19 & high-strung\_jj & 0 & -1511 \\
    20 & generous\_jj & 0 & -1154 \\
    21 & shy\_jj & 0 & -1099 \\
    22 & bold\_jj & 0 & -1022 \\
    23 & neat\_jj & 0 & -1013 \\
    24 & innovative\_jj & 0 & -1002 \\
    25 & philosophical\_jj & 0 & -987 \\
    26 & unadventurous\_jj & 0 & -921 \\
    27 & practical\_jj & 0 & -854 \\
    28 & demanding\_jj & 0 & -839 \\
    29 & conscientious\_jj & 0 & -769 \\
    30 & active\_jj & 0 & -755 \\
    61 & insecure\_jj & 0 & 601 \\
    62 & timid\_jj & 0 & 653 \\
    63 & undependable\_jj & 0 & 718 \\
    64 & self-pitying\_jj & 0 & 807 \\
    65 & uncharitable\_jj & 0 & 828 \\
    66 & unimaginative\_jj & 0 & 865 \\
    67 & rude\_jj & 0 & 885 \\
    68 & thorough\_jj & 0 & 933 \\
    69 & unreflective\_jj & 0 & 968 \\
    70 & unintelligent\_jj & 0 & 1046 \\
    71 & unsophisticated\_jj & 0 & 1132 \\
    72 & impractical\_jj & 0 & 1277 \\
    73 & organized\_jj & 0 & 1312 \\
    74 & unsympathetic\_jj & 0 & 1421 \\
    75 & distrustful\_jj & 0 & 1541 \\
    76 & verbal\_jj & 0 & 1543 \\
    77 & selfish\_jj & 0 & 1622 \\
    78 & fearful\_jj & 0 & 1663 \\
    79 & harsh\_jj & 0 & 1676 \\
    80 & haphazard\_jj & 0 & 1914 \\
    81 & prompt\_jj & 0 & 2217 \\
    82 & unrestrained\_jj & 0 & 2236 \\
    83 & careless\_jj & 0 & 2576 \\
    84 & uncooperative\_jj & 0 & 2734 \\
    85 & inefficient\_jj & 0 & 2860 \\
    86 & systematic\_jj & 0 & 2923 \\
    87 & sloppy\_jj & 0 & 2932 \\
    88 & inconsistent\_jj & 0 & 2961 \\
    89 & unsystematic\_jj & 0 & 3020 \\
    90 & negligent\_jj & 0 & 3299 \\
    \hline
    \caption{Scores and rankings for most extreme 30 words in component \#2} \\
\end{longtable}
\begin{longtable}[!htbp]{| rlr@{.}l |}
    \hline
    \textbf{Rank} & \textbf{Word} & \multicolumn{2}{c|}{\textbf{Score}} \\
    \hline
    \endhead
    1 & imaginative\_jj & 0 & -3037 \\
    2 & artistic\_jj & 0 & -2653 \\
    3 & innovative\_jj & 0 & -2597 \\
    4 & unreflective\_jj & 0 & -2464 \\
    5 & unadventurous\_jj & 0 & -2182 \\
    6 & unimaginative\_jj & 0 & -2045 \\
    7 & creative\_jj & 0 & -1913 \\
    8 & unintelligent\_jj & 0 & -1873 \\
    9 & self-pitying\_jj & 0 & -1842 \\
    10 & intellectual\_jj & 0 & -1811 \\
    11 & unsystematic\_jj & 0 & -1722 \\
    12 & unexcitable\_jj & 0 & -1683 \\
    13 & daring\_jj & 0 & -1645 \\
    14 & philosophical\_jj & 0 & -1309 \\
    15 & unrestrained\_jj & 0 & -1264 \\
    16 & selfish\_jj & 0 & -1250 \\
    17 & unsophisticated\_jj & 0 & -1217 \\
    18 & introspective\_jj & 0 & -1175 \\
    19 & extroverted\_jj & 0 & -1069 \\
    20 & trustful\_jj & 0 & -1048 \\
    21 & practical\_jj & 0 & -1040 \\
    22 & high-strung\_jj & 0 & -990 \\
    23 & undemanding\_jj & 0 & -977 \\
    24 & impractical\_jj & 0 & -962 \\
    25 & temperamental\_jj & 0 & -962 \\
    26 & neat\_jj & 0 & -915 \\
    27 & bold\_jj & 0 & -841 \\
    28 & imperturbable\_jj & 0 & -837 \\
    29 & efficient\_jj & 0 & -829 \\
    30 & undependable\_jj & 0 & -806 \\
    61 & talkative\_jj & 0 & 556 \\
    62 & organized\_jj & 0 & 602 \\
    63 & vigorous\_jj & 0 & 659 \\
    64 & assertive\_jj & 0 & 687 \\
    65 & reserved\_jj & 0 & 757 \\
    66 & cooperative\_jj & 0 & 757 \\
    67 & touchy\_jj & 0 & 864 \\
    68 & timid\_jj & 0 & 903 \\
    69 & jealous\_jj & 0 & 941 \\
    70 & insecure\_jj & 0 & 1067 \\
    71 & bright\_jj & 0 & 1183 \\
    72 & distrustful\_jj & 0 & 1190 \\
    73 & irritable\_jj & 0 & 1223 \\
    74 & generous\_jj & 0 & 1319 \\
    75 & deep\_jj & 0 & 1412 \\
    76 & pleasant\_jj & 0 & 1420 \\
    77 & sympathetic\_jj & 0 & 1433 \\
    78 & helpful\_jj & 0 & 1843 \\
    79 & kind\_jj & 0 & 1856 \\
    80 & shy\_jj & 0 & 1995 \\
    81 & active\_jj & 0 & 2040 \\
    82 & harsh\_jj & 0 & 2049 \\
    83 & steady\_jj & 0 & 2322 \\
    84 & careful\_jj & 0 & 2335 \\
    85 & quiet\_jj & 0 & 2462 \\
    86 & warm\_jj & 0 & 2477 \\
    87 & cold\_jj & 0 & 2999 \\
    88 & fearful\_jj & 0 & 3212 \\
    89 & anxious\_jj & 0 & 3401 \\
    90 & nervous\_jj & 0 & 3769 \\
    \hline
    \caption{Scores and rankings for most extreme 30 words in component \#3} \\
\end{longtable}
\begin{longtable}[!htbp]{| rlr@{.}l |}
    \hline
    \textbf{Rank} & \textbf{Word} & \multicolumn{2}{c|}{\textbf{Score}} \\
    \hline
    \endhead
    1 & active\_jj & 0 & -2782 \\
    2 & cooperative\_jj & 0 & -2493 \\
    3 & efficient\_jj & 0 & -2258 \\
    4 & sympathetic\_jj & 0 & -2258 \\
    5 & assertive\_jj & 0 & -2226 \\
    6 & unsympathetic\_jj & 0 & -1963 \\
    7 & distrustful\_jj & 0 & -1900 \\
    8 & innovative\_jj & 0 & -1829 \\
    9 & conscientious\_jj & 0 & -1784 \\
    10 & trustful\_jj & 0 & -1752 \\
    11 & considerate\_jj & 0 & -1631 \\
    12 & agreeable\_jj & 0 & -1630 \\
    13 & undependable\_jj & 0 & -1558 \\
    14 & helpful\_jj & 0 & -1454 \\
    15 & generous\_jj & 0 & -1323 \\
    16 & demanding\_jj & 0 & -1310 \\
    17 & inefficient\_jj & 0 & -1286 \\
    18 & fearful\_jj & 0 & -1173 \\
    19 & impractical\_jj & 0 & -1163 \\
    20 & uncooperative\_jj & 0 & -1128 \\
    21 & anxious\_jj & 0 & -1076 \\
    22 & insecure\_jj & 0 & -1065 \\
    23 & complex\_jj & 0 & -1054 \\
    24 & energetic\_jj & 0 & -1032 \\
    25 & organized\_jj & 0 & -1022 \\
    26 & extroverted\_jj & 0 & -971 \\
    27 & talkative\_jj & 0 & -962 \\
    28 & kind\_jj & 0 & -943 \\
    29 & unsystematic\_jj & 0 & -641 \\
    30 & nervous\_jj & 0 & -626 \\
    61 & bashful\_jj & 0 & 624 \\
    62 & high-strung\_jj & 0 & 658 \\
    63 & uncharitable\_jj & 0 & 663 \\
    64 & unadventurous\_jj & 0 & 694 \\
    65 & unexcitable\_jj & 0 & 701 \\
    66 & artistic\_jj & 0 & 712 \\
    67 & bold\_jj & 0 & 770 \\
    68 & unkind\_jj & 0 & 803 \\
    69 & rude\_jj & 0 & 807 \\
    70 & moody\_jj & 0 & 869 \\
    71 & shy\_jj & 0 & 929 \\
    72 & undemanding\_jj & 0 & 1114 \\
    73 & philosophical\_jj & 0 & 1170 \\
    74 & pleasant\_jj & 0 & 1435 \\
    75 & unreflective\_jj & 0 & 1528 \\
    76 & imperturbable\_jj & 0 & 1584 \\
    77 & simple\_jj & 0 & 1601 \\
    78 & self-pitying\_jj & 0 & 1610 \\
    79 & verbal\_jj & 0 & 1701 \\
    80 & sloppy\_jj & 0 & 1724 \\
    81 & quiet\_jj & 0 & 1745 \\
    82 & harsh\_jj & 0 & 1793 \\
    83 & careless\_jj & 0 & 1839 \\
    84 & bright\_jj & 0 & 2036 \\
    85 & steady\_jj & 0 & 2149 \\
    86 & shallow\_jj & 0 & 2396 \\
    87 & warm\_jj & 0 & 2484 \\
    88 & neat\_jj & 0 & 2551 \\
    89 & deep\_jj & 0 & 2784 \\
    90 & cold\_jj & 0 & 3130 \\
    \hline
    \caption{Scores and rankings for most extreme 30 words in component \#4} \\
\end{longtable}
\begin{longtable}[!htbp]{| rlr@{.}l |}
    \hline
    \textbf{Rank} & \textbf{Word} & \multicolumn{2}{c|}{\textbf{Score}} \\
    \hline
    \endhead
    1 & inefficient\_jj & 0 & -3173 \\
    2 & impractical\_jj & 0 & -2643 \\
    3 & bright\_jj & 0 & -2199 \\
    4 & neat\_jj & 0 & -1928 \\
    5 & undemanding\_jj & 0 & -1922 \\
    6 & generous\_jj & 0 & -1898 \\
    7 & efficient\_jj & 0 & -1837 \\
    8 & pleasant\_jj & 0 & -1796 \\
    9 & unimaginative\_jj & 0 & -1691 \\
    10 & undependable\_jj & 0 & -1613 \\
    11 & warm\_jj & 0 & -1554 \\
    12 & helpful\_jj & 0 & -1486 \\
    13 & haphazard\_jj & 0 & -1406 \\
    14 & unsophisticated\_jj & 0 & -1338 \\
    15 & cold\_jj & 0 & -1223 \\
    16 & talkative\_jj & 0 & -1221 \\
    17 & unintelligent\_jj & 0 & -1180 \\
    18 & shallow\_jj & 0 & -1175 \\
    19 & kind\_jj & 0 & -1153 \\
    20 & simple\_jj & 0 & -1005 \\
    21 & rude\_jj & 0 & -778 \\
    22 & considerate\_jj & 0 & -762 \\
    23 & sloppy\_jj & 0 & -746 \\
    24 & inconsistent\_jj & 0 & -729 \\
    25 & reserved\_jj & 0 & -679 \\
    26 & uncooperative\_jj & 0 & -634 \\
    27 & withdrawn\_jj & 0 & -621 \\
    28 & agreeable\_jj & 0 & -594 \\
    29 & trustful\_jj & 0 & -543 \\
    30 & innovative\_jj & 0 & -537 \\
    61 & unreflective\_jj & 0 & 519 \\
    62 & organized\_jj & 0 & 549 \\
    63 & conscientious\_jj & 0 & 555 \\
    64 & demanding\_jj & 0 & 592 \\
    65 & distrustful\_jj & 0 & 607 \\
    66 & touchy\_jj & 0 & 621 \\
    67 & prompt\_jj & 0 & 668 \\
    68 & energetic\_jj & 0 & 687 \\
    69 & creative\_jj & 0 & 881 \\
    70 & unemotional\_jj & 0 & 911 \\
    71 & thorough\_jj & 0 & 941 \\
    72 & introspective\_jj & 0 & 1043 \\
    73 & careful\_jj & 0 & 1072 \\
    74 & nervous\_jj & 0 & 1102 \\
    75 & imperturbable\_jj & 0 & 1120 \\
    76 & daring\_jj & 0 & 1144 \\
    77 & deep\_jj & 0 & 1202 \\
    78 & imaginative\_jj & 0 & 1217 \\
    79 & fretful\_jj & 0 & 1256 \\
    80 & systematic\_jj & 0 & 1272 \\
    81 & assertive\_jj & 0 & 1333 \\
    82 & fearful\_jj & 0 & 1623 \\
    83 & unrestrained\_jj & 0 & 1638 \\
    84 & anxious\_jj & 0 & 1789 \\
    85 & intellectual\_jj & 0 & 2347 \\
    86 & philosophical\_jj & 0 & 2372 \\
    87 & artistic\_jj & 0 & 2434 \\
    88 & vigorous\_jj & 0 & 2470 \\
    89 & verbal\_jj & 0 & 2952 \\
    90 & emotional\_jj & 0 & 3179 \\
    \hline
    \caption{Scores and rankings for most extreme 30 words in component \#5} \\
\end{longtable}
\begin{longtable}[!htbp]{| rlr@{.}l |}
    \hline
    \textbf{Rank} & \textbf{Word} & \multicolumn{2}{c|}{\textbf{Score}} \\
    \hline
    \endhead
    1 & cold\_jj & 0 & -2571 \\
    2 & trustful\_jj & 0 & -2306 \\
    3 & warm\_jj & 0 & -2061 \\
    4 & irritable\_jj & 0 & -2059 \\
    5 & deep\_jj & 0 & -2010 \\
    6 & complex\_jj & 0 & -1737 \\
    7 & undemanding\_jj & 0 & -1668 \\
    8 & harsh\_jj & 0 & -1545 \\
    9 & vigorous\_jj & 0 & -1366 \\
    10 & assertive\_jj & 0 & -1300 \\
    11 & emotional\_jj & 0 & -1268 \\
    12 & uncooperative\_jj & 0 & -1253 \\
    13 & active\_jj & 0 & -1228 \\
    14 & withdrawn\_jj & 0 & -1221 \\
    15 & undependable\_jj & 0 & -1202 \\
    16 & organized\_jj & 0 & -1197 \\
    17 & extroverted\_jj & 0 & -1145 \\
    18 & agreeable\_jj & 0 & -946 \\
    19 & touchy\_jj & 0 & -872 \\
    20 & introspective\_jj & 0 & -850 \\
    21 & reserved\_jj & 0 & -827 \\
    22 & distrustful\_jj & 0 & -774 \\
    23 & shallow\_jj & 0 & -658 \\
    24 & sympathetic\_jj & 0 & -599 \\
    25 & unemotional\_jj & 0 & -588 \\
    26 & high-strung\_jj & 0 & -564 \\
    27 & fretful\_jj & 0 & -559 \\
    28 & unreflective\_jj & 0 & -554 \\
    29 & self-pitying\_jj & 0 & -551 \\
    30 & demanding\_jj & 0 & -454 \\
    61 & imaginative\_jj & 0 & 403 \\
    62 & haphazard\_jj & 0 & 425 \\
    63 & nervous\_jj & 0 & 499 \\
    64 & daring\_jj & 0 & 530 \\
    65 & bold\_jj & 0 & 558 \\
    66 & anxious\_jj & 0 & 559 \\
    67 & bright\_jj & 0 & 580 \\
    68 & bashful\_jj & 0 & 637 \\
    69 & thorough\_jj & 0 & 679 \\
    70 & quiet\_jj & 0 & 708 \\
    71 & systematic\_jj & 0 & 718 \\
    72 & uncharitable\_jj & 0 & 795 \\
    73 & sloppy\_jj & 0 & 814 \\
    74 & conscientious\_jj & 0 & 851 \\
    75 & unkind\_jj & 0 & 859 \\
    76 & envious\_jj & 0 & 949 \\
    77 & rude\_jj & 0 & 1088 \\
    78 & energetic\_jj & 0 & 1161 \\
    79 & artistic\_jj & 0 & 1175 \\
    80 & timid\_jj & 0 & 1180 \\
    81 & helpful\_jj & 0 & 1307 \\
    82 & creative\_jj & 0 & 1626 \\
    83 & considerate\_jj & 0 & 1732 \\
    84 & selfish\_jj & 0 & 1972 \\
    85 & kind\_jj & 0 & 2028 \\
    86 & jealous\_jj & 0 & 2126 \\
    87 & negligent\_jj & 0 & 2221 \\
    88 & careful\_jj & 0 & 2245 \\
    89 & careless\_jj & 0 & 2802 \\
    90 & shy\_jj & 0 & 3237 \\
    \hline
    \caption{Scores and rankings for most extreme 30 words in component \#6} \\
\end{longtable}
\begin{longtable}[!htbp]{| rlr@{.}l |}
    \hline
    \textbf{Rank} & \textbf{Word} & \multicolumn{2}{c|}{\textbf{Score}} \\
    \hline
    \endhead
    1 & helpful\_jj & 0 & -2230 \\
    2 & practical\_jj & 0 & -2186 \\
    3 & intellectual\_jj & 0 & -2007 \\
    4 & emotional\_jj & 0 & -1926 \\
    5 & artistic\_jj & 0 & -1893 \\
    6 & fearful\_jj & 0 & -1789 \\
    7 & deep\_jj & 0 & -1769 \\
    8 & creative\_jj & 0 & -1566 \\
    9 & insecure\_jj & 0 & -1550 \\
    10 & philosophical\_jj & 0 & -1522 \\
    11 & shallow\_jj & 0 & -1516 \\
    12 & touchy\_jj & 0 & -1391 \\
    13 & simple\_jj & 0 & -1353 \\
    14 & impractical\_jj & 0 & -1161 \\
    15 & jealous\_jj & 0 & -1161 \\
    16 & unkind\_jj & 0 & -1151 \\
    17 & complex\_jj & 0 & -1143 \\
    18 & innovative\_jj & 0 & -1140 \\
    19 & unintelligent\_jj & 0 & -1105 \\
    20 & cold\_jj & 0 & -1082 \\
    21 & selfish\_jj & 0 & -1025 \\
    22 & distrustful\_jj & 0 & -992 \\
    23 & nervous\_jj & 0 & -973 \\
    24 & uncharitable\_jj & 0 & -947 \\
    25 & anxious\_jj & 0 & -889 \\
    26 & rude\_jj & 0 & -832 \\
    27 & envious\_jj & 0 & -630 \\
    28 & unreflective\_jj & 0 & -485 \\
    29 & undemanding\_jj & 0 & -467 \\
    30 & agreeable\_jj & 0 & -462 \\
    61 & extroverted\_jj & 0 & 613 \\
    62 & self-pitying\_jj & 0 & 647 \\
    63 & bright\_jj & 0 & 695 \\
    64 & uncooperative\_jj & 0 & 715 \\
    65 & unrestrained\_jj & 0 & 794 \\
    66 & introspective\_jj & 0 & 816 \\
    67 & negligent\_jj & 0 & 818 \\
    68 & bold\_jj & 0 & 835 \\
    69 & organized\_jj & 0 & 901 \\
    70 & bashful\_jj & 0 & 933 \\
    71 & quiet\_jj & 0 & 946 \\
    72 & daring\_jj & 0 & 1015 \\
    73 & timid\_jj & 0 & 1029 \\
    74 & assertive\_jj & 0 & 1046 \\
    75 & prompt\_jj & 0 & 1114 \\
    76 & considerate\_jj & 0 & 1175 \\
    77 & withdrawn\_jj & 0 & 1284 \\
    78 & unemotional\_jj & 0 & 1331 \\
    79 & imperturbable\_jj & 0 & 1436 \\
    80 & energetic\_jj & 0 & 1447 \\
    81 & sloppy\_jj & 0 & 1513 \\
    82 & high-strung\_jj & 0 & 1570 \\
    83 & cooperative\_jj & 0 & 1586 \\
    84 & haphazard\_jj & 0 & 1656 \\
    85 & reserved\_jj & 0 & 1703 \\
    86 & talkative\_jj & 0 & 1780 \\
    87 & steady\_jj & 0 & 1810 \\
    88 & systematic\_jj & 0 & 1836 \\
    89 & thorough\_jj & 0 & 2356 \\
    90 & vigorous\_jj & 0 & 2713 \\
    \hline
    \caption{Scores and rankings for most extreme 30 words in component \#7} \\
\end{longtable}
\begin{longtable}[!htbp]{| rlr@{.}l |}
    \hline
    \textbf{Rank} & \textbf{Word} & \multicolumn{2}{c|}{\textbf{Score}} \\
    \hline
    \endhead
    1 & bright\_jj & 0 & -1933 \\
    2 & moody\_jj & 0 & -1825 \\
    3 & energetic\_jj & 0 & -1822 \\
    4 & inefficient\_jj & 0 & -1757 \\
    5 & fretful\_jj & 0 & -1711 \\
    6 & bold\_jj & 0 & -1681 \\
    7 & unimaginative\_jj & 0 & -1591 \\
    8 & insecure\_jj & 0 & -1547 \\
    9 & nervous\_jj & 0 & -1526 \\
    10 & anxious\_jj & 0 & -1415 \\
    11 & fearful\_jj & 0 & -1375 \\
    12 & irritable\_jj & 0 & -1336 \\
    13 & assertive\_jj & 0 & -1314 \\
    14 & sloppy\_jj & 0 & -1313 \\
    15 & daring\_jj & 0 & -1278 \\
    16 & demanding\_jj & 0 & -1267 \\
    17 & efficient\_jj & 0 & -1184 \\
    18 & unsophisticated\_jj & 0 & -1173 \\
    19 & organized\_jj & 0 & -1146 \\
    20 & innovative\_jj & 0 & -1114 \\
    21 & active\_jj & 0 & -1108 \\
    22 & shy\_jj & 0 & -1075 \\
    23 & creative\_jj & 0 & -883 \\
    24 & undependable\_jj & 0 & -848 \\
    25 & imperturbable\_jj & 0 & -803 \\
    26 & imaginative\_jj & 0 & -778 \\
    27 & timid\_jj & 0 & -650 \\
    28 & complex\_jj & 0 & -620 \\
    29 & harsh\_jj & 0 & -593 \\
    30 & vigorous\_jj & 0 & -554 \\
    61 & simple\_jj & 0 & 668 \\
    62 & careless\_jj & 0 & 688 \\
    63 & unsympathetic\_jj & 0 & 707 \\
    64 & extroverted\_jj & 0 & 720 \\
    65 & verbal\_jj & 0 & 744 \\
    66 & selfish\_jj & 0 & 754 \\
    67 & bashful\_jj & 0 & 766 \\
    68 & warm\_jj & 0 & 836 \\
    69 & thorough\_jj & 0 & 865 \\
    70 & kind\_jj & 0 & 872 \\
    71 & prompt\_jj & 0 & 929 \\
    72 & talkative\_jj & 0 & 1015 \\
    73 & unemotional\_jj & 0 & 1049 \\
    74 & systematic\_jj & 0 & 1050 \\
    75 & sympathetic\_jj & 0 & 1080 \\
    76 & uncooperative\_jj & 0 & 1105 \\
    77 & careful\_jj & 0 & 1146 \\
    78 & pleasant\_jj & 0 & 1239 \\
    79 & helpful\_jj & 0 & 1244 \\
    80 & conscientious\_jj & 0 & 1458 \\
    81 & unkind\_jj & 0 & 1468 \\
    82 & practical\_jj & 0 & 1534 \\
    83 & unsystematic\_jj & 0 & 1609 \\
    84 & considerate\_jj & 0 & 1618 \\
    85 & reserved\_jj & 0 & 1745 \\
    86 & philosophical\_jj & 0 & 1826 \\
    87 & agreeable\_jj & 0 & 1833 \\
    88 & touchy\_jj & 0 & 1974 \\
    89 & cooperative\_jj & 0 & 2075 \\
    90 & trustful\_jj & 0 & 2453 \\
    \hline
    \caption{Scores and rankings for most extreme 30 words in component \#8} \\
\end{longtable}
\begin{longtable}[!htbp]{| rlr@{.}l |}
    \hline
    \textbf{Rank} & \textbf{Word} & \multicolumn{2}{c|}{\textbf{Score}} \\
    \hline
    \endhead
    1 & unsystematic\_jj & 0 & -2329 \\
    2 & active\_jj & 0 & -2318 \\
    3 & steady\_jj & 0 & -1909 \\
    4 & organized\_jj & 0 & -1826 \\
    5 & intellectual\_jj & 0 & -1733 \\
    6 & deep\_jj & 0 & -1607 \\
    7 & innovative\_jj & 0 & -1585 \\
    8 & undependable\_jj & 0 & -1582 \\
    9 & considerate\_jj & 0 & -1510 \\
    10 & negligent\_jj & 0 & -1505 \\
    11 & envious\_jj & 0 & -1412 \\
    12 & warm\_jj & 0 & -1255 \\
    13 & unrestrained\_jj & 0 & -1205 \\
    14 & systematic\_jj & 0 & -1185 \\
    15 & trustful\_jj & 0 & -1098 \\
    16 & artistic\_jj & 0 & -1090 \\
    17 & imperturbable\_jj & 0 & -1074 \\
    18 & unreflective\_jj & 0 & -1059 \\
    19 & energetic\_jj & 0 & -938 \\
    20 & creative\_jj & 0 & -931 \\
    21 & kind\_jj & 0 & -838 \\
    22 & generous\_jj & 0 & -821 \\
    23 & efficient\_jj & 0 & -776 \\
    24 & conscientious\_jj & 0 & -763 \\
    25 & withdrawn\_jj & 0 & -681 \\
    26 & unexcitable\_jj & 0 & -674 \\
    27 & neat\_jj & 0 & -564 \\
    28 & uncharitable\_jj & 0 & -562 \\
    29 & shallow\_jj & 0 & -554 \\
    30 & distrustful\_jj & 0 & -459 \\
    61 & unkind\_jj & 0 & 579 \\
    62 & bashful\_jj & 0 & 612 \\
    63 & high-strung\_jj & 0 & 678 \\
    64 & unemotional\_jj & 0 & 702 \\
    65 & introspective\_jj & 0 & 717 \\
    66 & verbal\_jj & 0 & 746 \\
    67 & imaginative\_jj & 0 & 748 \\
    68 & haphazard\_jj & 0 & 782 \\
    69 & vigorous\_jj & 0 & 794 \\
    70 & uncooperative\_jj & 0 & 871 \\
    71 & inconsistent\_jj & 0 & 939 \\
    72 & helpful\_jj & 0 & 943 \\
    73 & simple\_jj & 0 & 946 \\
    74 & sloppy\_jj & 0 & 969 \\
    75 & demanding\_jj & 0 & 1018 \\
    76 & unsympathetic\_jj & 0 & 1037 \\
    77 & philosophical\_jj & 0 & 1144 \\
    78 & unimaginative\_jj & 0 & 1256 \\
    79 & careful\_jj & 0 & 1276 \\
    80 & complex\_jj & 0 & 1345 \\
    81 & practical\_jj & 0 & 1402 \\
    82 & rude\_jj & 0 & 1513 \\
    83 & assertive\_jj & 0 & 1515 \\
    84 & harsh\_jj & 0 & 1528 \\
    85 & impractical\_jj & 0 & 1724 \\
    86 & touchy\_jj & 0 & 1729 \\
    87 & sympathetic\_jj & 0 & 1811 \\
    88 & timid\_jj & 0 & 1919 \\
    89 & daring\_jj & 0 & 2161 \\
    90 & bold\_jj & 0 & 2339 \\
    \hline
    \caption{Scores and rankings for most extreme 30 words in component \#9} \\
\end{longtable}
\begin{longtable}[!htbp]{| rlr@{.}l |}
    \hline
    \textbf{Rank} & \textbf{Word} & \multicolumn{2}{c|}{\textbf{Score}} \\
    \hline
    \endhead
    1 & irritable\_jj & 0 & -2482 \\
    2 & nervous\_jj & 0 & -2179 \\
    3 & verbal\_jj & 0 & -2031 \\
    4 & innovative\_jj & 0 & -2004 \\
    5 & helpful\_jj & 0 & -1633 \\
    6 & anxious\_jj & 0 & -1582 \\
    7 & negligent\_jj & 0 & -1503 \\
    8 & efficient\_jj & 0 & -1332 \\
    9 & emotional\_jj & 0 & -1331 \\
    10 & prompt\_jj & 0 & -1313 \\
    11 & withdrawn\_jj & 0 & -1246 \\
    12 & unsystematic\_jj & 0 & -1240 \\
    13 & uncooperative\_jj & 0 & -1041 \\
    14 & undemanding\_jj & 0 & -973 \\
    15 & sloppy\_jj & 0 & -960 \\
    16 & unemotional\_jj & 0 & -859 \\
    17 & rude\_jj & 0 & -843 \\
    18 & moody\_jj & 0 & -817 \\
    19 & bashful\_jj & 0 & -786 \\
    20 & talkative\_jj & 0 & -777 \\
    21 & high-strung\_jj & 0 & -762 \\
    22 & unkind\_jj & 0 & -748 \\
    23 & pleasant\_jj & 0 & -712 \\
    24 & careless\_jj & 0 & -697 \\
    25 & simple\_jj & 0 & -664 \\
    26 & imperturbable\_jj & 0 & -635 \\
    27 & uncharitable\_jj & 0 & -597 \\
    28 & cold\_jj & 0 & -578 \\
    29 & steady\_jj & 0 & -474 \\
    30 & trustful\_jj & 0 & -403 \\
    61 & insecure\_jj & 0 & 487 \\
    62 & energetic\_jj & 0 & 515 \\
    63 & fretful\_jj & 0 & 559 \\
    64 & unimaginative\_jj & 0 & 567 \\
    65 & undependable\_jj & 0 & 569 \\
    66 & active\_jj & 0 & 591 \\
    67 & neat\_jj & 0 & 595 \\
    68 & unsophisticated\_jj & 0 & 692 \\
    69 & systematic\_jj & 0 & 765 \\
    70 & kind\_jj & 0 & 834 \\
    71 & fearful\_jj & 0 & 877 \\
    72 & haphazard\_jj & 0 & 881 \\
    73 & vigorous\_jj & 0 & 886 \\
    74 & jealous\_jj & 0 & 886 \\
    75 & unintelligent\_jj & 0 & 919 \\
    76 & inefficient\_jj & 0 & 938 \\
    77 & assertive\_jj & 0 & 940 \\
    78 & touchy\_jj & 0 & 958 \\
    79 & selfish\_jj & 0 & 1012 \\
    80 & reserved\_jj & 0 & 1021 \\
    81 & organized\_jj & 0 & 1088 \\
    82 & unreflective\_jj & 0 & 1188 \\
    83 & bold\_jj & 0 & 1404 \\
    84 & conscientious\_jj & 0 & 1414 \\
    85 & timid\_jj & 0 & 1738 \\
    86 & shallow\_jj & 0 & 1791 \\
    87 & philosophical\_jj & 0 & 1840 \\
    88 & unrestrained\_jj & 0 & 1914 \\
    89 & deep\_jj & 0 & 2032 \\
    90 & distrustful\_jj & 0 & 2747 \\
    \hline
    \caption{Scores and rankings for most extreme 30 words in component \#10} \\
\end{longtable}
\begin{longtable}[!htbp]{| rlr@{.}l |}
    \hline
    \textbf{Rank} & \textbf{Word} & \multicolumn{2}{c|}{\textbf{Score}} \\
    \hline
    \endhead
    1 & unsystematic\_jj & 0 & -2174 \\
    2 & complex\_jj & 0 & -2161 \\
    3 & haphazard\_jj & 0 & -1858 \\
    4 & nervous\_jj & 0 & -1769 \\
    5 & neat\_jj & 0 & -1730 \\
    6 & simple\_jj & 0 & -1630 \\
    7 & thorough\_jj & 0 & -1622 \\
    8 & steady\_jj & 0 & -1314 \\
    9 & fearful\_jj & 0 & -1294 \\
    10 & withdrawn\_jj & 0 & -1236 \\
    11 & trustful\_jj & 0 & -1235 \\
    12 & fretful\_jj & 0 & -1080 \\
    13 & deep\_jj & 0 & -1026 \\
    14 & bashful\_jj & 0 & -1025 \\
    15 & unsophisticated\_jj & 0 & -975 \\
    16 & sloppy\_jj & 0 & -931 \\
    17 & shallow\_jj & 0 & -923 \\
    18 & insecure\_jj & 0 & -889 \\
    19 & distrustful\_jj & 0 & -820 \\
    20 & philosophical\_jj & 0 & -754 \\
    21 & jealous\_jj & 0 & -744 \\
    22 & touchy\_jj & 0 & -742 \\
    23 & envious\_jj & 0 & -732 \\
    24 & unadventurous\_jj & 0 & -666 \\
    25 & careful\_jj & 0 & -582 \\
    26 & high-strung\_jj & 0 & -527 \\
    27 & systematic\_jj & 0 & -513 \\
    28 & practical\_jj & 0 & -453 \\
    29 & moody\_jj & 0 & -432 \\
    30 & anxious\_jj & 0 & -400 \\
    61 & generous\_jj & 0 & 372 \\
    62 & conscientious\_jj & 0 & 402 \\
    63 & daring\_jj & 0 & 421 \\
    64 & quiet\_jj & 0 & 572 \\
    65 & assertive\_jj & 0 & 576 \\
    66 & pleasant\_jj & 0 & 582 \\
    67 & bold\_jj & 0 & 600 \\
    68 & unreflective\_jj & 0 & 614 \\
    69 & emotional\_jj & 0 & 645 \\
    70 & selfish\_jj & 0 & 671 \\
    71 & energetic\_jj & 0 & 673 \\
    72 & reserved\_jj & 0 & 676 \\
    73 & careless\_jj & 0 & 745 \\
    74 & inconsistent\_jj & 0 & 795 \\
    75 & artistic\_jj & 0 & 893 \\
    76 & agreeable\_jj & 0 & 921 \\
    77 & imperturbable\_jj & 0 & 1103 \\
    78 & intellectual\_jj & 0 & 1111 \\
    79 & unsympathetic\_jj & 0 & 1175 \\
    80 & rude\_jj & 0 & 1363 \\
    81 & active\_jj & 0 & 1376 \\
    82 & unkind\_jj & 0 & 1405 \\
    83 & warm\_jj & 0 & 1495 \\
    84 & impractical\_jj & 0 & 1499 \\
    85 & cold\_jj & 0 & 1634 \\
    86 & unrestrained\_jj & 0 & 1697 \\
    87 & organized\_jj & 0 & 1828 \\
    88 & uncooperative\_jj & 0 & 1871 \\
    89 & negligent\_jj & 0 & 1876 \\
    90 & harsh\_jj & 0 & 2505 \\
    \hline
    \caption{Scores and rankings for most extreme 30 words in component \#11} \\
\end{longtable}
\begin{longtable}[!htbp]{| rlr@{.}l |}
    \hline
    \textbf{Rank} & \textbf{Word} & \multicolumn{2}{c|}{\textbf{Score}} \\
    \hline
    \endhead
    1 & withdrawn\_jj & 0 & -2276 \\
    2 & unemotional\_jj & 0 & -1876 \\
    3 & unintelligent\_jj & 0 & -1766 \\
    4 & shallow\_jj & 0 & -1621 \\
    5 & systematic\_jj & 0 & -1457 \\
    6 & moody\_jj & 0 & -1349 \\
    7 & considerate\_jj & 0 & -1223 \\
    8 & warm\_jj & 0 & -1222 \\
    9 & emotional\_jj & 0 & -1177 \\
    10 & verbal\_jj & 0 & -1164 \\
    11 & rude\_jj & 0 & -1151 \\
    12 & undependable\_jj & 0 & -1047 \\
    13 & selfish\_jj & 0 & -1043 \\
    14 & insecure\_jj & 0 & -1030 \\
    15 & kind\_jj & 0 & -956 \\
    16 & bright\_jj & 0 & -942 \\
    17 & complex\_jj & 0 & -832 \\
    18 & intellectual\_jj & 0 & -811 \\
    19 & energetic\_jj & 0 & -806 \\
    20 & cooperative\_jj & 0 & -785 \\
    21 & reserved\_jj & 0 & -784 \\
    22 & prompt\_jj & 0 & -741 \\
    23 & shy\_jj & 0 & -703 \\
    24 & deep\_jj & 0 & -681 \\
    25 & daring\_jj & 0 & -671 \\
    26 & demanding\_jj & 0 & -658 \\
    27 & self-pitying\_jj & 0 & -651 \\
    28 & creative\_jj & 0 & -635 \\
    29 & talkative\_jj & 0 & -629 \\
    30 & cold\_jj & 0 & -568 \\
    61 & conscientious\_jj & 0 & 525 \\
    62 & timid\_jj & 0 & 562 \\
    63 & careful\_jj & 0 & 597 \\
    64 & inefficient\_jj & 0 & 640 \\
    65 & careless\_jj & 0 & 713 \\
    66 & agreeable\_jj & 0 & 783 \\
    67 & unimaginative\_jj & 0 & 785 \\
    68 & quiet\_jj & 0 & 812 \\
    69 & simple\_jj & 0 & 865 \\
    70 & efficient\_jj & 0 & 889 \\
    71 & active\_jj & 0 & 934 \\
    72 & neat\_jj & 0 & 960 \\
    73 & unexcitable\_jj & 0 & 963 \\
    74 & envious\_jj & 0 & 973 \\
    75 & assertive\_jj & 0 & 1005 \\
    76 & undemanding\_jj & 0 & 1069 \\
    77 & bashful\_jj & 0 & 1081 \\
    78 & helpful\_jj & 0 & 1099 \\
    79 & pleasant\_jj & 0 & 1107 \\
    80 & touchy\_jj & 0 & 1216 \\
    81 & trustful\_jj & 0 & 1216 \\
    82 & unkind\_jj & 0 & 1217 \\
    83 & vigorous\_jj & 0 & 1277 \\
    84 & uncharitable\_jj & 0 & 1294 \\
    85 & unreflective\_jj & 0 & 1306 \\
    86 & unrestrained\_jj & 0 & 1336 \\
    87 & steady\_jj & 0 & 1736 \\
    88 & unadventurous\_jj & 0 & 1905 \\
    89 & fretful\_jj & 0 & 1925 \\
    90 & imperturbable\_jj & 0 & 2606 \\
    \hline
    \caption{Scores and rankings for most extreme 30 words in component \#12} \\
\end{longtable}
\begin{longtable}[!htbp]{| rlr@{.}l |}
    \hline
    \textbf{Rank} & \textbf{Word} & \multicolumn{2}{c|}{\textbf{Score}} \\
    \hline
    \endhead
    1 & inconsistent\_jj & 0 & -2303 \\
    2 & shallow\_jj & 0 & -1654 \\
    3 & touchy\_jj & 0 & -1515 \\
    4 & sloppy\_jj & 0 & -1504 \\
    5 & emotional\_jj & 0 & -1470 \\
    6 & active\_jj & 0 & -1456 \\
    7 & uncooperative\_jj & 0 & -1444 \\
    8 & insecure\_jj & 0 & -1238 \\
    9 & unsophisticated\_jj & 0 & -1140 \\
    10 & undemanding\_jj & 0 & -1110 \\
    11 & intellectual\_jj & 0 & -967 \\
    12 & demanding\_jj & 0 & -940 \\
    13 & extroverted\_jj & 0 & -909 \\
    14 & talkative\_jj & 0 & -875 \\
    15 & haphazard\_jj & 0 & -865 \\
    16 & unimaginative\_jj & 0 & -856 \\
    17 & introspective\_jj & 0 & -840 \\
    18 & energetic\_jj & 0 & -764 \\
    19 & quiet\_jj & 0 & -737 \\
    20 & negligent\_jj & 0 & -683 \\
    21 & verbal\_jj & 0 & -657 \\
    22 & high-strung\_jj & 0 & -630 \\
    23 & shy\_jj & 0 & -624 \\
    24 & kind\_jj & 0 & -621 \\
    25 & complex\_jj & 0 & -605 \\
    26 & unrestrained\_jj & 0 & -585 \\
    27 & careless\_jj & 0 & -554 \\
    28 & unadventurous\_jj & 0 & -529 \\
    29 & reserved\_jj & 0 & -502 \\
    30 & philosophical\_jj & 0 & -464 \\
    61 & selfish\_jj & 0 & 296 \\
    62 & innovative\_jj & 0 & 323 \\
    63 & unexcitable\_jj & 0 & 340 \\
    64 & efficient\_jj & 0 & 391 \\
    65 & organized\_jj & 0 & 399 \\
    66 & rude\_jj & 0 & 447 \\
    67 & self-pitying\_jj & 0 & 526 \\
    68 & imperturbable\_jj & 0 & 558 \\
    69 & practical\_jj & 0 & 595 \\
    70 & trustful\_jj & 0 & 630 \\
    71 & anxious\_jj & 0 & 638 \\
    72 & envious\_jj & 0 & 669 \\
    73 & warm\_jj & 0 & 735 \\
    74 & simple\_jj & 0 & 744 \\
    75 & generous\_jj & 0 & 769 \\
    76 & harsh\_jj & 0 & 873 \\
    77 & unemotional\_jj & 0 & 915 \\
    78 & unsystematic\_jj & 0 & 1049 \\
    79 & sympathetic\_jj & 0 & 1068 \\
    80 & bright\_jj & 0 & 1075 \\
    81 & unsympathetic\_jj & 0 & 1107 \\
    82 & distrustful\_jj & 0 & 1200 \\
    83 & uncharitable\_jj & 0 & 1237 \\
    84 & withdrawn\_jj & 0 & 1249 \\
    85 & fearful\_jj & 0 & 1386 \\
    86 & unreflective\_jj & 0 & 1402 \\
    87 & undependable\_jj & 0 & 1529 \\
    88 & daring\_jj & 0 & 1582 \\
    89 & bold\_jj & 0 & 3030 \\
    90 & prompt\_jj & 0 & 3303 \\
    \hline
    \caption{Scores and rankings for most extreme 30 words in component \#13} \\
\end{longtable}
\begin{longtable}[!htbp]{| rlr@{.}l |}
    \hline
    \textbf{Rank} & \textbf{Word} & \multicolumn{2}{c|}{\textbf{Score}} \\
    \hline
    \endhead
    1 & organized\_jj & 0 & -1862 \\
    2 & unexcitable\_jj & 0 & -1703 \\
    3 & unkind\_jj & 0 & -1656 \\
    4 & envious\_jj & 0 & -1577 \\
    5 & unintelligent\_jj & 0 & -1572 \\
    6 & harsh\_jj & 0 & -1328 \\
    7 & steady\_jj & 0 & -1240 \\
    8 & kind\_jj & 0 & -1237 \\
    9 & undependable\_jj & 0 & -1169 \\
    10 & unimaginative\_jj & 0 & -1159 \\
    11 & verbal\_jj & 0 & -1154 \\
    12 & systematic\_jj & 0 & -1041 \\
    13 & extroverted\_jj & 0 & -1017 \\
    14 & generous\_jj & 0 & -987 \\
    15 & thorough\_jj & 0 & -970 \\
    16 & haphazard\_jj & 0 & -787 \\
    17 & uncharitable\_jj & 0 & -775 \\
    18 & unsympathetic\_jj & 0 & -765 \\
    19 & careful\_jj & 0 & -698 \\
    20 & undemanding\_jj & 0 & -691 \\
    21 & jealous\_jj & 0 & -591 \\
    22 & vigorous\_jj & 0 & -590 \\
    23 & sloppy\_jj & 0 & -565 \\
    24 & imaginative\_jj & 0 & -563 \\
    25 & emotional\_jj & 0 & -548 \\
    26 & talkative\_jj & 0 & -544 \\
    27 & intellectual\_jj & 0 & -523 \\
    28 & bashful\_jj & 0 & -521 \\
    29 & shy\_jj & 0 & -505 \\
    30 & considerate\_jj & 0 & -491 \\
    61 & assertive\_jj & 0 & 354 \\
    62 & distrustful\_jj & 0 & 374 \\
    63 & nervous\_jj & 0 & 393 \\
    64 & complex\_jj & 0 & 402 \\
    65 & shallow\_jj & 0 & 483 \\
    66 & neat\_jj & 0 & 584 \\
    67 & unrestrained\_jj & 0 & 587 \\
    68 & agreeable\_jj & 0 & 589 \\
    69 & simple\_jj & 0 & 641 \\
    70 & anxious\_jj & 0 & 673 \\
    71 & bright\_jj & 0 & 699 \\
    72 & temperamental\_jj & 0 & 762 \\
    73 & uncooperative\_jj & 0 & 867 \\
    74 & selfish\_jj & 0 & 887 \\
    75 & artistic\_jj & 0 & 889 \\
    76 & prompt\_jj & 0 & 891 \\
    77 & moody\_jj & 0 & 916 \\
    78 & active\_jj & 0 & 934 \\
    79 & unsystematic\_jj & 0 & 942 \\
    80 & unreflective\_jj & 0 & 1061 \\
    81 & quiet\_jj & 0 & 1152 \\
    82 & high-strung\_jj & 0 & 1167 \\
    83 & demanding\_jj & 0 & 1482 \\
    84 & impractical\_jj & 0 & 1485 \\
    85 & withdrawn\_jj & 0 & 1488 \\
    86 & careless\_jj & 0 & 1762 \\
    87 & conscientious\_jj & 0 & 1900 \\
    88 & reserved\_jj & 0 & 2146 \\
    89 & fretful\_jj & 0 & 2252 \\
    90 & negligent\_jj & 0 & 2484 \\
    \hline
    \caption{Scores and rankings for most extreme 30 words in component \#14} \\
\end{longtable}
\begin{longtable}[!htbp]{| rlr@{.}l |}
    \hline
    \textbf{Rank} & \textbf{Word} & \multicolumn{2}{c|}{\textbf{Score}} \\
    \hline
    \endhead
    1 & quiet\_jj & 0 & -2086 \\
    2 & unrestrained\_jj & 0 & -2020 \\
    3 & rude\_jj & 0 & -1924 \\
    4 & emotional\_jj & 0 & -1839 \\
    5 & unsophisticated\_jj & 0 & -1602 \\
    6 & simple\_jj & 0 & -1568 \\
    7 & introspective\_jj & 0 & -1193 \\
    8 & talkative\_jj & 0 & -1092 \\
    9 & practical\_jj & 0 & -1017 \\
    10 & self-pitying\_jj & 0 & -1008 \\
    11 & prompt\_jj & 0 & -989 \\
    12 & pleasant\_jj & 0 & -971 \\
    13 & inefficient\_jj & 0 & -971 \\
    14 & insecure\_jj & 0 & -864 \\
    15 & systematic\_jj & 0 & -801 \\
    16 & uncharitable\_jj & 0 & -728 \\
    17 & considerate\_jj & 0 & -705 \\
    18 & shy\_jj & 0 & -661 \\
    19 & fearful\_jj & 0 & -637 \\
    20 & impractical\_jj & 0 & -612 \\
    21 & efficient\_jj & 0 & -596 \\
    22 & kind\_jj & 0 & -595 \\
    23 & fretful\_jj & 0 & -569 \\
    24 & extroverted\_jj & 0 & -555 \\
    25 & uncooperative\_jj & 0 & -539 \\
    26 & energetic\_jj & 0 & -529 \\
    27 & vigorous\_jj & 0 & -486 \\
    28 & generous\_jj & 0 & -456 \\
    29 & unexcitable\_jj & 0 & -421 \\
    30 & nervous\_jj & 0 & -388 \\
    61 & complex\_jj & 0 & 277 \\
    62 & active\_jj & 0 & 336 \\
    63 & unemotional\_jj & 0 & 381 \\
    64 & touchy\_jj & 0 & 430 \\
    65 & careful\_jj & 0 & 478 \\
    66 & steady\_jj & 0 & 510 \\
    67 & harsh\_jj & 0 & 525 \\
    68 & haphazard\_jj & 0 & 556 \\
    69 & intellectual\_jj & 0 & 571 \\
    70 & agreeable\_jj & 0 & 588 \\
    71 & cold\_jj & 0 & 699 \\
    72 & negligent\_jj & 0 & 705 \\
    73 & unreflective\_jj & 0 & 714 \\
    74 & unsympathetic\_jj & 0 & 759 \\
    75 & neat\_jj & 0 & 787 \\
    76 & high-strung\_jj & 0 & 792 \\
    77 & assertive\_jj & 0 & 811 \\
    78 & trustful\_jj & 0 & 813 \\
    79 & cooperative\_jj & 0 & 889 \\
    80 & demanding\_jj & 0 & 986 \\
    81 & warm\_jj & 0 & 1241 \\
    82 & creative\_jj & 0 & 1244 \\
    83 & sympathetic\_jj & 0 & 1279 \\
    84 & bright\_jj & 0 & 1301 \\
    85 & sloppy\_jj & 0 & 1397 \\
    86 & artistic\_jj & 0 & 1497 \\
    87 & inconsistent\_jj & 0 & 1707 \\
    88 & envious\_jj & 0 & 1750 \\
    89 & jealous\_jj & 0 & 1911 \\
    90 & temperamental\_jj & 0 & 2474 \\
    \hline
    \caption{Scores and rankings for most extreme 30 words in component \#15} \\
\end{longtable}


\section{438 word list}
\subsection{Unnormalized PCA}
\label{app:rankedwordlists:438words:unnormalized}
\begin{longtable}[!htbp]{| rlr@{.}l |}
    \hline
    \textbf{Rank} & \textbf{Word} & \multicolumn{2}{c|}{\textbf{Score}} \\
    \hline
    \endhead
    1 & sociable\_jj & -2 & -5624 \\
    2 & vivacious\_jj & -2 & -2802 \\
    3 & considerate\_jj & -2 & -2300 \\
    4 & easygoing\_jj & -2 & -1304 \\
    5 & witty\_jj & -1 & -8083 \\
    6 & affectionate\_jj & -1 & -7541 \\
    7 & talkative\_jj & -1 & -7251 \\
    8 & gregarious\_jj & -1 & -6743 \\
    9 & down-to-earth\_jj & -1 & -6259 \\
    10 & courteous\_jj & -1 & -6241 \\
    11 & jovial\_jj & -1 & -4513 \\
    12 & cultured\_jj & -1 & -4384 \\
    13 & extroverted\_jj & -1 & -4120 \\
    14 & introspective\_jj & -1 & -3500 \\
    15 & inquisitive\_jj & -1 & -3152 \\
    16 & genial\_jj & -1 & -2395 \\
    17 & amiable\_jj & -1 & -2256 \\
    18 & mischievous\_jj & -1 & -2096 \\
    19 & perceptive\_jj & -1 & -1836 \\
    20 & humorous\_jj & -1 & -1829 \\
    21 & happy-go-lucky\_jj & -1 & -1803 \\
    22 & expressive\_jj & -1 & -1647 \\
    23 & folksy\_jj & -1 & -1568 \\
    24 & kind\_jj & -1 & -1473 \\
    25 & cheerful\_jj & -1 & -1178 \\
    26 & playful\_jj & -1 & -1067 \\
    27 & communicative\_jj & -1 & -731 \\
    28 & impetuous\_jj & -1 & -671 \\
    29 & gruff\_jj & -1 & -419 \\
    30 & bashful\_jj & -1 & -222 \\
    392 & inconsistent\_jj & 1 & 431 \\
    393 & decisive\_jj & 1 & 571 \\
    394 & charitable\_jj & 1 & 573 \\
    395 & prompt\_jj & 1 & 649 \\
    396 & responsible\_jj & 1 & 705 \\
    397 & suspicious\_jj & 1 & 708 \\
    398 & optimism & 1 & 739 \\
    399 & direct\_jj & 1 & 818 \\
    400 & organization & 1 & 870 \\
    401 & consistent\_jj & 1 & 968 \\
    402 & reckless\_jj & 1 & 971 \\
    403 & assertion & 1 & 1040 \\
    404 & diplomatic\_jj & 1 & 1041 \\
    405 & organized\_jj & 1 & 1077 \\
    406 & reasonable\_jj & 1 & 1457 \\
    407 & efficiency & 1 & 1532 \\
    408 & volatile\_jj & 1 & 1997 \\
    409 & systematic\_jj & 1 & 2004 \\
    410 & negligent\_jj & 1 & 2014 \\
    411 & explosive\_jj & 1 & 2147 \\
    412 & flexibility & 1 & 2414 \\
    413 & autonomous\_jj & 1 & 2430 \\
    414 & reserve & 1 & 2527 \\
    415 & caution & 1 & 2541 \\
    416 & sluggish\_jj & 1 & 2889 \\
    417 & intelligence & 1 & 3401 \\
    418 & cooperation & 1 & 4162 \\
    419 & volatility & 1 & 4444 \\
    420 & instability & 1 & 5375 \\
    421 & negligence & 1 & 6036 \\
    \hline
    \caption{Scores and rankings for most extreme 30 words in component \#1} \\
\end{longtable}
\begin{longtable}[!htbp]{| rlr@{.}l |}
    \hline
    \textbf{Rank} & \textbf{Word} & \multicolumn{2}{c|}{\textbf{Score}} \\
    \hline
    \endhead
    1 & callousness & -2 & -480 \\
    2 & selfishness & -1 & -8128 \\
    3 & gullibility & -1 & -6592 \\
    4 & stupidity & -1 & -6547 \\
    5 & recklessness & -1 & -6432 \\
    6 & rudeness & -1 & -6170 \\
    7 & deceit & -1 & -5653 \\
    8 & belligerence & -1 & -5329 \\
    9 & shallowness & -1 & -4978 \\
    10 & lethargy & -1 & -4899 \\
    11 & thoughtless\_jj & -1 & -4463 \\
    12 & passivity & -1 & -4443 \\
    13 & irritability & -1 & -4099 \\
    14 & pomposity & -1 & -3657 \\
    15 & self-pitying\_jj & -1 & -3591 \\
    16 & deceitful\_jj & -1 & -3588 \\
    17 & bigoted\_jj & -1 & -3513 \\
    18 & stubbornness & -1 & -3487 \\
    19 & indecisiveness & -1 & -3165 \\
    20 & unreflective\_jj & -1 & -2349 \\
    21 & disorganization & -1 & -1967 \\
    22 & aloofness & -1 & -1887 \\
    23 & inconsiderate\_jj & -1 & -1808 \\
    24 & vindictive\_jj & -1 & -1556 \\
    25 & sloth & -1 & -1249 \\
    26 & selfish\_jj & -1 & -1156 \\
    27 & ungracious\_jj & -1 & -1034 \\
    28 & shyness & -1 & -851 \\
    29 & unintelligent\_jj & -1 & -806 \\
    30 & forgetfulness & -1 & -803 \\
    392 & straightforward\_jj & 0 & 9959 \\
    393 & active\_jj & 0 & 9994 \\
    394 & quiet\_jj & 1 & 126 \\
    395 & economical\_jj & 1 & 154 \\
    396 & thorough\_jj & 1 & 201 \\
    397 & energetic\_jj & 1 & 499 \\
    398 & pleasant\_jj & 1 & 642 \\
    399 & gregarious\_jj & 1 & 865 \\
    400 & cautious\_jj & 1 & 1009 \\
    401 & generous\_jj & 1 & 1011 \\
    402 & cultured\_jj & 1 & 1516 \\
    403 & kind\_jj & 1 & 1728 \\
    404 & cordial\_jj & 1 & 1873 \\
    405 & vivacious\_jj & 1 & 1960 \\
    406 & confident\_jj & 1 & 2047 \\
    407 & courteous\_jj & 1 & 2165 \\
    408 & intelligent\_jj & 1 & 2292 \\
    409 & adventurous\_jj & 1 & 2513 \\
    410 & enthusiastic\_jj & 1 & 2694 \\
    411 & concise\_jj & 1 & 2723 \\
    412 & warm\_jj & 1 & 2959 \\
    413 & flexible\_jj & 1 & 3071 \\
    414 & dependable\_jj & 1 & 3499 \\
    415 & optimistic\_jj & 1 & 3683 \\
    416 & easygoing\_jj & 1 & 3707 \\
    417 & efficient\_jj & 1 & 4720 \\
    418 & reliable\_jj & 1 & 4725 \\
    419 & friendly\_jj & 1 & 4814 \\
    420 & considerate\_jj & 1 & 6954 \\
    421 & sociable\_jj & 2 & 1068 \\
    \hline
    \caption{Scores and rankings for most extreme 30 words in component \#2} \\
\end{longtable}
\begin{longtable}[!htbp]{| rlr@{.}l |}
    \hline
    \textbf{Rank} & \textbf{Word} & \multicolumn{2}{c|}{\textbf{Score}} \\
    \hline
    \endhead
    1 & abusive\_jj & -1 & -8964 \\
    2 & inconsiderate\_jj & -1 & -4689 \\
    3 & selfish\_jj & -1 & -4256 \\
    4 & disrespectful\_jj & -1 & -4183 \\
    5 & insensitive\_jj & -1 & -4052 \\
    6 & ignorant\_jj & -1 & -3918 \\
    7 & dishonest\_jj & -1 & -3804 \\
    8 & unscrupulous\_jj & -1 & -3194 \\
    9 & bigoted\_jj & -1 & -3191 \\
    10 & lenient\_jj & -1 & -2948 \\
    11 & gullible\_jj & -1 & -2421 \\
    12 & lazy\_jj & -1 & -2391 \\
    13 & greedy\_jj & -1 & -1989 \\
    14 & unfriendly\_jj & -1 & -1733 \\
    15 & vindictive\_jj & -1 & -1673 \\
    16 & unsympathetic\_jj & -1 & -1607 \\
    17 & pessimistic\_jj & -1 & -1577 \\
    18 & inefficient\_jj & -1 & -1288 \\
    19 & absent-minded\_jj & -1 & -1253 \\
    20 & impolite\_jj & -1 & -850 \\
    21 & naïve\_jj & -1 & -624 \\
    22 & rude\_jj & -1 & -570 \\
    23 & deceitful\_jj & -1 & -524 \\
    24 & intrusive\_jj & -1 & -516 \\
    25 & prejudiced\_jj & -1 & -40 \\
    26 & negligent\_jj & 0 & -9707 \\
    27 & unreliable\_jj & 0 & -9689 \\
    28 & thoughtless\_jj & 0 & -9283 \\
    29 & unintelligent\_jj & 0 & -9132 \\
    30 & egotistical\_jj & 0 & -9070 \\
    392 & shyness & 0 & 9431 \\
    393 & optimism & 0 & 9609 \\
    394 & melancholic\_jj & 0 & 9785 \\
    395 & empathy & 0 & 9820 \\
    396 & cunning & 0 & 9889 \\
    397 & inhibition & 0 & 9907 \\
    398 & modesty & 1 & 10 \\
    399 & courage & 1 & 50 \\
    400 & generosity & 1 & 386 \\
    401 & dependability & 1 & 651 \\
    402 & depth & 1 & 719 \\
    403 & artistic\_jj & 1 & 840 \\
    404 & spirit & 1 & 1401 \\
    405 & irritability & 1 & 1498 \\
    406 & persistence & 1 & 1789 \\
    407 & aloofness & 1 & 1899 \\
    408 & decisiveness & 1 & 2214 \\
    409 & precision & 1 & 2398 \\
    410 & meditative\_jj & 1 & 2500 \\
    411 & humor & 1 & 2613 \\
    412 & lethargy & 1 & 2816 \\
    413 & creativity & 1 & 3036 \\
    414 & candor & 1 & 3942 \\
    415 & warmth & 1 & 4490 \\
    416 & sophistication & 1 & 5041 \\
    417 & naturalness & 1 & 5154 \\
    418 & earthiness & 1 & 5299 \\
    419 & expressiveness & 1 & 5529 \\
    420 & spontaneity & 1 & 6318 \\
    421 & playfulness & 1 & 7505 \\
    \hline
    \caption{Scores and rankings for most extreme 30 words in component \#3} \\
\end{longtable}
\begin{longtable}[!htbp]{| rlr@{.}l |}
    \hline
    \textbf{Rank} & \textbf{Word} & \multicolumn{2}{c|}{\textbf{Score}} \\
    \hline
    \endhead
    1 & sincere\_jj & -1 & -6429 \\
    2 & considerate\_jj & -1 & -5523 \\
    3 & dignity & -1 & -4050 \\
    4 & courage & -1 & -3254 \\
    5 & selfless\_jj & -1 & -3060 \\
    6 & honest\_jj & -1 & -1151 \\
    7 & courageous\_jj & -1 & -1051 \\
    8 & principled\_jj & -1 & -641 \\
    9 & stupidity & -1 & -587 \\
    10 & negligence & -1 & -395 \\
    11 & selfish\_jj & -1 & -119 \\
    12 & compassionate\_jj & 0 & -9968 \\
    13 & candor & 0 & -9760 \\
    14 & courteous\_jj & 0 & -9644 \\
    15 & moral\_jj & 0 & -9586 \\
    16 & generosity & 0 & -9481 \\
    17 & recklessness & 0 & -9480 \\
    18 & negligent\_jj & 0 & -9415 \\
    19 & dishonest\_jj & 0 & -9201 \\
    20 & empathy & 0 & -8848 \\
    21 & systematic\_jj & 0 & -8828 \\
    22 & truthful\_jj & 0 & -8756 \\
    23 & selfishness & 0 & -8686 \\
    24 & sociable\_jj & 0 & -8681 \\
    25 & deliberate\_jj & 0 & -8557 \\
    26 & modesty & 0 & -8425 \\
    27 & cruelty & 0 & -8000 \\
    28 & ethical\_jj & 0 & -7968 \\
    29 & respectful\_jj & 0 & -7740 \\
    30 & morality & 0 & -7525 \\
    392 & thrifty\_jj & 0 & 6083 \\
    393 & aimlessness & 0 & 6159 \\
    394 & nonconforming\_jj & 0 & 6496 \\
    395 & insecure\_jj & 0 & 6646 \\
    396 & meditative\_jj & 0 & 6687 \\
    397 & grumpy\_jj & 0 & 6699 \\
    398 & placidity & 0 & 6732 \\
    399 & dominant\_jj & 0 & 6823 \\
    400 & surly\_jj & 0 & 6874 \\
    401 & extroverted\_jj & 0 & 6992 \\
    402 & aimless\_jj & 0 & 7000 \\
    403 & anxious\_jj & 0 & 7197 \\
    404 & quarrelsome\_jj & 0 & 7222 \\
    405 & cranky\_jj & 0 & 7401 \\
    406 & volatility & 0 & 7523 \\
    407 & morose\_jj & 0 & 7727 \\
    408 & volatile\_jj & 0 & 7792 \\
    409 & forgetful\_jj & 0 & 8178 \\
    410 & moody\_jj & 0 & 8266 \\
    411 & nervous\_jj & 0 & 8380 \\
    412 & cold\_jj & 0 & 9035 \\
    413 & forgetfulness & 0 & 9508 \\
    414 & erratic\_jj & 0 & 9927 \\
    415 & fretful\_jj & 0 & 9979 \\
    416 & sluggish\_jj & 1 & 2148 \\
    417 & lethargic\_jj & 1 & 2180 \\
    418 & lethargy & 1 & 5822 \\
    419 & irritability & 1 & 9494 \\
    420 & irritable\_jj & 1 & 9972 \\
    421 & absent-minded\_jj & 2 & 6972 \\
    \hline
    \caption{Scores and rankings for most extreme 30 words in component \#4} \\
\end{longtable}
\begin{longtable}[!htbp]{| rlr@{.}l |}
    \hline
    \textbf{Rank} & \textbf{Word} & \multicolumn{2}{c|}{\textbf{Score}} \\
    \hline
    \endhead
    1 & absent-minded\_jj & -2 & -7711 \\
    2 & refined\_jj & -1 & -6398 \\
    3 & economical\_jj & -1 & -5308 \\
    4 & concise\_jj & -1 & -4185 \\
    5 & efficient\_jj & -1 & -2772 \\
    6 & innovative\_jj & -1 & -2271 \\
    7 & inventive\_jj & -1 & -1779 \\
    8 & adaptable\_jj & -1 & -566 \\
    9 & analytical\_jj & -1 & -404 \\
    10 & underhanded\_jj & -1 & -225 \\
    11 & sophisticated\_jj & -1 & -97 \\
    12 & imaginative\_jj & 0 & -9499 \\
    13 & exacting\_jj & 0 & -9391 \\
    14 & devious\_jj & 0 & -8692 \\
    15 & cunning\_jj & 0 & -8664 \\
    16 & complex\_jj & 0 & -8188 \\
    17 & expressive\_jj & 0 & -8061 \\
    18 & unimaginative\_jj & 0 & -7710 \\
    19 & expressiveness & 0 & -7559 \\
    20 & insightful\_jj & 0 & -7486 \\
    21 & wordy\_jj & 0 & -7321 \\
    22 & inefficient\_jj & 0 & -7222 \\
    23 & meticulous\_jj & 0 & -7128 \\
    24 & unconventional\_jj & 0 & -6784 \\
    25 & perceptive\_jj & 0 & -6647 \\
    26 & creative\_jj & 0 & -6567 \\
    27 & manipulative\_jj & 0 & -6564 \\
    28 & precision & 0 & -6511 \\
    29 & individualistic\_jj & 0 & -6400 \\
    30 & unintelligent\_jj & 0 & -6350 \\
    392 & caution & 0 & 6874 \\
    393 & fear & 0 & 7265 \\
    394 & self-esteem & 0 & 7357 \\
    395 & cautious\_jj & 0 & 7381 \\
    396 & insecurity & 0 & 7385 \\
    397 & talkative\_jj & 0 & 7386 \\
    398 & polite\_jj & 0 & 7395 \\
    399 & bitter\_jj & 0 & 7400 \\
    400 & fearful\_jj & 0 & 7424 \\
    401 & instability & 0 & 7583 \\
    402 & quiet\_jj & 0 & 7677 \\
    403 & considerate\_jj & 0 & 7704 \\
    404 & jovial\_jj & 0 & 7926 \\
    405 & anxious\_jj & 0 & 8268 \\
    406 & warm\_jj & 0 & 8271 \\
    407 & gregarious\_jj & 0 & 8355 \\
    408 & pessimism & 0 & 8506 \\
    409 & distrust & 0 & 8666 \\
    410 & optimistic\_jj & 0 & 8765 \\
    411 & pessimistic\_jj & 0 & 8817 \\
    412 & kind\_jj & 0 & 9448 \\
    413 & silence & 0 & 9870 \\
    414 & sincere\_jj & 1 & 25 \\
    415 & irritable\_jj & 1 & 104 \\
    416 & cordial\_jj & 1 & 587 \\
    417 & nervous\_jj & 1 & 859 \\
    418 & sociable\_jj & 1 & 1660 \\
    419 & lethargy & 1 & 1905 \\
    420 & optimism & 1 & 2807 \\
    421 & irritability & 1 & 5846 \\
    \hline
    \caption{Scores and rankings for most extreme 30 words in component \#5} \\
\end{longtable}
\begin{longtable}[!htbp]{| rlr@{.}l |}
    \hline
    \textbf{Rank} & \textbf{Word} & \multicolumn{2}{c|}{\textbf{Score}} \\
    \hline
    \endhead
    1 & irritability & -3 & -3819 \\
    2 & lethargy & -2 & -529 \\
    3 & irritable\_jj & -1 & -9422 \\
    4 & forgetfulness & -1 & -7018 \\
    5 & economical\_jj & -1 & -6124 \\
    6 & absent-minded\_jj & -1 & -4449 \\
    7 & considerate\_jj & -1 & -4350 \\
    8 & abusive\_jj & -1 & -3781 \\
    9 & sociable\_jj & -1 & -2704 \\
    10 & self-esteem & -1 & -1708 \\
    11 & adaptable\_jj & 0 & -9977 \\
    12 & inhibition & 0 & -9802 \\
    13 & communicative\_jj & 0 & -9007 \\
    14 & compassionate\_jj & 0 & -8784 \\
    15 & intelligent\_jj & 0 & -8487 \\
    16 & kind\_jj & 0 & -8141 \\
    17 & disorganization & 0 & -7818 \\
    18 & efficient\_jj & 0 & -7559 \\
    19 & analytical\_jj & 0 & -7538 \\
    20 & empathy & 0 & -7210 \\
    21 & extroverted\_jj & 0 & -7181 \\
    22 & talkative\_jj & 0 & -6152 \\
    23 & volatility & 0 & -6097 \\
    24 & courteous\_jj & 0 & -6095 \\
    25 & suggestible\_jj & 0 & -5987 \\
    26 & forgetful\_jj & 0 & -5938 \\
    27 & selfishness & 0 & -5888 \\
    28 & dependability & 0 & -5718 \\
    29 & concise\_jj & 0 & -5708 \\
    30 & callousness & 0 & -5708 \\
    392 & quiet\_jj & 0 & 5390 \\
    393 & passionless\_jj & 0 & 5391 \\
    394 & formal\_jj & 0 & 5497 \\
    395 & crafty\_jj & 0 & 5573 \\
    396 & merry\_jj & 0 & 5597 \\
    397 & assertion & 0 & 5609 \\
    398 & surliness & 0 & 5668 \\
    399 & independence & 0 & 5678 \\
    400 & bullheaded\_jj & 0 & 5839 \\
    401 & cosmopolitan\_jj & 0 & 5842 \\
    402 & surly\_jj & 0 & 5965 \\
    403 & spirited\_jj & 0 & 5989 \\
    404 & skeptical\_jj & 0 & 6054 \\
    405 & scornful\_jj & 0 & 6141 \\
    406 & vain\_jj & 0 & 6151 \\
    407 & rambunctious\_jj & 0 & 6160 \\
    408 & earthiness & 0 & 6505 \\
    409 & tempestuous\_jj & 0 & 6634 \\
    410 & genial\_jj & 0 & 6675 \\
    411 & zestful\_jj & 0 & 6688 \\
    412 & curt\_jj & 0 & 6711 \\
    413 & homespun\_jj & 0 & 6843 \\
    414 & sly\_jj & 0 & 6910 \\
    415 & caustic\_jj & 0 & 6927 \\
    416 & bitter\_jj & 0 & 7191 \\
    417 & gruff\_jj & 0 & 7281 \\
    418 & flamboyant\_jj & 0 & 7322 \\
    419 & somber\_jj & 0 & 7645 \\
    420 & reserve & 0 & 7725 \\
    421 & folksy\_jj & 0 & 9946 \\
    \hline
    \caption{Scores and rankings for most extreme 30 words in component \#6} \\
\end{longtable}
\begin{longtable}[!htbp]{| rlr@{.}l |}
    \hline
    \textbf{Rank} & \textbf{Word} & \multicolumn{2}{c|}{\textbf{Score}} \\
    \hline
    \endhead
    1 & absent-minded\_jj & -6 & -1982 \\
    2 & cordial\_jj & -1 & -1886 \\
    3 & prompt\_jj & -1 & -256 \\
    4 & candor & 0 & -9374 \\
    5 & belligerence & 0 & -9062 \\
    6 & curt\_jj & 0 & -8807 \\
    7 & leniency & 0 & -8792 \\
    8 & frank\_jj & 0 & -8590 \\
    9 & respectful\_jj & 0 & -8170 \\
    10 & concise\_jj & 0 & -7606 \\
    11 & courage & 0 & -7172 \\
    12 & careful\_jj & 0 & -6995 \\
    13 & truthful\_jj & 0 & -6736 \\
    14 & decisiveness & 0 & -6681 \\
    15 & tactful\_jj & 0 & -6625 \\
    16 & sincere\_jj & 0 & -6545 \\
    17 & forceful\_jj & 0 & -6489 \\
    18 & polite\_jj & 0 & -6209 \\
    19 & principled\_jj & 0 & -6154 \\
    20 & touchy\_jj & 0 & -5513 \\
    21 & unemotional\_jj & 0 & -5446 \\
    22 & courteous\_jj & 0 & -5404 \\
    23 & pessimism & 0 & -5290 \\
    24 & stubbornness & 0 & -5251 \\
    25 & punctual\_jj & 0 & -4808 \\
    26 & compassionate\_jj & 0 & -4748 \\
    27 & dignity & 0 & -4721 \\
    28 & courageous\_jj & 0 & -4669 \\
    29 & cautious\_jj & 0 & -4593 \\
    30 & combative\_jj & 0 & -4514 \\
    392 & artistic\_jj & 0 & 4826 \\
    393 & volatile\_jj & 0 & 4850 \\
    394 & gullible\_jj & 0 & 4874 \\
    395 & irritability & 0 & 4952 \\
    396 & unconventional\_jj & 0 & 4988 \\
    397 & ambitious\_jj & 0 & 4992 \\
    398 & natural\_jj & 0 & 5003 \\
    399 & carefree\_jj & 0 & 5007 \\
    400 & instability & 0 & 5010 \\
    401 & happy-go-lucky\_jj & 0 & 5053 \\
    402 & sophisticated\_jj & 0 & 5055 \\
    403 & aimlessness & 0 & 5077 \\
    404 & insecure\_jj & 0 & 5120 \\
    405 & efficient\_jj & 0 & 5152 \\
    406 & rebellious\_jj & 0 & 5245 \\
    407 & refined\_jj & 0 & 5401 \\
    408 & cultured\_jj & 0 & 5445 \\
    409 & manipulative\_jj & 0 & 5479 \\
    410 & devious\_jj & 0 & 5521 \\
    411 & greedy\_jj & 0 & 5634 \\
    412 & egocentric\_jj & 0 & 5703 \\
    413 & conventional\_jj & 0 & 5904 \\
    414 & unimaginative\_jj & 0 & 5967 \\
    415 & sociable\_jj & 0 & 6416 \\
    416 & inefficient\_jj & 0 & 6428 \\
    417 & nonconforming\_jj & 0 & 7355 \\
    418 & adventurous\_jj & 0 & 7795 \\
    419 & unstable\_jj & 0 & 7967 \\
    420 & abusive\_jj & 0 & 8401 \\
    421 & autonomous\_jj & 0 & 9162 \\
    \hline
    \caption{Scores and rankings for most extreme 30 words in component \#7} \\
\end{longtable}
\begin{longtable}[!htbp]{| rlr@{.}l |}
    \hline
    \textbf{Rank} & \textbf{Word} & \multicolumn{2}{c|}{\textbf{Score}} \\
    \hline
    \endhead
    1 & concise\_jj & -1 & -4291 \\
    2 & cordial\_jj & -1 & -2890 \\
    3 & economical\_jj & -1 & -614 \\
    4 & antagonistic\_jj & 0 & -9475 \\
    5 & respectful\_jj & 0 & -9317 \\
    6 & lenient\_jj & 0 & -8754 \\
    7 & forceful\_jj & 0 & -8718 \\
    8 & belligerence & 0 & -8535 \\
    9 & assertive\_jj & 0 & -8487 \\
    10 & unemotional\_jj & 0 & -8465 \\
    11 & combative\_jj & 0 & -8422 \\
    12 & verbose\_jj & 0 & -7963 \\
    13 & restrained\_jj & 0 & -7723 \\
    14 & caustic\_jj & 0 & -7695 \\
    15 & somber\_jj & 0 & -7203 \\
    16 & frank\_jj & 0 & -7105 \\
    17 & truthful\_jj & 0 & -7086 \\
    18 & predictable\_jj & 0 & -7055 \\
    19 & insensitive\_jj & 0 & -6847 \\
    20 & self-critical\_jj & 0 & -6749 \\
    21 & refined\_jj & 0 & -6749 \\
    22 & dignified\_jj & 0 & -6533 \\
    23 & scornful\_jj & 0 & -6528 \\
    24 & indecisiveness & 0 & -6513 \\
    25 & intrusive\_jj & 0 & -6513 \\
    26 & impolite\_jj & 0 & -6455 \\
    27 & wordy\_jj & 0 & -6328 \\
    28 & candor & 0 & -6326 \\
    29 & prompt\_jj & 0 & -6250 \\
    30 & flippant\_jj & 0 & -6177 \\
    392 & creative\_jj & 0 & 5605 \\
    393 & suspicious\_jj & 0 & 5611 \\
    394 & jealous\_jj & 0 & 5681 \\
    395 & thrift & 0 & 5685 \\
    396 & adventurous\_jj & 0 & 5732 \\
    397 & cranky\_jj & 0 & 5957 \\
    398 & shy\_jj & 0 & 5958 \\
    399 & fear & 0 & 6031 \\
    400 & organized\_jj & 0 & 6055 \\
    401 & brave\_jj & 0 & 6109 \\
    402 & gregarious\_jj & 0 & 6127 \\
    403 & cruelty & 0 & 6313 \\
    404 & negligent\_jj & 0 & 6368 \\
    405 & intelligent\_jj & 0 & 6434 \\
    406 & generosity & 0 & 6469 \\
    407 & lazy\_jj & 0 & 6825 \\
    408 & courtesy & 0 & 6871 \\
    409 & gullible\_jj & 0 & 6914 \\
    410 & organization & 0 & 7090 \\
    411 & negligence & 0 & 7194 \\
    412 & spirit & 0 & 7333 \\
    413 & greedy\_jj & 0 & 7485 \\
    414 & inconsiderate\_jj & 0 & 7665 \\
    415 & kind\_jj & 0 & 8271 \\
    416 & proud\_jj & 0 & 9142 \\
    417 & vivacious\_jj & 0 & 9538 \\
    418 & sociable\_jj & 1 & 732 \\
    419 & unscrupulous\_jj & 1 & 1053 \\
    420 & charitable\_jj & 1 & 1183 \\
    421 & absent-minded\_jj & 3 & 9481 \\
    \hline
    \caption{Scores and rankings for most extreme 30 words in component \#8} \\
\end{longtable}
\begin{longtable}[!htbp]{| rlr@{.}l |}
    \hline
    \textbf{Rank} & \textbf{Word} & \multicolumn{2}{c|}{\textbf{Score}} \\
    \hline
    \endhead
    1 & distrustful\_jj & -1 & -6655 \\
    2 & individualistic\_jj & -1 & -3343 \\
    3 & distrust & -1 & -2525 \\
    4 & assertive\_jj & -1 & -2168 \\
    5 & aloofness & -1 & -1807 \\
    6 & belligerence & -1 & -1706 \\
    7 & antagonistic\_jj & -1 & -440 \\
    8 & autonomous\_jj & 0 & -9521 \\
    9 & accommodating\_jj & 0 & -9212 \\
    10 & dependability & 0 & -8902 \\
    11 & pessimistic\_jj & 0 & -8309 \\
    12 & uncritical\_jj & 0 & -8109 \\
    13 & obstinate\_jj & 0 & -7719 \\
    14 & insecure\_jj & 0 & -7309 \\
    15 & insecurity & 0 & -7292 \\
    16 & independence & 0 & -7257 \\
    17 & adaptable\_jj & 0 & -6925 \\
    18 & pessimism & 0 & -6885 \\
    19 & instability & 0 & -6648 \\
    20 & indecisive\_jj & 0 & -6529 \\
    21 & prejudiced\_jj & 0 & -6101 \\
    22 & cooperation & 0 & -6094 \\
    23 & extroverted\_jj & 0 & -6085 \\
    24 & unfriendly\_jj & 0 & -5924 \\
    25 & optimistic\_jj & 0 & -5834 \\
    26 & cosmopolitan\_jj & 0 & -5814 \\
    27 & quarrelsome\_jj & 0 & -5784 \\
    28 & ambition & 0 & -5758 \\
    29 & generosity & 0 & -5751 \\
    30 & optimism & 0 & -5715 \\
    392 & lazy\_jj & 0 & 6017 \\
    393 & spontaneous\_jj & 0 & 6083 \\
    394 & cold\_jj & 0 & 6109 \\
    395 & straightforward\_jj & 0 & 6167 \\
    396 & patient\_jj & 0 & 6211 \\
    397 & crabby\_jj & 0 & 6439 \\
    398 & curt\_jj & 0 & 6460 \\
    399 & humorous\_jj & 0 & 6830 \\
    400 & forgetfulness & 0 & 7039 \\
    401 & thorough\_jj & 0 & 7067 \\
    402 & cruel\_jj & 0 & 7096 \\
    403 & systematic\_jj & 0 & 7133 \\
    404 & verbal\_jj & 0 & 7327 \\
    405 & meditative\_jj & 0 & 7369 \\
    406 & simple\_jj & 0 & 7811 \\
    407 & reckless\_jj & 0 & 7845 \\
    408 & playful\_jj & 0 & 8169 \\
    409 & warm\_jj & 0 & 8181 \\
    410 & caustic\_jj & 0 & 8358 \\
    411 & sloppy\_jj & 0 & 8710 \\
    412 & folksy\_jj & 0 & 9048 \\
    413 & cruelty & 0 & 9132 \\
    414 & concise\_jj & 0 & 9629 \\
    415 & irritability & 1 & 10 \\
    416 & prompt\_jj & 1 & 475 \\
    417 & deliberate\_jj & 1 & 506 \\
    418 & abusive\_jj & 1 & 1352 \\
    419 & careless\_jj & 1 & 2187 \\
    420 & negligence & 1 & 4233 \\
    421 & negligent\_jj & 1 & 6293 \\
    \hline
    \caption{Scores and rankings for most extreme 30 words in component \#9} \\
\end{longtable}
\begin{longtable}[!htbp]{| rlr@{.}l |}
    \hline
    \textbf{Rank} & \textbf{Word} & \multicolumn{2}{c|}{\textbf{Score}} \\
    \hline
    \endhead
    1 & negligent\_jj & -1 & -7543 \\
    2 & friendly\_jj & -1 & -1854 \\
    3 & negligence & -1 & -1161 \\
    4 & cordial\_jj & -1 & -921 \\
    5 & reckless\_jj & -1 & -252 \\
    6 & recklessness & 0 & -9322 \\
    7 & belligerence & 0 & -9167 \\
    8 & economical\_jj & 0 & -8645 \\
    9 & easygoing\_jj & 0 & -8443 \\
    10 & leniency & 0 & -8226 \\
    11 & surly\_jj & 0 & -8164 \\
    12 & lenient\_jj & 0 & -7327 \\
    13 & cooperation & 0 & -7305 \\
    14 & unrestrained\_jj & 0 & -6980 \\
    15 & antagonistic\_jj & 0 & -6910 \\
    16 & mannerly\_jj & 0 & -6868 \\
    17 & cruelty & 0 & -6768 \\
    18 & cooperative\_jj & 0 & -6601 \\
    19 & gregarious\_jj & 0 & -6345 \\
    20 & orderly\_jj & 0 & -6180 \\
    21 & unscrupulous\_jj & 0 & -6061 \\
    22 & docile\_jj & 0 & -6026 \\
    23 & efficient\_jj & 0 & -5727 \\
    24 & erratic\_jj & 0 & -5697 \\
    25 & ruthless\_jj & 0 & -5679 \\
    26 & sociable\_jj & 0 & -5666 \\
    27 & jovial\_jj & 0 & -5654 \\
    28 & unfriendly\_jj & 0 & -5628 \\
    29 & inconsiderate\_jj & 0 & -5566 \\
    30 & inefficient\_jj & 0 & -5562 \\
    392 & proud\_jj & 0 & 5666 \\
    393 & gullibility & 0 & 5740 \\
    394 & precise\_jj & 0 & 5832 \\
    395 & understanding & 0 & 5887 \\
    396 & helpful\_jj & 0 & 6011 \\
    397 & lazy\_jj & 0 & 6068 \\
    398 & wordy\_jj & 0 & 6075 \\
    399 & cautious\_jj & 0 & 6086 \\
    400 & envious\_jj & 0 & 6159 \\
    401 & confident\_jj & 0 & 6489 \\
    402 & logic & 0 & 6592 \\
    403 & depth & 0 & 6713 \\
    404 & careful\_jj & 0 & 6766 \\
    405 & naïve\_jj & 0 & 6799 \\
    406 & wishy-washy\_jj & 0 & 6888 \\
    407 & truthful\_jj & 0 & 7029 \\
    408 & nervous\_jj & 0 & 7523 \\
    409 & gullible\_jj & 0 & 7616 \\
    410 & perceptive\_jj & 0 & 7651 \\
    411 & optimistic\_jj & 0 & 8222 \\
    412 & philosophical\_jj & 0 & 8621 \\
    413 & abusive\_jj & 0 & 8636 \\
    414 & ignorant\_jj & 0 & 8960 \\
    415 & unkind\_jj & 0 & 8961 \\
    416 & curious\_jj & 0 & 9215 \\
    417 & pessimistic\_jj & 1 & 66 \\
    418 & insightful\_jj & 1 & 494 \\
    419 & skeptical\_jj & 1 & 662 \\
    420 & insight & 1 & 5064 \\
    421 & concise\_jj & 1 & 6369 \\
    \hline
    \caption{Scores and rankings for most extreme 30 words in component \#10} \\
\end{longtable}
\begin{longtable}[!htbp]{| rlr@{.}l |}
    \hline
    \textbf{Rank} & \textbf{Word} & \multicolumn{2}{c|}{\textbf{Score}} \\
    \hline
    \endhead
    1 & economical\_jj & -1 & -728 \\
    2 & pessimistic\_jj & -1 & -54 \\
    3 & lethargic\_jj & 0 & -9462 \\
    4 & sloppy\_jj & 0 & -8373 \\
    5 & sluggish\_jj & 0 & -8113 \\
    6 & lazy\_jj & 0 & -7171 \\
    7 & optimism & 0 & -7045 \\
    8 & miserly\_jj & 0 & -6550 \\
    9 & lethargy & 0 & -6498 \\
    10 & optimistic\_jj & 0 & -6190 \\
    11 & reckless\_jj & 0 & -6151 \\
    12 & cautious\_jj & 0 & -5995 \\
    13 & careless\_jj & 0 & -5970 \\
    14 & indecisive\_jj & 0 & -5963 \\
    15 & unimaginative\_jj & 0 & -5843 \\
    16 & foolhardy\_jj & 0 & -5733 \\
    17 & dependable\_jj & 0 & -5722 \\
    18 & stupidity & 0 & -5709 \\
    19 & cynical\_jj & 0 & -5601 \\
    20 & pessimism & 0 & -5522 \\
    21 & rash\_jj & 0 & -5191 \\
    22 & indecisiveness & 0 & -5176 \\
    23 & selfish\_jj & 0 & -5120 \\
    24 & wishy-washy\_jj & 0 & -5061 \\
    25 & inefficient\_jj & 0 & -4971 \\
    26 & stingy\_jj & 0 & -4970 \\
    27 & recklessness & 0 & -4851 \\
    28 & gullible\_jj & 0 & -4771 \\
    29 & smug\_jj & 0 & -4726 \\
    30 & thrifty\_jj & 0 & -4675 \\
    392 & understanding & 0 & 4715 \\
    393 & orderly\_jj & 0 & 4729 \\
    394 & active\_jj & 0 & 4872 \\
    395 & cruelty & 0 & 4881 \\
    396 & distrustful\_jj & 0 & 4921 \\
    397 & respectful\_jj & 0 & 4935 \\
    398 & cooperation & 0 & 4963 \\
    399 & silence & 0 & 5006 \\
    400 & intrusiveness & 0 & 5010 \\
    401 & emotional\_jj & 0 & 5153 \\
    402 & verbal\_jj & 0 & 5157 \\
    403 & systematic\_jj & 0 & 5205 \\
    404 & touchy\_jj & 0 & 5289 \\
    405 & formal\_jj & 0 & 5316 \\
    406 & organized\_jj & 0 & 5474 \\
    407 & trustful\_jj & 0 & 5576 \\
    408 & organization & 0 & 5596 \\
    409 & informal\_jj & 0 & 5688 \\
    410 & melancholic\_jj & 0 & 5771 \\
    411 & belligerence & 0 & 6155 \\
    412 & earthiness & 0 & 6286 \\
    413 & independence & 0 & 6908 \\
    414 & suspicious\_jj & 0 & 7025 \\
    415 & independent\_jj & 0 & 7132 \\
    416 & explosive\_jj & 0 & 7264 \\
    417 & diplomatic\_jj & 0 & 7534 \\
    418 & nonconforming\_jj & 0 & 7649 \\
    419 & autonomous\_jj & 0 & 8119 \\
    420 & antagonistic\_jj & 0 & 8134 \\
    421 & abusive\_jj & 4 & 9517 \\
    \hline
    \caption{Scores and rankings for most extreme 30 words in component \#11} \\
\end{longtable}
\begin{longtable}[!htbp]{| rlr@{.}l |}
    \hline
    \textbf{Rank} & \textbf{Word} & \multicolumn{2}{c|}{\textbf{Score}} \\
    \hline
    \endhead
    1 & abusive\_jj & -1 & -9879 \\
    2 & erratic\_jj & -1 & -3016 \\
    3 & explosive\_jj & -1 & -351 \\
    4 & unpredictable\_jj & -1 & -20 \\
    5 & tenacious\_jj & 0 & -9758 \\
    6 & indecisive\_jj & 0 & -9260 \\
    7 & unstable\_jj & 0 & -8561 \\
    8 & expressive\_jj & 0 & -8499 \\
    9 & inventive\_jj & 0 & -8396 \\
    10 & assertive\_jj & 0 & -8048 \\
    11 & energetic\_jj & 0 & -7820 \\
    12 & obstinate\_jj & 0 & -7651 \\
    13 & forceful\_jj & 0 & -7259 \\
    14 & ruthless\_jj & 0 & -7143 \\
    15 & combative\_jj & 0 & -7133 \\
    16 & stubbornness & 0 & -7066 \\
    17 & reckless\_jj & 0 & -7037 \\
    18 & sluggish\_jj & 0 & -6492 \\
    19 & manipulative\_jj & 0 & -6421 \\
    20 & impetuous\_jj & 0 & -6337 \\
    21 & optimistic\_jj & 0 & -6317 \\
    22 & belligerence & 0 & -6293 \\
    23 & optimism & 0 & -6164 \\
    24 & volatile\_jj & 0 & -6126 \\
    25 & cunning\_jj & 0 & -6122 \\
    26 & stubborn\_jj & 0 & -6050 \\
    27 & pessimism & 0 & -5868 \\
    28 & decisiveness & 0 & -5858 \\
    29 & deceitful\_jj & 0 & -5817 \\
    30 & cunning & 0 & -5568 \\
    392 & independence & 0 & 5052 \\
    393 & polite\_jj & 0 & 5060 \\
    394 & understanding\_jj & 0 & 5074 \\
    395 & conventionality & 0 & 5283 \\
    396 & unsystematic\_jj & 0 & 5381 \\
    397 & traditional\_jj & 0 & 5527 \\
    398 & intellectuality & 0 & 5550 \\
    399 & meddlesome\_jj & 0 & 5688 \\
    400 & cordial\_jj & 0 & 5701 \\
    401 & pleasant\_jj & 0 & 5702 \\
    402 & nonconformity & 0 & 5846 \\
    403 & unsociable\_jj & 0 & 5942 \\
    404 & obliging\_jj & 0 & 5985 \\
    405 & mannerly\_jj & 0 & 6061 \\
    406 & crabby\_jj & 0 & 6109 \\
    407 & diplomatic\_jj & 0 & 6321 \\
    408 & rudeness & 0 & 6394 \\
    409 & aimlessness & 0 & 6430 \\
    410 & patient\_jj & 0 & 6757 \\
    411 & formal\_jj & 0 & 6870 \\
    412 & nosey\_jj & 0 & 6878 \\
    413 & inconsiderate\_jj & 0 & 7394 \\
    414 & trustful\_jj & 0 & 7497 \\
    415 & efficiency & 0 & 7715 \\
    416 & bossiness & 0 & 8500 \\
    417 & friendly\_jj & 0 & 8654 \\
    418 & punctuality & 0 & 8718 \\
    419 & punctual\_jj & 0 & 8736 \\
    420 & nonconforming\_jj & 1 & 618 \\
    421 & prompt\_jj & 1 & 640 \\
    \hline
    \caption{Scores and rankings for most extreme 30 words in component \#12} \\
\end{longtable}
\begin{longtable}[!htbp]{| rlr@{.}l |}
    \hline
    \textbf{Rank} & \textbf{Word} & \multicolumn{2}{c|}{\textbf{Score}} \\
    \hline
    \endhead
    1 & explosive\_jj & -1 & -1368 \\
    2 & suspicious\_jj & -1 & -274 \\
    3 & negligent\_jj & -1 & -177 \\
    4 & unscrupulous\_jj & -1 & -29 \\
    5 & organized\_jj & 0 & -8557 \\
    6 & intelligence & 0 & -7984 \\
    7 & distrust & 0 & -7601 \\
    8 & instability & 0 & -7505 \\
    9 & concise\_jj & 0 & -6962 \\
    10 & deliberate\_jj & 0 & -6921 \\
    11 & systematic\_jj & 0 & -6528 \\
    12 & folksy\_jj & 0 & -6527 \\
    13 & disorganization & 0 & -6322 \\
    14 & sophistication & 0 & -6017 \\
    15 & diplomatic\_jj & 0 & -5878 \\
    16 & antagonistic\_jj & 0 & -5713 \\
    17 & thorough\_jj & 0 & -5551 \\
    18 & wordy\_jj & 0 & -5544 \\
    19 & humorous\_jj & 0 & -5411 \\
    20 & self-critical\_jj & 0 & -5405 \\
    21 & secretive\_jj & 0 & -5389 \\
    22 & inquisitive\_jj & 0 & -5384 \\
    23 & inventive\_jj & 0 & -5368 \\
    24 & quarrelsome\_jj & 0 & -5361 \\
    25 & touchy\_jj & 0 & -5299 \\
    26 & philosophical\_jj & 0 & -5296 \\
    27 & cooperation & 0 & -5284 \\
    28 & irritability & 0 & -4969 \\
    29 & analytical\_jj & 0 & -4847 \\
    30 & autonomous\_jj & 0 & -4811 \\
    392 & reserve & 0 & 4557 \\
    393 & dependable\_jj & 0 & 4579 \\
    394 & sluggish\_jj & 0 & 4677 \\
    395 & courtesy & 0 & 4737 \\
    396 & decisiveness & 0 & 4763 \\
    397 & thrift & 0 & 4781 \\
    398 & generous\_jj & 0 & 4858 \\
    399 & ungracious\_jj & 0 & 4951 \\
    400 & warmth & 0 & 5124 \\
    401 & modest\_jj & 0 & 5277 \\
    402 & dignified\_jj & 0 & 5296 \\
    403 & cold\_jj & 0 & 5680 \\
    404 & modesty & 0 & 6075 \\
    405 & pleasant\_jj & 0 & 6156 \\
    406 & punctuality & 0 & 6568 \\
    407 & generosity & 0 & 6696 \\
    408 & rude\_jj & 0 & 7051 \\
    409 & dependability & 0 & 7064 \\
    410 & efficiency & 0 & 7354 \\
    411 & efficient\_jj & 0 & 7417 \\
    412 & warm\_jj & 0 & 7436 \\
    413 & selfless\_jj & 0 & 7439 \\
    414 & dignity & 0 & 7491 \\
    415 & courage & 0 & 7817 \\
    416 & brave\_jj & 0 & 8506 \\
    417 & stingy\_jj & 0 & 9633 \\
    418 & thrifty\_jj & 1 & 780 \\
    419 & economical\_jj & 1 & 4627 \\
    420 & refined\_jj & 1 & 4747 \\
    421 & abusive\_jj & 3 & 4811 \\
    \hline
    \caption{Scores and rankings for most extreme 30 words in component \#13} \\
\end{longtable}
\begin{longtable}[!htbp]{| rlr@{.}l |}
    \hline
    \textbf{Rank} & \textbf{Word} & \multicolumn{2}{c|}{\textbf{Score}} \\
    \hline
    \endhead
    1 & individualistic\_jj & -1 & -876 \\
    2 & peaceful\_jj & -1 & -645 \\
    3 & nonconformity & -1 & -608 \\
    4 & meditative\_jj & 0 & -9809 \\
    5 & cruelty & 0 & -8566 \\
    6 & brave\_jj & 0 & -8514 \\
    7 & contemplative\_jj & 0 & -8292 \\
    8 & lenient\_jj & 0 & -8258 \\
    9 & principled\_jj & 0 & -8103 \\
    10 & rebellious\_jj & 0 & -7763 \\
    11 & expressive\_jj & 0 & -7594 \\
    12 & expressiveness & 0 & -7333 \\
    13 & vigorous\_jj & 0 & -7176 \\
    14 & somber\_jj & 0 & -7137 \\
    15 & inventive\_jj & 0 & -6979 \\
    16 & dignified\_jj & 0 & -6807 \\
    17 & understanding\_jj & 0 & -6684 \\
    18 & unreflective\_jj & 0 & -6616 \\
    19 & prejudiced\_jj & 0 & -6614 \\
    20 & silent\_jj & 0 & -6461 \\
    21 & respectful\_jj & 0 & -6219 \\
    22 & silence & 0 & -6217 \\
    23 & assertive\_jj & 0 & -5973 \\
    24 & aimlessness & 0 & -5966 \\
    25 & philosophical\_jj & 0 & -5843 \\
    26 & morality & 0 & -5817 \\
    27 & accommodating\_jj & 0 & -5806 \\
    28 & courageous\_jj & 0 & -5777 \\
    29 & charitable\_jj & 0 & -5504 \\
    30 & restrained\_jj & 0 & -5440 \\
    392 & boastful\_jj & 0 & 5089 \\
    393 & dependable\_jj & 0 & 5114 \\
    394 & patient\_jj & 0 & 5146 \\
    395 & undependable\_jj & 0 & 5152 \\
    396 & volatility & 0 & 5270 \\
    397 & courteous\_jj & 0 & 5364 \\
    398 & aloofness & 0 & 5663 \\
    399 & depth & 0 & 5777 \\
    400 & underhanded\_jj & 0 & 5940 \\
    401 & verbose\_jj & 0 & 5941 \\
    402 & ungracious\_jj & 0 & 5985 \\
    403 & conceited\_jj & 0 & 5988 \\
    404 & suspicious\_jj & 0 & 6260 \\
    405 & friendly\_jj & 0 & 6315 \\
    406 & erratic\_jj & 0 & 6349 \\
    407 & candor & 0 & 6525 \\
    408 & diplomatic\_jj & 0 & 6556 \\
    409 & humor & 0 & 6825 \\
    410 & cordial\_jj & 0 & 7113 \\
    411 & unscrupulous\_jj & 0 & 7203 \\
    412 & belligerence & 0 & 7436 \\
    413 & sophistication & 0 & 8426 \\
    414 & insight & 0 & 8451 \\
    415 & reliable\_jj & 0 & 9142 \\
    416 & unreliable\_jj & 0 & 9238 \\
    417 & explosive\_jj & 0 & 9899 \\
    418 & efficiency & 0 & 9987 \\
    419 & punctuality & 1 & 393 \\
    420 & intelligence & 1 & 762 \\
    421 & dependability & 1 & 2078 \\
    \hline
    \caption{Scores and rankings for most extreme 30 words in component \#14} \\
\end{longtable}
\begin{longtable}[!htbp]{| rlr@{.}l |}
    \hline
    \textbf{Rank} & \textbf{Word} & \multicolumn{2}{c|}{\textbf{Score}} \\
    \hline
    \endhead
    1 & negligent\_jj & -1 & -6156 \\
    2 & negligence & 0 & -9431 \\
    3 & leniency & 0 & -8334 \\
    4 & dependability & 0 & -7922 \\
    5 & reserve & 0 & -7768 \\
    6 & adventurous\_jj & 0 & -7735 \\
    7 & careless\_jj & 0 & -7280 \\
    8 & optimistic\_jj & 0 & -7096 \\
    9 & lenient\_jj & 0 & -6629 \\
    10 & insight & 0 & -6573 \\
    11 & animation & 0 & -6227 \\
    12 & punctuality & 0 & -6174 \\
    13 & understanding\_jj & 0 & -6142 \\
    14 & fretful\_jj & 0 & -5773 \\
    15 & self-disciplined\_jj & 0 & -5716 \\
    16 & fastidious\_jj & 0 & -5661 \\
    17 & patient\_jj & 0 & -5637 \\
    18 & punctual\_jj & 0 & -5620 \\
    19 & pessimistic\_jj & 0 & -5446 \\
    20 & selfless\_jj & 0 & -5430 \\
    21 & surly\_jj & 0 & -5366 \\
    22 & nonconforming\_jj & 0 & -5365 \\
    23 & assured\_jj & 0 & -5291 \\
    24 & rambunctious\_jj & 0 & -5247 \\
    25 & indecisive\_jj & 0 & -5237 \\
    26 & unambitious\_jj & 0 & -5210 \\
    27 & impudent\_jj & 0 & -5187 \\
    28 & artistic\_jj & 0 & -5120 \\
    29 & inquisitive\_jj & 0 & -4967 \\
    30 & unsociable\_jj & 0 & -4838 \\
    392 & predictable\_jj & 0 & 4933 \\
    393 & unfriendly\_jj & 0 & 5052 \\
    394 & stupidity & 0 & 5195 \\
    395 & unintelligent\_jj & 0 & 5354 \\
    396 & warmth & 0 & 5632 \\
    397 & earthiness & 0 & 5706 \\
    398 & cordial\_jj & 0 & 5716 \\
    399 & pleasant\_jj & 0 & 5797 \\
    400 & cosmopolitan\_jj & 0 & 5851 \\
    401 & prejudice & 0 & 5873 \\
    402 & touchy\_jj & 0 & 6400 \\
    403 & bigoted\_jj & 0 & 6546 \\
    404 & economical\_jj & 0 & 6793 \\
    405 & unstable\_jj & 0 & 6818 \\
    406 & peaceful\_jj & 0 & 6862 \\
    407 & bitter\_jj & 0 & 6891 \\
    408 & harsh\_jj & 0 & 7136 \\
    409 & cold\_jj & 0 & 7329 \\
    410 & friendly\_jj & 0 & 7363 \\
    411 & deliberate\_jj & 0 & 7749 \\
    412 & shallow\_jj & 0 & 7911 \\
    413 & distrust & 0 & 8020 \\
    414 & insecurity & 0 & 9167 \\
    415 & folksy\_jj & 0 & 9543 \\
    416 & volatile\_jj & 0 & 9890 \\
    417 & cruel\_jj & 1 & 955 \\
    418 & absent-minded\_jj & 1 & 1975 \\
    419 & instability & 1 & 3269 \\
    420 & refined\_jj & 1 & 3821 \\
    421 & warm\_jj & 1 & 4556 \\
    \hline
    \caption{Scores and rankings for most extreme 30 words in component \#15} \\
\end{longtable}
\begin{longtable}[!htbp]{| rlr@{.}l |}
    \hline
    \textbf{Rank} & \textbf{Word} & \multicolumn{2}{c|}{\textbf{Score}} \\
    \hline
    \endhead
    1 & explosive\_jj & -1 & -812 \\
    2 & lethargy & -1 & -488 \\
    3 & courageous\_jj & -1 & -55 \\
    4 & defensive\_jj & 0 & -9608 \\
    5 & brave\_jj & 0 & -9506 \\
    6 & courage & 0 & -9432 \\
    7 & irritability & 0 & -9366 \\
    8 & independence & 0 & -8598 \\
    9 & selfless\_jj & 0 & -8254 \\
    10 & decisive\_jj & 0 & -8252 \\
    11 & irritable\_jj & 0 & -8033 \\
    12 & tenacious\_jj & 0 & -7737 \\
    13 & conceited\_jj & 0 & -7248 \\
    14 & unintelligent\_jj & 0 & -6989 \\
    15 & silence & 0 & -6667 \\
    16 & ungracious\_jj & 0 & -6560 \\
    17 & rash\_jj & 0 & -6432 \\
    18 & vain\_jj & 0 & -6408 \\
    19 & gregariousness & 0 & -6389 \\
    20 & rebellious\_jj & 0 & -6024 \\
    21 & peaceful\_jj & 0 & -5987 \\
    22 & withdrawn\_jj & 0 & -5436 \\
    23 & proud\_jj & 0 & -5303 \\
    24 & principled\_jj & 0 & -5246 \\
    25 & dignified\_jj & 0 & -5101 \\
    26 & foolhardy\_jj & 0 & -4976 \\
    27 & cunning\_jj & 0 & -4951 \\
    28 & ruthless\_jj & 0 & -4916 \\
    29 & concise\_jj & 0 & -4896 \\
    30 & egotistical\_jj & 0 & -4807 \\
    392 & sluggish\_jj & 0 & 4796 \\
    393 & distrust & 0 & 4804 \\
    394 & morose\_jj & 0 & 4923 \\
    395 & leniency & 0 & 4999 \\
    396 & earthy\_jj & 0 & 5004 \\
    397 & sociable\_jj & 0 & 5041 \\
    398 & passivity & 0 & 5049 \\
    399 & frivolous\_jj & 0 & 5049 \\
    400 & impersonal\_jj & 0 & 5089 \\
    401 & meditative\_jj & 0 & 5293 \\
    402 & skeptical\_jj & 0 & 5581 \\
    403 & curiosity & 0 & 5793 \\
    404 & insight & 0 & 5883 \\
    405 & refined\_jj & 0 & 5914 \\
    406 & frivolity & 0 & 6224 \\
    407 & cruelty & 0 & 6229 \\
    408 & playfulness & 0 & 6565 \\
    409 & contemplative\_jj & 0 & 7180 \\
    410 & cautious\_jj & 0 & 7430 \\
    411 & sophistication & 0 & 7809 \\
    412 & optimism & 0 & 8230 \\
    413 & surly\_jj & 0 & 8309 \\
    414 & negligent\_jj & 0 & 8565 \\
    415 & optimistic\_jj & 0 & 9486 \\
    416 & negligence & 0 & 9544 \\
    417 & abusive\_jj & 0 & 9989 \\
    418 & unscrupulous\_jj & 1 & 719 \\
    419 & lenient\_jj & 1 & 1383 \\
    420 & pessimism & 1 & 2614 \\
    421 & pessimistic\_jj & 1 & 2804 \\
    \hline
    \caption{Scores and rankings for most extreme 30 words in component \#16} \\
\end{longtable}
\begin{longtable}[!htbp]{| rlr@{.}l |}
    \hline
    \textbf{Rank} & \textbf{Word} & \multicolumn{2}{c|}{\textbf{Score}} \\
    \hline
    \endhead
    1 & concise\_jj & -2 & -2847 \\
    2 & thrift & -1 & -1333 \\
    3 & abusive\_jj & 0 & -9847 \\
    4 & sloth & 0 & -8662 \\
    5 & disorganization & 0 & -8496 \\
    6 & quarrelsome\_jj & 0 & -8256 \\
    7 & considerate\_jj & 0 & -7989 \\
    8 & thorough\_jj & 0 & -7798 \\
    9 & unscrupulous\_jj & 0 & -7221 \\
    10 & courteous\_jj & 0 & -7118 \\
    11 & systematic\_jj & 0 & -7115 \\
    12 & haphazard\_jj & 0 & -7043 \\
    13 & tenacious\_jj & 0 & -6704 \\
    14 & steady\_jj & 0 & -6667 \\
    15 & thrifty\_jj & 0 & -6363 \\
    16 & vigorous\_jj & 0 & -6295 \\
    17 & selfishness & 0 & -6291 \\
    18 & sloppy\_jj & 0 & -6192 \\
    19 & dependable\_jj & 0 & -6132 \\
    20 & slothful\_jj & 0 & -5842 \\
    21 & indecisiveness & 0 & -5656 \\
    22 & greedy\_jj & 0 & -5650 \\
    23 & stinginess & 0 & -5448 \\
    24 & surliness & 0 & -5318 \\
    25 & prompt\_jj & 0 & -5293 \\
    26 & docile\_jj & 0 & -5208 \\
    27 & distrust & 0 & -5144 \\
    28 & respectful\_jj & 0 & -4618 \\
    29 & bossiness & 0 & -4581 \\
    30 & ruthless\_jj & 0 & -4483 \\
    392 & candor & 0 & 4535 \\
    393 & negligent\_jj & 0 & 4577 \\
    394 & uninhibited\_jj & 0 & 4671 \\
    395 & earthy\_jj & 0 & 4690 \\
    396 & flippant\_jj & 0 & 4772 \\
    397 & unfriendly\_jj & 0 & 4794 \\
    398 & rebellious\_jj & 0 & 4974 \\
    399 & assertion & 0 & 5021 \\
    400 & efficiency & 0 & 5197 \\
    401 & irritable\_jj & 0 & 5319 \\
    402 & inconsiderate\_jj & 0 & 5338 \\
    403 & cruel\_jj & 0 & 5352 \\
    404 & leniency & 0 & 5817 \\
    405 & rude\_jj & 0 & 5861 \\
    406 & condescending\_jj & 0 & 5890 \\
    407 & impolite\_jj & 0 & 6092 \\
    408 & lenient\_jj & 0 & 6106 \\
    409 & impractical\_jj & 0 & 6123 \\
    410 & emotional\_jj & 0 & 6157 \\
    411 & caustic\_jj & 0 & 6274 \\
    412 & expressive\_jj & 0 & 6329 \\
    413 & unstable\_jj & 0 & 6833 \\
    414 & ungracious\_jj & 0 & 7736 \\
    415 & friendly\_jj & 0 & 7886 \\
    416 & disrespectful\_jj & 0 & 7908 \\
    417 & unkind\_jj & 0 & 7949 \\
    418 & earthiness & 0 & 8443 \\
    419 & insensitive\_jj & 0 & 9410 \\
    420 & economical\_jj & 1 & 535 \\
    421 & explosive\_jj & 1 & 3028 \\
    \hline
    \caption{Scores and rankings for most extreme 30 words in component \#17} \\
\end{longtable}
\begin{longtable}[!htbp]{| rlr@{.}l |}
    \hline
    \textbf{Rank} & \textbf{Word} & \multicolumn{2}{c|}{\textbf{Score}} \\
    \hline
    \endhead
    1 & prompt\_jj & -1 & -7620 \\
    2 & unscrupulous\_jj & -1 & -4103 \\
    3 & belligerence & -1 & -1266 \\
    4 & impolite\_jj & 0 & -9695 \\
    5 & lethargy & 0 & -9378 \\
    6 & sophistication & 0 & -9314 \\
    7 & charitable\_jj & 0 & -8983 \\
    8 & irritability & 0 & -7902 \\
    9 & thrift & 0 & -7476 \\
    10 & opportunistic\_jj & 0 & -6830 \\
    11 & inventive\_jj & 0 & -6616 \\
    12 & leniency & 0 & -6342 \\
    13 & frivolous\_jj & 0 & -6084 \\
    14 & scornful\_jj & 0 & -5687 \\
    15 & meddlesome\_jj & 0 & -5679 \\
    16 & intrusive\_jj & 0 & -5493 \\
    17 & extravagant\_jj & 0 & -5277 \\
    18 & secretive\_jj & 0 & -5233 \\
    19 & lenient\_jj & 0 & -5030 \\
    20 & adventurous\_jj & 0 & -5009 \\
    21 & gullible\_jj & 0 & -4927 \\
    22 & daring\_jj & 0 & -4521 \\
    23 & generosity & 0 & -4479 \\
    24 & unsympathetic\_jj & 0 & -4453 \\
    25 & skeptical\_jj & 0 & -4409 \\
    26 & diplomatic\_jj & 0 & -4388 \\
    27 & generous\_jj & 0 & -4328 \\
    28 & persistence & 0 & -4237 \\
    29 & brave\_jj & 0 & -4169 \\
    30 & imaginative\_jj & 0 & -4164 \\
    392 & reckless\_jj & 0 & 4833 \\
    393 & dominant\_jj & 0 & 4928 \\
    394 & inconsistent\_jj & 0 & 4952 \\
    395 & prejudiced\_jj & 0 & 5049 \\
    396 & individualistic\_jj & 0 & 5177 \\
    397 & morality & 0 & 5321 \\
    398 & self-disciplined\_jj & 0 & 5642 \\
    399 & negligence & 0 & 5698 \\
    400 & happy-go-lucky\_jj & 0 & 5895 \\
    401 & reliable\_jj & 0 & 5947 \\
    402 & careless\_jj & 0 & 5973 \\
    403 & tempestuous\_jj & 0 & 6052 \\
    404 & volatile\_jj & 0 & 6264 \\
    405 & cruelty & 0 & 6303 \\
    406 & moral\_jj & 0 & 6335 \\
    407 & communicative\_jj & 0 & 6449 \\
    408 & rebellious\_jj & 0 & 6671 \\
    409 & surly\_jj & 0 & 6808 \\
    410 & consistent\_jj & 0 & 6868 \\
    411 & understanding\_jj & 0 & 7369 \\
    412 & self-critical\_jj & 0 & 7427 \\
    413 & concise\_jj & 0 & 7798 \\
    414 & friendly\_jj & 0 & 7885 \\
    415 & explosive\_jj & 0 & 7998 \\
    416 & erratic\_jj & 0 & 8101 \\
    417 & dependability & 0 & 8809 \\
    418 & detached\_jj & 0 & 8855 \\
    419 & selfish\_jj & 0 & 9419 \\
    420 & negligent\_jj & 1 & 163 \\
    421 & nonconforming\_jj & 1 & 717 \\
    \hline
    \caption{Scores and rankings for most extreme 30 words in component \#18} \\
\end{longtable}
\begin{longtable}[!htbp]{| rlr@{.}l |}
    \hline
    \textbf{Rank} & \textbf{Word} & \multicolumn{2}{c|}{\textbf{Score}} \\
    \hline
    \endhead
    1 & prompt\_jj & -1 & -2011 \\
    2 & silence & 0 & -9646 \\
    3 & mannerly\_jj & 0 & -8678 \\
    4 & refined\_jj & 0 & -8666 \\
    5 & unscrupulous\_jj & 0 & -8512 \\
    6 & instability & 0 & -8436 \\
    7 & concise\_jj & 0 & -8101 \\
    8 & insightful\_jj & 0 & -6991 \\
    9 & quiet & 0 & -6876 \\
    10 & perceptive\_jj & 0 & -6712 \\
    11 & insight & 0 & -6651 \\
    12 & recklessness & 0 & -6527 \\
    13 & self-pitying\_jj & 0 & -6504 \\
    14 & brave\_jj & 0 & -6265 \\
    15 & aimlessness & 0 & -6171 \\
    16 & volatility & 0 & -6163 \\
    17 & intrusiveness & 0 & -6064 \\
    18 & orderly\_jj & 0 & -5992 \\
    19 & somber\_jj & 0 & -5667 \\
    20 & dignified\_jj & 0 & -5535 \\
    21 & assured\_jj & 0 & -5485 \\
    22 & pessimism & 0 & -5424 \\
    23 & explosive\_jj & 0 & -5371 \\
    24 & obstinate\_jj & 0 & -5166 \\
    25 & courageous\_jj & 0 & -4887 \\
    26 & sluggish\_jj & 0 & -4783 \\
    27 & economical\_jj & 0 & -4731 \\
    28 & peaceful\_jj & 0 & -4571 \\
    29 & earthiness & 0 & -4565 \\
    30 & unreflective\_jj & 0 & -4506 \\
    392 & forgetfulness & 0 & 4523 \\
    393 & combative\_jj & 0 & 4570 \\
    394 & disorganization & 0 & 4584 \\
    395 & unconventional\_jj & 0 & 4632 \\
    396 & passive\_jj & 0 & 4642 \\
    397 & homespun\_jj & 0 & 4653 \\
    398 & sophistication & 0 & 4742 \\
    399 & formal\_jj & 0 & 4775 \\
    400 & cosmopolitan\_jj & 0 & 4835 \\
    401 & intrusive\_jj & 0 & 5189 \\
    402 & self-esteem & 0 & 5208 \\
    403 & defensive\_jj & 0 & 5237 \\
    404 & courtesy & 0 & 5315 \\
    405 & lenient\_jj & 0 & 5407 \\
    406 & talkative\_jj & 0 & 5425 \\
    407 & cooperative\_jj & 0 & 5565 \\
    408 & assertive\_jj & 0 & 5668 \\
    409 & touchy\_jj & 0 & 5703 \\
    410 & casual\_jj & 0 & 5950 \\
    411 & charitable\_jj & 0 & 6476 \\
    412 & lazy\_jj & 0 & 6875 \\
    413 & irritability & 0 & 6911 \\
    414 & argumentative\_jj & 0 & 7027 \\
    415 & down-to-earth\_jj & 0 & 7229 \\
    416 & traditional\_jj & 0 & 8037 \\
    417 & friendly\_jj & 0 & 8086 \\
    418 & abusive\_jj & 0 & 8495 \\
    419 & deliberate\_jj & 0 & 9353 \\
    420 & thrift & 1 & 1873 \\
    421 & folksy\_jj & 2 & 1577 \\
    \hline
    \caption{Scores and rankings for most extreme 30 words in component \#19} \\
\end{longtable}
\begin{longtable}[!htbp]{| rlr@{.}l |}
    \hline
    \textbf{Rank} & \textbf{Word} & \multicolumn{2}{c|}{\textbf{Score}} \\
    \hline
    \endhead
    1 & insight & -1 & -5294 \\
    2 & warm\_jj & -1 & -955 \\
    3 & erratic\_jj & -1 & -382 \\
    4 & cold\_jj & 0 & -7623 \\
    5 & cordial\_jj & 0 & -7427 \\
    6 & tempestuous\_jj & 0 & -7322 \\
    7 & inefficient\_jj & 0 & -7220 \\
    8 & surly\_jj & 0 & -7170 \\
    9 & charitable\_jj & 0 & -6648 \\
    10 & aimless\_jj & 0 & -6522 \\
    11 & indecisive\_jj & 0 & -6139 \\
    12 & industrious\_jj & 0 & -5906 \\
    13 & belligerence & 0 & -5818 \\
    14 & artistic\_jj & 0 & -5697 \\
    15 & courtesy & 0 & -5488 \\
    16 & insightful\_jj & 0 & -5476 \\
    17 & joyless\_jj & 0 & -5448 \\
    18 & inconsistency & 0 & -5443 \\
    19 & naïve\_jj & 0 & -5434 \\
    20 & touchy\_jj & 0 & -5267 \\
    21 & cruel\_jj & 0 & -5219 \\
    22 & inventive\_jj & 0 & -5178 \\
    23 & unkind\_jj & 0 & -5108 \\
    24 & silence & 0 & -5077 \\
    25 & unimaginative\_jj & 0 & -4983 \\
    26 & compassionate\_jj & 0 & -4938 \\
    27 & unreliable\_jj & 0 & -4924 \\
    28 & unpredictable\_jj & 0 & -4689 \\
    29 & disorganization & 0 & -4657 \\
    30 & thrift & 0 & -4651 \\
    392 & sophistication & 0 & 4336 \\
    393 & economical\_jj & 0 & 4384 \\
    394 & withdrawn\_jj & 0 & 4563 \\
    395 & skeptical\_jj & 0 & 4661 \\
    396 & reasonable\_jj & 0 & 4740 \\
    397 & prejudice & 0 & 4764 \\
    398 & nervous\_jj & 0 & 4957 \\
    399 & steady\_jj & 0 & 5008 \\
    400 & caustic\_jj & 0 & 5040 \\
    401 & pessimism & 0 & 5053 \\
    402 & gullibility & 0 & 5108 \\
    403 & fearful\_jj & 0 & 5127 \\
    404 & optimistic\_jj & 0 & 5129 \\
    405 & distrustful\_jj & 0 & 5136 \\
    406 & sophisticated\_jj & 0 & 5212 \\
    407 & forceful\_jj & 0 & 5628 \\
    408 & deliberate\_jj & 0 & 5676 \\
    409 & cautious\_jj & 0 & 5813 \\
    410 & pessimistic\_jj & 0 & 5873 \\
    411 & condescending\_jj & 0 & 5956 \\
    412 & conventional\_jj & 0 & 6331 \\
    413 & inhibition & 0 & 6334 \\
    414 & cultured\_jj & 0 & 6490 \\
    415 & explosive\_jj & 0 & 6640 \\
    416 & earthiness & 0 & 6875 \\
    417 & nonconforming\_jj & 0 & 8176 \\
    418 & folksy\_jj & 0 & 9453 \\
    419 & suspicious\_jj & 1 & 2504 \\
    420 & prompt\_jj & 1 & 3351 \\
    421 & refined\_jj & 1 & 7023 \\
    \hline
    \caption{Scores and rankings for most extreme 30 words in component \#20} \\
\end{longtable}
\begin{longtable}[!htbp]{| rlr@{.}l |}
    \hline
    \textbf{Rank} & \textbf{Word} & \multicolumn{2}{c|}{\textbf{Score}} \\
    \hline
    \endhead
    1 & refined\_jj & -1 & -3869 \\
    2 & warm\_jj & -1 & -3602 \\
    3 & cold\_jj & -1 & -1875 \\
    4 & cruelty & -1 & -1055 \\
    5 & insight & -1 & -929 \\
    6 & perceptive\_jj & -1 & -590 \\
    7 & lenient\_jj & -1 & -352 \\
    8 & harsh\_jj & 0 & -8600 \\
    9 & submissive\_jj & 0 & -8026 \\
    10 & leniency & 0 & -7736 \\
    11 & sloth & 0 & -7324 \\
    12 & ruthless\_jj & 0 & -6845 \\
    13 & docile\_jj & 0 & -6776 \\
    14 & cunning & 0 & -6278 \\
    15 & expressive\_jj & 0 & -5850 \\
    16 & irritable\_jj & 0 & -5709 \\
    17 & cruel\_jj & 0 & -5588 \\
    18 & crafty\_jj & 0 & -5481 \\
    19 & callousness & 0 & -5303 \\
    20 & analytical\_jj & 0 & -5267 \\
    21 & belligerence & 0 & -5031 \\
    22 & adaptable\_jj & 0 & -4782 \\
    23 & verbal\_jj & 0 & -4764 \\
    24 & intelligence & 0 & -4731 \\
    25 & jealous\_jj & 0 & -4708 \\
    26 & sophisticated\_jj & 0 & -4697 \\
    27 & lazy\_jj & 0 & -4630 \\
    28 & sophistication & 0 & -4615 \\
    29 & negligent\_jj & 0 & -4564 \\
    30 & inquisitive\_jj & 0 & -4468 \\
    392 & extravagant\_jj & 0 & 4572 \\
    393 & volatile\_jj & 0 & 4613 \\
    394 & unassuming\_jj & 0 & 4632 \\
    395 & independence & 0 & 4693 \\
    396 & surliness & 0 & 4939 \\
    397 & optimism & 0 & 5060 \\
    398 & instability & 0 & 5081 \\
    399 & rash\_jj & 0 & 5114 \\
    400 & peaceful\_jj & 0 & 5191 \\
    401 & autonomous\_jj & 0 & 5228 \\
    402 & absent-minded\_jj & 0 & 5250 \\
    403 & disorganization & 0 & 5262 \\
    404 & homespun\_jj & 0 & 5297 \\
    405 & unrestrained\_jj & 0 & 5307 \\
    406 & aimless\_jj & 0 & 5351 \\
    407 & humorous\_jj & 0 & 5394 \\
    408 & economical\_jj & 0 & 5421 \\
    409 & concise\_jj & 0 & 5475 \\
    410 & foolhardy\_jj & 0 & 5497 \\
    411 & pessimism & 0 & 5541 \\
    412 & frivolity & 0 & 5769 \\
    413 & impudent\_jj & 0 & 5831 \\
    414 & haphazard\_jj & 0 & 6227 \\
    415 & spontaneous\_jj & 0 & 6423 \\
    416 & underhanded\_jj & 0 & 6829 \\
    417 & abusive\_jj & 0 & 8499 \\
    418 & frivolous\_jj & 0 & 8788 \\
    419 & volatility & 0 & 9012 \\
    420 & orderly\_jj & 1 & 201 \\
    421 & explosive\_jj & 1 & 798 \\
    \hline
    \caption{Scores and rankings for most extreme 30 words in component \#21} \\
\end{longtable}
\begin{longtable}[!htbp]{| rlr@{.}l |}
    \hline
    \textbf{Rank} & \textbf{Word} & \multicolumn{2}{c|}{\textbf{Score}} \\
    \hline
    \endhead
    1 & refined\_jj & -1 & -2445 \\
    2 & compassionate\_jj & -1 & -1379 \\
    3 & charitable\_jj & -1 & -1150 \\
    4 & earthiness & -1 & -924 \\
    5 & earthy\_jj & 0 & -8529 \\
    6 & concise\_jj & 0 & -8526 \\
    7 & lethargy & 0 & -7720 \\
    8 & lenient\_jj & 0 & -7673 \\
    9 & unscrupulous\_jj & 0 & -7530 \\
    10 & independent\_jj & 0 & -7177 \\
    11 & moral\_jj & 0 & -6888 \\
    12 & irritable\_jj & 0 & -6881 \\
    13 & negligence & 0 & -6513 \\
    14 & principled\_jj & 0 & -6434 \\
    15 & crabby\_jj & 0 & -6200 \\
    16 & stingy\_jj & 0 & -5887 \\
    17 & inconsistent\_jj & 0 & -5594 \\
    18 & philosophical\_jj & 0 & -5591 \\
    19 & aloofness & 0 & -5517 \\
    20 & cultured\_jj & 0 & -5305 \\
    21 & ethical\_jj & 0 & -5140 \\
    22 & organization & 0 & -5092 \\
    23 & obstinate\_jj & 0 & -5006 \\
    24 & caustic\_jj & 0 & -4990 \\
    25 & warm\_jj & 0 & -4966 \\
    26 & indecisive\_jj & 0 & -4943 \\
    27 & bitter\_jj & 0 & -4909 \\
    28 & deep\_jj & 0 & -4864 \\
    29 & defensive\_jj & 0 & -4734 \\
    30 & assertion & 0 & -4641 \\
    392 & uninhibited\_jj & 0 & 4217 \\
    393 & careful\_jj & 0 & 4285 \\
    394 & playfulness & 0 & 4297 \\
    395 & curious\_jj & 0 & 4310 \\
    396 & unpredictable\_jj & 0 & 4321 \\
    397 & brave\_jj & 0 & 4409 \\
    398 & cordial\_jj & 0 & 4488 \\
    399 & smart\_jj & 0 & 4500 \\
    400 & sophistication & 0 & 4551 \\
    401 & curiosity & 0 & 4729 \\
    402 & inconsiderate\_jj & 0 & 4737 \\
    403 & instability & 0 & 4808 \\
    404 & devious\_jj & 0 & 4830 \\
    405 & obliging\_jj & 0 & 4904 \\
    406 & dependability & 0 & 5146 \\
    407 & friendly\_jj & 0 & 5371 \\
    408 & flexibility & 0 & 5449 \\
    409 & intrusiveness & 0 & 5455 \\
    410 & sloth & 0 & 5458 \\
    411 & animation & 0 & 5492 \\
    412 & cunning\_jj & 0 & 5723 \\
    413 & understanding\_jj & 0 & 5819 \\
    414 & efficient\_jj & 0 & 5884 \\
    415 & silence & 0 & 6021 \\
    416 & docile\_jj & 0 & 6075 \\
    417 & rambunctious\_jj & 0 & 6725 \\
    418 & naturalness & 0 & 7739 \\
    419 & rude\_jj & 0 & 8548 \\
    420 & adaptable\_jj & 0 & 9527 \\
    421 & belligerence & 1 & 6252 \\
    \hline
    \caption{Scores and rankings for most extreme 30 words in component \#22} \\
\end{longtable}
\begin{longtable}[!htbp]{| rlr@{.}l |}
    \hline
    \textbf{Rank} & \textbf{Word} & \multicolumn{2}{c|}{\textbf{Score}} \\
    \hline
    \endhead
    1 & negligent\_jj & -1 & -2859 \\
    2 & unemotional\_jj & -1 & -1400 \\
    3 & orderly\_jj & 0 & -9278 \\
    4 & folksy\_jj & 0 & -8753 \\
    5 & inefficient\_jj & 0 & -7832 \\
    6 & unreliable\_jj & 0 & -7744 \\
    7 & distrustful\_jj & 0 & -7668 \\
    8 & disorganization & 0 & -7271 \\
    9 & expressiveness & 0 & -7077 \\
    10 & deliberate\_jj & 0 & -6919 \\
    11 & independence & 0 & -6868 \\
    12 & earthiness & 0 & -6779 \\
    13 & haphazard\_jj & 0 & -6446 \\
    14 & reliable\_jj & 0 & -6372 \\
    15 & melancholic\_jj & 0 & -6332 \\
    16 & insecure\_jj & 0 & -6239 \\
    17 & dependability & 0 & -5955 \\
    18 & autonomous\_jj & 0 & -5882 \\
    19 & independent\_jj & 0 & -5686 \\
    20 & analytical\_jj & 0 & -5662 \\
    21 & surly\_jj & 0 & -5655 \\
    22 & emotionality & 0 & -5628 \\
    23 & peaceful\_jj & 0 & -5459 \\
    24 & forgetful\_jj & 0 & -5159 \\
    25 & meditative\_jj & 0 & -4980 \\
    26 & precise\_jj & 0 & -4943 \\
    27 & underhanded\_jj & 0 & -4920 \\
    28 & bullheaded\_jj & 0 & -4696 \\
    29 & dignity & 0 & -4594 \\
    30 & systematic\_jj & 0 & -4564 \\
    392 & individualistic\_jj & 0 & 4180 \\
    393 & nonconformity & 0 & 4190 \\
    394 & animation & 0 & 4264 \\
    395 & careless\_jj & 0 & 4293 \\
    396 & selfish\_jj & 0 & 4520 \\
    397 & erratic\_jj & 0 & 4670 \\
    398 & conceited\_jj & 0 & 4799 \\
    399 & sloth & 0 & 4901 \\
    400 & mannerly\_jj & 0 & 4966 \\
    401 & egocentric\_jj & 0 & 5015 \\
    402 & antagonistic\_jj & 0 & 5026 \\
    403 & wordy\_jj & 0 & 5074 \\
    404 & suspicious\_jj & 0 & 5189 \\
    405 & volatility & 0 & 5372 \\
    406 & inventive\_jj & 0 & 5390 \\
    407 & opportunistic\_jj & 0 & 5508 \\
    408 & philosophical\_jj & 0 & 5640 \\
    409 & verbal\_jj & 0 & 5698 \\
    410 & thrifty\_jj & 0 & 5809 \\
    411 & pessimism & 0 & 5991 \\
    412 & spirited\_jj & 0 & 6333 \\
    413 & crafty\_jj & 0 & 6335 \\
    414 & miserly\_jj & 0 & 6751 \\
    415 & cordial\_jj & 0 & 7002 \\
    416 & tempestuous\_jj & 0 & 7054 \\
    417 & concise\_jj & 0 & 7342 \\
    418 & touchy\_jj & 0 & 9264 \\
    419 & unscrupulous\_jj & 0 & 9512 \\
    420 & friendly\_jj & 0 & 9632 \\
    421 & explosive\_jj & 1 & 531 \\
    \hline
    \caption{Scores and rankings for most extreme 30 words in component \#23} \\
\end{longtable}
\begin{longtable}[!htbp]{| rlr@{.}l |}
    \hline
    \textbf{Rank} & \textbf{Word} & \multicolumn{2}{c|}{\textbf{Score}} \\
    \hline
    \endhead
    1 & belligerence & -1 & -8741 \\
    2 & negligent\_jj & -1 & -7376 \\
    3 & concise\_jj & -1 & -3103 \\
    4 & independence & -1 & -2453 \\
    5 & thrift & -1 & -273 \\
    6 & refined\_jj & 0 & -9835 \\
    7 & folksy\_jj & 0 & -9021 \\
    8 & diplomatic\_jj & 0 & -7892 \\
    9 & instability & 0 & -7645 \\
    10 & mannerly\_jj & 0 & -7393 \\
    11 & dependability & 0 & -7288 \\
    12 & wishy-washy\_jj & 0 & -7138 \\
    13 & volatility & 0 & -6262 \\
    14 & impolite\_jj & 0 & -6256 \\
    15 & foolhardy\_jj & 0 & -6215 \\
    16 & sociable\_jj & 0 & -5723 \\
    17 & cosmopolitan\_jj & 0 & -5532 \\
    18 & indecisive\_jj & 0 & -5306 \\
    19 & punctuality & 0 & -5238 \\
    20 & careless\_jj & 0 & -5217 \\
    21 & cultured\_jj & 0 & -4869 \\
    22 & adventurous\_jj & 0 & -4803 \\
    23 & homespun\_jj & 0 & -4758 \\
    24 & deliberate\_jj & 0 & -4598 \\
    25 & touchy\_jj & 0 & -4473 \\
    26 & naïve\_jj & 0 & -4384 \\
    27 & frivolity & 0 & -4333 \\
    28 & thrifty\_jj & 0 & -4245 \\
    29 & negligence & 0 & -4209 \\
    30 & assertive\_jj & 0 & -4202 \\
    392 & deep\_jj & 0 & 3718 \\
    393 & unrestrained\_jj & 0 & 3750 \\
    394 & sincere\_jj & 0 & 3819 \\
    395 & naturalness & 0 & 3820 \\
    396 & silent\_jj & 0 & 3833 \\
    397 & honest\_jj & 0 & 3880 \\
    398 & passionless\_jj & 0 & 4010 \\
    399 & tenacious\_jj & 0 & 4030 \\
    400 & egocentric\_jj & 0 & 4138 \\
    401 & devious\_jj & 0 & 4289 \\
    402 & generosity & 0 & 4316 \\
    403 & intelligent\_jj & 0 & 4343 \\
    404 & distrust & 0 & 4411 \\
    405 & self-pitying\_jj & 0 & 4493 \\
    406 & empathy & 0 & 4572 \\
    407 & spirit & 0 & 4587 \\
    408 & friendly\_jj & 0 & 4588 \\
    409 & inconsistent\_jj & 0 & 4597 \\
    410 & greedy\_jj & 0 & 4677 \\
    411 & cordial\_jj & 0 & 4928 \\
    412 & orderly\_jj & 0 & 5016 \\
    413 & unscrupulous\_jj & 0 & 5355 \\
    414 & efficient\_jj & 0 & 5373 \\
    415 & thorough\_jj & 0 & 5576 \\
    416 & inefficient\_jj & 0 & 5638 \\
    417 & intrusive\_jj & 0 & 5923 \\
    418 & selfish\_jj & 0 & 6045 \\
    419 & erratic\_jj & 0 & 6350 \\
    420 & prompt\_jj & 0 & 9996 \\
    421 & surly\_jj & 1 & 6780 \\
    \hline
    \caption{Scores and rankings for most extreme 30 words in component \#24} \\
\end{longtable}
\begin{longtable}[!htbp]{| rlr@{.}l |}
    \hline
    \textbf{Rank} & \textbf{Word} & \multicolumn{2}{c|}{\textbf{Score}} \\
    \hline
    \endhead
    1 & refined\_jj & -1 & -3689 \\
    2 & charitable\_jj & -1 & -3450 \\
    3 & insight & -1 & -684 \\
    4 & belligerence & -1 & -531 \\
    5 & sluggish\_jj & 0 & -8219 \\
    6 & pessimism & 0 & -8012 \\
    7 & perceptive\_jj & 0 & -7379 \\
    8 & surly\_jj & 0 & -6471 \\
    9 & selfish\_jj & 0 & -6194 \\
    10 & economical\_jj & 0 & -6182 \\
    11 & organization & 0 & -6153 \\
    12 & volatility & 0 & -6082 \\
    13 & self-esteem & 0 & -5813 \\
    14 & independent\_jj & 0 & -5593 \\
    15 & reserve & 0 & -5524 \\
    16 & witty\_jj & 0 & -5012 \\
    17 & steady\_jj & 0 & -4995 \\
    18 & gullible\_jj & 0 & -4883 \\
    19 & inhibition & 0 & -4784 \\
    20 & thrift & 0 & -4768 \\
    21 & antagonistic\_jj & 0 & -4751 \\
    22 & meditative\_jj & 0 & -4594 \\
    23 & ungracious\_jj & 0 & -4558 \\
    24 & defensive\_jj & 0 & -4489 \\
    25 & dominant\_jj & 0 & -4412 \\
    26 & spirited\_jj & 0 & -4390 \\
    27 & manipulative\_jj & 0 & -4344 \\
    28 & silence & 0 & -4244 \\
    29 & verbose\_jj & 0 & -4215 \\
    30 & prompt\_jj & 0 & -4172 \\
    392 & inefficient\_jj & 0 & 3875 \\
    393 & unreliable\_jj & 0 & 3877 \\
    394 & unpredictable\_jj & 0 & 3917 \\
    395 & courteous\_jj & 0 & 3935 \\
    396 & distrust & 0 & 4013 \\
    397 & expressiveness & 0 & 4135 \\
    398 & courage & 0 & 4254 \\
    399 & impersonal\_jj & 0 & 4352 \\
    400 & unstable\_jj & 0 & 4373 \\
    401 & modesty & 0 & 4418 \\
    402 & playfulness & 0 & 4446 \\
    403 & accommodating\_jj & 0 & 4446 \\
    404 & organized\_jj & 0 & 4475 \\
    405 & pleasant\_jj & 0 & 4659 \\
    406 & decisiveness & 0 & 4664 \\
    407 & impractical\_jj & 0 & 4876 \\
    408 & intrusive\_jj & 0 & 4931 \\
    409 & punctuality & 0 & 5062 \\
    410 & sophistication & 0 & 5307 \\
    411 & individualistic\_jj & 0 & 5598 \\
    412 & leniency & 0 & 5669 \\
    413 & harsh\_jj & 0 & 6091 \\
    414 & punctual\_jj & 0 & 6331 \\
    415 & cold\_jj & 0 & 6561 \\
    416 & warmth & 0 & 6755 \\
    417 & warm\_jj & 0 & 8760 \\
    418 & suspicious\_jj & 0 & 9930 \\
    419 & lenient\_jj & 1 & 162 \\
    420 & concise\_jj & 1 & 3896 \\
    421 & explosive\_jj & 1 & 6910 \\
    \hline
    \caption{Scores and rankings for most extreme 30 words in component \#25} \\
\end{longtable}
\begin{longtable}[!htbp]{| rlr@{.}l |}
    \hline
    \textbf{Rank} & \textbf{Word} & \multicolumn{2}{c|}{\textbf{Score}} \\
    \hline
    \endhead
    1 & surly\_jj & -1 & -2692 \\
    2 & brave\_jj & 0 & -8967 \\
    3 & thorough\_jj & 0 & -7873 \\
    4 & thrift & 0 & -7209 \\
    5 & sophistication & 0 & -6882 \\
    6 & intelligence & 0 & -6811 \\
    7 & explosive\_jj & 0 & -5984 \\
    8 & enterprising\_jj & 0 & -5634 \\
    9 & scornful\_jj & 0 & -5588 \\
    10 & rudeness & 0 & -5433 \\
    11 & insight & 0 & -5238 \\
    12 & cruelty & 0 & -5209 \\
    13 & sociable\_jj & 0 & -5151 \\
    14 & impudent\_jj & 0 & -5064 \\
    15 & vigorous\_jj & 0 & -4952 \\
    16 & economical\_jj & 0 & -4901 \\
    17 & silence & 0 & -4686 \\
    18 & pessimism & 0 & -4636 \\
    19 & caustic\_jj & 0 & -4512 \\
    20 & lethargy & 0 & -4458 \\
    21 & independent\_jj & 0 & -4442 \\
    22 & foolhardy\_jj & 0 & -4404 \\
    23 & cultured\_jj & 0 & -4398 \\
    24 & daring & 0 & -4387 \\
    25 & industrious\_jj & 0 & -4267 \\
    26 & tenacious\_jj & 0 & -4127 \\
    27 & talkative\_jj & 0 & -4122 \\
    28 & insensitive\_jj & 0 & -4056 \\
    29 & cruel\_jj & 0 & -3990 \\
    30 & unimaginative\_jj & 0 & -3974 \\
    392 & distrustful\_jj & 0 & 4039 \\
    393 & stinginess & 0 & 4056 \\
    394 & warm\_jj & 0 & 4084 \\
    395 & unscrupulous\_jj & 0 & 4119 \\
    396 & extravagant\_jj & 0 & 4232 \\
    397 & instability & 0 & 4250 \\
    398 & careless\_jj & 0 & 4317 \\
    399 & artistic\_jj & 0 & 4404 \\
    400 & aloofness & 0 & 4429 \\
    401 & generous\_jj & 0 & 4845 \\
    402 & insecure\_jj & 0 & 4872 \\
    403 & boastful\_jj & 0 & 4921 \\
    404 & egotistical\_jj & 0 & 5052 \\
    405 & manipulative\_jj & 0 & 5056 \\
    406 & placidity & 0 & 5151 \\
    407 & earthiness & 0 & 5449 \\
    408 & greedy\_jj & 0 & 5767 \\
    409 & generosity & 0 & 6146 \\
    410 & expressiveness & 0 & 6223 \\
    411 & reckless\_jj & 0 & 6372 \\
    412 & inconsiderate\_jj & 0 & 6740 \\
    413 & volatile\_jj & 0 & 7493 \\
    414 & expressive\_jj & 0 & 8101 \\
    415 & naturalness & 0 & 9899 \\
    416 & prompt\_jj & 1 & 392 \\
    417 & volatility & 1 & 480 \\
    418 & selfish\_jj & 1 & 791 \\
    419 & erratic\_jj & 1 & 1563 \\
    420 & concise\_jj & 1 & 1575 \\
    421 & charitable\_jj & 2 & 310 \\
    \hline
    \caption{Scores and rankings for most extreme 30 words in component \#26} \\
\end{longtable}
\begin{longtable}[!htbp]{| rlr@{.}l |}
    \hline
    \textbf{Rank} & \textbf{Word} & \multicolumn{2}{c|}{\textbf{Score}} \\
    \hline
    \endhead
    1 & concise\_jj & -1 & -2307 \\
    2 & assured\_jj & 0 & -9998 \\
    3 & belligerence & 0 & -9904 \\
    4 & expressive\_jj & 0 & -8100 \\
    5 & proud\_jj & 0 & -7879 \\
    6 & charitable\_jj & 0 & -7149 \\
    7 & irritability & 0 & -7109 \\
    8 & envy & 0 & -7086 \\
    9 & suspicious\_jj & 0 & -6983 \\
    10 & refined\_jj & 0 & -6521 \\
    11 & envious\_jj & 0 & -6289 \\
    12 & folksy\_jj & 0 & -6229 \\
    13 & recklessness & 0 & -6172 \\
    14 & stupidity & 0 & -5989 \\
    15 & egotistical\_jj & 0 & -5799 \\
    16 & silence & 0 & -5747 \\
    17 & negligence & 0 & -5650 \\
    18 & optimistic\_jj & 0 & -5512 \\
    19 & benevolent\_jj & 0 & -5458 \\
    20 & egocentric\_jj & 0 & -5391 \\
    21 & surly\_jj & 0 & -5240 \\
    22 & economical\_jj & 0 & -5164 \\
    23 & autonomous\_jj & 0 & -5004 \\
    24 & confident\_jj & 0 & -4819 \\
    25 & enthusiastic\_jj & 0 & -4670 \\
    26 & punctual\_jj & 0 & -4579 \\
    27 & lethargy & 0 & -4433 \\
    28 & organization & 0 & -4395 \\
    29 & uncritical\_jj & 0 & -4167 \\
    30 & condescending\_jj & 0 & -4144 \\
    392 & intrusive\_jj & 0 & 4202 \\
    393 & philosophical\_jj & 0 & 4214 \\
    394 & foolhardy\_jj & 0 & 4268 \\
    395 & submissive\_jj & 0 & 4443 \\
    396 & thrifty\_jj & 0 & 4540 \\
    397 & leniency & 0 & 4547 \\
    398 & insecure\_jj & 0 & 4561 \\
    399 & zestful\_jj & 0 & 4663 \\
    400 & sociable\_jj & 0 & 4749 \\
    401 & wishy-washy\_jj & 0 & 4838 \\
    402 & morose\_jj & 0 & 4885 \\
    403 & unemotional\_jj & 0 & 4885 \\
    404 & demanding\_jj & 0 & 4942 \\
    405 & vivacious\_jj & 0 & 5075 \\
    406 & reserve & 0 & 5112 \\
    407 & complex\_jj & 0 & 5249 \\
    408 & flexibility & 0 & 5384 \\
    409 & withdrawn\_jj & 0 & 5637 \\
    410 & unstable\_jj & 0 & 5745 \\
    411 & self-esteem & 0 & 6041 \\
    412 & tactful\_jj & 0 & 6457 \\
    413 & manipulative\_jj & 0 & 6511 \\
    414 & volatility & 0 & 6718 \\
    415 & shyness & 0 & 6843 \\
    416 & independence & 0 & 7226 \\
    417 & compassionate\_jj & 0 & 8150 \\
    418 & touchy\_jj & 0 & 9173 \\
    419 & nonconforming\_jj & 0 & 9904 \\
    420 & insight & 1 & 1509 \\
    421 & prompt\_jj & 1 & 2073 \\
    \hline
    \caption{Scores and rankings for most extreme 30 words in component \#27} \\
\end{longtable}
\begin{longtable}[!htbp]{| rlr@{.}l |}
    \hline
    \textbf{Rank} & \textbf{Word} & \multicolumn{2}{c|}{\textbf{Score}} \\
    \hline
    \endhead
    1 & perceptive\_jj & 0 & -8801 \\
    2 & reliable\_jj & 0 & -8301 \\
    3 & negligent\_jj & 0 & -8257 \\
    4 & earthiness & 0 & -7524 \\
    5 & unscrupulous\_jj & 0 & -6543 \\
    6 & negligence & 0 & -6435 \\
    7 & tenacious\_jj & 0 & -6145 \\
    8 & independence & 0 & -6047 \\
    9 & silence & 0 & -6031 \\
    10 & passivity & 0 & -5752 \\
    11 & bitter\_jj & 0 & -5568 \\
    12 & touchy\_jj & 0 & -5478 \\
    13 & rebellious\_jj & 0 & -5389 \\
    14 & prejudice & 0 & -5309 \\
    15 & flippant\_jj & 0 & -5211 \\
    16 & conceited\_jj & 0 & -5202 \\
    17 & dominant\_jj & 0 & -5080 \\
    18 & patient\_jj & 0 & -4990 \\
    19 & surly\_jj & 0 & -4929 \\
    20 & dependable\_jj & 0 & -4832 \\
    21 & decisive\_jj & 0 & -4678 \\
    22 & impetuous\_jj & 0 & -4451 \\
    23 & miserly\_jj & 0 & -4440 \\
    24 & wishy-washy\_jj & 0 & -4416 \\
    25 & bossiness & 0 & -4401 \\
    26 & efficient\_jj & 0 & -4286 \\
    27 & organized\_jj & 0 & -4244 \\
    28 & egocentric\_jj & 0 & -4213 \\
    29 & surliness & 0 & -4206 \\
    30 & expressive\_jj & 0 & -4161 \\
    392 & aimlessness & 0 & 4037 \\
    393 & reckless\_jj & 0 & 4089 \\
    394 & inefficient\_jj & 0 & 4387 \\
    395 & thorough\_jj & 0 & 4451 \\
    396 & cranky\_jj & 0 & 4476 \\
    397 & impolite\_jj & 0 & 4515 \\
    398 & wordy\_jj & 0 & 4555 \\
    399 & lethargic\_jj & 0 & 4645 \\
    400 & haphazard\_jj & 0 & 4785 \\
    401 & unintelligent\_jj & 0 & 4813 \\
    402 & recklessness & 0 & 4835 \\
    403 & nonconforming\_jj & 0 & 4921 \\
    404 & quarrelsome\_jj & 0 & 4923 \\
    405 & intelligence & 0 & 5668 \\
    406 & morose\_jj & 0 & 5781 \\
    407 & sloth & 0 & 5840 \\
    408 & insight & 0 & 5858 \\
    409 & thoughtless\_jj & 0 & 6003 \\
    410 & orderly\_jj & 0 & 6243 \\
    411 & happy-go-lucky\_jj & 0 & 6331 \\
    412 & impersonal\_jj & 0 & 6331 \\
    413 & selfish\_jj & 0 & 6403 \\
    414 & shallow\_jj & 0 & 7076 \\
    415 & reserve & 0 & 7543 \\
    416 & aimless\_jj & 0 & 7810 \\
    417 & cordial\_jj & 0 & 8045 \\
    418 & explosive\_jj & 0 & 9649 \\
    419 & refined\_jj & 1 & 831 \\
    420 & suspicious\_jj & 1 & 7207 \\
    421 & thrift & 1 & 9363 \\
    \hline
    \caption{Scores and rankings for most extreme 30 words in component \#28} \\
\end{longtable}
\begin{longtable}[!htbp]{| rlr@{.}l |}
    \hline
    \textbf{Rank} & \textbf{Word} & \multicolumn{2}{c|}{\textbf{Score}} \\
    \hline
    \endhead
    1 & insight & -1 & -177 \\
    2 & concise\_jj & 0 & -8411 \\
    3 & withdrawn\_jj & 0 & -7854 \\
    4 & happy-go-lucky\_jj & 0 & -7108 \\
    5 & rash\_jj & 0 & -6808 \\
    6 & impudent\_jj & 0 & -6353 \\
    7 & leniency & 0 & -6158 \\
    8 & impetuous\_jj & 0 & -6114 \\
    9 & lenient\_jj & 0 & -6091 \\
    10 & curt\_jj & 0 & -5974 \\
    11 & logic & 0 & -5413 \\
    12 & naturalness & 0 & -5350 \\
    13 & decisiveness & 0 & -5310 \\
    14 & antagonistic\_jj & 0 & -5287 \\
    15 & docile\_jj & 0 & -5044 \\
    16 & assured\_jj & 0 & -4975 \\
    17 & warm\_jj & 0 & -4901 \\
    18 & quarrelsome\_jj & 0 & -4807 \\
    19 & prompt\_jj & 0 & -4688 \\
    20 & obliging\_jj & 0 & -4652 \\
    21 & frivolity & 0 & -4611 \\
    22 & aimless\_jj & 0 & -4526 \\
    23 & unemotional\_jj & 0 & -4511 \\
    24 & conventional\_jj & 0 & -4459 \\
    25 & logical\_jj & 0 & -4442 \\
    26 & unfriendly\_jj & 0 & -4344 \\
    27 & flexibility & 0 & -4109 \\
    28 & optimism & 0 & -4103 \\
    29 & friendly\_jj & 0 & -4076 \\
    30 & spirit & 0 & -4059 \\
    392 & dignified\_jj & 0 & 4387 \\
    393 & thrift & 0 & 4452 \\
    394 & impolite\_jj & 0 & 4499 \\
    395 & dignity & 0 & 4854 \\
    396 & quiet\_jj & 0 & 5005 \\
    397 & fastidious\_jj & 0 & 5169 \\
    398 & self-pitying\_jj & 0 & 5177 \\
    399 & refined\_jj & 0 & 5290 \\
    400 & self-esteem & 0 & 5333 \\
    401 & insecure\_jj & 0 & 5376 \\
    402 & shy\_jj & 0 & 5439 \\
    403 & respectful\_jj & 0 & 5542 \\
    404 & suspicious\_jj & 0 & 5621 \\
    405 & sluggish\_jj & 0 & 5723 \\
    406 & proud\_jj & 0 & 5790 \\
    407 & exacting\_jj & 0 & 6162 \\
    408 & charitable\_jj & 0 & 6269 \\
    409 & verbose\_jj & 0 & 6427 \\
    410 & touchy\_jj & 0 & 6729 \\
    411 & mannerly\_jj & 0 & 6762 \\
    412 & secretive\_jj & 0 & 6798 \\
    413 & demanding\_jj & 0 & 7074 \\
    414 & meticulous\_jj & 0 & 7150 \\
    415 & explosive\_jj & 0 & 7332 \\
    416 & worldly\_jj & 0 & 7361 \\
    417 & thorough\_jj & 0 & 8277 \\
    418 & punctual\_jj & 0 & 8870 \\
    419 & punctuality & 0 & 9938 \\
    420 & sophistication & 1 & 946 \\
    421 & surly\_jj & 1 & 2500 \\
    \hline
    \caption{Scores and rankings for most extreme 30 words in component \#29} \\
\end{longtable}
\begin{longtable}[!htbp]{| rlr@{.}l |}
    \hline
    \textbf{Rank} & \textbf{Word} & \multicolumn{2}{c|}{\textbf{Score}} \\
    \hline
    \endhead
    1 & insight & -1 & -3162 \\
    2 & independence & -1 & -1938 \\
    3 & prompt\_jj & -1 & -1249 \\
    4 & concise\_jj & -1 & -758 \\
    5 & autonomous\_jj & -1 & -417 \\
    6 & rude\_jj & 0 & -9342 \\
    7 & economical\_jj & 0 & -7717 \\
    8 & rambunctious\_jj & 0 & -7127 \\
    9 & rebellious\_jj & 0 & -6797 \\
    10 & scornful\_jj & 0 & -6768 \\
    11 & withdrawn\_jj & 0 & -6432 \\
    12 & thrift & 0 & -6374 \\
    13 & dependability & 0 & -6285 \\
    14 & caustic\_jj & 0 & -5609 \\
    15 & excitable\_jj & 0 & -5491 \\
    16 & cosmopolitan\_jj & 0 & -5360 \\
    17 & reserve & 0 & -5296 \\
    18 & humor & 0 & -5200 \\
    19 & happy-go-lucky\_jj & 0 & -5148 \\
    20 & pomposity & 0 & -5111 \\
    21 & refined\_jj & 0 & -4946 \\
    22 & bigoted\_jj & 0 & -4929 \\
    23 & rudeness & 0 & -4833 \\
    24 & docile\_jj & 0 & -4823 \\
    25 & gruff\_jj & 0 & -4590 \\
    26 & erratic\_jj & 0 & -4568 \\
    27 & unemotional\_jj & 0 & -4556 \\
    28 & analytical\_jj & 0 & -4324 \\
    29 & crabby\_jj & 0 & -4245 \\
    30 & wordy\_jj & 0 & -4100 \\
    392 & contemplative\_jj & 0 & 4083 \\
    393 & pleasant\_jj & 0 & 4101 \\
    394 & self-critical\_jj & 0 & 4319 \\
    395 & unintelligent\_jj & 0 & 4343 \\
    396 & diplomatic\_jj & 0 & 4441 \\
    397 & kind\_jj & 0 & 4446 \\
    398 & shallow\_jj & 0 & 4467 \\
    399 & melancholic\_jj & 0 & 4513 \\
    400 & unambitious\_jj & 0 & 4519 \\
    401 & pessimism & 0 & 4537 \\
    402 & trustful\_jj & 0 & 4619 \\
    403 & suggestible\_jj & 0 & 4626 \\
    404 & patient\_jj & 0 & 4647 \\
    405 & rash\_jj & 0 & 4906 \\
    406 & placidity & 0 & 5040 \\
    407 & explosive\_jj & 0 & 5060 \\
    408 & negligence & 0 & 5069 \\
    409 & inventive\_jj & 0 & 5079 \\
    410 & conceited\_jj & 0 & 5112 \\
    411 & exacting\_jj & 0 & 5123 \\
    412 & naturalness & 0 & 5246 \\
    413 & naïve\_jj & 0 & 5440 \\
    414 & warm\_jj & 0 & 5599 \\
    415 & unimaginative\_jj & 0 & 5720 \\
    416 & deliberate\_jj & 0 & 6113 \\
    417 & nonconforming\_jj & 0 & 6158 \\
    418 & foolhardy\_jj & 0 & 6268 \\
    419 & earthiness & 0 & 6370 \\
    420 & compassionate\_jj & 0 & 7760 \\
    421 & belligerence & 1 & 322 \\
    \hline
    \caption{Scores and rankings for most extreme 30 words in component \#30} \\
\end{longtable}
\begin{longtable}[!htbp]{| rlr@{.}l |}
    \hline
    \textbf{Rank} & \textbf{Word} & \multicolumn{2}{c|}{\textbf{Score}} \\
    \hline
    \endhead
    1 & economical\_jj & -1 & -288 \\
    2 & reserve & 0 & -9969 \\
    3 & stingy\_jj & 0 & -9830 \\
    4 & scornful\_jj & 0 & -8713 \\
    5 & expressive\_jj & 0 & -7572 \\
    6 & earthiness & 0 & -7465 \\
    7 & expressiveness & 0 & -6351 \\
    8 & undemanding\_jj & 0 & -6092 \\
    9 & defensive\_jj & 0 & -6072 \\
    10 & caustic\_jj & 0 & -6005 \\
    11 & recklessness & 0 & -5687 \\
    12 & intrusiveness & 0 & -5652 \\
    13 & kind\_jj & 0 & -5462 \\
    14 & sociable\_jj & 0 & -5422 \\
    15 & disorganization & 0 & -5326 \\
    16 & touchy\_jj & 0 & -5316 \\
    17 & depth & 0 & -5162 \\
    18 & explosive\_jj & 0 & -4963 \\
    19 & careless\_jj & 0 & -4955 \\
    20 & natural\_jj & 0 & -4925 \\
    21 & organized\_jj & 0 & -4883 \\
    22 & warm\_jj & 0 & -4702 \\
    23 & deep\_jj & 0 & -4411 \\
    24 & impolite\_jj & 0 & -4301 \\
    25 & inquisitive\_jj & 0 & -4253 \\
    26 & rambunctious\_jj & 0 & -4240 \\
    27 & sloth & 0 & -4212 \\
    28 & spontaneous\_jj & 0 & -4195 \\
    29 & opportunistic\_jj & 0 & -4072 \\
    30 & assertive\_jj & 0 & -3801 \\
    392 & erratic\_jj & 0 & 4272 \\
    393 & cosmopolitan\_jj & 0 & 4308 \\
    394 & withdrawn\_jj & 0 & 4321 \\
    395 & pompous\_jj & 0 & 4330 \\
    396 & perceptive\_jj & 0 & 4343 \\
    397 & melancholic\_jj & 0 & 4578 \\
    398 & conceited\_jj & 0 & 4601 \\
    399 & predictability & 0 & 4606 \\
    400 & pessimism & 0 & 4627 \\
    401 & morose\_jj & 0 & 4789 \\
    402 & reliable\_jj & 0 & 4877 \\
    403 & vain\_jj & 0 & 4927 \\
    404 & cunning\_jj & 0 & 5006 \\
    405 & forgetful\_jj & 0 & 5124 \\
    406 & homespun\_jj & 0 & 5280 \\
    407 & abusive\_jj & 0 & 5399 \\
    408 & moody\_jj & 0 & 5464 \\
    409 & rude\_jj & 0 & 5526 \\
    410 & rebellious\_jj & 0 & 5541 \\
    411 & cooperation & 0 & 5600 \\
    412 & intelligence & 0 & 5722 \\
    413 & detached\_jj & 0 & 5801 \\
    414 & prompt\_jj & 0 & 6558 \\
    415 & grumpy\_jj & 0 & 6754 \\
    416 & lethargy & 0 & 6859 \\
    417 & lenient\_jj & 0 & 7560 \\
    418 & surly\_jj & 0 & 9591 \\
    419 & belligerence & 1 & 647 \\
    420 & leniency & 1 & 1400 \\
    421 & compassionate\_jj & 1 & 3389 \\
    \hline
    \caption{Scores and rankings for most extreme 30 words in component \#31} \\
\end{longtable}
\begin{longtable}[!htbp]{| rlr@{.}l |}
    \hline
    \textbf{Rank} & \textbf{Word} & \multicolumn{2}{c|}{\textbf{Score}} \\
    \hline
    \endhead
    1 & belligerence & -1 & -4393 \\
    2 & prompt\_jj & -1 & -3433 \\
    3 & warm\_jj & -1 & -1566 \\
    4 & suspicious\_jj & 0 & -9344 \\
    5 & inhibition & 0 & -8979 \\
    6 & unemotional\_jj & 0 & -8508 \\
    7 & distrustful\_jj & 0 & -8149 \\
    8 & folksy\_jj & 0 & -7788 \\
    9 & cultured\_jj & 0 & -7610 \\
    10 & explosive\_jj & 0 & -7485 \\
    11 & erratic\_jj & 0 & -6328 \\
    12 & excitable\_jj & 0 & -6315 \\
    13 & self-disciplined\_jj & 0 & -6027 \\
    14 & shyness & 0 & -5820 \\
    15 & courtesy & 0 & -5615 \\
    16 & cold\_jj & 0 & -5475 \\
    17 & thrifty\_jj & 0 & -5358 \\
    18 & uncritical\_jj & 0 & -5277 \\
    19 & selfish\_jj & 0 & -5251 \\
    20 & detached\_jj & 0 & -5121 \\
    21 & deliberate\_jj & 0 & -5114 \\
    22 & rudeness & 0 & -4819 \\
    23 & surly\_jj & 0 & -4806 \\
    24 & unconventional\_jj & 0 & -4356 \\
    25 & orderly\_jj & 0 & -4312 \\
    26 & individualistic\_jj & 0 & -4289 \\
    27 & stubbornness & 0 & -4195 \\
    28 & adaptable\_jj & 0 & -4175 \\
    29 & undemanding\_jj & 0 & -4175 \\
    30 & generosity & 0 & -3872 \\
    392 & peaceful\_jj & 0 & 4075 \\
    393 & flexibility & 0 & 4091 \\
    394 & frank\_jj & 0 & 4130 \\
    395 & refined\_jj & 0 & 4156 \\
    396 & mannerly\_jj & 0 & 4159 \\
    397 & cunning\_jj & 0 & 4228 \\
    398 & antagonistic\_jj & 0 & 4249 \\
    399 & absent-minded\_jj & 0 & 4251 \\
    400 & deceitful\_jj & 0 & 4261 \\
    401 & cranky\_jj & 0 & 4354 \\
    402 & understanding\_jj & 0 & 4655 \\
    403 & decisiveness & 0 & 4813 \\
    404 & sly\_jj & 0 & 4939 \\
    405 & playful\_jj & 0 & 5233 \\
    406 & jealous\_jj & 0 & 5331 \\
    407 & instability & 0 & 5338 \\
    408 & cordial\_jj & 0 & 5343 \\
    409 & lenient\_jj & 0 & 5415 \\
    410 & negligent\_jj & 0 & 5554 \\
    411 & cooperative\_jj & 0 & 6005 \\
    412 & manipulative\_jj & 0 & 6024 \\
    413 & vindictive\_jj & 0 & 6054 \\
    414 & self-esteem & 0 & 6336 \\
    415 & devious\_jj & 0 & 6837 \\
    416 & concise\_jj & 0 & 6861 \\
    417 & economical\_jj & 0 & 6895 \\
    418 & independence & 0 & 7939 \\
    419 & cooperation & 0 & 8545 \\
    420 & quarrelsome\_jj & 1 & 181 \\
    421 & naturalness & 1 & 824 \\
    \hline
    \caption{Scores and rankings for most extreme 30 words in component \#32} \\
\end{longtable}
\begin{longtable}[!htbp]{| rlr@{.}l |}
    \hline
    \textbf{Rank} & \textbf{Word} & \multicolumn{2}{c|}{\textbf{Score}} \\
    \hline
    \endhead
    1 & mannerly\_jj & -1 & -4497 \\
    2 & nonconforming\_jj & -1 & -210 \\
    3 & courtesy & 0 & -8370 \\
    4 & compassionate\_jj & 0 & -8370 \\
    5 & callousness & 0 & -7974 \\
    6 & lenient\_jj & 0 & -7792 \\
    7 & generous\_jj & 0 & -6556 \\
    8 & sluggish\_jj & 0 & -6232 \\
    9 & thrift & 0 & -6045 \\
    10 & shallowness & 0 & -6024 \\
    11 & reserve & 0 & -5997 \\
    12 & aloofness & 0 & -5982 \\
    13 & wordy\_jj & 0 & -5533 \\
    14 & witty\_jj & 0 & -5272 \\
    15 & steady\_jj & 0 & -4909 \\
    16 & expressive\_jj & 0 & -4865 \\
    17 & impudent\_jj & 0 & -4863 \\
    18 & friendly\_jj & 0 & -4676 \\
    19 & leniency & 0 & -4458 \\
    20 & reckless\_jj & 0 & -4362 \\
    21 & uncritical\_jj & 0 & -4315 \\
    22 & cosmopolitan\_jj & 0 & -4253 \\
    23 & pomposity & 0 & -4014 \\
    24 & silence & 0 & -4008 \\
    25 & insight & 0 & -3952 \\
    26 & modest\_jj & 0 & -3940 \\
    27 & animation & 0 & -3918 \\
    28 & bigoted\_jj & 0 & -3904 \\
    29 & dishonest\_jj & 0 & -3866 \\
    30 & self-pitying\_jj & 0 & -3849 \\
    392 & crabby\_jj & 0 & 4159 \\
    393 & inquisitive\_jj & 0 & 4244 \\
    394 & cruelty & 0 & 4251 \\
    395 & autonomous\_jj & 0 & 4308 \\
    396 & spirit & 0 & 4416 \\
    397 & careful\_jj & 0 & 4420 \\
    398 & happy-go-lucky\_jj & 0 & 4531 \\
    399 & agreeable\_jj & 0 & 4753 \\
    400 & prompt\_jj & 0 & 4791 \\
    401 & exacting\_jj & 0 & 4850 \\
    402 & bitter\_jj & 0 & 4876 \\
    403 & meddlesome\_jj & 0 & 4918 \\
    404 & curious\_jj & 0 & 5105 \\
    405 & wishy-washy\_jj & 0 & 5124 \\
    406 & ethical\_jj & 0 & 5312 \\
    407 & modesty & 0 & 5592 \\
    408 & crafty\_jj & 0 & 5682 \\
    409 & philosophical\_jj & 0 & 5762 \\
    410 & touchy\_jj & 0 & 5914 \\
    411 & cranky\_jj & 0 & 6009 \\
    412 & curiosity & 0 & 6073 \\
    413 & rambunctious\_jj & 0 & 6542 \\
    414 & unkind\_jj & 0 & 6947 \\
    415 & naïve\_jj & 0 & 7147 \\
    416 & grumpy\_jj & 0 & 7168 \\
    417 & argumentative\_jj & 0 & 8779 \\
    418 & warm\_jj & 0 & 9073 \\
    419 & economical\_jj & 0 & 9235 \\
    420 & surly\_jj & 1 & 704 \\
    421 & refined\_jj & 1 & 3081 \\
    \hline
    \caption{Scores and rankings for most extreme 30 words in component \#33} \\
\end{longtable}
\begin{longtable}[!htbp]{| rlr@{.}l |}
    \hline
    \textbf{Rank} & \textbf{Word} & \multicolumn{2}{c|}{\textbf{Score}} \\
    \hline
    \endhead
    1 & unscrupulous\_jj & -1 & -4697 \\
    2 & orderly\_jj & 0 & -9970 \\
    3 & conceited\_jj & 0 & -8918 \\
    4 & refined\_jj & 0 & -7613 \\
    5 & opportunistic\_jj & 0 & -7229 \\
    6 & underhanded\_jj & 0 & -7133 \\
    7 & rebellious\_jj & 0 & -6836 \\
    8 & sophistication & 0 & -6725 \\
    9 & insight & 0 & -6725 \\
    10 & deceitful\_jj & 0 & -6098 \\
    11 & inconsiderate\_jj & 0 & -5930 \\
    12 & impolite\_jj & 0 & -5713 \\
    13 & unsociable\_jj & 0 & -5567 \\
    14 & easygoing\_jj & 0 & -5452 \\
    15 & submissive\_jj & 0 & -5330 \\
    16 & shallowness & 0 & -5189 \\
    17 & thrift & 0 & -5101 \\
    18 & quiet\_jj & 0 & -5019 \\
    19 & respectful\_jj & 0 & -4970 \\
    20 & manipulative\_jj & 0 & -4927 \\
    21 & somber\_jj & 0 & -4851 \\
    22 & predictability & 0 & -4627 \\
    23 & depth & 0 & -4548 \\
    24 & reserved\_jj & 0 & -4528 \\
    25 & ungracious\_jj & 0 & -4478 \\
    26 & informal\_jj & 0 & -4437 \\
    27 & courteous\_jj & 0 & -4306 \\
    28 & flexibility & 0 & -4249 \\
    29 & withdrawn\_jj & 0 & -4200 \\
    30 & traditional\_jj & 0 & -4157 \\
    392 & careless\_jj & 0 & 4058 \\
    393 & joyless\_jj & 0 & 4073 \\
    394 & selfish\_jj & 0 & 4158 \\
    395 & vivacious\_jj & 0 & 4344 \\
    396 & vigorous\_jj & 0 & 4510 \\
    397 & friendly\_jj & 0 & 4543 \\
    398 & courtesy & 0 & 4576 \\
    399 & bossiness & 0 & 4590 \\
    400 & diplomatic\_jj & 0 & 4603 \\
    401 & cranky\_jj & 0 & 4695 \\
    402 & instability & 0 & 4799 \\
    403 & communicative\_jj & 0 & 4898 \\
    404 & frivolous\_jj & 0 & 4947 \\
    405 & surliness & 0 & 5069 \\
    406 & belligerence & 0 & 5196 \\
    407 & verbose\_jj & 0 & 5252 \\
    408 & perceptive\_jj & 0 & 5498 \\
    409 & disorganization & 0 & 5782 \\
    410 & inefficient\_jj & 0 & 5946 \\
    411 & negligence & 0 & 6248 \\
    412 & charitable\_jj & 0 & 6252 \\
    413 & sloth & 0 & 6375 \\
    414 & suspicious\_jj & 0 & 6395 \\
    415 & animation & 0 & 6594 \\
    416 & melancholic\_jj & 0 & 6633 \\
    417 & lazy\_jj & 0 & 6795 \\
    418 & quarrelsome\_jj & 0 & 7543 \\
    419 & economical\_jj & 0 & 9598 \\
    420 & compassionate\_jj & 0 & 9641 \\
    421 & prompt\_jj & 1 & 5207 \\
    \hline
    \caption{Scores and rankings for most extreme 30 words in component \#34} \\
\end{longtable}
\begin{longtable}[!htbp]{| rlr@{.}l |}
    \hline
    \textbf{Rank} & \textbf{Word} & \multicolumn{2}{c|}{\textbf{Score}} \\
    \hline
    \endhead
    1 & silence & -1 & -1879 \\
    2 & suspicious\_jj & -1 & -427 \\
    3 & insight & -1 & -117 \\
    4 & friendly\_jj & 0 & -9475 \\
    5 & irritable\_jj & 0 & -9317 \\
    6 & unscrupulous\_jj & 0 & -7811 \\
    7 & dignity & 0 & -7181 \\
    8 & thrifty\_jj & 0 & -6530 \\
    9 & spontaneity & 0 & -5790 \\
    10 & naturalness & 0 & -5785 \\
    11 & forgetful\_jj & 0 & -5679 \\
    12 & negligence & 0 & -5490 \\
    13 & nosey\_jj & 0 & -5448 \\
    14 & candor & 0 & -5442 \\
    15 & folksy\_jj & 0 & -5135 \\
    16 & dignified\_jj & 0 & -5102 \\
    17 & bashful\_jj & 0 & -5024 \\
    18 & homespun\_jj & 0 & -4968 \\
    19 & tempestuous\_jj & 0 & -4864 \\
    20 & happy-go-lucky\_jj & 0 & -4781 \\
    21 & emotional\_jj & 0 & -4732 \\
    22 & sentimental\_jj & 0 & -4516 \\
    23 & recklessness & 0 & -4433 \\
    24 & modesty & 0 & -4428 \\
    25 & secretive\_jj & 0 & -4330 \\
    26 & vain\_jj & 0 & -4326 \\
    27 & compassionate\_jj & 0 & -4290 \\
    28 & efficient\_jj & 0 & -4256 \\
    29 & cruel\_jj & 0 & -3981 \\
    30 & quiet & 0 & -3941 \\
    392 & gregarious\_jj & 0 & 3822 \\
    393 & sincere\_jj & 0 & 3854 \\
    394 & expressiveness & 0 & 3889 \\
    395 & merry\_jj & 0 & 3959 \\
    396 & jovial\_jj & 0 & 4026 \\
    397 & polite\_jj & 0 & 4131 \\
    398 & careless\_jj & 0 & 4175 \\
    399 & egotistical\_jj & 0 & 4215 \\
    400 & mannerly\_jj & 0 & 4354 \\
    401 & considerate\_jj & 0 & 4408 \\
    402 & gruff\_jj & 0 & 4419 \\
    403 & understanding\_jj & 0 & 4537 \\
    404 & analytical\_jj & 0 & 4859 \\
    405 & brave\_jj & 0 & 5105 \\
    406 & pessimistic\_jj & 0 & 5168 \\
    407 & pessimism & 0 & 5189 \\
    408 & rude\_jj & 0 & 5280 \\
    409 & animation & 0 & 5407 \\
    410 & unemotional\_jj & 0 & 5631 \\
    411 & punctuality & 0 & 5740 \\
    412 & prompt\_jj & 0 & 5970 \\
    413 & inventive\_jj & 0 & 6023 \\
    414 & unreflective\_jj & 0 & 7043 \\
    415 & inconsiderate\_jj & 0 & 7165 \\
    416 & shallowness & 0 & 7226 \\
    417 & autonomous\_jj & 0 & 7286 \\
    418 & punctual\_jj & 0 & 7474 \\
    419 & warm\_jj & 0 & 8969 \\
    420 & nonconforming\_jj & 0 & 9207 \\
    421 & surly\_jj & 0 & 9890 \\
    \hline
    \caption{Scores and rankings for most extreme 30 words in component \#35} \\
\end{longtable}
\begin{longtable}[!htbp]{| rlr@{.}l |}
    \hline
    \textbf{Rank} & \textbf{Word} & \multicolumn{2}{c|}{\textbf{Score}} \\
    \hline
    \endhead
    1 & silence & 0 & -7533 \\
    2 & refined\_jj & 0 & -7325 \\
    3 & somber\_jj & 0 & -7324 \\
    4 & indecisiveness & 0 & -6821 \\
    5 & scornful\_jj & 0 & -6597 \\
    6 & warm\_jj & 0 & -6569 \\
    7 & wishy-washy\_jj & 0 & -6386 \\
    8 & intrusiveness & 0 & -5993 \\
    9 & irritability & 0 & -5745 \\
    10 & erratic\_jj & 0 & -5556 \\
    11 & skeptical\_jj & 0 & -5510 \\
    12 & punctual\_jj & 0 & -5506 \\
    13 & stinginess & 0 & -5407 \\
    14 & intelligence & 0 & -5388 \\
    15 & selfless\_jj & 0 & -5356 \\
    16 & brave\_jj & 0 & -5296 \\
    17 & reliable\_jj & 0 & -5109 \\
    18 & thrift & 0 & -4815 \\
    19 & animation & 0 & -4754 \\
    20 & distrustful\_jj & 0 & -4677 \\
    21 & boastful\_jj & 0 & -4487 \\
    22 & disorganization & 0 & -4487 \\
    23 & zestful\_jj & 0 & -4447 \\
    24 & homespun\_jj & 0 & -4377 \\
    25 & talkative\_jj & 0 & -4221 \\
    26 & undependable\_jj & 0 & -4213 \\
    27 & economical\_jj & 0 & -4167 \\
    28 & unsympathetic\_jj & 0 & -4136 \\
    29 & insight & 0 & -4078 \\
    30 & volatility & 0 & -4030 \\
    392 & reserve & 0 & 4385 \\
    393 & submissive\_jj & 0 & 4392 \\
    394 & rude\_jj & 0 & 4394 \\
    395 & flexibility & 0 & 4602 \\
    396 & lazy\_jj & 0 & 4630 \\
    397 & sophistication & 0 & 4638 \\
    398 & sociable\_jj & 0 & 4642 \\
    399 & ungracious\_jj & 0 & 4700 \\
    400 & thoughtless\_jj & 0 & 4764 \\
    401 & inhibition & 0 & 4778 \\
    402 & unintelligent\_jj & 0 & 4954 \\
    403 & obliging\_jj & 0 & 4965 \\
    404 & flexible\_jj & 0 & 5189 \\
    405 & sloppy\_jj & 0 & 5400 \\
    406 & shallow\_jj & 0 & 5523 \\
    407 & sluggish\_jj & 0 & 5548 \\
    408 & easygoing\_jj & 0 & 5594 \\
    409 & prompt\_jj & 0 & 5735 \\
    410 & earthiness & 0 & 5886 \\
    411 & concise\_jj & 0 & 5925 \\
    412 & earthy\_jj & 0 & 5941 \\
    413 & sloth & 0 & 6137 \\
    414 & mannerly\_jj & 0 & 6478 \\
    415 & insecure\_jj & 0 & 6626 \\
    416 & unrestrained\_jj & 0 & 6639 \\
    417 & crabby\_jj & 0 & 7533 \\
    418 & surly\_jj & 0 & 7591 \\
    419 & lethargic\_jj & 0 & 9813 \\
    420 & belligerence & 1 & 1286 \\
    421 & autonomous\_jj & 1 & 1706 \\
    \hline
    \caption{Scores and rankings for most extreme 30 words in component \#36} \\
\end{longtable}
\begin{longtable}[!htbp]{| rlr@{.}l |}
    \hline
    \textbf{Rank} & \textbf{Word} & \multicolumn{2}{c|}{\textbf{Score}} \\
    \hline
    \endhead
    1 & nonconforming\_jj & -1 & -169 \\
    2 & careless\_jj & 0 & -8053 \\
    3 & self-pitying\_jj & 0 & -7893 \\
    4 & earthiness & 0 & -7828 \\
    5 & homespun\_jj & 0 & -6965 \\
    6 & erratic\_jj & 0 & -6851 \\
    7 & distrustful\_jj & 0 & -6666 \\
    8 & self-esteem & 0 & -6491 \\
    9 & mannerly\_jj & 0 & -6151 \\
    10 & joyless\_jj & 0 & -6045 \\
    11 & insecure\_jj & 0 & -5807 \\
    12 & negligent\_jj & 0 & -5529 \\
    13 & rude\_jj & 0 & -5103 \\
    14 & brave\_jj & 0 & -5064 \\
    15 & thrift & 0 & -4955 \\
    16 & economical\_jj & 0 & -4911 \\
    17 & nosey\_jj & 0 & -4845 \\
    18 & touchy\_jj & 0 & -4811 \\
    19 & frank\_jj & 0 & -4710 \\
    20 & lethargy & 0 & -4642 \\
    21 & fastidious\_jj & 0 & -4634 \\
    22 & reckless\_jj & 0 & -4477 \\
    23 & bossiness & 0 & -4441 \\
    24 & helpful\_jj & 0 & -4430 \\
    25 & unfriendly\_jj & 0 & -4289 \\
    26 & impolite\_jj & 0 & -4282 \\
    27 & deliberate\_jj & 0 & -4281 \\
    28 & insight & 0 & -4274 \\
    29 & inefficient\_jj & 0 & -4159 \\
    30 & unreflective\_jj & 0 & -4051 \\
    392 & playfulness & 0 & 3803 \\
    393 & melancholic\_jj & 0 & 3810 \\
    394 & restrained\_jj & 0 & 3812 \\
    395 & happy-go-lucky\_jj & 0 & 3841 \\
    396 & egotistical\_jj & 0 & 3942 \\
    397 & natural\_jj & 0 & 4036 \\
    398 & meddlesome\_jj & 0 & 4070 \\
    399 & steady\_jj & 0 & 4107 \\
    400 & somber\_jj & 0 & 4351 \\
    401 & wishy-washy\_jj & 0 & 4388 \\
    402 & indecisive\_jj & 0 & 4869 \\
    403 & self-indulgent\_jj & 0 & 4922 \\
    404 & condescending\_jj & 0 & 4926 \\
    405 & gullibility & 0 & 4956 \\
    406 & naïve\_jj & 0 & 5069 \\
    407 & volatility & 0 & 5366 \\
    408 & insensitive\_jj & 0 & 5388 \\
    409 & docile\_jj & 0 & 5438 \\
    410 & warm\_jj & 0 & 5503 \\
    411 & sloth & 0 & 5577 \\
    412 & bigoted\_jj & 0 & 5700 \\
    413 & shallowness & 0 & 5790 \\
    414 & silent\_jj & 0 & 6284 \\
    415 & silence & 0 & 6677 \\
    416 & lazy\_jj & 0 & 7171 \\
    417 & punctuality & 0 & 7379 \\
    418 & argumentative\_jj & 0 & 9611 \\
    419 & sophistication & 0 & 9798 \\
    420 & compassionate\_jj & 1 & 485 \\
    421 & explosive\_jj & 1 & 680 \\
    \hline
    \caption{Scores and rankings for most extreme 30 words in component \#37} \\
\end{longtable}
\begin{longtable}[!htbp]{| rlr@{.}l |}
    \hline
    \textbf{Rank} & \textbf{Word} & \multicolumn{2}{c|}{\textbf{Score}} \\
    \hline
    \endhead
    1 & impudent\_jj & -1 & -830 \\
    2 & inconsiderate\_jj & 0 & -8620 \\
    3 & rash\_jj & 0 & -7709 \\
    4 & dependability & 0 & -7061 \\
    5 & aimless\_jj & 0 & -7006 \\
    6 & cruelty & 0 & -6797 \\
    7 & shallowness & 0 & -6751 \\
    8 & folksy\_jj & 0 & -6724 \\
    9 & punctuality & 0 & -6720 \\
    10 & exacting\_jj & 0 & -6361 \\
    11 & lenient\_jj & 0 & -5999 \\
    12 & precise\_jj & 0 & -5907 \\
    13 & courtesy & 0 & -5407 \\
    14 & careless\_jj & 0 & -5274 \\
    15 & haphazard\_jj & 0 & -5107 \\
    16 & callousness & 0 & -5050 \\
    17 & caustic\_jj & 0 & -4929 \\
    18 & silence & 0 & -4863 \\
    19 & shy\_jj & 0 & -4728 \\
    20 & secretive\_jj & 0 & -4678 \\
    21 & cunning & 0 & -4638 \\
    22 & charitable\_jj & 0 & -4575 \\
    23 & wordy\_jj & 0 & -4516 \\
    24 & lethargy & 0 & -4373 \\
    25 & distrust & 0 & -4185 \\
    26 & autonomous\_jj & 0 & -4090 \\
    27 & touchy\_jj & 0 & -4018 \\
    28 & rudeness & 0 & -3950 \\
    29 & impetuous\_jj & 0 & -3927 \\
    30 & forgetfulness & 0 & -3927 \\
    392 & devious\_jj & 0 & 3722 \\
    393 & curt\_jj & 0 & 3731 \\
    394 & creative\_jj & 0 & 3769 \\
    395 & ungracious\_jj & 0 & 3853 \\
    396 & self-esteem & 0 & 3894 \\
    397 & frivolous\_jj & 0 & 3898 \\
    398 & envy & 0 & 4098 \\
    399 & homespun\_jj & 0 & 4316 \\
    400 & innovative\_jj & 0 & 4334 \\
    401 & bright\_jj & 0 & 4467 \\
    402 & concise\_jj & 0 & 4560 \\
    403 & intellectual\_jj & 0 & 4591 \\
    404 & nonconforming\_jj & 0 & 4823 \\
    405 & prejudice & 0 & 5184 \\
    406 & forgetful\_jj & 0 & 5210 \\
    407 & independence & 0 & 5390 \\
    408 & prejudiced\_jj & 0 & 5419 \\
    409 & smart\_jj & 0 & 5578 \\
    410 & volatility & 0 & 5587 \\
    411 & animation & 0 & 5789 \\
    412 & thrift & 0 & 6142 \\
    413 & boastful\_jj & 0 & 6207 \\
    414 & irritable\_jj & 0 & 6998 \\
    415 & prompt\_jj & 0 & 7272 \\
    416 & sophistication & 0 & 7575 \\
    417 & cordial\_jj & 0 & 8176 \\
    418 & reserve & 0 & 8350 \\
    419 & surly\_jj & 0 & 8585 \\
    420 & warm\_jj & 1 & 907 \\
    421 & negligent\_jj & 1 & 1548 \\
    \hline
    \caption{Scores and rankings for most extreme 30 words in component \#38} \\
\end{longtable}
\begin{longtable}[!htbp]{| rlr@{.}l |}
    \hline
    \textbf{Rank} & \textbf{Word} & \multicolumn{2}{c|}{\textbf{Score}} \\
    \hline
    \endhead
    1 & prompt\_jj & -1 & -3142 \\
    2 & folksy\_jj & -1 & -523 \\
    3 & silence & 0 & -8619 \\
    4 & friendly\_jj & 0 & -8609 \\
    5 & cordial\_jj & 0 & -8602 \\
    6 & shallowness & 0 & -8266 \\
    7 & negligent\_jj & 0 & -8094 \\
    8 & lazy\_jj & 0 & -6515 \\
    9 & adventurous\_jj & 0 & -6257 \\
    10 & self-critical\_jj & 0 & -5890 \\
    11 & opportunistic\_jj & 0 & -5884 \\
    12 & mannerly\_jj & 0 & -5778 \\
    13 & slothful\_jj & 0 & -5617 \\
    14 & lethargy & 0 & -5613 \\
    15 & egotistical\_jj & 0 & -5576 \\
    16 & warmth & 0 & -5183 \\
    17 & punctual\_jj & 0 & -5135 \\
    18 & intellectual\_jj & 0 & -5114 \\
    19 & unscrupulous\_jj & 0 & -5087 \\
    20 & warm\_jj & 0 & -4881 \\
    21 & expressiveness & 0 & -4623 \\
    22 & insight & 0 & -4508 \\
    23 & thrifty\_jj & 0 & -4497 \\
    24 & quarrelsome\_jj & 0 & -4361 \\
    25 & artistic\_jj & 0 & -4259 \\
    26 & selfish\_jj & 0 & -4245 \\
    27 & inconsiderate\_jj & 0 & -4242 \\
    28 & deliberate\_jj & 0 & -4223 \\
    29 & condescending\_jj & 0 & -4133 \\
    30 & emotional\_jj & 0 & -4010 \\
    392 & straightforward\_jj & 0 & 3687 \\
    393 & animation & 0 & 3745 \\
    394 & stingy\_jj & 0 & 3791 \\
    395 & naturalness & 0 & 3842 \\
    396 & careful\_jj & 0 & 4140 \\
    397 & volatility & 0 & 4220 \\
    398 & tactless\_jj & 0 & 4249 \\
    399 & thorough\_jj & 0 & 4361 \\
    400 & touchy\_jj & 0 & 4589 \\
    401 & erratic\_jj & 0 & 4594 \\
    402 & truthful\_jj & 0 & 4653 \\
    403 & instability & 0 & 4854 \\
    404 & gregarious\_jj & 0 & 4939 \\
    405 & compassionate\_jj & 0 & 5012 \\
    406 & miserly\_jj & 0 & 5023 \\
    407 & recklessness & 0 & 5268 \\
    408 & naïve\_jj & 0 & 5367 \\
    409 & happy-go-lucky\_jj & 0 & 5420 \\
    410 & curt\_jj & 0 & 5437 \\
    411 & unrestrained\_jj & 0 & 5449 \\
    412 & suspicious\_jj & 0 & 5484 \\
    413 & courtesy & 0 & 5490 \\
    414 & pomposity & 0 & 5779 \\
    415 & surly\_jj & 0 & 6036 \\
    416 & understanding\_jj & 0 & 6123 \\
    417 & autonomous\_jj & 0 & 6659 \\
    418 & refined\_jj & 0 & 6817 \\
    419 & lenient\_jj & 0 & 6999 \\
    420 & charitable\_jj & 0 & 7457 \\
    421 & belligerence & 0 & 7691 \\
    \hline
    \caption{Scores and rankings for most extreme 30 words in component \#39} \\
\end{longtable}
\begin{longtable}[!htbp]{| rlr@{.}l |}
    \hline
    \textbf{Rank} & \textbf{Word} & \multicolumn{2}{c|}{\textbf{Score}} \\
    \hline
    \endhead
    1 & negligent\_jj & 0 & -9487 \\
    2 & surly\_jj & 0 & -8797 \\
    3 & disrespectful\_jj & 0 & -8047 \\
    4 & foolhardy\_jj & 0 & -7032 \\
    5 & explosive\_jj & 0 & -6990 \\
    6 & suspicious\_jj & 0 & -6975 \\
    7 & earthiness & 0 & -6872 \\
    8 & stinginess & 0 & -5914 \\
    9 & courteous\_jj & 0 & -5825 \\
    10 & individualistic\_jj & 0 & -5666 \\
    11 & intelligence & 0 & -5566 \\
    12 & shallow\_jj & 0 & -5495 \\
    13 & vain\_jj & 0 & -5430 \\
    14 & rude\_jj & 0 & -5323 \\
    15 & charitable\_jj & 0 & -5232 \\
    16 & helpful\_jj & 0 & -5062 \\
    17 & irritability & 0 & -5050 \\
    18 & lethargy & 0 & -4845 \\
    19 & stingy\_jj & 0 & -4839 \\
    20 & prompt\_jj & 0 & -4811 \\
    21 & mannerly\_jj & 0 & -4659 \\
    22 & devious\_jj & 0 & -4582 \\
    23 & cruel\_jj & 0 & -4574 \\
    24 & deceitful\_jj & 0 & -4443 \\
    25 & impudent\_jj & 0 & -4439 \\
    26 & accommodating\_jj & 0 & -4430 \\
    27 & pessimistic\_jj & 0 & -4423 \\
    28 & thrifty\_jj & 0 & -4416 \\
    29 & manipulative\_jj & 0 & -4400 \\
    30 & defensive\_jj & 0 & -4375 \\
    392 & animation & 0 & 4076 \\
    393 & artistic\_jj & 0 & 4122 \\
    394 & unassuming\_jj & 0 & 4151 \\
    395 & wishy-washy\_jj & 0 & 4164 \\
    396 & unambitious\_jj & 0 & 4203 \\
    397 & lenient\_jj & 0 & 4319 \\
    398 & rebellious\_jj & 0 & 4321 \\
    399 & insecure\_jj & 0 & 4324 \\
    400 & thrift & 0 & 4350 \\
    401 & negligence & 0 & 4525 \\
    402 & shyness & 0 & 4547 \\
    403 & recklessness & 0 & 4671 \\
    404 & moral\_jj & 0 & 4786 \\
    405 & harsh\_jj & 0 & 4844 \\
    406 & independent\_jj & 0 & 4860 \\
    407 & orderly\_jj & 0 & 5141 \\
    408 & bashful\_jj & 0 & 5166 \\
    409 & inconsiderate\_jj & 0 & 5173 \\
    410 & instability & 0 & 5292 \\
    411 & warm\_jj & 0 & 5752 \\
    412 & nonconforming\_jj & 0 & 5939 \\
    413 & frank\_jj & 0 & 5986 \\
    414 & indecisive\_jj & 0 & 6115 \\
    415 & impolite\_jj & 0 & 6140 \\
    416 & informal\_jj & 0 & 6203 \\
    417 & aloofness & 0 & 6377 \\
    418 & verbal\_jj & 0 & 6587 \\
    419 & thorough\_jj & 0 & 8221 \\
    420 & economical\_jj & 0 & 9808 \\
    421 & assured\_jj & 1 & 299 \\
    \hline
    \caption{Scores and rankings for most extreme 30 words in component \#40} \\
\end{longtable}
\begin{longtable}[!htbp]{| rlr@{.}l |}
    \hline
    \textbf{Rank} & \textbf{Word} & \multicolumn{2}{c|}{\textbf{Score}} \\
    \hline
    \endhead
    1 & orderly\_jj & -1 & -1421 \\
    2 & refined\_jj & -1 & -247 \\
    3 & quarrelsome\_jj & 0 & -8557 \\
    4 & haphazard\_jj & 0 & -7596 \\
    5 & negligent\_jj & 0 & -7369 \\
    6 & detached\_jj & 0 & -7208 \\
    7 & condescending\_jj & 0 & -6900 \\
    8 & cosmopolitan\_jj & 0 & -6298 \\
    9 & sloth & 0 & -6269 \\
    10 & agreeable\_jj & 0 & -5887 \\
    11 & selfish\_jj & 0 & -5746 \\
    12 & artistic\_jj & 0 & -5600 \\
    13 & flippant\_jj & 0 & -5512 \\
    14 & natural\_jj & 0 & -5261 \\
    15 & logical\_jj & 0 & -5130 \\
    16 & adventurous\_jj & 0 & -5065 \\
    17 & sophistication & 0 & -5035 \\
    18 & earthiness & 0 & -4570 \\
    19 & tempestuous\_jj & 0 & -4403 \\
    20 & envious\_jj & 0 & -4328 \\
    21 & responsible\_jj & 0 & -4217 \\
    22 & antagonistic\_jj & 0 & -4189 \\
    23 & negligence & 0 & -4012 \\
    24 & formal\_jj & 0 & -3932 \\
    25 & insensitive\_jj & 0 & -3887 \\
    26 & easygoing\_jj & 0 & -3845 \\
    27 & forgetfulness & 0 & -3795 \\
    28 & sincere\_jj & 0 & -3794 \\
    29 & extravagant\_jj & 0 & -3761 \\
    30 & pomposity & 0 & -3708 \\
    392 & deceit & 0 & 4287 \\
    393 & underhanded\_jj & 0 & 4306 \\
    394 & communicative\_jj & 0 & 4452 \\
    395 & boastful\_jj & 0 & 4529 \\
    396 & argumentative\_jj & 0 & 4586 \\
    397 & self-esteem & 0 & 4586 \\
    398 & talkative\_jj & 0 & 4602 \\
    399 & warm\_jj & 0 & 4647 \\
    400 & withdrawn\_jj & 0 & 4842 \\
    401 & obliging\_jj & 0 & 4846 \\
    402 & defensive\_jj & 0 & 5234 \\
    403 & cruelty & 0 & 5249 \\
    404 & morose\_jj & 0 & 5315 \\
    405 & merry\_jj & 0 & 5406 \\
    406 & surly\_jj & 0 & 5463 \\
    407 & contemplative\_jj & 0 & 5510 \\
    408 & meditative\_jj & 0 & 5586 \\
    409 & dependability & 0 & 5594 \\
    410 & rebellious\_jj & 0 & 5632 \\
    411 & independence & 0 & 5652 \\
    412 & efficiency & 0 & 5686 \\
    413 & self-pitying\_jj & 0 & 5727 \\
    414 & devious\_jj & 0 & 5736 \\
    415 & expressive\_jj & 0 & 7007 \\
    416 & gullible\_jj & 0 & 7698 \\
    417 & compassionate\_jj & 0 & 7879 \\
    418 & deliberate\_jj & 0 & 7927 \\
    419 & unscrupulous\_jj & 0 & 7964 \\
    420 & economical\_jj & 0 & 8497 \\
    421 & folksy\_jj & 1 & 2349 \\
    \hline
    \caption{Scores and rankings for most extreme 30 words in component \#41} \\
\end{longtable}
\begin{longtable}[!htbp]{| rlr@{.}l |}
    \hline
    \textbf{Rank} & \textbf{Word} & \multicolumn{2}{c|}{\textbf{Score}} \\
    \hline
    \endhead
    1 & mannerly\_jj & -1 & -6792 \\
    2 & compassionate\_jj & 0 & -7685 \\
    3 & thorough\_jj & 0 & -7280 \\
    4 & pomposity & 0 & -6623 \\
    5 & volatility & 0 & -6480 \\
    6 & sloth & 0 & -6338 \\
    7 & warm\_jj & 0 & -5940 \\
    8 & charitable\_jj & 0 & -5890 \\
    9 & deliberate\_jj & 0 & -5584 \\
    10 & independent\_jj & 0 & -5472 \\
    11 & unkind\_jj & 0 & -5399 \\
    12 & pessimism & 0 & -5397 \\
    13 & careful\_jj & 0 & -5349 \\
    14 & caution & 0 & -5228 \\
    15 & efficient\_jj & 0 & -4950 \\
    16 & underhanded\_jj & 0 & -4942 \\
    17 & fastidious\_jj & 0 & -4871 \\
    18 & communicative\_jj & 0 & -4790 \\
    19 & individualistic\_jj & 0 & -4624 \\
    20 & volatile\_jj & 0 & -4593 \\
    21 & docile\_jj & 0 & -4528 \\
    22 & intelligence & 0 & -4516 \\
    23 & thrift & 0 & -4491 \\
    24 & irritability & 0 & -4490 \\
    25 & frank\_jj & 0 & -4337 \\
    26 & indecisiveness & 0 & -4305 \\
    27 & reckless\_jj & 0 & -4297 \\
    28 & verbose\_jj & 0 & -4144 \\
    29 & deceitful\_jj & 0 & -4130 \\
    30 & cautious\_jj & 0 & -4039 \\
    392 & stupidity & 0 & 3747 \\
    393 & unsympathetic\_jj & 0 & 3763 \\
    394 & introspective\_jj & 0 & 3854 \\
    395 & suspicious\_jj & 0 & 3950 \\
    396 & impudent\_jj & 0 & 3981 \\
    397 & inconsistency & 0 & 4162 \\
    398 & friendly\_jj & 0 & 4252 \\
    399 & humorous\_jj & 0 & 4266 \\
    400 & negligence & 0 & 4355 \\
    401 & disrespectful\_jj & 0 & 4569 \\
    402 & surliness & 0 & 4767 \\
    403 & intrusiveness & 0 & 5073 \\
    404 & peaceful\_jj & 0 & 5156 \\
    405 & leniency & 0 & 5288 \\
    406 & defensive\_jj & 0 & 5536 \\
    407 & belligerence & 0 & 5564 \\
    408 & rebellious\_jj & 0 & 5618 \\
    409 & conceited\_jj & 0 & 5704 \\
    410 & irritable\_jj & 0 & 5944 \\
    411 & expressiveness & 0 & 6007 \\
    412 & wordy\_jj & 0 & 6270 \\
    413 & distrustful\_jj & 0 & 6410 \\
    414 & rude\_jj & 0 & 6487 \\
    415 & nonconforming\_jj & 0 & 6708 \\
    416 & vain\_jj & 0 & 6822 \\
    417 & cordial\_jj & 0 & 6898 \\
    418 & aimless\_jj & 0 & 7076 \\
    419 & refined\_jj & 0 & 7806 \\
    420 & disorganization & 0 & 7816 \\
    421 & punctuality & 0 & 8288 \\
    \hline
    \caption{Scores and rankings for most extreme 30 words in component \#42} \\
\end{longtable}
\begin{longtable}[!htbp]{| rlr@{.}l |}
    \hline
    \textbf{Rank} & \textbf{Word} & \multicolumn{2}{c|}{\textbf{Score}} \\
    \hline
    \endhead
    1 & selfish\_jj & 0 & -9367 \\
    2 & self-pitying\_jj & 0 & -8896 \\
    3 & leniency & 0 & -8512 \\
    4 & recklessness & 0 & -8485 \\
    5 & surly\_jj & 0 & -7946 \\
    6 & independence & 0 & -6557 \\
    7 & reckless\_jj & 0 & -6525 \\
    8 & shallow\_jj & 0 & -6394 \\
    9 & inhibition & 0 & -5931 \\
    10 & punctual\_jj & 0 & -5798 \\
    11 & reserve & 0 & -5717 \\
    12 & autonomous\_jj & 0 & -5677 \\
    13 & depth & 0 & -5438 \\
    14 & abusive\_jj & 0 & -5068 \\
    15 & insight & 0 & -5060 \\
    16 & deliberate\_jj & 0 & -4727 \\
    17 & boastful\_jj & 0 & -4629 \\
    18 & cosmopolitan\_jj & 0 & -4566 \\
    19 & crabby\_jj & 0 & -4529 \\
    20 & genial\_jj & 0 & -4513 \\
    21 & distrustful\_jj & 0 & -4439 \\
    22 & logical\_jj & 0 & -4369 \\
    23 & diplomatic\_jj & 0 & -4363 \\
    24 & understanding & 0 & -4341 \\
    25 & gruff\_jj & 0 & -4300 \\
    26 & natural\_jj & 0 & -4215 \\
    27 & cordial\_jj & 0 & -4155 \\
    28 & curiosity & 0 & -4117 \\
    29 & shallowness & 0 & -4068 \\
    30 & forgetfulness & 0 & -4050 \\
    392 & pomposity & 0 & 3957 \\
    393 & rude\_jj & 0 & 4164 \\
    394 & firm\_jj & 0 & 4214 \\
    395 & snobbish\_jj & 0 & 4224 \\
    396 & suspicious\_jj & 0 & 4247 \\
    397 & spirit & 0 & 4266 \\
    398 & inventive\_jj & 0 & 4340 \\
    399 & meticulous\_jj & 0 & 4579 \\
    400 & nonconforming\_jj & 0 & 4606 \\
    401 & kind\_jj & 0 & 4615 \\
    402 & disrespectful\_jj & 0 & 4683 \\
    403 & nonconformity & 0 & 4770 \\
    404 & dependability & 0 & 4897 \\
    405 & expressive\_jj & 0 & 4991 \\
    406 & friendly\_jj & 0 & 4992 \\
    407 & passivity & 0 & 5010 \\
    408 & expressiveness & 0 & 5043 \\
    409 & homespun\_jj & 0 & 5206 \\
    410 & proud\_jj & 0 & 5276 \\
    411 & steady\_jj & 0 & 5345 \\
    412 & unreflective\_jj & 0 & 5386 \\
    413 & unemotional\_jj & 0 & 5464 \\
    414 & inconsistent\_jj & 0 & 5586 \\
    415 & unkind\_jj & 0 & 6068 \\
    416 & unreliable\_jj & 0 & 6560 \\
    417 & charitable\_jj & 0 & 6790 \\
    418 & prompt\_jj & 0 & 7070 \\
    419 & prejudice & 0 & 7711 \\
    420 & animation & 0 & 7725 \\
    421 & quarrelsome\_jj & 0 & 9981 \\
    \hline
    \caption{Scores and rankings for most extreme 30 words in component \#43} \\
\end{longtable}
\begin{longtable}[!htbp]{| rlr@{.}l |}
    \hline
    \textbf{Rank} & \textbf{Word} & \multicolumn{2}{c|}{\textbf{Score}} \\
    \hline
    \endhead
    1 & compassionate\_jj & -1 & -152 \\
    2 & concise\_jj & -1 & -10 \\
    3 & surly\_jj & 0 & -9160 \\
    4 & nonconforming\_jj & 0 & -7774 \\
    5 & caustic\_jj & 0 & -7146 \\
    6 & surliness & 0 & -6327 \\
    7 & curt\_jj & 0 & -6225 \\
    8 & lazy\_jj & 0 & -6164 \\
    9 & ethical\_jj & 0 & -6047 \\
    10 & smart\_jj & 0 & -5672 \\
    11 & ungracious\_jj & 0 & -5504 \\
    12 & inhibition & 0 & -5488 \\
    13 & silence & 0 & -5405 \\
    14 & refined\_jj & 0 & -5157 \\
    15 & innovative\_jj & 0 & -5138 \\
    16 & withdrawn\_jj & 0 & -5118 \\
    17 & indecisive\_jj & 0 & -5083 \\
    18 & nervous\_jj & 0 & -4928 \\
    19 & cultured\_jj & 0 & -4873 \\
    20 & thrifty\_jj & 0 & -4872 \\
    21 & tempestuous\_jj & 0 & -4544 \\
    22 & impolite\_jj & 0 & -4494 \\
    23 & dependability & 0 & -4458 \\
    24 & assured\_jj & 0 & -4421 \\
    25 & impudent\_jj & 0 & -4404 \\
    26 & cosmopolitan\_jj & 0 & -4354 \\
    27 & self-disciplined\_jj & 0 & -4136 \\
    28 & cruelty & 0 & -4092 \\
    29 & zestful\_jj & 0 & -4047 \\
    30 & kind\_jj & 0 & -3895 \\
    392 & extroverted\_jj & 0 & 3677 \\
    393 & unsympathetic\_jj & 0 & 3741 \\
    394 & consistent\_jj & 0 & 3834 \\
    395 & vain\_jj & 0 & 3909 \\
    396 & unassuming\_jj & 0 & 4006 \\
    397 & meditative\_jj & 0 & 4037 \\
    398 & self-critical\_jj & 0 & 4102 \\
    399 & unreliable\_jj & 0 & 4118 \\
    400 & economical\_jj & 0 & 4142 \\
    401 & cranky\_jj & 0 & 4262 \\
    402 & intrusive\_jj & 0 & 4298 \\
    403 & antagonistic\_jj & 0 & 4381 \\
    404 & obstinate\_jj & 0 & 4723 \\
    405 & uncritical\_jj & 0 & 4751 \\
    406 & humorous\_jj & 0 & 4767 \\
    407 & reasonable\_jj & 0 & 4952 \\
    408 & pleasant\_jj & 0 & 5023 \\
    409 & accommodating\_jj & 0 & 5082 \\
    410 & unemotional\_jj & 0 & 5318 \\
    411 & self-pitying\_jj & 0 & 5410 \\
    412 & cold\_jj & 0 & 5585 \\
    413 & meddlesome\_jj & 0 & 5788 \\
    414 & forgetful\_jj & 0 & 5823 \\
    415 & tactful\_jj & 0 & 5845 \\
    416 & suspicious\_jj & 0 & 6465 \\
    417 & negligence & 0 & 6757 \\
    418 & insight & 0 & 7168 \\
    419 & mannerly\_jj & 0 & 9335 \\
    420 & unreflective\_jj & 0 & 9534 \\
    421 & detached\_jj & 0 & 9949 \\
    \hline
    \caption{Scores and rankings for most extreme 30 words in component \#44} \\
\end{longtable}
\begin{longtable}[!htbp]{| rlr@{.}l |}
    \hline
    \textbf{Rank} & \textbf{Word} & \multicolumn{2}{c|}{\textbf{Score}} \\
    \hline
    \endhead
    1 & insight & -1 & -1471 \\
    2 & shallowness & 0 & -8876 \\
    3 & concise\_jj & 0 & -7636 \\
    4 & negligence & 0 & -7212 \\
    5 & ungracious\_jj & 0 & -7088 \\
    6 & rude\_jj & 0 & -6959 \\
    7 & thorough\_jj & 0 & -6718 \\
    8 & economical\_jj & 0 & -6688 \\
    9 & impolite\_jj & 0 & -6271 \\
    10 & sloth & 0 & -5879 \\
    11 & prejudiced\_jj & 0 & -5625 \\
    12 & somber\_jj & 0 & -5474 \\
    13 & shallow\_jj & 0 & -5416 \\
    14 & detached\_jj & 0 & -5408 \\
    15 & deep\_jj & 0 & -5379 \\
    16 & belligerence & 0 & -5039 \\
    17 & friendly\_jj & 0 & -4838 \\
    18 & combative\_jj & 0 & -4674 \\
    19 & natural\_jj & 0 & -4647 \\
    20 & distrust & 0 & -4299 \\
    21 & egotistical\_jj & 0 & -4251 \\
    22 & insecurity & 0 & -4231 \\
    23 & dignified\_jj & 0 & -4218 \\
    24 & unsympathetic\_jj & 0 & -4129 \\
    25 & indecisive\_jj & 0 & -3996 \\
    26 & generous\_jj & 0 & -3992 \\
    27 & earthiness & 0 & -3911 \\
    28 & traditional\_jj & 0 & -3849 \\
    29 & inefficient\_jj & 0 & -3752 \\
    30 & fretful\_jj & 0 & -3741 \\
    392 & cooperation & 0 & 3645 \\
    393 & uninhibited\_jj & 0 & 3692 \\
    394 & sloppy\_jj & 0 & 3725 \\
    395 & vain\_jj & 0 & 3773 \\
    396 & unreliable\_jj & 0 & 3864 \\
    397 & truthful\_jj & 0 & 3914 \\
    398 & reserve & 0 & 3961 \\
    399 & envious\_jj & 0 & 4423 \\
    400 & inconsistent\_jj & 0 & 4538 \\
    401 & stubbornness & 0 & 4584 \\
    402 & wordy\_jj & 0 & 4728 \\
    403 & steady\_jj & 0 & 4743 \\
    404 & uncritical\_jj & 0 & 4898 \\
    405 & unscrupulous\_jj & 0 & 4980 \\
    406 & unkind\_jj & 0 & 5134 \\
    407 & erratic\_jj & 0 & 5247 \\
    408 & underhanded\_jj & 0 & 5257 \\
    409 & melancholic\_jj & 0 & 5293 \\
    410 & selfish\_jj & 0 & 5332 \\
    411 & lethargic\_jj & 0 & 5725 \\
    412 & candor & 0 & 5865 \\
    413 & communicative\_jj & 0 & 6173 \\
    414 & lazy\_jj & 0 & 6484 \\
    415 & mannerly\_jj & 0 & 6938 \\
    416 & cordial\_jj & 0 & 7243 \\
    417 & peaceful\_jj & 0 & 7344 \\
    418 & intelligence & 0 & 7504 \\
    419 & tempestuous\_jj & 0 & 7866 \\
    420 & independence & 0 & 8324 \\
    421 & refined\_jj & 1 & 2969 \\
    \hline
    \caption{Scores and rankings for most extreme 30 words in component \#45} \\
\end{longtable}
\begin{longtable}[!htbp]{| rlr@{.}l |}
    \hline
    \textbf{Rank} & \textbf{Word} & \multicolumn{2}{c|}{\textbf{Score}} \\
    \hline
    \endhead
    1 & inhibition & -1 & -28 \\
    2 & cunning\_jj & 0 & -7411 \\
    3 & peaceful\_jj & 0 & -7322 \\
    4 & wishy-washy\_jj & 0 & -7297 \\
    5 & orderly\_jj & 0 & -7225 \\
    6 & diplomatic\_jj & 0 & -7021 \\
    7 & extroverted\_jj & 0 & -6973 \\
    8 & defensive\_jj & 0 & -6784 \\
    9 & aimless\_jj & 0 & -6508 \\
    10 & earthiness & 0 & -6270 \\
    11 & self-disciplined\_jj & 0 & -6013 \\
    12 & refined\_jj & 0 & -5854 \\
    13 & mannerly\_jj & 0 & -5709 \\
    14 & lazy\_jj & 0 & -5599 \\
    15 & obstinate\_jj & 0 & -5528 \\
    16 & unsociable\_jj & 0 & -5359 \\
    17 & ungracious\_jj & 0 & -5140 \\
    18 & quiet\_jj & 0 & -4934 \\
    19 & inconsiderate\_jj & 0 & -4868 \\
    20 & indecisive\_jj & 0 & -4827 \\
    21 & optimism & 0 & -4800 \\
    22 & unintelligent\_jj & 0 & -4625 \\
    23 & unassuming\_jj & 0 & -4410 \\
    24 & patient\_jj & 0 & -4061 \\
    25 & insecurity & 0 & -4044 \\
    26 & bashful\_jj & 0 & -3997 \\
    27 & expressive\_jj & 0 & -3991 \\
    28 & compassionate\_jj & 0 & -3963 \\
    29 & creative\_jj & 0 & -3959 \\
    30 & crafty\_jj & 0 & -3907 \\
    392 & reasonable\_jj & 0 & 3887 \\
    393 & cordial\_jj & 0 & 3987 \\
    394 & proud\_jj & 0 & 3996 \\
    395 & impetuous\_jj & 0 & 4008 \\
    396 & volatile\_jj & 0 & 4045 \\
    397 & logical\_jj & 0 & 4055 \\
    398 & modesty & 0 & 4089 \\
    399 & intrusive\_jj & 0 & 4216 \\
    400 & expressiveness & 0 & 4365 \\
    401 & shallowness & 0 & 4469 \\
    402 & sophisticated\_jj & 0 & 4496 \\
    403 & foolhardy\_jj & 0 & 4507 \\
    404 & lethargy & 0 & 4568 \\
    405 & temperamental\_jj & 0 & 4616 \\
    406 & sincere\_jj & 0 & 4770 \\
    407 & autonomous\_jj & 0 & 4828 \\
    408 & jealous\_jj & 0 & 4925 \\
    409 & predictability & 0 & 5025 \\
    410 & naturalness & 0 & 5104 \\
    411 & demanding\_jj & 0 & 5358 \\
    412 & thorough\_jj & 0 & 5382 \\
    413 & exacting\_jj & 0 & 5532 \\
    414 & reserve & 0 & 5883 \\
    415 & thrift & 0 & 6087 \\
    416 & perceptive\_jj & 0 & 6298 \\
    417 & irritability & 0 & 6574 \\
    418 & unreflective\_jj & 0 & 6724 \\
    419 & unscrupulous\_jj & 0 & 6908 \\
    420 & dependability & 0 & 6956 \\
    421 & tempestuous\_jj & 0 & 8368 \\
    \hline
    \caption{Scores and rankings for most extreme 30 words in component \#46} \\
\end{longtable}
\begin{longtable}[!htbp]{| rlr@{.}l |}
    \hline
    \textbf{Rank} & \textbf{Word} & \multicolumn{2}{c|}{\textbf{Score}} \\
    \hline
    \endhead
    1 & erratic\_jj & 0 & -9460 \\
    2 & selfish\_jj & 0 & -9073 \\
    3 & independent\_jj & 0 & -8969 \\
    4 & irritability & 0 & -7052 \\
    5 & intelligence & 0 & -6951 \\
    6 & modesty & 0 & -6679 \\
    7 & friendly\_jj & 0 & -6623 \\
    8 & shallowness & 0 & -6606 \\
    9 & suspicious\_jj & 0 & -6397 \\
    10 & meddlesome\_jj & 0 & -6328 \\
    11 & docile\_jj & 0 & -6171 \\
    12 & ethical\_jj & 0 & -5663 \\
    13 & slothful\_jj & 0 & -5398 \\
    14 & fastidious\_jj & 0 & -5292 \\
    15 & rude\_jj & 0 & -5282 \\
    16 & argumentative\_jj & 0 & -5037 \\
    17 & assertive\_jj & 0 & -5010 \\
    18 & caution & 0 & -4999 \\
    19 & bullheaded\_jj & 0 & -4786 \\
    20 & intrusiveness & 0 & -4608 \\
    21 & unimaginative\_jj & 0 & -4573 \\
    22 & reckless\_jj & 0 & -4468 \\
    23 & honest\_jj & 0 & -4454 \\
    24 & peaceful\_jj & 0 & -4334 \\
    25 & frank\_jj & 0 & -4304 \\
    26 & earthy\_jj & 0 & -4274 \\
    27 & intrusive\_jj & 0 & -4141 \\
    28 & grumpy\_jj & 0 & -4007 \\
    29 & animation & 0 & -3796 \\
    30 & submissive\_jj & 0 & -3785 \\
    392 & lethargic\_jj & 0 & 3588 \\
    393 & individualistic\_jj & 0 & 3589 \\
    394 & demanding\_jj & 0 & 3623 \\
    395 & assured\_jj & 0 & 3626 \\
    396 & gregarious\_jj & 0 & 3802 \\
    397 & reserve & 0 & 3936 \\
    398 & sluggish\_jj & 0 & 3982 \\
    399 & steady\_jj & 0 & 3995 \\
    400 & inhibition & 0 & 4084 \\
    401 & cruelty & 0 & 4194 \\
    402 & negligent\_jj & 0 & 4205 \\
    403 & instability & 0 & 4205 \\
    404 & inventive\_jj & 0 & 4351 \\
    405 & volatility & 0 & 4407 \\
    406 & leniency & 0 & 4657 \\
    407 & generous\_jj & 0 & 4988 \\
    408 & jealous\_jj & 0 & 5100 \\
    409 & excitable\_jj & 0 & 5178 \\
    410 & crafty\_jj & 0 & 5196 \\
    411 & predictability & 0 & 5317 \\
    412 & distrustful\_jj & 0 & 5469 \\
    413 & charitable\_jj & 0 & 5722 \\
    414 & envious\_jj & 0 & 5746 \\
    415 & talkative\_jj & 0 & 6063 \\
    416 & understanding\_jj & 0 & 6210 \\
    417 & gullibility & 0 & 6216 \\
    418 & generosity & 0 & 6680 \\
    419 & envy & 0 & 6853 \\
    420 & touchy\_jj & 0 & 7084 \\
    421 & unemotional\_jj & 1 & 1356 \\
    \hline
    \caption{Scores and rankings for most extreme 30 words in component \#47} \\
\end{longtable}
\begin{longtable}[!htbp]{| rlr@{.}l |}
    \hline
    \textbf{Rank} & \textbf{Word} & \multicolumn{2}{c|}{\textbf{Score}} \\
    \hline
    \endhead
    1 & aimless\_jj & 0 & -8730 \\
    2 & intelligence & 0 & -7435 \\
    3 & underhanded\_jj & 0 & -6847 \\
    4 & unreflective\_jj & 0 & -6746 \\
    5 & intrusive\_jj & 0 & -6163 \\
    6 & thrift & 0 & -5810 \\
    7 & prompt\_jj & 0 & -5665 \\
    8 & frivolity & 0 & -5550 \\
    9 & frivolous\_jj & 0 & -5549 \\
    10 & unfriendly\_jj & 0 & -5486 \\
    11 & recklessness & 0 & -5424 \\
    12 & direct\_jj & 0 & -5087 \\
    13 & spontaneity & 0 & -5034 \\
    14 & obstinate\_jj & 0 & -5025 \\
    15 & shallowness & 0 & -4990 \\
    16 & vivacious\_jj & 0 & -4907 \\
    17 & concise\_jj & 0 & -4896 \\
    18 & verbose\_jj & 0 & -4696 \\
    19 & indecisive\_jj & 0 & -4656 \\
    20 & irritable\_jj & 0 & -4599 \\
    21 & demanding\_jj & 0 & -4536 \\
    22 & surly\_jj & 0 & -4320 \\
    23 & perceptive\_jj & 0 & -4288 \\
    24 & zestful\_jj & 0 & -4270 \\
    25 & friendly\_jj & 0 & -4154 \\
    26 & ethical\_jj & 0 & -4084 \\
    27 & impersonal\_jj & 0 & -4059 \\
    28 & adventurous\_jj & 0 & -4018 \\
    29 & cruelty & 0 & -4002 \\
    30 & happy-go-lucky\_jj & 0 & -3871 \\
    392 & greedy\_jj & 0 & 4299 \\
    393 & expressiveness & 0 & 4320 \\
    394 & unemotional\_jj & 0 & 4337 \\
    395 & indecisiveness & 0 & 4547 \\
    396 & intellectuality & 0 & 4565 \\
    397 & melancholic\_jj & 0 & 4597 \\
    398 & caution & 0 & 4605 \\
    399 & negligent\_jj & 0 & 4624 \\
    400 & wordy\_jj & 0 & 4659 \\
    401 & slothful\_jj & 0 & 4674 \\
    402 & unimaginative\_jj & 0 & 4686 \\
    403 & respectful\_jj & 0 & 4694 \\
    404 & reserve & 0 & 4712 \\
    405 & meticulous\_jj & 0 & 4728 \\
    406 & aloofness & 0 & 4916 \\
    407 & tactful\_jj & 0 & 4997 \\
    408 & quarrelsome\_jj & 0 & 5002 \\
    409 & aimlessness & 0 & 5005 \\
    410 & distrustful\_jj & 0 & 5108 \\
    411 & bitter\_jj & 0 & 5120 \\
    412 & miserly\_jj & 0 & 5337 \\
    413 & careful\_jj & 0 & 5416 \\
    414 & caustic\_jj & 0 & 5884 \\
    415 & merry\_jj & 0 & 6618 \\
    416 & leniency & 0 & 6666 \\
    417 & self-critical\_jj & 0 & 7125 \\
    418 & selfish\_jj & 0 & 7159 \\
    419 & autonomous\_jj & 0 & 7795 \\
    420 & animation & 0 & 7824 \\
    421 & thrifty\_jj & 0 & 8617 \\
    \hline
    \caption{Scores and rankings for most extreme 30 words in component \#48} \\
\end{longtable}
\begin{longtable}[!htbp]{| rlr@{.}l |}
    \hline
    \textbf{Rank} & \textbf{Word} & \multicolumn{2}{c|}{\textbf{Score}} \\
    \hline
    \endhead
    1 & surliness & 0 & -8528 \\
    2 & surly\_jj & 0 & -7906 \\
    3 & forgetful\_jj & 0 & -7764 \\
    4 & impolite\_jj & 0 & -6729 \\
    5 & quarrelsome\_jj & 0 & -6465 \\
    6 & disorganization & 0 & -6465 \\
    7 & thorough\_jj & 0 & -6389 \\
    8 & cosmopolitan\_jj & 0 & -6154 \\
    9 & rude\_jj & 0 & -6089 \\
    10 & unintelligent\_jj & 0 & -5889 \\
    11 & impudent\_jj & 0 & -5765 \\
    12 & rash\_jj & 0 & -5059 \\
    13 & friendly\_jj & 0 & -4967 \\
    14 & leniency & 0 & -4920 \\
    15 & instability & 0 & -4916 \\
    16 & emotionality & 0 & -4895 \\
    17 & mannerly\_jj & 0 & -4639 \\
    18 & erratic\_jj & 0 & -4571 \\
    19 & gullibility & 0 & -4403 \\
    20 & insight & 0 & -4383 \\
    21 & prejudiced\_jj & 0 & -4338 \\
    22 & reserve & 0 & -4316 \\
    23 & inconsiderate\_jj & 0 & -4285 \\
    24 & bigoted\_jj & 0 & -4230 \\
    25 & expressive\_jj & 0 & -4217 \\
    26 & stinginess & 0 & -4084 \\
    27 & passionless\_jj & 0 & -4061 \\
    28 & unsociable\_jj & 0 & -4054 \\
    29 & truthful\_jj & 0 & -4006 \\
    30 & daring\_jj & 0 & -3959 \\
    392 & crabby\_jj & 0 & 3999 \\
    393 & easygoing\_jj & 0 & 4119 \\
    394 & irritable\_jj & 0 & 4196 \\
    395 & gruff\_jj & 0 & 4219 \\
    396 & selfishness & 0 & 4258 \\
    397 & withdrawn\_jj & 0 & 4273 \\
    398 & miserly\_jj & 0 & 4337 \\
    399 & condescending\_jj & 0 & 4357 \\
    400 & unassuming\_jj & 0 & 4415 \\
    401 & predictability & 0 & 4522 \\
    402 & nonconforming\_jj & 0 & 4561 \\
    403 & uncritical\_jj & 0 & 4625 \\
    404 & intrusive\_jj & 0 & 4658 \\
    405 & inhibition & 0 & 4713 \\
    406 & lazy\_jj & 0 & 4716 \\
    407 & deceitful\_jj & 0 & 4852 \\
    408 & silence & 0 & 5039 \\
    409 & obstinate\_jj & 0 & 5509 \\
    410 & timid\_jj & 0 & 5589 \\
    411 & lethargy & 0 & 5808 \\
    412 & grumpy\_jj & 0 & 5823 \\
    413 & unambitious\_jj & 0 & 6004 \\
    414 & pomposity & 0 & 6162 \\
    415 & understanding\_jj & 0 & 6168 \\
    416 & deceit & 0 & 6499 \\
    417 & inefficient\_jj & 0 & 6675 \\
    418 & efficiency & 0 & 7009 \\
    419 & scornful\_jj & 0 & 7307 \\
    420 & belligerence & 0 & 7430 \\
    421 & animation & 0 & 7966 \\
    \hline
    \caption{Scores and rankings for most extreme 30 words in component \#49} \\
\end{longtable}
\begin{longtable}[!htbp]{| rlr@{.}l |}
    \hline
    \textbf{Rank} & \textbf{Word} & \multicolumn{2}{c|}{\textbf{Score}} \\
    \hline
    \endhead
    1 & adventurous\_jj & -1 & -1035 \\
    2 & economical\_jj & 0 & -8124 \\
    3 & mannerly\_jj & 0 & -7892 \\
    4 & thorough\_jj & 0 & -7578 \\
    5 & sincere\_jj & 0 & -6921 \\
    6 & warm\_jj & 0 & -6854 \\
    7 & compassionate\_jj & 0 & -5724 \\
    8 & withdrawn\_jj & 0 & -5691 \\
    9 & dependability & 0 & -5343 \\
    10 & shallow\_jj & 0 & -5113 \\
    11 & belligerence & 0 & -5103 \\
    12 & systematic\_jj & 0 & -5098 \\
    13 & happy-go-lucky\_jj & 0 & -5018 \\
    14 & crabby\_jj & 0 & -4990 \\
    15 & deep\_jj & 0 & -4938 \\
    16 & carefree\_jj & 0 & -4911 \\
    17 & uncritical\_jj & 0 & -4900 \\
    18 & rebellious\_jj & 0 & -4857 \\
    19 & sophistication & 0 & -4795 \\
    20 & intelligence & 0 & -4777 \\
    21 & miserly\_jj & 0 & -4772 \\
    22 & flippant\_jj & 0 & -4545 \\
    23 & diplomatic\_jj & 0 & -4374 \\
    24 & spontaneous\_jj & 0 & -4190 \\
    25 & haphazard\_jj & 0 & -4144 \\
    26 & self-esteem & 0 & -4082 \\
    27 & insensitive\_jj & 0 & -3978 \\
    28 & shyness & 0 & -3835 \\
    29 & underhanded\_jj & 0 & -3786 \\
    30 & aloofness & 0 & -3761 \\
    392 & deceit & 0 & 3433 \\
    393 & tactful\_jj & 0 & 3467 \\
    394 & friendly\_jj & 0 & 3480 \\
    395 & inquisitive\_jj & 0 & 3606 \\
    396 & dignity & 0 & 3630 \\
    397 & morose\_jj & 0 & 3643 \\
    398 & aimless\_jj & 0 & 3732 \\
    399 & courtesy & 0 & 3792 \\
    400 & argumentative\_jj & 0 & 3805 \\
    401 & gullible\_jj & 0 & 4003 \\
    402 & intellectual\_jj & 0 & 4174 \\
    403 & gregarious\_jj & 0 & 4217 \\
    404 & expressiveness & 0 & 4417 \\
    405 & rude\_jj & 0 & 4456 \\
    406 & intrusive\_jj & 0 & 4480 \\
    407 & adaptable\_jj & 0 & 4557 \\
    408 & silence & 0 & 4693 \\
    409 & courteous\_jj & 0 & 4746 \\
    410 & negligence & 0 & 4860 \\
    411 & cruelty & 0 & 6144 \\
    412 & quarrelsome\_jj & 0 & 6163 \\
    413 & talkative\_jj & 0 & 6336 \\
    414 & orderly\_jj & 0 & 6566 \\
    415 & modesty & 0 & 6679 \\
    416 & punctual\_jj & 0 & 7801 \\
    417 & volatility & 0 & 7861 \\
    418 & verbose\_jj & 0 & 7879 \\
    419 & cultured\_jj & 0 & 8337 \\
    420 & reserve & 0 & 8744 \\
    421 & thrift & 1 & 294 \\
    \hline
    \caption{Scores and rankings for most extreme 30 words in component \#50} \\
\end{longtable}
\begin{longtable}[!htbp]{| rlr@{.}l |}
    \hline
    \textbf{Rank} & \textbf{Word} & \multicolumn{2}{c|}{\textbf{Score}} \\
    \hline
    \endhead
    1 & folksy\_jj & -1 & -234 \\
    2 & autonomous\_jj & 0 & -8341 \\
    3 & tactless\_jj & 0 & -8119 \\
    4 & wishy-washy\_jj & 0 & -7321 \\
    5 & thoughtless\_jj & 0 & -6984 \\
    6 & understanding\_jj & 0 & -6673 \\
    7 & unambitious\_jj & 0 & -6669 \\
    8 & instability & 0 & -6197 \\
    9 & erratic\_jj & 0 & -5982 \\
    10 & flexibility & 0 & -5867 \\
    11 & wordy\_jj & 0 & -5823 \\
    12 & self-critical\_jj & 0 & -5764 \\
    13 & intrusiveness & 0 & -5266 \\
    14 & passivity & 0 & -5247 \\
    15 & cultured\_jj & 0 & -5167 \\
    16 & flippant\_jj & 0 & -4910 \\
    17 & selfish\_jj & 0 & -4899 \\
    18 & haphazard\_jj & 0 & -4605 \\
    19 & conceited\_jj & 0 & -4456 \\
    20 & brave\_jj & 0 & -4415 \\
    21 & ambitious\_jj & 0 & -4378 \\
    22 & prompt\_jj & 0 & -4302 \\
    23 & accommodating\_jj & 0 & -4143 \\
    24 & efficiency & 0 & -4107 \\
    25 & sophistication & 0 & -3878 \\
    26 & forgetful\_jj & 0 & -3769 \\
    27 & indecisive\_jj & 0 & -3756 \\
    28 & warm\_jj & 0 & -3500 \\
    29 & patient\_jj & 0 & -3457 \\
    30 & miserly\_jj & 0 & -3368 \\
    392 & worldly\_jj & 0 & 3848 \\
    393 & verbose\_jj & 0 & 3859 \\
    394 & inconsiderate\_jj & 0 & 3879 \\
    395 & adventurous\_jj & 0 & 3940 \\
    396 & envy & 0 & 3941 \\
    397 & courtesy & 0 & 3942 \\
    398 & lenient\_jj & 0 & 4015 \\
    399 & stubbornness & 0 & 4180 \\
    400 & deceit & 0 & 4248 \\
    401 & respectful\_jj & 0 & 4370 \\
    402 & explosive\_jj & 0 & 4476 \\
    403 & dependable\_jj & 0 & 4576 \\
    404 & logical\_jj & 0 & 4808 \\
    405 & aimless\_jj & 0 & 4827 \\
    406 & demonstrative\_jj & 0 & 4831 \\
    407 & artistic\_jj & 0 & 4855 \\
    408 & prejudice & 0 & 4925 \\
    409 & leniency & 0 & 5109 \\
    410 & inquisitive\_jj & 0 & 5419 \\
    411 & reserve & 0 & 5517 \\
    412 & lazy\_jj & 0 & 5681 \\
    413 & modesty & 0 & 5697 \\
    414 & shallow\_jj & 0 & 5991 \\
    415 & caustic\_jj & 0 & 6065 \\
    416 & punctual\_jj & 0 & 6092 \\
    417 & silent\_jj & 0 & 6827 \\
    418 & belligerence & 0 & 8113 \\
    419 & nonconforming\_jj & 0 & 8206 \\
    420 & economical\_jj & 1 & 1 \\
    421 & argumentative\_jj & 1 & 1617 \\
    \hline
    \caption{Scores and rankings for most extreme 30 words in component \#51} \\
\end{longtable}
\begin{longtable}[!htbp]{| rlr@{.}l |}
    \hline
    \textbf{Rank} & \textbf{Word} & \multicolumn{2}{c|}{\textbf{Score}} \\
    \hline
    \endhead
    1 & verbose\_jj & 0 & -9267 \\
    2 & defensive\_jj & 0 & -8033 \\
    3 & aimlessness & 0 & -7990 \\
    4 & independence & 0 & -7661 \\
    5 & erratic\_jj & 0 & -6653 \\
    6 & principled\_jj & 0 & -5899 \\
    7 & logical\_jj & 0 & -5701 \\
    8 & morose\_jj & 0 & -5234 \\
    9 & dependability & 0 & -5053 \\
    10 & organized\_jj & 0 & -4997 \\
    11 & bitter\_jj & 0 & -4947 \\
    12 & mannerly\_jj & 0 & -4826 \\
    13 & folksy\_jj & 0 & -4612 \\
    14 & depth & 0 & -4578 \\
    15 & disrespectful\_jj & 0 & -4379 \\
    16 & reasonable\_jj & 0 & -4338 \\
    17 & talkative\_jj & 0 & -4260 \\
    18 & spirited\_jj & 0 & -4229 \\
    19 & unscrupulous\_jj & 0 & -4222 \\
    20 & patient\_jj & 0 & -4217 \\
    21 & extroverted\_jj & 0 & -4158 \\
    22 & ungracious\_jj & 0 & -4144 \\
    23 & gruff\_jj & 0 & -4100 \\
    24 & predictability & 0 & -4065 \\
    25 & predictable\_jj & 0 & -3980 \\
    26 & surliness & 0 & -3929 \\
    27 & fastidious\_jj & 0 & -3903 \\
    28 & callousness & 0 & -3901 \\
    29 & expressiveness & 0 & -3849 \\
    30 & sophistication & 0 & -3682 \\
    392 & negligence & 0 & 3690 \\
    393 & impudent\_jj & 0 & 3712 \\
    394 & thrift & 0 & 3803 \\
    395 & egotistical\_jj & 0 & 3830 \\
    396 & stingy\_jj & 0 & 3860 \\
    397 & manipulative\_jj & 0 & 3893 \\
    398 & silent\_jj & 0 & 3895 \\
    399 & meditative\_jj & 0 & 3927 \\
    400 & earthiness & 0 & 4127 \\
    401 & timid\_jj & 0 & 4162 \\
    402 & vain\_jj & 0 & 4354 \\
    403 & cunning\_jj & 0 & 4486 \\
    404 & forgetfulness & 0 & 4582 \\
    405 & cunning & 0 & 4606 \\
    406 & quiet\_jj & 0 & 4609 \\
    407 & active\_jj & 0 & 4711 \\
    408 & self-esteem & 0 & 4830 \\
    409 & surly\_jj & 0 & 4946 \\
    410 & gullible\_jj & 0 & 4979 \\
    411 & caustic\_jj & 0 & 5477 \\
    412 & pomposity & 0 & 5644 \\
    413 & bossiness & 0 & 5647 \\
    414 & insight & 0 & 5782 \\
    415 & explosive\_jj & 0 & 5859 \\
    416 & rebellious\_jj & 0 & 6893 \\
    417 & volatile\_jj & 0 & 6905 \\
    418 & aloofness & 0 & 8247 \\
    419 & shallowness & 0 & 8930 \\
    420 & nonconformity & 1 & 140 \\
    421 & cordial\_jj & 1 & 167 \\
    \hline
    \caption{Scores and rankings for most extreme 30 words in component \#52} \\
\end{longtable}
\begin{longtable}[!htbp]{| rlr@{.}l |}
    \hline
    \textbf{Rank} & \textbf{Word} & \multicolumn{2}{c|}{\textbf{Score}} \\
    \hline
    \endhead
    1 & disorganization & 0 & -8166 \\
    2 & concise\_jj & 0 & -7435 \\
    3 & mannerly\_jj & 0 & -6891 \\
    4 & surly\_jj & 0 & -6736 \\
    5 & reserve & 0 & -6608 \\
    6 & condescending\_jj & 0 & -6431 \\
    7 & shallow\_jj & 0 & -6237 \\
    8 & shyness & 0 & -5679 \\
    9 & unconventional\_jj & 0 & -5560 \\
    10 & impudent\_jj & 0 & -5546 \\
    11 & irritable\_jj & 0 & -5354 \\
    12 & aloofness & 0 & -5353 \\
    13 & self-esteem & 0 & -5217 \\
    14 & friendly\_jj & 0 & -5197 \\
    15 & unreflective\_jj & 0 & -5194 \\
    16 & flippant\_jj & 0 & -5067 \\
    17 & silence & 0 & -4939 \\
    18 & talkative\_jj & 0 & -4645 \\
    19 & unscrupulous\_jj & 0 & -4626 \\
    20 & natural\_jj & 0 & -4545 \\
    21 & leniency & 0 & -4545 \\
    22 & intrusiveness & 0 & -4505 \\
    23 & callousness & 0 & -4362 \\
    24 & expressive\_jj & 0 & -4357 \\
    25 & passivity & 0 & -4344 \\
    26 & insecure\_jj & 0 & -4240 \\
    27 & unimaginative\_jj & 0 & -4217 \\
    28 & vain\_jj & 0 & -4194 \\
    29 & tactful\_jj & 0 & -4096 \\
    30 & selfish\_jj & 0 & -4069 \\
    392 & spontaneous\_jj & 0 & 3604 \\
    393 & stingy\_jj & 0 & 3617 \\
    394 & rebellious\_jj & 0 & 3825 \\
    395 & tactless\_jj & 0 & 3833 \\
    396 & unrestrained\_jj & 0 & 3884 \\
    397 & dependability & 0 & 4120 \\
    398 & spontaneity & 0 & 4172 \\
    399 & cosmopolitan\_jj & 0 & 4174 \\
    400 & foolhardy\_jj & 0 & 4221 \\
    401 & underhanded\_jj & 0 & 4345 \\
    402 & bitter\_jj & 0 & 4633 \\
    403 & deceitful\_jj & 0 & 4663 \\
    404 & verbal\_jj & 0 & 4703 \\
    405 & uncritical\_jj & 0 & 4732 \\
    406 & frivolous\_jj & 0 & 4844 \\
    407 & vindictive\_jj & 0 & 5018 \\
    408 & crabby\_jj & 0 & 5053 \\
    409 & unsociable\_jj & 0 & 5152 \\
    410 & thorough\_jj & 0 & 5180 \\
    411 & earthiness & 0 & 5198 \\
    412 & demanding\_jj & 0 & 5934 \\
    413 & insight & 0 & 6037 \\
    414 & recklessness & 0 & 6088 \\
    415 & understanding\_jj & 0 & 6168 \\
    416 & punctuality & 0 & 6203 \\
    417 & naïve\_jj & 0 & 6244 \\
    418 & irritability & 0 & 6463 \\
    419 & ungracious\_jj & 0 & 7814 \\
    420 & shallowness & 0 & 8565 \\
    421 & punctual\_jj & 0 & 9309 \\
    \hline
    \caption{Scores and rankings for most extreme 30 words in component \#53} \\
\end{longtable}
\begin{longtable}[!htbp]{| rlr@{.}l |}
    \hline
    \textbf{Rank} & \textbf{Word} & \multicolumn{2}{c|}{\textbf{Score}} \\
    \hline
    \endhead
    1 & intelligence & 0 & -9090 \\
    2 & selfish\_jj & 0 & -7587 \\
    3 & irritability & 0 & -7561 \\
    4 & unscrupulous\_jj & 0 & -7250 \\
    5 & reserve & 0 & -7147 \\
    6 & surliness & 0 & -7084 \\
    7 & punctuality & 0 & -6541 \\
    8 & compassionate\_jj & 0 & -6405 \\
    9 & brave\_jj & 0 & -5784 \\
    10 & artistic\_jj & 0 & -4857 \\
    11 & cooperation & 0 & -4752 \\
    12 & docile\_jj & 0 & -4731 \\
    13 & intellectual\_jj & 0 & -4692 \\
    14 & nonconforming\_jj & 0 & -4683 \\
    15 & defensive\_jj & 0 & -4637 \\
    16 & economical\_jj & 0 & -4636 \\
    17 & patient\_jj & 0 & -4619 \\
    18 & stingy\_jj & 0 & -4577 \\
    19 & rash\_jj & 0 & -4542 \\
    20 & sophistication & 0 & -4490 \\
    21 & unintelligent\_jj & 0 & -4454 \\
    22 & unreliable\_jj & 0 & -4440 \\
    23 & diplomatic\_jj & 0 & -4414 \\
    24 & modesty & 0 & -4407 \\
    25 & flamboyant\_jj & 0 & -4365 \\
    26 & analytical\_jj & 0 & -4360 \\
    27 & folksy\_jj & 0 & -4327 \\
    28 & happy-go-lucky\_jj & 0 & -4300 \\
    29 & underhanded\_jj & 0 & -4084 \\
    30 & reckless\_jj & 0 & -3959 \\
    392 & joyless\_jj & 0 & 3704 \\
    393 & assured\_jj & 0 & 3716 \\
    394 & self-pitying\_jj & 0 & 3799 \\
    395 & spontaneity & 0 & 3959 \\
    396 & friendly\_jj & 0 & 4013 \\
    397 & submissive\_jj & 0 & 4166 \\
    398 & argumentative\_jj & 0 & 4217 \\
    399 & spirit & 0 & 4237 \\
    400 & insight & 0 & 4298 \\
    401 & communicative\_jj & 0 & 4333 \\
    402 & impersonal\_jj & 0 & 4376 \\
    403 & flexible\_jj & 0 & 4439 \\
    404 & aloofness & 0 & 4565 \\
    405 & thorough\_jj & 0 & 4594 \\
    406 & independent\_jj & 0 & 4632 \\
    407 & deceit & 0 & 4703 \\
    408 & negligent\_jj & 0 & 4761 \\
    409 & charitable\_jj & 0 & 5000 \\
    410 & refined\_jj & 0 & 5010 \\
    411 & casual\_jj & 0 & 5077 \\
    412 & boastful\_jj & 0 & 5185 \\
    413 & lenient\_jj & 0 & 5559 \\
    414 & aimless\_jj & 0 & 5945 \\
    415 & organized\_jj & 0 & 5954 \\
    416 & self-disciplined\_jj & 0 & 5994 \\
    417 & lazy\_jj & 0 & 6222 \\
    418 & indecisiveness & 0 & 7404 \\
    419 & merry\_jj & 0 & 7463 \\
    420 & independence & 0 & 7607 \\
    421 & explosive\_jj & 0 & 8513 \\
    \hline
    \caption{Scores and rankings for most extreme 30 words in component \#54} \\
\end{longtable}
\begin{longtable}[!htbp]{| rlr@{.}l |}
    \hline
    \textbf{Rank} & \textbf{Word} & \multicolumn{2}{c|}{\textbf{Score}} \\
    \hline
    \endhead
    1 & punctual\_jj & 0 & -8749 \\
    2 & perceptive\_jj & 0 & -8108 \\
    3 & cruelty & 0 & -7879 \\
    4 & docile\_jj & 0 & -7538 \\
    5 & reserve & 0 & -6993 \\
    6 & self-esteem & 0 & -6816 \\
    7 & intelligence & 0 & -6262 \\
    8 & cosmopolitan\_jj & 0 & -6084 \\
    9 & disorganization & 0 & -5648 \\
    10 & pessimistic\_jj & 0 & -5519 \\
    11 & sloth & 0 & -5310 \\
    12 & punctuality & 0 & -5232 \\
    13 & lenient\_jj & 0 & -5088 \\
    14 & courteous\_jj & 0 & -4952 \\
    15 & naturalness & 0 & -4871 \\
    16 & down-to-earth\_jj & 0 & -4732 \\
    17 & joyless\_jj & 0 & -4646 \\
    18 & slothful\_jj & 0 & -4479 \\
    19 & unkind\_jj & 0 & -4376 \\
    20 & self-disciplined\_jj & 0 & -4337 \\
    21 & amiability & 0 & -4247 \\
    22 & miserly\_jj & 0 & -4197 \\
    23 & insecurity & 0 & -4190 \\
    24 & prompt\_jj & 0 & -4145 \\
    25 & prejudiced\_jj & 0 & -3959 \\
    26 & self-pitying\_jj & 0 & -3935 \\
    27 & cunning\_jj & 0 & -3934 \\
    28 & bigoted\_jj & 0 & -3929 \\
    29 & shallow\_jj & 0 & -3743 \\
    30 & modest\_jj & 0 & -3643 \\
    392 & temperamental\_jj & 0 & 3593 \\
    393 & moody\_jj & 0 & 3629 \\
    394 & skeptical\_jj & 0 & 3652 \\
    395 & wordy\_jj & 0 & 3770 \\
    396 & decisiveness & 0 & 3775 \\
    397 & fearful\_jj & 0 & 3782 \\
    398 & touchy\_jj & 0 & 3799 \\
    399 & stupidity & 0 & 3842 \\
    400 & expressiveness & 0 & 3921 \\
    401 & nonconforming\_jj & 0 & 4022 \\
    402 & considerate\_jj & 0 & 4043 \\
    403 & traditional\_jj & 0 & 4087 \\
    404 & argumentative\_jj & 0 & 4155 \\
    405 & truthful\_jj & 0 & 4235 \\
    406 & gullibility & 0 & 4344 \\
    407 & thrift & 0 & 4443 \\
    408 & recklessness & 0 & 4482 \\
    409 & detached\_jj & 0 & 4549 \\
    410 & anxious\_jj & 0 & 4895 \\
    411 & assertive\_jj & 0 & 4898 \\
    412 & moral\_jj & 0 & 5137 \\
    413 & nonconformity & 0 & 5224 \\
    414 & accommodating\_jj & 0 & 5292 \\
    415 & friendly\_jj & 0 & 5440 \\
    416 & silence & 0 & 5678 \\
    417 & careful\_jj & 0 & 6092 \\
    418 & quarrelsome\_jj & 0 & 6388 \\
    419 & passivity & 0 & 6915 \\
    420 & surly\_jj & 0 & 8135 \\
    421 & mannerly\_jj & 1 & 4884 \\
    \hline
    \caption{Scores and rankings for most extreme 30 words in component \#55} \\
\end{longtable}
\begin{longtable}[!htbp]{| rlr@{.}l |}
    \hline
    \textbf{Rank} & \textbf{Word} & \multicolumn{2}{c|}{\textbf{Score}} \\
    \hline
    \endhead
    1 & expressiveness & 0 & -7047 \\
    2 & self-critical\_jj & 0 & -7037 \\
    3 & erratic\_jj & 0 & -6801 \\
    4 & wordy\_jj & 0 & -6669 \\
    5 & insight & 0 & -6229 \\
    6 & independence & 0 & -6220 \\
    7 & recklessness & 0 & -6113 \\
    8 & refined\_jj & 0 & -5913 \\
    9 & irritable\_jj & 0 & -5751 \\
    10 & merry\_jj & 0 & -5740 \\
    11 & prompt\_jj & 0 & -5571 \\
    12 & easygoing\_jj & 0 & -5465 \\
    13 & modesty & 0 & -5370 \\
    14 & careful\_jj & 0 & -5101 \\
    15 & self-disciplined\_jj & 0 & -5023 \\
    16 & egotistical\_jj & 0 & -4975 \\
    17 & indecisiveness & 0 & -4955 \\
    18 & ambitious\_jj & 0 & -4879 \\
    19 & cruelty & 0 & -4762 \\
    20 & nonconformity & 0 & -4664 \\
    21 & aimlessness & 0 & -4585 \\
    22 & gullibility & 0 & -4519 \\
    23 & rude\_jj & 0 & -4497 \\
    24 & kind\_jj & 0 & -4392 \\
    25 & argumentative\_jj & 0 & -4365 \\
    26 & harsh\_jj & 0 & -4303 \\
    27 & obliging\_jj & 0 & -4264 \\
    28 & conceited\_jj & 0 & -4084 \\
    29 & uncritical\_jj & 0 & -4063 \\
    30 & flamboyant\_jj & 0 & -4053 \\
    392 & sophistication & 0 & 3897 \\
    393 & cultured\_jj & 0 & 4148 \\
    394 & scornful\_jj & 0 & 4225 \\
    395 & playfulness & 0 & 4302 \\
    396 & curiosity & 0 & 4309 \\
    397 & vindictive\_jj & 0 & 4482 \\
    398 & careless\_jj & 0 & 4509 \\
    399 & nosey\_jj & 0 & 4512 \\
    400 & quarrelsome\_jj & 0 & 4534 \\
    401 & inhibition & 0 & 4649 \\
    402 & passive\_jj & 0 & 4686 \\
    403 & communicative\_jj & 0 & 4853 \\
    404 & ungracious\_jj & 0 & 4889 \\
    405 & distrustful\_jj & 0 & 4947 \\
    406 & principled\_jj & 0 & 4954 \\
    407 & bossiness & 0 & 4963 \\
    408 & autonomous\_jj & 0 & 4976 \\
    409 & detached\_jj & 0 & 5214 \\
    410 & moody\_jj & 0 & 5225 \\
    411 & silence & 0 & 5262 \\
    412 & selfish\_jj & 0 & 5472 \\
    413 & spirited\_jj & 0 & 5494 \\
    414 & curt\_jj & 0 & 5558 \\
    415 & candor & 0 & 5706 \\
    416 & intrusiveness & 0 & 5775 \\
    417 & manipulative\_jj & 0 & 5917 \\
    418 & inconsiderate\_jj & 0 & 6178 \\
    419 & leniency & 0 & 6395 \\
    420 & dependability & 0 & 6903 \\
    421 & thrift & 0 & 8311 \\
    \hline
    \caption{Scores and rankings for most extreme 30 words in component \#56} \\
\end{longtable}
\begin{longtable}[!htbp]{| rlr@{.}l |}
    \hline
    \textbf{Rank} & \textbf{Word} & \multicolumn{2}{c|}{\textbf{Score}} \\
    \hline
    \endhead
    1 & sophistication & -1 & -3388 \\
    2 & intelligence & 0 & -8918 \\
    3 & impudent\_jj & 0 & -8570 \\
    4 & economical\_jj & 0 & -7272 \\
    5 & sloppy\_jj & 0 & -6133 \\
    6 & assured\_jj & 0 & -5908 \\
    7 & quiet & 0 & -5895 \\
    8 & cosmopolitan\_jj & 0 & -5428 \\
    9 & quiet\_jj & 0 & -5407 \\
    10 & extroverted\_jj & 0 & -5372 \\
    11 & self-critical\_jj & 0 & -5241 \\
    12 & individualistic\_jj & 0 & -5139 \\
    13 & argumentative\_jj & 0 & -5002 \\
    14 & zestful\_jj & 0 & -4910 \\
    15 & negligent\_jj & 0 & -4893 \\
    16 & careless\_jj & 0 & -4817 \\
    17 & smart\_jj & 0 & -4718 \\
    18 & compassionate\_jj & 0 & -4690 \\
    19 & nonconformity & 0 & -4588 \\
    20 & courtesy & 0 & -4460 \\
    21 & tenacious\_jj & 0 & -4292 \\
    22 & impetuous\_jj & 0 & -4118 \\
    23 & sophisticated\_jj & 0 & -3968 \\
    24 & depth & 0 & -3811 \\
    25 & boastful\_jj & 0 & -3706 \\
    26 & aimlessness & 0 & -3703 \\
    27 & volatility & 0 & -3698 \\
    28 & impersonal\_jj & 0 & -3693 \\
    29 & self-pitying\_jj & 0 & -3693 \\
    30 & cordial\_jj & 0 & -3599 \\
    392 & bullheaded\_jj & 0 & 3427 \\
    393 & slothful\_jj & 0 & 3438 \\
    394 & humor & 0 & 3514 \\
    395 & dependability & 0 & 3606 \\
    396 & forgetfulness & 0 & 3611 \\
    397 & sociable\_jj & 0 & 3614 \\
    398 & daring & 0 & 3650 \\
    399 & mannerly\_jj & 0 & 3802 \\
    400 & adventurous\_jj & 0 & 3854 \\
    401 & punctuality & 0 & 3867 \\
    402 & grumpy\_jj & 0 & 4021 \\
    403 & reserve & 0 & 4034 \\
    404 & earthiness & 0 & 4035 \\
    405 & undependable\_jj & 0 & 4069 \\
    406 & foolhardy\_jj & 0 & 4281 \\
    407 & brave\_jj & 0 & 4295 \\
    408 & irritable\_jj & 0 & 4395 \\
    409 & belligerence & 0 & 4580 \\
    410 & sluggish\_jj & 0 & 4633 \\
    411 & prejudiced\_jj & 0 & 4835 \\
    412 & quarrelsome\_jj & 0 & 5258 \\
    413 & indecisiveness & 0 & 5779 \\
    414 & cruelty & 0 & 6027 \\
    415 & aloofness & 0 & 6469 \\
    416 & wishy-washy\_jj & 0 & 6795 \\
    417 & surly\_jj & 0 & 6858 \\
    418 & unemotional\_jj & 0 & 7151 \\
    419 & pomposity & 0 & 7593 \\
    420 & unscrupulous\_jj & 0 & 8041 \\
    421 & surliness & 0 & 8647 \\
    \hline
    \caption{Scores and rankings for most extreme 30 words in component \#57} \\
\end{longtable}
\begin{longtable}[!htbp]{| rlr@{.}l |}
    \hline
    \textbf{Rank} & \textbf{Word} & \multicolumn{2}{c|}{\textbf{Score}} \\
    \hline
    \endhead
    1 & nonconforming\_jj & -1 & -122 \\
    2 & sophistication & 0 & -7176 \\
    3 & forgetful\_jj & 0 & -6661 \\
    4 & perceptive\_jj & 0 & -6576 \\
    5 & friendly\_jj & 0 & -6293 \\
    6 & disorganization & 0 & -6098 \\
    7 & decisiveness & 0 & -5759 \\
    8 & quarrelsome\_jj & 0 & -5669 \\
    9 & unscrupulous\_jj & 0 & -5597 \\
    10 & generosity & 0 & -5571 \\
    11 & charitable\_jj & 0 & -5542 \\
    12 & folksy\_jj & 0 & -5482 \\
    13 & rude\_jj & 0 & -5286 \\
    14 & punctuality & 0 & -5151 \\
    15 & sloth & 0 & -4893 \\
    16 & aimless\_jj & 0 & -4703 \\
    17 & uncritical\_jj & 0 & -4464 \\
    18 & conceit & 0 & -4451 \\
    19 & understanding\_jj & 0 & -4419 \\
    20 & courage & 0 & -4167 \\
    21 & impersonal\_jj & 0 & -4110 \\
    22 & negligent\_jj & 0 & -4043 \\
    23 & cunning & 0 & -3966 \\
    24 & reckless\_jj & 0 & -3905 \\
    25 & timid\_jj & 0 & -3875 \\
    26 & caustic\_jj & 0 & -3777 \\
    27 & tactless\_jj & 0 & -3743 \\
    28 & earthiness & 0 & -3577 \\
    29 & organization & 0 & -3559 \\
    30 & reasonable\_jj & 0 & -3554 \\
    392 & self-disciplined\_jj & 0 & 3687 \\
    393 & submissive\_jj & 0 & 3740 \\
    394 & deceit & 0 & 3782 \\
    395 & depth & 0 & 3802 \\
    396 & bullheaded\_jj & 0 & 3972 \\
    397 & passionless\_jj & 0 & 3973 \\
    398 & merry\_jj & 0 & 4029 \\
    399 & systematic\_jj & 0 & 4030 \\
    400 & envy & 0 & 4206 \\
    401 & proud\_jj & 0 & 4241 \\
    402 & lenient\_jj & 0 & 4457 \\
    403 & prompt\_jj & 0 & 4465 \\
    404 & traditional\_jj & 0 & 4588 \\
    405 & insight & 0 & 4660 \\
    406 & accommodating\_jj & 0 & 4686 \\
    407 & animation & 0 & 4844 \\
    408 & dependability & 0 & 4856 \\
    409 & touchy\_jj & 0 & 5124 \\
    410 & gregarious\_jj & 0 & 5204 \\
    411 & vivacious\_jj & 0 & 5207 \\
    412 & thrift & 0 & 5302 \\
    413 & demanding\_jj & 0 & 5395 \\
    414 & intrusive\_jj & 0 & 5404 \\
    415 & inconsistency & 0 & 5542 \\
    416 & silence & 0 & 5915 \\
    417 & underhanded\_jj & 0 & 5933 \\
    418 & surliness & 0 & 6246 \\
    419 & negligence & 0 & 6908 \\
    420 & bright\_jj & 0 & 7016 \\
    421 & inconsiderate\_jj & 0 & 9171 \\
    \hline
    \caption{Scores and rankings for most extreme 30 words in component \#58} \\
\end{longtable}
\begin{longtable}[!htbp]{| rlr@{.}l |}
    \hline
    \textbf{Rank} & \textbf{Word} & \multicolumn{2}{c|}{\textbf{Score}} \\
    \hline
    \endhead
    1 & understanding\_jj & -1 & -577 \\
    2 & inconsiderate\_jj & 0 & -9117 \\
    3 & irritable\_jj & 0 & -8840 \\
    4 & steady\_jj & 0 & -8673 \\
    5 & animation & 0 & -6339 \\
    6 & folksy\_jj & 0 & -5942 \\
    7 & reserve & 0 & -5770 \\
    8 & stinginess & 0 & -5519 \\
    9 & compassionate\_jj & 0 & -5310 \\
    10 & morose\_jj & 0 & -5092 \\
    11 & assured\_jj & 0 & -4767 \\
    12 & adaptable\_jj & 0 & -4673 \\
    13 & prompt\_jj & 0 & -4570 \\
    14 & cruel\_jj & 0 & -4569 \\
    15 & ethical\_jj & 0 & -4477 \\
    16 & impolite\_jj & 0 & -4467 \\
    17 & antagonistic\_jj & 0 & -4332 \\
    18 & friendly\_jj & 0 & -4229 \\
    19 & meticulous\_jj & 0 & -4190 \\
    20 & mannerly\_jj & 0 & -4059 \\
    21 & aloofness & 0 & -3967 \\
    22 & orderly\_jj & 0 & -3921 \\
    23 & unstable\_jj & 0 & -3910 \\
    24 & stingy\_jj & 0 & -3903 \\
    25 & devious\_jj & 0 & -3805 \\
    26 & exacting\_jj & 0 & -3764 \\
    27 & inquisitive\_jj & 0 & -3716 \\
    28 & shallowness & 0 & -3689 \\
    29 & careless\_jj & 0 & -3545 \\
    30 & homespun\_jj & 0 & -3471 \\
    392 & submissive\_jj & 0 & 3687 \\
    393 & warm\_jj & 0 & 3724 \\
    394 & self-esteem & 0 & 3741 \\
    395 & bossy\_jj & 0 & 3772 \\
    396 & intrusiveness & 0 & 3865 \\
    397 & caustic\_jj & 0 & 3892 \\
    398 & verbal\_jj & 0 & 3940 \\
    399 & wishy-washy\_jj & 0 & 4041 \\
    400 & artistic\_jj & 0 & 4117 \\
    401 & self-critical\_jj & 0 & 4201 \\
    402 & shallow\_jj & 0 & 4233 \\
    403 & charitable\_jj & 0 & 4247 \\
    404 & nonconforming\_jj & 0 & 4273 \\
    405 & prejudiced\_jj & 0 & 4282 \\
    406 & expressive\_jj & 0 & 4323 \\
    407 & slothful\_jj & 0 & 4426 \\
    408 & thoughtless\_jj & 0 & 4577 \\
    409 & expressiveness & 0 & 4661 \\
    410 & wordy\_jj & 0 & 4828 \\
    411 & generous\_jj & 0 & 4830 \\
    412 & surly\_jj & 0 & 5074 \\
    413 & cunning\_jj & 0 & 5085 \\
    414 & economical\_jj & 0 & 5362 \\
    415 & lenient\_jj & 0 & 5466 \\
    416 & rebellious\_jj & 0 & 5545 \\
    417 & vivacious\_jj & 0 & 5890 \\
    418 & self-disciplined\_jj & 0 & 5986 \\
    419 & suspicious\_jj & 0 & 6471 \\
    420 & efficiency & 0 & 6994 \\
    421 & efficient\_jj & 0 & 7685 \\
    \hline
    \caption{Scores and rankings for most extreme 30 words in component \#59} \\
\end{longtable}
\begin{longtable}[!htbp]{| rlr@{.}l |}
    \hline
    \textbf{Rank} & \textbf{Word} & \multicolumn{2}{c|}{\textbf{Score}} \\
    \hline
    \endhead
    1 & zestful\_jj & 0 & -9822 \\
    2 & shallowness & 0 & -8323 \\
    3 & courteous\_jj & 0 & -6361 \\
    4 & fretful\_jj & 0 & -6261 \\
    5 & charitable\_jj & 0 & -6169 \\
    6 & steady\_jj & 0 & -5964 \\
    7 & surliness & 0 & -5859 \\
    8 & shallow\_jj & 0 & -5572 \\
    9 & sluggish\_jj & 0 & -5509 \\
    10 & rudeness & 0 & -5503 \\
    11 & expressive\_jj & 0 & -5066 \\
    12 & talkative\_jj & 0 & -5037 \\
    13 & autonomous\_jj & 0 & -5020 \\
    14 & negligent\_jj & 0 & -4901 \\
    15 & caustic\_jj & 0 & -4817 \\
    16 & instability & 0 & -4721 \\
    17 & belligerence & 0 & -4580 \\
    18 & unintelligent\_jj & 0 & -4540 \\
    19 & flexible\_jj & 0 & -4445 \\
    20 & communicative\_jj & 0 & -4439 \\
    21 & reserved\_jj & 0 & -4393 \\
    22 & unrestrained\_jj & 0 & -4385 \\
    23 & conceited\_jj & 0 & -4271 \\
    24 & morose\_jj & 0 & -4155 \\
    25 & boastful\_jj & 0 & -4130 \\
    26 & insight & 0 & -4094 \\
    27 & detached\_jj & 0 & -4075 \\
    28 & careless\_jj & 0 & -3922 \\
    29 & stingy\_jj & 0 & -3871 \\
    30 & excitable\_jj & 0 & -3860 \\
    392 & easygoing\_jj & 0 & 3549 \\
    393 & sociable\_jj & 0 & 3634 \\
    394 & deliberate\_jj & 0 & 3655 \\
    395 & rebellious\_jj & 0 & 3662 \\
    396 & punctuality & 0 & 3697 \\
    397 & unconventional\_jj & 0 & 3715 \\
    398 & mannerly\_jj & 0 & 3749 \\
    399 & vivacious\_jj & 0 & 4029 \\
    400 & nonconformity & 0 & 4058 \\
    401 & predictability & 0 & 4077 \\
    402 & gregarious\_jj & 0 & 4089 \\
    403 & courtesy & 0 & 4330 \\
    404 & contemplative\_jj & 0 & 4410 \\
    405 & argumentative\_jj & 0 & 4438 \\
    406 & harsh\_jj & 0 & 4748 \\
    407 & joyless\_jj & 0 & 4856 \\
    408 & bitter\_jj & 0 & 5006 \\
    409 & earthiness & 0 & 5055 \\
    410 & volatility & 0 & 5061 \\
    411 & cunning\_jj & 0 & 5347 \\
    412 & undemanding\_jj & 0 & 5352 \\
    413 & inhibition & 0 & 5399 \\
    414 & cold\_jj & 0 & 5445 \\
    415 & recklessness & 0 & 5896 \\
    416 & irritable\_jj & 0 & 6483 \\
    417 & distrustful\_jj & 0 & 7042 \\
    418 & dependability & 0 & 7053 \\
    419 & merry\_jj & 0 & 7253 \\
    420 & unreflective\_jj & 0 & 8141 \\
    421 & reserve & 0 & 9244 \\
    \hline
    \caption{Scores and rankings for most extreme 30 words in component \#60} \\
\end{longtable}
\begin{longtable}[!htbp]{| rlr@{.}l |}
    \hline
    \textbf{Rank} & \textbf{Word} & \multicolumn{2}{c|}{\textbf{Score}} \\
    \hline
    \endhead
    1 & detached\_jj & -1 & -2767 \\
    2 & surly\_jj & 0 & -7582 \\
    3 & volatility & 0 & -6710 \\
    4 & self-critical\_jj & 0 & -6190 \\
    5 & depth & 0 & -6086 \\
    6 & volatile\_jj & 0 & -5630 \\
    7 & inventive\_jj & 0 & -5333 \\
    8 & naturalness & 0 & -5029 \\
    9 & folksy\_jj & 0 & -4838 \\
    10 & modest\_jj & 0 & -4827 \\
    11 & independence & 0 & -4819 \\
    12 & impolite\_jj & 0 & -4805 \\
    13 & timid\_jj & 0 & -4673 \\
    14 & truthful\_jj & 0 & -4580 \\
    15 & obstinate\_jj & 0 & -4547 \\
    16 & expressiveness & 0 & -4536 \\
    17 & docile\_jj & 0 & -4397 \\
    18 & brave\_jj & 0 & -4381 \\
    19 & rudeness & 0 & -4328 \\
    20 & inhibition & 0 & -4263 \\
    21 & distrustful\_jj & 0 & -4241 \\
    22 & careless\_jj & 0 & -3968 \\
    23 & thorough\_jj & 0 & -3960 \\
    24 & sophistication & 0 & -3925 \\
    25 & explosive\_jj & 0 & -3889 \\
    26 & consistent\_jj & 0 & -3728 \\
    27 & deliberate\_jj & 0 & -3525 \\
    28 & peaceful\_jj & 0 & -3517 \\
    29 & prompt\_jj & 0 & -3474 \\
    30 & negligence & 0 & -3463 \\
    392 & crabby\_jj & 0 & 3595 \\
    393 & optimistic\_jj & 0 & 3595 \\
    394 & suspicious\_jj & 0 & 3663 \\
    395 & philosophical\_jj & 0 & 3674 \\
    396 & opportunistic\_jj & 0 & 3888 \\
    397 & emotionality & 0 & 3896 \\
    398 & unconventional\_jj & 0 & 4043 \\
    399 & forgetfulness & 0 & 4160 \\
    400 & unfriendly\_jj & 0 & 4214 \\
    401 & pomposity & 0 & 4310 \\
    402 & frivolity & 0 & 4386 \\
    403 & ethical\_jj & 0 & 4520 \\
    404 & rash\_jj & 0 & 4548 \\
    405 & self-disciplined\_jj & 0 & 4641 \\
    406 & bright\_jj & 0 & 4707 \\
    407 & friendly\_jj & 0 & 4748 \\
    408 & rude\_jj & 0 & 4749 \\
    409 & sloppy\_jj & 0 & 4765 \\
    410 & impersonal\_jj & 0 & 5011 \\
    411 & somber\_jj & 0 & 5088 \\
    412 & moral\_jj & 0 & 5170 \\
    413 & warm\_jj & 0 & 5558 \\
    414 & unstable\_jj & 0 & 5702 \\
    415 & intellectual\_jj & 0 & 5874 \\
    416 & unintelligent\_jj & 0 & 5914 \\
    417 & self-pitying\_jj & 0 & 6143 \\
    418 & shallowness & 0 & 6562 \\
    419 & analytical\_jj & 0 & 7614 \\
    420 & disorganization & 0 & 7857 \\
    421 & verbal\_jj & 0 & 9019 \\
    \hline
    \caption{Scores and rankings for most extreme 30 words in component \#61} \\
\end{longtable}
\begin{longtable}[!htbp]{| rlr@{.}l |}
    \hline
    \textbf{Rank} & \textbf{Word} & \multicolumn{2}{c|}{\textbf{Score}} \\
    \hline
    \endhead
    1 & thrift & -1 & -285 \\
    2 & compassionate\_jj & 0 & -8590 \\
    3 & self-critical\_jj & 0 & -8214 \\
    4 & disorganization & 0 & -7799 \\
    5 & economical\_jj & 0 & -7274 \\
    6 & rebellious\_jj & 0 & -6715 \\
    7 & adventurous\_jj & 0 & -6500 \\
    8 & volatility & 0 & -6195 \\
    9 & detached\_jj & 0 & -6165 \\
    10 & volatile\_jj & 0 & -5541 \\
    11 & excitable\_jj & 0 & -5300 \\
    12 & zestful\_jj & 0 & -5144 \\
    13 & crabby\_jj & 0 & -5086 \\
    14 & friendly\_jj & 0 & -4939 \\
    15 & wordy\_jj & 0 & -4683 \\
    16 & conceited\_jj & 0 & -4681 \\
    17 & suspicious\_jj & 0 & -4666 \\
    18 & underhanded\_jj & 0 & -4599 \\
    19 & undemanding\_jj & 0 & -4423 \\
    20 & rudeness & 0 & -4333 \\
    21 & expressiveness & 0 & -4295 \\
    22 & assertive\_jj & 0 & -4266 \\
    23 & reserve & 0 & -4265 \\
    24 & amiability & 0 & -4028 \\
    25 & warm\_jj & 0 & -3990 \\
    26 & reserved\_jj & 0 & -3856 \\
    27 & perceptive\_jj & 0 & -3774 \\
    28 & recklessness & 0 & -3726 \\
    29 & ungracious\_jj & 0 & -3589 \\
    30 & ethical\_jj & 0 & -3553 \\
    392 & melancholic\_jj & 0 & 3415 \\
    393 & submissive\_jj & 0 & 3445 \\
    394 & spontaneous\_jj & 0 & 3522 \\
    395 & meddlesome\_jj & 0 & 3573 \\
    396 & formal\_jj & 0 & 3635 \\
    397 & wishy-washy\_jj & 0 & 3656 \\
    398 & reckless\_jj & 0 & 3678 \\
    399 & unreliable\_jj & 0 & 3729 \\
    400 & folksy\_jj & 0 & 3788 \\
    401 & passionless\_jj & 0 & 3816 \\
    402 & irritability & 0 & 3889 \\
    403 & efficiency & 0 & 3894 \\
    404 & dominant\_jj & 0 & 3910 \\
    405 & nonconformity & 0 & 4018 \\
    406 & somber\_jj & 0 & 4077 \\
    407 & explosive\_jj & 0 & 4314 \\
    408 & merry\_jj & 0 & 4460 \\
    409 & inhibition & 0 & 4694 \\
    410 & forgetful\_jj & 0 & 4710 \\
    411 & gregarious\_jj & 0 & 4876 \\
    412 & sluggish\_jj & 0 & 4981 \\
    413 & inefficient\_jj & 0 & 5088 \\
    414 & unscrupulous\_jj & 0 & 5878 \\
    415 & modesty & 0 & 6509 \\
    416 & traditional\_jj & 0 & 6535 \\
    417 & worldly\_jj & 0 & 6864 \\
    418 & careless\_jj & 0 & 6993 \\
    419 & indecisive\_jj & 0 & 8413 \\
    420 & insight & 0 & 8931 \\
    421 & unreflective\_jj & 0 & 9128 \\
    \hline
    \caption{Scores and rankings for most extreme 30 words in component \#62} \\
\end{longtable}
\begin{longtable}[!htbp]{| rlr@{.}l |}
    \hline
    \textbf{Rank} & \textbf{Word} & \multicolumn{2}{c|}{\textbf{Score}} \\
    \hline
    \endhead
    1 & caustic\_jj & 0 & -7632 \\
    2 & intelligence & 0 & -6625 \\
    3 & self-critical\_jj & 0 & -6310 \\
    4 & pomposity & 0 & -6072 \\
    5 & defensive\_jj & 0 & -5653 \\
    6 & flippant\_jj & 0 & -5306 \\
    7 & instability & 0 & -5138 \\
    8 & thrifty\_jj & 0 & -5111 \\
    9 & animation & 0 & -4903 \\
    10 & assured\_jj & 0 & -4819 \\
    11 & efficient\_jj & 0 & -4798 \\
    12 & reserve & 0 & -4757 \\
    13 & thrift & 0 & -4647 \\
    14 & obliging\_jj & 0 & -4262 \\
    15 & uncritical\_jj & 0 & -4255 \\
    16 & sincere\_jj & 0 & -4227 \\
    17 & demonstrative\_jj & 0 & -4136 \\
    18 & melancholic\_jj & 0 & -4092 \\
    19 & meditative\_jj & 0 & -4024 \\
    20 & dependability & 0 & -3998 \\
    21 & unrestrained\_jj & 0 & -3899 \\
    22 & callousness & 0 & -3878 \\
    23 & impolite\_jj & 0 & -3818 \\
    24 & sociable\_jj & 0 & -3667 \\
    25 & anxious\_jj & 0 & -3636 \\
    26 & prejudice & 0 & -3635 \\
    27 & harsh\_jj & 0 & -3550 \\
    28 & suggestible\_jj & 0 & -3545 \\
    29 & honest\_jj & 0 & -3521 \\
    30 & careless\_jj & 0 & -3505 \\
    392 & detached\_jj & 0 & 3970 \\
    393 & vigorous\_jj & 0 & 3984 \\
    394 & envy & 0 & 3987 \\
    395 & inquisitive\_jj & 0 & 4058 \\
    396 & disorganization & 0 & 4103 \\
    397 & opportunistic\_jj & 0 & 4139 \\
    398 & deceit & 0 & 4142 \\
    399 & extravagant\_jj & 0 & 4191 \\
    400 & insensitive\_jj & 0 & 4205 \\
    401 & economical\_jj & 0 & 4221 \\
    402 & vivacious\_jj & 0 & 4249 \\
    403 & impractical\_jj & 0 & 4268 \\
    404 & formal\_jj & 0 & 4309 \\
    405 & aimlessness & 0 & 4554 \\
    406 & organized\_jj & 0 & 4556 \\
    407 & compassionate\_jj & 0 & 4600 \\
    408 & understanding\_jj & 0 & 4675 \\
    409 & insight & 0 & 4695 \\
    410 & caution & 0 & 4899 \\
    411 & self-disciplined\_jj & 0 & 4970 \\
    412 & courtesy & 0 & 5158 \\
    413 & pessimism & 0 & 5361 \\
    414 & indecisive\_jj & 0 & 5477 \\
    415 & unreflective\_jj & 0 & 5547 \\
    416 & scornful\_jj & 0 & 5578 \\
    417 & leniency & 0 & 5601 \\
    418 & touchy\_jj & 0 & 6415 \\
    419 & bullheaded\_jj & 0 & 6435 \\
    420 & independence & 0 & 7003 \\
    421 & bright\_jj & 0 & 8596 \\
    \hline
    \caption{Scores and rankings for most extreme 30 words in component \#63} \\
\end{longtable}
\begin{longtable}[!htbp]{| rlr@{.}l |}
    \hline
    \textbf{Rank} & \textbf{Word} & \multicolumn{2}{c|}{\textbf{Score}} \\
    \hline
    \endhead
    1 & disorganization & 0 & -7451 \\
    2 & predictability & 0 & -6194 \\
    3 & dependability & 0 & -5830 \\
    4 & surly\_jj & 0 & -5613 \\
    5 & crabby\_jj & 0 & -5603 \\
    6 & worldly\_jj & 0 & -5303 \\
    7 & withdrawn\_jj & 0 & -5249 \\
    8 & vain\_jj & 0 & -5149 \\
    9 & volatility & 0 & -5109 \\
    10 & unimaginative\_jj & 0 & -5005 \\
    11 & frivolous\_jj & 0 & -4991 \\
    12 & casual\_jj & 0 & -4887 \\
    13 & cordial\_jj & 0 & -4646 \\
    14 & cunning\_jj & 0 & -4630 \\
    15 & mannerly\_jj & 0 & -4490 \\
    16 & shallowness & 0 & -4421 \\
    17 & negligence & 0 & -4376 \\
    18 & indecisive\_jj & 0 & -4355 \\
    19 & unrestrained\_jj & 0 & -4280 \\
    20 & enterprising\_jj & 0 & -4222 \\
    21 & careless\_jj & 0 & -4168 \\
    22 & submissive\_jj & 0 & -4142 \\
    23 & adventurous\_jj & 0 & -4116 \\
    24 & traditional\_jj & 0 & -4037 \\
    25 & modesty & 0 & -3986 \\
    26 & animation & 0 & -3971 \\
    27 & somber\_jj & 0 & -3875 \\
    28 & intelligence & 0 & -3815 \\
    29 & indecisiveness & 0 & -3804 \\
    30 & bullheaded\_jj & 0 & -3726 \\
    392 & vindictive\_jj & 0 & 3742 \\
    393 & ethical\_jj & 0 & 3765 \\
    394 & curt\_jj & 0 & 3890 \\
    395 & aloofness & 0 & 3895 \\
    396 & lethargic\_jj & 0 & 3904 \\
    397 & crafty\_jj & 0 & 3985 \\
    398 & cosmopolitan\_jj & 0 & 4001 \\
    399 & belligerence & 0 & 4020 \\
    400 & tempestuous\_jj & 0 & 4137 \\
    401 & passivity & 0 & 4154 \\
    402 & exacting\_jj & 0 & 4256 \\
    403 & sincere\_jj & 0 & 4358 \\
    404 & cunning & 0 & 4403 \\
    405 & autonomous\_jj & 0 & 4418 \\
    406 & obliging\_jj & 0 & 4454 \\
    407 & self-esteem & 0 & 4502 \\
    408 & impetuous\_jj & 0 & 4607 \\
    409 & detached\_jj & 0 & 4643 \\
    410 & lenient\_jj & 0 & 5014 \\
    411 & zestful\_jj & 0 & 5804 \\
    412 & independence & 0 & 6004 \\
    413 & verbose\_jj & 0 & 6031 \\
    414 & friendly\_jj & 0 & 6136 \\
    415 & negligent\_jj & 0 & 6270 \\
    416 & rebellious\_jj & 0 & 6520 \\
    417 & sophistication & 0 & 6820 \\
    418 & inhibition & 0 & 7179 \\
    419 & slothful\_jj & 0 & 7437 \\
    420 & surliness & 0 & 7569 \\
    421 & unreflective\_jj & 0 & 9880 \\
    \hline
    \caption{Scores and rankings for most extreme 30 words in component \#64} \\
\end{longtable}
\begin{longtable}[!htbp]{| rlr@{.}l |}
    \hline
    \textbf{Rank} & \textbf{Word} & \multicolumn{2}{c|}{\textbf{Score}} \\
    \hline
    \endhead
    1 & cruelty & 0 & -8723 \\
    2 & independence & 0 & -8489 \\
    3 & boastful\_jj & 0 & -7810 \\
    4 & somber\_jj & 0 & -6832 \\
    5 & verbose\_jj & 0 & -6506 \\
    6 & friendly\_jj & 0 & -6090 \\
    7 & forgetfulness & 0 & -5980 \\
    8 & fastidious\_jj & 0 & -5784 \\
    9 & self-critical\_jj & 0 & -5264 \\
    10 & deceit & 0 & -5004 \\
    11 & deceitful\_jj & 0 & -4881 \\
    12 & submissive\_jj & 0 & -4508 \\
    13 & recklessness & 0 & -4496 \\
    14 & skeptical\_jj & 0 & -4434 \\
    15 & cosmopolitan\_jj & 0 & -4388 \\
    16 & reckless\_jj & 0 & -4354 \\
    17 & bright\_jj & 0 & -4353 \\
    18 & systematic\_jj & 0 & -4296 \\
    19 & organization & 0 & -4287 \\
    20 & self-disciplined\_jj & 0 & -4205 \\
    21 & bigoted\_jj & 0 & -4126 \\
    22 & shallowness & 0 & -3956 \\
    23 & pleasant\_jj & 0 & -3773 \\
    24 & careless\_jj & 0 & -3679 \\
    25 & inquisitive\_jj & 0 & -3643 \\
    26 & humorous\_jj & 0 & -3577 \\
    27 & generosity & 0 & -3503 \\
    28 & unassuming\_jj & 0 & -3440 \\
    29 & instability & 0 & -3439 \\
    30 & playful\_jj & 0 & -3436 \\
    392 & rude\_jj & 0 & 3709 \\
    393 & courage & 0 & 3743 \\
    394 & refined\_jj & 0 & 3761 \\
    395 & candor & 0 & 3774 \\
    396 & curt\_jj & 0 & 3825 \\
    397 & underhanded\_jj & 0 & 3917 \\
    398 & kind\_jj & 0 & 3926 \\
    399 & rebellious\_jj & 0 & 4000 \\
    400 & cunning & 0 & 4003 \\
    401 & punctual\_jj & 0 & 4049 \\
    402 & lenient\_jj & 0 & 4102 \\
    403 & aimlessness & 0 & 4180 \\
    404 & selfish\_jj & 0 & 4220 \\
    405 & conceited\_jj & 0 & 4303 \\
    406 & obstinate\_jj & 0 & 4384 \\
    407 & mannerly\_jj & 0 & 4478 \\
    408 & verbal\_jj & 0 & 4548 \\
    409 & impractical\_jj & 0 & 4803 \\
    410 & animation & 0 & 4825 \\
    411 & self-pitying\_jj & 0 & 5149 \\
    412 & reserve & 0 & 5181 \\
    413 & nonconformity & 0 & 5416 \\
    414 & compassionate\_jj & 0 & 5460 \\
    415 & silence & 0 & 5513 \\
    416 & frivolous\_jj & 0 & 5730 \\
    417 & frivolity & 0 & 5759 \\
    418 & negligent\_jj & 0 & 6068 \\
    419 & perceptive\_jj & 0 & 6100 \\
    420 & unscrupulous\_jj & 0 & 6274 \\
    421 & naturalness & 0 & 6438 \\
    \hline
    \caption{Scores and rankings for most extreme 30 words in component \#65} \\
\end{longtable}
\begin{longtable}[!htbp]{| rlr@{.}l |}
    \hline
    \textbf{Rank} & \textbf{Word} & \multicolumn{2}{c|}{\textbf{Score}} \\
    \hline
    \endhead
    1 & modesty & 0 & -8439 \\
    2 & naturalness & 0 & -6728 \\
    3 & expressiveness & 0 & -6501 \\
    4 & demonstrative\_jj & 0 & -6493 \\
    5 & autonomous\_jj & 0 & -6202 \\
    6 & independence & 0 & -6168 \\
    7 & callousness & 0 & -5919 \\
    8 & pessimism & 0 & -5840 \\
    9 & organized\_jj & 0 & -5693 \\
    10 & stingy\_jj & 0 & -5630 \\
    11 & detached\_jj & 0 & -5582 \\
    12 & ungracious\_jj & 0 & -5290 \\
    13 & self-pitying\_jj & 0 & -5285 \\
    14 & haphazard\_jj & 0 & -5188 \\
    15 & economical\_jj & 0 & -5064 \\
    16 & talkative\_jj & 0 & -4990 \\
    17 & meticulous\_jj & 0 & -4956 \\
    18 & mannerly\_jj & 0 & -4567 \\
    19 & independent\_jj & 0 & -4549 \\
    20 & opportunistic\_jj & 0 & -4441 \\
    21 & cultured\_jj & 0 & -4364 \\
    22 & understanding\_jj & 0 & -4210 \\
    23 & thrifty\_jj & 0 & -4200 \\
    24 & somber\_jj & 0 & -4151 \\
    25 & warm\_jj & 0 & -4129 \\
    26 & animation & 0 & -3961 \\
    27 & surliness & 0 & -3937 \\
    28 & cold\_jj & 0 & -3807 \\
    29 & conceit & 0 & -3720 \\
    30 & restrained\_jj & 0 & -3635 \\
    392 & antagonistic\_jj & 0 & 3533 \\
    393 & uninhibited\_jj & 0 & 3539 \\
    394 & unstable\_jj & 0 & 3571 \\
    395 & gregarious\_jj & 0 & 3642 \\
    396 & negligent\_jj & 0 & 3675 \\
    397 & trustful\_jj & 0 & 3895 \\
    398 & shyness & 0 & 3992 \\
    399 & aimlessness & 0 & 4021 \\
    400 & rambunctious\_jj & 0 & 4071 \\
    401 & obstinate\_jj & 0 & 4151 \\
    402 & impolite\_jj & 0 & 4208 \\
    403 & meddlesome\_jj & 0 & 4365 \\
    404 & self-disciplined\_jj & 0 & 4637 \\
    405 & helpful\_jj & 0 & 4665 \\
    406 & self-critical\_jj & 0 & 4710 \\
    407 & bigoted\_jj & 0 & 4800 \\
    408 & stinginess & 0 & 4813 \\
    409 & leniency & 0 & 4819 \\
    410 & inefficient\_jj & 0 & 4933 \\
    411 & humor & 0 & 5116 \\
    412 & proud\_jj & 0 & 5311 \\
    413 & compassionate\_jj & 0 & 5449 \\
    414 & earthy\_jj & 0 & 5692 \\
    415 & meditative\_jj & 0 & 5820 \\
    416 & argumentative\_jj & 0 & 5924 \\
    417 & reserve & 0 & 6521 \\
    418 & curious\_jj & 0 & 6736 \\
    419 & wordy\_jj & 0 & 6824 \\
    420 & unreflective\_jj & 0 & 7483 \\
    421 & earthiness & 0 & 7964 \\
    \hline
    \caption{Scores and rankings for most extreme 30 words in component \#66} \\
\end{longtable}
\begin{longtable}[!htbp]{| rlr@{.}l |}
    \hline
    \textbf{Rank} & \textbf{Word} & \multicolumn{2}{c|}{\textbf{Score}} \\
    \hline
    \endhead
    1 & wordy\_jj & 0 & -8137 \\
    2 & inhibition & 0 & -8084 \\
    3 & nonconforming\_jj & 0 & -7652 \\
    4 & perceptive\_jj & 0 & -7434 \\
    5 & economical\_jj & 0 & -7071 \\
    6 & careless\_jj & 0 & -6186 \\
    7 & temperamental\_jj & 0 & -6035 \\
    8 & unreflective\_jj & 0 & -5564 \\
    9 & rash\_jj & 0 & -5554 \\
    10 & ungracious\_jj & 0 & -5540 \\
    11 & flippant\_jj & 0 & -5157 \\
    12 & unimaginative\_jj & 0 & -4988 \\
    13 & industrious\_jj & 0 & -4829 \\
    14 & courteous\_jj & 0 & -4669 \\
    15 & impudent\_jj & 0 & -4599 \\
    16 & pomposity & 0 & -4560 \\
    17 & purposeful\_jj & 0 & -4263 \\
    18 & erratic\_jj & 0 & -4222 \\
    19 & spontaneous\_jj & 0 & -4140 \\
    20 & disorganization & 0 & -4093 \\
    21 & demonstrative\_jj & 0 & -4054 \\
    22 & gullible\_jj & 0 & -4033 \\
    23 & prompt\_jj & 0 & -4021 \\
    24 & easygoing\_jj & 0 & -3971 \\
    25 & independence & 0 & -3885 \\
    26 & orderly\_jj & 0 & -3766 \\
    27 & respectful\_jj & 0 & -3754 \\
    28 & merry\_jj & 0 & -3560 \\
    29 & curious\_jj & 0 & -3521 \\
    30 & suspicious\_jj & 0 & -3497 \\
    392 & caustic\_jj & 0 & 3391 \\
    393 & harsh\_jj & 0 & 3488 \\
    394 & tactless\_jj & 0 & 3490 \\
    395 & ambitious\_jj & 0 & 3518 \\
    396 & rudeness & 0 & 3645 \\
    397 & natural\_jj & 0 & 3748 \\
    398 & reserved\_jj & 0 & 3823 \\
    399 & candor & 0 & 3826 \\
    400 & manipulative\_jj & 0 & 3877 \\
    401 & punctuality & 0 & 4032 \\
    402 & unconventional\_jj & 0 & 4036 \\
    403 & nosey\_jj & 0 & 4036 \\
    404 & unambitious\_jj & 0 & 4114 \\
    405 & scornful\_jj & 0 & 4274 \\
    406 & shy\_jj & 0 & 4279 \\
    407 & sincere\_jj & 0 & 4330 \\
    408 & passivity & 0 & 4379 \\
    409 & obliging\_jj & 0 & 4490 \\
    410 & indecisiveness & 0 & 4572 \\
    411 & surliness & 0 & 4809 \\
    412 & nonconformity & 0 & 5115 \\
    413 & concise\_jj & 0 & 5206 \\
    414 & self-critical\_jj & 0 & 5280 \\
    415 & worldly\_jj & 0 & 5743 \\
    416 & recklessness & 0 & 6197 \\
    417 & self-disciplined\_jj & 0 & 6202 \\
    418 & autonomous\_jj & 0 & 6700 \\
    419 & extroverted\_jj & 0 & 7319 \\
    420 & rude\_jj & 0 & 7465 \\
    421 & unintelligent\_jj & 0 & 7466 \\
    \hline
    \caption{Scores and rankings for most extreme 30 words in component \#67} \\
\end{longtable}

\subsection{Normalized PCA}
\label{app:rankedwordlists:438words:normalized}
\begin{longtable}[!htbp]{| rlr@{.}l |}
    \hline
    \textbf{Rank} & \textbf{Word} & \multicolumn{2}{c|}{\textbf{Score}} \\
    \hline
    \endhead
    1 & sociable\_jj & -17 & -9174 \\
    2 & vivacious\_jj & -16 & -77 \\
    3 & considerate\_jj & -15 & -7630 \\
    4 & easygoing\_jj & -14 & -8620 \\
    5 & witty\_jj & -12 & -5432 \\
    6 & talkative\_jj & -12 & -1966 \\
    7 & affectionate\_jj & -12 & -1878 \\
    8 & gregarious\_jj & -11 & -9660 \\
    9 & down-to-earth\_jj & -11 & -5647 \\
    10 & courteous\_jj & -11 & -5580 \\
    11 & jovial\_jj & -10 & -4218 \\
    12 & cultured\_jj & -10 & -1437 \\
    13 & extroverted\_jj & -10 & -800 \\
    14 & inquisitive\_jj & -9 & -4452 \\
    15 & introspective\_jj & -9 & -4351 \\
    16 & genial\_jj & -8 & -8250 \\
    17 & amiable\_jj & -8 & -5934 \\
    18 & perceptive\_jj & -8 & -4035 \\
    19 & mischievous\_jj & -8 & -3870 \\
    20 & folksy\_jj & -8 & -3864 \\
    21 & happy-go-lucky\_jj & -8 & -3401 \\
    22 & humorous\_jj & -8 & -1115 \\
    23 & expressive\_jj & -8 & -135 \\
    24 & cheerful\_jj & -7 & -9088 \\
    25 & kind\_jj & -7 & -8460 \\
    26 & playful\_jj & -7 & -6389 \\
    27 & surly\_jj & -7 & -4900 \\
    28 & impetuous\_jj & -7 & -4806 \\
    29 & communicative\_jj & -7 & -4656 \\
    30 & gruff\_jj & -7 & -4370 \\
    392 & fear & 7 & 4294 \\
    393 & independence & 7 & 4747 \\
    394 & responsible\_jj & 7 & 4857 \\
    395 & direct\_jj & 7 & 6232 \\
    396 & decisive\_jj & 7 & 6354 \\
    397 & prompt\_jj & 7 & 6829 \\
    398 & reckless\_jj & 7 & 6875 \\
    399 & suspicious\_jj & 7 & 6933 \\
    400 & organization & 7 & 7106 \\
    401 & diplomatic\_jj & 7 & 7528 \\
    402 & optimism & 7 & 7859 \\
    403 & assertion & 7 & 8563 \\
    404 & organized\_jj & 7 & 9213 \\
    405 & consistent\_jj & 7 & 9269 \\
    406 & efficiency & 8 & 2494 \\
    407 & reasonable\_jj & 8 & 3474 \\
    408 & negligent\_jj & 8 & 4584 \\
    409 & systematic\_jj & 8 & 4825 \\
    410 & volatile\_jj & 8 & 5729 \\
    411 & explosive\_jj & 8 & 6354 \\
    412 & reserve & 8 & 8266 \\
    413 & flexibility & 8 & 9127 \\
    414 & autonomous\_jj & 9 & 852 \\
    415 & caution & 9 & 1057 \\
    416 & sluggish\_jj & 9 & 2293 \\
    417 & intelligence & 9 & 5304 \\
    418 & cooperation & 9 & 9414 \\
    419 & volatility & 10 & 4852 \\
    420 & instability & 10 & 6089 \\
    421 & negligence & 11 & 3632 \\
    \hline
    \caption{Scores and rankings for most extreme 30 words in component \#1} \\
\end{longtable}
\begin{longtable}[!htbp]{| rlr@{.}l |}
    \hline
    \textbf{Rank} & \textbf{Word} & \multicolumn{2}{c|}{\textbf{Score}} \\
    \hline
    \endhead
    1 & callousness & -14 & -7345 \\
    2 & selfishness & -12 & -9045 \\
    3 & gullibility & -11 & -7229 \\
    4 & recklessness & -11 & -6896 \\
    5 & stupidity & -11 & -6396 \\
    6 & rudeness & -11 & -4240 \\
    7 & deceit & -11 & -940 \\
    8 & shallowness & -10 & -7380 \\
    9 & belligerence & -10 & -7331 \\
    10 & lethargy & -10 & -5262 \\
    11 & thoughtless\_jj & -10 & -3675 \\
    12 & passivity & -10 & -2520 \\
    13 & irritability & -10 & -2315 \\
    14 & pomposity & -9 & -8070 \\
    15 & deceitful\_jj & -9 & -6951 \\
    16 & stubbornness & -9 & -6779 \\
    17 & bigoted\_jj & -9 & -5756 \\
    18 & self-pitying\_jj & -9 & -4671 \\
    19 & indecisiveness & -9 & -3955 \\
    20 & unreflective\_jj & -8 & -7417 \\
    21 & aloofness & -8 & -4852 \\
    22 & disorganization & -8 & -4452 \\
    23 & inconsiderate\_jj & -8 & -2819 \\
    24 & vindictive\_jj & -8 & -1176 \\
    25 & ungracious\_jj & -7 & -9552 \\
    26 & selfish\_jj & -7 & -9546 \\
    27 & sloth & -7 & -8394 \\
    28 & forgetfulness & -7 & -8393 \\
    29 & shyness & -7 & -6504 \\
    30 & unintelligent\_jj & -7 & -5097 \\
    392 & active\_jj & 7 & 119 \\
    393 & cooperative\_jj & 7 & 831 \\
    394 & quiet\_jj & 7 & 3417 \\
    395 & thorough\_jj & 7 & 3420 \\
    396 & economical\_jj & 7 & 3974 \\
    397 & energetic\_jj & 7 & 4680 \\
    398 & cautious\_jj & 7 & 4782 \\
    399 & pleasant\_jj & 7 & 5701 \\
    400 & generous\_jj & 7 & 9052 \\
    401 & gregarious\_jj & 8 & 928 \\
    402 & cultured\_jj & 8 & 3465 \\
    403 & cordial\_jj & 8 & 3841 \\
    404 & kind\_jj & 8 & 3986 \\
    405 & confident\_jj & 8 & 5209 \\
    406 & vivacious\_jj & 8 & 7426 \\
    407 & courteous\_jj & 8 & 8863 \\
    408 & enthusiastic\_jj & 8 & 9927 \\
    409 & intelligent\_jj & 9 & 284 \\
    410 & adventurous\_jj & 9 & 798 \\
    411 & warm\_jj & 9 & 2619 \\
    412 & concise\_jj & 9 & 2832 \\
    413 & flexible\_jj & 9 & 3619 \\
    414 & optimistic\_jj & 9 & 4434 \\
    415 & dependable\_jj & 9 & 7301 \\
    416 & easygoing\_jj & 9 & 9472 \\
    417 & efficient\_jj & 10 & 4419 \\
    418 & reliable\_jj & 10 & 4431 \\
    419 & friendly\_jj & 10 & 4577 \\
    420 & considerate\_jj & 12 & 5096 \\
    421 & sociable\_jj & 15 & 3877 \\
    \hline
    \caption{Scores and rankings for most extreme 30 words in component \#2} \\
\end{longtable}
\begin{longtable}[!htbp]{| rlr@{.}l |}
    \hline
    \textbf{Rank} & \textbf{Word} & \multicolumn{2}{c|}{\textbf{Score}} \\
    \hline
    \endhead
    1 & abusive\_jj & -13 & -3632 \\
    2 & inconsiderate\_jj & -10 & -1426 \\
    3 & insensitive\_jj & -10 & -887 \\
    4 & selfish\_jj & -10 & -718 \\
    5 & ignorant\_jj & -9 & -8699 \\
    6 & disrespectful\_jj & -9 & -8228 \\
    7 & dishonest\_jj & -9 & -8069 \\
    8 & unscrupulous\_jj & -9 & -6504 \\
    9 & lenient\_jj & -9 & -5210 \\
    10 & bigoted\_jj & -9 & -2880 \\
    11 & gullible\_jj & -8 & -9342 \\
    12 & pessimistic\_jj & -8 & -8256 \\
    13 & lazy\_jj & -8 & -8019 \\
    14 & unfriendly\_jj & -8 & -6011 \\
    15 & greedy\_jj & -8 & -5481 \\
    16 & unsympathetic\_jj & -8 & -4162 \\
    17 & vindictive\_jj & -8 & -2567 \\
    18 & inefficient\_jj & -8 & -446 \\
    19 & absent-minded\_jj & -7 & -7814 \\
    20 & impolite\_jj & -7 & -7545 \\
    21 & naïve\_jj & -7 & -6804 \\
    22 & intrusive\_jj & -7 & -5694 \\
    23 & deceitful\_jj & -7 & -5477 \\
    24 & rude\_jj & -7 & -3775 \\
    25 & negligent\_jj & -7 & -2219 \\
    26 & prejudiced\_jj & -7 & -1641 \\
    27 & unreliable\_jj & -6 & -8554 \\
    28 & distrustful\_jj & -6 & -7541 \\
    29 & wishy-washy\_jj & -6 & -5846 \\
    30 & thoughtless\_jj & -6 & -5108 \\
    392 & earthy\_jj & 6 & 6761 \\
    393 & optimism & 6 & 8439 \\
    394 & empathy & 6 & 9342 \\
    395 & melancholic\_jj & 7 & 747 \\
    396 & courage & 7 & 1202 \\
    397 & inhibition & 7 & 2338 \\
    398 & cunning & 7 & 2736 \\
    399 & modesty & 7 & 3009 \\
    400 & generosity & 7 & 3261 \\
    401 & dependability & 7 & 5045 \\
    402 & artistic\_jj & 7 & 6467 \\
    403 & depth & 7 & 6587 \\
    404 & persistence & 8 & 4394 \\
    405 & spirit & 8 & 4413 \\
    406 & aloofness & 8 & 5817 \\
    407 & irritability & 8 & 6366 \\
    408 & decisiveness & 8 & 6633 \\
    409 & humor & 8 & 9401 \\
    410 & precision & 8 & 9959 \\
    411 & meditative\_jj & 9 & 2197 \\
    412 & lethargy & 9 & 3520 \\
    413 & creativity & 9 & 4649 \\
    414 & candor & 10 & 964 \\
    415 & warmth & 10 & 4259 \\
    416 & earthiness & 10 & 8084 \\
    417 & sophistication & 10 & 9239 \\
    418 & naturalness & 10 & 9434 \\
    419 & expressiveness & 11 & 1021 \\
    420 & spontaneity & 11 & 7606 \\
    421 & playfulness & 12 & 5981 \\
    \hline
    \caption{Scores and rankings for most extreme 30 words in component \#3} \\
\end{longtable}
\begin{longtable}[!htbp]{| rlr@{.}l |}
    \hline
    \textbf{Rank} & \textbf{Word} & \multicolumn{2}{c|}{\textbf{Score}} \\
    \hline
    \endhead
    1 & sincere\_jj & -11 & -5699 \\
    2 & considerate\_jj & -10 & -8882 \\
    3 & dignity & -10 & -1223 \\
    4 & courage & -9 & -5567 \\
    5 & selfless\_jj & -9 & -3142 \\
    6 & courageous\_jj & -8 & -454 \\
    7 & honest\_jj & -7 & -9668 \\
    8 & stupidity & -7 & -7833 \\
    9 & negligence & -7 & -7304 \\
    10 & principled\_jj & -7 & -6420 \\
    11 & compassionate\_jj & -7 & -3757 \\
    12 & selfish\_jj & -7 & -1749 \\
    13 & moral\_jj & -7 & -973 \\
    14 & negligent\_jj & -7 & -955 \\
    15 & generosity & -6 & -9729 \\
    16 & recklessness & -6 & -8772 \\
    17 & dishonest\_jj & -6 & -8090 \\
    18 & candor & -6 & -6757 \\
    19 & selfishness & -6 & -4613 \\
    20 & courteous\_jj & -6 & -4086 \\
    21 & deliberate\_jj & -6 & -3750 \\
    22 & empathy & -6 & -2984 \\
    23 & truthful\_jj & -6 & -2199 \\
    24 & systematic\_jj & -6 & -1835 \\
    25 & sociable\_jj & -6 & -101 \\
    26 & cruelty & -5 & -9335 \\
    27 & modesty & -5 & -9094 \\
    28 & ethical\_jj & -5 & -8270 \\
    29 & morality & -5 & -5443 \\
    30 & reckless\_jj & -5 & -4066 \\
    392 & aimlessness & 4 & 3788 \\
    393 & nonconforming\_jj & 4 & 4259 \\
    394 & tempestuous\_jj & 4 & 6624 \\
    395 & insecure\_jj & 4 & 7949 \\
    396 & grumpy\_jj & 4 & 9079 \\
    397 & aimless\_jj & 4 & 9381 \\
    398 & meditative\_jj & 5 & 335 \\
    399 & placidity & 5 & 739 \\
    400 & surly\_jj & 5 & 1262 \\
    401 & cranky\_jj & 5 & 1824 \\
    402 & dominant\_jj & 5 & 1945 \\
    403 & quarrelsome\_jj & 5 & 2269 \\
    404 & anxious\_jj & 5 & 3232 \\
    405 & extroverted\_jj & 5 & 3974 \\
    406 & forgetful\_jj & 5 & 4411 \\
    407 & morose\_jj & 5 & 4433 \\
    408 & volatile\_jj & 5 & 6673 \\
    409 & volatility & 5 & 6935 \\
    410 & moody\_jj & 5 & 8277 \\
    411 & nervous\_jj & 6 & 1201 \\
    412 & cold\_jj & 6 & 3646 \\
    413 & forgetfulness & 6 & 9360 \\
    414 & fretful\_jj & 7 & 1163 \\
    415 & erratic\_jj & 7 & 1323 \\
    416 & lethargic\_jj & 8 & 7704 \\
    417 & sluggish\_jj & 8 & 9476 \\
    418 & lethargy & 11 & 1712 \\
    419 & irritability & 14 & 3020 \\
    420 & irritable\_jj & 14 & 4806 \\
    421 & absent-minded\_jj & 18 & 6782 \\
    \hline
    \caption{Scores and rankings for most extreme 30 words in component \#4} \\
\end{longtable}
\begin{longtable}[!htbp]{| rlr@{.}l |}
    \hline
    \textbf{Rank} & \textbf{Word} & \multicolumn{2}{c|}{\textbf{Score}} \\
    \hline
    \endhead
    1 & absent-minded\_jj & -22 & -510 \\
    2 & refined\_jj & -11 & -5546 \\
    3 & economical\_jj & -11 & -3412 \\
    4 & concise\_jj & -10 & -4225 \\
    5 & efficient\_jj & -9 & -2003 \\
    6 & innovative\_jj & -8 & -8193 \\
    7 & inventive\_jj & -8 & -4952 \\
    8 & adaptable\_jj & -7 & -7483 \\
    9 & underhanded\_jj & -7 & -4146 \\
    10 & analytical\_jj & -7 & -3982 \\
    11 & sophisticated\_jj & -7 & -2095 \\
    12 & imaginative\_jj & -6 & -9387 \\
    13 & exacting\_jj & -6 & -8588 \\
    14 & devious\_jj & -6 & -2850 \\
    15 & expressive\_jj & -6 & -2559 \\
    16 & cunning\_jj & -6 & -2285 \\
    17 & complex\_jj & -5 & -8993 \\
    18 & expressiveness & -5 & -4704 \\
    19 & insightful\_jj & -5 & -4303 \\
    20 & unimaginative\_jj & -5 & -3459 \\
    21 & meticulous\_jj & -5 & -1384 \\
    22 & inefficient\_jj & -5 & -1231 \\
    23 & manipulative\_jj & -5 & -457 \\
    24 & perceptive\_jj & -4 & -9858 \\
    25 & wordy\_jj & -4 & -9200 \\
    26 & precision & -4 & -8448 \\
    27 & unconventional\_jj & -4 & -7500 \\
    28 & creative\_jj & -4 & -6648 \\
    29 & flexible\_jj & -4 & -5927 \\
    30 & unintelligent\_jj & -4 & -5398 \\
    392 & caution & 4 & 9439 \\
    393 & somber\_jj & 4 & 9486 \\
    394 & talkative\_jj & 5 & 663 \\
    395 & fear & 5 & 1831 \\
    396 & fearful\_jj & 5 & 2880 \\
    397 & insecurity & 5 & 3482 \\
    398 & polite\_jj & 5 & 3502 \\
    399 & considerate\_jj & 5 & 4851 \\
    400 & instability & 5 & 5099 \\
    401 & bitter\_jj & 5 & 6343 \\
    402 & cautious\_jj & 5 & 6552 \\
    403 & warm\_jj & 5 & 7360 \\
    404 & jovial\_jj & 5 & 7361 \\
    405 & quiet\_jj & 5 & 7645 \\
    406 & anxious\_jj & 5 & 8030 \\
    407 & gregarious\_jj & 6 & 1419 \\
    408 & irritable\_jj & 6 & 3110 \\
    409 & pessimism & 6 & 4152 \\
    410 & distrust & 6 & 5182 \\
    411 & optimistic\_jj & 6 & 5851 \\
    412 & pessimistic\_jj & 6 & 7790 \\
    413 & kind\_jj & 6 & 7849 \\
    414 & silence & 7 & 1660 \\
    415 & sincere\_jj & 7 & 4242 \\
    416 & lethargy & 7 & 4995 \\
    417 & nervous\_jj & 7 & 5489 \\
    418 & cordial\_jj & 8 & 1827 \\
    419 & sociable\_jj & 8 & 4052 \\
    420 & irritability & 9 & 6663 \\
    421 & optimism & 9 & 6918 \\
    \hline
    \caption{Scores and rankings for most extreme 30 words in component \#5} \\
\end{longtable}
\begin{longtable}[!htbp]{| rlr@{.}l |}
    \hline
    \textbf{Rank} & \textbf{Word} & \multicolumn{2}{c|}{\textbf{Score}} \\
    \hline
    \endhead
    1 & irritability & -24 & -8525 \\
    2 & lethargy & -14 & -8777 \\
    3 & irritable\_jj & -14 & -3164 \\
    4 & economical\_jj & -12 & -2556 \\
    5 & forgetfulness & -12 & -1658 \\
    6 & considerate\_jj & -10 & -5745 \\
    7 & sociable\_jj & -9 & -2906 \\
    8 & abusive\_jj & -9 & -768 \\
    9 & self-esteem & -8 & -6816 \\
    10 & adaptable\_jj & -7 & -1513 \\
    11 & inhibition & -7 & -783 \\
    12 & communicative\_jj & -6 & -6596 \\
    13 & compassionate\_jj & -6 & -2091 \\
    14 & efficient\_jj & -5 & -9327 \\
    15 & extroverted\_jj & -5 & -9171 \\
    16 & disorganization & -5 & -8787 \\
    17 & intelligent\_jj & -5 & -8122 \\
    18 & kind\_jj & -5 & -4527 \\
    19 & analytical\_jj & -5 & -3023 \\
    20 & empathy & -5 & -2580 \\
    21 & talkative\_jj & -4 & -7939 \\
    22 & selfishness & -4 & -7313 \\
    23 & suggestible\_jj & -4 & -6729 \\
    24 & courteous\_jj & -4 & -6361 \\
    25 & insecurity & -4 & -5838 \\
    26 & forgetful\_jj & -4 & -5151 \\
    27 & volatility & -4 & -3636 \\
    28 & callousness & -4 & -3469 \\
    29 & rudeness & -4 & -3308 \\
    30 & impersonal\_jj & -4 & -2861 \\
    392 & unassuming\_jj & 3 & 9097 \\
    393 & cosmopolitan\_jj & 3 & 9693 \\
    394 & independence & 4 & 410 \\
    395 & surliness & 4 & 740 \\
    396 & passionless\_jj & 4 & 1021 \\
    397 & merry\_jj & 4 & 1261 \\
    398 & defensive\_jj & 4 & 1431 \\
    399 & courtesy & 4 & 1716 \\
    400 & assertion & 4 & 3017 \\
    401 & scornful\_jj & 4 & 3099 \\
    402 & crafty\_jj & 4 & 4455 \\
    403 & rambunctious\_jj & 4 & 4544 \\
    404 & bullheaded\_jj & 4 & 4719 \\
    405 & spirited\_jj & 4 & 5241 \\
    406 & skeptical\_jj & 4 & 5422 \\
    407 & genial\_jj & 4 & 7815 \\
    408 & somber\_jj & 4 & 8086 \\
    409 & vain\_jj & 4 & 8215 \\
    410 & earthiness & 4 & 8326 \\
    411 & caustic\_jj & 4 & 8531 \\
    412 & tempestuous\_jj & 5 & 198 \\
    413 & bitter\_jj & 5 & 1455 \\
    414 & zestful\_jj & 5 & 2506 \\
    415 & curt\_jj & 5 & 2758 \\
    416 & sly\_jj & 5 & 3347 \\
    417 & gruff\_jj & 5 & 3735 \\
    418 & homespun\_jj & 5 & 4537 \\
    419 & flamboyant\_jj & 5 & 4548 \\
    420 & reserve & 5 & 7156 \\
    421 & folksy\_jj & 7 & 6019 \\
    \hline
    \caption{Scores and rankings for most extreme 30 words in component \#6} \\
\end{longtable}
\begin{longtable}[!htbp]{| rlr@{.}l |}
    \hline
    \textbf{Rank} & \textbf{Word} & \multicolumn{2}{c|}{\textbf{Score}} \\
    \hline
    \endhead
    1 & absent-minded\_jj & -47 & -1248 \\
    2 & cordial\_jj & -7 & -9274 \\
    3 & prompt\_jj & -7 & -1502 \\
    4 & candor & -6 & -6573 \\
    5 & leniency & -6 & -15 \\
    6 & belligerence & -5 & -9037 \\
    7 & frank\_jj & -5 & -8832 \\
    8 & respectful\_jj & -5 & -7494 \\
    9 & courage & -5 & -6945 \\
    10 & curt\_jj & -5 & -6712 \\
    11 & concise\_jj & -5 & -2383 \\
    12 & sincere\_jj & -5 & -383 \\
    13 & decisiveness & -4 & -8760 \\
    14 & tactful\_jj & -4 & -8233 \\
    15 & careful\_jj & -4 & -7894 \\
    16 & truthful\_jj & -4 & -6451 \\
    17 & principled\_jj & -4 & -5736 \\
    18 & courteous\_jj & -4 & -3229 \\
    19 & polite\_jj & -4 & -2291 \\
    20 & compassionate\_jj & -4 & -2009 \\
    21 & stubbornness & -3 & -9360 \\
    22 & pessimism & -3 & -8840 \\
    23 & forceful\_jj & -3 & -8595 \\
    24 & touchy\_jj & -3 & -7960 \\
    25 & dignity & -3 & -6842 \\
    26 & unemotional\_jj & -3 & -6141 \\
    27 & persistence & -3 & -5333 \\
    28 & courageous\_jj & -3 & -4769 \\
    29 & punctual\_jj & -3 & -4362 \\
    30 & insight & -3 & -3325 \\
    392 & natural\_jj & 3 & 2166 \\
    393 & happy-go-lucky\_jj & 3 & 2890 \\
    394 & volatile\_jj & 3 & 3214 \\
    395 & shallow\_jj & 3 & 3730 \\
    396 & aimlessness & 3 & 4042 \\
    397 & snobbish\_jj & 3 & 4078 \\
    398 & instability & 3 & 4396 \\
    399 & greedy\_jj & 3 & 4461 \\
    400 & egotistical\_jj & 3 & 4581 \\
    401 & cultured\_jj & 3 & 4622 \\
    402 & efficient\_jj & 3 & 4875 \\
    403 & cunning\_jj & 3 & 5250 \\
    404 & sophisticated\_jj & 3 & 5802 \\
    405 & rebellious\_jj & 3 & 5881 \\
    406 & carefree\_jj & 3 & 5958 \\
    407 & ambitious\_jj & 3 & 6139 \\
    408 & unconventional\_jj & 3 & 6722 \\
    409 & artistic\_jj & 3 & 7236 \\
    410 & manipulative\_jj & 3 & 7935 \\
    411 & egocentric\_jj & 4 & 18 \\
    412 & devious\_jj & 4 & 1848 \\
    413 & conventional\_jj & 4 & 2295 \\
    414 & inefficient\_jj & 4 & 4903 \\
    415 & refined\_jj & 4 & 5449 \\
    416 & unimaginative\_jj & 4 & 8323 \\
    417 & nonconforming\_jj & 5 & 5 \\
    418 & adventurous\_jj & 5 & 1558 \\
    419 & unstable\_jj & 5 & 6260 \\
    420 & abusive\_jj & 6 & 3450 \\
    421 & autonomous\_jj & 6 & 4942 \\
    \hline
    \caption{Scores and rankings for most extreme 30 words in component \#7} \\
\end{longtable}
\begin{longtable}[!htbp]{| rlr@{.}l |}
    \hline
    \textbf{Rank} & \textbf{Word} & \multicolumn{2}{c|}{\textbf{Score}} \\
    \hline
    \endhead
    1 & concise\_jj & -10 & -8954 \\
    2 & cordial\_jj & -10 & -2676 \\
    3 & economical\_jj & -7 & -2126 \\
    4 & respectful\_jj & -6 & -9970 \\
    5 & antagonistic\_jj & -6 & -8741 \\
    6 & forceful\_jj & -6 & -6765 \\
    7 & belligerence & -6 & -6238 \\
    8 & lenient\_jj & -6 & -6227 \\
    9 & combative\_jj & -6 & -4999 \\
    10 & unemotional\_jj & -6 & -3554 \\
    11 & verbose\_jj & -6 & -691 \\
    12 & assertive\_jj & -6 & -464 \\
    13 & caustic\_jj & -5 & -9788 \\
    14 & restrained\_jj & -5 & -8759 \\
    15 & frank\_jj & -5 & -6577 \\
    16 & somber\_jj & -5 & -4546 \\
    17 & truthful\_jj & -5 & -4499 \\
    18 & scornful\_jj & -5 & -2559 \\
    19 & predictable\_jj & -5 & -1481 \\
    20 & insensitive\_jj & -5 & -36 \\
    21 & dignified\_jj & -4 & -9963 \\
    22 & candor & -4 & -9348 \\
    23 & self-critical\_jj & -4 & -9098 \\
    24 & prompt\_jj & -4 & -8556 \\
    25 & inconsistent\_jj & -4 & -7830 \\
    26 & flippant\_jj & -4 & -7118 \\
    27 & wordy\_jj & -4 & -6901 \\
    28 & indecisiveness & -4 & -6381 \\
    29 & condescending\_jj & -4 & -6069 \\
    30 & cautious\_jj & -4 & -5855 \\
    392 & enterprising\_jj & 4 & 1047 \\
    393 & brave\_jj & 4 & 1135 \\
    394 & creative\_jj & 4 & 1192 \\
    395 & cultured\_jj & 4 & 2836 \\
    396 & generosity & 4 & 3062 \\
    397 & energetic\_jj & 4 & 3746 \\
    398 & cranky\_jj & 4 & 4252 \\
    399 & shy\_jj & 4 & 4596 \\
    400 & courtesy & 4 & 5641 \\
    401 & organized\_jj & 4 & 6465 \\
    402 & fear & 4 & 6585 \\
    403 & gregarious\_jj & 4 & 7700 \\
    404 & negligent\_jj & 4 & 9003 \\
    405 & lazy\_jj & 4 & 9473 \\
    406 & adventurous\_jj & 4 & 9756 \\
    407 & gullible\_jj & 5 & 140 \\
    408 & cruelty & 5 & 572 \\
    409 & spirit & 5 & 807 \\
    410 & intelligent\_jj & 5 & 1657 \\
    411 & organization & 5 & 5042 \\
    412 & negligence & 5 & 6886 \\
    413 & greedy\_jj & 5 & 6917 \\
    414 & inconsiderate\_jj & 5 & 9727 \\
    415 & proud\_jj & 6 & 3149 \\
    416 & kind\_jj & 6 & 3645 \\
    417 & vivacious\_jj & 7 & 4914 \\
    418 & charitable\_jj & 8 & 2313 \\
    419 & sociable\_jj & 8 & 6254 \\
    420 & unscrupulous\_jj & 8 & 7309 \\
    421 & absent-minded\_jj & 25 & 9667 \\
    \hline
    \caption{Scores and rankings for most extreme 30 words in component \#8} \\
\end{longtable}
\begin{longtable}[!htbp]{| rlr@{.}l |}
    \hline
    \textbf{Rank} & \textbf{Word} & \multicolumn{2}{c|}{\textbf{Score}} \\
    \hline
    \endhead
    1 & distrustful\_jj & -12 & -2522 \\
    2 & individualistic\_jj & -9 & -7235 \\
    3 & aloofness & -8 & -6395 \\
    4 & assertive\_jj & -8 & -6200 \\
    5 & distrust & -8 & -6149 \\
    6 & belligerence & -7 & -8943 \\
    7 & antagonistic\_jj & -7 & -5190 \\
    8 & autonomous\_jj & -7 & -158 \\
    9 & accommodating\_jj & -6 & -8087 \\
    10 & dependability & -6 & -4803 \\
    11 & pessimistic\_jj & -6 & -2216 \\
    12 & uncritical\_jj & -5 & -9978 \\
    13 & obstinate\_jj & -5 & -6727 \\
    14 & independence & -5 & -4377 \\
    15 & adaptable\_jj & -5 & -25 \\
    16 & insecure\_jj & -4 & -9194 \\
    17 & insecurity & -4 & -7897 \\
    18 & pessimism & -4 & -7672 \\
    19 & prejudiced\_jj & -4 & -7650 \\
    20 & generosity & -4 & -4769 \\
    21 & economical\_jj & -4 & -3748 \\
    22 & optimistic\_jj & -4 & -3592 \\
    23 & cooperation & -4 & -3515 \\
    24 & indecisive\_jj & -4 & -2698 \\
    25 & unfriendly\_jj & -4 & -2576 \\
    26 & ambition & -4 & -2393 \\
    27 & cosmopolitan\_jj & -4 & -1897 \\
    28 & instability & -4 & -1314 \\
    29 & extroverted\_jj & -4 & -663 \\
    30 & principled\_jj & -4 & -501 \\
    392 & witty\_jj & 4 & 5170 \\
    393 & dull\_jj & 4 & 5293 \\
    394 & lazy\_jj & 4 & 5320 \\
    395 & spontaneous\_jj & 4 & 5456 \\
    396 & straightforward\_jj & 4 & 5760 \\
    397 & patient\_jj & 4 & 6912 \\
    398 & curt\_jj & 4 & 7398 \\
    399 & humorous\_jj & 5 & 204 \\
    400 & meditative\_jj & 5 & 385 \\
    401 & cruel\_jj & 5 & 1807 \\
    402 & systematic\_jj & 5 & 2098 \\
    403 & thorough\_jj & 5 & 4257 \\
    404 & verbal\_jj & 5 & 4268 \\
    405 & reckless\_jj & 5 & 5184 \\
    406 & simple\_jj & 5 & 6147 \\
    407 & forgetfulness & 5 & 6618 \\
    408 & playful\_jj & 5 & 9259 \\
    409 & warm\_jj & 5 & 9614 \\
    410 & cruelty & 6 & 2118 \\
    411 & caustic\_jj & 6 & 2805 \\
    412 & sloppy\_jj & 6 & 4671 \\
    413 & folksy\_jj & 6 & 8319 \\
    414 & concise\_jj & 6 & 8444 \\
    415 & prompt\_jj & 7 & 4056 \\
    416 & deliberate\_jj & 7 & 9721 \\
    417 & abusive\_jj & 8 & 3578 \\
    418 & irritability & 8 & 7058 \\
    419 & careless\_jj & 8 & 7204 \\
    420 & negligence & 10 & 3312 \\
    421 & negligent\_jj & 11 & 5314 \\
    \hline
    \caption{Scores and rankings for most extreme 30 words in component \#9} \\
\end{longtable}
\begin{longtable}[!htbp]{| rlr@{.}l |}
    \hline
    \textbf{Rank} & \textbf{Word} & \multicolumn{2}{c|}{\textbf{Score}} \\
    \hline
    \endhead
    1 & negligent\_jj & -12 & -3212 \\
    2 & friendly\_jj & -9 & -625 \\
    3 & negligence & -8 & -2343 \\
    4 & cordial\_jj & -8 & -1989 \\
    5 & reckless\_jj & -6 & -9314 \\
    6 & recklessness & -6 & -7269 \\
    7 & belligerence & -6 & -6244 \\
    8 & economical\_jj & -6 & -4819 \\
    9 & easygoing\_jj & -6 & -1759 \\
    10 & leniency & -6 & -85 \\
    11 & surly\_jj & -5 & -7071 \\
    12 & cooperation & -5 & -4620 \\
    13 & mannerly\_jj & -5 & -2186 \\
    14 & unrestrained\_jj & -5 & -1548 \\
    15 & lenient\_jj & -5 & -421 \\
    16 & antagonistic\_jj & -4 & -9653 \\
    17 & orderly\_jj & -4 & -7773 \\
    18 & refined\_jj & -4 & -6666 \\
    19 & gregarious\_jj & -4 & -5573 \\
    20 & efficient\_jj & -4 & -4797 \\
    21 & cooperative\_jj & -4 & -4409 \\
    22 & aloofness & -4 & -4094 \\
    23 & cruelty & -4 & -4005 \\
    24 & unfriendly\_jj & -4 & -2002 \\
    25 & inefficient\_jj & -4 & -1760 \\
    26 & miserly\_jj & -4 & -1596 \\
    27 & exacting\_jj & -4 & -143 \\
    28 & courteous\_jj & -4 & -117 \\
    29 & unscrupulous\_jj & -4 & -8 \\
    30 & docile\_jj & -3 & -9736 \\
    392 & helpful\_jj & 4 & 2504 \\
    393 & gullibility & 4 & 2856 \\
    394 & precise\_jj & 4 & 2882 \\
    395 & proud\_jj & 4 & 3458 \\
    396 & jealous\_jj & 4 & 3676 \\
    397 & abusive\_jj & 4 & 3872 \\
    398 & envious\_jj & 4 & 4134 \\
    399 & wordy\_jj & 4 & 4477 \\
    400 & cautious\_jj & 4 & 7306 \\
    401 & wishy-washy\_jj & 4 & 7504 \\
    402 & depth & 4 & 7563 \\
    403 & lazy\_jj & 4 & 8680 \\
    404 & truthful\_jj & 4 & 8951 \\
    405 & confident\_jj & 4 & 9073 \\
    406 & logic & 4 & 9183 \\
    407 & careful\_jj & 4 & 9524 \\
    408 & naïve\_jj & 5 & 479 \\
    409 & gullible\_jj & 5 & 5497 \\
    410 & nervous\_jj & 5 & 6623 \\
    411 & unkind\_jj & 5 & 9200 \\
    412 & perceptive\_jj & 6 & 1712 \\
    413 & optimistic\_jj & 6 & 3832 \\
    414 & ignorant\_jj & 6 & 3909 \\
    415 & philosophical\_jj & 6 & 3919 \\
    416 & curious\_jj & 7 & 3118 \\
    417 & insightful\_jj & 7 & 6245 \\
    418 & skeptical\_jj & 7 & 9378 \\
    419 & pessimistic\_jj & 7 & 9479 \\
    420 & insight & 10 & 8727 \\
    421 & concise\_jj & 11 & 5624 \\
    \hline
    \caption{Scores and rankings for most extreme 30 words in component \#10} \\
\end{longtable}
\begin{longtable}[!htbp]{| rlr@{.}l |}
    \hline
    \textbf{Rank} & \textbf{Word} & \multicolumn{2}{c|}{\textbf{Score}} \\
    \hline
    \endhead
    1 & abusive\_jj & -39 & -9226 \\
    2 & explosive\_jj & -6 & -48 \\
    3 & suspicious\_jj & -5 & -4410 \\
    4 & belligerence & -5 & -3044 \\
    5 & antagonistic\_jj & -5 & -2618 \\
    6 & autonomous\_jj & -4 & -7470 \\
    7 & earthiness & -4 & -5281 \\
    8 & expressive\_jj & -4 & -4488 \\
    9 & emotional\_jj & -4 & -3104 \\
    10 & independent\_jj & -4 & -2082 \\
    11 & verbal\_jj & -4 & -1815 \\
    12 & melancholic\_jj & -4 & -458 \\
    13 & diplomatic\_jj & -3 & -9871 \\
    14 & independence & -3 & -8945 \\
    15 & systematic\_jj & -3 & -8301 \\
    16 & cruelty & -3 & -6890 \\
    17 & respectful\_jj & -3 & -6855 \\
    18 & tempestuous\_jj & -3 & -6445 \\
    19 & distrustful\_jj & -3 & -4701 \\
    20 & organized\_jj & -3 & -4653 \\
    21 & active\_jj & -3 & -3726 \\
    22 & affectionate\_jj & -3 & -3713 \\
    23 & nonconforming\_jj & -3 & -3430 \\
    24 & silence & -3 & -1949 \\
    25 & unfriendly\_jj & -3 & -531 \\
    26 & assertive\_jj & -3 & -89 \\
    27 & rebellious\_jj & -2 & -9757 \\
    28 & informal\_jj & -2 & -9245 \\
    29 & organization & -2 & -9175 \\
    30 & sympathetic\_jj & -2 & -8778 \\
    392 & cynical\_jj & 3 & 2316 \\
    393 & selfish\_jj & 3 & 2667 \\
    394 & inconsiderate\_jj & 3 & 3149 \\
    395 & stingy\_jj & 3 & 3300 \\
    396 & forgetful\_jj & 3 & 4280 \\
    397 & greedy\_jj & 3 & 4764 \\
    398 & naïve\_jj & 3 & 4953 \\
    399 & foolhardy\_jj & 3 & 5305 \\
    400 & frivolous\_jj & 3 & 5831 \\
    401 & thrifty\_jj & 3 & 5990 \\
    402 & optimism & 3 & 6890 \\
    403 & stupidity & 3 & 7102 \\
    404 & lethargy & 3 & 7522 \\
    405 & smug\_jj & 3 & 8142 \\
    406 & gullible\_jj & 3 & 9812 \\
    407 & punctuality & 4 & 148 \\
    408 & wishy-washy\_jj & 4 & 223 \\
    409 & dependable\_jj & 4 & 1291 \\
    410 & unimaginative\_jj & 4 & 1403 \\
    411 & efficient\_jj & 4 & 1829 \\
    412 & efficiency & 4 & 3098 \\
    413 & inefficient\_jj & 4 & 6245 \\
    414 & sluggish\_jj & 4 & 6475 \\
    415 & careless\_jj & 4 & 6684 \\
    416 & miserly\_jj & 5 & 1424 \\
    417 & sloppy\_jj & 5 & 1503 \\
    418 & lazy\_jj & 5 & 1862 \\
    419 & lethargic\_jj & 5 & 6959 \\
    420 & pessimistic\_jj & 5 & 9554 \\
    421 & economical\_jj & 8 & 2683 \\
    \hline
    \caption{Scores and rankings for most extreme 30 words in component \#11} \\
\end{longtable}
\begin{longtable}[!htbp]{| rlr@{.}l |}
    \hline
    \textbf{Rank} & \textbf{Word} & \multicolumn{2}{c|}{\textbf{Score}} \\
    \hline
    \endhead
    1 & erratic\_jj & -9 & -4304 \\
    2 & indecisive\_jj & -8 & -776 \\
    3 & tenacious\_jj & -7 & -4607 \\
    4 & unpredictable\_jj & -7 & -1938 \\
    5 & unstable\_jj & -6 & -5558 \\
    6 & reckless\_jj & -6 & -3949 \\
    7 & sluggish\_jj & -6 & -1815 \\
    8 & explosive\_jj & -5 & -9946 \\
    9 & energetic\_jj & -5 & -8407 \\
    10 & inventive\_jj & -5 & -7951 \\
    11 & assertive\_jj & -5 & -6537 \\
    12 & obstinate\_jj & -5 & -6300 \\
    13 & expressive\_jj & -5 & -2676 \\
    14 & optimistic\_jj & -5 & -2528 \\
    15 & combative\_jj & -5 & -1256 \\
    16 & optimism & -5 & -628 \\
    17 & ruthless\_jj & -5 & -503 \\
    18 & forceful\_jj & -4 & -9408 \\
    19 & stubborn\_jj & -4 & -8757 \\
    20 & stubbornness & -4 & -8674 \\
    21 & lethargic\_jj & -4 & -8587 \\
    22 & cunning\_jj & -4 & -7318 \\
    23 & cunning & -4 & -6965 \\
    24 & pessimism & -4 & -6418 \\
    25 & sophistication & -4 & -5983 \\
    26 & decisiveness & -4 & -5958 \\
    27 & manipulative\_jj & -4 & -5941 \\
    28 & pessimistic\_jj & -4 & -5565 \\
    29 & impetuous\_jj & -4 & -4384 \\
    30 & cautious\_jj & -4 & -3704 \\
    392 & polite\_jj & 3 & 9375 \\
    393 & understanding\_jj & 3 & 9963 \\
    394 & ethical\_jj & 4 & 239 \\
    395 & intrusiveness & 4 & 1261 \\
    396 & mannerly\_jj & 4 & 1506 \\
    397 & unsystematic\_jj & 4 & 1834 \\
    398 & organization & 4 & 2477 \\
    399 & traditional\_jj & 4 & 2729 \\
    400 & cordial\_jj & 4 & 2754 \\
    401 & independence & 4 & 3947 \\
    402 & crabby\_jj & 4 & 4048 \\
    403 & rudeness & 4 & 4743 \\
    404 & inconsiderate\_jj & 4 & 4896 \\
    405 & obliging\_jj & 4 & 4925 \\
    406 & nonconformity & 4 & 6189 \\
    407 & warm\_jj & 4 & 6666 \\
    408 & efficiency & 4 & 8474 \\
    409 & intellectuality & 5 & 216 \\
    410 & patient\_jj & 5 & 1254 \\
    411 & aimlessness & 5 & 1656 \\
    412 & punctuality & 5 & 2487 \\
    413 & diplomatic\_jj & 5 & 2589 \\
    414 & nosey\_jj & 5 & 4211 \\
    415 & formal\_jj & 5 & 5908 \\
    416 & friendly\_jj & 5 & 9776 \\
    417 & bossiness & 6 & 1987 \\
    418 & trustful\_jj & 6 & 2307 \\
    419 & punctual\_jj & 6 & 4845 \\
    420 & prompt\_jj & 7 & 4004 \\
    421 & nonconforming\_jj & 9 & 1655 \\
    \hline
    \caption{Scores and rankings for most extreme 30 words in component \#12} \\
\end{longtable}
\begin{longtable}[!htbp]{| rlr@{.}l |}
    \hline
    \textbf{Rank} & \textbf{Word} & \multicolumn{2}{c|}{\textbf{Score}} \\
    \hline
    \endhead
    1 & abusive\_jj & -24 & -9777 \\
    2 & economical\_jj & -10 & -3321 \\
    3 & refined\_jj & -9 & -456 \\
    4 & thrifty\_jj & -8 & -1101 \\
    5 & stingy\_jj & -6 & -9917 \\
    6 & brave\_jj & -6 & -8772 \\
    7 & courage & -6 & -1708 \\
    8 & selfless\_jj & -5 & -9533 \\
    9 & generosity & -5 & -3941 \\
    10 & punctuality & -5 & -2954 \\
    11 & dignity & -5 & -1490 \\
    12 & efficiency & -5 & -374 \\
    13 & efficient\_jj & -5 & -36 \\
    14 & dependability & -4 & -9879 \\
    15 & rude\_jj & -4 & -9776 \\
    16 & reserve & -4 & -4586 \\
    17 & modesty & -4 & -3517 \\
    18 & thrift & -4 & -3252 \\
    19 & modest\_jj & -3 & -9104 \\
    20 & restrained\_jj & -3 & -8190 \\
    21 & courtesy & -3 & -7977 \\
    22 & pleasant\_jj & -3 & -7261 \\
    23 & decisiveness & -3 & -7057 \\
    24 & sluggish\_jj & -3 & -6572 \\
    25 & persistence & -3 & -5638 \\
    26 & courageous\_jj & -3 & -5502 \\
    27 & ungracious\_jj & -3 & -4118 \\
    28 & optimism & -3 & -3286 \\
    29 & flexibility & -3 & -2827 \\
    30 & dignified\_jj & -3 & -2636 \\
    392 & informal\_jj & 3 & 4766 \\
    393 & thorough\_jj & 3 & 5064 \\
    394 & self-critical\_jj & 3 & 5237 \\
    395 & deceit & 3 & 6015 \\
    396 & autonomous\_jj & 3 & 7046 \\
    397 & humorous\_jj & 3 & 7313 \\
    398 & cordial\_jj & 3 & 7590 \\
    399 & sophistication & 3 & 8737 \\
    400 & wordy\_jj & 4 & 675 \\
    401 & unstable\_jj & 4 & 786 \\
    402 & philosophical\_jj & 4 & 1290 \\
    403 & cooperation & 4 & 4432 \\
    404 & disorganization & 4 & 5403 \\
    405 & secretive\_jj & 4 & 5499 \\
    406 & quarrelsome\_jj & 4 & 5551 \\
    407 & antagonistic\_jj & 4 & 5682 \\
    408 & concise\_jj & 4 & 9612 \\
    409 & touchy\_jj & 4 & 9780 \\
    410 & systematic\_jj & 5 & 453 \\
    411 & deliberate\_jj & 5 & 3753 \\
    412 & negligent\_jj & 5 & 7009 \\
    413 & diplomatic\_jj & 5 & 7410 \\
    414 & folksy\_jj & 5 & 7957 \\
    415 & intelligence & 6 & 6072 \\
    416 & distrust & 6 & 6834 \\
    417 & organized\_jj & 6 & 8274 \\
    418 & suspicious\_jj & 6 & 8411 \\
    419 & unscrupulous\_jj & 6 & 8721 \\
    420 & instability & 7 & 4746 \\
    421 & explosive\_jj & 8 & 373 \\
    \hline
    \caption{Scores and rankings for most extreme 30 words in component \#13} \\
\end{longtable}
\begin{longtable}[!htbp]{| rlr@{.}l |}
    \hline
    \textbf{Rank} & \textbf{Word} & \multicolumn{2}{c|}{\textbf{Score}} \\
    \hline
    \endhead
    1 & nonconformity & -7 & -6241 \\
    2 & individualistic\_jj & -7 & -6089 \\
    3 & meditative\_jj & -7 & -734 \\
    4 & lenient\_jj & -6 & -8618 \\
    5 & peaceful\_jj & -6 & -8281 \\
    6 & cruelty & -6 & -582 \\
    7 & inventive\_jj & -5 & -8780 \\
    8 & rebellious\_jj & -5 & -8356 \\
    9 & principled\_jj & -5 & -7248 \\
    10 & contemplative\_jj & -5 & -6860 \\
    11 & brave\_jj & -5 & -6560 \\
    12 & vigorous\_jj & -5 & -6056 \\
    13 & negligent\_jj & -5 & -4917 \\
    14 & somber\_jj & -5 & -2950 \\
    15 & understanding\_jj & -5 & -2473 \\
    16 & expressiveness & -5 & -2448 \\
    17 & unreflective\_jj & -5 & -1806 \\
    18 & expressive\_jj & -4 & -9748 \\
    19 & adventurous\_jj & -4 & -9578 \\
    20 & charitable\_jj & -4 & -7037 \\
    21 & assertive\_jj & -4 & -6222 \\
    22 & respectful\_jj & -4 & -3791 \\
    23 & accommodating\_jj & -4 & -3770 \\
    24 & silent\_jj & -4 & -2558 \\
    25 & silence & -4 & -2151 \\
    26 & inhibition & -4 & -1318 \\
    27 & dignified\_jj & -4 & -1074 \\
    28 & prejudiced\_jj & -4 & -691 \\
    29 & restrained\_jj & -4 & -555 \\
    30 & courageous\_jj & -4 & -115 \\
    392 & smart\_jj & 3 & 7745 \\
    393 & volatility & 3 & 8629 \\
    394 & predictability & 3 & 9210 \\
    395 & verbose\_jj & 3 & 9214 \\
    396 & dependable\_jj & 4 & 1170 \\
    397 & unscrupulous\_jj & 4 & 1405 \\
    398 & erratic\_jj & 4 & 2967 \\
    399 & candor & 4 & 3658 \\
    400 & warmth & 4 & 4171 \\
    401 & instability & 4 & 4979 \\
    402 & ungracious\_jj & 4 & 5372 \\
    403 & depth & 4 & 5815 \\
    404 & aloofness & 4 & 6693 \\
    405 & conceited\_jj & 4 & 7365 \\
    406 & diplomatic\_jj & 4 & 7677 \\
    407 & warm\_jj & 5 & 2424 \\
    408 & friendly\_jj & 5 & 2823 \\
    409 & humor & 5 & 4748 \\
    410 & abusive\_jj & 5 & 5932 \\
    411 & sophistication & 5 & 7815 \\
    412 & insight & 5 & 8585 \\
    413 & cordial\_jj & 5 & 8669 \\
    414 & belligerence & 6 & 1028 \\
    415 & punctuality & 6 & 3531 \\
    416 & explosive\_jj & 6 & 3927 \\
    417 & unreliable\_jj & 6 & 7753 \\
    418 & intelligence & 7 & 12 \\
    419 & reliable\_jj & 7 & 234 \\
    420 & efficiency & 7 & 325 \\
    421 & dependability & 7 & 8940 \\
    \hline
    \caption{Scores and rankings for most extreme 30 words in component \#14} \\
\end{longtable}
\begin{longtable}[!htbp]{| rlr@{.}l |}
    \hline
    \textbf{Rank} & \textbf{Word} & \multicolumn{2}{c|}{\textbf{Score}} \\
    \hline
    \endhead
    1 & negligent\_jj & -11 & -9153 \\
    2 & negligence & -7 & -2623 \\
    3 & dependability & -6 & -6914 \\
    4 & insight & -6 & -2288 \\
    5 & leniency & -5 & -7782 \\
    6 & careless\_jj & -5 & -4251 \\
    7 & punctuality & -5 & -2983 \\
    8 & adventurous\_jj & -4 & -9110 \\
    9 & animation & -4 & -8222 \\
    10 & reserve & -4 & -7864 \\
    11 & patient\_jj & -4 & -6650 \\
    12 & fastidious\_jj & -4 & -4980 \\
    13 & punctual\_jj & -4 & -4080 \\
    14 & optimistic\_jj & -4 & -3669 \\
    15 & indecisive\_jj & -4 & -3086 \\
    16 & candor & -4 & -2851 \\
    17 & perceptive\_jj & -4 & -1669 \\
    18 & unambitious\_jj & -4 & -1379 \\
    19 & boastful\_jj & -4 & -1082 \\
    20 & surly\_jj & -3 & -9494 \\
    21 & assured\_jj & -3 & -9243 \\
    22 & suspicious\_jj & -3 & -9234 \\
    23 & lenient\_jj & -3 & -8559 \\
    24 & unsociable\_jj & -3 & -7789 \\
    25 & nonconforming\_jj & -3 & -7007 \\
    26 & self-disciplined\_jj & -3 & -6719 \\
    27 & inquisitive\_jj & -3 & -6611 \\
    28 & forgetful\_jj & -3 & -6354 \\
    29 & understanding\_jj & -3 & -6184 \\
    30 & envious\_jj & -3 & -5523 \\
    392 & unfriendly\_jj & 3 & 6356 \\
    393 & touchy\_jj & 3 & 6820 \\
    394 & dull\_jj & 3 & 7366 \\
    395 & predictable\_jj & 3 & 7392 \\
    396 & quiet\_jj & 3 & 8510 \\
    397 & friendly\_jj & 3 & 9022 \\
    398 & earthiness & 3 & 9937 \\
    399 & stupidity & 4 & 206 \\
    400 & unstable\_jj & 4 & 2681 \\
    401 & bigoted\_jj & 4 & 4983 \\
    402 & distrust & 4 & 5062 \\
    403 & dignified\_jj & 4 & 5288 \\
    404 & bitter\_jj & 4 & 5769 \\
    405 & cosmopolitan\_jj & 4 & 9410 \\
    406 & prejudice & 5 & 405 \\
    407 & pleasant\_jj & 5 & 753 \\
    408 & cold\_jj & 5 & 2109 \\
    409 & shallow\_jj & 5 & 6594 \\
    410 & deliberate\_jj & 5 & 6988 \\
    411 & harsh\_jj & 5 & 7149 \\
    412 & folksy\_jj & 6 & 859 \\
    413 & economical\_jj & 6 & 3723 \\
    414 & insecurity & 6 & 5301 \\
    415 & peaceful\_jj & 6 & 7781 \\
    416 & volatile\_jj & 6 & 8586 \\
    417 & instability & 8 & 651 \\
    418 & cruel\_jj & 8 & 3208 \\
    419 & absent-minded\_jj & 9 & 6994 \\
    420 & warm\_jj & 10 & 1193 \\
    421 & refined\_jj & 11 & 3348 \\
    \hline
    \caption{Scores and rankings for most extreme 30 words in component \#15} \\
\end{longtable}

\subsection{MDS}
\label{app:rankedwordlists:438words:mds}
\begin{longtable}[!htbp]{| rlr@{.}l |}
    \hline
    \textbf{Rank} & \textbf{Word} & \multicolumn{2}{c|}{\textbf{Score}} \\
    \hline
    \endhead
    1 & self-pitying\_jj & 0 & -2892 \\
    2 & scatterbrained\_jj & 0 & -2879 \\
    3 & pomposity & 0 & -2815 \\
    4 & mischievous\_jj & 0 & -2752 \\
    5 & sly\_jj & 0 & -2744 \\
    6 & pompous\_jj & 0 & -2743 \\
    7 & snobbish\_jj & 0 & -2659 \\
    8 & genial\_jj & 0 & -2551 \\
    9 & condescending\_jj & 0 & -2469 \\
    10 & shyness & 0 & -2444 \\
    11 & introspective\_jj & 0 & -2425 \\
    12 & exhibitionistic\_jj & 0 & -2399 \\
    13 & self-indulgent\_jj & 0 & -2397 \\
    14 & witty\_jj & 0 & -2380 \\
    15 & playfulness & 0 & -2339 \\
    16 & egotistical\_jj & 0 & -2319 \\
    17 & bossy\_jj & 0 & -2318 \\
    18 & passionless\_jj & 0 & -2317 \\
    19 & gruff\_jj & 0 & -2253 \\
    20 & amiable\_jj & 0 & -2197 \\
    21 & bashful\_jj & 0 & -2191 \\
    22 & vivacious\_jj & 0 & -2184 \\
    23 & amiability & 0 & -2179 \\
    24 & happy-go-lucky\_jj & 0 & -2146 \\
    25 & egocentric\_jj & 0 & -2145 \\
    26 & flippant\_jj & 0 & -2129 \\
    27 & aloofness & 0 & -2086 \\
    28 & extroverted\_jj & 0 & -2082 \\
    29 & boastful\_jj & 0 & -2073 \\
    30 & forgetful\_jj & 0 & -2064 \\
    392 & informal\_jj & 0 & 2691 \\
    393 & dominant\_jj & 0 & 2713 \\
    394 & patient\_jj & 0 & 2759 \\
    395 & confident\_jj & 0 & 2760 \\
    396 & active\_jj & 0 & 2791 \\
    397 & diplomatic\_jj & 0 & 2802 \\
    398 & efficiency & 0 & 2824 \\
    399 & sluggish\_jj & 0 & 2837 \\
    400 & thorough\_jj & 0 & 2850 \\
    401 & careful\_jj & 0 & 2881 \\
    402 & firm\_jj & 0 & 2900 \\
    403 & responsible\_jj & 0 & 2900 \\
    404 & decisive\_jj & 0 & 2908 \\
    405 & cautious\_jj & 0 & 2948 \\
    406 & caution & 0 & 2961 \\
    407 & optimistic\_jj & 0 & 3007 \\
    408 & autonomous\_jj & 0 & 3012 \\
    409 & helpful\_jj & 0 & 3070 \\
    410 & organization & 0 & 3078 \\
    411 & efficient\_jj & 0 & 3137 \\
    412 & flexible\_jj & 0 & 3145 \\
    413 & modest\_jj & 0 & 3150 \\
    414 & formal\_jj & 0 & 3239 \\
    415 & volatile\_jj & 0 & 3275 \\
    416 & reasonable\_jj & 0 & 3357 \\
    417 & flexibility & 0 & 3364 \\
    418 & reliable\_jj & 0 & 3378 \\
    419 & direct\_jj & 0 & 3459 \\
    420 & cooperation & 0 & 3696 \\
    421 & consistent\_jj & 0 & 3897 \\
    \hline
    \caption{Scores and rankings for most extreme 30 words in component \#1} \\
\end{longtable}
\begin{longtable}[!htbp]{| rlr@{.}l |}
    \hline
    \textbf{Rank} & \textbf{Word} & \multicolumn{2}{c|}{\textbf{Score}} \\
    \hline
    \endhead
    1 & sociable\_jj & 0 & -3990 \\
    2 & cheerful\_jj & 0 & -3496 \\
    3 & vivacious\_jj & 0 & -3412 \\
    4 & easygoing\_jj & 0 & -3398 \\
    5 & considerate\_jj & 0 & -3397 \\
    6 & enthusiastic\_jj & 0 & -3370 \\
    7 & dependable\_jj & 0 & -3225 \\
    8 & gregarious\_jj & 0 & -3163 \\
    9 & affectionate\_jj & 0 & -3127 \\
    10 & energetic\_jj & 0 & -3067 \\
    11 & talkative\_jj & 0 & -3020 \\
    12 & down-to-earth\_jj & 0 & -2993 \\
    13 & courteous\_jj & 0 & -2988 \\
    14 & intelligent\_jj & 0 & -2962 \\
    15 & cultured\_jj & 0 & -2933 \\
    16 & adventurous\_jj & 0 & -2927 \\
    17 & kind\_jj & 0 & -2823 \\
    18 & amiable\_jj & 0 & -2815 \\
    19 & pleasant\_jj & 0 & -2771 \\
    20 & witty\_jj & 0 & -2737 \\
    21 & unassuming\_jj & 0 & -2727 \\
    22 & jovial\_jj & 0 & -2558 \\
    23 & inquisitive\_jj & 0 & -2509 \\
    24 & humble\_jj & 0 & -2448 \\
    25 & quiet\_jj & 0 & -2410 \\
    26 & confident\_jj & 0 & -2407 \\
    27 & polite\_jj & 0 & -2368 \\
    28 & flexible\_jj & 0 & -2331 \\
    29 & generous\_jj & 0 & -2276 \\
    30 & agreeable\_jj & 0 & -2229 \\
    392 & unrestrained\_jj & 0 & 2224 \\
    393 & cruelty & 0 & 2224 \\
    394 & intrusiveness & 0 & 2227 \\
    395 & morality & 0 & 2262 \\
    396 & sloth & 0 & 2296 \\
    397 & thoughtless\_jj & 0 & 2306 \\
    398 & moral\_jj & 0 & 2327 \\
    399 & belligerence & 0 & 2376 \\
    400 & shallowness & 0 & 2424 \\
    401 & instability & 0 & 2430 \\
    402 & reckless\_jj & 0 & 2476 \\
    403 & lethargy & 0 & 2646 \\
    404 & inconsistency & 0 & 2684 \\
    405 & rudeness & 0 & 2752 \\
    406 & disorganization & 0 & 2823 \\
    407 & negligence & 0 & 2829 \\
    408 & insecurity & 0 & 2856 \\
    409 & indecisiveness & 0 & 2857 \\
    410 & fear & 0 & 2858 \\
    411 & prejudice & 0 & 2859 \\
    412 & thoughtlessness & 0 & 2872 \\
    413 & stubbornness & 0 & 2885 \\
    414 & distrust & 0 & 2888 \\
    415 & passivity & 0 & 2946 \\
    416 & deceit & 0 & 3386 \\
    417 & recklessness & 0 & 3606 \\
    418 & gullibility & 0 & 3656 \\
    419 & selfishness & 0 & 3687 \\
    420 & callousness & 0 & 3779 \\
    421 & stupidity & 0 & 4079 \\
    \hline
    \caption{Scores and rankings for most extreme 30 words in component \#2} \\
\end{longtable}
\begin{longtable}[!htbp]{| rlr@{.}l |}
    \hline
    \textbf{Rank} & \textbf{Word} & \multicolumn{2}{c|}{\textbf{Score}} \\
    \hline
    \endhead
    1 & ignorant\_jj & 0 & -3469 \\
    2 & dishonest\_jj & 0 & -3272 \\
    3 & insensitive\_jj & 0 & -3153 \\
    4 & greedy\_jj & 0 & -3034 \\
    5 & disrespectful\_jj & 0 & -2984 \\
    6 & bigoted\_jj & 0 & -2926 \\
    7 & unfriendly\_jj & 0 & -2902 \\
    8 & unsympathetic\_jj & 0 & -2834 \\
    9 & inconsiderate\_jj & 0 & -2820 \\
    10 & gullible\_jj & 0 & -2679 \\
    11 & vindictive\_jj & 0 & -2673 \\
    12 & inefficient\_jj & 0 & -2621 \\
    13 & lenient\_jj & 0 & -2447 \\
    14 & pessimistic\_jj & 0 & -2433 \\
    15 & intrusive\_jj & 0 & -2408 \\
    16 & naïve\_jj & 0 & -2401 \\
    17 & lazy\_jj & 0 & -2383 \\
    18 & unreliable\_jj & 0 & -2318 \\
    19 & deceitful\_jj & 0 & -2283 \\
    20 & cynical\_jj & 0 & -2245 \\
    21 & selfish\_jj & 0 & -2235 \\
    22 & unscrupulous\_jj & 0 & -2169 \\
    23 & prejudiced\_jj & 0 & -2165 \\
    24 & impolite\_jj & 0 & -2153 \\
    25 & timid\_jj & 0 & -2145 \\
    26 & thoughtless\_jj & 0 & -2035 \\
    27 & distrustful\_jj & 0 & -1970 \\
    28 & wishy-washy\_jj & 0 & -1964 \\
    29 & rude\_jj & 0 & -1963 \\
    30 & egotistical\_jj & 0 & -1949 \\
    392 & homespun\_jj & 0 & 2031 \\
    393 & gregariousness & 0 & 2038 \\
    394 & earthy\_jj & 0 & 2060 \\
    395 & dependability & 0 & 2060 \\
    396 & predictability & 0 & 2064 \\
    397 & curiosity & 0 & 2068 \\
    398 & understanding & 0 & 2081 \\
    399 & empathy & 0 & 2167 \\
    400 & ambition & 0 & 2199 \\
    401 & optimism & 0 & 2263 \\
    402 & generosity & 0 & 2285 \\
    403 & courage & 0 & 2342 \\
    404 & decisiveness & 0 & 2452 \\
    405 & meditative\_jj & 0 & 2542 \\
    406 & earthiness & 0 & 2561 \\
    407 & depth & 0 & 2598 \\
    408 & flexibility & 0 & 2606 \\
    409 & sophistication & 0 & 2651 \\
    410 & naturalness & 0 & 2727 \\
    411 & candor & 0 & 2752 \\
    412 & humor & 0 & 2769 \\
    413 & expressiveness & 0 & 2779 \\
    414 & artistic\_jj & 0 & 2836 \\
    415 & spirit & 0 & 2973 \\
    416 & persistence & 0 & 2990 \\
    417 & playfulness & 0 & 3330 \\
    418 & spontaneity & 0 & 3429 \\
    419 & precision & 0 & 3560 \\
    420 & warmth & 0 & 3576 \\
    421 & creativity & 0 & 3693 \\
    \hline
    \caption{Scores and rankings for most extreme 30 words in component \#3} \\
\end{longtable}
\begin{longtable}[!htbp]{| rlr@{.}l |}
    \hline
    \textbf{Rank} & \textbf{Word} & \multicolumn{2}{c|}{\textbf{Score}} \\
    \hline
    \endhead
    1 & innovative\_jj & 0 & -2810 \\
    2 & dishonest\_jj & 0 & -2432 \\
    3 & intelligent\_jj & 0 & -2392 \\
    4 & ethical\_jj & 0 & -2332 \\
    5 & imaginative\_jj & 0 & -2248 \\
    6 & truthful\_jj & 0 & -2208 \\
    7 & honest\_jj & 0 & -2180 \\
    8 & principled\_jj & 0 & -2178 \\
    9 & analytical\_jj & 0 & -2126 \\
    10 & courageous\_jj & 0 & -2117 \\
    11 & insightful\_jj & 0 & -2058 \\
    12 & cunning\_jj & 0 & -2055 \\
    13 & creative\_jj & 0 & -2012 \\
    14 & logic & 0 & -1955 \\
    15 & moral\_jj & 0 & -1851 \\
    16 & adaptable\_jj & 0 & -1826 \\
    17 & sophisticated\_jj & 0 & -1793 \\
    18 & selfless\_jj & 0 & -1779 \\
    19 & efficient\_jj & 0 & -1777 \\
    20 & systematic\_jj & 0 & -1770 \\
    21 & morality & 0 & -1753 \\
    22 & intellectual\_jj & 0 & -1723 \\
    23 & logical\_jj & 0 & -1721 \\
    24 & dignity & 0 & -1717 \\
    25 & straightforward\_jj & 0 & -1708 \\
    26 & devious\_jj & 0 & -1697 \\
    27 & creativity & 0 & -1695 \\
    28 & unintelligent\_jj & 0 & -1663 \\
    29 & intellectuality & 0 & -1622 \\
    30 & deceitful\_jj & 0 & -1614 \\
    392 & cranky\_jj & 0 & 1632 \\
    393 & quiet\_jj & 0 & 1678 \\
    394 & aimlessness & 0 & 1680 \\
    395 & dominant\_jj & 0 & 1692 \\
    396 & pessimism & 0 & 1711 \\
    397 & cautious\_jj & 0 & 1764 \\
    398 & insecurity & 0 & 1770 \\
    399 & warm\_jj & 0 & 1783 \\
    400 & pessimistic\_jj & 0 & 1796 \\
    401 & reserve & 0 & 1832 \\
    402 & somber\_jj & 0 & 1874 \\
    403 & optimistic\_jj & 0 & 1903 \\
    404 & quiet & 0 & 1911 \\
    405 & volatile\_jj & 0 & 1978 \\
    406 & volatility & 0 & 2067 \\
    407 & persistent\_jj & 0 & 2081 \\
    408 & instability & 0 & 2100 \\
    409 & steady\_jj & 0 & 2114 \\
    410 & bitter\_jj & 0 & 2425 \\
    411 & fretful\_jj & 0 & 2435 \\
    412 & irritability & 0 & 2465 \\
    413 & optimism & 0 & 2474 \\
    414 & fearful\_jj & 0 & 2531 \\
    415 & lethargy & 0 & 2614 \\
    416 & irritable\_jj & 0 & 2704 \\
    417 & cold\_jj & 0 & 2799 \\
    418 & anxious\_jj & 0 & 2889 \\
    419 & lethargic\_jj & 0 & 3009 \\
    420 & sluggish\_jj & 0 & 3334 \\
    421 & nervous\_jj & 0 & 3340 \\
    \hline
    \caption{Scores and rankings for most extreme 30 words in component \#4} \\
\end{longtable}
\begin{longtable}[!htbp]{| rlr@{.}l |}
    \hline
    \textbf{Rank} & \textbf{Word} & \multicolumn{2}{c|}{\textbf{Score}} \\
    \hline
    \endhead
    1 & innovative\_jj & 0 & -2590 \\
    2 & complex\_jj & 0 & -2548 \\
    3 & efficient\_jj & 0 & -2391 \\
    4 & inefficient\_jj & 0 & -2309 \\
    5 & conventional\_jj & 0 & -2295 \\
    6 & extravagant\_jj & 0 & -2172 \\
    7 & refined\_jj & 0 & -2167 \\
    8 & sophisticated\_jj & 0 & -2096 \\
    9 & traditional\_jj & 0 & -2032 \\
    10 & undemanding\_jj & 0 & -2012 \\
    11 & unimaginative\_jj & 0 & -1903 \\
    12 & unconventional\_jj & 0 & -1880 \\
    13 & inventive\_jj & 0 & -1867 \\
    14 & economical\_jj & 0 & -1826 \\
    15 & undependable\_jj & 0 & -1823 \\
    16 & animation & 0 & -1751 \\
    17 & flexible\_jj & 0 & -1684 \\
    18 & slothful\_jj & 0 & -1662 \\
    19 & nonconforming\_jj & 0 & -1642 \\
    20 & conventionality & 0 & -1624 \\
    21 & impractical\_jj & 0 & -1606 \\
    22 & unaggressive\_jj & 0 & -1579 \\
    23 & impersonal\_jj & 0 & -1559 \\
    24 & adaptable\_jj & 0 & -1527 \\
    25 & imaginative\_jj & 0 & -1516 \\
    26 & ambitious\_jj & 0 & -1478 \\
    27 & autonomous\_jj & 0 & -1473 \\
    28 & exacting\_jj & 0 & -1466 \\
    29 & unsociable\_jj & 0 & -1466 \\
    30 & demanding\_jj & 0 & -1416 \\
    392 & cordial\_jj & 0 & 1602 \\
    393 & skeptical\_jj & 0 & 1627 \\
    394 & leniency & 0 & 1653 \\
    395 & prejudice & 0 & 1664 \\
    396 & disrespectful\_jj & 0 & 1698 \\
    397 & modesty & 0 & 1702 \\
    398 & quiet\_jj & 0 & 1713 \\
    399 & brave\_jj & 0 & 1716 \\
    400 & silence & 0 & 1724 \\
    401 & fear & 0 & 1728 \\
    402 & courteous\_jj & 0 & 1731 \\
    403 & confident\_jj & 0 & 1732 \\
    404 & considerate\_jj & 0 & 1761 \\
    405 & principled\_jj & 0 & 1825 \\
    406 & empathy & 0 & 2012 \\
    407 & spirit & 0 & 2066 \\
    408 & selfless\_jj & 0 & 2100 \\
    409 & generosity & 0 & 2131 \\
    410 & polite\_jj & 0 & 2131 \\
    411 & respectful\_jj & 0 & 2140 \\
    412 & candor & 0 & 2172 \\
    413 & kind\_jj & 0 & 2214 \\
    414 & courageous\_jj & 0 & 2290 \\
    415 & proud\_jj & 0 & 2406 \\
    416 & frank\_jj & 0 & 2447 \\
    417 & honest\_jj & 0 & 2471 \\
    418 & optimism & 0 & 2530 \\
    419 & dignity & 0 & 2850 \\
    420 & courage & 0 & 3026 \\
    421 & sincere\_jj & 0 & 3605 \\
    \hline
    \caption{Scores and rankings for most extreme 30 words in component \#5} \\
\end{longtable}
\begin{longtable}[!htbp]{| rlr@{.}l |}
    \hline
    \textbf{Rank} & \textbf{Word} & \multicolumn{2}{c|}{\textbf{Score}} \\
    \hline
    \endhead
    1 & organization & 0 & -2410 \\
    2 & proud\_jj & 0 & -2379 \\
    3 & autonomous\_jj & 0 & -2287 \\
    4 & fear & 0 & -2253 \\
    5 & active\_jj & 0 & -2176 \\
    6 & independent\_jj & 0 & -2155 \\
    7 & sociable\_jj & 0 & -2133 \\
    8 & kind\_jj & 0 & -1885 \\
    9 & adventurous\_jj & 0 & -1872 \\
    10 & intelligent\_jj & 0 & -1848 \\
    11 & envy & 0 & -1832 \\
    12 & envious\_jj & 0 & -1820 \\
    13 & enterprising\_jj & 0 & -1812 \\
    14 & creative\_jj & 0 & -1673 \\
    15 & insecurity & 0 & -1655 \\
    16 & insecure\_jj & 0 & -1650 \\
    17 & fearful\_jj & 0 & -1633 \\
    18 & considerate\_jj & 0 & -1633 \\
    19 & greedy\_jj & 0 & -1623 \\
    20 & curiosity & 0 & -1573 \\
    21 & vivacious\_jj & 0 & -1552 \\
    22 & ambitious\_jj & 0 & -1531 \\
    23 & independence & 0 & -1522 \\
    24 & gullible\_jj & 0 & -1520 \\
    25 & cultured\_jj & 0 & -1515 \\
    26 & ambition & 0 & -1476 \\
    27 & ignorant\_jj & 0 & -1465 \\
    28 & distrustful\_jj & 0 & -1453 \\
    29 & jealous\_jj & 0 & -1441 \\
    30 & charitable\_jj & 0 & -1435 \\
    392 & dull\_jj & 0 & 1498 \\
    393 & careful\_jj & 0 & 1511 \\
    394 & cordial\_jj & 0 & 1516 \\
    395 & concise\_jj & 0 & 1536 \\
    396 & rash\_jj & 0 & 1557 \\
    397 & inconsistency & 0 & 1574 \\
    398 & harsh\_jj & 0 & 1579 \\
    399 & spirited\_jj & 0 & 1596 \\
    400 & frank\_jj & 0 & 1599 \\
    401 & decisive\_jj & 0 & 1618 \\
    402 & truthful\_jj & 0 & 1637 \\
    403 & haphazard\_jj & 0 & 1667 \\
    404 & unemotional\_jj & 0 & 1670 \\
    405 & verbose\_jj & 0 & 1696 \\
    406 & precise\_jj & 0 & 1698 \\
    407 & tempestuous\_jj & 0 & 1767 \\
    408 & predictable\_jj & 0 & 1816 \\
    409 & deliberate\_jj & 0 & 1823 \\
    410 & curt\_jj & 0 & 1959 \\
    411 & caustic\_jj & 0 & 1970 \\
    412 & verbal\_jj & 0 & 2020 \\
    413 & consistent\_jj & 0 & 2027 \\
    414 & somber\_jj & 0 & 2048 \\
    415 & vigorous\_jj & 0 & 2063 \\
    416 & combative\_jj & 0 & 2107 \\
    417 & inconsistent\_jj & 0 & 2173 \\
    418 & straightforward\_jj & 0 & 2402 \\
    419 & restrained\_jj & 0 & 2414 \\
    420 & sloppy\_jj & 0 & 2578 \\
    421 & forceful\_jj & 0 & 2671 \\
    \hline
    \caption{Scores and rankings for most extreme 30 words in component \#6} \\
\end{longtable}
\begin{longtable}[!htbp]{| rlr@{.}l |}
    \hline
    \textbf{Rank} & \textbf{Word} & \multicolumn{2}{c|}{\textbf{Score}} \\
    \hline
    \endhead
    1 & tenacious\_jj & 0 & -2298 \\
    2 & ruthless\_jj & 0 & -2188 \\
    3 & flamboyant\_jj & 0 & -2149 \\
    4 & crafty\_jj & 0 & -2063 \\
    5 & cunning\_jj & 0 & -2049 \\
    6 & daring & 0 & -1969 \\
    7 & sloppy\_jj & 0 & -1905 \\
    8 & brave\_jj & 0 & -1893 \\
    9 & daring\_jj & 0 & -1855 \\
    10 & defensive\_jj & 0 & -1693 \\
    11 & extravagant\_jj & 0 & -1684 \\
    12 & shy\_jj & 0 & -1658 \\
    13 & decisive\_jj & 0 & -1650 \\
    14 & cunning & 0 & -1646 \\
    15 & ambition & 0 & -1611 \\
    16 & vain\_jj & 0 & -1605 \\
    17 & ambitious\_jj & 0 & -1541 \\
    18 & spirited\_jj & 0 & -1536 \\
    19 & courtesy & 0 & -1524 \\
    20 & stubborn\_jj & 0 & -1483 \\
    21 & dominant\_jj & 0 & -1454 \\
    22 & homespun\_jj & 0 & -1443 \\
    23 & bright\_jj & 0 & -1410 \\
    24 & reckless\_jj & 0 & -1401 \\
    25 & sly\_jj & 0 & -1383 \\
    26 & stingy\_jj & 0 & -1360 \\
    27 & proud\_jj & 0 & -1357 \\
    28 & modest\_jj & 0 & -1344 \\
    29 & smart\_jj & 0 & -1325 \\
    30 & devious\_jj & 0 & -1318 \\
    392 & emotional\_jj & 0 & 1337 \\
    393 & self-critical\_jj & 0 & 1347 \\
    394 & courteous\_jj & 0 & 1360 \\
    395 & talkative\_jj & 0 & 1374 \\
    396 & disorganization & 0 & 1383 \\
    397 & agreeable\_jj & 0 & 1471 \\
    398 & self-disciplined\_jj & 0 & 1475 \\
    399 & cooperation & 0 & 1495 \\
    400 & rudeness & 0 & 1511 \\
    401 & unsociable\_jj & 0 & 1619 \\
    402 & inhibition & 0 & 1627 \\
    403 & self-esteem & 0 & 1640 \\
    404 & punctual\_jj & 0 & 1669 \\
    405 & understanding\_jj & 0 & 1685 \\
    406 & understanding & 0 & 1693 \\
    407 & cordial\_jj & 0 & 1717 \\
    408 & communicative\_jj & 0 & 1743 \\
    409 & patient\_jj & 0 & 1755 \\
    410 & nonconforming\_jj & 0 & 1772 \\
    411 & cooperative\_jj & 0 & 1777 \\
    412 & antagonistic\_jj & 0 & 1969 \\
    413 & forgetfulness & 0 & 1981 \\
    414 & extroverted\_jj & 0 & 2038 \\
    415 & respectful\_jj & 0 & 2103 \\
    416 & irritable\_jj & 0 & 2108 \\
    417 & suggestible\_jj & 0 & 2202 \\
    418 & talkativeness & 0 & 2234 \\
    419 & unsystematic\_jj & 0 & 2464 \\
    420 & irritability & 0 & 2467 \\
    421 & trustful\_jj & 0 & 3167 \\
    \hline
    \caption{Scores and rankings for most extreme 30 words in component \#7} \\
\end{longtable}
\begin{longtable}[!htbp]{| rlr@{.}l |}
    \hline
    \textbf{Rank} & \textbf{Word} & \multicolumn{2}{c|}{\textbf{Score}} \\
    \hline
    \endhead
    1 & simple\_jj & 0 & -2710 \\
    2 & careless\_jj & 0 & -2217 \\
    3 & lazy\_jj & 0 & -2110 \\
    4 & pleasant\_jj & 0 & -2074 \\
    5 & patient\_jj & 0 & -2021 \\
    6 & cold\_jj & 0 & -1970 \\
    7 & negligence & 0 & -1903 \\
    8 & warm\_jj & 0 & -1879 \\
    9 & shallow\_jj & 0 & -1800 \\
    10 & courtesy & 0 & -1686 \\
    11 & bright\_jj & 0 & -1639 \\
    12 & casual\_jj & 0 & -1636 \\
    13 & merry\_jj & 0 & -1610 \\
    14 & negligent\_jj & 0 & -1580 \\
    15 & irritability & 0 & -1575 \\
    16 & smart\_jj & 0 & -1545 \\
    17 & quiet\_jj & 0 & -1528 \\
    18 & cruelty & 0 & -1525 \\
    19 & unsociable\_jj & 0 & -1491 \\
    20 & crabby\_jj & 0 & -1490 \\
    21 & reasonable\_jj & 0 & -1487 \\
    22 & inconsiderate\_jj & 0 & -1467 \\
    23 & rude\_jj & 0 & -1445 \\
    24 & dull\_jj & 0 & -1419 \\
    25 & forgetfulness & 0 & -1403 \\
    26 & shy\_jj & 0 & -1385 \\
    27 & cruel\_jj & 0 & -1358 \\
    28 & scatterbrained\_jj & 0 & -1347 \\
    29 & quiet & 0 & -1344 \\
    30 & nosey\_jj & 0 & -1332 \\
    392 & courageous\_jj & 0 & 1298 \\
    393 & dominant\_jj & 0 & 1309 \\
    394 & secretive\_jj & 0 & 1315 \\
    395 & cautious\_jj & 0 & 1317 \\
    396 & uncritical\_jj & 0 & 1409 \\
    397 & cooperation & 0 & 1411 \\
    398 & sympathetic\_jj & 0 & 1435 \\
    399 & demonstrative\_jj & 0 & 1441 \\
    400 & ruthless\_jj & 0 & 1534 \\
    401 & exacting\_jj & 0 & 1534 \\
    402 & aloofness & 0 & 1561 \\
    403 & tenacious\_jj & 0 & 1565 \\
    404 & principled\_jj & 0 & 1568 \\
    405 & unsympathetic\_jj & 0 & 1586 \\
    406 & ambitious\_jj & 0 & 1596 \\
    407 & belligerence & 0 & 1661 \\
    408 & forceful\_jj & 0 & 1715 \\
    409 & indecisive\_jj & 0 & 1786 \\
    410 & decisiveness & 0 & 1804 \\
    411 & ambition & 0 & 1819 \\
    412 & stubbornness & 0 & 1822 \\
    413 & distrust & 0 & 1838 \\
    414 & stubborn\_jj & 0 & 1961 \\
    415 & obstinate\_jj & 0 & 2044 \\
    416 & accommodating\_jj & 0 & 2137 \\
    417 & combative\_jj & 0 & 2268 \\
    418 & distrustful\_jj & 0 & 2311 \\
    419 & individualistic\_jj & 0 & 2359 \\
    420 & antagonistic\_jj & 0 & 2674 \\
    421 & assertive\_jj & 0 & 3646 \\
    \hline
    \caption{Scores and rankings for most extreme 30 words in component \#8} \\
\end{longtable}
\begin{longtable}[!htbp]{| rlr@{.}l |}
    \hline
    \textbf{Rank} & \textbf{Word} & \multicolumn{2}{c|}{\textbf{Score}} \\
    \hline
    \endhead
    1 & irritability & 0 & -2224 \\
    2 & nervous\_jj & 0 & -2130 \\
    3 & insecurity & 0 & -1930 \\
    4 & unpredictable\_jj & 0 & -1897 \\
    5 & irritable\_jj & 0 & -1852 \\
    6 & fearful\_jj & 0 & -1844 \\
    7 & forgetfulness & 0 & -1800 \\
    8 & lethargy & 0 & -1774 \\
    9 & emotional\_jj & 0 & -1769 \\
    10 & insecure\_jj & 0 & -1749 \\
    11 & pessimistic\_jj & 0 & -1683 \\
    12 & adaptable\_jj & 0 & -1653 \\
    13 & imaginative\_jj & 0 & -1646 \\
    14 & suggestible\_jj & 0 & -1604 \\
    15 & predictable\_jj & 0 & -1572 \\
    16 & unstable\_jj & 0 & -1501 \\
    17 & dull\_jj & 0 & -1481 \\
    18 & cynical\_jj & 0 & -1457 \\
    19 & optimistic\_jj & 0 & -1439 \\
    20 & logic & 0 & -1437 \\
    21 & insightful\_jj & 0 & -1416 \\
    22 & confident\_jj & 0 & -1414 \\
    23 & cautious\_jj & 0 & -1399 \\
    24 & complex\_jj & 0 & -1398 \\
    25 & self-esteem & 0 & -1389 \\
    26 & pessimism & 0 & -1383 \\
    27 & persistent\_jj & 0 & -1382 \\
    28 & curiosity & 0 & -1381 \\
    29 & lethargic\_jj & 0 & -1358 \\
    30 & creativity & 0 & -1344 \\
    392 & bossiness & 0 & 1242 \\
    393 & genial\_jj & 0 & 1248 \\
    394 & prompt\_jj & 0 & 1257 \\
    395 & antagonistic\_jj & 0 & 1281 \\
    396 & crabby\_jj & 0 & 1287 \\
    397 & bullheaded\_jj & 0 & 1310 \\
    398 & peaceful\_jj & 0 & 1354 \\
    399 & unsystematic\_jj & 0 & 1423 \\
    400 & unexcitable\_jj & 0 & 1430 \\
    401 & unassuming\_jj & 0 & 1432 \\
    402 & reserve & 0 & 1464 \\
    403 & intelligence & 0 & 1473 \\
    404 & meddlesome\_jj & 0 & 1521 \\
    405 & gruff\_jj & 0 & 1569 \\
    406 & curt\_jj & 0 & 1606 \\
    407 & obliging\_jj & 0 & 1675 \\
    408 & informal\_jj & 0 & 1695 \\
    409 & direct\_jj & 0 & 1714 \\
    410 & charitable\_jj & 0 & 1720 \\
    411 & leniency & 0 & 1721 \\
    412 & friendly\_jj & 0 & 1748 \\
    413 & foresighted\_jj & 0 & 1768 \\
    414 & independent\_jj & 0 & 1826 \\
    415 & cordial\_jj & 0 & 1826 \\
    416 & intellectuality & 0 & 1895 \\
    417 & independence & 0 & 1989 \\
    418 & cooperation & 0 & 2272 \\
    419 & formal\_jj & 0 & 2451 \\
    420 & organization & 0 & 2550 \\
    421 & diplomatic\_jj & 0 & 2832 \\
    \hline
    \caption{Scores and rankings for most extreme 30 words in component \#9} \\
\end{longtable}
\begin{longtable}[!htbp]{| rlr@{.}l |}
    \hline
    \textbf{Rank} & \textbf{Word} & \multicolumn{2}{c|}{\textbf{Score}} \\
    \hline
    \endhead
    1 & lethargic\_jj & 0 & -2138 \\
    2 & erratic\_jj & 0 & -1970 \\
    3 & tenacious\_jj & 0 & -1960 \\
    4 & negligent\_jj & 0 & -1942 \\
    5 & irritability & 0 & -1942 \\
    6 & careless\_jj & 0 & -1902 \\
    7 & selfless\_jj & 0 & -1877 \\
    8 & reckless\_jj & 0 & -1875 \\
    9 & recklessness & 0 & -1842 \\
    10 & negligence & 0 & -1828 \\
    11 & punctuality & 0 & -1825 \\
    12 & lethargy & 0 & -1692 \\
    13 & dependability & 0 & -1676 \\
    14 & rash\_jj & 0 & -1634 \\
    15 & persistence & 0 & -1522 \\
    16 & efficiency & 0 & -1494 \\
    17 & patient\_jj & 0 & -1472 \\
    18 & irritable\_jj & 0 & -1464 \\
    19 & dependable\_jj & 0 & -1425 \\
    20 & forgetfulness & 0 & -1420 \\
    21 & indecisive\_jj & 0 & -1358 \\
    22 & miserly\_jj & 0 & -1355 \\
    23 & withdrawn\_jj & 0 & -1346 \\
    24 & economical\_jj & 0 & -1330 \\
    25 & impetuous\_jj & 0 & -1271 \\
    26 & defensive\_jj & 0 & -1252 \\
    27 & courteous\_jj & 0 & -1249 \\
    28 & energetic\_jj & 0 & -1248 \\
    29 & sloppy\_jj & 0 & -1218 \\
    30 & industrious\_jj & 0 & -1180 \\
    392 & unkind\_jj & 0 & 1225 \\
    393 & wordy\_jj & 0 & 1229 \\
    394 & melancholic\_jj & 0 & 1246 \\
    395 & prejudice & 0 & 1258 \\
    396 & gullible\_jj & 0 & 1267 \\
    397 & ignorant\_jj & 0 & 1300 \\
    398 & complex\_jj & 0 & 1318 \\
    399 & exhibitionistic\_jj & 0 & 1320 \\
    400 & simple\_jj & 0 & 1321 \\
    401 & unobservant\_jj & 0 & 1335 \\
    402 & cynical\_jj & 0 & 1379 \\
    403 & uncritical\_jj & 0 & 1414 \\
    404 & unaggressive\_jj & 0 & 1434 \\
    405 & prejudiced\_jj & 0 & 1444 \\
    406 & conceit & 0 & 1447 \\
    407 & nonconformity & 0 & 1463 \\
    408 & logical\_jj & 0 & 1471 \\
    409 & sentimental\_jj & 0 & 1499 \\
    410 & cosmopolitan\_jj & 0 & 1510 \\
    411 & distrust & 0 & 1555 \\
    412 & silent\_jj & 0 & 1575 \\
    413 & formal\_jj & 0 & 1680 \\
    414 & curious\_jj & 0 & 1730 \\
    415 & distrustful\_jj & 0 & 1818 \\
    416 & traditional\_jj & 0 & 1864 \\
    417 & skeptical\_jj & 0 & 1901 \\
    418 & touchy\_jj & 0 & 1947 \\
    419 & morality & 0 & 2000 \\
    420 & logic & 0 & 2186 \\
    421 & philosophical\_jj & 0 & 2841 \\
    \hline
    \caption{Scores and rankings for most extreme 30 words in component \#10} \\
\end{longtable}
\begin{longtable}[!htbp]{| rlr@{.}l |}
    \hline
    \textbf{Rank} & \textbf{Word} & \multicolumn{2}{c|}{\textbf{Score}} \\
    \hline
    \endhead
    1 & unstable\_jj & 0 & -2408 \\
    2 & systematic\_jj & 0 & -2347 \\
    3 & cruelty & 0 & -2294 \\
    4 & organized\_jj & 0 & -2156 \\
    5 & unpredictable\_jj & 0 & -2111 \\
    6 & emotional\_jj & 0 & -2035 \\
    7 & spontaneous\_jj & 0 & -2005 \\
    8 & cruel\_jj & 0 & -2001 \\
    9 & rebellious\_jj & 0 & -1782 \\
    10 & instability & 0 & -1689 \\
    11 & inventive\_jj & 0 & -1630 \\
    12 & peaceful\_jj & 0 & -1522 \\
    13 & complex\_jj & 0 & -1481 \\
    14 & silence & 0 & -1434 \\
    15 & deliberate\_jj & 0 & -1411 \\
    16 & orderly\_jj & 0 & -1393 \\
    17 & bitter\_jj & 0 & -1371 \\
    18 & explosive\_jj & 0 & -1369 \\
    19 & informal\_jj & 0 & -1354 \\
    20 & manipulative\_jj & 0 & -1335 \\
    21 & vigorous\_jj & 0 & -1308 \\
    22 & volatile\_jj & 0 & -1289 \\
    23 & negligent\_jj & 0 & -1289 \\
    24 & devious\_jj & 0 & -1281 \\
    25 & energetic\_jj & 0 & -1244 \\
    26 & reckless\_jj & 0 & -1221 \\
    27 & caustic\_jj & 0 & -1192 \\
    28 & secretive\_jj & 0 & -1173 \\
    29 & inquisitive\_jj & 0 & -1138 \\
    30 & unscrupulous\_jj & 0 & -1115 \\
    392 & decisiveness & 0 & 1192 \\
    393 & undemanding\_jj & 0 & 1212 \\
    394 & accommodating\_jj & 0 & 1226 \\
    395 & unfriendliness & 0 & 1236 \\
    396 & bashful\_jj & 0 & 1259 \\
    397 & punctual\_jj & 0 & 1268 \\
    398 & wishy-washy\_jj & 0 & 1291 \\
    399 & dependable\_jj & 0 & 1292 \\
    400 & cautious\_jj & 0 & 1361 \\
    401 & unadventurous\_jj & 0 & 1396 \\
    402 & confident\_jj & 0 & 1409 \\
    403 & truthful\_jj & 0 & 1410 \\
    404 & miserly\_jj & 0 & 1433 \\
    405 & consistent\_jj & 0 & 1435 \\
    406 & unambitious\_jj & 0 & 1480 \\
    407 & predictability & 0 & 1482 \\
    408 & optimistic\_jj & 0 & 1488 \\
    409 & optimism & 0 & 1521 \\
    410 & stinginess & 0 & 1587 \\
    411 & flexibility & 0 & 1600 \\
    412 & reliable\_jj & 0 & 1633 \\
    413 & modest\_jj & 0 & 1641 \\
    414 & smug\_jj & 0 & 1735 \\
    415 & envious\_jj & 0 & 1902 \\
    416 & pessimistic\_jj & 0 & 1903 \\
    417 & thrifty\_jj & 0 & 1978 \\
    418 & dependability & 0 & 2132 \\
    419 & stingy\_jj & 0 & 2170 \\
    420 & punctuality & 0 & 2406 \\
    421 & efficiency & 0 & 2536 \\
    \hline
    \caption{Scores and rankings for most extreme 30 words in component \#11} \\
\end{longtable}
\begin{longtable}[!htbp]{| rlr@{.}l |}
    \hline
    \textbf{Rank} & \textbf{Word} & \multicolumn{2}{c|}{\textbf{Score}} \\
    \hline
    \endhead
    1 & assertion & 0 & -2127 \\
    2 & intelligence & 0 & -2123 \\
    3 & explosive\_jj & 0 & -2060 \\
    4 & insight & 0 & -2035 \\
    5 & suspicious\_jj & 0 & -1996 \\
    6 & skeptical\_jj & 0 & -1869 \\
    7 & analytical\_jj & 0 & -1708 \\
    8 & verbal\_jj & 0 & -1659 \\
    9 & enthusiastic\_jj & 0 & -1629 \\
    10 & insightful\_jj & 0 & -1624 \\
    11 & nervous\_jj & 0 & -1603 \\
    12 & jealous\_jj & 0 & -1554 \\
    13 & anxious\_jj & 0 & -1455 \\
    14 & precision & 0 & -1383 \\
    15 & unkind\_jj & 0 & -1374 \\
    16 & animation & 0 & -1357 \\
    17 & passionless\_jj & 0 & -1333 \\
    18 & perceptive\_jj & 0 & -1333 \\
    19 & unreliable\_jj & 0 & -1315 \\
    20 & shy\_jj & 0 & -1314 \\
    21 & defensive\_jj & 0 & -1309 \\
    22 & depth & 0 & -1304 \\
    23 & verbose\_jj & 0 & -1245 \\
    24 & boastful\_jj & 0 & -1239 \\
    25 & sly\_jj & 0 & -1199 \\
    26 & unexcitable\_jj & 0 & -1194 \\
    27 & envious\_jj & 0 & -1194 \\
    28 & frank\_jj & 0 & -1186 \\
    29 & assured\_jj & 0 & -1172 \\
    30 & independent\_jj & 0 & -1156 \\
    392 & moral\_jj & 0 & 1209 \\
    393 & carefree\_jj & 0 & 1210 \\
    394 & humble\_jj & 0 & 1221 \\
    395 & restrained\_jj & 0 & 1230 \\
    396 & miserly\_jj & 0 & 1245 \\
    397 & frivolity & 0 & 1248 \\
    398 & reserved\_jj & 0 & 1250 \\
    399 & generous\_jj & 0 & 1287 \\
    400 & selfless\_jj & 0 & 1289 \\
    401 & predictable\_jj & 0 & 1299 \\
    402 & aimlessness & 0 & 1368 \\
    403 & docile\_jj & 0 & 1380 \\
    404 & foolhardy\_jj & 0 & 1426 \\
    405 & nonconformity & 0 & 1466 \\
    406 & impractical\_jj & 0 & 1496 \\
    407 & unrestrained\_jj & 0 & 1523 \\
    408 & cosmopolitan\_jj & 0 & 1565 \\
    409 & flexible\_jj & 0 & 1568 \\
    410 & selfishness & 0 & 1599 \\
    411 & dignity & 0 & 1611 \\
    412 & accommodating\_jj & 0 & 1637 \\
    413 & traditional\_jj & 0 & 1637 \\
    414 & benevolent\_jj & 0 & 1643 \\
    415 & quiet\_jj & 0 & 1704 \\
    416 & modest\_jj & 0 & 1806 \\
    417 & pleasant\_jj & 0 & 1810 \\
    418 & individualistic\_jj & 0 & 2027 \\
    419 & peaceful\_jj & 0 & 2157 \\
    420 & thrifty\_jj & 0 & 2229 \\
    421 & dignified\_jj & 0 & 2279 \\
    \hline
    \caption{Scores and rankings for most extreme 30 words in component \#12} \\
\end{longtable}
\begin{longtable}[!htbp]{| rlr@{.}l |}
    \hline
    \textbf{Rank} & \textbf{Word} & \multicolumn{2}{c|}{\textbf{Score}} \\
    \hline
    \endhead
    1 & understanding\_jj & 0 & -1855 \\
    2 & vigorous\_jj & 0 & -1730 \\
    3 & brave\_jj & 0 & -1687 \\
    4 & nonconformity & 0 & -1624 \\
    5 & reserve & 0 & -1542 \\
    6 & quiet & 0 & -1533 \\
    7 & animation & 0 & -1522 \\
    8 & fretful\_jj & 0 & -1474 \\
    9 & artistic\_jj & 0 & -1378 \\
    10 & daring & 0 & -1363 \\
    11 & meditative\_jj & 0 & -1342 \\
    12 & adventurous\_jj & 0 & -1338 \\
    13 & inventive\_jj & 0 & -1309 \\
    14 & optimistic\_jj & 0 & -1308 \\
    15 & negligent\_jj & 0 & -1303 \\
    16 & leniency & 0 & -1283 \\
    17 & unsociable\_jj & 0 & -1280 \\
    18 & inhibition & 0 & -1264 \\
    19 & rash\_jj & 0 & -1264 \\
    20 & suggestible\_jj & 0 & -1240 \\
    21 & rambunctious\_jj & 0 & -1196 \\
    22 & unaggressive\_jj & 0 & -1191 \\
    23 & talkativeness & 0 & -1179 \\
    24 & absent-minded\_jj & 0 & -1164 \\
    25 & bullheaded\_jj & 0 & -1147 \\
    26 & vain\_jj & 0 & -1144 \\
    27 & impudent\_jj & 0 & -1138 \\
    28 & charitable\_jj & 0 & -1136 \\
    29 & cautious\_jj & 0 & -1113 \\
    30 & confident\_jj & 0 & -1101 \\
    392 & unpredictable\_jj & 0 & 1114 \\
    393 & earthy\_jj & 0 & 1119 \\
    394 & depth & 0 & 1139 \\
    395 & gregarious\_jj & 0 & 1140 \\
    396 & predictable\_jj & 0 & 1168 \\
    397 & insecure\_jj & 0 & 1215 \\
    398 & easygoing\_jj & 0 & 1240 \\
    399 & shallow\_jj & 0 & 1271 \\
    400 & predictability & 0 & 1303 \\
    401 & deceit & 0 & 1321 \\
    402 & envy & 0 & 1339 \\
    403 & aloofness & 0 & 1357 \\
    404 & unstable\_jj & 0 & 1431 \\
    405 & unfriendliness & 0 & 1484 \\
    406 & friendly\_jj & 0 & 1489 \\
    407 & diplomatic\_jj & 0 & 1499 \\
    408 & humor & 0 & 1507 \\
    409 & bitter\_jj & 0 & 1508 \\
    410 & warm\_jj & 0 & 1589 \\
    411 & cordial\_jj & 0 & 1600 \\
    412 & undependable\_jj & 0 & 1669 \\
    413 & inefficient\_jj & 0 & 1693 \\
    414 & insecurity & 0 & 1758 \\
    415 & volatile\_jj & 0 & 1759 \\
    416 & warmth & 0 & 1819 \\
    417 & unreliable\_jj & 0 & 1846 \\
    418 & dependable\_jj & 0 & 1976 \\
    419 & reliable\_jj & 0 & 2132 \\
    420 & distrust & 0 & 2147 \\
    421 & instability & 0 & 2150 \\
    \hline
    \caption{Scores and rankings for most extreme 30 words in component \#13} \\
\end{longtable}
\begin{longtable}[!htbp]{| rlr@{.}l |}
    \hline
    \textbf{Rank} & \textbf{Word} & \multicolumn{2}{c|}{\textbf{Score}} \\
    \hline
    \endhead
    1 & dominant\_jj & 0 & -1845 \\
    2 & defensive\_jj & 0 & -1754 \\
    3 & brave\_jj & 0 & -1692 \\
    4 & gregariousness & 0 & -1683 \\
    5 & deep\_jj & 0 & -1681 \\
    6 & stubborn\_jj & 0 & -1624 \\
    7 & steady\_jj & 0 & -1606 \\
    8 & tenacious\_jj & 0 & -1519 \\
    9 & decisive\_jj & 0 & -1469 \\
    10 & principled\_jj & 0 & -1436 \\
    11 & courageous\_jj & 0 & -1413 \\
    12 & foresighted\_jj & 0 & -1344 \\
    13 & prejudiced\_jj & 0 & -1310 \\
    14 & moral\_jj & 0 & -1283 \\
    15 & shallow\_jj & 0 & -1282 \\
    16 & nonconforming\_jj & 0 & -1275 \\
    17 & vigorous\_jj & 0 & -1268 \\
    18 & natural\_jj & 0 & -1219 \\
    19 & reliable\_jj & 0 & -1216 \\
    20 & vain\_jj & 0 & -1211 \\
    21 & submissive\_jj & 0 & -1207 \\
    22 & unintelligent\_jj & 0 & -1193 \\
    23 & depth & 0 & -1184 \\
    24 & courage & 0 & -1180 \\
    25 & active\_jj & 0 & -1141 \\
    26 & consistent\_jj & 0 & -1121 \\
    27 & demonstrative\_jj & 0 & -1117 \\
    28 & inhibition & 0 & -1094 \\
    29 & adaptable\_jj & 0 & -1091 \\
    30 & analytical\_jj & 0 & -1079 \\
    392 & generous\_jj & 0 & 1067 \\
    393 & spontaneous\_jj & 0 & 1076 \\
    394 & reckless\_jj & 0 & 1092 \\
    395 & inconsistency & 0 & 1100 \\
    396 & playfulness & 0 & 1123 \\
    397 & enterprising\_jj & 0 & 1136 \\
    398 & restrained\_jj & 0 & 1146 \\
    399 & unrestrained\_jj & 0 & 1154 \\
    400 & inventive\_jj & 0 & 1180 \\
    401 & optimistic\_jj & 0 & 1191 \\
    402 & negligence & 0 & 1197 \\
    403 & somber\_jj & 0 & 1199 \\
    404 & curiosity & 0 & 1203 \\
    405 & pessimistic\_jj & 0 & 1213 \\
    406 & adventurous\_jj & 0 & 1239 \\
    407 & imaginative\_jj & 0 & 1240 \\
    408 & carefree\_jj & 0 & 1292 \\
    409 & spontaneity & 0 & 1319 \\
    410 & pessimism & 0 & 1336 \\
    411 & unfriendly\_jj & 0 & 1348 \\
    412 & optimism & 0 & 1395 \\
    413 & unscrupulous\_jj & 0 & 1621 \\
    414 & frivolity & 0 & 1642 \\
    415 & leniency & 0 & 1718 \\
    416 & impractical\_jj & 0 & 1720 \\
    417 & casual\_jj & 0 & 1799 \\
    418 & lenient\_jj & 0 & 2002 \\
    419 & frivolous\_jj & 0 & 2147 \\
    420 & informal\_jj & 0 & 2385 \\
    421 & extravagant\_jj & 0 & 2968 \\
    \hline
    \caption{Scores and rankings for most extreme 30 words in component \#14} \\
\end{longtable}
\begin{longtable}[!htbp]{| rlr@{.}l |}
    \hline
    \textbf{Rank} & \textbf{Word} & \multicolumn{2}{c|}{\textbf{Score}} \\
    \hline
    \endhead
    1 & formal\_jj & 0 & -1635 \\
    2 & sloth & 0 & -1549 \\
    3 & ruthless\_jj & 0 & -1472 \\
    4 & systematic\_jj & 0 & -1465 \\
    5 & sophisticated\_jj & 0 & -1439 \\
    6 & distrust & 0 & -1402 \\
    7 & casual\_jj & 0 & -1334 \\
    8 & direct\_jj & 0 & -1314 \\
    9 & cooperative\_jj & 0 & -1300 \\
    10 & meticulous\_jj & 0 & -1298 \\
    11 & vigorous\_jj & 0 & -1287 \\
    12 & thorough\_jj & 0 & -1240 \\
    13 & passive\_jj & 0 & -1186 \\
    14 & flexible\_jj & 0 & -1153 \\
    15 & traditional\_jj & 0 & -1144 \\
    16 & cynical\_jj & 0 & -1120 \\
    17 & rambunctious\_jj & 0 & -1109 \\
    18 & crabby\_jj & 0 & -1102 \\
    19 & mischievous\_jj & 0 & -1082 \\
    20 & shy\_jj & 0 & -1080 \\
    21 & fearful\_jj & 0 & -1064 \\
    22 & surliness & 0 & -1054 \\
    23 & understanding\_jj & 0 & -1051 \\
    24 & gullibility & 0 & -1047 \\
    25 & stinginess & 0 & -1045 \\
    26 & cautious\_jj & 0 & -1041 \\
    27 & envy & 0 & -1030 \\
    28 & happy-go-lucky\_jj & 0 & -1030 \\
    29 & crafty\_jj & 0 & -1025 \\
    30 & fastidious\_jj & 0 & -998 \\
    392 & assertion & 0 & 1104 \\
    393 & unpredictable\_jj & 0 & 1152 \\
    394 & ungracious\_jj & 0 & 1173 \\
    395 & rash\_jj & 0 & 1180 \\
    396 & dignity & 0 & 1186 \\
    397 & earthy\_jj & 0 & 1199 \\
    398 & economical\_jj & 0 & 1212 \\
    399 & insight & 0 & 1245 \\
    400 & unfriendly\_jj & 0 & 1254 \\
    401 & harsh\_jj & 0 & 1263 \\
    402 & cruel\_jj & 0 & 1289 \\
    403 & naïve\_jj & 0 & 1308 \\
    404 & autonomous\_jj & 0 & 1328 \\
    405 & warm\_jj & 0 & 1332 \\
    406 & emotional\_jj & 0 & 1335 \\
    407 & impractical\_jj & 0 & 1348 \\
    408 & quiet & 0 & 1354 \\
    409 & explosive\_jj & 0 & 1393 \\
    410 & orderly\_jj & 0 & 1426 \\
    411 & foolhardy\_jj & 0 & 1428 \\
    412 & unstable\_jj & 0 & 1431 \\
    413 & insensitive\_jj & 0 & 1482 \\
    414 & unreliable\_jj & 0 & 1504 \\
    415 & independent\_jj & 0 & 1524 \\
    416 & undemanding\_jj & 0 & 1611 \\
    417 & silence & 0 & 1627 \\
    418 & independence & 0 & 1707 \\
    419 & insightful\_jj & 0 & 1737 \\
    420 & cold\_jj & 0 & 1838 \\
    421 & unkind\_jj & 0 & 2131 \\
    \hline
    \caption{Scores and rankings for most extreme 30 words in component \#15} \\
\end{longtable}
\begin{longtable}[!htbp]{| rlr@{.}l |}
    \hline
    \textbf{Rank} & \textbf{Word} & \multicolumn{2}{c|}{\textbf{Score}} \\
    \hline
    \endhead
    1 & daring\_jj & 0 & -1764 \\
    2 & courage & 0 & -1645 \\
    3 & impolite\_jj & 0 & -1614 \\
    4 & daring & 0 & -1568 \\
    5 & flexibility & 0 & -1545 \\
    6 & sympathetic\_jj & 0 & -1438 \\
    7 & intrusive\_jj & 0 & -1371 \\
    8 & harsh\_jj & 0 & -1361 \\
    9 & leniency & 0 & -1354 \\
    10 & brave\_jj & 0 & -1349 \\
    11 & diplomatic\_jj & 0 & -1347 \\
    12 & warmth & 0 & -1344 \\
    13 & gregariousness & 0 & -1335 \\
    14 & courageous\_jj & 0 & -1316 \\
    15 & decisiveness & 0 & -1252 \\
    16 & fearful\_jj & 0 & -1242 \\
    17 & efficient\_jj & 0 & -1238 \\
    18 & impractical\_jj & 0 & -1232 \\
    19 & cold\_jj & 0 & -1228 \\
    20 & sophisticated\_jj & 0 & -1205 \\
    21 & rude\_jj & 0 & -1187 \\
    22 & ruthless\_jj & 0 & -1183 \\
    23 & nervous\_jj & 0 & -1181 \\
    24 & generous\_jj & 0 & -1136 \\
    25 & jealous\_jj & 0 & -1133 \\
    26 & smart\_jj & 0 & -1116 \\
    27 & helpful\_jj & 0 & -1098 \\
    28 & obliging\_jj & 0 & -1080 \\
    29 & belligerence & 0 & -1041 \\
    30 & precision & 0 & -1022 \\
    392 & surly\_jj & 0 & 1074 \\
    393 & tempestuous\_jj & 0 & 1078 \\
    394 & unadventurous\_jj & 0 & 1083 \\
    395 & stubborn\_jj & 0 & 1108 \\
    396 & concise\_jj & 0 & 1113 \\
    397 & negligence & 0 & 1131 \\
    398 & optimistic\_jj & 0 & 1140 \\
    399 & systematic\_jj & 0 & 1152 \\
    400 & steady\_jj & 0 & 1160 \\
    401 & unrestrained\_jj & 0 & 1164 \\
    402 & spirited\_jj & 0 & 1168 \\
    403 & dominant\_jj & 0 & 1219 \\
    404 & principled\_jj & 0 & 1251 \\
    405 & assured\_jj & 0 & 1255 \\
    406 & industrious\_jj & 0 & 1279 \\
    407 & honest\_jj & 0 & 1284 \\
    408 & morality & 0 & 1292 \\
    409 & sloppy\_jj & 0 & 1303 \\
    410 & independent\_jj & 0 & 1315 \\
    411 & miserly\_jj & 0 & 1344 \\
    412 & impudent\_jj & 0 & 1383 \\
    413 & detached\_jj & 0 & 1400 \\
    414 & optimism & 0 & 1450 \\
    415 & firm\_jj & 0 & 1467 \\
    416 & pessimism & 0 & 1596 \\
    417 & sluggish\_jj & 0 & 1668 \\
    418 & inconsistency & 0 & 1676 \\
    419 & consistent\_jj & 0 & 1690 \\
    420 & philosophical\_jj & 0 & 1711 \\
    421 & inconsistent\_jj & 0 & 2032 \\
    \hline
    \caption{Scores and rankings for most extreme 30 words in component \#16} \\
\end{longtable}
\begin{longtable}[!htbp]{| rlr@{.}l |}
    \hline
    \textbf{Rank} & \textbf{Word} & \multicolumn{2}{c|}{\textbf{Score}} \\
    \hline
    \endhead
    1 & orderly\_jj & 0 & -1715 \\
    2 & haphazard\_jj & 0 & -1686 \\
    3 & thorough\_jj & 0 & -1532 \\
    4 & systematic\_jj & 0 & -1527 \\
    5 & bullheaded\_jj & 0 & -1503 \\
    6 & quiet & 0 & -1499 \\
    7 & intrusiveness & 0 & -1494 \\
    8 & secretive\_jj & 0 & -1464 \\
    9 & intelligence & 0 & -1429 \\
    10 & foresighted\_jj & 0 & -1356 \\
    11 & peaceful\_jj & 0 & -1356 \\
    12 & meticulous\_jj & 0 & -1354 \\
    13 & unsystematic\_jj & 0 & -1324 \\
    14 & dignified\_jj & 0 & -1311 \\
    15 & aimlessness & 0 & -1291 \\
    16 & inefficient\_jj & 0 & -1259 \\
    17 & fearful\_jj & 0 & -1218 \\
    18 & unscrupulous\_jj & 0 & -1158 \\
    19 & somber\_jj & 0 & -1126 \\
    20 & instability & 0 & -1095 \\
    21 & fear & 0 & -1088 \\
    22 & skeptical\_jj & 0 & -1072 \\
    23 & devious\_jj & 0 & -1067 \\
    24 & silent\_jj & 0 & -1067 \\
    25 & precise\_jj & 0 & -1063 \\
    26 & greedy\_jj & 0 & -1059 \\
    27 & prompt\_jj & 0 & -1058 \\
    28 & steady\_jj & 0 & -1042 \\
    29 & assured\_jj & 0 & -1040 \\
    30 & fretful\_jj & 0 & -1033 \\
    392 & philosophical\_jj & 0 & 1030 \\
    393 & snobbish\_jj & 0 & 1040 \\
    394 & nonconformity & 0 & 1047 \\
    395 & smart\_jj & 0 & 1049 \\
    396 & ungracious\_jj & 0 & 1064 \\
    397 & rebellious\_jj & 0 & 1071 \\
    398 & egotistical\_jj & 0 & 1079 \\
    399 & shyness & 0 & 1090 \\
    400 & submissive\_jj & 0 & 1095 \\
    401 & dominant\_jj & 0 & 1099 \\
    402 & argumentative\_jj & 0 & 1101 \\
    403 & emotional\_jj & 0 & 1103 \\
    404 & inhibition & 0 & 1109 \\
    405 & unconventional\_jj & 0 & 1117 \\
    406 & inconsistent\_jj & 0 & 1178 \\
    407 & active\_jj & 0 & 1201 \\
    408 & direct\_jj & 0 & 1231 \\
    409 & prejudice & 0 & 1233 \\
    410 & verbal\_jj & 0 & 1264 \\
    411 & flamboyant\_jj & 0 & 1332 \\
    412 & charitable\_jj & 0 & 1346 \\
    413 & courtesy & 0 & 1349 \\
    414 & disrespectful\_jj & 0 & 1388 \\
    415 & traditional\_jj & 0 & 1426 \\
    416 & self-esteem & 0 & 1437 \\
    417 & defensive\_jj & 0 & 1443 \\
    418 & erratic\_jj & 0 & 1445 \\
    419 & friendly\_jj & 0 & 1475 \\
    420 & touchy\_jj & 0 & 1742 \\
    421 & casual\_jj & 0 & 1880 \\
    \hline
    \caption{Scores and rankings for most extreme 30 words in component \#17} \\
\end{longtable}
\begin{longtable}[!htbp]{| rlr@{.}l |}
    \hline
    \textbf{Rank} & \textbf{Word} & \multicolumn{2}{c|}{\textbf{Score}} \\
    \hline
    \endhead
    1 & conventional\_jj & 0 & -1821 \\
    2 & assertion & 0 & -1562 \\
    3 & happy-go-lucky\_jj & 0 & -1558 \\
    4 & reliable\_jj & 0 & -1497 \\
    5 & excitable\_jj & 0 & -1333 \\
    6 & consistent\_jj & 0 & -1326 \\
    7 & passive\_jj & 0 & -1306 \\
    8 & volatile\_jj & 0 & -1289 \\
    9 & rambunctious\_jj & 0 & -1272 \\
    10 & confident\_jj & 0 & -1264 \\
    11 & optimistic\_jj & 0 & -1241 \\
    12 & responsible\_jj & 0 & -1220 \\
    13 & reckless\_jj & 0 & -1207 \\
    14 & natural\_jj & 0 & -1202 \\
    15 & rebellious\_jj & 0 & -1182 \\
    16 & efficient\_jj & 0 & -1135 \\
    17 & nonconforming\_jj & 0 & -1135 \\
    18 & efficiency & 0 & -1107 \\
    19 & unrestrained\_jj & 0 & -1105 \\
    20 & steady\_jj & 0 & -1085 \\
    21 & logic & 0 & -1059 \\
    22 & naturalness & 0 & -1047 \\
    23 & dependability & 0 & -1018 \\
    24 & gregariousness & 0 & -1004 \\
    25 & explosive\_jj & 0 & -1003 \\
    26 & uninhibited\_jj & 0 & -998 \\
    27 & curt\_jj & 0 & -973 \\
    28 & direct\_jj & 0 & -930 \\
    29 & optimism & 0 & -928 \\
    30 & unstable\_jj & 0 & -927 \\
    392 & sloth & 0 & 980 \\
    393 & humble\_jj & 0 & 986 \\
    394 & impolite\_jj & 0 & 997 \\
    395 & wordy\_jj & 0 & 1009 \\
    396 & shallow\_jj & 0 & 1011 \\
    397 & tenacious\_jj & 0 & 1024 \\
    398 & meticulous\_jj & 0 & 1058 \\
    399 & ethical\_jj & 0 & 1117 \\
    400 & enterprising\_jj & 0 & 1136 \\
    401 & persistence & 0 & 1147 \\
    402 & crafty\_jj & 0 & 1202 \\
    403 & warm\_jj & 0 & 1214 \\
    404 & sloppy\_jj & 0 & 1229 \\
    405 & disorganization & 0 & 1261 \\
    406 & absent-minded\_jj & 0 & 1278 \\
    407 & secretive\_jj & 0 & 1294 \\
    408 & unscrupulous\_jj & 0 & 1337 \\
    409 & creative\_jj & 0 & 1362 \\
    410 & philosophical\_jj & 0 & 1422 \\
    411 & lethargy & 0 & 1424 \\
    412 & thrifty\_jj & 0 & 1428 \\
    413 & charitable\_jj & 0 & 1434 \\
    414 & thrift & 0 & 1466 \\
    415 & cold\_jj & 0 & 1498 \\
    416 & insight & 0 & 1581 \\
    417 & deep\_jj & 0 & 1685 \\
    418 & diplomatic\_jj & 0 & 1696 \\
    419 & stingy\_jj & 0 & 1714 \\
    420 & bitter\_jj & 0 & 1957 \\
    421 & touchy\_jj & 0 & 2085 \\
    \hline
    \caption{Scores and rankings for most extreme 30 words in component \#18} \\
\end{longtable}
\begin{longtable}[!htbp]{| rlr@{.}l |}
    \hline
    \textbf{Rank} & \textbf{Word} & \multicolumn{2}{c|}{\textbf{Score}} \\
    \hline
    \endhead
    1 & decisive\_jj & 0 & -1846 \\
    2 & diplomatic\_jj & 0 & -1826 \\
    3 & spirited\_jj & 0 & -1814 \\
    4 & volatility & 0 & -1806 \\
    5 & foolhardy\_jj & 0 & -1767 \\
    6 & vigorous\_jj & 0 & -1659 \\
    7 & frivolous\_jj & 0 & -1597 \\
    8 & instability & 0 & -1475 \\
    9 & humorous\_jj & 0 & -1444 \\
    10 & spontaneous\_jj & 0 & -1411 \\
    11 & organized\_jj & 0 & -1365 \\
    12 & rash\_jj & 0 & -1362 \\
    13 & mannerly\_jj & 0 & -1281 \\
    14 & conceited\_jj & 0 & -1269 \\
    15 & prompt\_jj & 0 & -1234 \\
    16 & bossiness & 0 & -1190 \\
    17 & explosive\_jj & 0 & -1156 \\
    18 & smart\_jj & 0 & -1155 \\
    19 & impolite\_jj & 0 & -1153 \\
    20 & daring\_jj & 0 & -1143 \\
    21 & talkativeness & 0 & -1074 \\
    22 & independence & 0 & -1060 \\
    23 & meddlesome\_jj & 0 & -1023 \\
    24 & enterprising\_jj & 0 & -1018 \\
    25 & modest\_jj & 0 & -1011 \\
    26 & energetic\_jj & 0 & -971 \\
    27 & peaceful\_jj & 0 & -927 \\
    28 & economical\_jj & 0 & -922 \\
    29 & verbose\_jj & 0 & -915 \\
    30 & fear & 0 & -914 \\
    392 & callousness & 0 & 937 \\
    393 & negligent\_jj & 0 & 938 \\
    394 & careful\_jj & 0 & 942 \\
    395 & complex\_jj & 0 & 973 \\
    396 & modesty & 0 & 977 \\
    397 & melancholic\_jj & 0 & 992 \\
    398 & morose\_jj & 0 & 992 \\
    399 & unreliable\_jj & 0 & 993 \\
    400 & detached\_jj & 0 & 999 \\
    401 & gruff\_jj & 0 & 1010 \\
    402 & precision & 0 & 1038 \\
    403 & cooperative\_jj & 0 & 1042 \\
    404 & harsh\_jj & 0 & 1056 \\
    405 & proud\_jj & 0 & 1058 \\
    406 & earthy\_jj & 0 & 1066 \\
    407 & jealous\_jj & 0 & 1103 \\
    408 & understanding & 0 & 1149 \\
    409 & sloth & 0 & 1149 \\
    410 & meticulous\_jj & 0 & 1150 \\
    411 & warmth & 0 & 1202 \\
    412 & cruel\_jj & 0 & 1213 \\
    413 & dignity & 0 & 1224 \\
    414 & exacting\_jj & 0 & 1236 \\
    415 & precise\_jj & 0 & 1242 \\
    416 & warm\_jj & 0 & 1382 \\
    417 & lenient\_jj & 0 & 1515 \\
    418 & inconsistent\_jj & 0 & 1520 \\
    419 & expressiveness & 0 & 1675 \\
    420 & cold\_jj & 0 & 1838 \\
    421 & cruelty & 0 & 1967 \\
    \hline
    \caption{Scores and rankings for most extreme 30 words in component \#19} \\
\end{longtable}
\begin{longtable}[!htbp]{| rlr@{.}l |}
    \hline
    \textbf{Rank} & \textbf{Word} & \multicolumn{2}{c|}{\textbf{Score}} \\
    \hline
    \endhead
    1 & harsh\_jj & 0 & -1903 \\
    2 & cold\_jj & 0 & -1859 \\
    3 & caution & 0 & -1813 \\
    4 & enterprising\_jj & 0 & -1607 \\
    5 & cruelty & 0 & -1466 \\
    6 & warm\_jj & 0 & -1389 \\
    7 & pessimism & 0 & -1352 \\
    8 & caustic\_jj & 0 & -1308 \\
    9 & adaptable\_jj & 0 & -1296 \\
    10 & cruel\_jj & 0 & -1240 \\
    11 & unscrupulous\_jj & 0 & -1231 \\
    12 & sophisticated\_jj & 0 & -1137 \\
    13 & curt\_jj & 0 & -1130 \\
    14 & miserly\_jj & 0 & -1124 \\
    15 & bitter\_jj & 0 & -1122 \\
    16 & friendly\_jj & 0 & -1097 \\
    17 & perceptive\_jj & 0 & -1088 \\
    18 & trustful\_jj & 0 & -1086 \\
    19 & leniency & 0 & -1056 \\
    20 & conventional\_jj & 0 & -1053 \\
    21 & obliging\_jj & 0 & -1053 \\
    22 & insight & 0 & -1043 \\
    23 & suggestible\_jj & 0 & -1026 \\
    24 & firm\_jj & 0 & -1020 \\
    25 & lenient\_jj & 0 & -997 \\
    26 & devious\_jj & 0 & -977 \\
    27 & absent-minded\_jj & 0 & -969 \\
    28 & obstinate\_jj & 0 & -948 \\
    29 & nosey\_jj & 0 & -941 \\
    30 & logic & 0 & -937 \\
    392 & dull\_jj & 0 & 975 \\
    393 & conventionality & 0 & 984 \\
    394 & aimless\_jj & 0 & 1000 \\
    395 & flamboyant\_jj & 0 & 1013 \\
    396 & informal\_jj & 0 & 1015 \\
    397 & sentimental\_jj & 0 & 1069 \\
    398 & meticulous\_jj & 0 & 1076 \\
    399 & traditional\_jj & 0 & 1080 \\
    400 & unassuming\_jj & 0 & 1102 \\
    401 & underhanded\_jj & 0 & 1119 \\
    402 & quiet\_jj & 0 & 1120 \\
    403 & meditative\_jj & 0 & 1182 \\
    404 & organization & 0 & 1216 \\
    405 & extravagant\_jj & 0 & 1222 \\
    406 & insecure\_jj & 0 & 1282 \\
    407 & orderly\_jj & 0 & 1301 \\
    408 & worldly\_jj & 0 & 1303 \\
    409 & self-indulgent\_jj & 0 & 1314 \\
    410 & indecisive\_jj & 0 & 1334 \\
    411 & artistic\_jj & 0 & 1340 \\
    412 & shy\_jj & 0 & 1368 \\
    413 & pompous\_jj & 0 & 1410 \\
    414 & formal\_jj & 0 & 1458 \\
    415 & proud\_jj & 0 & 1475 \\
    416 & secretive\_jj & 0 & 1504 \\
    417 & thrift & 0 & 1667 \\
    418 & disorganization & 0 & 1671 \\
    419 & self-esteem & 0 & 1678 \\
    420 & haphazard\_jj & 0 & 1710 \\
    421 & charitable\_jj & 0 & 1976 \\
    \hline
    \caption{Scores and rankings for most extreme 30 words in component \#20} \\
\end{longtable}
\begin{longtable}[!htbp]{| rlr@{.}l |}
    \hline
    \textbf{Rank} & \textbf{Word} & \multicolumn{2}{c|}{\textbf{Score}} \\
    \hline
    \endhead
    1 & unstable\_jj & 0 & -1765 \\
    2 & cranky\_jj & 0 & -1586 \\
    3 & diplomatic\_jj & 0 & -1513 \\
    4 & nonconforming\_jj & 0 & -1445 \\
    5 & trustful\_jj & 0 & -1438 \\
    6 & moral\_jj & 0 & -1305 \\
    7 & morose\_jj & 0 & -1304 \\
    8 & explosive\_jj & 0 & -1262 \\
    9 & instability & 0 & -1240 \\
    10 & careless\_jj & 0 & -1220 \\
    11 & self-critical\_jj & 0 & -1220 \\
    12 & exacting\_jj & 0 & -1205 \\
    13 & rash\_jj & 0 & -1187 \\
    14 & unambitious\_jj & 0 & -1169 \\
    15 & intellectuality & 0 & -1162 \\
    16 & foolhardy\_jj & 0 & -1160 \\
    17 & unsystematic\_jj & 0 & -1110 \\
    18 & foresighted\_jj & 0 & -1109 \\
    19 & ambition & 0 & -1107 \\
    20 & rebellious\_jj & 0 & -1105 \\
    21 & cooperation & 0 & -1051 \\
    22 & self-esteem & 0 & -1044 \\
    23 & grumpy\_jj & 0 & -1034 \\
    24 & unobservant\_jj & 0 & -1032 \\
    25 & insecure\_jj & 0 & -1003 \\
    26 & complex\_jj & 0 & -980 \\
    27 & predictability & 0 & -979 \\
    28 & philosophical\_jj & 0 & -977 \\
    29 & independence & 0 & -953 \\
    30 & pessimism & 0 & -942 \\
    392 & insensitive\_jj & 0 & 904 \\
    393 & assured\_jj & 0 & 908 \\
    394 & predictable\_jj & 0 & 913 \\
    395 & innovative\_jj & 0 & 918 \\
    396 & spontaneous\_jj & 0 & 924 \\
    397 & unsympathetic\_jj & 0 & 938 \\
    398 & forgetfulness & 0 & 981 \\
    399 & proud\_jj & 0 & 1000 \\
    400 & stingy\_jj & 0 & 1002 \\
    401 & excitable\_jj & 0 & 1007 \\
    402 & direct\_jj & 0 & 1011 \\
    403 & spirited\_jj & 0 & 1012 \\
    404 & rudeness & 0 & 1076 \\
    405 & miserly\_jj & 0 & 1076 \\
    406 & generosity & 0 & 1097 \\
    407 & verbal\_jj & 0 & 1118 \\
    408 & thoughtlessness & 0 & 1174 \\
    409 & passive\_jj & 0 & 1185 \\
    410 & ignorant\_jj & 0 & 1212 \\
    411 & enthusiastic\_jj & 0 & 1230 \\
    412 & envy & 0 & 1274 \\
    413 & expressive\_jj & 0 & 1383 \\
    414 & active\_jj & 0 & 1421 \\
    415 & dominant\_jj & 0 & 1516 \\
    416 & uncritical\_jj & 0 & 1530 \\
    417 & organized\_jj & 0 & 1552 \\
    418 & silent\_jj & 0 & 1659 \\
    419 & scornful\_jj & 0 & 1659 \\
    420 & silence & 0 & 1671 \\
    421 & courtesy & 0 & 1950 \\
    \hline
    \caption{Scores and rankings for most extreme 30 words in component \#21} \\
\end{longtable}
\begin{longtable}[!htbp]{| rlr@{.}l |}
    \hline
    \textbf{Rank} & \textbf{Word} & \multicolumn{2}{c|}{\textbf{Score}} \\
    \hline
    \endhead
    1 & aimless\_jj & 0 & -1771 \\
    2 & cooperative\_jj & 0 & -1605 \\
    3 & lethargy & 0 & -1511 \\
    4 & trustful\_jj & 0 & -1408 \\
    5 & helpful\_jj & 0 & -1404 \\
    6 & logical\_jj & 0 & -1334 \\
    7 & deep\_jj & 0 & -1234 \\
    8 & impractical\_jj & 0 & -1230 \\
    9 & quarrelsome\_jj & 0 & -1198 \\
    10 & conventionality & 0 & -1197 \\
    11 & irritability & 0 & -1151 \\
    12 & purposeful\_jj & 0 & -1129 \\
    13 & sympathetic\_jj & 0 & -1127 \\
    14 & forgetfulness & 0 & -1119 \\
    15 & inventive\_jj & 0 & -1089 \\
    16 & tenacious\_jj & 0 & -1089 \\
    17 & rash\_jj & 0 & -1062 \\
    18 & logic & 0 & -1012 \\
    19 & spirit & 0 & -1008 \\
    20 & irritable\_jj & 0 & -976 \\
    21 & responsible\_jj & 0 & -970 \\
    22 & haphazard\_jj & 0 & -947 \\
    23 & cooperation & 0 & -944 \\
    24 & innovative\_jj & 0 & -941 \\
    25 & organization & 0 & -936 \\
    26 & simple\_jj & 0 & -929 \\
    27 & distrust & 0 & -917 \\
    28 & unsystematic\_jj & 0 & -911 \\
    29 & gregariousness & 0 & -909 \\
    30 & peaceful\_jj & 0 & -904 \\
    392 & snobbish\_jj & 0 & 919 \\
    393 & morality & 0 & 920 \\
    394 & silent\_jj & 0 & 932 \\
    395 & suspicious\_jj & 0 & 942 \\
    396 & intellectual\_jj & 0 & 954 \\
    397 & perceptive\_jj & 0 & 991 \\
    398 & punctuality & 0 & 992 \\
    399 & touchy\_jj & 0 & 1012 \\
    400 & modest\_jj & 0 & 1040 \\
    401 & worldly\_jj & 0 & 1047 \\
    402 & mannerly\_jj & 0 & 1048 \\
    403 & silence & 0 & 1050 \\
    404 & thrifty\_jj & 0 & 1060 \\
    405 & refined\_jj & 0 & 1088 \\
    406 & shy\_jj & 0 & 1097 \\
    407 & predictability & 0 & 1120 \\
    408 & steady\_jj & 0 & 1148 \\
    409 & cruelty & 0 & 1152 \\
    410 & moral\_jj & 0 & 1196 \\
    411 & intrusiveness & 0 & 1204 \\
    412 & secretive\_jj & 0 & 1206 \\
    413 & quiet\_jj & 0 & 1248 \\
    414 & ethical\_jj & 0 & 1331 \\
    415 & meticulous\_jj & 0 & 1368 \\
    416 & volatile\_jj & 0 & 1395 \\
    417 & excitable\_jj & 0 & 1433 \\
    418 & demanding\_jj & 0 & 1538 \\
    419 & exacting\_jj & 0 & 1611 \\
    420 & dependability & 0 & 1728 \\
    421 & volatility & 0 & 1919 \\
    \hline
    \caption{Scores and rankings for most extreme 30 words in component \#22} \\
\end{longtable}
\begin{longtable}[!htbp]{| rlr@{.}l |}
    \hline
    \textbf{Rank} & \textbf{Word} & \multicolumn{2}{c|}{\textbf{Score}} \\
    \hline
    \endhead
    1 & quiet & 0 & -1924 \\
    2 & tempestuous\_jj & 0 & -1641 \\
    3 & silence & 0 & -1532 \\
    4 & vain\_jj & 0 & -1523 \\
    5 & crafty\_jj & 0 & -1410 \\
    6 & logic & 0 & -1374 \\
    7 & joyless\_jj & 0 & -1334 \\
    8 & egocentric\_jj & 0 & -1281 \\
    9 & rambunctious\_jj & 0 & -1253 \\
    10 & erratic\_jj & 0 & -1253 \\
    11 & curiosity & 0 & -1215 \\
    12 & unpredictable\_jj & 0 & -1205 \\
    13 & cordial\_jj & 0 & -1081 \\
    14 & adaptable\_jj & 0 & -1071 \\
    15 & trustful\_jj & 0 & -1071 \\
    16 & understanding & 0 & -1066 \\
    17 & belligerence & 0 & -1066 \\
    18 & nosey\_jj & 0 & -1061 \\
    19 & touchy\_jj & 0 & -1061 \\
    20 & informal\_jj & 0 & -1039 \\
    21 & antagonistic\_jj & 0 & -1024 \\
    22 & understanding\_jj & 0 & -972 \\
    23 & inconsiderate\_jj & 0 & -959 \\
    24 & agreeable\_jj & 0 & -943 \\
    25 & temperamental\_jj & 0 & -918 \\
    26 & envy & 0 & -900 \\
    27 & unfriendly\_jj & 0 & -886 \\
    28 & industrious\_jj & 0 & -840 \\
    29 & unadventurous\_jj & 0 & -836 \\
    30 & envious\_jj & 0 & -814 \\
    392 & unintelligent\_jj & 0 & 824 \\
    393 & ungracious\_jj & 0 & 845 \\
    394 & sophistication & 0 & 866 \\
    395 & reasonable\_jj & 0 & 872 \\
    396 & reserved\_jj & 0 & 876 \\
    397 & condescending\_jj & 0 & 913 \\
    398 & withdrawn\_jj & 0 & 920 \\
    399 & expressiveness & 0 & 925 \\
    400 & principled\_jj & 0 & 986 \\
    401 & sluggish\_jj & 0 & 992 \\
    402 & organization & 0 & 1003 \\
    403 & vivacious\_jj & 0 & 1080 \\
    404 & crabby\_jj & 0 & 1081 \\
    405 & restrained\_jj & 0 & 1087 \\
    406 & irritable\_jj & 0 & 1099 \\
    407 & cultured\_jj & 0 & 1155 \\
    408 & independent\_jj & 0 & 1168 \\
    409 & callousness & 0 & 1182 \\
    410 & compassionate\_jj & 0 & 1182 \\
    411 & lenient\_jj & 0 & 1189 \\
    412 & talkativeness & 0 & 1246 \\
    413 & generous\_jj & 0 & 1252 \\
    414 & charitable\_jj & 0 & 1258 \\
    415 & dishonest\_jj & 0 & 1300 \\
    416 & refined\_jj & 0 & 1502 \\
    417 & shallow\_jj & 0 & 1599 \\
    418 & depth & 0 & 1612 \\
    419 & earthiness & 0 & 1795 \\
    420 & earthy\_jj & 0 & 1859 \\
    421 & deep\_jj & 0 & 1961 \\
    \hline
    \caption{Scores and rankings for most extreme 30 words in component \#23} \\
\end{longtable}
\begin{longtable}[!htbp]{| rlr@{.}l |}
    \hline
    \textbf{Rank} & \textbf{Word} & \multicolumn{2}{c|}{\textbf{Score}} \\
    \hline
    \endhead
    1 & fastidious\_jj & 0 & -1504 \\
    2 & bossiness & 0 & -1488 \\
    3 & tenacious\_jj & 0 & -1435 \\
    4 & fear & 0 & -1317 \\
    5 & traditional\_jj & 0 & -1305 \\
    6 & conventional\_jj & 0 & -1278 \\
    7 & shyness & 0 & -1198 \\
    8 & unassuming\_jj & 0 & -1177 \\
    9 & patient\_jj & 0 & -1165 \\
    10 & simple\_jj & 0 & -1156 \\
    11 & modesty & 0 & -1146 \\
    12 & tactful\_jj & 0 & -1099 \\
    13 & stubborn\_jj & 0 & -1095 \\
    14 & assertion & 0 & -1078 \\
    15 & rebellious\_jj & 0 & -1038 \\
    16 & unreliable\_jj & 0 & -1032 \\
    17 & meticulous\_jj & 0 & -1009 \\
    18 & organized\_jj & 0 & -1006 \\
    19 & obstinate\_jj & 0 & -1004 \\
    20 & persistent\_jj & 0 & -995 \\
    21 & unrestrained\_jj & 0 & -972 \\
    22 & morality & 0 & -947 \\
    23 & frivolous\_jj & 0 & -929 \\
    24 & orderly\_jj & 0 & -915 \\
    25 & gruff\_jj & 0 & -905 \\
    26 & disorganization & 0 & -901 \\
    27 & enterprising\_jj & 0 & -893 \\
    28 & bitter\_jj & 0 & -884 \\
    29 & earthiness & 0 & -875 \\
    30 & precision & 0 & -872 \\
    392 & diplomatic\_jj & 0 & 918 \\
    393 & belligerence & 0 & 936 \\
    394 & pessimistic\_jj & 0 & 966 \\
    395 & refined\_jj & 0 & 981 \\
    396 & selfish\_jj & 0 & 1007 \\
    397 & devious\_jj & 0 & 1016 \\
    398 & quarrelsome\_jj & 0 & 1019 \\
    399 & intellectuality & 0 & 1021 \\
    400 & animation & 0 & 1022 \\
    401 & communicative\_jj & 0 & 1028 \\
    402 & contemplative\_jj & 0 & 1053 \\
    403 & gullibility & 0 & 1089 \\
    404 & volatility & 0 & 1099 \\
    405 & depth & 0 & 1106 \\
    406 & thrift & 0 & 1110 \\
    407 & benevolent\_jj & 0 & 1117 \\
    408 & sluggish\_jj & 0 & 1158 \\
    409 & insight & 0 & 1163 \\
    410 & sloth & 0 & 1227 \\
    411 & placidity & 0 & 1230 \\
    412 & cooperation & 0 & 1316 \\
    413 & flexibility & 0 & 1426 \\
    414 & understanding\_jj & 0 & 1465 \\
    415 & cordial\_jj & 0 & 1474 \\
    416 & playful\_jj & 0 & 1505 \\
    417 & natural\_jj & 0 & 1530 \\
    418 & somber\_jj & 0 & 1556 \\
    419 & shallow\_jj & 0 & 1641 \\
    420 & cooperative\_jj & 0 & 1751 \\
    421 & reserve & 0 & 1766 \\
    \hline
    \caption{Scores and rankings for most extreme 30 words in component \#24} \\
\end{longtable}
\begin{longtable}[!htbp]{| rlr@{.}l |}
    \hline
    \textbf{Rank} & \textbf{Word} & \multicolumn{2}{c|}{\textbf{Score}} \\
    \hline
    \endhead
    1 & unconventional\_jj & 0 & -1512 \\
    2 & wishy-washy\_jj & 0 & -1306 \\
    3 & natural\_jj & 0 & -1302 \\
    4 & independence & 0 & -1282 \\
    5 & thrift & 0 & -1274 \\
    6 & insight & 0 & -1260 \\
    7 & undependable\_jj & 0 & -1201 \\
    8 & argumentative\_jj & 0 & -1184 \\
    9 & negligent\_jj & 0 & -1127 \\
    10 & thrifty\_jj & 0 & -1097 \\
    11 & humorous\_jj & 0 & -1086 \\
    12 & volatility & 0 & -1083 \\
    13 & impetuous\_jj & 0 & -1078 \\
    14 & disorganization & 0 & -1067 \\
    15 & cautious\_jj & 0 & -1056 \\
    16 & foolhardy\_jj & 0 & -1042 \\
    17 & distrustful\_jj & 0 & -1014 \\
    18 & conventional\_jj & 0 & -1007 \\
    19 & passivity & 0 & -1006 \\
    20 & unreliable\_jj & 0 & -1004 \\
    21 & frivolity & 0 & -999 \\
    22 & indecisiveness & 0 & -996 \\
    23 & perceptive\_jj & 0 & -995 \\
    24 & analytical\_jj & 0 & -966 \\
    25 & leniency & 0 & -932 \\
    26 & responsible\_jj & 0 & -931 \\
    27 & intelligence & 0 & -923 \\
    28 & impudent\_jj & 0 & -874 \\
    29 & deliberate\_jj & 0 & -858 \\
    30 & uncritical\_jj & 0 & -855 \\
    392 & prejudice & 0 & 899 \\
    393 & orderly\_jj & 0 & 902 \\
    394 & egocentric\_jj & 0 & 904 \\
    395 & dignified\_jj & 0 & 904 \\
    396 & cordial\_jj & 0 & 913 \\
    397 & humble\_jj & 0 & 940 \\
    398 & assertion & 0 & 952 \\
    399 & unscrupulous\_jj & 0 & 971 \\
    400 & formal\_jj & 0 & 1004 \\
    401 & frank\_jj & 0 & 1023 \\
    402 & thorough\_jj & 0 & 1069 \\
    403 & generous\_jj & 0 & 1082 \\
    404 & conceited\_jj & 0 & 1083 \\
    405 & lethargic\_jj & 0 & 1100 \\
    406 & confident\_jj & 0 & 1121 \\
    407 & ambition & 0 & 1129 \\
    408 & jealous\_jj & 0 & 1146 \\
    409 & informal\_jj & 0 & 1167 \\
    410 & exacting\_jj & 0 & 1174 \\
    411 & insecure\_jj & 0 & 1175 \\
    412 & emotional\_jj & 0 & 1258 \\
    413 & proud\_jj & 0 & 1287 \\
    414 & deep\_jj & 0 & 1315 \\
    415 & assured\_jj & 0 & 1387 \\
    416 & extravagant\_jj & 0 & 1450 \\
    417 & envy & 0 & 1477 \\
    418 & unaggressive\_jj & 0 & 1485 \\
    419 & modest\_jj & 0 & 1509 \\
    420 & verbal\_jj & 0 & 1718 \\
    421 & ambitious\_jj & 0 & 1758 \\
    \hline
    \caption{Scores and rankings for most extreme 30 words in component \#25} \\
\end{longtable}
\begin{longtable}[!htbp]{| rlr@{.}l |}
    \hline
    \textbf{Rank} & \textbf{Word} & \multicolumn{2}{c|}{\textbf{Score}} \\
    \hline
    \endhead
    1 & punctuality & 0 & -1754 \\
    2 & lenient\_jj & 0 & -1716 \\
    3 & cooperation & 0 & -1465 \\
    4 & forgetful\_jj & 0 & -1396 \\
    5 & leniency & 0 & -1395 \\
    6 & negligence & 0 & -1391 \\
    7 & grumpy\_jj & 0 & -1248 \\
    8 & detached\_jj & 0 & -1222 \\
    9 & punctual\_jj & 0 & -1181 \\
    10 & responsible\_jj & 0 & -1162 \\
    11 & negligent\_jj & 0 & -1128 \\
    12 & animation & 0 & -1104 \\
    13 & demanding\_jj & 0 & -1103 \\
    14 & lethargy & 0 & -1040 \\
    15 & rudeness & 0 & -1032 \\
    16 & formal\_jj & 0 & -1031 \\
    17 & organization & 0 & -1017 \\
    18 & vigorous\_jj & 0 & -1013 \\
    19 & compassionate\_jj & 0 & -1007 \\
    20 & crabby\_jj & 0 & -986 \\
    21 & cruelty & 0 & -980 \\
    22 & reliable\_jj & 0 & -969 \\
    23 & ambitious\_jj & 0 & -956 \\
    24 & surly\_jj & 0 & -954 \\
    25 & cranky\_jj & 0 & -941 \\
    26 & quarrelsome\_jj & 0 & -910 \\
    27 & perceptive\_jj & 0 & -893 \\
    28 & envy & 0 & -889 \\
    29 & lethargic\_jj & 0 & -888 \\
    30 & silence & 0 & -885 \\
    392 & naturalness & 0 & 887 \\
    393 & unfriendly\_jj & 0 & 891 \\
    394 & quiet\_jj & 0 & 898 \\
    395 & stinginess & 0 & 900 \\
    396 & tempestuous\_jj & 0 & 911 \\
    397 & persistence & 0 & 934 \\
    398 & helpful\_jj & 0 & 958 \\
    399 & volatile\_jj & 0 & 968 \\
    400 & selfless\_jj & 0 & 972 \\
    401 & generous\_jj & 0 & 976 \\
    402 & informal\_jj & 0 & 980 \\
    403 & manipulative\_jj & 0 & 981 \\
    404 & cautious\_jj & 0 & 1012 \\
    405 & boastful\_jj & 0 & 1056 \\
    406 & generosity & 0 & 1076 \\
    407 & unobservant\_jj & 0 & 1082 \\
    408 & reserve & 0 & 1083 \\
    409 & kind\_jj & 0 & 1122 \\
    410 & active\_jj & 0 & 1130 \\
    411 & careful\_jj & 0 & 1137 \\
    412 & abusive\_jj & 0 & 1145 \\
    413 & orderly\_jj & 0 & 1165 \\
    414 & unsystematic\_jj & 0 & 1251 \\
    415 & defensive\_jj & 0 & 1255 \\
    416 & stingy\_jj & 0 & 1261 \\
    417 & thrifty\_jj & 0 & 1279 \\
    418 & exhibitionistic\_jj & 0 & 1488 \\
    419 & unconventional\_jj & 0 & 1655 \\
    420 & volatility & 0 & 1790 \\
    421 & opportunistic\_jj & 0 & 1950 \\
    \hline
    \caption{Scores and rankings for most extreme 30 words in component \#26} \\
\end{longtable}
\begin{longtable}[!htbp]{| rlr@{.}l |}
    \hline
    \textbf{Rank} & \textbf{Word} & \multicolumn{2}{c|}{\textbf{Score}} \\
    \hline
    \endhead
    1 & thrift & 0 & -1994 \\
    2 & reserve & 0 & -1938 \\
    3 & enterprising\_jj & 0 & -1570 \\
    4 & informal\_jj & 0 & -1537 \\
    5 & bossy\_jj & 0 & -1288 \\
    6 & lethargic\_jj & 0 & -1264 \\
    7 & withdrawn\_jj & 0 & -1255 \\
    8 & thorough\_jj & 0 & -1243 \\
    9 & flexibility & 0 & -1185 \\
    10 & impersonal\_jj & 0 & -1165 \\
    11 & scornful\_jj & 0 & -1100 \\
    12 & insensitive\_jj & 0 & -1090 \\
    13 & independent\_jj & 0 & -1090 \\
    14 & analytical\_jj & 0 & -1080 \\
    15 & timid\_jj & 0 & -1061 \\
    16 & happy-go-lucky\_jj & 0 & -1054 \\
    17 & orderly\_jj & 0 & -1042 \\
    18 & crabby\_jj & 0 & -1037 \\
    19 & intelligence & 0 & -1016 \\
    20 & caution & 0 & -1016 \\
    21 & autonomous\_jj & 0 & -1006 \\
    22 & thoughtless\_jj & 0 & -1006 \\
    23 & unrestrained\_jj & 0 & -1005 \\
    24 & systematic\_jj & 0 & -997 \\
    25 & insecure\_jj & 0 & -991 \\
    26 & unassuming\_jj & 0 & -974 \\
    27 & unconventional\_jj & 0 & -968 \\
    28 & unemotional\_jj & 0 & -962 \\
    29 & intrusive\_jj & 0 & -915 \\
    30 & insight & 0 & -899 \\
    392 & erratic\_jj & 0 & 835 \\
    393 & reckless\_jj & 0 & 841 \\
    394 & volatility & 0 & 854 \\
    395 & temperamental\_jj & 0 & 871 \\
    396 & direct\_jj & 0 & 871 \\
    397 & unpredictable\_jj & 0 & 888 \\
    398 & unscrupulous\_jj & 0 & 903 \\
    399 & touchy\_jj & 0 & 910 \\
    400 & absent-minded\_jj & 0 & 931 \\
    401 & precise\_jj & 0 & 932 \\
    402 & decisive\_jj & 0 & 952 \\
    403 & organization & 0 & 962 \\
    404 & expressive\_jj & 0 & 972 \\
    405 & tempestuous\_jj & 0 & 974 \\
    406 & instability & 0 & 981 \\
    407 & punctual\_jj & 0 & 1001 \\
    408 & self-indulgent\_jj & 0 & 1017 \\
    409 & frivolous\_jj & 0 & 1019 \\
    410 & helpful\_jj & 0 & 1044 \\
    411 & volatile\_jj & 0 & 1068 \\
    412 & earthiness & 0 & 1080 \\
    413 & truthful\_jj & 0 & 1094 \\
    414 & active\_jj & 0 & 1127 \\
    415 & generous\_jj & 0 & 1153 \\
    416 & miserly\_jj & 0 & 1218 \\
    417 & conventionality & 0 & 1251 \\
    418 & negligence & 0 & 1376 \\
    419 & responsible\_jj & 0 & 1521 \\
    420 & stinginess & 0 & 2274 \\
    421 & charitable\_jj & 0 & 2315 \\
    \hline
    \caption{Scores and rankings for most extreme 30 words in component \#27} \\
\end{longtable}


\section{2797 word list}
\subsection{Unnormalized PCA}
\label{app:rankedwordlists:2797words:unnormalized}
\begin{table}[tbp]
    \begin{tabular}{| rlr@{.}l | rlr@{.}l |}
    \hline
    \textbf{Rank} & \textbf{Word} & \multicolumn{2}{c|}{\textbf{Score}} & \textbf{Rank} & \textbf{Word} & \multicolumn{2}{c|}{\textbf{Score}} \\
    \hline
    1 & stringent\_jj & -1 & 6632    &    1860 & mousy\_jj & 1 & 6471 \\
    2 & beneficial\_jj & -1 & 5634    &    1859 & vivacious\_jj & 1 & 5495 \\
    3 & indirect\_jj & -1 & 5575    &    1858 & girlish\_jj & 1 & 4221 \\
    4 & contentious\_jj & -1 & 5114    &    1857 & self-possessed\_jj & 1 & 4207 \\
    5 & prudent\_jj & -1 & 4918    &    1856 & go-getter & 1 & 4159 \\
    6 & dependent\_jj & -1 & 4899    &    1855 & guileless\_jj & 1 & 3911 \\
    7 & lax\_jj & -1 & 4735    &    1854 & sardonic\_jj & 1 & 3874 \\
    8 & corrective\_jj & -1 & 4600    &    1853 & tomboy & 1 & 3780 \\
    9 & reasonable\_jj & -1 & 4582    &    1852 & coquette & 1 & 3749 \\
    10 & volatile\_jj & -1 & 4504    &    1851 & sassy\_jj & 1 & 3745 \\
    11 & discretionary\_jj & -1 & 4425    &    1850 & impish\_jj & 1 & 3651 \\
    12 & autonomous\_jj & -1 & 4388    &    1849 & coquettish\_jj & 1 & 3606 \\
    13 & drastic\_jj & -1 & 4301    &    1848 & stoic & 1 & 3330 \\
    14 & affected\_jj & -1 & 4080    &    1847 & boyish\_jj & 1 & 3072 \\
    15 & fraudulent\_jj & -1 & 4038    &    1846 & puckish\_jj & 1 & 2916 \\
    16 & systematic\_jj & -1 & 4027    &    1845 & debonair\_jj & 1 & 2837 \\
    17 & indefinite\_jj & -1 & 3976    &    1844 & bubbly\_jj & 1 & 2736 \\
    18 & confidential\_jj & -1 & 3951    &    1843 & witty\_jj & 1 & 2726 \\
    19 & exclusive\_jj & -1 & 3916    &    1842 & beatific\_jj & 1 & 2681 \\
    20 & unfair\_jj & -1 & 3896    &    1841 & flirtatious\_jj & 1 & 2613 \\
    21 & inaccurate\_jj & -1 & 3610    &    1840 & misanthropic\_jj & 1 & 2545 \\
    22 & rigorous\_jj & -1 & 3588    &    1839 & roguish\_jj & 1 & 2485 \\
    23 & stable\_jj & -1 & 3564    &    1838 & suave\_jj & 1 & 2313 \\
    24 & expeditious\_jj & -1 & 3392    &    1837 & rakish\_jj & 1 & 2242 \\
    25 & critical\_jj & -1 & 3378    &    1836 & pert\_jj & 1 & 2204 \\
    26 & forward-looking\_jj & -1 & 3306    &    1835 & poised\_jj & 1 & 2204 \\
    27 & arbitrary\_jj & -1 & 3195    &    1834 & shrewish\_jj & 1 & 2180 \\
    28 & severe\_jj & -1 & 3183    &    1833 & improviser & 1 & 2078 \\
    29 & productive\_jj & -1 & 3155    &    1832 & brassy\_jj & 1 & 2037 \\
    30 & consistent\_jj & -1 & 3132    &    1831 & wry\_jj & 1 & 1954 \\
    \hline
    \end{tabular}
    \caption{Scores and rankings for most extreme 30 words in component \#1} 
\end{table}
\clearpage
\begin{table}[tbp]
    \begin{tabular}{| rlr@{.}l | rlr@{.}l |}
    \hline
    \textbf{Rank} & \textbf{Word} & \multicolumn{2}{c|}{\textbf{Score}} & \textbf{Rank} & \textbf{Word} & \multicolumn{2}{c|}{\textbf{Score}} \\
    \hline
    1 & warm\_jj & -1 & 5098    &    1860 & defamatory\_jj & 1 & 9503 \\
    2 & elegant\_jj & -1 & 4327    &    1859 & bigoted\_jj & 1 & 8854 \\
    3 & vibrant\_jj & -1 & 3495    &    1858 & untruthful\_jj & 1 & 7759 \\
    4 & savant & -1 & 3309    &    1857 & deceitful\_jj & 1 & 7643 \\
    5 & sparkling\_jj & -1 & 3174    &    1856 & cowardly\_jj & 1 & 7461 \\
    6 & luxurious\_jj & -1 & 2799    &    1855 & selfish\_jj & 1 & 7218 \\
    7 & graceful\_jj & -1 & 2748    &    1854 & slanderous\_jj & 1 & 6950 \\
    8 & airy\_jj & -1 & 2745    &    1853 & inhuman\_jj & 1 & 6835 \\
    9 & buttery\_jj & -1 & 2416    &    1852 & hypocritical\_jj & 1 & 6213 \\
    10 & polished\_jj & -1 & 2154    &    1851 & irresponsible\_jj & 1 & 6140 \\
    11 & sunny\_jj & -1 & 2040    &    1850 & vindictive\_jj & 1 & 5512 \\
    12 & sultry\_jj & -1 & 1748    &    1849 & dishonest\_jj & 1 & 5493 \\
    13 & vivacious\_jj & -1 & 1657    &    1848 & disrespectful\_jj & 1 & 5392 \\
    14 & versatile\_jj & -1 & 1600    &    1847 & thoughtless\_jj & 1 & 5376 \\
    15 & lively\_jj & -1 & 1394    &    1846 & unethical\_jj & 1 & 5369 \\
    16 & sociable\_jj & -1 & 1333    &    1845 & mendacious\_jj & 1 & 4831 \\
    17 & intuitive\_jj & -1 & 1201    &    1844 & godless\_jj & 1 & 4676 \\
    18 & rugged\_jj & -1 & 1118    &    1843 & callous\_jj & 1 & 4654 \\
    19 & brisk\_jj & -1 & 1003    &    1842 & narrow-minded\_jj & 1 & 4639 \\
    20 & bright\_jj & -1 & 895    &    1841 & insensitive\_jj & 1 & 4592 \\
    21 & dainty\_jj & -1 & 856    &    1840 & inconsiderate\_jj & 1 & 4581 \\
    22 & sensual\_jj & -1 & 794    &    1839 & polemic\_jj & 1 & 4445 \\
    23 & musical\_jj & -1 & 739    &    1838 & unfeeling\_jj & 1 & 4299 \\
    24 & ethereal\_jj & -1 & 692    &    1837 & intolerant\_jj & 1 & 4180 \\
    25 & calm\_jj & -1 & 683    &    1836 & ignorant\_jj & 1 & 4169 \\
    26 & serene\_jj & -1 & 573    &    1835 & unintelligent\_jj & 1 & 4069 \\
    27 & breezy\_jj & -1 & 496    &    1834 & unprincipled\_jj & 1 & 3858 \\
    28 & outdoor\_jj & -1 & 407    &    1833 & discourteous\_jj & 1 & 3715 \\
    29 & tender\_jj & -1 & 399    &    1832 & inhumane\_jj & 1 & 3687 \\
    30 & gingery\_jj & -1 & 380    &    1831 & boorish\_jj & 1 & 3654 \\
    \hline
    \end{tabular}
    \caption{Scores and rankings for most extreme 30 words in component \#2} 
\end{table}
\clearpage
\begin{table}[tbp]
    \begin{tabular}{| rlr@{.}l | rlr@{.}l |}
    \hline
    \textbf{Rank} & \textbf{Word} & \multicolumn{2}{c|}{\textbf{Score}} & \textbf{Rank} & \textbf{Word} & \multicolumn{2}{c|}{\textbf{Score}} \\
    \hline
    1 & considerate\_jj & -2 & 5489    &    1860 & comforter & 1 & 3462 \\
    2 & sociable\_jj & -2 & 1856    &    1859 & gingery\_jj & 1 & 3287 \\
    3 & courteous\_jj & -2 & 245    &    1858 & butterfly & 1 & 2879 \\
    4 & trustworthy\_jj & -1 & 8932    &    1857 & pixy & 1 & 2802 \\
    5 & articulate\_jj & -1 & 8701    &    1856 & bendable\_jj & 1 & 1698 \\
    6 & open-minded\_jj & -1 & 7427    &    1855 & soft-shelled\_jj & 1 & 1617 \\
    7 & approachable\_jj & -1 & 6488    &    1854 & broiler & 1 & 1602 \\
    8 & kind\_jj & -1 & 6478    &    1853 & boneless\_jj & 1 & 1360 \\
    9 & respectful\_jj & -1 & 6144    &    1852 & low-pressure\_jj & 1 & 1118 \\
    10 & thoughtful\_jj & -1 & 5968    &    1851 & vinegary\_jj & 1 & 534 \\
    11 & self-confident\_jj & -1 & 5586    &    1850 & baneful\_jj & 1 & 522 \\
    12 & easy-going\_jj & -1 & 5185    &    1849 & yellow\_jj & 1 & 346 \\
    13 & level-headed\_jj & -1 & 4879    &    1848 & wildcat\_jj & 1 & 294 \\
    14 & honest\_jj & -1 & 4660    &    1847 & hard-shell\_jj & 1 & 224 \\
    15 & down-to-earth\_jj & -1 & 4561    &    1846 & clam & 1 & 52 \\
    16 & talkative\_jj & -1 & 4395    &    1845 & aplastic\_jj & 0 & 9918 \\
    17 & circumspect\_jj & -1 & 4125    &    1844 & acrid\_jj & 0 & 9855 \\
    18 & intelligent\_jj & -1 & 3709    &    1843 & croaking & 0 & 9774 \\
    19 & perceptive\_jj & -1 & 3562    &    1842 & gourmand & 0 & 9703 \\
    20 & sincere\_jj & -1 & 3528    &    1841 & unmanly\_jj & 0 & 9560 \\
    21 & pragmatic\_jj & -1 & 3465    &    1840 & driftless\_jj & 0 & 9538 \\
    22 & cordial\_jj & -1 & 3340    &    1839 & gooey\_jj & 0 & 9500 \\
    23 & accommodating\_jj & -1 & 3258    &    1838 & bristly\_jj & 0 & 9379 \\
    24 & opinionated\_jj & -1 & 3191    &    1837 & high-hat & 0 & 9373 \\
    25 & affectionate\_jj & -1 & 2924    &    1836 & dare-devil & 0 & 9332 \\
    26 & gregarious\_jj & -1 & 2923    &    1835 & plastic\_jj & 0 & 9279 \\
    27 & knowledgeable\_jj & -1 & 2813    &    1834 & satanic\_jj & 0 & 9209 \\
    28 & deferential\_jj & -1 & 2766    &    1833 & fossil & 0 & 9201 \\
    29 & fair-minded\_jj & -1 & 2745    &    1832 & cry-baby & 0 & 9185 \\
    30 & adaptable\_jj & -1 & 2720    &    1831 & magnetic\_jj & 0 & 9136 \\
    \hline
    \end{tabular}
    \caption{Scores and rankings for most extreme 30 words in component \#3} 
\end{table}
\clearpage
\begin{table}[tbp]
    \begin{tabular}{| rlr@{.}l | rlr@{.}l |}
    \hline
    \textbf{Rank} & \textbf{Word} & \multicolumn{2}{c|}{\textbf{Score}} & \textbf{Rank} & \textbf{Word} & \multicolumn{2}{c|}{\textbf{Score}} \\
    \hline
    1 & loyal\_jj & -1 & 2946    &    1860 & unvarying\_jj & 1 & 3880 \\
    2 & objector & -1 & 2704    &    1859 & limpid\_jj & 1 & 2827 \\
    3 & staunch\_jj & -1 & 2414    &    1858 & dissonant\_jj & 1 & 2230 \\
    4 & devout\_jj & -1 & 2340    &    1857 & imprecise\_jj & 1 & 2080 \\
    5 & sociable\_jj & -1 & 2258    &    1856 & defamatory\_jj & 1 & 1977 \\
    6 & alcoholic & -1 & 1930    &    1855 & didactic\_jj & 1 & 1859 \\
    7 & avid\_jj & -1 & 1584    &    1854 & contrived\_jj & 1 & 1803 \\
    8 & cheater & -1 & 1427    &    1853 & oblique\_jj & 1 & 1784 \\
    9 & kind-hearted\_jj & -1 & 1377    &    1852 & poetic\_jj & 1 & 1614 \\
    10 & stalwart & -1 & 1214    &    1851 & discursive\_jj & 1 & 1497 \\
    11 & gambler & -1 & 1095    &    1850 & abstruse\_jj & 1 & 1321 \\
    12 & stay-at-home\_jj & -1 & 1086    &    1849 & lyrical\_jj & 1 & 1296 \\
    13 & disciplinarian & -1 & 985    &    1848 & meditative\_jj & 1 & 1290 \\
    14 & angler & -1 & 716    &    1847 & stilted\_jj & 1 & 1209 \\
    15 & teetotaler & -1 & 589    &    1846 & astringent\_jj & 1 & 1108 \\
    16 & intrepid\_jj & -1 & 579    &    1845 & deterministic\_jj & 1 & 721 \\
    17 & loving\_jj & -1 & 455    &    1844 & melodramatic\_jj & 1 & 468 \\
    18 & samaritan & -1 & 428    &    1843 & subjective\_jj & 1 & 282 \\
    19 & visionary & -1 & 387    &    1842 & abstract\_jj & 1 & 267 \\
    20 & gregarious\_jj & -1 & 374    &    1841 & morbid\_jj & 1 & 126 \\
    21 & fan & -1 & 254    &    1840 & pithy\_jj & 1 & 80 \\
    22 & go-getter & -1 & 141    &    1839 & mawkish\_jj & 0 & 9893 \\
    23 & outgoing\_jj & 0 & 9961    &    1838 & tasteless\_jj & 0 & 9852 \\
    24 & has-been & 0 & 9890    &    1837 & ethereal\_jj & 0 & 9816 \\
    25 & brat & 0 & 9625    &    1836 & nonsensical\_jj & 0 & 9686 \\
    26 & lucky\_jj & 0 & 9602    &    1835 & discordant\_jj & 0 & 9599 \\
    27 & brahmin & 0 & 9473    &    1834 & economical\_jj & 0 & 9497 \\
    28 & bulldog & 0 & 9427    &    1833 & declamatory\_jj & 0 & 9448 \\
    29 & proud\_jj & 0 & 9236    &    1832 & intricate\_jj & 0 & 9335 \\
    30 & dissident & 0 & 9218    &    1831 & literal\_jj & 0 & 9216 \\
    \hline
    \end{tabular}
    \caption{Scores and rankings for most extreme 30 words in component \#4} 
\end{table}
\clearpage
\begin{table}[tbp]
    \begin{tabular}{| rlr@{.}l | rlr@{.}l |}
    \hline
    \textbf{Rank} & \textbf{Word} & \multicolumn{2}{c|}{\textbf{Score}} & \textbf{Rank} & \textbf{Word} & \multicolumn{2}{c|}{\textbf{Score}} \\
    \hline
    1 & deterministic\_jj & -1 & 3347    &    1860 & testy\_jj & 1 & 5996 \\
    2 & self-reliant\_jj & -1 & 2717    &    1859 & lewd\_jj & 1 & 5434 \\
    3 & adaptable\_jj & -1 & 2382    &    1858 & conciliatory\_jj & 1 & 3503 \\
    4 & theistic\_jj & -1 & 1938    &    1857 & disorderly\_jj & 1 & 2755 \\
    5 & mutable\_jj & -1 & 1543    &    1856 & unsportsmanlike\_jj & 1 & 2691 \\
    6 & educated\_jj & -1 & 1473    &    1855 & defiant\_jj & 1 & 2281 \\
    7 & unchangeable\_jj & -1 & 1213    &    1854 & rowdy\_jj & 1 & 2165 \\
    8 & closed-minded\_jj & -1 & 1175    &    1853 & derisive\_jj & 1 & 2149 \\
    9 & individualistic\_jj & -1 & 1115    &    1852 & vitriolic\_jj & 1 & 2016 \\
    10 & open-minded\_jj & -1 & 850    &    1851 & congratulatory\_jj & 1 & 1838 \\
    11 & conformist\_jj & -1 & 520    &    1850 & risqué\_jj & 1 & 1716 \\
    12 & self-sufficient\_jj & -1 & 477    &    1849 & acrimonious\_jj & 1 & 1677 \\
    13 & modifiable\_jj & -1 & 204    &    1848 & thunderous\_jj & 1 & 1593 \\
    14 & materialistic\_jj & -1 & 140    &    1847 & caustic\_jj & 1 & 1590 \\
    15 & equitable\_jj & -1 & 15    &    1846 & rancorous\_jj & 1 & 1356 \\
    16 & intelligent\_jj & 0 & 9669    &    1845 & stormy\_jj & 1 & 1290 \\
    17 & unchanging\_jj & 0 & 9554    &    1844 & stern\_jj & 1 & 1220 \\
    18 & considerate\_jj & 0 & 9539    &    1843 & defamatory\_jj & 1 & 1208 \\
    19 & perspicuous\_jj & 0 & 9437    &    1842 & loud\_jj & 1 & 1102 \\
    20 & altruistic\_jj & 0 & 9322    &    1841 & raucous\_jj & 1 & 657 \\
    21 & enlightened\_jj & 0 & 9314    &    1840 & frosty\_jj & 1 & 576 \\
    22 & intuitive\_jj & 0 & 9281    &    1839 & verbal\_jj & 1 & 491 \\
    23 & irreligious\_jj & 0 & 8973    &    1838 & lukewarm\_jj & 1 & 389 \\
    24 & celibate & 0 & 8957    &    1837 & lascivious\_jj & 1 & 357 \\
    25 & utopian\_jj & 0 & 8913    &    1836 & obscene\_jj & 1 & 331 \\
    26 & immutable\_jj & 0 & 8864    &    1835 & torrid\_jj & 1 & 249 \\
    27 & ethnocentric\_jj & 0 & 8841    &    1834 & bitter\_jj & 1 & 121 \\
    28 & sociable\_jj & 0 & 8814    &    1833 & curt\_jj & 1 & 76 \\
    29 & imitative\_jj & 0 & 8807    &    1832 & derogatory\_jj & 0 & 9997 \\
    30 & variant\_jj & 0 & 8788    &    1831 & foul-mouthed\_jj & 0 & 9727 \\
    \hline
    \end{tabular}
    \caption{Scores and rankings for most extreme 30 words in component \#5} 
\end{table}
\clearpage
\begin{table}[tbp]
    \begin{tabular}{| rlr@{.}l | rlr@{.}l |}
    \hline
    \textbf{Rank} & \textbf{Word} & \multicolumn{2}{c|}{\textbf{Score}} & \textbf{Rank} & \textbf{Word} & \multicolumn{2}{c|}{\textbf{Score}} \\
    \hline
    1 & unbending\_jj & -1 & 4267    &    1860 & sociable\_jj & 1 & 4490 \\
    2 & unswerving\_jj & -1 & 4115    &    1859 & picky\_jj & 1 & 4147 \\
    3 & ultraconservative\_jj & -1 & 3240    &    1858 & irritable\_jj & 1 & 3734 \\
    4 & unfailing\_jj & -1 & 3170    &    1857 & lazy\_jj & 1 & 2923 \\
    5 & unflagging\_jj & -1 & 3101    &    1856 & inconsiderate\_jj & 1 & 2263 \\
    6 & unshakable\_jj & -1 & 3083    &    1855 & lewd\_jj & 1 & 2187 \\
    7 & firebrand & -1 & 2413    &    1854 & abusive\_jj & 1 & 1609 \\
    8 & steadfast\_jj & -1 & 2163    &    1853 & clingy\_jj & 1 & 1464 \\
    9 & untiring\_jj & -1 & 1629    &    1852 & fussy\_jj & 1 & 1435 \\
    10 & unflinching\_jj & -1 & 1620    &    1851 & rude\_jj & 1 & 1224 \\
    11 & uncompromising\_jj & -1 & 1535    &    1850 & sexy\_jj & 1 & 813 \\
    12 & unquestioning\_jj & -1 & 1397    &    1849 & defamatory\_jj & 1 & 711 \\
    13 & secular\_jj & -1 & 1332    &    1848 & gooey\_jj & 1 & 683 \\
    14 & dissident & -1 & 1248    &    1847 & lascivious\_jj & 1 & 419 \\
    15 & unyielding\_jj & -1 & 1186    &    1846 & frisky\_jj & 1 & 351 \\
    16 & trenchant\_jj & -1 & 1019    &    1845 & talkative\_jj & 1 & 128 \\
    17 & oratorical\_jj & -1 & 948    &    1844 & bubbly\_jj & 0 & 9948 \\
    18 & staunch\_jj & -1 & 900    &    1843 & disorderly\_jj & 0 & 9945 \\
    19 & imperturbable\_jj & -1 & 731    &    1842 & sedentary\_jj & 0 & 9859 \\
    20 & fervent\_jj & -1 & 248    &    1841 & choosy\_jj & 0 & 9801 \\
    21 & doctrinaire\_jj & 0 & 9908    &    1840 & oily\_jj & 0 & 9748 \\
    22 & outspoken\_jj & 0 & 9887    &    1839 & risqué\_jj & 0 & 9620 \\
    23 & autocratic\_jj & 0 & 9848    &    1838 & addicted\_jj & 0 & 9571 \\
    24 & unerring\_jj & 0 & 9832    &    1837 & distractible\_jj & 0 & 9377 \\
    25 & magisterial\_jj & 0 & 9767    &    1836 & kind\_jj & 0 & 9276 \\
    26 & militant & 0 & 9683    &    1835 & considerate\_jj & 0 & 9252 \\
    27 & scholarly\_jj & 0 & 9305    &    1834 & autistic\_jj & 0 & 9188 \\
    28 & dogged\_jj & 0 & 9246    &    1833 & tasteful\_jj & 0 & 9120 \\
    29 & religious\_jj & 0 & 9162    &    1832 & buttery\_jj & 0 & 9075 \\
    30 & ascendant\_jj & 0 & 9142    &    1831 & bitch & 0 & 9026 \\
    \hline
    \end{tabular}
    \caption{Scores and rankings for most extreme 30 words in component \#6} 
\end{table}
\clearpage
\begin{table}[tbp]
    \begin{tabular}{| rlr@{.}l | rlr@{.}l |}
    \hline
    \textbf{Rank} & \textbf{Word} & \multicolumn{2}{c|}{\textbf{Score}} & \textbf{Rank} & \textbf{Word} & \multicolumn{2}{c|}{\textbf{Score}} \\
    \hline
    1 & assertive\_jj & -1 & 1824    &    1860 & savant & 2 & 2627 \\
    2 & inhospitable\_jj & -1 & 1821    &    1859 & lewd\_jj & 1 & 4703 \\
    3 & stagnant\_jj & -1 & 1554    &    1858 & defamatory\_jj & 1 & 4234 \\
    4 & distrustful\_jj & -1 & 1357    &    1857 & lascivious\_jj & 1 & 3840 \\
    5 & impervious\_jj & -1 & 1018    &    1856 & slanderous\_jj & 1 & 1310 \\
    6 & conformist\_jj & -1 & 731    &    1855 & sincere\_jj & 1 & 503 \\
    7 & mistrustful\_jj & -1 & 276    &    1854 & unbiased\_jj & 1 & 495 \\
    8 & fractious\_jj & -1 & 233    &    1853 & musical\_jj & 1 & 364 \\
    9 & apathetic\_jj & -1 & 231    &    1852 & obscene\_jj & 1 & 225 \\
    10 & blase\_jj & -1 & 63    &    1851 & liar & 1 & 222 \\
    11 & autocratic\_jj & 0 & 9912    &    1850 & comedian & 1 & 212 \\
    12 & docile\_jj & 0 & 9729    &    1849 & teachable\_jj & 1 & 205 \\
    13 & laggard\_jj & 0 & 9629    &    1848 & cry-baby & 1 & 88 \\
    14 & reliant\_jj & 0 & 9578    &    1847 & autistic\_jj & 1 & 87 \\
    15 & sanguine\_jj & 0 & 9468    &    1846 & cognitive\_jj & 1 & 79 \\
    16 & spendthrift\_jj & 0 & 9458    &    1845 & quitter & 1 & 43 \\
    17 & intransigent\_jj & 0 & 8998    &    1844 & concise\_jj & 1 & 22 \\
    18 & inclement\_jj & 0 & 8975    &    1843 & malicious\_jj & 0 & 9812 \\
    19 & pliable\_jj & 0 & 8870    &    1842 & unreserved\_jj & 0 & 9754 \\
    20 & treacherous\_jj & 0 & 8806    &    1841 & derogatory\_jj & 0 & 9458 \\
    21 & clement\_jj & 0 & 8784    &    1840 & objective\_jj & 0 & 9401 \\
    22 & fickle\_jj & 0 & 8779    &    1839 & considerate\_jj & 0 & 9255 \\
    23 & hospitable\_jj & 0 & 8732    &    1838 & thorough\_jj & 0 & 9224 \\
    24 & icy\_jj & 0 & 8685    &    1837 & objector & 0 & 9198 \\
    25 & choosy\_jj & 0 & 8662    &    1836 & retrospective\_jj & 0 & 9142 \\
    26 & rapacious\_jj & 0 & 8605    &    1835 & insightful\_jj & 0 & 9115 \\
    27 & wary\_jj & 0 & 8535    &    1834 & negligent\_jj & 0 & 9016 \\
    28 & clannish\_jj & 0 & 8527    &    1833 & disorderly\_jj & 0 & 8961 \\
    29 & indecisive\_jj & 0 & 8501    &    1832 & systematic\_jj & 0 & 8882 \\
    30 & sluggish\_jj & 0 & 8383    &    1831 & fraudulent\_jj & 0 & 8802 \\
    \hline
    \end{tabular}
    \caption{Scores and rankings for most extreme 30 words in component \#7} 
\end{table}
\clearpage
\begin{table}[tbp]
    \begin{tabular}{| rlr@{.}l | rlr@{.}l |}
    \hline
    \textbf{Rank} & \textbf{Word} & \multicolumn{2}{c|}{\textbf{Score}} & \textbf{Rank} & \textbf{Word} & \multicolumn{2}{c|}{\textbf{Score}} \\
    \hline
    1 & inhuman\_jj & -1 & 6995    &    1860 & spendthrift & 1 & 5559 \\
    2 & savant & -1 & 4788    &    1859 & remiss\_jj & 1 & 4284 \\
    3 & lascivious\_jj & -1 & 4545    &    1858 & quitter & 1 & 1649 \\
    4 & inhumane\_jj & -1 & 2918    &    1857 & tightwad & 1 & 371 \\
    5 & disorderly\_jj & -1 & 2663    &    1856 & lukewarm\_jj & 1 & 180 \\
    6 & forcible\_jj & -1 & 2618    &    1855 & wishy-washy\_jj & 1 & 101 \\
    7 & wanton\_jj & -1 & 1721    &    1854 & facetious\_jj & 1 & 80 \\
    8 & abusive\_jj & -1 & 1437    &    1853 & squeamish\_jj & 1 & 49 \\
    9 & libidinous\_jj & -1 & 1321    &    1852 & pessimistic\_jj & 0 & 9911 \\
    10 & impulsive\_jj & -1 & 1157    &    1851 & sanguine\_jj & 0 & 9816 \\
    11 & obstructive\_jj & -1 & 1018    &    1850 & coy\_jj & 0 & 9788 \\
    12 & autistic\_jj & -1 & 635    &    1849 & circumspect\_jj & 0 & 9767 \\
    13 & lewd\_jj & -1 & 630    &    1848 & pessimist & 0 & 9527 \\
    14 & sadistic\_jj & -1 & 524    &    1847 & cautionary\_jj & 0 & 9445 \\
    15 & unstable\_jj & -1 & 179    &    1846 & do-nothing & 0 & 9310 \\
    16 & overactive\_jj & 0 & 9890    &    1845 & sophistic\_jj & 0 & 9251 \\
    17 & antisocial\_jj & 0 & 9544    &    1844 & picky\_jj & 0 & 9193 \\
    18 & manipulative\_jj & 0 & 9451    &    1843 & choosy\_jj & 0 & 9044 \\
    19 & uncontrolled\_jj & 0 & 9132    &    1842 & preachy\_jj & 0 & 9001 \\
    20 & expressive\_jj & 0 & 9043    &    1841 & churlish\_jj & 0 & 8749 \\
    21 & unpredictable\_jj & 0 & 9000    &    1840 & wishful\_jj & 0 & 8601 \\
    22 & brutal\_jj & 0 & 8906    &    1839 & gun-shy\_jj & 0 & 8489 \\
    23 & uncooperative\_jj & 0 & 8654    &    1838 & cogent\_jj & 0 & 8485 \\
    24 & violent\_jj & 0 & 8641    &    1837 & concise\_jj & 0 & 8451 \\
    25 & illiterate\_jj & 0 & 8525    &    1836 & unenthusiastic\_jj & 0 & 8284 \\
    26 & unruly\_jj & 0 & 8512    &    1835 & presumptuous\_jj & 0 & 8203 \\
    27 & acute\_jj & 0 & 8349    &    1834 & skeptical\_jj & 0 & 8138 \\
    28 & aplastic\_jj & 0 & 8269    &    1833 & highfalutin\_jj & 0 & 8134 \\
    29 & possessive\_jj & 0 & 8163    &    1832 & cautious\_jj & 0 & 8035 \\
    30 & cerebral\_jj & 0 & 8072    &    1831 & carper & 0 & 8002 \\
    \hline
    \end{tabular}
    \caption{Scores and rankings for most extreme 30 words in component \#8} 
\end{table}
\clearpage
\begin{table}[tbp]
    \begin{tabular}{| rlr@{.}l | rlr@{.}l |}
    \hline
    \textbf{Rank} & \textbf{Word} & \multicolumn{2}{c|}{\textbf{Score}} & \textbf{Rank} & \textbf{Word} & \multicolumn{2}{c|}{\textbf{Score}} \\
    \hline
    1 & vegetative\_jj & -1 & 8537    &    1860 & chic\_jj & 1 & 2239 \\
    2 & irritable\_jj & -1 & 8318    &    1859 & cowardly\_jj & 1 & 1353 \\
    3 & abstinent\_jj & -1 & 7522    &    1858 & elegant\_jj & 1 & 471 \\
    4 & aplastic\_jj & -1 & 7437    &    1857 & cunning\_jj & 0 & 9793 \\
    5 & overactive\_jj & -1 & 6945    &    1856 & flashy\_jj & 0 & 9420 \\
    6 & obstructive\_jj & -1 & 4896    &    1855 & brazen\_jj & 0 & 8982 \\
    7 & autistic\_jj & -1 & 4887    &    1854 & luxurious\_jj & 0 & 8877 \\
    8 & cognitive\_jj & -1 & 4731    &    1853 & gourmet & 0 & 8824 \\
    9 & refractory\_jj & -1 & 4541    &    1852 & risque\_jj & 0 & 8820 \\
    10 & affective\_jj & -1 & 4177    &    1851 & downright\_jj & 0 & 8782 \\
    11 & malignant\_jj & -1 & 3867    &    1850 & barbarous\_jj & 0 & 8561 \\
    12 & modifiable\_jj & -1 & 1457    &    1849 & sexy\_jj & 0 & 8378 \\
    13 & cerebral\_jj & -1 & 463    &    1848 & wasteful\_jj & 0 & 8310 \\
    14 & acute\_jj & -1 & 224    &    1847 & clever\_jj & 0 & 8038 \\
    15 & hypersensitive\_jj & -1 & 178    &    1846 & homespun\_jj & 0 & 7997 \\
    16 & testy\_jj & -1 & 47    &    1845 & extravagant\_jj & 0 & 7968 \\
    17 & unemotional\_jj & 0 & 9772    &    1844 & ruthless\_jj & 0 & 7896 \\
    18 & maternal\_jj & 0 & 9332    &    1843 & cosmopolitan\_jj & 0 & 7833 \\
    19 & extroverted\_jj & 0 & 9247    &    1842 & greedy\_jj & 0 & 7813 \\
    20 & nonconforming\_jj & 0 & 8947    &    1841 & devious\_jj & 0 & 7774 \\
    21 & mild\_jj & 0 & 8858    &    1840 & clown & 0 & 7732 \\
    22 & impulsive\_jj & 0 & 8739    &    1839 & sophisticated\_jj & 0 & 7673 \\
    23 & unfaithful\_jj & 0 & 8728    &    1838 & witch & 0 & 7650 \\
    24 & questioning\_jj & 0 & 8721    &    1837 & undemocratic\_jj & 0 & 7594 \\
    25 & distractible\_jj & 0 & 8606    &    1836 & fanciful\_jj & 0 & 7584 \\
    26 & obsessive & 0 & 8579    &    1835 & lavish\_jj & 0 & 7460 \\
    27 & possessive\_jj & 0 & 8566    &    1834 & gullible\_jj & 0 & 7456 \\
    28 & objector & 0 & 8541    &    1833 & quirky\_jj & 0 & 7418 \\
    29 & unresponsive\_jj & 0 & 8530    &    1832 & ostentatious\_jj & 0 & 7340 \\
    30 & compulsive & 0 & 8378    &    1831 & unscrupulous\_jj & 0 & 7314 \\
    \hline
    \end{tabular}
    \caption{Scores and rankings for most extreme 30 words in component \#9} 
\end{table}
\clearpage
\begin{table}[tbp]
    \begin{tabular}{| rlr@{.}l | rlr@{.}l |}
    \hline
    \textbf{Rank} & \textbf{Word} & \multicolumn{2}{c|}{\textbf{Score}} & \textbf{Rank} & \textbf{Word} & \multicolumn{2}{c|}{\textbf{Score}} \\
    \hline
    1 & lascivious\_jj & -1 & 5148    &    1860 & savant & 1 & 1857 \\
    2 & secular\_jj & -1 & 5081    &    1859 & inexact\_jj & 1 & 1317 \\
    3 & religious\_jj & -1 & 3663    &    1858 & unguarded\_jj & 1 & 152 \\
    4 & devout\_jj & -1 & 2977    &    1857 & mathematical\_jj & 0 & 9971 \\
    5 & celibate & -1 & 2739    &    1856 & indestructible\_jj & 0 & 9625 \\
    6 & defamatory\_jj & -1 & 2712    &    1855 & accomplished\_jj & 0 & 9246 \\
    7 & forcible\_jj & -1 & 2430    &    1854 & adroit\_jj & 0 & 8891 \\
    8 & blasphemous\_jj & -1 & 1629    &    1853 & unreliable\_jj & 0 & 8882 \\
    9 & nonreligious\_jj & -1 & 1517    &    1852 & balky\_jj & 0 & 8771 \\
    10 & peaceful\_jj & -1 & 1458    &    1851 & egghead & 0 & 8721 \\
    11 & prayerful\_jj & -1 & 1260    &    1850 & imperturbable\_jj & 0 & 8552 \\
    12 & buttery\_jj & -1 & 809    &    1849 & inexperienced\_jj & 0 & 8148 \\
    13 & ultraconservative\_jj & -1 & 731    &    1848 & overactive\_jj & 0 & 7749 \\
    14 & inhuman\_jj & -1 & 502    &    1847 & canny\_jj & 0 & 7703 \\
    15 & heretical\_jj & -1 & 403    &    1846 & astute\_jj & 0 & 7693 \\
    16 & solemn\_jj & -1 & 381    &    1845 & cunning & 0 & 7622 \\
    17 & warm\_jj & -1 & 282    &    1844 & audacious\_jj & 0 & 7580 \\
    18 & lewd\_jj & -1 & 152    &    1843 & unerring\_jj & 0 & 7560 \\
    19 & respectful\_jj & -1 & 150    &    1842 & assured\_jj & 0 & 7517 \\
    20 & tolerant\_jj & -1 & 114    &    1841 & overconfident\_jj & 0 & 7478 \\
    21 & pious\_jj & 0 & 9930    &    1840 & accurate\_jj & 0 & 7355 \\
    22 & cordial\_jj & 0 & 9863    &    1839 & irascible\_jj & 0 & 7299 \\
    23 & loving\_jj & 0 & 9840    &    1838 & iffy\_jj & 0 & 7251 \\
    24 & hearty\_jj & 0 & 9840    &    1837 & absent-minded\_jj & 0 & 7194 \\
    25 & monastic\_jj & 0 & 9796    &    1836 & opportunist & 0 & 7139 \\
    26 & peppery\_jj & 0 & 9776    &    1835 & upstart\_jj & 0 & 7052 \\
    27 & sincere\_jj & 0 & 9762    &    1834 & hard-nosed\_jj & 0 & 6980 \\
    28 & satanic\_jj & 0 & 9354    &    1833 & introvert & 0 & 6947 \\
    29 & godless\_jj & 0 & 9313    &    1832 & improviser & 0 & 6895 \\
    30 & puritanical\_jj & 0 & 9165    &    1831 & abrupt\_jj & 0 & 6842 \\
    \hline
    \end{tabular}
    \caption{Scores and rankings for most extreme 30 words in component \#10} 
\end{table}
\clearpage
\begin{table}[tbp]
    \begin{tabular}{| rlr@{.}l | rlr@{.}l |}
    \hline
    \textbf{Rank} & \textbf{Word} & \multicolumn{2}{c|}{\textbf{Score}} & \textbf{Rank} & \textbf{Word} & \multicolumn{2}{c|}{\textbf{Score}} \\
    \hline
    1 & ultraconservative\_jj & -1 & 7353    &    1860 & cheater & 1 & 2231 \\
    2 & exclusive\_jj & -1 & 5433    &    1859 & cowardly\_jj & 1 & 1531 \\
    3 & eclectic\_jj & -1 & 3794    &    1858 & methodical\_jj & 1 & 1111 \\
    4 & defamatory\_jj & -1 & 3677    &    1857 & merciless\_jj & 1 & 547 \\
    5 & savant & -1 & 3523    &    1856 & cold-blooded\_jj & 1 & 467 \\
    6 & outspoken\_jj & -1 & 2438    &    1855 & deliberate\_jj & 0 & 9892 \\
    7 & risqué\_jj & -1 & 2410    &    1854 & callous\_jj & 0 & 9480 \\
    8 & avid\_jj & -1 & 2098    &    1853 & doer & 0 & 8722 \\
    9 & unguarded\_jj & -1 & 1173    &    1852 & calculating\_jj & 0 & 8655 \\
    10 & obscene\_jj & -1 & 1100    &    1851 & remorseless\_jj & 0 & 8611 \\
    11 & risque\_jj & -1 & 335    &    1850 & cunning & 0 & 8478 \\
    12 & acrimonious\_jj & -1 & 166    &    1849 & merciful\_jj & 0 & 8255 \\
    13 & explicit\_jj & 0 & 9539    &    1848 & cunning\_jj & 0 & 8216 \\
    14 & irreverent\_jj & 0 & 9051    &    1847 & humane\_jj & 0 & 8036 \\
    15 & observant\_jj & 0 & 8950    &    1846 & brute\_jj & 0 & 7998 \\
    16 & outdoor\_jj & 0 & 8868    &    1845 & speedy\_jj & 0 & 7968 \\
    17 & derogatory\_jj & 0 & 8657    &    1844 & ruthless\_jj & 0 & 7935 \\
    18 & abusive\_jj & 0 & 8484    &    1843 & selfless\_jj & 0 & 7881 \\
    19 & immodest\_jj & 0 & 8444    &    1842 & valiant\_jj & 0 & 7816 \\
    20 & uncooperative\_jj & 0 & 8321    &    1841 & vegetative\_jj & 0 & 7769 \\
    21 & animated\_jj & 0 & 8283    &    1840 & barbarous\_jj & 0 & 7743 \\
    22 & encyclopedic\_jj & 0 & 8267    &    1839 & deliberative\_jj & 0 & 7629 \\
    23 & highbrow\_jj & 0 & 8191    &    1838 & inhuman\_jj & 0 & 7574 \\
    24 & itinerant & 0 & 8188    &    1837 & devious\_jj & 0 & 7537 \\
    25 & independent\_jj & 0 & 8115    &    1836 & sadistic\_jj & 0 & 7479 \\
    26 & unruly\_jj & 0 & 8057    &    1835 & heroic\_jj & 0 & 7329 \\
    27 & lewd\_jj & 0 & 8030    &    1834 & brave\_jj & 0 & 7261 \\
    28 & intellectual & 0 & 8021    &    1833 & thorough\_jj & 0 & 7179 \\
    29 & blasphemous\_jj & 0 & 7827    &    1832 & precipitous\_jj & 0 & 7143 \\
    30 & informal\_jj & 0 & 7746    &    1831 & quitter & 0 & 7081 \\
    \hline
    \end{tabular}
    \caption{Scores and rankings for most extreme 30 words in component \#11} 
\end{table}
\clearpage
\begin{table}[tbp]
    \begin{tabular}{| rlr@{.}l | rlr@{.}l |}
    \hline
    \textbf{Rank} & \textbf{Word} & \multicolumn{2}{c|}{\textbf{Score}} & \textbf{Rank} & \textbf{Word} & \multicolumn{2}{c|}{\textbf{Score}} \\
    \hline
    1 & morbid\_jj & -1 & 2780    &    1860 & vinegary\_jj & 1 & 7602 \\
    2 & rebellious\_jj & -1 & 491    &    1859 & peppery\_jj & 1 & 5106 \\
    3 & macabre\_jj & -1 & 446    &    1858 & gingery\_jj & 1 & 4939 \\
    4 & celibate & -1 & 342    &    1857 & broiler & 1 & 4573 \\
    5 & mystical\_jj & -1 & 148    &    1856 & buttery\_jj & 1 & 4437 \\
    6 & chaste\_jj & 0 & 9332    &    1855 & oily\_jj & 1 & 3574 \\
    7 & nomadic\_jj & 0 & 9185    &    1854 & unctuous\_jj & 1 & 2890 \\
    8 & bleak\_jj & 0 & 9126    &    1853 & unsportsmanlike\_jj & 1 & 2420 \\
    9 & materialistic\_jj & 0 & 8713    &    1852 & boneless\_jj & 1 & 2097 \\
    10 & dispiriting\_jj & 0 & 8597    &    1851 & impartial\_jj & 1 & 1892 \\
    11 & satanic\_jj & 0 & 8502    &    1850 & unrefined\_jj & 1 & 156 \\
    12 & risqué\_jj & 0 & 8414    &    1849 & gooey\_jj & 0 & 9535 \\
    13 & lawless\_jj & 0 & 8357    &    1848 & imperturbable\_jj & 0 & 9296 \\
    14 & violent\_jj & 0 & 8272    &    1847 & fruit & 0 & 8959 \\
    15 & fatalistic\_jj & 0 & 8195    &    1846 & expeditious\_jj & 0 & 8947 \\
    16 & cry-baby & 0 & 8139    &    1845 & absent-minded\_jj & 0 & 8866 \\
    17 & cautionary\_jj & 0 & 7811    &    1844 & pungent\_jj & 0 & 8865 \\
    18 & curious\_jj & 0 & 7664    &    1843 & ham & 0 & 8862 \\
    19 & mundane\_jj & 0 & 7623    &    1842 & earthy\_jj & 0 & 8766 \\
    20 & raunchy\_jj & 0 & 7483    &    1841 & clam & 0 & 8708 \\
    21 & bloodthirsty\_jj & 0 & 7459    &    1840 & prompt\_jj & 0 & 8637 \\
    22 & hectic\_jj & 0 & 7379    &    1839 & unreserved\_jj & 0 & 8521 \\
    23 & distant\_jj & 0 & 7351    &    1838 & ungracious\_jj & 0 & 8501 \\
    24 & autistic\_jj & 0 & 7307    &    1837 & unbending\_jj & 0 & 8300 \\
    25 & mawkish\_jj & 0 & 7295    &    1836 & hearty\_jj & 0 & 7995 \\
    26 & profound\_jj & 0 & 7290    &    1835 & crusty\_jj & 0 & 7851 \\
    27 & otherworldly\_jj & 0 & 7210    &    1834 & pert\_jj & 0 & 7846 \\
    28 & rootless\_jj & 0 & 7030    &    1833 & unbiased\_jj & 0 & 7757 \\
    29 & self-conscious\_jj & 0 & 7017    &    1832 & uncooperative\_jj & 0 & 7734 \\
    30 & chaotic\_jj & 0 & 6983    &    1831 & comforter & 0 & 7721 \\
    \hline
    \end{tabular}
    \caption{Scores and rankings for most extreme 30 words in component \#12} 
\end{table}
\clearpage
\begin{table}[tbp]
    \begin{tabular}{| rlr@{.}l | rlr@{.}l |}
    \hline
    \textbf{Rank} & \textbf{Word} & \multicolumn{2}{c|}{\textbf{Score}} & \textbf{Rank} & \textbf{Word} & \multicolumn{2}{c|}{\textbf{Score}} \\
    \hline
    1 & lenient\_jj & -1 & 1034    &    1860 & peppery\_jj & 1 & 4891 \\
    2 & coquette & -1 & 665    &    1859 & pungent\_jj & 1 & 2129 \\
    3 & corrective\_jj & -1 & 528    &    1858 & broiler & 1 & 1920 \\
    4 & stringent\_jj & -1 & 341    &    1857 & tender\_jj & 1 & 1424 \\
    5 & expeditious\_jj & 0 & 9692    &    1856 & acrid\_jj & 1 & 675 \\
    6 & lascivious\_jj & 0 & 9657    &    1855 & gingery\_jj & 1 & 384 \\
    7 & uncooperative\_jj & 0 & 9410    &    1854 & poisonous\_jj & 1 & 239 \\
    8 & equitable\_jj & 0 & 9127    &    1853 & sour\_jj & 0 & 9939 \\
    9 & demure\_jj & 0 & 9115    &    1852 & boneless\_jj & 0 & 9875 \\
    10 & ladylike\_jj & 0 & 9023    &    1851 & downright\_jj & 0 & 9615 \\
    11 & disciplinarian & 0 & 8786    &    1850 & fiery\_jj & 0 & 9484 \\
    12 & prim\_jj & 0 & 8737    &    1849 & fruit & 0 & 9458 \\
    13 & lewd\_jj & 0 & 8720    &    1848 & oily\_jj & 0 & 9441 \\
    14 & libidinous\_jj & 0 & 8425    &    1847 & venomous\_jj & 0 & 9379 \\
    15 & transparent\_jj & 0 & 8360    &    1846 & mushy\_jj & 0 & 9324 \\
    16 & playboy & 0 & 8111    &    1845 & vinegary\_jj & 0 & 9230 \\
    17 & flirtatious\_jj & 0 & 8104    &    1844 & buttery\_jj & 0 & 9169 \\
    18 & insubordinate\_jj & 0 & 8085    &    1843 & hard-boiled\_jj & 0 & 8940 \\
    19 & risqué\_jj & 0 & 7881    &    1842 & hearty\_jj & 0 & 8707 \\
    20 & mousy\_jj & 0 & 7877    &    1841 & gooey\_jj & 0 & 8474 \\
    21 & rigorous\_jj & 0 & 7768    &    1840 & tasteless\_jj & 0 & 8446 \\
    22 & intrusive\_jj & 0 & 7763    &    1839 & ignorant\_jj & 0 & 8385 \\
    23 & lax\_jj & 0 & 7747    &    1838 & uneducated\_jj & 0 & 8288 \\
    24 & rakish\_jj & 0 & 7688    &    1837 & bitter\_jj & 0 & 8035 \\
    25 & unsportsmanlike\_jj & 0 & 7572    &    1836 & pundit & 0 & 7939 \\
    26 & forcible\_jj & 0 & 7451    &    1835 & ham & 0 & 7793 \\
    27 & strict\_jj & 0 & 7398    &    1834 & clam & 0 & 7703 \\
    28 & monopolistic\_jj & 0 & 7360    &    1833 & erudite\_jj & 0 & 7693 \\
    29 & disorderly\_jj & 0 & 7351    &    1832 & insightful\_jj & 0 & 7642 \\
    30 & submissive\_jj & 0 & 7308    &    1831 & crusty\_jj & 0 & 7612 \\
    \hline
    \end{tabular}
    \caption{Scores and rankings for most extreme 30 words in component \#13} 
\end{table}
\clearpage
\begin{table}[tbp]
    \begin{tabular}{| rlr@{.}l | rlr@{.}l |}
    \hline
    \textbf{Rank} & \textbf{Word} & \multicolumn{2}{c|}{\textbf{Score}} & \textbf{Rank} & \textbf{Word} & \multicolumn{2}{c|}{\textbf{Score}} \\
    \hline
    1 & unreserved\_jj & -1 & 2678    &    1860 & cerebral\_jj & 1 & 1148 \\
    2 & indeterminate\_jj & -1 & 1912    &    1859 & variant\_jj & 1 & 229 \\
    3 & amicable\_jj & -1 & 1681    &    1858 & mathematical\_jj & 1 & 176 \\
    4 & indefinite\_jj & -1 & 1471    &    1857 & firebrand & 0 & 9969 \\
    5 & remiss\_jj & -1 & 427    &    1856 & refractory\_jj & 0 & 9841 \\
    6 & unfailing\_jj & -1 & 131    &    1855 & resistive\_jj & 0 & 9732 \\
    7 & unguarded\_jj & -1 & 125    &    1854 & aplastic\_jj & 0 & 9457 \\
    8 & imprudent\_jj & 0 & 9914    &    1853 & cognitive\_jj & 0 & 9364 \\
    9 & indomitable\_jj & 0 & 9457    &    1852 & malignant\_jj & 0 & 9167 \\
    10 & unsportsmanlike\_jj & 0 & 8840    &    1851 & sophisticated\_jj & 0 & 8808 \\
    11 & abject\_jj & 0 & 8769    &    1850 & moderate\_jj & 0 & 8544 \\
    12 & unfaithful\_jj & 0 & 8734    &    1849 & obstructive\_jj & 0 & 8157 \\
    13 & unselfish\_jj & 0 & 8595    &    1848 & folksy\_jj & 0 & 8054 \\
    14 & intrepid\_jj & 0 & 8449    &    1847 & affective\_jj & 0 & 7813 \\
    15 & immovable\_jj & 0 & 8375    &    1846 & secular\_jj & 0 & 7768 \\
    16 & sincere\_jj & 0 & 8227    &    1845 & analytical\_jj & 0 & 7678 \\
    17 & unshakable\_jj & 0 & 8180    &    1844 & pundit & 0 & 7614 \\
    18 & unreasonable\_jj & 0 & 8174    &    1843 & nonvolatile\_jj & 0 & 7545 \\
    19 & unequivocal\_jj & 0 & 8169    &    1842 & rigorous\_jj & 0 & 7417 \\
    20 & selfless\_jj & 0 & 8016    &    1841 & refined\_jj & 0 & 7403 \\
    21 & emphatic\_jj & 0 & 7944    &    1840 & strident\_jj & 0 & 7342 \\
    22 & orderly\_jj & 0 & 7935    &    1839 & deterministic\_jj & 0 & 7273 \\
    23 & prayerful\_jj & 0 & 7748    &    1838 & combative\_jj & 0 & 7254 \\
    24 & indulgent\_jj & 0 & 7743    &    1837 & magnetic\_jj & 0 & 7228 \\
    25 & do-or-die\_jj & 0 & 7653    &    1836 & stringent\_jj & 0 & 7131 \\
    26 & inhuman\_jj & 0 & 7651    &    1835 & staunch\_jj & 0 & 6991 \\
    27 & unswerving\_jj & 0 & 7518    &    1834 & generalist & 0 & 6846 \\
    28 & immutable\_jj & 0 & 7491    &    1833 & clownish\_jj & 0 & 6805 \\
    29 & fortunate\_jj & 0 & 7449    &    1832 & modifiable\_jj & 0 & 6664 \\
    30 & unflagging\_jj & 0 & 7263    &    1831 & sharp-tongued\_jj & 0 & 6623 \\
    \hline
    \end{tabular}
    \caption{Scores and rankings for most extreme 30 words in component \#14} 
\end{table}
\clearpage
\begin{table}[tbp]
    \begin{tabular}{| rlr@{.}l | rlr@{.}l |}
    \hline
    \textbf{Rank} & \textbf{Word} & \multicolumn{2}{c|}{\textbf{Score}} & \textbf{Rank} & \textbf{Word} & \multicolumn{2}{c|}{\textbf{Score}} \\
    \hline
    1 & convivial\_jj & -1 & 1526    &    1860 & angelic\_jj & 1 & 2483 \\
    2 & rowdy\_jj & -1 & 800    &    1859 & optimistic\_jj & 1 & 112 \\
    3 & leisurely\_jj & -1 & 211    &    1858 & cherubic\_jj & 0 & 9884 \\
    4 & lewd\_jj & 0 & 9453    &    1857 & impassive\_jj & 0 & 9774 \\
    5 & lascivious\_jj & 0 & 9143    &    1856 & quitter & 0 & 9515 \\
    6 & courteous\_jj & 0 & 8724    &    1855 & girlish\_jj & 0 & 9435 \\
    7 & testy\_jj & 0 & 8688    &    1854 & demure\_jj & 0 & 9342 \\
    8 & considerate\_jj & 0 & 8530    &    1853 & lifeless\_jj & 0 & 9260 \\
    9 & low-pressure\_jj & 0 & 8272    &    1852 & beatific\_jj & 0 & 9077 \\
    10 & expeditious\_jj & 0 & 8175    &    1851 & fluttery\_jj & 0 & 8985 \\
    11 & mannerly\_jj & 0 & 8135    &    1850 & pessimistic\_jj & 0 & 8933 \\
    12 & sociable\_jj & 0 & 8049    &    1849 & sexy\_jj & 0 & 8708 \\
    13 & hectic\_jj & 0 & 7840    &    1848 & bright\_jj & 0 & 8534 \\
    14 & gourmet & 0 & 7687    &    1847 & womanly\_jj & 0 & 8496 \\
    15 & unsociable\_jj & 0 & 7674    &    1846 & double-faced\_jj & 0 & 8353 \\
    16 & unhurried\_jj & 0 & 7658    &    1845 & unsure\_jj & 0 & 8327 \\
    17 & raucous\_jj & 0 & 7520    &    1844 & wishful\_jj & 0 & 8282 \\
    18 & rancorous\_jj & 0 & 7464    &    1843 & overactive\_jj & 0 & 8189 \\
    19 & epicurean\_jj & 0 & 7318    &    1842 & vegetative\_jj & 0 & 8114 \\
    20 & circuitous\_jj & 0 & 6984    &    1841 & clingy\_jj & 0 & 8102 \\
    21 & congenial\_jj & 0 & 6931    &    1840 & piercing\_jj & 0 & 7988 \\
    22 & wildcat\_jj & 0 & 6851    &    1839 & acute\_jj & 0 & 7925 \\
    23 & unsportsmanlike\_jj & 0 & 6807    &    1838 & forward-looking\_jj & 0 & 7797 \\
    24 & buzzy\_jj & 0 & 6775    &    1837 & benign\_jj & 0 & 7795 \\
    25 & outdoor\_jj & 0 & 6753    &    1836 & ethereal\_jj & 0 & 7781 \\
    26 & risqué\_jj & 0 & 6563    &    1835 & bullish\_jj & 0 & 7638 \\
    27 & boisterous\_jj & 0 & 6352    &    1834 & voluptuous\_jj & 0 & 7634 \\
    28 & sedate\_jj & 0 & 6300    &    1833 & cautionary\_jj & 0 & 7604 \\
    29 & frenetic\_jj & 0 & 6279    &    1832 & complacent\_jj & 0 & 7571 \\
    30 & hit-or-miss\_jj & 0 & 6276    &    1831 & feminine\_jj & 0 & 7501 \\
    \hline
    \end{tabular}
    \caption{Scores and rankings for most extreme 30 words in component \#15} 
\end{table}
\clearpage

\subsection{Normalized PCA}
\label{app:rankedwordlists:2797words:normalized}
\begin{table}[tbp]
    \begin{tabular}{| rlr@{.}l | rlr@{.}l |}
    \hline
    \textbf{Rank} & \textbf{Word} & \multicolumn{2}{c|}{\textbf{Score}} & \textbf{Rank} & \textbf{Word} & \multicolumn{2}{c|}{\textbf{Score}} \\
    \hline
    1 & stringent\_jj & -11 & 7823    &    1860 & mousy\_jj & 11 & 5354 \\
    2 & beneficial\_jj & -11 & 389    &    1859 & vivacious\_jj & 10 & 8396 \\
    3 & indirect\_jj & -10 & 9367    &    1858 & girlish\_jj & 10 & 160 \\
    4 & contentious\_jj & -10 & 6561    &    1857 & go-getter & 10 & 145 \\
    5 & dependent\_jj & -10 & 4823    &    1856 & self-possessed\_jj & 10 & 37 \\
    6 & prudent\_jj & -10 & 4781    &    1855 & sardonic\_jj & 9 & 7821 \\
    7 & lax\_jj & -10 & 4447    &    1854 & guileless\_jj & 9 & 7706 \\
    8 & reasonable\_jj & -10 & 4223    &    1853 & coquette & 9 & 7511 \\
    9 & corrective\_jj & -10 & 3057    &    1852 & tomboy & 9 & 7110 \\
    10 & autonomous\_jj & -10 & 2312    &    1851 & sassy\_jj & 9 & 6109 \\
    11 & volatile\_jj & -10 & 2089    &    1850 & impish\_jj & 9 & 6044 \\
    12 & discretionary\_jj & -10 & 486    &    1849 & coquettish\_jj & 9 & 5768 \\
    13 & drastic\_jj & -9 & 9012    &    1848 & stoic & 9 & 3944 \\
    14 & systematic\_jj & -9 & 8961    &    1847 & boyish\_jj & 9 & 1788 \\
    15 & affected\_jj & -9 & 8706    &    1846 & debonair\_jj & 9 & 1006 \\
    16 & fraudulent\_jj & -9 & 8602    &    1845 & puckish\_jj & 9 & 904 \\
    17 & unfair\_jj & -9 & 7402    &    1844 & bubbly\_jj & 8 & 9660 \\
    18 & indefinite\_jj & -9 & 7028    &    1843 & flirtatious\_jj & 8 & 9327 \\
    19 & confidential\_jj & -9 & 6747    &    1842 & beatific\_jj & 8 & 9030 \\
    20 & exclusive\_jj & -9 & 6214    &    1841 & misanthropic\_jj & 8 & 8868 \\
    21 & rigorous\_jj & -9 & 5911    &    1840 & suave\_jj & 8 & 7831 \\
    22 & forward-looking\_jj & -9 & 5500    &    1839 & roguish\_jj & 8 & 7774 \\
    23 & stable\_jj & -9 & 5474    &    1838 & witty\_jj & 8 & 7654 \\
    24 & inaccurate\_jj & -9 & 5174    &    1837 & shrewish\_jj & 8 & 7202 \\
    25 & arbitrary\_jj & -9 & 4235    &    1836 & rakish\_jj & 8 & 6029 \\
    26 & critical\_jj & -9 & 4005    &    1835 & brassy\_jj & 8 & 5344 \\
    27 & expeditious\_jj & -9 & 3958    &    1834 & bumpkin & 8 & 5300 \\
    28 & consistent\_jj & -9 & 2636    &    1833 & improviser & 8 & 5215 \\
    29 & prompt\_jj & -9 & 2429    &    1832 & poised\_jj & 8 & 4951 \\
    30 & productive\_jj & -9 & 1692    &    1831 & diffident\_jj & 8 & 4299 \\
    \hline
    \end{tabular}
    \caption{Scores and rankings for most extreme 30 words in component \#1} 
\end{table}
\clearpage
\begin{table}[tbp]
    \begin{tabular}{| rlr@{.}l | rlr@{.}l |}
    \hline
    \textbf{Rank} & \textbf{Word} & \multicolumn{2}{c|}{\textbf{Score}} & \textbf{Rank} & \textbf{Word} & \multicolumn{2}{c|}{\textbf{Score}} \\
    \hline
    1 & warm\_jj & -10 & 6733    &    1860 & defamatory\_jj & 13 & 5043 \\
    2 & elegant\_jj & -10 & 1415    &    1859 & bigoted\_jj & 13 & 2072 \\
    3 & savant & -9 & 6352    &    1858 & cowardly\_jj & 12 & 5562 \\
    4 & vibrant\_jj & -9 & 6014    &    1857 & untruthful\_jj & 12 & 5230 \\
    5 & luxurious\_jj & -9 & 4690    &    1856 & deceitful\_jj & 12 & 3746 \\
    6 & sparkling\_jj & -9 & 1402    &    1855 & slanderous\_jj & 12 & 20 \\
    7 & graceful\_jj & -9 & 182    &    1854 & inhuman\_jj & 11 & 9412 \\
    8 & airy\_jj & -8 & 8338    &    1853 & selfish\_jj & 11 & 8718 \\
    9 & buttery\_jj & -8 & 7818    &    1852 & hypocritical\_jj & 11 & 4287 \\
    10 & polished\_jj & -8 & 7270    &    1851 & irresponsible\_jj & 11 & 3544 \\
    11 & vivacious\_jj & -8 & 5733    &    1850 & dishonest\_jj & 10 & 9012 \\
    12 & sunny\_jj & -8 & 5005    &    1849 & vindictive\_jj & 10 & 8869 \\
    13 & sociable\_jj & -8 & 4743    &    1848 & unethical\_jj & 10 & 7493 \\
    14 & versatile\_jj & -8 & 3683    &    1847 & thoughtless\_jj & 10 & 6769 \\
    15 & sultry\_jj & -8 & 1723    &    1846 & disrespectful\_jj & 10 & 5628 \\
    16 & lively\_jj & -8 & 1397    &    1845 & godless\_jj & 10 & 3948 \\
    17 & intuitive\_jj & -8 & 1306    &    1844 & insensitive\_jj & 10 & 3777 \\
    18 & brisk\_jj & -7 & 8117    &    1843 & mendacious\_jj & 10 & 3737 \\
    19 & rugged\_jj & -7 & 7490    &    1842 & polemic\_jj & 10 & 3373 \\
    20 & sensual\_jj & -7 & 7202    &    1841 & callous\_jj & 10 & 2715 \\
    21 & dainty\_jj & -7 & 6958    &    1840 & narrow-minded\_jj & 10 & 2314 \\
    22 & calm\_jj & -7 & 6667    &    1839 & unfeeling\_jj & 10 & 2228 \\
    23 & bright\_jj & -7 & 6145    &    1838 & intolerant\_jj & 10 & 1870 \\
    24 & musical\_jj & -7 & 5529    &    1837 & inconsiderate\_jj & 10 & 1494 \\
    25 & ethereal\_jj & -7 & 5061    &    1836 & ignorant\_jj & 9 & 8972 \\
    26 & serene\_jj & -7 & 5034    &    1835 & unintelligent\_jj & 9 & 8318 \\
    27 & cultured\_jj & -7 & 4478    &    1834 & unprincipled\_jj & 9 & 7729 \\
    28 & pleasant\_jj & -7 & 3800    &    1833 & inhumane\_jj & 9 & 7466 \\
    29 & tender\_jj & -7 & 3487    &    1832 & blasphemous\_jj & 9 & 6323 \\
    30 & breezy\_jj & -7 & 3254    &    1831 & boorish\_jj & 9 & 5084 \\
    \hline
    \end{tabular}
    \caption{Scores and rankings for most extreme 30 words in component \#2} 
\end{table}
\clearpage
\begin{table}[tbp]
    \begin{tabular}{| rlr@{.}l | rlr@{.}l |}
    \hline
    \textbf{Rank} & \textbf{Word} & \multicolumn{2}{c|}{\textbf{Score}} & \textbf{Rank} & \textbf{Word} & \multicolumn{2}{c|}{\textbf{Score}} \\
    \hline
    1 & considerate\_jj & -18 & 174    &    1860 & gingery\_jj & 9 & 4861 \\
    2 & sociable\_jj & -15 & 4380    &    1859 & comforter & 9 & 4194 \\
    3 & courteous\_jj & -14 & 3290    &    1858 & pixy & 9 & 1549 \\
    4 & trustworthy\_jj & -13 & 4448    &    1857 & butterfly & 9 & 490 \\
    5 & articulate\_jj & -13 & 1559    &    1856 & broiler & 8 & 4296 \\
    6 & open-minded\_jj & -12 & 2807    &    1855 & soft-shelled\_jj & 8 & 3148 \\
    7 & approachable\_jj & -11 & 7430    &    1854 & bendable\_jj & 8 & 2963 \\
    8 & kind\_jj & -11 & 6195    &    1853 & boneless\_jj & 8 & 723 \\
    9 & respectful\_jj & -11 & 5974    &    1852 & low-pressure\_jj & 7 & 9602 \\
    10 & thoughtful\_jj & -11 & 2645    &    1851 & baneful\_jj & 7 & 5941 \\
    11 & self-confident\_jj & -11 & 522    &    1850 & vinegary\_jj & 7 & 4943 \\
    12 & easy-going\_jj & -10 & 6411    &    1849 & wildcat\_jj & 7 & 3797 \\
    13 & level-headed\_jj & -10 & 5379    &    1848 & yellow\_jj & 7 & 3225 \\
    14 & honest\_jj & -10 & 4062    &    1847 & clam & 7 & 2497 \\
    15 & talkative\_jj & -10 & 2432    &    1846 & hard-shell\_jj & 7 & 2246 \\
    16 & down-to-earth\_jj & -10 & 2285    &    1845 & aplastic\_jj & 7 & 1946 \\
    17 & circumspect\_jj & -10 & 386    &    1844 & acrid\_jj & 6 & 9548 \\
    18 & sincere\_jj & -9 & 6918    &    1843 & croaking & 6 & 8872 \\
    19 & intelligent\_jj & -9 & 6606    &    1842 & gourmand & 6 & 8685 \\
    20 & perceptive\_jj & -9 & 6492    &    1841 & driftless\_jj & 6 & 8636 \\
    21 & cordial\_jj & -9 & 5777    &    1840 & gooey\_jj & 6 & 7841 \\
    22 & accommodating\_jj & -9 & 4771    &    1839 & bristly\_jj & 6 & 7038 \\
    23 & pragmatic\_jj & -9 & 4561    &    1838 & unmanly\_jj & 6 & 7020 \\
    24 & opinionated\_jj & -9 & 2531    &    1837 & high-hat & 6 & 6974 \\
    25 & deferential\_jj & -9 & 1686    &    1836 & magnetic\_jj & 6 & 6677 \\
    26 & adaptable\_jj & -9 & 953    &    1835 & dare-devil & 6 & 6558 \\
    27 & gregarious\_jj & -9 & 572    &    1834 & fossil & 6 & 6114 \\
    28 & fair-minded\_jj & -9 & 492    &    1833 & plastic\_jj & 6 & 5926 \\
    29 & affectionate\_jj & -9 & 230    &    1832 & cry-baby & 6 & 4821 \\
    30 & responsive\_jj & -9 & 122    &    1831 & ductile\_jj & 6 & 3997 \\
    \hline
    \end{tabular}
    \caption{Scores and rankings for most extreme 30 words in component \#3} 
\end{table}
\clearpage
\begin{table}[tbp]
    \begin{tabular}{| rlr@{.}l | rlr@{.}l |}
    \hline
    \textbf{Rank} & \textbf{Word} & \multicolumn{2}{c|}{\textbf{Score}} & \textbf{Rank} & \textbf{Word} & \multicolumn{2}{c|}{\textbf{Score}} \\
    \hline
    1 & objector & -9 & 3294    &    1860 & unvarying\_jj & 9 & 7292 \\
    2 & loyal\_jj & -9 & 2699    &    1859 & limpid\_jj & 9 & 647 \\
    3 & staunch\_jj & -9 & 1096    &    1858 & defamatory\_jj & 8 & 9127 \\
    4 & devout\_jj & -8 & 9843    &    1857 & dissonant\_jj & 8 & 8557 \\
    5 & avid\_jj & -8 & 5698    &    1856 & imprecise\_jj & 8 & 6403 \\
    6 & sociable\_jj & -8 & 5595    &    1855 & contrived\_jj & 8 & 6268 \\
    7 & alcoholic & -8 & 2759    &    1854 & oblique\_jj & 8 & 5895 \\
    8 & stalwart & -8 & 2729    &    1853 & didactic\_jj & 8 & 2826 \\
    9 & kind-hearted\_jj & -8 & 2206    &    1852 & poetic\_jj & 8 & 2295 \\
    10 & cheater & -8 & 266    &    1851 & stilted\_jj & 8 & 1167 \\
    11 & stay-at-home\_jj & -7 & 9674    &    1850 & meditative\_jj & 8 & 281 \\
    12 & disciplinarian & -7 & 8650    &    1849 & lyrical\_jj & 7 & 9954 \\
    13 & gambler & -7 & 8217    &    1848 & discursive\_jj & 7 & 9785 \\
    14 & angler & -7 & 7885    &    1847 & astringent\_jj & 7 & 8847 \\
    15 & visionary & -7 & 7578    &    1846 & abstruse\_jj & 7 & 7490 \\
    16 & samaritan & -7 & 5789    &    1845 & melodramatic\_jj & 7 & 6193 \\
    17 & loving\_jj & -7 & 5323    &    1844 & deterministic\_jj & 7 & 4560 \\
    18 & teetotaler & -7 & 5154    &    1843 & subjective\_jj & 7 & 3013 \\
    19 & intrepid\_jj & -7 & 3661    &    1842 & tasteless\_jj & 7 & 2943 \\
    20 & gregarious\_jj & -7 & 3358    &    1841 & abstract\_jj & 7 & 2334 \\
    21 & fan & -7 & 2471    &    1840 & mawkish\_jj & 7 & 1959 \\
    22 & go-getter & -7 & 1238    &    1839 & pithy\_jj & 7 & 1525 \\
    23 & brahmin & -6 & 8533    &    1838 & morbid\_jj & 7 & 1104 \\
    24 & has-been & -6 & 8523    &    1837 & nonsensical\_jj & 7 & 1091 \\
    25 & outgoing\_jj & -6 & 8476    &    1836 & ethereal\_jj & 6 & 9967 \\
    26 & brat & -6 & 8222    &    1835 & discordant\_jj & 6 & 8369 \\
    27 & dissident & -6 & 8045    &    1834 & intricate\_jj & 6 & 6629 \\
    28 & proud\_jj & -6 & 6968    &    1833 & sensuous\_jj & 6 & 5929 \\
    29 & educated\_jj & -6 & 6139    &    1832 & economical\_jj & 6 & 5361 \\
    30 & lucky\_jj & -6 & 5906    &    1831 & downright\_jj & 6 & 5344 \\
    \hline
    \end{tabular}
    \caption{Scores and rankings for most extreme 30 words in component \#4} 
\end{table}
\clearpage
\begin{table}[tbp]
    \begin{tabular}{| rlr@{.}l | rlr@{.}l |}
    \hline
    \textbf{Rank} & \textbf{Word} & \multicolumn{2}{c|}{\textbf{Score}} & \textbf{Rank} & \textbf{Word} & \multicolumn{2}{c|}{\textbf{Score}} \\
    \hline
    1 & deterministic\_jj & -9 & 7116    &    1860 & testy\_jj & 11 & 3643 \\
    2 & self-reliant\_jj & -9 & 352    &    1859 & lewd\_jj & 11 & 1062 \\
    3 & adaptable\_jj & -8 & 8656    &    1858 & conciliatory\_jj & 9 & 2793 \\
    4 & theistic\_jj & -8 & 7633    &    1857 & disorderly\_jj & 9 & 417 \\
    5 & mutable\_jj & -8 & 4690    &    1856 & unsportsmanlike\_jj & 9 & 177 \\
    6 & individualistic\_jj & -8 & 2411    &    1855 & rowdy\_jj & 8 & 9171 \\
    7 & unchangeable\_jj & -8 & 1631    &    1854 & risqué\_jj & 8 & 8654 \\
    8 & closed-minded\_jj & -8 & 981    &    1853 & defiant\_jj & 8 & 7730 \\
    9 & educated\_jj & -7 & 8096    &    1852 & derisive\_jj & 8 & 5321 \\
    10 & conformist\_jj & -7 & 7470    &    1851 & congratulatory\_jj & 8 & 4535 \\
    11 & open-minded\_jj & -7 & 6925    &    1850 & acrimonious\_jj & 8 & 3997 \\
    12 & modifiable\_jj & -7 & 4205    &    1849 & vitriolic\_jj & 8 & 2946 \\
    13 & materialistic\_jj & -7 & 3409    &    1848 & caustic\_jj & 8 & 2924 \\
    14 & self-sufficient\_jj & -7 & 3402    &    1847 & stormy\_jj & 8 & 1132 \\
    15 & equitable\_jj & -7 & 2210    &    1846 & thunderous\_jj & 8 & 361 \\
    16 & unchanging\_jj & -7 & 191    &    1845 & rancorous\_jj & 8 & 307 \\
    17 & enlightened\_jj & -6 & 8079    &    1844 & loud\_jj & 8 & 55 \\
    18 & perspicuous\_jj & -6 & 7858    &    1843 & stern\_jj & 7 & 9411 \\
    19 & intuitive\_jj & -6 & 7408    &    1842 & raucous\_jj & 7 & 6743 \\
    20 & irreligious\_jj & -6 & 7386    &    1841 & lascivious\_jj & 7 & 5156 \\
    21 & utopian\_jj & -6 & 7250    &    1840 & defamatory\_jj & 7 & 4787 \\
    22 & altruistic\_jj & -6 & 7156    &    1839 & verbal\_jj & 7 & 4658 \\
    23 & intelligent\_jj & -6 & 6360    &    1838 & torrid\_jj & 7 & 4052 \\
    24 & imitative\_jj & -6 & 5814    &    1837 & bitter\_jj & 7 & 4041 \\
    25 & immutable\_jj & -6 & 5708    &    1836 & frosty\_jj & 7 & 3967 \\
    26 & mechanistic\_jj & -6 & 5377    &    1835 & lukewarm\_jj & 7 & 3194 \\
    27 & ethnocentric\_jj & -6 & 5376    &    1834 & obscene\_jj & 7 & 2845 \\
    28 & considerate\_jj & -6 & 4476    &    1833 & derogatory\_jj & 7 & 1511 \\
    29 & empathic\_jj & -6 & 2649    &    1832 & curt\_jj & 7 & 665 \\
    30 & celibate & -6 & 2527    &    1831 & foul-mouthed\_jj & 6 & 9674 \\
    \hline
    \end{tabular}
    \caption{Scores and rankings for most extreme 30 words in component \#5} 
\end{table}
\clearpage
\begin{table}[tbp]
    \begin{tabular}{| rlr@{.}l | rlr@{.}l |}
    \hline
    \textbf{Rank} & \textbf{Word} & \multicolumn{2}{c|}{\textbf{Score}} & \textbf{Rank} & \textbf{Word} & \multicolumn{2}{c|}{\textbf{Score}} \\
    \hline
    1 & unswerving\_jj & -10 & 973    &    1860 & sociable\_jj & 10 & 4757 \\
    2 & unbending\_jj & -10 & 158    &    1859 & picky\_jj & 10 & 4029 \\
    3 & unfailing\_jj & -9 & 7423    &    1858 & irritable\_jj & 9 & 6358 \\
    4 & unflagging\_jj & -9 & 5061    &    1857 & lazy\_jj & 9 & 2608 \\
    5 & unshakable\_jj & -9 & 3913    &    1856 & inconsiderate\_jj & 8 & 6344 \\
    6 & ultraconservative\_jj & -9 & 3465    &    1855 & lewd\_jj & 8 & 2708 \\
    7 & firebrand & -8 & 8826    &    1854 & abusive\_jj & 8 & 2036 \\
    8 & steadfast\_jj & -8 & 7961    &    1853 & fussy\_jj & 8 & 1156 \\
    9 & unflinching\_jj & -8 & 3822    &    1852 & clingy\_jj & 8 & 833 \\
    10 & uncompromising\_jj & -8 & 3471    &    1851 & gooey\_jj & 7 & 6692 \\
    11 & untiring\_jj & -8 & 2820    &    1850 & sexy\_jj & 7 & 6320 \\
    12 & unyielding\_jj & -8 & 2711    &    1849 & rude\_jj & 7 & 5910 \\
    13 & trenchant\_jj & -8 & 1214    &    1848 & frisky\_jj & 7 & 3886 \\
    14 & oratorical\_jj & -8 & 192    &    1847 & choosy\_jj & 7 & 3174 \\
    15 & unquestioning\_jj & -7 & 9994    &    1846 & talkative\_jj & 7 & 3044 \\
    16 & secular\_jj & -7 & 8617    &    1845 & sedentary\_jj & 7 & 2803 \\
    17 & imperturbable\_jj & -7 & 7842    &    1844 & defamatory\_jj & 7 & 1163 \\
    18 & dissident & -7 & 7212    &    1843 & addicted\_jj & 7 & 785 \\
    19 & staunch\_jj & -7 & 6880    &    1842 & lascivious\_jj & 7 & 497 \\
    20 & fervent\_jj & -7 & 2602    &    1841 & oily\_jj & 7 & 320 \\
    21 & magisterial\_jj & -7 & 1406    &    1840 & bubbly\_jj & 6 & 9951 \\
    22 & unerring\_jj & -7 & 1001    &    1839 & distractible\_jj & 6 & 8018 \\
    23 & doctrinaire\_jj & -7 & 252    &    1838 & risqué\_jj & 6 & 7424 \\
    24 & outspoken\_jj & -6 & 9769    &    1837 & autistic\_jj & 6 & 7210 \\
    25 & autocratic\_jj & -6 & 9762    &    1836 & disorderly\_jj & 6 & 6883 \\
    26 & scholarly\_jj & -6 & 7639    &    1835 & considerate\_jj & 6 & 6859 \\
    27 & militant & -6 & 7108    &    1834 & forgetful\_jj & 6 & 5899 \\
    28 & dogged\_jj & -6 & 5315    &    1833 & kind\_jj & 6 & 5662 \\
    29 & religious\_jj & -6 & 4888    &    1832 & bitch & 6 & 4654 \\
    30 & tireless\_jj & -6 & 4394    &    1831 & lucky\_jj & 6 & 4314 \\
    \hline
    \end{tabular}
    \caption{Scores and rankings for most extreme 30 words in component \#6} 
\end{table}
\clearpage
\begin{table}[tbp]
    \begin{tabular}{| rlr@{.}l | rlr@{.}l |}
    \hline
    \textbf{Rank} & \textbf{Word} & \multicolumn{2}{c|}{\textbf{Score}} & \textbf{Rank} & \textbf{Word} & \multicolumn{2}{c|}{\textbf{Score}} \\
    \hline
    1 & assertive\_jj & -8 & 5864    &    1860 & savant & 15 & 9616 \\
    2 & inhospitable\_jj & -8 & 4546    &    1859 & lewd\_jj & 10 & 2022 \\
    3 & distrustful\_jj & -8 & 2723    &    1858 & defamatory\_jj & 10 & 525 \\
    4 & stagnant\_jj & -8 & 1937    &    1857 & lascivious\_jj & 9 & 4354 \\
    5 & impervious\_jj & -7 & 6995    &    1856 & slanderous\_jj & 7 & 9077 \\
    6 & conformist\_jj & -7 & 6450    &    1855 & unbiased\_jj & 7 & 6774 \\
    7 & mistrustful\_jj & -7 & 3258    &    1854 & quitter & 7 & 5588 \\
    8 & autocratic\_jj & -7 & 2505    &    1853 & musical\_jj & 7 & 3767 \\
    9 & fractious\_jj & -7 & 2211    &    1852 & teachable\_jj & 7 & 3156 \\
    10 & apathetic\_jj & -7 & 1988    &    1851 & liar & 7 & 3011 \\
    11 & blase\_jj & -7 & 1185    &    1850 & cry-baby & 7 & 2952 \\
    12 & docile\_jj & -6 & 9998    &    1849 & obscene\_jj & 7 & 2526 \\
    13 & sanguine\_jj & -6 & 9070    &    1848 & sincere\_jj & 7 & 2363 \\
    14 & reliant\_jj & -6 & 7686    &    1847 & comedian & 7 & 2060 \\
    15 & laggard\_jj & -6 & 7295    &    1846 & concise\_jj & 7 & 1460 \\
    16 & spendthrift\_jj & -6 & 7257    &    1845 & autistic\_jj & 7 & 1317 \\
    17 & intransigent\_jj & -6 & 4267    &    1844 & cognitive\_jj & 7 & 1128 \\
    18 & wary\_jj & -6 & 3603    &    1843 & thorough\_jj & 6 & 8162 \\
    19 & rapacious\_jj & -6 & 2916    &    1842 & objective\_jj & 6 & 7344 \\
    20 & hospitable\_jj & -6 & 2432    &    1841 & malicious\_jj & 6 & 7333 \\
    21 & fickle\_jj & -6 & 2321    &    1840 & unreserved\_jj & 6 & 7048 \\
    22 & clannish\_jj & -6 & 2031    &    1839 & insightful\_jj & 6 & 5880 \\
    23 & pliable\_jj & -6 & 1944    &    1838 & considerate\_jj & 6 & 5376 \\
    24 & inclement\_jj & -6 & 1470    &    1837 & retrospective\_jj & 6 & 5320 \\
    25 & treacherous\_jj & -6 & 1400    &    1836 & derogatory\_jj & 6 & 5074 \\
    26 & sluggish\_jj & -6 & 1065    &    1835 & objector & 6 & 4843 \\
    27 & icy\_jj & -6 & 1047    &    1834 & disorderly\_jj & 6 & 4164 \\
    28 & clement\_jj & -6 & 859    &    1833 & impartial\_jj & 6 & 3395 \\
    29 & choosy\_jj & -6 & 616    &    1832 & fraudulent\_jj & 6 & 3382 \\
    30 & frosty\_jj & -6 & 67    &    1831 & negligent\_jj & 6 & 3036 \\
    \hline
    \end{tabular}
    \caption{Scores and rankings for most extreme 30 words in component \#7} 
\end{table}
\clearpage
\begin{table}[tbp]
    \begin{tabular}{| rlr@{.}l | rlr@{.}l |}
    \hline
    \textbf{Rank} & \textbf{Word} & \multicolumn{2}{c|}{\textbf{Score}} & \textbf{Rank} & \textbf{Word} & \multicolumn{2}{c|}{\textbf{Score}} \\
    \hline
    1 & inhuman\_jj & -11 & 8875    &    1860 & spendthrift & 10 & 9301 \\
    2 & savant & -10 & 6709    &    1859 & remiss\_jj & 9 & 9351 \\
    3 & lascivious\_jj & -10 & 4117    &    1858 & quitter & 8 & 4778 \\
    4 & inhumane\_jj & -9 & 1652    &    1857 & tightwad & 7 & 3571 \\
    5 & disorderly\_jj & -9 & 655    &    1856 & squeamish\_jj & 7 & 3503 \\
    6 & forcible\_jj & -9 & 258    &    1855 & lukewarm\_jj & 7 & 3030 \\
    7 & wanton\_jj & -8 & 4001    &    1854 & wishy-washy\_jj & 7 & 2307 \\
    8 & impulsive\_jj & -8 & 3547    &    1853 & facetious\_jj & 7 & 1310 \\
    9 & obstructive\_jj & -8 & 3039    &    1852 & sanguine\_jj & 7 & 1153 \\
    10 & autistic\_jj & -8 & 1644    &    1851 & coy\_jj & 7 & 885 \\
    11 & abusive\_jj & -8 & 1095    &    1850 & pessimistic\_jj & 7 & 602 \\
    12 & libidinous\_jj & -7 & 8318    &    1849 & circumspect\_jj & 6 & 9242 \\
    13 & overactive\_jj & -7 & 6964    &    1848 & cautionary\_jj & 6 & 9147 \\
    14 & lewd\_jj & -7 & 4852    &    1847 & pessimist & 6 & 6238 \\
    15 & unstable\_jj & -7 & 3135    &    1846 & do-nothing & 6 & 5887 \\
    16 & sadistic\_jj & -7 & 2911    &    1845 & picky\_jj & 6 & 5781 \\
    17 & antisocial\_jj & -7 & 2637    &    1844 & choosy\_jj & 6 & 5073 \\
    18 & uncontrolled\_jj & -6 & 6928    &    1843 & preachy\_jj & 6 & 4586 \\
    19 & manipulative\_jj & -6 & 4762    &    1842 & concise\_jj & 6 & 4503 \\
    20 & uncooperative\_jj & -6 & 4605    &    1841 & sophistic\_jj & 6 & 4294 \\
    21 & acute\_jj & -6 & 4232    &    1840 & churlish\_jj & 6 & 3181 \\
    22 & unpredictable\_jj & -6 & 3968    &    1839 & cogent\_jj & 6 & 1962 \\
    23 & expressive\_jj & -6 & 2961    &    1838 & wishful\_jj & 6 & 1953 \\
    24 & unruly\_jj & -6 & 2724    &    1837 & gun-shy\_jj & 6 & 1298 \\
    25 & illiterate\_jj & -6 & 2573    &    1836 & skeptical\_jj & 5 & 8442 \\
    26 & violent\_jj & -6 & 2359    &    1835 & unread\_jj & 5 & 8184 \\
    27 & brutal\_jj & -6 & 2131    &    1834 & highfalutin\_jj & 5 & 8149 \\
    28 & irritable\_jj & -6 & 1536    &    1833 & morbid\_jj & 5 & 8088 \\
    29 & aplastic\_jj & -6 & 1439    &    1832 & carper & 5 & 7800 \\
    30 & cerebral\_jj & -6 & 699    &    1831 & cautious\_jj & 5 & 7735 \\
    \hline
    \end{tabular}
    \caption{Scores and rankings for most extreme 30 words in component \#8} 
\end{table}
\clearpage
\begin{table}[tbp]
    \begin{tabular}{| rlr@{.}l | rlr@{.}l |}
    \hline
    \textbf{Rank} & \textbf{Word} & \multicolumn{2}{c|}{\textbf{Score}} & \textbf{Rank} & \textbf{Word} & \multicolumn{2}{c|}{\textbf{Score}} \\
    \hline
    1 & vegetative\_jj & -13 & 4862    &    1860 & chic\_jj & 8 & 9421 \\
    2 & irritable\_jj & -12 & 6442    &    1859 & cowardly\_jj & 8 & 1620 \\
    3 & abstinent\_jj & -12 & 5720    &    1858 & elegant\_jj & 7 & 6828 \\
    4 & aplastic\_jj & -12 & 4113    &    1857 & cunning\_jj & 6 & 9338 \\
    5 & overactive\_jj & -11 & 7090    &    1856 & flashy\_jj & 6 & 8375 \\
    6 & cognitive\_jj & -10 & 5124    &    1855 & downright\_jj & 6 & 4972 \\
    7 & autistic\_jj & -10 & 1037    &    1854 & risque\_jj & 6 & 4480 \\
    8 & refractory\_jj & -10 & 865    &    1853 & brazen\_jj & 6 & 2036 \\
    9 & malignant\_jj & -10 & 626    &    1852 & barbarous\_jj & 6 & 1988 \\
    10 & obstructive\_jj & -9 & 9422    &    1851 & luxurious\_jj & 6 & 1093 \\
    11 & affective\_jj & -9 & 8284    &    1850 & sexy\_jj & 6 & 772 \\
    12 & modifiable\_jj & -7 & 9278    &    1849 & gourmet & 6 & 507 \\
    13 & cerebral\_jj & -7 & 4652    &    1848 & wasteful\_jj & 6 & 389 \\
    14 & hypersensitive\_jj & -7 & 2202    &    1847 & extravagant\_jj & 5 & 9477 \\
    15 & testy\_jj & -7 & 1719    &    1846 & undemocratic\_jj & 5 & 9013 \\
    16 & unemotional\_jj & -7 & 352    &    1845 & eclectic\_jj & 5 & 6885 \\
    17 & maternal\_jj & -7 & 132    &    1844 & homespun\_jj & 5 & 6682 \\
    18 & acute\_jj & -6 & 9496    &    1843 & clever\_jj & 5 & 6384 \\
    19 & mild\_jj & -6 & 5135    &    1842 & ruthless\_jj & 5 & 5820 \\
    20 & extroverted\_jj & -6 & 5046    &    1841 & devious\_jj & 5 & 5788 \\
    21 & objector & -6 & 4853    &    1840 & cosmopolitan\_jj & 5 & 5595 \\
    22 & distractible\_jj & -6 & 2877    &    1839 & fanciful\_jj & 5 & 5528 \\
    23 & questioning\_jj & -6 & 2542    &    1838 & ostentatious\_jj & 5 & 4882 \\
    24 & nonconforming\_jj & -6 & 1201    &    1837 & greedy\_jj & 5 & 4848 \\
    25 & unfaithful\_jj & -6 & 897    &    1836 & quirky\_jj & 5 & 4594 \\
    26 & impulsive\_jj & -6 & 512    &    1835 & witch & 5 & 4199 \\
    27 & obsessive & -6 & 298    &    1834 & clown & 5 & 3956 \\
    28 & unresponsive\_jj & -5 & 9544    &    1833 & lavish\_jj & 5 & 3846 \\
    29 & possessive\_jj & -5 & 9249    &    1832 & sophisticated\_jj & 5 & 3683 \\
    30 & compulsive & -5 & 8352    &    1831 & gullible\_jj & 5 & 2533 \\
    \hline
    \end{tabular}
    \caption{Scores and rankings for most extreme 30 words in component \#9} 
\end{table}
\clearpage
\begin{table}[tbp]
    \begin{tabular}{| rlr@{.}l | rlr@{.}l |}
    \hline
    \textbf{Rank} & \textbf{Word} & \multicolumn{2}{c|}{\textbf{Score}} & \textbf{Rank} & \textbf{Word} & \multicolumn{2}{c|}{\textbf{Score}} \\
    \hline
    1 & lascivious\_jj & -11 & 1368    &    1860 & savant & 8 & 2324 \\
    2 & secular\_jj & -10 & 9045    &    1859 & inexact\_jj & 8 & 867 \\
    3 & religious\_jj & -9 & 7914    &    1858 & unguarded\_jj & 7 & 5758 \\
    4 & celibate & -9 & 3433    &    1857 & mathematical\_jj & 6 & 9072 \\
    5 & forcible\_jj & -9 & 2773    &    1856 & accomplished\_jj & 6 & 8257 \\
    6 & devout\_jj & -9 & 999    &    1855 & indestructible\_jj & 6 & 7491 \\
    7 & defamatory\_jj & -8 & 7470    &    1854 & adroit\_jj & 6 & 5762 \\
    8 & peaceful\_jj & -8 & 3764    &    1853 & balky\_jj & 6 & 5583 \\
    9 & blasphemous\_jj & -8 & 1950    &    1852 & imperturbable\_jj & 6 & 4133 \\
    10 & nonreligious\_jj & -8 & 749    &    1851 & egghead & 6 & 2937 \\
    11 & inhuman\_jj & -7 & 9155    &    1850 & unreliable\_jj & 6 & 1893 \\
    12 & prayerful\_jj & -7 & 9149    &    1849 & absent-minded\_jj & 5 & 9856 \\
    13 & buttery\_jj & -7 & 7095    &    1848 & overactive\_jj & 5 & 9277 \\
    14 & solemn\_jj & -7 & 6461    &    1847 & inexperienced\_jj & 5 & 8666 \\
    15 & heretical\_jj & -7 & 4700    &    1846 & astute\_jj & 5 & 5944 \\
    16 & warm\_jj & -7 & 4223    &    1845 & unerring\_jj & 5 & 5832 \\
    17 & respectful\_jj & -7 & 3814    &    1844 & canny\_jj & 5 & 5294 \\
    18 & lewd\_jj & -7 & 3216    &    1843 & overconfident\_jj & 5 & 4709 \\
    19 & loving\_jj & -7 & 2644    &    1842 & assured\_jj & 5 & 4382 \\
    20 & hearty\_jj & -7 & 1946    &    1841 & cunning & 5 & 4300 \\
    21 & pious\_jj & -7 & 1270    &    1840 & audacious\_jj & 5 & 4201 \\
    22 & ultraconservative\_jj & -7 & 1192    &    1839 & irascible\_jj & 5 & 3328 \\
    23 & tolerant\_jj & -7 & 1018    &    1838 & avid\_jj & 5 & 2726 \\
    24 & sincere\_jj & -7 & 531    &    1837 & upstart\_jj & 5 & 2246 \\
    25 & monastic\_jj & -7 & 324    &    1836 & improviser & 5 & 2208 \\
    26 & peppery\_jj & -6 & 8957    &    1835 & introvert & 5 & 1516 \\
    27 & satanic\_jj & -6 & 8775    &    1834 & iffy\_jj & 5 & 932 \\
    28 & cordial\_jj & -6 & 7312    &    1833 & opportunist & 5 & 836 \\
    29 & godless\_jj & -6 & 6229    &    1832 & accurate\_jj & 5 & 548 \\
    30 & civilized\_jj & -6 & 6039    &    1831 & abrupt\_jj & 5 & 244 \\
    \hline
    \end{tabular}
    \caption{Scores and rankings for most extreme 30 words in component \#10} 
\end{table}
\clearpage
\begin{table}[tbp]
    \begin{tabular}{| rlr@{.}l | rlr@{.}l |}
    \hline
    \textbf{Rank} & \textbf{Word} & \multicolumn{2}{c|}{\textbf{Score}} & \textbf{Rank} & \textbf{Word} & \multicolumn{2}{c|}{\textbf{Score}} \\
    \hline
    1 & vinegary\_jj & -9 & 9537    &    1860 & risqué\_jj & 10 & 5306 \\
    2 & gingery\_jj & -7 & 8145    &    1859 & ultraconservative\_jj & 8 & 7202 \\
    3 & peppery\_jj & -7 & 7975    &    1858 & risque\_jj & 8 & 5622 \\
    4 & buttery\_jj & -7 & 7265    &    1857 & savant & 8 & 3579 \\
    5 & boneless\_jj & -7 & 5383    &    1856 & autistic\_jj & 7 & 6499 \\
    6 & broiler & -7 & 4799    &    1855 & exclusive\_jj & 7 & 4712 \\
    7 & unctuous\_jj & -7 & 3408    &    1854 & raunchy\_jj & 7 & 4213 \\
    8 & oily\_jj & -7 & 1951    &    1853 & eclectic\_jj & 7 & 1314 \\
    9 & expeditious\_jj & -6 & 8764    &    1852 & celibate & 7 & 616 \\
    10 & methodical\_jj & -6 & 8032    &    1851 & explicit\_jj & 6 & 5465 \\
    11 & prompt\_jj & -6 & 2558    &    1850 & religious\_jj & 6 & 5314 \\
    12 & speedy\_jj & -6 & 105    &    1849 & impressionable\_jj & 6 & 4488 \\
    13 & deliberate\_jj & -5 & 9566    &    1848 & morbid\_jj & 6 & 4374 \\
    14 & comforter & -5 & 8794    &    1847 & chaste\_jj & 6 & 4106 \\
    15 & backhanded\_jj & -5 & 6539    &    1846 & musical\_jj & 6 & 3949 \\
    16 & merciless\_jj & -5 & 6491    &    1845 & prudish\_jj & 6 & 3643 \\
    17 & cunning & -5 & 5650    &    1844 & animated\_jj & 6 & 3382 \\
    18 & doer & -5 & 3805    &    1843 & highbrow\_jj & 6 & 3145 \\
    19 & deliberative\_jj & -5 & 3156    &    1842 & lewd\_jj & 6 & 1988 \\
    20 & gooey\_jj & -5 & 2849    &    1841 & observant\_jj & 6 & 1958 \\
    21 & courageous\_jj & -5 & 2515    &    1840 & secular\_jj & 6 & 1616 \\
    22 & tender\_jj & -5 & 2386    &    1839 & intimate\_jj & 6 & 1236 \\
    23 & calculating\_jj & -5 & 412    &    1838 & immodest\_jj & 6 & 915 \\
    24 & transparent\_jj & -5 & 210    &    1837 & individualistic\_jj & 6 & 878 \\
    25 & prudent\_jj & -5 & 201    &    1836 & die-hard\_jj & 6 & 522 \\
    26 & pliant\_jj & -4 & 9916    &    1835 & acrimonious\_jj & 5 & 8815 \\
    27 & pungent\_jj & -4 & 9905    &    1834 & addicted\_jj & 5 & 8211 \\
    28 & proven\_jj & -4 & 9879    &    1833 & salacious\_jj & 5 & 7768 \\
    29 & bloody-minded\_jj & -4 & 9550    &    1832 & cosmopolitan\_jj & 5 & 6337 \\
    30 & thorough\_jj & -4 & 9349    &    1831 & blasphemous\_jj & 5 & 6295 \\
    \hline
    \end{tabular}
    \caption{Scores and rankings for most extreme 30 words in component \#11} 
\end{table}
\clearpage
\begin{table}[tbp]
    \begin{tabular}{| rlr@{.}l | rlr@{.}l |}
    \hline
    \textbf{Rank} & \textbf{Word} & \multicolumn{2}{c|}{\textbf{Score}} & \textbf{Rank} & \textbf{Word} & \multicolumn{2}{c|}{\textbf{Score}} \\
    \hline
    1 & outspoken\_jj & -9 & 6569    &    1860 & cheater & 8 & 2790 \\
    2 & defamatory\_jj & -9 & 1299    &    1859 & sadistic\_jj & 7 & 3054 \\
    3 & ultraconservative\_jj & -9 & 503    &    1858 & cowardly\_jj & 7 & 2123 \\
    4 & vinegary\_jj & -8 & 5387    &    1857 & vegetative\_jj & 7 & 1499 \\
    5 & uncooperative\_jj & -8 & 1627    &    1856 & cold-blooded\_jj & 7 & 1232 \\
    6 & peppery\_jj & -8 & 210    &    1855 & macabre\_jj & 6 & 8655 \\
    7 & unreserved\_jj & -7 & 9998    &    1854 & barbarous\_jj & 6 & 6662 \\
    8 & exclusive\_jj & -7 & 8613    &    1853 & morbid\_jj & 6 & 4451 \\
    9 & broiler & -7 & 7357    &    1852 & torrid\_jj & 6 & 2103 \\
    10 & gingery\_jj & -7 & 7166    &    1851 & mechanistic\_jj & 6 & 1384 \\
    11 & buttery\_jj & -7 & 6928    &    1850 & deterministic\_jj & 6 & 380 \\
    12 & unsportsmanlike\_jj & -7 & 6574    &    1849 & dispiriting\_jj & 6 & 10 \\
    13 & impartial\_jj & -7 & 508    &    1848 & vicious\_jj & 5 & 9624 \\
    14 & oily\_jj & -7 & 401    &    1847 & bloodthirsty\_jj & 5 & 9308 \\
    15 & obscene\_jj & -6 & 9997    &    1846 & murderous\_jj & 5 & 9175 \\
    16 & avid\_jj & -6 & 9117    &    1845 & heroic\_jj & 5 & 8996 \\
    17 & unguarded\_jj & -6 & 8057    &    1844 & bleak\_jj & 5 & 8947 \\
    18 & earthy\_jj & -6 & 6755    &    1843 & valiant\_jj & 5 & 8708 \\
    19 & ham & -6 & 6070    &    1842 & brutal\_jj & 5 & 6083 \\
    20 & eclectic\_jj & -6 & 6055    &    1841 & vengeful\_jj & 5 & 3989 \\
    21 & sugary\_jj & -6 & 3015    &    1840 & pitiless\_jj & 5 & 3644 \\
    22 & intemperate\_jj & -6 & 707    &    1839 & lethargic\_jj & 5 & 3408 \\
    23 & offhand\_jj & -6 & 372    &    1838 & rebellious\_jj & 5 & 2559 \\
    24 & unbiased\_jj & -5 & 9101    &    1837 & fiendish\_jj & 5 & 2464 \\
    25 & unctuous\_jj & -5 & 8210    &    1836 & precipitous\_jj & 5 & 2143 \\
    26 & insincere\_jj & -5 & 6776    &    1835 & relentless\_jj & 5 & 1921 \\
    27 & astringent\_jj & -5 & 6546    &    1834 & cruel\_jj & 5 & 1801 \\
    28 & affirmative\_jj & -5 & 6288    &    1833 & mystical\_jj & 5 & 1364 \\
    29 & independent\_jj & -5 & 5896    &    1832 & spasmodic\_jj & 5 & 1218 \\
    30 & unrefined\_jj & -5 & 5657    &    1831 & torturous\_jj & 5 & 1199 \\
    \hline
    \end{tabular}
    \caption{Scores and rankings for most extreme 30 words in component \#12} 
\end{table}
\clearpage
\begin{table}[tbp]
    \begin{tabular}{| rlr@{.}l | rlr@{.}l |}
    \hline
    \textbf{Rank} & \textbf{Word} & \multicolumn{2}{c|}{\textbf{Score}} & \textbf{Rank} & \textbf{Word} & \multicolumn{2}{c|}{\textbf{Score}} \\
    \hline
    1 & lenient\_jj & -7 & 8453    &    1860 & peppery\_jj & 10 & 6506 \\
    2 & coquette & -7 & 6181    &    1859 & pungent\_jj & 8 & 8690 \\
    3 & corrective\_jj & -7 & 5358    &    1858 & broiler & 8 & 6826 \\
    4 & stringent\_jj & -7 & 3697    &    1857 & tender\_jj & 8 & 872 \\
    5 & expeditious\_jj & -6 & 7143    &    1856 & acrid\_jj & 7 & 5796 \\
    6 & demure\_jj & -6 & 6391    &    1855 & gingery\_jj & 7 & 2942 \\
    7 & lascivious\_jj & -6 & 5639    &    1854 & sour\_jj & 7 & 2680 \\
    8 & uncooperative\_jj & -6 & 5517    &    1853 & poisonous\_jj & 7 & 2298 \\
    9 & ladylike\_jj & -6 & 4982    &    1852 & downright\_jj & 6 & 9629 \\
    10 & equitable\_jj & -6 & 4930    &    1851 & boneless\_jj & 6 & 9443 \\
    11 & prim\_jj & -6 & 3591    &    1850 & fiery\_jj & 6 & 9192 \\
    12 & disciplinarian & -6 & 2973    &    1849 & oily\_jj & 6 & 8542 \\
    13 & transparent\_jj & -6 & 609    &    1848 & fruit & 6 & 8368 \\
    14 & lewd\_jj & -5 & 9578    &    1847 & mushy\_jj & 6 & 8326 \\
    15 & insubordinate\_jj & -5 & 9496    &    1846 & venomous\_jj & 6 & 6461 \\
    16 & flirtatious\_jj & -5 & 8922    &    1845 & vinegary\_jj & 6 & 6447 \\
    17 & libidinous\_jj & -5 & 8383    &    1844 & buttery\_jj & 6 & 6383 \\
    18 & risqué\_jj & -5 & 8105    &    1843 & hard-boiled\_jj & 6 & 6356 \\
    19 & mousy\_jj & -5 & 8028    &    1842 & hearty\_jj & 6 & 4385 \\
    20 & unsportsmanlike\_jj & -5 & 7330    &    1841 & gooey\_jj & 6 & 1805 \\
    21 & lax\_jj & -5 & 6860    &    1840 & tasteless\_jj & 6 & 1244 \\
    22 & playboy & -5 & 6007    &    1839 & bitter\_jj & 5 & 9719 \\
    23 & rakish\_jj & -5 & 5887    &    1838 & ignorant\_jj & 5 & 8162 \\
    24 & gentle-hearted\_jj & -5 & 5640    &    1837 & uneducated\_jj & 5 & 8078 \\
    25 & intrusive\_jj & -5 & 5129    &    1836 & pundit & 5 & 7781 \\
    26 & rigorous\_jj & -5 & 4967    &    1835 & cheater & 5 & 5978 \\
    27 & strict\_jj & -5 & 4185    &    1834 & ham & 5 & 5820 \\
    28 & obtrusive\_jj & -5 & 2499    &    1833 & erudite\_jj & 5 & 4927 \\
    29 & womanly\_jj & -5 & 2185    &    1832 & vicious\_jj & 5 & 4903 \\
    30 & forcible\_jj & -5 & 2079    &    1831 & clam & 5 & 4256 \\
    \hline
    \end{tabular}
    \caption{Scores and rankings for most extreme 30 words in component \#13} 
\end{table}
\clearpage
\begin{table}[tbp]
    \begin{tabular}{| rlr@{.}l | rlr@{.}l |}
    \hline
    \textbf{Rank} & \textbf{Word} & \multicolumn{2}{c|}{\textbf{Score}} & \textbf{Rank} & \textbf{Word} & \multicolumn{2}{c|}{\textbf{Score}} \\
    \hline
    1 & indeterminate\_jj & -8 & 7283    &    1860 & mathematical\_jj & 7 & 3578 \\
    2 & indefinite\_jj & -8 & 6266    &    1859 & resistive\_jj & 7 & 2264 \\
    3 & unreserved\_jj & -8 & 5581    &    1858 & cerebral\_jj & 6 & 9686 \\
    4 & amicable\_jj & -7 & 9610    &    1857 & firebrand & 6 & 9655 \\
    5 & remiss\_jj & -7 & 4800    &    1856 & variant\_jj & 6 & 7812 \\
    6 & unfailing\_jj & -7 & 3048    &    1855 & cognitive\_jj & 6 & 4521 \\
    7 & unguarded\_jj & -7 & 2898    &    1854 & sophisticated\_jj & 6 & 3950 \\
    8 & indomitable\_jj & -6 & 8549    &    1853 & aplastic\_jj & 6 & 267 \\
    9 & imprudent\_jj & -6 & 7075    &    1852 & refractory\_jj & 5 & 9853 \\
    10 & immovable\_jj & -6 & 6905    &    1851 & deterministic\_jj & 5 & 5756 \\
    11 & unreasonable\_jj & -6 & 5543    &    1850 & analytical\_jj & 5 & 5734 \\
    12 & unequivocal\_jj & -6 & 3554    &    1849 & rigorous\_jj & 5 & 5333 \\
    13 & abject\_jj & -6 & 3294    &    1848 & refined\_jj & 5 & 5100 \\
    14 & unshakable\_jj & -6 & 3097    &    1847 & malignant\_jj & 5 & 3687 \\
    15 & imperturbable\_jj & -6 & 2507    &    1846 & nonvolatile\_jj & 5 & 3486 \\
    16 & sincere\_jj & -5 & 9723    &    1845 & combative\_jj & 5 & 2638 \\
    17 & emphatic\_jj & -5 & 9432    &    1844 & generalist & 5 & 2187 \\
    18 & unselfish\_jj & -5 & 8705    &    1843 & moderate\_jj & 5 & 1773 \\
    19 & angelic\_jj & -5 & 7932    &    1842 & stringent\_jj & 5 & 1065 \\
    20 & unfaithful\_jj & -5 & 6429    &    1841 & magnetic\_jj & 5 & 725 \\
    21 & orderly\_jj & -5 & 6025    &    1840 & folksy\_jj & 4 & 9696 \\
    22 & selfless\_jj & -5 & 5621    &    1839 & sharp-tongued\_jj & 4 & 9546 \\
    23 & unswerving\_jj & -5 & 5169    &    1838 & affective\_jj & 4 & 9532 \\
    24 & immutable\_jj & -5 & 4961    &    1837 & clownish\_jj & 4 & 9194 \\
    25 & abrupt\_jj & -5 & 4315    &    1836 & pundit & 4 & 9105 \\
    26 & unerring\_jj & -5 & 3794    &    1835 & strident\_jj & 4 & 8691 \\
    27 & otherworldly\_jj & -5 & 3641    &    1834 & secular\_jj & 4 & 8583 \\
    28 & indulgent\_jj & -5 & 3387    &    1833 & risqué\_jj & 4 & 8175 \\
    29 & unflagging\_jj & -5 & 3363    &    1832 & obstructive\_jj & 4 & 7907 \\
    30 & prayerful\_jj & -5 & 3351    &    1831 & responsive\_jj & 4 & 7288 \\
    \hline
    \end{tabular}
    \caption{Scores and rankings for most extreme 30 words in component \#14} 
\end{table}
\clearpage
\begin{table}[tbp]
    \begin{tabular}{| rlr@{.}l | rlr@{.}l |}
    \hline
    \textbf{Rank} & \textbf{Word} & \multicolumn{2}{c|}{\textbf{Score}} & \textbf{Rank} & \textbf{Word} & \multicolumn{2}{c|}{\textbf{Score}} \\
    \hline
    1 & convivial\_jj & -8 & 3507    &    1860 & angelic\_jj & 7 & 8274 \\
    2 & rowdy\_jj & -7 & 8645    &    1859 & optimistic\_jj & 7 & 1038 \\
    3 & leisurely\_jj & -7 & 7716    &    1858 & quitter & 7 & 678 \\
    4 & lewd\_jj & -6 & 7624    &    1857 & cherubic\_jj & 6 & 8139 \\
    5 & considerate\_jj & -6 & 5081    &    1856 & demure\_jj & 6 & 6386 \\
    6 & expeditious\_jj & -6 & 4999    &    1855 & pessimistic\_jj & 6 & 6196 \\
    7 & lascivious\_jj & -6 & 4830    &    1854 & vegetative\_jj & 6 & 5306 \\
    8 & courteous\_jj & -6 & 2167    &    1853 & girlish\_jj & 6 & 5153 \\
    9 & hectic\_jj & -6 & 1233    &    1852 & fluttery\_jj & 6 & 4082 \\
    10 & sociable\_jj & -5 & 9735    &    1851 & impassive\_jj & 6 & 3748 \\
    11 & outdoor\_jj & -5 & 9296    &    1850 & sexy\_jj & 6 & 3397 \\
    12 & unhurried\_jj & -5 & 9158    &    1849 & forward-looking\_jj & 6 & 2377 \\
    13 & unsociable\_jj & -5 & 9147    &    1848 & cautionary\_jj & 6 & 454 \\
    14 & unsportsmanlike\_jj & -5 & 8772    &    1847 & double-faced\_jj & 6 & 383 \\
    15 & testy\_jj & -5 & 7916    &    1846 & benign\_jj & 6 & 182 \\
    16 & low-pressure\_jj & -5 & 6226    &    1845 & acute\_jj & 5 & 9992 \\
    17 & raucous\_jj & -5 & 6162    &    1844 & wishful\_jj & 5 & 9821 \\
    18 & mannerly\_jj & -5 & 5450    &    1843 & unsure\_jj & 5 & 9715 \\
    19 & epicurean\_jj & -5 & 3123    &    1842 & secular\_jj & 5 & 9627 \\
    20 & inconsiderate\_jj & -5 & 2779    &    1841 & overactive\_jj & 5 & 9282 \\
    21 & unreserved\_jj & -4 & 9985    &    1840 & malignant\_jj & 5 & 8663 \\
    22 & wildcat\_jj & -4 & 9655    &    1839 & beatific\_jj & 5 & 8620 \\
    23 & rancorous\_jj & -4 & 8789    &    1838 & piercing\_jj & 5 & 7440 \\
    24 & circuitous\_jj & -4 & 8687    &    1837 & voluptuous\_jj & 5 & 7055 \\
    25 & frenetic\_jj & -4 & 7902    &    1836 & bright\_jj & 5 & 6756 \\
    26 & buzzy\_jj & -4 & 7770    &    1835 & lifeless\_jj & 5 & 6608 \\
    27 & sedate\_jj & -4 & 7666    &    1834 & womanly\_jj & 5 & 6145 \\
    28 & friendly\_jj & -4 & 7623    &    1833 & mannish\_jj & 5 & 5693 \\
    29 & gourmet & -4 & 7570    &    1832 & wary\_jj & 5 & 5548 \\
    30 & congenial\_jj & -4 & 7470    &    1831 & bullish\_jj & 5 & 5331 \\
    \hline
    \end{tabular}
    \caption{Scores and rankings for most extreme 30 words in component \#15} 
\end{table}
\clearpage

\subsection{MDS}
\label{app:rankedwordlists:2797words:mds}
\begin{longtable}[!htbp]{| rlr@{.}l |}
    \hline
    \textbf{Rank} & \textbf{Word} & \multicolumn{2}{c|}{\textbf{Score}} \\
    \hline
    \endhead
    1 & gamin & 0 & -3363 \\
    2 & coquettish\_jj & 0 & -3060 \\
    3 & kittenish\_jj & 0 & -2977 \\
    4 & tender-hearted\_jj & 0 & -2928 \\
    5 & shrewish\_jj & 0 & -2846 \\
    6 & donnish\_jj & 0 & -2833 \\
    7 & slangy\_jj & 0 & -2775 \\
    8 & sassy\_jj & 0 & -2738 \\
    9 & stuck-up\_jj & 0 & -2715 \\
    10 & sly\_jj & 0 & -2714 \\
    11 & ingenue & 0 & -2702 \\
    12 & melancholic & 0 & -2702 \\
    13 & tomboy & 0 & -2670 \\
    14 & brassy\_jj & 0 & -2666 \\
    15 & impish\_jj & 0 & -2582 \\
    16 & girlish\_jj & 0 & -2582 \\
    17 & sardonic\_jj & 0 & -2574 \\
    18 & suave\_jj & 0 & -2532 \\
    19 & virginal\_jj & 0 & -2513 \\
    20 & untamable\_jj & 0 & -2512 \\
    21 & go-getter & 0 & -2497 \\
    22 & coltish\_jj & 0 & -2494 \\
    23 & growly\_jj & 0 & -2489 \\
    24 & boyish\_jj & 0 & -2485 \\
    25 & mannish\_jj & 0 & -2480 \\
    26 & duffer & 0 & -2437 \\
    27 & mousy\_jj & 0 & -2433 \\
    28 & falstaffian\_jj & 0 & -2426 \\
    29 & stoic & 0 & -2418 \\
    30 & rakish\_jj & 0 & -2395 \\
    1831 & expensive\_jj & 0 & 3236 \\
    1832 & severe\_jj & 0 & 3245 \\
    1833 & flexible\_jj & 0 & 3262 \\
    1834 & wary\_jj & 0 & 3283 \\
    1835 & inaccurate\_jj & 0 & 3298 \\
    1836 & disruptive\_jj & 0 & 3320 \\
    1837 & inconsistent\_jj & 0 & 3322 \\
    1838 & critical\_jj & 0 & 3338 \\
    1839 & affected\_jj & 0 & 3358 \\
    1840 & responsible\_jj & 0 & 3365 \\
    1841 & confidential\_jj & 0 & 3397 \\
    1842 & lax\_jj & 0 & 3398 \\
    1843 & accurate\_jj & 0 & 3400 \\
    1844 & indirect\_jj & 0 & 3424 \\
    1845 & systematic\_jj & 0 & 3452 \\
    1846 & unfair\_jj & 0 & 3469 \\
    1847 & rigorous\_jj & 0 & 3511 \\
    1848 & drastic\_jj & 0 & 3511 \\
    1849 & helpful\_jj & 0 & 3557 \\
    1850 & dependent\_jj & 0 & 3569 \\
    1851 & productive\_jj & 0 & 3585 \\
    1852 & stable\_jj & 0 & 3585 \\
    1853 & volatile\_jj & 0 & 3601 \\
    1854 & direct\_jj & 0 & 3658 \\
    1855 & reasonable\_jj & 0 & 3829 \\
    1856 & consistent\_jj & 0 & 3851 \\
    1857 & contentious\_jj & 0 & 3868 \\
    1858 & prudent\_jj & 0 & 4006 \\
    1859 & stringent\_jj & 0 & 4020 \\
    1860 & beneficial\_jj & 0 & 4144 \\
    \hline
    \caption{Scores and rankings for most extreme 30 words in component \#1} \\
\end{longtable}
\begin{longtable}[!htbp]{| rlr@{.}l |}
    \hline
    \textbf{Rank} & \textbf{Word} & \multicolumn{2}{c|}{\textbf{Score}} \\
    \hline
    \endhead
    1 & bigoted\_jj & 0 & -3508 \\
    2 & unethical\_jj & 0 & -3268 \\
    3 & hypocritical\_jj & 0 & -3165 \\
    4 & untruthful\_jj & 0 & -3102 \\
    5 & godless\_jj & 0 & -3087 \\
    6 & untransparent\_jj & 0 & -3016 \\
    7 & dishonest\_jj & 0 & -3010 \\
    8 & deceitful\_jj & 0 & -2985 \\
    9 & mendacious\_jj & 0 & -2952 \\
    10 & unreasoning\_jj & 0 & -2934 \\
    11 & oversensitive\_jj & 0 & -2924 \\
    12 & ignorant\_jj & 0 & -2894 \\
    13 & greedy\_jj & 0 & -2828 \\
    14 & gutless\_jj & 0 & -2827 \\
    15 & unfeeling\_jj & 0 & -2803 \\
    16 & irresponsible\_jj & 0 & -2801 \\
    17 & muddle-headed\_jj & 0 & -2773 \\
    18 & uncivilized\_jj & 0 & -2767 \\
    19 & unprincipled\_jj & 0 & -2754 \\
    20 & egoistic\_jj & 0 & -2750 \\
    21 & pig-headed\_jj & 0 & -2746 \\
    22 & misguided\_jj & 0 & -2738 \\
    23 & thoughtless\_jj & 0 & -2724 \\
    24 & irrational\_jj & 0 & -2711 \\
    25 & vindictive\_jj & 0 & -2708 \\
    26 & spineless\_jj & 0 & -2690 \\
    27 & unconstructive\_jj & 0 & -2680 \\
    28 & weak-minded\_jj & 0 & -2665 \\
    29 & unmanly\_jj & 0 & -2644 \\
    30 & denigratory\_jj & 0 & -2640 \\
    1831 & confident\_jj & 0 & 2597 \\
    1832 & breezy\_jj & 0 & 2605 \\
    1833 & sociable\_jj & 0 & 2615 \\
    1834 & intimate\_jj & 0 & 2615 \\
    1835 & flexible\_jj & 0 & 2622 \\
    1836 & vivid\_jj & 0 & 2644 \\
    1837 & somber\_jj & 0 & 2652 \\
    1838 & cheerful\_jj & 0 & 2658 \\
    1839 & cool\_jj & 0 & 2707 \\
    1840 & airy\_jj & 0 & 2741 \\
    1841 & vivacious\_jj & 0 & 2751 \\
    1842 & bright\_jj & 0 & 2751 \\
    1843 & energetic\_jj & 0 & 2787 \\
    1844 & warm\_jj & 0 & 2791 \\
    1845 & brisk\_jj & 0 & 2837 \\
    1846 & quiet\_jj & 0 & 2842 \\
    1847 & buoyant\_jj & 0 & 2842 \\
    1848 & sparkling\_jj & 0 & 2843 \\
    1849 & versatile\_jj & 0 & 2866 \\
    1850 & serene\_jj & 0 & 2896 \\
    1851 & pleasant\_jj & 0 & 2916 \\
    1852 & dependable\_jj & 0 & 2943 \\
    1853 & gentle\_jj & 0 & 2981 \\
    1854 & sunny\_jj & 0 & 3037 \\
    1855 & polished\_jj & 0 & 3061 \\
    1856 & elegant\_jj & 0 & 3138 \\
    1857 & vibrant\_jj & 0 & 3332 \\
    1858 & calm\_jj & 0 & 3351 \\
    1859 & graceful\_jj & 0 & 3382 \\
    1860 & lively\_jj & 0 & 3629 \\
    \hline
    \caption{Scores and rankings for most extreme 30 words in component \#2} \\
\end{longtable}
\begin{longtable}[!htbp]{| rlr@{.}l |}
    \hline
    \textbf{Rank} & \textbf{Word} & \multicolumn{2}{c|}{\textbf{Score}} \\
    \hline
    \endhead
    1 & thoughtful\_jj & 0 & -3383 \\
    2 & honest\_jj & 0 & -3220 \\
    3 & pragmatic\_jj & 0 & -3094 \\
    4 & trustworthy\_jj & 0 & -3053 \\
    5 & open-minded\_jj & 0 & -3051 \\
    6 & forthright\_jj & 0 & -3025 \\
    7 & articulate\_jj & 0 & -3006 \\
    8 & cynical\_jj & 0 & -3005 \\
    9 & arrogant\_jj & 0 & -2964 \\
    10 & considerate\_jj & 0 & -2959 \\
    11 & level-headed\_jj & 0 & -2956 \\
    12 & naive\_jj & 0 & -2939 \\
    13 & deferential\_jj & 0 & -2856 \\
    14 & courageous\_jj & 0 & -2830 \\
    15 & courteous\_jj & 0 & -2824 \\
    16 & circumspect\_jj & 0 & -2784 \\
    17 & self-confident\_jj & 0 & -2741 \\
    18 & combative\_jj & 0 & -2714 \\
    19 & aloof\_jj & 0 & -2671 \\
    20 & approachable\_jj & 0 & -2654 \\
    21 & respectful\_jj & 0 & -2603 \\
    22 & sympathetic\_jj & 0 & -2583 \\
    23 & self-assured\_jj & 0 & -2559 \\
    24 & principled\_jj & 0 & -2536 \\
    25 & kind\_jj & 0 & -2518 \\
    26 & opinionated\_jj & 0 & -2498 \\
    27 & forceful\_jj & 0 & -2490 \\
    28 & accommodating\_jj & 0 & -2490 \\
    29 & astute\_jj & 0 & -2481 \\
    30 & gracious\_jj & 0 & -2442 \\
    1831 & migrant\_jj & 0 & 2492 \\
    1832 & epicurean & 0 & 2513 \\
    1833 & eruptive\_jj & 0 & 2536 \\
    1834 & fossil & 0 & 2551 \\
    1835 & clam & 0 & 2559 \\
    1836 & aplastic\_jj & 0 & 2571 \\
    1837 & comforter & 0 & 2586 \\
    1838 & dauber & 0 & 2602 \\
    1839 & exclusive\_jj & 0 & 2603 \\
    1840 & indoor\_jj & 0 & 2642 \\
    1841 & nonvolatile\_jj & 0 & 2652 \\
    1842 & hard-shelled\_jj & 0 & 2653 \\
    1843 & bubbler & 0 & 2663 \\
    1844 & boneless\_jj & 0 & 2712 \\
    1845 & fruit & 0 & 2722 \\
    1846 & ductile\_jj & 0 & 2758 \\
    1847 & magnetic\_jj & 0 & 2778 \\
    1848 & dare-devil & 0 & 2797 \\
    1849 & driftless\_jj & 0 & 2799 \\
    1850 & high-hat & 0 & 2805 \\
    1851 & outdoor\_jj & 0 & 2848 \\
    1852 & plastic\_jj & 0 & 2864 \\
    1853 & low-pressure\_jj & 0 & 2903 \\
    1854 & yellow\_jj & 0 & 2970 \\
    1855 & butterfly & 0 & 3125 \\
    1856 & gingery\_jj & 0 & 3139 \\
    1857 & bendable\_jj & 0 & 3147 \\
    1858 & broiler & 0 & 3177 \\
    1859 & pixy & 0 & 3416 \\
    1860 & soft-shelled\_jj & 0 & 3444 \\
    \hline
    \caption{Scores and rankings for most extreme 30 words in component \#3} \\
\end{longtable}
\begin{longtable}[!htbp]{| rlr@{.}l |}
    \hline
    \textbf{Rank} & \textbf{Word} & \multicolumn{2}{c|}{\textbf{Score}} \\
    \hline
    \endhead
    1 & fan & 0 & -2643 \\
    2 & lucky\_jj & 0 & -2481 \\
    3 & staunch\_jj & 0 & -2402 \\
    4 & bulldog & 0 & -2382 \\
    5 & rowdy\_jj & 0 & -2367 \\
    6 & stalwart & 0 & -2323 \\
    7 & jealous\_jj & 0 & -2311 \\
    8 & loyal\_jj & 0 & -2262 \\
    9 & hapless\_jj & 0 & -2252 \\
    10 & stalwart\_jj & 0 & -2217 \\
    11 & clown & 0 & -2198 \\
    12 & congratulatory\_jj & 0 & -2191 \\
    13 & long-suffering\_jj & 0 & -2154 \\
    14 & barker & 0 & -2148 \\
    15 & torrid\_jj & 0 & -2138 \\
    16 & tough\_jj & 0 & -2105 \\
    17 & cackler & 0 & -2054 \\
    18 & gambler & 0 & -2036 \\
    19 & hot-tempered\_jj & 0 & -2023 \\
    20 & shy\_jj & 0 & -1972 \\
    21 & foul-mouthed\_jj & 0 & -1958 \\
    22 & has-been & 0 & -1953 \\
    23 & bitch & 0 & -1945 \\
    24 & bitter\_jj & 0 & -1920 \\
    25 & cheater & 0 & -1914 \\
    26 & plucky\_jj & 0 & -1904 \\
    27 & brat & 0 & -1886 \\
    28 & driftless\_jj & 0 & -1884 \\
    29 & comedian & 0 & -1878 \\
    30 & geezer & 0 & -1868 \\
    1831 & untheatrical\_jj & 0 & 2042 \\
    1832 & analytical\_jj & 0 & 2045 \\
    1833 & immutable\_jj & 0 & 2048 \\
    1834 & calculable\_jj & 0 & 2074 \\
    1835 & unchangeable\_jj & 0 & 2089 \\
    1836 & objective\_jj & 0 & 2103 \\
    1837 & unspontaneous\_jj & 0 & 2114 \\
    1838 & anachronistic\_jj & 0 & 2146 \\
    1839 & unalterable\_jj & 0 & 2147 \\
    1840 & intuitive\_jj & 0 & 2223 \\
    1841 & discursive\_jj & 0 & 2228 \\
    1842 & inquisitorial\_jj & 0 & 2238 \\
    1843 & meditative\_jj & 0 & 2241 \\
    1844 & imitative\_jj & 0 & 2302 \\
    1845 & self-revealing\_jj & 0 & 2331 \\
    1846 & mutable\_jj & 0 & 2356 \\
    1847 & unvarying\_jj & 0 & 2395 \\
    1848 & mechanistic\_jj & 0 & 2399 \\
    1849 & intricate\_jj & 0 & 2407 \\
    1850 & deterministic\_jj & 0 & 2423 \\
    1851 & complex\_jj & 0 & 2462 \\
    1852 & unconstrained\_jj & 0 & 2469 \\
    1853 & subjective\_jj & 0 & 2501 \\
    1854 & poetic\_jj & 0 & 2520 \\
    1855 & imaginative\_jj & 0 & 2668 \\
    1856 & recondite\_jj & 0 & 2668 \\
    1857 & assimilative\_jj & 0 & 2712 \\
    1858 & participative\_jj & 0 & 2812 \\
    1859 & undogmatic\_jj & 0 & 2988 \\
    1860 & abstract\_jj & 0 & 3198 \\
    \hline
    \caption{Scores and rankings for most extreme 30 words in component \#4} \\
\end{longtable}
\begin{longtable}[!htbp]{| rlr@{.}l |}
    \hline
    \textbf{Rank} & \textbf{Word} & \multicolumn{2}{c|}{\textbf{Score}} \\
    \hline
    \endhead
    1 & farcical\_jj & 0 & -2682 \\
    2 & vitriolic\_jj & 0 & -2572 \\
    3 & one-sided\_jj & 0 & -2564 \\
    4 & thunderous\_jj & 0 & -2518 \\
    5 & clumsy\_jj & 0 & -2358 \\
    6 & terse\_jj & 0 & -2345 \\
    7 & melodramatic\_jj & 0 & -2308 \\
    8 & verbal\_jj & 0 & -2307 \\
    9 & rancorous\_jj & 0 & -2299 \\
    10 & testy\_jj & 0 & -2237 \\
    11 & conciliatory\_jj & 0 & -2220 \\
    12 & brief\_jj & 0 & -2205 \\
    13 & nonsensical\_jj & 0 & -2183 \\
    14 & ferocious\_jj & 0 & -2170 \\
    15 & caustic\_jj & 0 & -2160 \\
    16 & derisive\_jj & 0 & -2147 \\
    17 & quick-fire\_jj & 0 & -2139 \\
    18 & raucous\_jj & 0 & -2138 \\
    19 & stilted\_jj & 0 & -2120 \\
    20 & frenetic\_jj & 0 & -2118 \\
    21 & harsh\_jj & 0 & -2100 \\
    22 & contradictory\_jj & 0 & -2084 \\
    23 & blunt\_jj & 0 & -2066 \\
    24 & rhetorical\_jj & 0 & -2034 \\
    25 & chaotic\_jj & 0 & -2015 \\
    26 & bellicose\_jj & 0 & -2008 \\
    27 & strident\_jj & 0 & -2001 \\
    28 & sloppy\_jj & 0 & -2000 \\
    29 & acrimonious\_jj & 0 & -1978 \\
    30 & stormy\_jj & 0 & -1973 \\
    1831 & kind\_jj & 0 & 1968 \\
    1832 & proud\_jj & 0 & 2028 \\
    1833 & conscientious\_jj & 0 & 2029 \\
    1834 & articulate\_jj & 0 & 2033 \\
    1835 & adaptable\_jj & 0 & 2038 \\
    1836 & celibate & 0 & 2045 \\
    1837 & god-fearing\_jj & 0 & 2052 \\
    1838 & cultured\_jj & 0 & 2054 \\
    1839 & kind-hearted\_jj & 0 & 2063 \\
    1840 & inquiring\_jj & 0 & 2069 \\
    1841 & easy-going\_jj & 0 & 2087 \\
    1842 & literate\_jj & 0 & 2088 \\
    1843 & trustful\_jj & 0 & 2095 \\
    1844 & illiterate\_jj & 0 & 2112 \\
    1845 & loyal\_jj & 0 & 2149 \\
    1846 & fogy & 0 & 2200 \\
    1847 & stay-at-home\_jj & 0 & 2225 \\
    1848 & avid\_jj & 0 & 2287 \\
    1849 & knowledgeable\_jj & 0 & 2295 \\
    1850 & dedicated\_jj & 0 & 2472 \\
    1851 & loving\_jj & 0 & 2526 \\
    1852 & considerate\_jj & 0 & 2543 \\
    1853 & open-minded\_jj & 0 & 2561 \\
    1854 & active\_jj & 0 & 2604 \\
    1855 & intelligent\_jj & 0 & 2606 \\
    1856 & trustworthy\_jj & 0 & 2615 \\
    1857 & self-reliant\_jj & 0 & 2707 \\
    1858 & self-sufficient\_jj & 0 & 2753 \\
    1859 & sociable\_jj & 0 & 2813 \\
    1860 & educated\_jj & 0 & 3090 \\
    \hline
    \caption{Scores and rankings for most extreme 30 words in component \#5} \\
\end{longtable}
\begin{longtable}[!htbp]{| rlr@{.}l |}
    \hline
    \textbf{Rank} & \textbf{Word} & \multicolumn{2}{c|}{\textbf{Score}} \\
    \hline
    \endhead
    1 & picky\_jj & 0 & -2647 \\
    2 & lazy\_jj & 0 & -2601 \\
    3 & fussy\_jj & 0 & -2596 \\
    4 & sexy\_jj & 0 & -2419 \\
    5 & tasteless\_jj & 0 & -2326 \\
    6 & cool\_jj & 0 & -2256 \\
    7 & expensive\_jj & 0 & -2220 \\
    8 & gooey\_jj & 0 & -2139 \\
    9 & rude\_jj & 0 & -2130 \\
    10 & sneaky\_jj & 0 & -2078 \\
    11 & clingy\_jj & 0 & -2018 \\
    12 & bland\_jj & 0 & -2009 \\
    13 & pleasant\_jj & 0 & -1994 \\
    14 & bitch & 0 & -1987 \\
    15 & inconsiderate\_jj & 0 & -1981 \\
    16 & irritable\_jj & 0 & -1936 \\
    17 & lucky\_jj & 0 & -1929 \\
    18 & frisky\_jj & 0 & -1920 \\
    19 & pretentious\_jj & 0 & -1912 \\
    20 & self-conscious\_jj & 0 & -1899 \\
    21 & helpful\_jj & 0 & -1888 \\
    22 & nosey\_jj & 0 & -1854 \\
    23 & obtrusive\_jj & 0 & -1846 \\
    24 & choosy\_jj & 0 & -1806 \\
    25 & chicken-hearted\_jj & 0 & -1805 \\
    26 & casual\_jj & 0 & -1791 \\
    27 & impractical\_jj & 0 & -1786 \\
    28 & smug\_jj & 0 & -1781 \\
    29 & inaccurate\_jj & 0 & -1759 \\
    30 & indulgent\_jj & 0 & -1735 \\
    1831 & unflagging\_jj & 0 & 2215 \\
    1832 & stalwart\_jj & 0 & 2215 \\
    1833 & political\_jj & 0 & 2220 \\
    1834 & unflinching\_jj & 0 & 2223 \\
    1835 & factious\_jj & 0 & 2249 \\
    1836 & ultrareligious\_jj & 0 & 2262 \\
    1837 & tireless\_jj & 0 & 2303 \\
    1838 & unfaltering\_jj & 0 & 2318 \\
    1839 & stout-hearted\_jj & 0 & 2338 \\
    1840 & secular\_jj & 0 & 2340 \\
    1841 & ultraconservative & 0 & 2345 \\
    1842 & unfailing\_jj & 0 & 2368 \\
    1843 & dogged\_jj & 0 & 2438 \\
    1844 & indefatigable\_jj & 0 & 2450 \\
    1845 & autocratic\_jj & 0 & 2465 \\
    1846 & ultraconservative\_jj & 0 & 2486 \\
    1847 & religious\_jj & 0 & 2497 \\
    1848 & unshakable\_jj & 0 & 2595 \\
    1849 & outspoken\_jj & 0 & 2647 \\
    1850 & militant & 0 & 2666 \\
    1851 & democratic\_jj & 0 & 2701 \\
    1852 & stalwart & 0 & 2776 \\
    1853 & uncompromising\_jj & 0 & 2813 \\
    1854 & fervent\_jj & 0 & 2823 \\
    1855 & firebrand & 0 & 2867 \\
    1856 & unbending\_jj & 0 & 2924 \\
    1857 & steadfast\_jj & 0 & 2948 \\
    1858 & unswerving\_jj & 0 & 2992 \\
    1859 & staunch\_jj & 0 & 3009 \\
    1860 & untiring\_jj & 0 & 3128 \\
    \hline
    \caption{Scores and rankings for most extreme 30 words in component \#6} \\
\end{longtable}
\begin{longtable}[!htbp]{| rlr@{.}l |}
    \hline
    \textbf{Rank} & \textbf{Word} & \multicolumn{2}{c|}{\textbf{Score}} \\
    \hline
    \endhead
    1 & retrospective\_jj & 0 & -2377 \\
    2 & genius & 0 & -2377 \\
    3 & original\_jj & 0 & -2338 \\
    4 & unbiased\_jj & 0 & -2247 \\
    5 & facetious\_jj & 0 & -2179 \\
    6 & cogent\_jj & 0 & -2177 \\
    7 & sincere\_jj & 0 & -2176 \\
    8 & objective\_jj & 0 & -2172 \\
    9 & confidential\_jj & 0 & -2169 \\
    10 & remiss\_jj & 0 & -2153 \\
    11 & scholarly\_jj & 0 & -2151 \\
    12 & literary\_jj & 0 & -2139 \\
    13 & truthful\_jj & 0 & -2104 \\
    14 & insightful\_jj & 0 & -2066 \\
    15 & derogatory\_jj & 0 & -2064 \\
    16 & thorough\_jj & 0 & -2032 \\
    17 & giving\_jj & 0 & -2017 \\
    18 & unreserved\_jj & 0 & -1975 \\
    19 & candid\_jj & 0 & -1948 \\
    20 & teachable\_jj & 0 & -1940 \\
    21 & categorical\_jj & 0 & -1936 \\
    22 & bitch & 0 & -1915 \\
    23 & pithy\_jj & 0 & -1906 \\
    24 & musical\_jj & 0 & -1904 \\
    25 & exhaustive\_jj & 0 & -1901 \\
    26 & informative\_jj & 0 & -1879 \\
    27 & comedian & 0 & -1868 \\
    28 & teaser & 0 & -1865 \\
    29 & practical\_jj & 0 & -1860 \\
    30 & risque\_jj & 0 & -1855 \\
    1831 & fretful\_jj & 0 & 1888 \\
    1832 & conformist\_jj & 0 & 1894 \\
    1833 & placid\_jj & 0 & 1911 \\
    1834 & sluggish\_jj & 0 & 1919 \\
    1835 & distrustful\_jj & 0 & 1920 \\
    1836 & mistrustful\_jj & 0 & 1936 \\
    1837 & autocratic\_jj & 0 & 1954 \\
    1838 & impatient\_jj & 0 & 1961 \\
    1839 & inefficient\_jj & 0 & 1962 \\
    1840 & volatile\_jj & 0 & 1970 \\
    1841 & unpredictable\_jj & 0 & 1980 \\
    1842 & lethargic\_jj & 0 & 2007 \\
    1843 & hospitable\_jj & 0 & 2012 \\
    1844 & unproductive\_jj & 0 & 2017 \\
    1845 & unstable\_jj & 0 & 2024 \\
    1846 & apathetic\_jj & 0 & 2052 \\
    1847 & treacherous\_jj & 0 & 2068 \\
    1848 & reliant\_jj & 0 & 2122 \\
    1849 & icy\_jj & 0 & 2183 \\
    1850 & indifferent\_jj & 0 & 2193 \\
    1851 & rapacious\_jj & 0 & 2217 \\
    1852 & fickle\_jj & 0 & 2224 \\
    1853 & assertive\_jj & 0 & 2227 \\
    1854 & docile\_jj & 0 & 2231 \\
    1855 & fractious\_jj & 0 & 2247 \\
    1856 & clement\_jj & 0 & 2302 \\
    1857 & impervious\_jj & 0 & 2366 \\
    1858 & exhaustible\_jj & 0 & 2446 \\
    1859 & inhospitable\_jj & 0 & 2786 \\
    1860 & stagnant\_jj & 0 & 2956 \\
    \hline
    \caption{Scores and rankings for most extreme 30 words in component \#7} \\
\end{longtable}
\begin{longtable}[!htbp]{| rlr@{.}l |}
    \hline
    \textbf{Rank} & \textbf{Word} & \multicolumn{2}{c|}{\textbf{Score}} \\
    \hline
    \endhead
    1 & lukewarm\_jj & 0 & -2438 \\
    2 & circumspect\_jj & 0 & -2409 \\
    3 & frosty\_jj & 0 & -2361 \\
    4 & unenthusiastic\_jj & 0 & -2344 \\
    5 & sanguine\_jj & 0 & -2280 \\
    6 & abstinent\_jj & 0 & -2208 \\
    7 & remiss\_jj & 0 & -2202 \\
    8 & cautious\_jj & 0 & -2193 \\
    9 & tight-lipped\_jj & 0 & -2106 \\
    10 & conciliatory\_jj & 0 & -2101 \\
    11 & well-disposed\_jj & 0 & -2080 \\
    12 & congratulatory\_jj & 0 & -2044 \\
    13 & nagging\_jj & 0 & -1966 \\
    14 & pessimistic\_jj & 0 & -1963 \\
    15 & testy\_jj & 0 & -1945 \\
    16 & trustful\_jj & 0 & -1926 \\
    17 & touchy\_jj & 0 & -1914 \\
    18 & bullish\_jj & 0 & -1844 \\
    19 & doctrinaire & 0 & -1832 \\
    20 & adulatory\_jj & 0 & -1819 \\
    21 & prayerful\_jj & 0 & -1809 \\
    22 & curt\_jj & 0 & -1785 \\
    23 & reticent\_jj & 0 & -1766 \\
    24 & wary\_jj & 0 & -1759 \\
    25 & coy\_jj & 0 & -1741 \\
    26 & negative\_jj & 0 & -1730 \\
    27 & terse\_jj & 0 & -1716 \\
    28 & respectful\_jj & 0 & -1712 \\
    29 & skeptical\_jj & 0 & -1672 \\
    30 & reconciliatory\_jj & 0 & -1663 \\
    1831 & brutal\_jj & 0 & 1794 \\
    1832 & cruel\_jj & 0 & 1809 \\
    1833 & eclectic\_jj & 0 & 1810 \\
    1834 & malicious\_jj & 0 & 1817 \\
    1835 & inhuman\_jj & 0 & 1827 \\
    1836 & intuitive\_jj & 0 & 1832 \\
    1837 & ambitious\_jj & 0 & 1842 \\
    1838 & intelligent\_jj & 0 & 1844 \\
    1839 & archaic\_jj & 0 & 1852 \\
    1840 & predatory\_jj & 0 & 1881 \\
    1841 & unfair\_jj & 0 & 1885 \\
    1842 & adaptive\_jj & 0 & 1910 \\
    1843 & irrepressible\_jj & 0 & 1923 \\
    1844 & extravagant\_jj & 0 & 1929 \\
    1845 & clever\_jj & 0 & 1940 \\
    1846 & unpredictable\_jj & 0 & 1975 \\
    1847 & sadistic\_jj & 0 & 1981 \\
    1848 & inhumane\_jj & 0 & 1994 \\
    1849 & imaginative\_jj & 0 & 2103 \\
    1850 & cunning\_jj & 0 & 2132 \\
    1851 & sophisticated\_jj & 0 & 2143 \\
    1852 & accomplished\_jj & 0 & 2163 \\
    1853 & devious\_jj & 0 & 2166 \\
    1854 & ingenious\_jj & 0 & 2176 \\
    1855 & ruthless\_jj & 0 & 2276 \\
    1856 & inventive\_jj & 0 & 2305 \\
    1857 & manipulative\_jj & 0 & 2325 \\
    1858 & audacious\_jj & 0 & 2381 \\
    1859 & innovative\_jj & 0 & 2532 \\
    1860 & intricate\_jj & 0 & 2729 \\
    \hline
    \caption{Scores and rankings for most extreme 30 words in component \#8} \\
\end{longtable}
\begin{longtable}[!htbp]{| rlr@{.}l |}
    \hline
    \textbf{Rank} & \textbf{Word} & \multicolumn{2}{c|}{\textbf{Score}} \\
    \hline
    \endhead
    1 & lavish\_jj & 0 & -2568 \\
    2 & fanciful\_jj & 0 & -2405 \\
    3 & expensive\_jj & 0 & -2400 \\
    4 & grandiose\_jj & 0 & -2385 \\
    5 & chic\_jj & 0 & -2265 \\
    6 & staid\_jj & 0 & -2166 \\
    7 & fancier & 0 & -2160 \\
    8 & extravagant\_jj & 0 & -2126 \\
    9 & quaint\_jj & 0 & -1987 \\
    10 & bold\_jj & 0 & -1975 \\
    11 & traditional\_jj & 0 & -1974 \\
    12 & ostentatious\_jj & 0 & -1966 \\
    13 & stodgy\_jj & 0 & -1890 \\
    14 & utopian\_jj & 0 & -1885 \\
    15 & do-nothing & 0 & -1864 \\
    16 & glamorous\_jj & 0 & -1808 \\
    17 & fuddy-duddy & 0 & -1766 \\
    18 & original\_jj & 0 & -1744 \\
    19 & dodgy\_jj & 0 & -1733 \\
    20 & highfalutin\_jj & 0 & -1721 \\
    21 & risque\_jj & 0 & -1716 \\
    22 & bigheaded\_jj & 0 & -1710 \\
    23 & old-fashioned\_jj & 0 & -1677 \\
    24 & gourmet & 0 & -1671 \\
    25 & cosmopolitan\_jj & 0 & -1655 \\
    26 & profligate\_jj & 0 & -1648 \\
    27 & ludicrous\_jj & 0 & -1636 \\
    28 & die-hard\_jj & 0 & -1634 \\
    29 & flashy\_jj & 0 & -1633 \\
    30 & highbrow\_jj & 0 & -1621 \\
    1831 & vegetative\_jj & 0 & 1784 \\
    1832 & modifiable\_jj & 0 & 1791 \\
    1833 & malignant\_jj & 0 & 1793 \\
    1834 & unrelenting\_jj & 0 & 1794 \\
    1835 & maternal\_jj & 0 & 1811 \\
    1836 & neglectful\_jj & 0 & 1851 \\
    1837 & affective\_jj & 0 & 1851 \\
    1838 & piercing\_jj & 0 & 1852 \\
    1839 & questioning\_jj & 0 & 1953 \\
    1840 & withdrawn\_jj & 0 & 1970 \\
    1841 & antisocial\_jj & 0 & 1999 \\
    1842 & uncontrolled\_jj & 0 & 2010 \\
    1843 & asocial\_jj & 0 & 2034 \\
    1844 & noncompliant\_jj & 0 & 2043 \\
    1845 & possessive\_jj & 0 & 2085 \\
    1846 & unemotional\_jj & 0 & 2085 \\
    1847 & refractory\_jj & 0 & 2119 \\
    1848 & cackler & 0 & 2123 \\
    1849 & cerebral\_jj & 0 & 2172 \\
    1850 & obstructive\_jj & 0 & 2201 \\
    1851 & severe\_jj & 0 & 2287 \\
    1852 & acute\_jj & 0 & 2521 \\
    1853 & overactive\_jj & 0 & 2547 \\
    1854 & impulsive\_jj & 0 & 2563 \\
    1855 & abstinent\_jj & 0 & 2629 \\
    1856 & emotional\_jj & 0 & 2675 \\
    1857 & irritable\_jj & 0 & 2752 \\
    1858 & cognitive\_jj & 0 & 2828 \\
    1859 & autistic\_jj & 0 & 2837 \\
    1860 & aplastic\_jj & 0 & 3409 \\
    \hline
    \caption{Scores and rankings for most extreme 30 words in component \#9} \\
\end{longtable}
\begin{longtable}[!htbp]{| rlr@{.}l |}
    \hline
    \textbf{Rank} & \textbf{Word} & \multicolumn{2}{c|}{\textbf{Score}} \\
    \hline
    \endhead
    1 & quick-fire\_jj & 0 & -2341 \\
    2 & jack-of-all-trades & 0 & -2298 \\
    3 & adroit\_jj & 0 & -2239 \\
    4 & hard-nosed\_jj & 0 & -2217 \\
    5 & balky\_jj & 0 & -1940 \\
    6 & obdurate\_jj & 0 & -1886 \\
    7 & dogged\_jj & 0 & -1871 \\
    8 & assured\_jj & 0 & -1868 \\
    9 & sloppy\_jj & 0 & -1861 \\
    10 & cunning & 0 & -1857 \\
    11 & shrewd\_jj & 0 & -1814 \\
    12 & proven\_jj & 0 & -1797 \\
    13 & methodical\_jj & 0 & -1715 \\
    14 & inexperienced\_jj & 0 & -1688 \\
    15 & incisive\_jj & 0 & -1669 \\
    16 & iffy\_jj & 0 & -1652 \\
    17 & gutsy\_jj & 0 & -1624 \\
    18 & unerring\_jj & 0 & -1623 \\
    19 & versatile\_jj & 0 & -1618 \\
    20 & yes-man & 0 & -1606 \\
    21 & overconfident\_jj & 0 & -1602 \\
    22 & rash\_jj & 0 & -1571 \\
    23 & indestructible\_jj & 0 & -1569 \\
    24 & emphatic\_jj & 0 & -1562 \\
    25 & luckless\_jj & 0 & -1544 \\
    26 & speedy\_jj & 0 & -1540 \\
    27 & indefatigable\_jj & 0 & -1518 \\
    28 & forward\_jj & 0 & -1516 \\
    29 & imperious\_jj & 0 & -1515 \\
    30 & name-dropper & 0 & -1509 \\
    1831 & godless\_jj & 0 & 1762 \\
    1832 & mystical\_jj & 0 & 1770 \\
    1833 & quiet\_jj & 0 & 1770 \\
    1834 & clannish\_jj & 0 & 1794 \\
    1835 & cosmopolitan\_jj & 0 & 1815 \\
    1836 & austere\_jj & 0 & 1830 \\
    1837 & observant\_jj & 0 & 1850 \\
    1838 & ultraconservative\_jj & 0 & 1850 \\
    1839 & spontaneous\_jj & 0 & 1862 \\
    1840 & puritanical\_jj & 0 & 1864 \\
    1841 & heretical\_jj & 0 & 1866 \\
    1842 & extremist & 0 & 1895 \\
    1843 & noisy\_jj & 0 & 1921 \\
    1844 & raucous\_jj & 0 & 1924 \\
    1845 & prudish\_jj & 0 & 1959 \\
    1846 & loving\_jj & 0 & 1976 \\
    1847 & vibrant\_jj & 0 & 2018 \\
    1848 & irreligious\_jj & 0 & 2026 \\
    1849 & celibate & 0 & 2026 \\
    1850 & peaceful\_jj & 0 & 2082 \\
    1851 & tolerant\_jj & 0 & 2168 \\
    1852 & nonreligious\_jj & 0 & 2247 \\
    1853 & satanic\_jj & 0 & 2348 \\
    1854 & blasphemous\_jj & 0 & 2350 \\
    1855 & monastic\_jj & 0 & 2363 \\
    1856 & pious\_jj & 0 & 2372 \\
    1857 & secular\_jj & 0 & 2436 \\
    1858 & devout\_jj & 0 & 2694 \\
    1859 & violent\_jj & 0 & 2817 \\
    1860 & religious\_jj & 0 & 3803 \\
    \hline
    \caption{Scores and rankings for most extreme 30 words in component \#10} \\
\end{longtable}
\begin{longtable}[!htbp]{| rlr@{.}l |}
    \hline
    \textbf{Rank} & \textbf{Word} & \multicolumn{2}{c|}{\textbf{Score}} \\
    \hline
    \endhead
    1 & merciless\_jj & 0 & -2148 \\
    2 & humane\_jj & 0 & -2093 \\
    3 & dignified\_jj & 0 & -2046 \\
    4 & murderous\_jj & 0 & -2016 \\
    5 & merciful\_jj & 0 & -2010 \\
    6 & ruthless\_jj & 0 & -1995 \\
    7 & cold-blooded\_jj & 0 & -1981 \\
    8 & systematic\_jj & 0 & -1980 \\
    9 & deliberate\_jj & 0 & -1952 \\
    10 & inalterable\_jj & 0 & -1934 \\
    11 & courageous\_jj & 0 & -1929 \\
    12 & calm\_jj & 0 & -1920 \\
    13 & speedy\_jj & 0 & -1897 \\
    14 & brave\_jj & 0 & -1873 \\
    15 & gentle\_jj & 0 & -1857 \\
    16 & selfless\_jj & 0 & -1848 \\
    17 & brutal\_jj & 0 & -1846 \\
    18 & callous\_jj & 0 & -1817 \\
    19 & cowardly\_jj & 0 & -1792 \\
    20 & rational\_jj & 0 & -1759 \\
    21 & methodical\_jj & 0 & -1749 \\
    22 & remorseless\_jj & 0 & -1749 \\
    23 & cunning\_jj & 0 & -1717 \\
    24 & decisive\_jj & 0 & -1717 \\
    25 & consistent\_jj & 0 & -1664 \\
    26 & barbarous\_jj & 0 & -1660 \\
    27 & peaceful\_jj & 0 & -1649 \\
    28 & prudent\_jj & 0 & -1630 \\
    29 & sadistic\_jj & 0 & -1621 \\
    30 & relentless\_jj & 0 & -1617 \\
    1831 & irreverent\_jj & 0 & 1634 \\
    1832 & highbrow\_jj & 0 & 1644 \\
    1833 & offhand\_jj & 0 & 1684 \\
    1834 & oblique\_jj & 0 & 1684 \\
    1835 & outdated\_jj & 0 & 1713 \\
    1836 & original\_jj & 0 & 1718 \\
    1837 & intimate\_jj & 0 & 1733 \\
    1838 & unruly\_jj & 0 & 1756 \\
    1839 & inexact\_jj & 0 & 1788 \\
    1840 & explicit\_jj & 0 & 1800 \\
    1841 & ebullient\_jj & 0 & 1825 \\
    1842 & exhaustive\_jj & 0 & 1825 \\
    1843 & itinerant & 0 & 1828 \\
    1844 & unenthusiastic\_jj & 0 & 1868 \\
    1845 & encyclopedic\_jj & 0 & 1871 \\
    1846 & abrupt\_jj & 0 & 1912 \\
    1847 & unguarded\_jj & 0 & 1941 \\
    1848 & upstart\_jj & 0 & 1960 \\
    1849 & exclusive\_jj & 0 & 1987 \\
    1850 & independent\_jj & 0 & 1991 \\
    1851 & ill-tempered\_jj & 0 & 2002 \\
    1852 & accomplished\_jj & 0 & 2040 \\
    1853 & awkward\_jj & 0 & 2102 \\
    1854 & eagle-eyed\_jj & 0 & 2193 \\
    1855 & informal\_jj & 0 & 2217 \\
    1856 & adulatory\_jj & 0 & 2452 \\
    1857 & outspoken\_jj & 0 & 2603 \\
    1858 & avid\_jj & 0 & 2621 \\
    1859 & eclectic\_jj & 0 & 2765 \\
    1860 & acrimonious\_jj & 0 & 2777 \\
    \hline
    \caption{Scores and rankings for most extreme 30 words in component \#11} \\
\end{longtable}
\begin{longtable}[!htbp]{| rlr@{.}l |}
    \hline
    \textbf{Rank} & \textbf{Word} & \multicolumn{2}{c|}{\textbf{Score}} \\
    \hline
    \endhead
    1 & dispiriting\_jj & 0 & -2148 \\
    2 & bleak\_jj & 0 & -2035 \\
    3 & nagging\_jj & 0 & -1971 \\
    4 & mystical\_jj & 0 & -1937 \\
    5 & profound\_jj & 0 & -1922 \\
    6 & elusive\_jj & 0 & -1922 \\
    7 & genius & 0 & -1894 \\
    8 & whirlwind & 0 & -1852 \\
    9 & torrid\_jj & 0 & -1839 \\
    10 & purposeless\_jj & 0 & -1807 \\
    11 & morbid\_jj & 0 & -1797 \\
    12 & distant\_jj & 0 & -1793 \\
    13 & mundane\_jj & 0 & -1760 \\
    14 & madcap\_jj & 0 & -1745 \\
    15 & philosophical\_jj & 0 & -1745 \\
    16 & hectic\_jj & 0 & -1725 \\
    17 & curious\_jj & 0 & -1653 \\
    18 & otherworldly\_jj & 0 & -1616 \\
    19 & emotional\_jj & 0 & -1601 \\
    20 & fickle\_jj & 0 & -1593 \\
    21 & vicious\_jj & 0 & -1563 \\
    22 & levity & 0 & -1481 \\
    23 & exhortative\_jj & 0 & -1474 \\
    24 & fatalistic\_jj & 0 & -1468 \\
    25 & lethargic\_jj & 0 & -1467 \\
    26 & lucky\_jj & 0 & -1462 \\
    27 & shrinking & 0 & -1458 \\
    28 & valiant\_jj & 0 & -1447 \\
    29 & tricky\_jj & 0 & -1426 \\
    30 & prodigal & 0 & -1416 \\
    1831 & pliable\_jj & 0 & 1566 \\
    1832 & undemocratic\_jj & 0 & 1567 \\
    1833 & unsportsmanlike\_jj & 0 & 1570 \\
    1834 & unreserved\_jj & 0 & 1570 \\
    1835 & fruit & 0 & 1594 \\
    1836 & double-dealer & 0 & 1594 \\
    1837 & unrefined\_jj & 0 & 1598 \\
    1838 & high-handed\_jj & 0 & 1616 \\
    1839 & expeditious\_jj & 0 & 1628 \\
    1840 & conciliatory\_jj & 0 & 1653 \\
    1841 & coarse\_jj & 0 & 1702 \\
    1842 & pontifical\_jj & 0 & 1707 \\
    1843 & mannish\_jj & 0 & 1713 \\
    1844 & discreet\_jj & 0 & 1715 \\
    1845 & dainty\_jj & 0 & 1735 \\
    1846 & compliant\_jj & 0 & 1798 \\
    1847 & yellow\_jj & 0 & 1799 \\
    1848 & noncompliant\_jj & 0 & 1884 \\
    1849 & earthy\_jj & 0 & 1893 \\
    1850 & airy\_jj & 0 & 1941 \\
    1851 & broiler & 0 & 1949 \\
    1852 & flowery\_jj & 0 & 1958 \\
    1853 & double-faced\_jj & 0 & 1985 \\
    1854 & peppery\_jj & 0 & 2029 \\
    1855 & bristly\_jj & 0 & 2067 \\
    1856 & oily\_jj & 0 & 2068 \\
    1857 & transparent\_jj & 0 & 2150 \\
    1858 & gingery\_jj & 0 & 2222 \\
    1859 & vinegary\_jj & 0 & 2306 \\
    1860 & buttery\_jj & 0 & 2444 \\
    \hline
    \caption{Scores and rankings for most extreme 30 words in component \#12} \\
\end{longtable}
\begin{longtable}[!htbp]{| rlr@{.}l |}
    \hline
    \textbf{Rank} & \textbf{Word} & \multicolumn{2}{c|}{\textbf{Score}} \\
    \hline
    \endhead
    1 & unreserved\_jj & 0 & -2015 \\
    2 & bountiful\_jj & 0 & -1939 \\
    3 & angler & 0 & -1891 \\
    4 & ham & 0 & -1884 \\
    5 & abject\_jj & 0 & -1830 \\
    6 & indulgent\_jj & 0 & -1830 \\
    7 & enterprising\_jj & 0 & -1813 \\
    8 & fruit & 0 & -1775 \\
    9 & emphatic\_jj & 0 & -1770 \\
    10 & peppery\_jj & 0 & -1767 \\
    11 & fortunate\_jj & 0 & -1739 \\
    12 & instructive\_jj & 0 & -1724 \\
    13 & hearty\_jj & 0 & -1712 \\
    14 & industrious\_jj & 0 & -1709 \\
    15 & broiler & 0 & -1708 \\
    16 & unselfish\_jj & 0 & -1679 \\
    17 & unkind\_jj & 0 & -1667 \\
    18 & tender\_jj & 0 & -1662 \\
    19 & epicurean\_jj & 0 & -1661 \\
    20 & injudicious\_jj & 0 & -1657 \\
    21 & magnanimous\_jj & 0 & -1655 \\
    22 & selfless\_jj & 0 & -1640 \\
    23 & indomitable\_jj & 0 & -1608 \\
    24 & prayerful\_jj & 0 & -1604 \\
    25 & soft-shelled\_jj & 0 & -1586 \\
    26 & consolatory\_jj & 0 & -1584 \\
    27 & brambly\_jj & 0 & -1557 \\
    28 & sunny\_jj & 0 & -1549 \\
    29 & clement\_jj & 0 & -1538 \\
    30 & unremitting\_jj & 0 & -1514 \\
    1831 & high-strung\_jj & 0 & 1341 \\
    1832 & mannish\_jj & 0 & 1353 \\
    1833 & transparent\_jj & 0 & 1353 \\
    1834 & nonconforming\_jj & 0 & 1360 \\
    1835 & protective\_jj & 0 & 1361 \\
    1836 & compliant\_jj & 0 & 1367 \\
    1837 & tomboy & 0 & 1369 \\
    1838 & certain\_jj & 0 & 1373 \\
    1839 & dapper\_jj & 0 & 1381 \\
    1840 & bureaucratic\_jj & 0 & 1392 \\
    1841 & sexy\_jj & 0 & 1392 \\
    1842 & demure\_jj & 0 & 1422 \\
    1843 & passive\_jj & 0 & 1445 \\
    1844 & lax\_jj & 0 & 1460 \\
    1845 & formal\_jj & 0 & 1461 \\
    1846 & unauthoritative\_jj & 0 & 1497 \\
    1847 & masculine\_jj & 0 & 1548 \\
    1848 & coquette & 0 & 1554 \\
    1849 & conventional\_jj & 0 & 1594 \\
    1850 & mousy\_jj & 0 & 1624 \\
    1851 & rakish\_jj & 0 & 1673 \\
    1852 & flexible\_jj & 0 & 1710 \\
    1853 & complex\_jj & 0 & 1755 \\
    1854 & sophisticated\_jj & 0 & 1915 \\
    1855 & strict\_jj & 0 & 1921 \\
    1856 & complicated\_jj & 0 & 1922 \\
    1857 & intrusive\_jj & 0 & 1969 \\
    1858 & rigorous\_jj & 0 & 2013 \\
    1859 & stringent\_jj & 0 & 2222 \\
    1860 & rigid\_jj & 0 & 2224 \\
    \hline
    \caption{Scores and rankings for most extreme 30 words in component \#13} \\
\end{longtable}
\begin{longtable}[!htbp]{| rlr@{.}l |}
    \hline
    \textbf{Rank} & \textbf{Word} & \multicolumn{2}{c|}{\textbf{Score}} \\
    \hline
    \endhead
    1 & peppery\_jj & 0 & -2262 \\
    2 & fiery\_jj & 0 & -2158 \\
    3 & vocal\_jj & 0 & -2151 \\
    4 & coarse\_jj & 0 & -1902 \\
    5 & sour\_jj & 0 & -1865 \\
    6 & political\_jj & 0 & -1785 \\
    7 & mushy\_jj & 0 & -1774 \\
    8 & ignorant\_jj & 0 & -1765 \\
    9 & pungent\_jj & 0 & -1727 \\
    10 & moderate\_jj & 0 & -1703 \\
    11 & gingery\_jj & 0 & -1679 \\
    12 & buttery\_jj & 0 & -1675 \\
    13 & strident\_jj & 0 & -1637 \\
    14 & crusty\_jj & 0 & -1636 \\
    15 & deep\_jj & 0 & -1635 \\
    16 & oily\_jj & 0 & -1567 \\
    17 & wary\_jj & 0 & -1566 \\
    18 & broiler & 0 & -1558 \\
    19 & earthy\_jj & 0 & -1557 \\
    20 & skeptical\_jj & 0 & -1557 \\
    21 & acute\_jj & 0 & -1532 \\
    22 & boneless\_jj & 0 & -1527 \\
    23 & hard-shelled\_jj & 0 & -1518 \\
    24 & outspoken\_jj & 0 & -1510 \\
    25 & fruit & 0 & -1498 \\
    26 & tender\_jj & 0 & -1488 \\
    27 & epigrammatic\_jj & 0 & -1470 \\
    28 & pundit & 0 & -1451 \\
    29 & yellow\_jj & 0 & -1438 \\
    30 & unctuous\_jj & 0 & -1432 \\
    1831 & cooperative\_jj & 0 & 1439 \\
    1832 & cloistered\_jj & 0 & 1444 \\
    1833 & intimate\_jj & 0 & 1479 \\
    1834 & tempestuous\_jj & 0 & 1488 \\
    1835 & light-hearted\_jj & 0 & 1492 \\
    1836 & painstaking\_jj & 0 & 1496 \\
    1837 & torturous\_jj & 0 & 1504 \\
    1838 & humane\_jj & 0 & 1508 \\
    1839 & buccaneer & 0 & 1515 \\
    1840 & outdoor\_jj & 0 & 1534 \\
    1841 & formal\_jj & 0 & 1537 \\
    1842 & businesslike\_jj & 0 & 1537 \\
    1843 & frenetic\_jj & 0 & 1543 \\
    1844 & lenient\_jj & 0 & 1565 \\
    1845 & unsociable\_jj & 0 & 1620 \\
    1846 & staid\_jj & 0 & 1633 \\
    1847 & convivial\_jj & 0 & 1672 \\
    1848 & orderly\_jj & 0 & 1707 \\
    1849 & genteel\_jj & 0 & 1730 \\
    1850 & circuitous\_jj & 0 & 1750 \\
    1851 & lavish\_jj & 0 & 1812 \\
    1852 & haphazard\_jj & 0 & 1833 \\
    1853 & equitable\_jj & 0 & 1838 \\
    1854 & discreet\_jj & 0 & 1861 \\
    1855 & amicable\_jj & 0 & 1985 \\
    1856 & informal\_jj & 0 & 1993 \\
    1857 & sedate\_jj & 0 & 1997 \\
    1858 & hectic\_jj & 0 & 2203 \\
    1859 & expeditious\_jj & 0 & 2458 \\
    1860 & leisurely\_jj & 0 & 2544 \\
    \hline
    \caption{Scores and rankings for most extreme 30 words in component \#14} \\
\end{longtable}
\begin{longtable}[!htbp]{| rlr@{.}l |}
    \hline
    \textbf{Rank} & \textbf{Word} & \multicolumn{2}{c|}{\textbf{Score}} \\
    \hline
    \endhead
    1 & rowdy\_jj & 0 & -1950 \\
    2 & rancorous\_jj & 0 & -1872 \\
    3 & convivial\_jj & 0 & -1848 \\
    4 & noisy\_jj & 0 & -1783 \\
    5 & raucous\_jj & 0 & -1736 \\
    6 & knowledgeable\_jj & 0 & -1704 \\
    7 & conversational\_jj & 0 & -1637 \\
    8 & dedicated\_jj & 0 & -1539 \\
    9 & slothful\_jj & 0 & -1539 \\
    10 & testy\_jj & 0 & -1534 \\
    11 & productive\_jj & 0 & -1516 \\
    12 & political\_jj & 0 & -1503 \\
    13 & lively\_jj & 0 & -1491 \\
    14 & wordy\_jj & 0 & -1439 \\
    15 & opinionated\_jj & 0 & -1435 \\
    16 & venomous\_jj & 0 & -1426 \\
    17 & bawdy\_jj & 0 & -1419 \\
    18 & generalist & 0 & -1404 \\
    19 & boisterous\_jj & 0 & -1403 \\
    20 & methodical\_jj & 0 & -1402 \\
    21 & long-winded\_jj & 0 & -1380 \\
    22 & leisurely\_jj & 0 & -1376 \\
    23 & foul-mouthed\_jj & 0 & -1375 \\
    24 & complex\_jj & 0 & -1355 \\
    25 & diplomatic\_jj & 0 & -1352 \\
    26 & noncoercive\_jj & 0 & -1352 \\
    27 & gossipy\_jj & 0 & -1341 \\
    28 & digressive\_jj & 0 & -1335 \\
    29 & painstaking\_jj & 0 & -1331 \\
    30 & polemical\_jj & 0 & -1326 \\
    1831 & unfailing\_jj & 0 & 1448 \\
    1832 & ethereal\_jj & 0 & 1448 \\
    1833 & definite\_jj & 0 & 1458 \\
    1834 & exact\_jj & 0 & 1470 \\
    1835 & sexy\_jj & 0 & 1476 \\
    1836 & lifeless\_jj & 0 & 1497 \\
    1837 & unalterable\_jj & 0 & 1502 \\
    1838 & indifferent\_jj & 0 & 1514 \\
    1839 & pessimistic\_jj & 0 & 1516 \\
    1840 & bullish\_jj & 0 & 1525 \\
    1841 & careworn\_jj & 0 & 1545 \\
    1842 & indefinite\_jj & 0 & 1563 \\
    1843 & coquettish\_jj & 0 & 1579 \\
    1844 & womanly\_jj & 0 & 1644 \\
    1845 & bleak\_jj & 0 & 1650 \\
    1846 & indomitable\_jj & 0 & 1690 \\
    1847 & girlish\_jj & 0 & 1714 \\
    1848 & bright\_jj & 0 & 1721 \\
    1849 & contrary\_jj & 0 & 1732 \\
    1850 & virginal\_jj & 0 & 1733 \\
    1851 & confident\_jj & 0 & 1862 \\
    1852 & beatific\_jj & 0 & 1882 \\
    1853 & unshakable\_jj & 0 & 1900 \\
    1854 & cherubic\_jj & 0 & 1974 \\
    1855 & demure\_jj & 0 & 2104 \\
    1856 & optimistic\_jj & 0 & 2136 \\
    1857 & unequivocal\_jj & 0 & 2166 \\
    1858 & unreasonable\_jj & 0 & 2175 \\
    1859 & impassive\_jj & 0 & 2310 \\
    1860 & angelic\_jj & 0 & 2893 \\
    \hline
    \caption{Scores and rankings for most extreme 30 words in component \#15} \\
\end{longtable}
\begin{longtable}[!htbp]{| rlr@{.}l |}
    \hline
    \textbf{Rank} & \textbf{Word} & \multicolumn{2}{c|}{\textbf{Score}} \\
    \hline
    \endhead
    1 & fossil & 0 & -2086 \\
    2 & bleak\_jj & 0 & -1972 \\
    3 & backboned\_jj & 0 & -1925 \\
    4 & tight-lipped\_jj & 0 & -1921 \\
    5 & migratory\_jj & 0 & -1894 \\
    6 & cautionary\_jj & 0 & -1864 \\
    7 & sanguine\_jj & 0 & -1758 \\
    8 & prophetic\_jj & 0 & -1754 \\
    9 & temperate\_jj & 0 & -1732 \\
    10 & inhospitable\_jj & 0 & -1719 \\
    11 & unreliable\_jj & 0 & -1701 \\
    12 & confidential\_jj & 0 & -1655 \\
    13 & reticent\_jj & 0 & -1644 \\
    14 & contradictory\_jj & 0 & -1609 \\
    15 & scholarly\_jj & 0 & -1571 \\
    16 & pessimistic\_jj & 0 & -1568 \\
    17 & derelict\_jj & 0 & -1559 \\
    18 & amnesic\_jj & 0 & -1544 \\
    19 & cold-blooded\_jj & 0 & -1544 \\
    20 & imprecise\_jj & 0 & -1529 \\
    21 & prescient\_jj & 0 & -1515 \\
    22 & impassive\_jj & 0 & -1509 \\
    23 & shallow\_jj & 0 & -1509 \\
    24 & inaccurate\_jj & 0 & -1494 \\
    25 & heretical\_jj & 0 & -1492 \\
    26 & unread\_jj & 0 & -1492 \\
    27 & hard-shelled\_jj & 0 & -1460 \\
    28 & buccaneer & 0 & -1422 \\
    29 & cryptic\_jj & 0 & -1421 \\
    30 & painstaking\_jj & 0 & -1412 \\
    1831 & relentless\_jj & 0 & 1328 \\
    1832 & peppy\_jj & 0 & 1361 \\
    1833 & naysaying & 0 & 1362 \\
    1834 & adaptive\_jj & 0 & 1383 \\
    1835 & loud\_jj & 0 & 1420 \\
    1836 & spontaneous\_jj & 0 & 1421 \\
    1837 & infantile\_jj & 0 & 1439 \\
    1838 & innovative\_jj & 0 & 1453 \\
    1839 & thunderous\_jj & 0 & 1455 \\
    1840 & energetic\_jj & 0 & 1507 \\
    1841 & imperturbable\_jj & 0 & 1508 \\
    1842 & intense\_jj & 0 & 1515 \\
    1843 & headlong\_jj & 0 & 1519 \\
    1844 & rowdy\_jj & 0 & 1521 \\
    1845 & social\_jj & 0 & 1544 \\
    1846 & progressive\_jj & 0 & 1561 \\
    1847 & vigorous\_jj & 0 & 1569 \\
    1848 & spirited\_jj & 0 & 1569 \\
    1849 & frenetic\_jj & 0 & 1602 \\
    1850 & emotional\_jj & 0 & 1609 \\
    1851 & vim & 0 & 1674 \\
    1852 & casual\_jj & 0 & 1727 \\
    1853 & labile\_jj & 0 & 1734 \\
    1854 & aggressive\_jj & 0 & 1814 \\
    1855 & unrestrained\_jj & 0 & 1830 \\
    1856 & penny-wise\_jj & 0 & 1836 \\
    1857 & boisterous\_jj & 0 & 1862 \\
    1858 & raucous\_jj & 0 & 1885 \\
    1859 & excessive\_jj & 0 & 2035 \\
    1860 & unbridled\_jj & 0 & 2242 \\
    \hline
    \caption{Scores and rankings for most extreme 30 words in component \#16} \\
\end{longtable}
\begin{longtable}[!htbp]{| rlr@{.}l |}
    \hline
    \textbf{Rank} & \textbf{Word} & \multicolumn{2}{c|}{\textbf{Score}} \\
    \hline
    \endhead
    1 & orderly\_jj & 0 & -1980 \\
    2 & inefficient\_jj & 0 & -1957 \\
    3 & messy\_jj & 0 & -1825 \\
    4 & overemotional\_jj & 0 & -1825 \\
    5 & informal\_jj & 0 & -1823 \\
    6 & antiquated\_jj & 0 & -1764 \\
    7 & factious\_jj & 0 & -1691 \\
    8 & bureaucratic\_jj & 0 & -1686 \\
    9 & autonomous\_jj & 0 & -1653 \\
    10 & taunter & 0 & -1627 \\
    11 & fractious\_jj & 0 & -1627 \\
    12 & immovable\_jj & 0 & -1625 \\
    13 & chaotic\_jj & 0 & -1621 \\
    14 & fogy & 0 & -1580 \\
    15 & outgoing\_jj & 0 & -1535 \\
    16 & amicable\_jj & 0 & -1472 \\
    17 & complaisant\_jj & 0 & -1461 \\
    18 & illiterate\_jj & 0 & -1449 \\
    19 & untransparent\_jj & 0 & -1443 \\
    20 & outdated\_jj & 0 & -1420 \\
    21 & balky\_jj & 0 & -1405 \\
    22 & awkward\_jj & 0 & -1403 \\
    23 & unstable\_jj & 0 & -1380 \\
    24 & narrow\_jj & 0 & -1377 \\
    25 & amiable\_jj & 0 & -1369 \\
    26 & lawless\_jj & 0 & -1367 \\
    27 & honest\_jj & 0 & -1360 \\
    28 & trustful\_jj & 0 & -1333 \\
    29 & buffoon & 0 & -1288 \\
    30 & provincial\_jj & 0 & -1284 \\
    1831 & abstinent\_jj & 0 & 1271 \\
    1832 & migratory\_jj & 0 & 1296 \\
    1833 & zealous\_jj & 0 & 1297 \\
    1834 & retrospective\_jj & 0 & 1307 \\
    1835 & altruistic\_jj & 0 & 1309 \\
    1836 & malignant\_jj & 0 & 1317 \\
    1837 & thrifty\_jj & 0 & 1317 \\
    1838 & epicurean\_jj & 0 & 1332 \\
    1839 & sedentary\_jj & 0 & 1333 \\
    1840 & vociferous\_jj & 0 & 1359 \\
    1841 & drastic\_jj & 0 & 1367 \\
    1842 & fervent\_jj & 0 & 1370 \\
    1843 & cutthroat & 0 & 1371 \\
    1844 & acquisitive\_jj & 0 & 1386 \\
    1845 & cognitive\_jj & 0 & 1390 \\
    1846 & daring\_jj & 0 & 1436 \\
    1847 & vestal & 0 & 1437 \\
    1848 & savoir-faire & 0 & 1443 \\
    1849 & predatory\_jj & 0 & 1502 \\
    1850 & exacting\_jj & 0 & 1513 \\
    1851 & circumspect\_jj & 0 & 1513 \\
    1852 & cavalier\_jj & 0 & 1518 \\
    1853 & wanton\_jj & 0 & 1521 \\
    1854 & choosy\_jj & 0 & 1586 \\
    1855 & antisocial\_jj & 0 & 1625 \\
    1856 & penetrative\_jj & 0 & 1645 \\
    1857 & unsportsmanlike\_jj & 0 & 1664 \\
    1858 & compulsive\_jj & 0 & 1727 \\
    1859 & aggressive\_jj & 0 & 1884 \\
    1860 & discerning\_jj & 0 & 1998 \\
    \hline
    \caption{Scores and rankings for most extreme 30 words in component \#17} \\
\end{longtable}
\begin{longtable}[!htbp]{| rlr@{.}l |}
    \hline
    \textbf{Rank} & \textbf{Word} & \multicolumn{2}{c|}{\textbf{Score}} \\
    \hline
    \endhead
    1 & eager\_jj & 0 & -1833 \\
    2 & gun-shy\_jj & 0 & -1828 \\
    3 & unscrupulous\_jj & 0 & -1707 \\
    4 & wary\_jj & 0 & -1552 \\
    5 & explicit\_jj & 0 & -1522 \\
    6 & long-suffering\_jj & 0 & -1516 \\
    7 & bullheaded\_jj & 0 & -1460 \\
    8 & vocal\_jj & 0 & -1424 \\
    9 & timid & 0 & -1369 \\
    10 & obliging\_jj & 0 & -1356 \\
    11 & tight-lipped\_jj & 0 & -1346 \\
    12 & jealous\_jj & 0 & -1328 \\
    13 & lachrymose\_jj & 0 & -1306 \\
    14 & careful\_jj & 0 & -1301 \\
    15 & skeptical\_jj & 0 & -1300 \\
    16 & loyal\_jj & 0 & -1288 \\
    17 & avaricious\_jj & 0 & -1283 \\
    18 & confident\_jj & 0 & -1280 \\
    19 & greedy\_jj & 0 & -1279 \\
    20 & unwary\_jj & 0 & -1265 \\
    21 & brigandish\_jj & 0 & -1260 \\
    22 & reverent\_jj & 0 & -1255 \\
    23 & ribald\_jj & 0 & -1234 \\
    24 & impatient\_jj & 0 & -1234 \\
    25 & sophisticated\_jj & 0 & -1233 \\
    26 & alert\_jj & 0 & -1221 \\
    27 & unemphatic\_jj & 0 & -1206 \\
    28 & mistrustful\_jj & 0 & -1201 \\
    29 & sensitive\_jj & 0 & -1198 \\
    30 & lucky\_jj & 0 & -1196 \\
    1831 & doer & 0 & 1380 \\
    1832 & placid\_jj & 0 & 1409 \\
    1833 & blustery\_jj & 0 & 1418 \\
    1834 & obsessive & 0 & 1433 \\
    1835 & outdated\_jj & 0 & 1449 \\
    1836 & stuffy\_jj & 0 & 1449 \\
    1837 & quaint\_jj & 0 & 1456 \\
    1838 & overactive\_jj & 0 & 1463 \\
    1839 & compulsive\_jj & 0 & 1464 \\
    1840 & genteel\_jj & 0 & 1470 \\
    1841 & dour\_jj & 0 & 1553 \\
    1842 & unchanging\_jj & 0 & 1575 \\
    1843 & detached\_jj & 0 & 1611 \\
    1844 & pessimist & 0 & 1621 \\
    1845 & cerebral\_jj & 0 & 1630 \\
    1846 & pundit & 0 & 1646 \\
    1847 & vegetative\_jj & 0 & 1684 \\
    1848 & severe\_jj & 0 & 1710 \\
    1849 & aplastic\_jj & 0 & 1718 \\
    1850 & laggard & 0 & 1722 \\
    1851 & malignant\_jj & 0 & 1823 \\
    1852 & progressive\_jj & 0 & 1871 \\
    1853 & refractory\_jj & 0 & 1873 \\
    1854 & cognitive\_jj & 0 & 1893 \\
    1855 & benign\_jj & 0 & 1909 \\
    1856 & moderate\_jj & 0 & 1985 \\
    1857 & variant\_jj & 0 & 2014 \\
    1858 & obstructive\_jj & 0 & 2302 \\
    1859 & mild\_jj & 0 & 2404 \\
    1860 & affective\_jj & 0 & 2708 \\
    \hline
    \caption{Scores and rankings for most extreme 30 words in component \#18} \\
\end{longtable}
\begin{longtable}[!htbp]{| rlr@{.}l |}
    \hline
    \textbf{Rank} & \textbf{Word} & \multicolumn{2}{c|}{\textbf{Score}} \\
    \hline
    \endhead
    1 & buoyant\_jj & 0 & -2056 \\
    2 & bullish\_jj & 0 & -1904 \\
    3 & steady\_jj & 0 & -1754 \\
    4 & participative\_jj & 0 & -1703 \\
    5 & inward\_jj & 0 & -1687 \\
    6 & bearish\_jj & 0 & -1622 \\
    7 & derisive\_jj & 0 & -1597 \\
    8 & resistive\_jj & 0 & -1534 \\
    9 & sluggish\_jj & 0 & -1525 \\
    10 & slacker & 0 & -1435 \\
    11 & provident\_jj & 0 & -1406 \\
    12 & clattery\_jj & 0 & -1404 \\
    13 & nonvolatile\_jj & 0 & -1385 \\
    14 & stagnant\_jj & 0 & -1376 \\
    15 & mimetic\_jj & 0 & -1350 \\
    16 & sardonic\_jj & 0 & -1289 \\
    17 & spendthrift\_jj & 0 & -1273 \\
    18 & constrained\_jj & 0 & -1263 \\
    19 & miserly\_jj & 0 & -1251 \\
    20 & unbridled\_jj & 0 & -1234 \\
    21 & verve & 0 & -1220 \\
    22 & cautious\_jj & 0 & -1219 \\
    23 & sanguine\_jj & 0 & -1219 \\
    24 & live\_jj & 0 & -1216 \\
    25 & helter-skelter\_jj & 0 & -1213 \\
    26 & affected\_jj & 0 & -1205 \\
    27 & metropolitan\_jj & 0 & -1193 \\
    28 & buccaneer & 0 & -1192 \\
    29 & giving\_jj & 0 & -1192 \\
    30 & calculable\_jj & 0 & -1186 \\
    1831 & fierce\_jj & 0 & 1323 \\
    1832 & duffer & 0 & 1345 \\
    1833 & hussy & 0 & 1345 \\
    1834 & stormy\_jj & 0 & 1363 \\
    1835 & adulterous\_jj & 0 & 1370 \\
    1836 & philosophical\_jj & 0 & 1391 \\
    1837 & unladylike\_jj & 0 & 1432 \\
    1838 & glamorous\_jj & 0 & 1465 \\
    1839 & ungraceful\_jj & 0 & 1465 \\
    1840 & ethical\_jj & 0 & 1495 \\
    1841 & fractious\_jj & 0 & 1524 \\
    1842 & tough\_jj & 0 & 1541 \\
    1843 & unloving\_jj & 0 & 1546 \\
    1844 & serious\_jj & 0 & 1552 \\
    1845 & rancorous\_jj & 0 & 1595 \\
    1846 & awkward\_jj & 0 & 1618 \\
    1847 & sensitive\_jj & 0 & 1633 \\
    1848 & intense\_jj & 0 & 1649 \\
    1849 & complex\_jj & 0 & 1864 \\
    1850 & acrimonious\_jj & 0 & 1898 \\
    1851 & indelicate\_jj & 0 & 2040 \\
    1852 & intractable\_jj & 0 & 2055 \\
    1853 & touchy\_jj & 0 & 2075 \\
    1854 & complicated\_jj & 0 & 2232 \\
    1855 & bitter\_jj & 0 & 2399 \\
    1856 & tricky\_jj & 0 & 2422 \\
    1857 & messy\_jj & 0 & 2524 \\
    1858 & divisive\_jj & 0 & 2739 \\
    1859 & thorny\_jj & 0 & 2808 \\
    1860 & contentious\_jj & 0 & 3005 \\
    \hline
    \caption{Scores and rankings for most extreme 30 words in component \#19} \\
\end{longtable}
\begin{longtable}[!htbp]{| rlr@{.}l |}
    \hline
    \textbf{Rank} & \textbf{Word} & \multicolumn{2}{c|}{\textbf{Score}} \\
    \hline
    \endhead
    1 & do-or-die\_jj & 0 & -2000 \\
    2 & suppressive\_jj & 0 & -1962 \\
    3 & adaptable\_jj & 0 & -1796 \\
    4 & indestructible\_jj & 0 & -1734 \\
    5 & magnanimous\_jj & 0 & -1732 \\
    6 & warlike\_jj & 0 & -1721 \\
    7 & brigandish\_jj & 0 & -1697 \\
    8 & insurgent\_jj & 0 & -1602 \\
    9 & self-defensive\_jj & 0 & -1586 \\
    10 & militant & 0 & -1550 \\
    11 & invincible\_jj & 0 & -1535 \\
    12 & statesmanlike\_jj & 0 & -1516 \\
    13 & hospitable\_jj & 0 & -1429 \\
    14 & supersensitive\_jj & 0 & -1428 \\
    15 & unchangeable\_jj & 0 & -1418 \\
    16 & risque\_jj & 0 & -1409 \\
    17 & obtrusive\_jj & 0 & -1390 \\
    18 & antagonistic\_jj & 0 & -1387 \\
    19 & fossil & 0 & -1375 \\
    20 & adaptive\_jj & 0 & -1373 \\
    21 & extremist & 0 & -1359 \\
    22 & heretical\_jj & 0 & -1356 \\
    23 & blunderbuss & 0 & -1335 \\
    24 & elastic\_jj & 0 & -1324 \\
    25 & resistive\_jj & 0 & -1319 \\
    26 & versatile\_jj & 0 & -1319 \\
    27 & bellicose\_jj & 0 & -1303 \\
    28 & ductile\_jj & 0 & -1300 \\
    29 & discriminative\_jj & 0 & -1299 \\
    30 & uncivilized\_jj & 0 & -1298 \\
    1831 & picayune\_jj & 0 & 1216 \\
    1832 & dishonest\_jj & 0 & 1230 \\
    1833 & hurly-burly\_jj & 0 & 1233 \\
    1834 & punctilious\_jj & 0 & 1249 \\
    1835 & savoir-faire & 0 & 1249 \\
    1836 & slipshod\_jj & 0 & 1269 \\
    1837 & skinflint & 0 & 1274 \\
    1838 & intellectual\_jj & 0 & 1275 \\
    1839 & obsessive\_jj & 0 & 1301 \\
    1840 & scholarly\_jj & 0 & 1304 \\
    1841 & confidential\_jj & 0 & 1319 \\
    1842 & deep\_jj & 0 & 1328 \\
    1843 & careless\_jj & 0 & 1333 \\
    1844 & sloppy\_jj & 0 & 1343 \\
    1845 & lavish\_jj & 0 & 1349 \\
    1846 & profligate & 0 & 1365 \\
    1847 & social\_jj & 0 & 1366 \\
    1848 & reckless\_jj & 0 & 1403 \\
    1849 & unbridled\_jj & 0 & 1412 \\
    1850 & provident\_jj & 0 & 1457 \\
    1851 & consolatory\_jj & 0 & 1483 \\
    1852 & greedy\_jj & 0 & 1488 \\
    1853 & fraudulent\_jj & 0 & 1549 \\
    1854 & unscrupulous\_jj & 0 & 1550 \\
    1855 & dodgy\_jj & 0 & 1595 \\
    1856 & munificent\_jj & 0 & 1705 \\
    1857 & bureaucratic\_jj & 0 & 1725 \\
    1858 & sour\_jj & 0 & 1777 \\
    1859 & lax\_jj & 0 & 2076 \\
    1860 & excessive\_jj & 0 & 2191 \\
    \hline
    \caption{Scores and rankings for most extreme 30 words in component \#20} \\
\end{longtable}
\begin{longtable}[!htbp]{| rlr@{.}l |}
    \hline
    \textbf{Rank} & \textbf{Word} & \multicolumn{2}{c|}{\textbf{Score}} \\
    \hline
    \endhead
    1 & noisy\_jj & 0 & -1740 \\
    2 & unsociable\_jj & 0 & -1723 \\
    3 & unruly\_jj & 0 & -1720 \\
    4 & meticulous\_jj & 0 & -1610 \\
    5 & untrained\_jj & 0 & -1609 \\
    6 & loud\_jj & 0 & -1541 \\
    7 & wide-awake\_jj & 0 & -1482 \\
    8 & outdoor\_jj & 0 & -1477 \\
    9 & alert\_jj & 0 & -1440 \\
    10 & rigorous\_jj & 0 & -1418 \\
    11 & silent\_jj & 0 & -1406 \\
    12 & fidgety\_jj & 0 & -1365 \\
    13 & thorough\_jj & 0 & -1364 \\
    14 & narrow\_jj & 0 & -1348 \\
    15 & proper\_jj & 0 & -1323 \\
    16 & cursory\_jj & 0 & -1314 \\
    17 & fanatical\_jj & 0 & -1297 \\
    18 & dedicated\_jj & 0 & -1297 \\
    19 & solemn\_jj & 0 & -1294 \\
    20 & stern\_jj & 0 & -1283 \\
    21 & indoor\_jj & 0 & -1267 \\
    22 & strenuous\_jj & 0 & -1264 \\
    23 & accessible\_jj & 0 & -1260 \\
    24 & icy\_jj & 0 & -1232 \\
    25 & vigorous\_jj & 0 & -1230 \\
    26 & slick\_jj & 0 & -1226 \\
    27 & monastic\_jj & 0 & -1223 \\
    28 & speedy\_jj & 0 & -1222 \\
    29 & unmindful\_jj & 0 & -1212 \\
    30 & die-hard\_jj & 0 & -1206 \\
    1831 & gooey\_jj & 0 & 1298 \\
    1832 & unfaithful\_jj & 0 & 1298 \\
    1833 & boneless\_jj & 0 & 1298 \\
    1834 & munificent\_jj & 0 & 1301 \\
    1835 & stable\_jj & 0 & 1313 \\
    1836 & broiler & 0 & 1315 \\
    1837 & unproductive\_jj & 0 & 1331 \\
    1838 & reconciliatory\_jj & 0 & 1338 \\
    1839 & peppery\_jj & 0 & 1361 \\
    1840 & tender\_jj & 0 & 1365 \\
    1841 & dependent\_jj & 0 & 1365 \\
    1842 & laggard\_jj & 0 & 1374 \\
    1843 & neighborly\_jj & 0 & 1398 \\
    1844 & spunky\_jj & 0 & 1411 \\
    1845 & generous\_jj & 0 & 1426 \\
    1846 & notional\_jj & 0 & 1448 \\
    1847 & testy\_jj & 0 & 1449 \\
    1848 & uneconomical\_jj & 0 & 1470 \\
    1849 & equitable\_jj & 0 & 1503 \\
    1850 & acquisitive\_jj & 0 & 1527 \\
    1851 & amicable\_jj & 0 & 1529 \\
    1852 & playboy & 0 & 1547 \\
    1853 & tempestuous\_jj & 0 & 1561 \\
    1854 & indirect\_jj & 0 & 1609 \\
    1855 & adulterous\_jj & 0 & 1611 \\
    1856 & provident\_jj & 0 & 1755 \\
    1857 & bountiful\_jj & 0 & 1796 \\
    1858 & cordial\_jj & 0 & 1912 \\
    1859 & beneficial\_jj & 0 & 1945 \\
    1860 & sour\_jj & 0 & 2087 \\
    \hline
    \caption{Scores and rankings for most extreme 30 words in component \#21} \\
\end{longtable}
\begin{longtable}[!htbp]{| rlr@{.}l |}
    \hline
    \textbf{Rank} & \textbf{Word} & \multicolumn{2}{c|}{\textbf{Score}} \\
    \hline
    \endhead
    1 & constant\_jj & 0 & -1983 \\
    2 & unquestioning\_jj & 0 & -1944 \\
    3 & unswerving\_jj & 0 & -1699 \\
    4 & incontrovertible\_jj & 0 & -1609 \\
    5 & faithful\_jj & 0 & -1593 \\
    6 & die-hard\_jj & 0 & -1590 \\
    7 & flinty\_jj & 0 & -1582 \\
    8 & fickle\_jj & 0 & -1564 \\
    9 & reliable\_jj & 0 & -1533 \\
    10 & celibate & 0 & -1533 \\
    11 & steadfast\_jj & 0 & -1516 \\
    12 & fanatical\_jj & 0 & -1496 \\
    13 & single-minded\_jj & 0 & -1475 \\
    14 & finicky\_jj & 0 & -1459 \\
    15 & fervent\_jj & 0 & -1429 \\
    16 & variable\_jj & 0 & -1389 \\
    17 & resistive\_jj & 0 & -1367 \\
    18 & unfailing\_jj & 0 & -1287 \\
    19 & dependable\_jj & 0 & -1261 \\
    20 & uncritical\_jj & 0 & -1255 \\
    21 & remorseless\_jj & 0 & -1251 \\
    22 & loyal\_jj & 0 & -1230 \\
    23 & intense\_jj & 0 & -1218 \\
    24 & long-suffering\_jj & 0 & -1218 \\
    25 & adjustable\_jj & 0 & -1217 \\
    26 & indestructible\_jj & 0 & -1214 \\
    27 & proven\_jj & 0 & -1212 \\
    28 & unflinching\_jj & 0 & -1202 \\
    29 & steady\_jj & 0 & -1173 \\
    30 & unshakable\_jj & 0 & -1145 \\
    1831 & reckless\_jj & 0 & 1204 \\
    1832 & inventive\_jj & 0 & 1204 \\
    1833 & rash\_jj & 0 & 1205 \\
    1834 & chameleonlike\_jj & 0 & 1218 \\
    1835 & butterfly & 0 & 1220 \\
    1836 & corrective\_jj & 0 & 1237 \\
    1837 & irresponsible\_jj & 0 & 1244 \\
    1838 & provincial\_jj & 0 & 1245 \\
    1839 & somber\_jj & 0 & 1251 \\
    1840 & audacious\_jj & 0 & 1252 \\
    1841 & venturesome\_jj & 0 & 1255 \\
    1842 & nonhostile\_jj & 0 & 1259 \\
    1843 & humanitarian\_jj & 0 & 1267 \\
    1844 & backboned\_jj & 0 & 1300 \\
    1845 & clear-headed\_jj & 0 & 1302 \\
    1846 & clerkish\_jj & 0 & 1308 \\
    1847 & migrant & 0 & 1321 \\
    1848 & daring\_jj & 0 & 1324 \\
    1849 & migratory\_jj & 0 & 1362 \\
    1850 & musical\_jj & 0 & 1370 \\
    1851 & unwise\_jj & 0 & 1382 \\
    1852 & humanitarian & 0 & 1512 \\
    1853 & drastic\_jj & 0 & 1517 \\
    1854 & brave\_jj & 0 & 1523 \\
    1855 & unversed\_jj & 0 & 1544 \\
    1856 & obsessive & 0 & 1545 \\
    1857 & initiative & 0 & 1792 \\
    1858 & decisive\_jj & 0 & 1807 \\
    1859 & ambitious\_jj & 0 & 1915 \\
    1860 & bold\_jj & 0 & 2399 \\
    \hline
    \caption{Scores and rankings for most extreme 30 words in component \#22} \\
\end{longtable}
\begin{longtable}[!htbp]{| rlr@{.}l |}
    \hline
    \textbf{Rank} & \textbf{Word} & \multicolumn{2}{c|}{\textbf{Score}} \\
    \hline
    \endhead
    1 & stable\_jj & 0 & -1701 \\
    2 & cyclonic\_jj & 0 & -1659 \\
    3 & vibrant\_jj & 0 & -1629 \\
    4 & dauber & 0 & -1617 \\
    5 & adulatory\_jj & 0 & -1611 \\
    6 & resilient\_jj & 0 & -1603 \\
    7 & transparent\_jj & 0 & -1556 \\
    8 & bigheaded\_jj & 0 & -1546 \\
    9 & inaccurate\_jj & 0 & -1514 \\
    10 & sphinxlike\_jj & 0 & -1506 \\
    11 & vestal & 0 & -1497 \\
    12 & slanderous\_jj & 0 & -1441 \\
    13 & squally\_jj & 0 & -1401 \\
    14 & unrelenting\_jj & 0 & -1383 \\
    15 & tolerant\_jj & 0 & -1360 \\
    16 & goody-goody & 0 & -1336 \\
    17 & cutthroat\_jj & 0 & -1306 \\
    18 & consistent\_jj & 0 & -1299 \\
    19 & egotistic\_jj & 0 & -1299 \\
    20 & buoyant\_jj & 0 & -1274 \\
    21 & treacherous\_jj & 0 & -1270 \\
    22 & unbiased\_jj & 0 & -1260 \\
    23 & inhospitable\_jj & 0 & -1238 \\
    24 & dynamic\_jj & 0 & -1231 \\
    25 & malcontent & 0 & -1213 \\
    26 & pixy & 0 & -1197 \\
    27 & unloving\_jj & 0 & -1178 \\
    28 & accessible\_jj & 0 & -1174 \\
    29 & taunter & 0 & -1174 \\
    30 & vicious\_jj & 0 & -1174 \\
    1831 & presumptuous\_jj & 0 & 1076 \\
    1832 & extremist & 0 & 1077 \\
    1833 & clear-headed\_jj & 0 & 1081 \\
    1834 & amorous\_jj & 0 & 1088 \\
    1835 & superstitious\_jj & 0 & 1090 \\
    1836 & picky\_jj & 0 & 1092 \\
    1837 & crafty\_jj & 0 & 1097 \\
    1838 & ascetic\_jj & 0 & 1106 \\
    1839 & nagging\_jj & 0 & 1111 \\
    1840 & niggardly\_jj & 0 & 1112 \\
    1841 & remiss\_jj & 0 & 1113 \\
    1842 & hard-boiled\_jj & 0 & 1118 \\
    1843 & ham & 0 & 1144 \\
    1844 & nonchalant\_jj & 0 & 1156 \\
    1845 & fussy\_jj & 0 & 1180 \\
    1846 & gingery\_jj & 0 & 1189 \\
    1847 & brute\_jj & 0 & 1221 \\
    1848 & rash\_jj & 0 & 1239 \\
    1849 & boneless\_jj & 0 & 1245 \\
    1850 & self-defensive\_jj & 0 & 1287 \\
    1851 & squeamish\_jj & 0 & 1287 \\
    1852 & indefinite\_jj & 0 & 1329 \\
    1853 & covert\_jj & 0 & 1337 \\
    1854 & suppressive\_jj & 0 & 1342 \\
    1855 & oily\_jj & 0 & 1352 \\
    1856 & certain\_jj & 0 & 1357 \\
    1857 & insurgent\_jj & 0 & 1424 \\
    1858 & sugary\_jj & 0 & 1478 \\
    1859 & militant\_jj & 0 & 1533 \\
    1860 & reconciliatory\_jj & 0 & 1818 \\
    \hline
    \caption{Scores and rankings for most extreme 30 words in component \#23} \\
\end{longtable}
\begin{longtable}[!htbp]{| rlr@{.}l |}
    \hline
    \textbf{Rank} & \textbf{Word} & \multicolumn{2}{c|}{\textbf{Score}} \\
    \hline
    \endhead
    1 & charitable\_jj & 0 & -1685 \\
    2 & unwary\_jj & 0 & -1582 \\
    3 & oratorical\_jj & 0 & -1462 \\
    4 & malicious\_jj & 0 & -1451 \\
    5 & lavish\_jj & 0 & -1406 \\
    6 & reproachful\_jj & 0 & -1389 \\
    7 & furtive\_jj & 0 & -1377 \\
    8 & discretionary\_jj & 0 & -1309 \\
    9 & diplomatic\_jj & 0 & -1292 \\
    10 & ostentatious\_jj & 0 & -1289 \\
    11 & aboveboard\_jj & 0 & -1287 \\
    12 & incautious\_jj & 0 & -1270 \\
    13 & truckling & 0 & -1259 \\
    14 & direct\_jj & 0 & -1253 \\
    15 & unguarded\_jj & 0 & -1248 \\
    16 & jocular\_jj & 0 & -1245 \\
    17 & volcanic\_jj & 0 & -1234 \\
    18 & venomous\_jj & 0 & -1228 \\
    19 & inward\_jj & 0 & -1226 \\
    20 & extravagant\_jj & 0 & -1217 \\
    21 & glamorous\_jj & 0 & -1212 \\
    22 & falstaffian\_jj & 0 & -1211 \\
    23 & irresolute\_jj & 0 & -1207 \\
    24 & covetous\_jj & 0 & -1205 \\
    25 & fraudulent\_jj & 0 & -1188 \\
    26 & social\_jj & 0 & -1184 \\
    27 & hair-trigger\_jj & 0 & -1154 \\
    28 & philanthropic\_jj & 0 & -1140 \\
    29 & discreet\_jj & 0 & -1136 \\
    30 & backhanded\_jj & 0 & -1136 \\
    1831 & unrefined\_jj & 0 & 1153 \\
    1832 & brutal\_jj & 0 & 1162 \\
    1833 & eclectic & 0 & 1165 \\
    1834 & humane\_jj & 0 & 1173 \\
    1835 & nonreligious\_jj & 0 & 1179 \\
    1836 & vinegary\_jj & 0 & 1214 \\
    1837 & witch & 0 & 1229 \\
    1838 & gooey\_jj & 0 & 1261 \\
    1839 & miserabilist & 0 & 1265 \\
    1840 & torturous\_jj & 0 & 1265 \\
    1841 & retrospective\_jj & 0 & 1293 \\
    1842 & peppery\_jj & 0 & 1328 \\
    1843 & sentimental\_jj & 0 & 1328 \\
    1844 & picky\_jj & 0 & 1347 \\
    1845 & original\_jj & 0 & 1348 \\
    1846 & mechanistic\_jj & 0 & 1354 \\
    1847 & bitch & 0 & 1363 \\
    1848 & prodigal & 0 & 1391 \\
    1849 & stalwart & 0 & 1398 \\
    1850 & celebrative\_jj & 0 & 1424 \\
    1851 & carper & 0 & 1449 \\
    1852 & lenient\_jj & 0 & 1472 \\
    1853 & rigorous\_jj & 0 & 1493 \\
    1854 & inhumane\_jj & 0 & 1515 \\
    1855 & sissy & 0 & 1552 \\
    1856 & inflexible\_jj & 0 & 1571 \\
    1857 & strict\_jj & 0 & 1630 \\
    1858 & tender\_jj & 0 & 1685 \\
    1859 & cry-baby & 0 & 1857 \\
    1860 & mushy\_jj & 0 & 1990 \\
    \hline
    \caption{Scores and rankings for most extreme 30 words in component \#24} \\
\end{longtable}
\begin{longtable}[!htbp]{| rlr@{.}l |}
    \hline
    \textbf{Rank} & \textbf{Word} & \multicolumn{2}{c|}{\textbf{Score}} \\
    \hline
    \endhead
    1 & eruptive\_jj & 0 & -1893 \\
    2 & buccaneer & 0 & -1644 \\
    3 & amnesic\_jj & 0 & -1590 \\
    4 & suppressive\_jj & 0 & -1580 \\
    5 & uncontrolled\_jj & 0 & -1501 \\
    6 & remiss\_jj & 0 & -1486 \\
    7 & granitic\_jj & 0 & -1411 \\
    8 & daredevil & 0 & -1409 \\
    9 & intense\_jj & 0 & -1367 \\
    10 & unaccommodating\_jj & 0 & -1329 \\
    11 & deliberative\_jj & 0 & -1321 \\
    12 & witch & 0 & -1295 \\
    13 & immoderate\_jj & 0 & -1258 \\
    14 & naysaying & 0 & -1244 \\
    15 & tight-lipped\_jj & 0 & -1233 \\
    16 & duplicity & 0 & -1229 \\
    17 & unpredictable\_jj & 0 & -1183 \\
    18 & cry-baby & 0 & -1183 \\
    19 & mimetic\_jj & 0 & -1173 \\
    20 & precipitous\_jj & 0 & -1169 \\
    21 & epicurean\_jj & 0 & -1147 \\
    22 & careful\_jj & 0 & -1143 \\
    23 & ungallant\_jj & 0 & -1134 \\
    24 & inner-directed\_jj & 0 & -1118 \\
    25 & brigandish\_jj & 0 & -1096 \\
    26 & evasive\_jj & 0 & -1093 \\
    27 & bullheaded\_jj & 0 & -1076 \\
    28 & exact\_jj & 0 & -1051 \\
    29 & cutthroat\_jj & 0 & -1046 \\
    30 & poisonous\_jj & 0 & -1039 \\
    1831 & industrious\_jj & 0 & 1279 \\
    1832 & tame\_jj & 0 & 1308 \\
    1833 & do-or-die\_jj & 0 & 1318 \\
    1834 & ham & 0 & 1330 \\
    1835 & migrant\_jj & 0 & 1356 \\
    1836 & christlike\_jj & 0 & 1368 \\
    1837 & wayward\_jj & 0 & 1370 \\
    1838 & derogatory\_jj & 0 & 1371 \\
    1839 & itinerant & 0 & 1371 \\
    1840 & gutsy\_jj & 0 & 1371 \\
    1841 & libidinous\_jj & 0 & 1393 \\
    1842 & stingy\_jj & 0 & 1407 \\
    1843 & autistic\_jj & 0 & 1421 \\
    1844 & nonreligious\_jj & 0 & 1488 \\
    1845 & fluky\_jj & 0 & 1489 \\
    1846 & emphatic\_jj & 0 & 1494 \\
    1847 & devout\_jj & 0 & 1503 \\
    1848 & card & 0 & 1508 \\
    1849 & torrid\_jj & 0 & 1520 \\
    1850 & sloppy\_jj & 0 & 1539 \\
    1851 & lewd\_jj & 0 & 1567 \\
    1852 & one-sided\_jj & 0 & 1588 \\
    1853 & religious\_jj & 0 & 1631 \\
    1854 & lion-hearted\_jj & 0 & 1690 \\
    1855 & consolatory\_jj & 0 & 1733 \\
    1856 & penetrative\_jj & 0 & 1850 \\
    1857 & obdurate\_jj & 0 & 1962 \\
    1858 & nervy\_jj & 0 & 1967 \\
    1859 & quick-fire\_jj & 0 & 2125 \\
    1860 & butter-fingered\_jj & 0 & 2567 \\
    \hline
    \caption{Scores and rankings for most extreme 30 words in component \#25} \\
\end{longtable}
\begin{longtable}[!htbp]{| rlr@{.}l |}
    \hline
    \textbf{Rank} & \textbf{Word} & \multicolumn{2}{c|}{\textbf{Score}} \\
    \hline
    \endhead
    1 & wild\_jj & 0 & -1917 \\
    2 & predatory\_jj & 0 & -1796 \\
    3 & cagy\_jj & 0 & -1414 \\
    4 & informal\_jj & 0 & -1368 \\
    5 & strict\_jj & 0 & -1341 \\
    6 & blunderbuss & 0 & -1329 \\
    7 & temperate\_jj & 0 & -1329 \\
    8 & egghead & 0 & -1328 \\
    9 & unsuspicious\_jj & 0 & -1286 \\
    10 & devil-may-care\_jj & 0 & -1270 \\
    11 & sunny\_jj & 0 & -1265 \\
    12 & monopolistic\_jj & 0 & -1264 \\
    13 & stormy\_jj & 0 & -1222 \\
    14 & firm\_jj & 0 & -1192 \\
    15 & upstart & 0 & -1170 \\
    16 & affirmative\_jj & 0 & -1165 \\
    17 & explicit\_jj & 0 & -1153 \\
    18 & antagonistic\_jj & 0 & -1149 \\
    19 & slick\_jj & 0 & -1140 \\
    20 & harsh\_jj & 0 & -1135 \\
    21 & contrary\_jj & 0 & -1120 \\
    22 & hurly-burly\_jj & 0 & -1116 \\
    23 & rational\_jj & 0 & -1111 \\
    24 & icy\_jj & 0 & -1111 \\
    25 & barker & 0 & -1104 \\
    26 & deceptive\_jj & 0 & -1099 \\
    27 & clerkish\_jj & 0 & -1092 \\
    28 & prudent\_jj & 0 & -1056 \\
    29 & ornery\_jj & 0 & -1035 \\
    30 & reckless\_jj & 0 & -1030 \\
    1831 & worldly\_jj & 0 & 1190 \\
    1832 & slapdash\_jj & 0 & 1191 \\
    1833 & lifeless\_jj & 0 & 1198 \\
    1834 & acute\_jj & 0 & 1207 \\
    1835 & painstaking\_jj & 0 & 1217 \\
    1836 & obstructive\_jj & 0 & 1237 \\
    1837 & durable\_jj & 0 & 1248 \\
    1838 & grandiose\_jj & 0 & 1274 \\
    1839 & comforter & 0 & 1290 \\
    1840 & prayerful\_jj & 0 & 1308 \\
    1841 & charitable\_jj & 0 & 1316 \\
    1842 & unread\_jj & 0 & 1319 \\
    1843 & expensive\_jj & 0 & 1323 \\
    1844 & nonvolatile\_jj & 0 & 1338 \\
    1845 & tidy & 0 & 1348 \\
    1846 & malignant\_jj & 0 & 1375 \\
    1847 & unresponsive\_jj & 0 & 1387 \\
    1848 & hypersensitive\_jj & 0 & 1397 \\
    1849 & strenuous\_jj & 0 & 1402 \\
    1850 & untiring\_jj & 0 & 1421 \\
    1851 & tireless\_jj & 0 & 1431 \\
    1852 & overactive\_jj & 0 & 1577 \\
    1853 & philanthropic\_jj & 0 & 1587 \\
    1854 & lavish\_jj & 0 & 1617 \\
    1855 & valiant\_jj & 0 & 1692 \\
    1856 & irritable\_jj & 0 & 1703 \\
    1857 & vegetative\_jj & 0 & 1706 \\
    1858 & giving\_jj & 0 & 1889 \\
    1859 & aplastic\_jj & 0 & 2063 \\
    1860 & refractory\_jj & 0 & 2221 \\
    \hline
    \caption{Scores and rankings for most extreme 30 words in component \#26} \\
\end{longtable}
\begin{longtable}[!htbp]{| rlr@{.}l |}
    \hline
    \textbf{Rank} & \textbf{Word} & \multicolumn{2}{c|}{\textbf{Score}} \\
    \hline
    \endhead
    1 & censorial\_jj & 0 & -1625 \\
    2 & close-mouthed\_jj & 0 & -1614 \\
    3 & cagy\_jj & 0 & -1567 \\
    4 & unflagging\_jj & 0 & -1509 \\
    5 & treacherous\_jj & 0 & -1503 \\
    6 & inclement\_jj & 0 & -1500 \\
    7 & moral\_jj & 0 & -1449 \\
    8 & blase\_jj & 0 & -1396 \\
    9 & risqué\_jj & 0 & -1381 \\
    10 & insubordinate\_jj & 0 & -1328 \\
    11 & artistic\_jj & 0 & -1320 \\
    12 & chameleonlike\_jj & 0 & -1307 \\
    13 & balky\_jj & 0 & -1300 \\
    14 & vestal & 0 & -1298 \\
    15 & low-pressure\_jj & 0 & -1293 \\
    16 & cackler & 0 & -1283 \\
    17 & inalterable\_jj & 0 & -1282 \\
    18 & selfless\_jj & 0 & -1280 \\
    19 & immutable\_jj & 0 & -1258 \\
    20 & rugged\_jj & 0 & -1254 \\
    21 & levity & 0 & -1227 \\
    22 & overconfident\_jj & 0 & -1205 \\
    23 & risque\_jj & 0 & -1169 \\
    24 & verve & 0 & -1160 \\
    25 & careless\_jj & 0 & -1157 \\
    26 & supersensitive\_jj & 0 & -1149 \\
    27 & mathematical\_jj & 0 & -1147 \\
    28 & headstrong\_jj & 0 & -1143 \\
    29 & proud\_jj & 0 & -1139 \\
    30 & brash\_jj & 0 & -1138 \\
    1831 & vicious\_jj & 0 & 1075 \\
    1832 & acute\_jj & 0 & 1076 \\
    1833 & lukewarm\_jj & 0 & 1080 \\
    1834 & niggardly\_jj & 0 & 1083 \\
    1835 & stuck-up\_jj & 0 & 1084 \\
    1836 & vociferous\_jj & 0 & 1087 \\
    1837 & christlike\_jj & 0 & 1108 \\
    1838 & sadistic\_jj & 0 & 1131 \\
    1839 & brazen\_jj & 0 & 1145 \\
    1840 & gentle-hearted\_jj & 0 & 1159 \\
    1841 & moderate\_jj & 0 & 1175 \\
    1842 & bloodthirsty\_jj & 0 & 1192 \\
    1843 & mild\_jj & 0 & 1204 \\
    1844 & double-dealer & 0 & 1209 \\
    1845 & informal\_jj & 0 & 1217 \\
    1846 & ingenious\_jj & 0 & 1306 \\
    1847 & acquisitive\_jj & 0 & 1306 \\
    1848 & disobliging\_jj & 0 & 1308 \\
    1849 & vigorous\_jj & 0 & 1310 \\
    1850 & cursory\_jj & 0 & 1360 \\
    1851 & well-disposed\_jj & 0 & 1371 \\
    1852 & benign\_jj & 0 & 1374 \\
    1853 & insidious\_jj & 0 & 1377 \\
    1854 & murderous\_jj & 0 & 1379 \\
    1855 & ambitious\_jj & 0 & 1441 \\
    1856 & opportunist & 0 & 1445 \\
    1857 & incorrupt\_jj & 0 & 1450 \\
    1858 & audacious\_jj & 0 & 1453 \\
    1859 & aggressive\_jj & 0 & 1642 \\
    1860 & amnesic\_jj & 0 & 1870 \\
    \hline
    \caption{Scores and rankings for most extreme 30 words in component \#27} \\
\end{longtable}
\begin{longtable}[!htbp]{| rlr@{.}l |}
    \hline
    \textbf{Rank} & \textbf{Word} & \multicolumn{2}{c|}{\textbf{Score}} \\
    \hline
    \endhead
    1 & fossil & 0 & -1843 \\
    2 & cooperative\_jj & 0 & -1682 \\
    3 & backboned\_jj & 0 & -1634 \\
    4 & croaking & 0 & -1579 \\
    5 & babbler & 0 & -1545 \\
    6 & wild\_jj & 0 & -1527 \\
    7 & unresponsive\_jj & 0 & -1517 \\
    8 & comforter & 0 & -1513 \\
    9 & antagonistic\_jj & 0 & -1451 \\
    10 & hypochondriac\_jj & 0 & -1425 \\
    11 & thick-headed\_jj & 0 & -1415 \\
    12 & venomous\_jj & 0 & -1399 \\
    13 & thorny\_jj & 0 & -1399 \\
    14 & argumentative\_jj & 0 & -1352 \\
    15 & childlike\_jj & 0 & -1333 \\
    16 & gruff\_jj & 0 & -1309 \\
    17 & sentient\_jj & 0 & -1297 \\
    18 & mystical\_jj & 0 & -1283 \\
    19 & magpie & 0 & -1270 \\
    20 & adaptive\_jj & 0 & -1245 \\
    21 & constructive\_jj & 0 & -1233 \\
    22 & childish\_jj & 0 & -1227 \\
    23 & migratory\_jj & 0 & -1226 \\
    24 & prodigal & 0 & -1199 \\
    25 & testy\_jj & 0 & -1199 \\
    26 & lifeless\_jj & 0 & -1189 \\
    27 & rambunctious\_jj & 0 & -1182 \\
    28 & soft-shelled\_jj & 0 & -1174 \\
    29 & bullheaded\_jj & 0 & -1161 \\
    30 & defiant\_jj & 0 & -1137 \\
    1831 & provincial\_jj & 0 & 1147 \\
    1832 & unauthoritative\_jj & 0 & 1149 \\
    1833 & breezy\_jj & 0 & 1158 \\
    1834 & foolhardy\_jj & 0 & 1164 \\
    1835 & well-read\_jj & 0 & 1166 \\
    1836 & improvident\_jj & 0 & 1178 \\
    1837 & retributive\_jj & 0 & 1181 \\
    1838 & unsophisticated\_jj & 0 & 1186 \\
    1839 & faint-hearted\_jj & 0 & 1188 \\
    1840 & sunny\_jj & 0 & 1213 \\
    1841 & educated\_jj & 0 & 1213 \\
    1842 & loyal\_jj & 0 & 1216 \\
    1843 & lawless\_jj & 0 & 1226 \\
    1844 & treacherous\_jj & 0 & 1228 \\
    1845 & incorruptible\_jj & 0 & 1299 \\
    1846 & ladylike\_jj & 0 & 1367 \\
    1847 & snooty\_jj & 0 & 1378 \\
    1848 & extremist & 0 & 1395 \\
    1849 & ameliorative\_jj & 0 & 1396 \\
    1850 & undemanding\_jj & 0 & 1406 \\
    1851 & penetrative\_jj & 0 & 1419 \\
    1852 & squally\_jj & 0 & 1446 \\
    1853 & moderate\_jj & 0 & 1486 \\
    1854 & mild\_jj & 0 & 1509 \\
    1855 & brutal\_jj & 0 & 1511 \\
    1856 & militant\_jj & 0 & 1727 \\
    1857 & severe\_jj & 0 & 1777 \\
    1858 & incorrupt\_jj & 0 & 1807 \\
    1859 & insurgent\_jj & 0 & 1891 \\
    1860 & inclement\_jj & 0 & 1924 \\
    \hline
    \caption{Scores and rankings for most extreme 30 words in component \#28} \\
\end{longtable}
\begin{longtable}[!htbp]{| rlr@{.}l |}
    \hline
    \textbf{Rank} & \textbf{Word} & \multicolumn{2}{c|}{\textbf{Score}} \\
    \hline
    \endhead
    1 & forcible\_jj & 0 & -1606 \\
    2 & meteoric\_jj & 0 & -1568 \\
    3 & systematic\_jj & 0 & -1551 \\
    4 & exhaustive\_jj & 0 & -1524 \\
    5 & matronly\_jj & 0 & -1493 \\
    6 & public-minded\_jj & 0 & -1487 \\
    7 & methodical\_jj & 0 & -1482 \\
    8 & naysaying & 0 & -1437 \\
    9 & flowery\_jj & 0 & -1429 \\
    10 & chic\_jj & 0 & -1391 \\
    11 & vigorous\_jj & 0 & -1352 \\
    12 & forensic\_jj & 0 & -1343 \\
    13 & juvenile\_jj & 0 & -1297 \\
    14 & willful\_jj & 0 & -1293 \\
    15 & rancorous\_jj & 0 & -1260 \\
    16 & affirmative\_jj & 0 & -1258 \\
    17 & doer & 0 & -1237 \\
    18 & uneducated\_jj & 0 & -1222 \\
    19 & sluggish\_jj & 0 & -1191 \\
    20 & lewd\_jj & 0 & -1178 \\
    21 & stagnant\_jj & 0 & -1170 \\
    22 & vibrant\_jj & 0 & -1167 \\
    23 & uncivil\_jj & 0 & -1162 \\
    24 & painstaking\_jj & 0 & -1160 \\
    25 & educated\_jj & 0 & -1159 \\
    26 & incontrovertible\_jj & 0 & -1146 \\
    27 & unread\_jj & 0 & -1142 \\
    28 & dissenter & 0 & -1135 \\
    29 & tame\_jj & 0 & -1133 \\
    30 & lethargic\_jj & 0 & -1123 \\
    1831 & abrasive\_jj & 0 & 1055 \\
    1832 & fair-weather\_jj & 0 & 1061 \\
    1833 & tricky\_jj & 0 & 1083 \\
    1834 & conventional\_jj & 0 & 1102 \\
    1835 & obstructive\_jj & 0 & 1106 \\
    1836 & prodigal\_jj & 0 & 1108 \\
    1837 & ascetic\_jj & 0 & 1108 \\
    1838 & efficient\_jj & 0 & 1111 \\
    1839 & temperamental\_jj & 0 & 1127 \\
    1840 & stringent\_jj & 0 & 1140 \\
    1841 & wise\_jj & 0 & 1145 \\
    1842 & mild\_jj & 0 & 1149 \\
    1843 & inclement\_jj & 0 & 1153 \\
    1844 & unloving\_jj & 0 & 1154 \\
    1845 & platitudinous\_jj & 0 & 1182 \\
    1846 & friendly\_jj & 0 & 1189 \\
    1847 & misanthropic\_jj & 0 & 1204 \\
    1848 & irascible\_jj & 0 & 1207 \\
    1849 & sentimental\_jj & 0 & 1227 \\
    1850 & beneficial\_jj & 0 & 1227 \\
    1851 & cold\_jj & 0 & 1273 \\
    1852 & pleasant\_jj & 0 & 1319 \\
    1853 & high-hat & 0 & 1337 \\
    1854 & icy\_jj & 0 & 1367 \\
    1855 & helpful\_jj & 0 & 1397 \\
    1856 & severe\_jj & 0 & 1437 \\
    1857 & treacherous\_jj & 0 & 1527 \\
    1858 & expensive\_jj & 0 & 1586 \\
    1859 & sympathetic\_jj & 0 & 1605 \\
    1860 & generous\_jj & 0 & 1632 \\
    \hline
    \caption{Scores and rankings for most extreme 30 words in component \#29} \\
\end{longtable}
\begin{longtable}[!htbp]{| rlr@{.}l |}
    \hline
    \textbf{Rank} & \textbf{Word} & \multicolumn{2}{c|}{\textbf{Score}} \\
    \hline
    \endhead
    1 & covert\_jj & 0 & -1452 \\
    2 & nonvolatile\_jj & 0 & -1428 \\
    3 & analytical\_jj & 0 & -1389 \\
    4 & giving\_jj & 0 & -1362 \\
    5 & hypercritical\_jj & 0 & -1340 \\
    6 & innovative\_jj & 0 & -1313 \\
    7 & creative\_jj & 0 & -1312 \\
    8 & icy\_jj & 0 & -1296 \\
    9 & unostentatious\_jj & 0 & -1294 \\
    10 & dodgy\_jj & 0 & -1290 \\
    11 & clairvoyant & 0 & -1261 \\
    12 & intuitive\_jj & 0 & -1248 \\
    13 & untiring\_jj & 0 & -1202 \\
    14 & lackadaisical\_jj & 0 & -1197 \\
    15 & unhelpful\_jj & 0 & -1191 \\
    16 & frosty\_jj & 0 & -1189 \\
    17 & disdainful\_jj & 0 & -1188 \\
    18 & shortsighted\_jj & 0 & -1181 \\
    19 & negative\_jj & 0 & -1180 \\
    20 & militant\_jj & 0 & -1171 \\
    21 & adaptive\_jj & 0 & -1155 \\
    22 & self-defensive\_jj & 0 & -1131 \\
    23 & helpful\_jj & 0 & -1111 \\
    24 & vixenish\_jj & 0 & -1107 \\
    25 & unconventional\_jj & 0 & -1104 \\
    26 & waggish\_jj & 0 & -1104 \\
    27 & kind\_jj & 0 & -1101 \\
    28 & blustery\_jj & 0 & -1091 \\
    29 & truckling & 0 & -1091 \\
    30 & obsessive & 0 & -1079 \\
    1831 & rowdy\_jj & 0 & 1040 \\
    1832 & loud\_jj & 0 & 1043 \\
    1833 & clear-cut\_jj & 0 & 1048 \\
    1834 & unresponsive\_jj & 0 & 1055 \\
    1835 & pliant\_jj & 0 & 1056 \\
    1836 & versatile\_jj & 0 & 1073 \\
    1837 & controlled\_jj & 0 & 1076 \\
    1838 & solemn\_jj & 0 & 1076 \\
    1839 & stringent\_jj & 0 & 1080 \\
    1840 & profligate & 0 & 1081 \\
    1841 & tractable\_jj & 0 & 1082 \\
    1842 & reliable\_jj & 0 & 1100 \\
    1843 & reckless\_jj & 0 & 1106 \\
    1844 & durable\_jj & 0 & 1112 \\
    1845 & improvident\_jj & 0 & 1137 \\
    1846 & wanton\_jj & 0 & 1150 \\
    1847 & gluttonous\_jj & 0 & 1171 \\
    1848 & virile\_jj & 0 & 1184 \\
    1849 & silent\_jj & 0 & 1184 \\
    1850 & innocuous\_jj & 0 & 1189 \\
    1851 & decorous\_jj & 0 & 1219 \\
    1852 & explicit\_jj & 0 & 1279 \\
    1853 & inconstant\_jj & 0 & 1323 \\
    1854 & sentient\_jj & 0 & 1327 \\
    1855 & dissenter & 0 & 1346 \\
    1856 & reasonable\_jj & 0 & 1347 \\
    1857 & self-respecting\_jj & 0 & 1375 \\
    1858 & just\_jj & 0 & 1416 \\
    1859 & faint-hearted\_jj & 0 & 1494 \\
    1860 & disorderly\_jj & 0 & 1509 \\
    \hline
    \caption{Scores and rankings for most extreme 30 words in component \#30} \\
\end{longtable}
\begin{longtable}[!htbp]{| rlr@{.}l |}
    \hline
    \textbf{Rank} & \textbf{Word} & \multicolumn{2}{c|}{\textbf{Score}} \\
    \hline
    \endhead
    1 & noncompliant\_jj & 0 & -1556 \\
    2 & flammable\_jj & 0 & -1517 \\
    3 & scrappy\_jj & 0 & -1473 \\
    4 & confident\_jj & 0 & -1445 \\
    5 & provident\_jj & 0 & -1438 \\
    6 & proud\_jj & 0 & -1401 \\
    7 & dauber & 0 & -1400 \\
    8 & one-sided\_jj & 0 & -1393 \\
    9 & kind\_jj & 0 & -1362 \\
    10 & yellow\_jj & 0 & -1272 \\
    11 & tight-lipped\_jj & 0 & -1235 \\
    12 & stout-hearted\_jj & 0 & -1231 \\
    13 & fan & 0 & -1228 \\
    14 & innocuous\_jj & 0 & -1206 \\
    15 & impudent\_jj & 0 & -1198 \\
    16 & friendly\_jj & 0 & -1182 \\
    17 & jovial\_jj & 0 & -1179 \\
    18 & bacchanalian\_jj & 0 & -1162 \\
    19 & spasmodic\_jj & 0 & -1153 \\
    20 & invincible\_jj & 0 & -1131 \\
    21 & upstart\_jj & 0 & -1109 \\
    22 & loyal\_jj & 0 & -1099 \\
    23 & quick-tempered\_jj & 0 & -1083 \\
    24 & unsophisticated\_jj & 0 & -1081 \\
    25 & refractory\_jj & 0 & -1077 \\
    26 & gregarious\_jj & 0 & -1073 \\
    27 & double-dealer & 0 & -1072 \\
    28 & lion-hearted\_jj & 0 & -1044 \\
    29 & many-sided\_jj & 0 & -1039 \\
    30 & hindsight & 0 & -1038 \\
    1831 & lenient\_jj & 0 & 1097 \\
    1832 & unauthoritative\_jj & 0 & 1098 \\
    1833 & crank & 0 & 1110 \\
    1834 & gainful\_jj & 0 & 1110 \\
    1835 & low-pressure\_jj & 0 & 1120 \\
    1836 & scold & 0 & 1130 \\
    1837 & rigorous\_jj & 0 & 1133 \\
    1838 & stringent\_jj & 0 & 1136 \\
    1839 & humanitarian\_jj & 0 & 1142 \\
    1840 & croaking & 0 & 1144 \\
    1841 & sultry\_jj & 0 & 1155 \\
    1842 & unrelenting\_jj & 0 & 1164 \\
    1843 & treacherous\_jj & 0 & 1190 \\
    1844 & cyclonic\_jj & 0 & 1202 \\
    1845 & exhortative\_jj & 0 & 1219 \\
    1846 & weepy\_jj & 0 & 1227 \\
    1847 & illiterate\_jj & 0 & 1231 \\
    1848 & tough-minded\_jj & 0 & 1232 \\
    1849 & stern\_jj & 0 & 1242 \\
    1850 & expansive\_jj & 0 & 1263 \\
    1851 & severe\_jj & 0 & 1342 \\
    1852 & temperate\_jj & 0 & 1362 \\
    1853 & cold\_jj & 0 & 1386 \\
    1854 & sophistic\_jj & 0 & 1401 \\
    1855 & sunny\_jj & 0 & 1450 \\
    1856 & blustery\_jj & 0 & 1478 \\
    1857 & icy\_jj & 0 & 1548 \\
    1858 & inclement\_jj & 0 & 1609 \\
    1859 & harsh\_jj & 0 & 1648 \\
    1860 & drastic\_jj & 0 & 1791 \\
    \hline
    \caption{Scores and rankings for most extreme 30 words in component \#31} \\
\end{longtable}
\begin{longtable}[!htbp]{| rlr@{.}l |}
    \hline
    \textbf{Rank} & \textbf{Word} & \multicolumn{2}{c|}{\textbf{Score}} \\
    \hline
    \endhead
    1 & obstructive\_jj & 0 & -1903 \\
    2 & butter-fingered\_jj & 0 & -1622 \\
    3 & obsessive & 0 & -1497 \\
    4 & factious\_jj & 0 & -1471 \\
    5 & wildcat\_jj & 0 & -1462 \\
    6 & aplastic\_jj & 0 & -1422 \\
    7 & variant\_jj & 0 & -1420 \\
    8 & bilious\_jj & 0 & -1379 \\
    9 & flighty\_jj & 0 & -1305 \\
    10 & refractory\_jj & 0 & -1293 \\
    11 & overactive\_jj & 0 & -1280 \\
    12 & unscrupulous\_jj & 0 & -1262 \\
    13 & unfair\_jj & 0 & -1255 \\
    14 & thorny\_jj & 0 & -1235 \\
    15 & backboned\_jj & 0 & -1233 \\
    16 & compulsive & 0 & -1221 \\
    17 & bovine\_jj & 0 & -1207 \\
    18 & progressive\_jj & 0 & -1193 \\
    19 & cognitive\_jj & 0 & -1192 \\
    20 & malignant\_jj & 0 & -1185 \\
    21 & fly-by-night\_jj & 0 & -1179 \\
    22 & frivolous\_jj & 0 & -1178 \\
    23 & feline\_jj & 0 & -1159 \\
    24 & irritable\_jj & 0 & -1136 \\
    25 & butterfly & 0 & -1128 \\
    26 & truckling & 0 & -1119 \\
    27 & intractable\_jj & 0 & -1098 \\
    28 & inquisitive\_jj & 0 & -1079 \\
    29 & ill-tempered\_jj & 0 & -1069 \\
    30 & deceptive\_jj & 0 & -1059 \\
    1831 & carnal\_jj & 0 & 1030 \\
    1832 & violent\_jj & 0 & 1046 \\
    1833 & unstable\_jj & 0 & 1051 \\
    1834 & asocial\_jj & 0 & 1054 \\
    1835 & carouser & 0 & 1056 \\
    1836 & flammable\_jj & 0 & 1058 \\
    1837 & visionary\_jj & 0 & 1063 \\
    1838 & caustic\_jj & 0 & 1072 \\
    1839 & indefinite\_jj & 0 & 1120 \\
    1840 & unobservant\_jj & 0 & 1122 \\
    1841 & stable\_jj & 0 & 1132 \\
    1842 & destructive\_jj & 0 & 1133 \\
    1843 & unemotional\_jj & 0 & 1153 \\
    1844 & snoopy\_jj & 0 & 1158 \\
    1845 & unfeminine\_jj & 0 & 1159 \\
    1846 & belligerent\_jj & 0 & 1161 \\
    1847 & libertine\_jj & 0 & 1174 \\
    1848 & lewd\_jj & 0 & 1181 \\
    1849 & stern\_jj & 0 & 1195 \\
    1850 & blunt\_jj & 0 & 1203 \\
    1851 & hayseed & 0 & 1206 \\
    1852 & slovenly\_jj & 0 & 1211 \\
    1853 & jack-of-all-trades & 0 & 1214 \\
    1854 & admonitory\_jj & 0 & 1235 \\
    1855 & reckless\_jj & 0 & 1275 \\
    1856 & explosive\_jj & 0 & 1308 \\
    1857 & unreserved\_jj & 0 & 1395 \\
    1858 & lascivious\_jj & 0 & 1443 \\
    1859 & disorderly\_jj & 0 & 1911 \\
    1860 & tattletale & 0 & 1956 \\
    \hline
    \caption{Scores and rankings for most extreme 30 words in component \#32} \\
\end{longtable}
\begin{longtable}[!htbp]{| rlr@{.}l |}
    \hline
    \textbf{Rank} & \textbf{Word} & \multicolumn{2}{c|}{\textbf{Score}} \\
    \hline
    \endhead
    1 & squally\_jj & 0 & -1788 \\
    2 & reflective\_jj & 0 & -1453 \\
    3 & inclement\_jj & 0 & -1411 \\
    4 & irresolute\_jj & 0 & -1366 \\
    5 & icy\_jj & 0 & -1360 \\
    6 & cyclonic\_jj & 0 & -1274 \\
    7 & guileful\_jj & 0 & -1250 \\
    8 & reckless\_jj & 0 & -1248 \\
    9 & crank & 0 & -1230 \\
    10 & wishful\_jj & 0 & -1210 \\
    11 & low-pressure\_jj & 0 & -1174 \\
    12 & acrid\_jj & 0 & -1163 \\
    13 & wanton\_jj & 0 & -1158 \\
    14 & flammable\_jj & 0 & -1145 \\
    15 & philosophical\_jj & 0 & -1115 \\
    16 & opportunist & 0 & -1110 \\
    17 & objector & 0 & -1110 \\
    18 & impartial\_jj & 0 & -1106 \\
    19 & laggard\_jj & 0 & -1102 \\
    20 & unflagging\_jj & 0 & -1098 \\
    21 & disorderly\_jj & 0 & -1089 \\
    22 & fretful\_jj & 0 & -1087 \\
    23 & masochistic\_jj & 0 & -1085 \\
    24 & stormy\_jj & 0 & -1069 \\
    25 & dogmatic\_jj & 0 & -1066 \\
    26 & cogent\_jj & 0 & -1064 \\
    27 & breezy\_jj & 0 & -1055 \\
    28 & crowder & 0 & -1054 \\
    29 & antisocial\_jj & 0 & -1050 \\
    30 & prayerful\_jj & 0 & -1044 \\
    1831 & migratory\_jj & 0 & 1026 \\
    1832 & lax\_jj & 0 & 1041 \\
    1833 & derisive\_jj & 0 & 1044 \\
    1834 & uncultivated\_jj & 0 & 1048 \\
    1835 & inexpressive\_jj & 0 & 1064 \\
    1836 & soft-shelled\_jj & 0 & 1068 \\
    1837 & archaic\_jj & 0 & 1075 \\
    1838 & chameleonlike\_jj & 0 & 1110 \\
    1839 & maternal\_jj & 0 & 1112 \\
    1840 & precise\_jj & 0 & 1123 \\
    1841 & exact\_jj & 0 & 1127 \\
    1842 & variant\_jj & 0 & 1129 \\
    1843 & fossil & 0 & 1132 \\
    1844 & censorial\_jj & 0 & 1141 \\
    1845 & cultivated\_jj & 0 & 1149 \\
    1846 & overtrained\_jj & 0 & 1170 \\
    1847 & meteoric\_jj & 0 & 1171 \\
    1848 & cerebral\_jj & 0 & 1214 \\
    1849 & harsh\_jj & 0 & 1298 \\
    1850 & stodgy\_jj & 0 & 1304 \\
    1851 & feline\_jj & 0 & 1335 \\
    1852 & migrant & 0 & 1340 \\
    1853 & gourmand & 0 & 1341 \\
    1854 & derogatory\_jj & 0 & 1346 \\
    1855 & humble\_jj & 0 & 1381 \\
    1856 & reclusive\_jj & 0 & 1449 \\
    1857 & migrant\_jj & 0 & 1632 \\
    1858 & wild\_jj & 0 & 1663 \\
    1859 & cruel\_jj & 0 & 1706 \\
    1860 & nomadic\_jj & 0 & 1815 \\
    \hline
    \caption{Scores and rankings for most extreme 30 words in component \#33} \\
\end{longtable}
\begin{longtable}[!htbp]{| rlr@{.}l |}
    \hline
    \textbf{Rank} & \textbf{Word} & \multicolumn{2}{c|}{\textbf{Score}} \\
    \hline
    \endhead
    1 & unconventional\_jj & 0 & -1643 \\
    2 & ameliorative\_jj & 0 & -1570 \\
    3 & brave\_jj & 0 & -1443 \\
    4 & impractical\_jj & 0 & -1438 \\
    5 & variable\_jj & 0 & -1436 \\
    6 & drastic\_jj & 0 & -1317 \\
    7 & munificent\_jj & 0 & -1222 \\
    8 & prescient\_jj & 0 & -1220 \\
    9 & croaking & 0 & -1175 \\
    10 & passionate\_jj & 0 & -1171 \\
    11 & vociferous\_jj & 0 & -1169 \\
    12 & courageous\_jj & 0 & -1166 \\
    13 & flammable\_jj & 0 & -1159 \\
    14 & inexorable\_jj & 0 & -1156 \\
    15 & disruptive\_jj & 0 & -1142 \\
    16 & scornful\_jj & 0 & -1138 \\
    17 & abrupt\_jj & 0 & -1121 \\
    18 & unmovable\_jj & 0 & -1120 \\
    19 & excitable\_jj & 0 & -1118 \\
    20 & niggardly\_jj & 0 & -1105 \\
    21 & loud\_jj & 0 & -1080 \\
    22 & nonconforming\_jj & 0 & -1054 \\
    23 & bookish\_jj & 0 & -1054 \\
    24 & meteoric\_jj & 0 & -1045 \\
    25 & foresighted\_jj & 0 & -1039 \\
    26 & harsh\_jj & 0 & -1027 \\
    27 & foolhardy\_jj & 0 & -1022 \\
    28 & adulatory\_jj & 0 & -1018 \\
    29 & inefficient\_jj & 0 & -1014 \\
    30 & strong-minded\_jj & 0 & -1005 \\
    1831 & precise\_jj & 0 & 1000 \\
    1832 & unbiased\_jj & 0 & 1025 \\
    1833 & double-dealer & 0 & 1044 \\
    1834 & realistic\_jj & 0 & 1045 \\
    1835 & political\_jj & 0 & 1048 \\
    1836 & vestal & 0 & 1055 \\
    1837 & confident\_jj & 0 & 1075 \\
    1838 & acute\_jj & 0 & 1105 \\
    1839 & assured\_jj & 0 & 1109 \\
    1840 & comedian & 0 & 1113 \\
    1841 & detached\_jj & 0 & 1129 \\
    1842 & objective\_jj & 0 & 1141 \\
    1843 & obstructive\_jj & 0 & 1156 \\
    1844 & luxurious\_jj & 0 & 1165 \\
    1845 & clannish\_jj & 0 & 1168 \\
    1846 & gourmet & 0 & 1188 \\
    1847 & provincial\_jj & 0 & 1190 \\
    1848 & proper\_jj & 0 & 1202 \\
    1849 & affected\_jj & 0 & 1262 \\
    1850 & driftless\_jj & 0 & 1269 \\
    1851 & compulsive & 0 & 1280 \\
    1852 & exclusive\_jj & 0 & 1290 \\
    1853 & militant\_jj & 0 & 1308 \\
    1854 & compliant\_jj & 0 & 1321 \\
    1855 & clubby\_jj & 0 & 1322 \\
    1856 & aplomb & 0 & 1426 \\
    1857 & impartial\_jj & 0 & 1502 \\
    1858 & metropolitan\_jj & 0 & 1521 \\
    1859 & lawless\_jj & 0 & 1599 \\
    1860 & neighborly\_jj & 0 & 1696 \\
    \hline
    \caption{Scores and rankings for most extreme 30 words in component \#34} \\
\end{longtable}
\begin{longtable}[!htbp]{| rlr@{.}l |}
    \hline
    \textbf{Rank} & \textbf{Word} & \multicolumn{2}{c|}{\textbf{Score}} \\
    \hline
    \endhead
    1 & contradictory\_jj & 0 & -1550 \\
    2 & inaccurate\_jj & 0 & -1530 \\
    3 & brisk\_jj & 0 & -1444 \\
    4 & ascetic & 0 & -1397 \\
    5 & exact\_jj & 0 & -1388 \\
    6 & untrustworthy\_jj & 0 & -1271 \\
    7 & elusive\_jj & 0 & -1262 \\
    8 & accurate\_jj & 0 & -1259 \\
    9 & fair-weather\_jj & 0 & -1171 \\
    10 & undependable\_jj & 0 & -1161 \\
    11 & chameleonlike\_jj & 0 & -1150 \\
    12 & frenetic\_jj & 0 & -1135 \\
    13 & modifiable\_jj & 0 & -1124 \\
    14 & abstinent\_jj & 0 & -1096 \\
    15 & unreliable\_jj & 0 & -1090 \\
    16 & constant\_jj & 0 & -1076 \\
    17 & tireless\_jj & 0 & -1075 \\
    18 & peaceful\_jj & 0 & -1070 \\
    19 & saintly\_jj & 0 & -1069 \\
    20 & manipulative\_jj & 0 & -1061 \\
    21 & sedentary\_jj & 0 & -1054 \\
    22 & unbiased\_jj & 0 & -1049 \\
    23 & fraudulent\_jj & 0 & -1045 \\
    24 & apolitical\_jj & 0 & -1043 \\
    25 & beneficent\_jj & 0 & -1040 \\
    26 & educable\_jj & 0 & -1038 \\
    27 & bloodless\_jj & 0 & -1027 \\
    28 & compulsive\_jj & 0 & -1026 \\
    29 & unethical\_jj & 0 & -1019 \\
    30 & sharp-tongued\_jj & 0 & -1014 \\
    1831 & civilized\_jj & 0 & 1106 \\
    1832 & severe\_jj & 0 & 1109 \\
    1833 & blunderbuss & 0 & 1122 \\
    1834 & disobliging\_jj & 0 & 1123 \\
    1835 & discerning\_jj & 0 & 1129 \\
    1836 & frank\_jj & 0 & 1144 \\
    1837 & stringent\_jj & 0 & 1144 \\
    1838 & rigid\_jj & 0 & 1145 \\
    1839 & pervious\_jj & 0 & 1146 \\
    1840 & thorny\_jj & 0 & 1181 \\
    1841 & intellectual & 0 & 1185 \\
    1842 & detached\_jj & 0 & 1205 \\
    1843 & cloistered\_jj & 0 & 1216 \\
    1844 & moral\_jj & 0 & 1222 \\
    1845 & flammable\_jj & 0 & 1236 \\
    1846 & intractable\_jj & 0 & 1248 \\
    1847 & geezer & 0 & 1317 \\
    1848 & derelict\_jj & 0 & 1357 \\
    1849 & protective\_jj & 0 & 1359 \\
    1850 & intellectual\_jj & 0 & 1422 \\
    1851 & profound\_jj & 0 & 1432 \\
    1852 & migrant\_jj & 0 & 1481 \\
    1853 & plastic\_jj & 0 & 1551 \\
    1854 & immovable\_jj & 0 & 1574 \\
    1855 & serious\_jj & 0 & 1602 \\
    1856 & litigious\_jj & 0 & 1613 \\
    1857 & abandoned\_jj & 0 & 1653 \\
    1858 & barbed\_jj & 0 & 1834 \\
    1859 & deep\_jj & 0 & 1866 \\
    1860 & corrosive\_jj & 0 & 1929 \\
    \hline
    \caption{Scores and rankings for most extreme 30 words in component \#35} \\
\end{longtable}
\begin{longtable}[!htbp]{| rlr@{.}l |}
    \hline
    \textbf{Rank} & \textbf{Word} & \multicolumn{2}{c|}{\textbf{Score}} \\
    \hline
    \endhead
    1 & ungentle\_jj & 0 & -1544 \\
    2 & artistic\_jj & 0 & -1532 \\
    3 & unfaithful\_jj & 0 & -1404 \\
    4 & cyclonic\_jj & 0 & -1402 \\
    5 & lavish\_jj & 0 & -1390 \\
    6 & variable\_jj & 0 & -1364 \\
    7 & amorous\_jj & 0 & -1334 \\
    8 & aristocratic\_jj & 0 & -1327 \\
    9 & intricate\_jj & 0 & -1287 \\
    10 & hot-blooded\_jj & 0 & -1264 \\
    11 & jealous\_jj & 0 & -1237 \\
    12 & magnetic\_jj & 0 & -1228 \\
    13 & congratulatory\_jj & 0 & -1210 \\
    14 & untruthful\_jj & 0 & -1210 \\
    15 & selfish\_jj & 0 & -1169 \\
    16 & adjustable\_jj & 0 & -1165 \\
    17 & unanchored\_jj & 0 & -1157 \\
    18 & assured\_jj & 0 & -1148 \\
    19 & warm\_jj & 0 & -1131 \\
    20 & adulterous\_jj & 0 & -1124 \\
    21 & tempestuous\_jj & 0 & -1124 \\
    22 & frosty\_jj & 0 & -1100 \\
    23 & musical\_jj & 0 & -1094 \\
    24 & outgoing\_jj & 0 & -1075 \\
    25 & greedy\_jj & 0 & -1075 \\
    26 & stormy\_jj & 0 & -1069 \\
    27 & servile\_jj & 0 & -1063 \\
    28 & theoretic\_jj & 0 & -1045 \\
    29 & unsystematic\_jj & 0 & -1039 \\
    30 & lukewarm\_jj & 0 & -1020 \\
    1831 & responsible\_jj & 0 & 958 \\
    1832 & thorny\_jj & 0 & 966 \\
    1833 & unfair\_jj & 0 & 968 \\
    1834 & teachable\_jj & 0 & 975 \\
    1835 & dispiriting\_jj & 0 & 979 \\
    1836 & immoderate\_jj & 0 & 983 \\
    1837 & meditative\_jj & 0 & 1000 \\
    1838 & hardened\_jj & 0 & 1003 \\
    1839 & harum-scarum\_jj & 0 & 1004 \\
    1840 & thoroughgoing\_jj & 0 & 1007 \\
    1841 & dolorous\_jj & 0 & 1016 \\
    1842 & undisguised\_jj & 0 & 1022 \\
    1843 & hair-trigger\_jj & 0 & 1034 \\
    1844 & falstaffian\_jj & 0 & 1054 \\
    1845 & bristly\_jj & 0 & 1074 \\
    1846 & close-mouthed\_jj & 0 & 1081 \\
    1847 & double-dealer & 0 & 1082 \\
    1848 & consolatory\_jj & 0 & 1111 \\
    1849 & smutty\_jj & 0 & 1127 \\
    1850 & obsessive & 0 & 1162 \\
    1851 & masochistic\_jj & 0 & 1167 \\
    1852 & compulsive & 0 & 1185 \\
    1853 & instructive\_jj & 0 & 1218 \\
    1854 & inhumane\_jj & 0 & 1231 \\
    1855 & levity & 0 & 1247 \\
    1856 & impractical\_jj & 0 & 1269 \\
    1857 & serious\_jj & 0 & 1304 \\
    1858 & trembly\_jj & 0 & 1368 \\
    1859 & intractable\_jj & 0 & 1525 \\
    1860 & dare-devil & 0 & 1571 \\
    \hline
    \caption{Scores and rankings for most extreme 30 words in component \#36} \\
\end{longtable}
\begin{longtable}[!htbp]{| rlr@{.}l |}
    \hline
    \textbf{Rank} & \textbf{Word} & \multicolumn{2}{c|}{\textbf{Score}} \\
    \hline
    \endhead
    1 & upstart\_jj & 0 & -1593 \\
    2 & hard-shell\_jj & 0 & -1575 \\
    3 & valiant\_jj & 0 & -1525 \\
    4 & exhortative\_jj & 0 & -1452 \\
    5 & retributive\_jj & 0 & -1383 \\
    6 & refractory\_jj & 0 & -1339 \\
    7 & earnest\_jj & 0 & -1309 \\
    8 & fly-by-night\_jj & 0 & -1216 \\
    9 & cautionary\_jj & 0 & -1175 \\
    10 & indestructible\_jj & 0 & -1171 \\
    11 & pacifistic\_jj & 0 & -1140 \\
    12 & acid\_jj & 0 & -1122 \\
    13 & aboveboard\_jj & 0 & -1117 \\
    14 & high-minded\_jj & 0 & -1100 \\
    15 & plastic\_jj & 0 & -1090 \\
    16 & bold\_jj & 0 & -1072 \\
    17 & hard-nosed\_jj & 0 & -1067 \\
    18 & sharp-witted\_jj & 0 & -1063 \\
    19 & withdrawn\_jj & 0 & -1053 \\
    20 & unshakable\_jj & 0 & -1053 \\
    21 & gun-shy\_jj & 0 & -1047 \\
    22 & backboned\_jj & 0 & -1040 \\
    23 & hardened\_jj & 0 & -1031 \\
    24 & babyish\_jj & 0 & -1028 \\
    25 & ethical\_jj & 0 & -1025 \\
    26 & innovative\_jj & 0 & -1015 \\
    27 & old-fashioned\_jj & 0 & -1006 \\
    28 & unrelenting\_jj & 0 & -985 \\
    29 & smart-alecky\_jj & 0 & -980 \\
    30 & well-spoken\_jj & 0 & -975 \\
    1831 & volatile\_jj & 0 & 988 \\
    1832 & flint-hearted\_jj & 0 & 996 \\
    1833 & precipitous\_jj & 0 & 998 \\
    1834 & destructive\_jj & 0 & 1013 \\
    1835 & dynamic\_jj & 0 & 1027 \\
    1836 & eruptive\_jj & 0 & 1033 \\
    1837 & overhasty\_jj & 0 & 1033 \\
    1838 & piercing\_jj & 0 & 1037 \\
    1839 & cognizant\_jj & 0 & 1038 \\
    1840 & thunderous\_jj & 0 & 1039 \\
    1841 & leisurely\_jj & 0 & 1048 \\
    1842 & hurly-burly\_jj & 0 & 1049 \\
    1843 & fan & 0 & 1074 \\
    1844 & show-off & 0 & 1083 \\
    1845 & languid\_jj & 0 & 1117 \\
    1846 & autocratic\_jj & 0 & 1146 \\
    1847 & whirlwind & 0 & 1154 \\
    1848 & firebrand & 0 & 1156 \\
    1849 & ruminative\_jj & 0 & 1184 \\
    1850 & baleful\_jj & 0 & 1194 \\
    1851 & remiss\_jj & 0 & 1205 \\
    1852 & volcanic\_jj & 0 & 1248 \\
    1853 & majestic\_jj & 0 & 1250 \\
    1854 & secretive\_jj & 0 & 1293 \\
    1855 & jealous\_jj & 0 & 1324 \\
    1856 & meditative\_jj & 0 & 1385 \\
    1857 & dictatorial\_jj & 0 & 1401 \\
    1858 & connoisseur & 0 & 1409 \\
    1859 & proud\_jj & 0 & 1470 \\
    1860 & devotee & 0 & 1616 \\
    \hline
    \caption{Scores and rankings for most extreme 30 words in component \#37} \\
\end{longtable}
\begin{longtable}[!htbp]{| rlr@{.}l |}
    \hline
    \textbf{Rank} & \textbf{Word} & \multicolumn{2}{c|}{\textbf{Score}} \\
    \hline
    \endhead
    1 & unteachable\_jj & 0 & -1397 \\
    2 & persuasive\_jj & 0 & -1349 \\
    3 & impervious\_jj & 0 & -1346 \\
    4 & mollycoddle & 0 & -1308 \\
    5 & verbal\_jj & 0 & -1290 \\
    6 & frosty\_jj & 0 & -1270 \\
    7 & tricky\_jj & 0 & -1249 \\
    8 & persistent\_jj & 0 & -1239 \\
    9 & undefeatable\_jj & 0 & -1219 \\
    10 & recalcitrant\_jj & 0 & -1212 \\
    11 & martial\_jj & 0 & -1210 \\
    12 & warm\_jj & 0 & -1202 \\
    13 & obscene\_jj & 0 & -1179 \\
    14 & incontrovertible\_jj & 0 & -1163 \\
    15 & other-directed\_jj & 0 & -1132 \\
    16 & eclectic & 0 & -1130 \\
    17 & musical\_jj & 0 & -1120 \\
    18 & lewd\_jj & 0 & -1085 \\
    19 & loud\_jj & 0 & -1072 \\
    20 & brute\_jj & 0 & -1070 \\
    21 & fickle\_jj & 0 & -1063 \\
    22 & invariable\_jj & 0 & -1053 \\
    23 & coercive\_jj & 0 & -1046 \\
    24 & cognitive\_jj & 0 & -1043 \\
    25 & cold\_jj & 0 & -1043 \\
    26 & unshakable\_jj & 0 & -1030 \\
    27 & helpful\_jj & 0 & -1015 \\
    28 & vestal\_jj & 0 & -1012 \\
    29 & oratorical\_jj & 0 & -990 \\
    30 & clubbable\_jj & 0 & -990 \\
    1831 & indiscreet\_jj & 0 & 966 \\
    1832 & alarmist & 0 & 976 \\
    1833 & discretionary\_jj & 0 & 987 \\
    1834 & exact\_jj & 0 & 990 \\
    1835 & self-indulgent\_jj & 0 & 996 \\
    1836 & unvarying\_jj & 0 & 1005 \\
    1837 & backhanded\_jj & 0 & 1007 \\
    1838 & closed-minded\_jj & 0 & 1010 \\
    1839 & nonhostile\_jj & 0 & 1025 \\
    1840 & fraternal\_jj & 0 & 1028 \\
    1841 & refractory\_jj & 0 & 1030 \\
    1842 & munificent\_jj & 0 & 1039 \\
    1843 & considerate\_jj & 0 & 1053 \\
    1844 & bubbler & 0 & 1053 \\
    1845 & circuitous\_jj & 0 & 1066 \\
    1846 & detached\_jj & 0 & 1067 \\
    1847 & daredevil & 0 & 1072 \\
    1848 & impudent\_jj & 0 & 1085 \\
    1849 & malcontent & 0 & 1101 \\
    1850 & self-controlled\_jj & 0 & 1109 \\
    1851 & variable\_jj & 0 & 1124 \\
    1852 & unguarded\_jj & 0 & 1131 \\
    1853 & volcanic\_jj & 0 & 1147 \\
    1854 & calculable\_jj & 0 & 1182 \\
    1855 & self-centered\_jj & 0 & 1188 \\
    1856 & penny-wise\_jj & 0 & 1202 \\
    1857 & zealous\_jj & 0 & 1257 \\
    1858 & saccharine\_jj & 0 & 1330 \\
    1859 & preachy\_jj & 0 & 1366 \\
    1860 & eruptive\_jj & 0 & 1538 \\
    \hline
    \caption{Scores and rankings for most extreme 30 words in component \#38} \\
\end{longtable}
\begin{longtable}[!htbp]{| rlr@{.}l |}
    \hline
    \textbf{Rank} & \textbf{Word} & \multicolumn{2}{c|}{\textbf{Score}} \\
    \hline
    \endhead
    1 & loud\_jj & 0 & -1623 \\
    2 & conclusive\_jj & 0 & -1600 \\
    3 & definite\_jj & 0 & -1553 \\
    4 & die-hard\_jj & 0 & -1461 \\
    5 & clear-cut\_jj & 0 & -1457 \\
    6 & withdrawn\_jj & 0 & -1407 \\
    7 & prodigal\_jj & 0 & -1332 \\
    8 & vociferous\_jj & 0 & -1325 \\
    9 & many-sided\_jj & 0 & -1250 \\
    10 & thunderous\_jj & 0 & -1236 \\
    11 & objective\_jj & 0 & -1228 \\
    12 & excitable\_jj & 0 & -1169 \\
    13 & chesty\_jj & 0 & -1165 \\
    14 & overhasty\_jj & 0 & -1136 \\
    15 & nosey\_jj & 0 & -1130 \\
    16 & strong-minded\_jj & 0 & -1122 \\
    17 & rancorous\_jj & 0 & -1121 \\
    18 & barker & 0 & -1121 \\
    19 & discerning\_jj & 0 & -1119 \\
    20 & boisterous\_jj & 0 & -1110 \\
    21 & unschooled\_jj & 0 & -1109 \\
    22 & unruly\_jj & 0 & -1105 \\
    23 & stay-at-home\_jj & 0 & -1098 \\
    24 & unbiased\_jj & 0 & -1081 \\
    25 & fly-by-night\_jj & 0 & -1057 \\
    26 & bright-eyed\_jj & 0 & -1047 \\
    27 & clear-headed\_jj & 0 & -1041 \\
    28 & faint-hearted\_jj & 0 & -1037 \\
    29 & plastic\_jj & 0 & -1033 \\
    30 & cassandra & 0 & -1008 \\
    1831 & digressive\_jj & 0 & 960 \\
    1832 & exhortative\_jj & 0 & 966 \\
    1833 & oblique\_jj & 0 & 969 \\
    1834 & secretive\_jj & 0 & 970 \\
    1835 & coy\_jj & 0 & 972 \\
    1836 & frugal\_jj & 0 & 973 \\
    1837 & stingy\_jj & 0 & 978 \\
    1838 & elusive\_jj & 0 & 979 \\
    1839 & cold\_jj & 0 & 981 \\
    1840 & careful\_jj & 0 & 989 \\
    1841 & quiet\_jj & 0 & 994 \\
    1842 & compulsive\_jj & 0 & 1008 \\
    1843 & hard-boiled\_jj & 0 & 1030 \\
    1844 & ultraconservative\_jj & 0 & 1033 \\
    1845 & discretionary\_jj & 0 & 1038 \\
    1846 & strenuous\_jj & 0 & 1065 \\
    1847 & stable\_jj & 0 & 1079 \\
    1848 & strict\_jj & 0 & 1085 \\
    1849 & sensitive\_jj & 0 & 1088 \\
    1850 & lackadaisical\_jj & 0 & 1136 \\
    1851 & chaste\_jj & 0 & 1151 \\
    1852 & true-hearted\_jj & 0 & 1152 \\
    1853 & close-mouthed\_jj & 0 & 1171 \\
    1854 & torrid\_jj & 0 & 1375 \\
    1855 & brisk\_jj & 0 & 1398 \\
    1856 & steady\_jj & 0 & 1449 \\
    1857 & tame\_jj & 0 & 1572 \\
    1858 & stagnant\_jj & 0 & 1624 \\
    1859 & silent\_jj & 0 & 1675 \\
    1860 & sluggish\_jj & 0 & 1721 \\
    \hline
    \caption{Scores and rankings for most extreme 30 words in component \#39} \\
\end{longtable}
\begin{longtable}[!htbp]{| rlr@{.}l |}
    \hline
    \textbf{Rank} & \textbf{Word} & \multicolumn{2}{c|}{\textbf{Score}} \\
    \hline
    \endhead
    1 & derelict\_jj & 0 & -1331 \\
    2 & hit-or-miss\_jj & 0 & -1296 \\
    3 & deceptive\_jj & 0 & -1269 \\
    4 & highbrow\_jj & 0 & -1266 \\
    5 & aplastic\_jj & 0 & -1240 \\
    6 & proven\_jj & 0 & -1187 \\
    7 & fraudulent\_jj & 0 & -1110 \\
    8 & milquetoast & 0 & -1105 \\
    9 & conventional\_jj & 0 & -1100 \\
    10 & untutored\_jj & 0 & -1079 \\
    11 & fuddy-duddy & 0 & -1075 \\
    12 & penetrative\_jj & 0 & -1065 \\
    13 & clandestine\_jj & 0 & -1062 \\
    14 & brute & 0 & -1037 \\
    15 & coercive\_jj & 0 & -1034 \\
    16 & true-hearted\_jj & 0 & -1030 \\
    17 & undiscerning\_jj & 0 & -1029 \\
    18 & moralistic\_jj & 0 & -1028 \\
    19 & unresponsive\_jj & 0 & -1027 \\
    20 & stodgy\_jj & 0 & -1008 \\
    21 & gabby\_jj & 0 & -996 \\
    22 & mimetic\_jj & 0 & -989 \\
    23 & overactive\_jj & 0 & -969 \\
    24 & naive\_jj & 0 & -944 \\
    25 & bohemian & 0 & -937 \\
    26 & deliberate\_jj & 0 & -936 \\
    27 & controlled\_jj & 0 & -933 \\
    28 & quaint\_jj & 0 & -929 \\
    29 & proprietary\_jj & 0 & -916 \\
    30 & clear-cut\_jj & 0 & -902 \\
    1831 & exhaustive\_jj & 0 & 943 \\
    1832 & comedian & 0 & 946 \\
    1833 & strict\_jj & 0 & 946 \\
    1834 & mock & 0 & 950 \\
    1835 & mannish\_jj & 0 & 953 \\
    1836 & buoyant\_jj & 0 & 964 \\
    1837 & monkish\_jj & 0 & 982 \\
    1838 & martial\_jj & 0 & 983 \\
    1839 & generous\_jj & 0 & 984 \\
    1840 & complaisant\_jj & 0 & 987 \\
    1841 & slanderous\_jj & 0 & 987 \\
    1842 & inflexible\_jj & 0 & 998 \\
    1843 & brutal\_jj & 0 & 1004 \\
    1844 & ultrareligious\_jj & 0 & 1004 \\
    1845 & precise\_jj & 0 & 1005 \\
    1846 & acrimonious\_jj & 0 & 1025 \\
    1847 & meticulous\_jj & 0 & 1034 \\
    1848 & clown & 0 & 1039 \\
    1849 & undemanding\_jj & 0 & 1053 \\
    1850 & flexible\_jj & 0 & 1066 \\
    1851 & closed-minded\_jj & 0 & 1071 \\
    1852 & unsociable\_jj & 0 & 1138 \\
    1853 & jester & 0 & 1161 \\
    1854 & provident\_jj & 0 & 1161 \\
    1855 & dapper\_jj & 0 & 1180 \\
    1856 & exact\_jj & 0 & 1283 \\
    1857 & persnickety\_jj & 0 & 1285 \\
    1858 & discretionary\_jj & 0 & 1390 \\
    1859 & disobliging\_jj & 0 & 1409 \\
    1860 & unsocial\_jj & 0 & 1621 \\
    \hline
    \caption{Scores and rankings for most extreme 30 words in component \#40} \\
\end{longtable}
\begin{longtable}[!htbp]{| rlr@{.}l |}
    \hline
    \textbf{Rank} & \textbf{Word} & \multicolumn{2}{c|}{\textbf{Score}} \\
    \hline
    \endhead
    1 & rugged\_jj & 0 & -1488 \\
    2 & self-sufficient\_jj & 0 & -1454 \\
    3 & driftless\_jj & 0 & -1410 \\
    4 & unequivocal\_jj & 0 & -1370 \\
    5 & conclusive\_jj & 0 & -1325 \\
    6 & gluttonous\_jj & 0 & -1279 \\
    7 & addicted\_jj & 0 & -1226 \\
    8 & accessible\_jj & 0 & -1214 \\
    9 & explicit\_jj & 0 & -1198 \\
    10 & boastful\_jj & 0 & -1173 \\
    11 & sedentary\_jj & 0 & -1163 \\
    12 & brigandish\_jj & 0 & -1155 \\
    13 & headlong\_jj & 0 & -1153 \\
    14 & brief\_jj & 0 & -1146 \\
    15 & cultivated\_jj & 0 & -1130 \\
    16 & hurly-burly\_jj & 0 & -1128 \\
    17 & vestal & 0 & -1118 \\
    18 & alcoholic & 0 & -1092 \\
    19 & cryptic\_jj & 0 & -1089 \\
    20 & exhaustive\_jj & 0 & -1068 \\
    21 & unread\_jj & 0 & -1062 \\
    22 & strenuous\_jj & 0 & -1058 \\
    23 & terse\_jj & 0 & -1029 \\
    24 & avid\_jj & 0 & -1014 \\
    25 & verbal\_jj & 0 & -956 \\
    26 & argumentative\_jj & 0 & -956 \\
    27 & productive\_jj & 0 & -951 \\
    28 & granitic\_jj & 0 & -938 \\
    29 & obsessive\_jj & 0 & -931 \\
    30 & discriminative\_jj & 0 & -927 \\
    1831 & stalwart & 0 & 1012 \\
    1832 & altruistic\_jj & 0 & 1047 \\
    1833 & iffy\_jj & 0 & 1054 \\
    1834 & funereal\_jj & 0 & 1057 \\
    1835 & silent\_jj & 0 & 1085 \\
    1836 & clubby\_jj & 0 & 1105 \\
    1837 & canny\_jj & 0 & 1110 \\
    1838 & unappreciative\_jj & 0 & 1118 \\
    1839 & lenient\_jj & 0 & 1127 \\
    1840 & baleful\_jj & 0 & 1155 \\
    1841 & birdlike\_jj & 0 & 1157 \\
    1842 & inquiring\_jj & 0 & 1160 \\
    1843 & dissident\_jj & 0 & 1199 \\
    1844 & macabre\_jj & 0 & 1202 \\
    1845 & tight-lipped\_jj & 0 & 1216 \\
    1846 & fluky\_jj & 0 & 1234 \\
    1847 & unfair\_jj & 0 & 1234 \\
    1848 & august\_jj & 0 & 1237 \\
    1849 & ghoulish\_jj & 0 & 1241 \\
    1850 & true-hearted\_jj & 0 & 1242 \\
    1851 & disorderly\_jj & 0 & 1242 \\
    1852 & munificent\_jj & 0 & 1249 \\
    1853 & iconoclastic\_jj & 0 & 1266 \\
    1854 & public-minded\_jj & 0 & 1269 \\
    1855 & theoretic\_jj & 0 & 1365 \\
    1856 & variable\_jj & 0 & 1403 \\
    1857 & impassive\_jj & 0 & 1403 \\
    1858 & forcible\_jj & 0 & 1510 \\
    1859 & adjustable\_jj & 0 & 1538 \\
    1860 & lascivious\_jj & 0 & 1635 \\
    \hline
    \caption{Scores and rankings for most extreme 30 words in component \#41} \\
\end{longtable}
\begin{longtable}[!htbp]{| rlr@{.}l |}
    \hline
    \textbf{Rank} & \textbf{Word} & \multicolumn{2}{c|}{\textbf{Score}} \\
    \hline
    \endhead
    1 & admonitory\_jj & 0 & -1486 \\
    2 & noncoercive\_jj & 0 & -1430 \\
    3 & ruffian & 0 & -1248 \\
    4 & encyclopedic\_jj & 0 & -1234 \\
    5 & smarty & 0 & -1194 \\
    6 & rambunctious\_jj & 0 & -1190 \\
    7 & double-dealer & 0 & -1177 \\
    8 & malignant\_jj & 0 & -1172 \\
    9 & ornery\_jj & 0 & -1129 \\
    10 & comforter & 0 & -1124 \\
    11 & brazen\_jj & 0 & -1085 \\
    12 & quick-tempered\_jj & 0 & -1083 \\
    13 & boastful\_jj & 0 & -1075 \\
    14 & democratic\_jj & 0 & -1073 \\
    15 & malleable\_jj & 0 & -1066 \\
    16 & unscrupulous\_jj & 0 & -1050 \\
    17 & talkative\_jj & 0 & -1047 \\
    18 & aboveboard\_jj & 0 & -1041 \\
    19 & distractible\_jj & 0 & -1039 \\
    20 & true-hearted\_jj & 0 & -1035 \\
    21 & lascivious\_jj & 0 & -1011 \\
    22 & cerebral\_jj & 0 & -1010 \\
    23 & blunderbuss & 0 & -1006 \\
    24 & hypochondriac & 0 & -1001 \\
    25 & libidinous\_jj & 0 & -989 \\
    26 & alcoholic & 0 & -985 \\
    27 & explosive\_jj & 0 & -979 \\
    28 & overcautious\_jj & 0 & -976 \\
    29 & sophistic\_jj & 0 & -975 \\
    30 & retributive\_jj & 0 & -966 \\
    1831 & cowardly\_jj & 0 & 938 \\
    1832 & hostile\_jj & 0 & 950 \\
    1833 & unhelpful\_jj & 0 & 952 \\
    1834 & earnest\_jj & 0 & 983 \\
    1835 & felicitous\_jj & 0 & 985 \\
    1836 & coward & 0 & 986 \\
    1837 & unanchored\_jj & 0 & 987 \\
    1838 & unflinching\_jj & 0 & 991 \\
    1839 & critical\_jj & 0 & 994 \\
    1840 & objector & 0 & 1021 \\
    1841 & supercilious\_jj & 0 & 1031 \\
    1842 & self-respecting\_jj & 0 & 1035 \\
    1843 & self-important\_jj & 0 & 1060 \\
    1844 & outdoor\_jj & 0 & 1073 \\
    1845 & prompt\_jj & 0 & 1081 \\
    1846 & indirect\_jj & 0 & 1093 \\
    1847 & direct\_jj & 0 & 1143 \\
    1848 & hellion & 0 & 1147 \\
    1849 & affected\_jj & 0 & 1166 \\
    1850 & particular\_jj & 0 & 1188 \\
    1851 & conscientious\_jj & 0 & 1222 \\
    1852 & dedicated\_jj & 0 & 1242 \\
    1853 & hapless\_jj & 0 & 1246 \\
    1854 & negative\_jj & 0 & 1270 \\
    1855 & generalist & 0 & 1296 \\
    1856 & undemonstrative\_jj & 0 & 1347 \\
    1857 & humanitarian\_jj & 0 & 1389 \\
    1858 & complimentary\_jj & 0 & 1400 \\
    1859 & slacker & 0 & 1503 \\
    1860 & passive\_jj & 0 & 1689 \\
    \hline
    \caption{Scores and rankings for most extreme 30 words in component \#42} \\
\end{longtable}
\begin{longtable}[!htbp]{| rlr@{.}l |}
    \hline
    \textbf{Rank} & \textbf{Word} & \multicolumn{2}{c|}{\textbf{Score}} \\
    \hline
    \endhead
    1 & philanthropic\_jj & 0 & -1533 \\
    2 & babbler & 0 & -1473 \\
    3 & humanitarian\_jj & 0 & -1378 \\
    4 & tight-lipped\_jj & 0 & -1374 \\
    5 & humanitarian & 0 & -1350 \\
    6 & notional\_jj & 0 & -1310 \\
    7 & good-tempered\_jj & 0 & -1303 \\
    8 & bovine\_jj & 0 & -1297 \\
    9 & charitable\_jj & 0 & -1231 \\
    10 & vague\_jj & 0 & -1188 \\
    11 & bitch & 0 & -1162 \\
    12 & unsocial\_jj & 0 & -1153 \\
    13 & aimless\_jj & 0 & -1115 \\
    14 & responsible\_jj & 0 & -1089 \\
    15 & haphazard\_jj & 0 & -1075 \\
    16 & coy\_jj & 0 & -1062 \\
    17 & lax\_jj & 0 & -1058 \\
    18 & categorical\_jj & 0 & -1055 \\
    19 & unenthusiastic\_jj & 0 & -1007 \\
    20 & flighty\_jj & 0 & -994 \\
    21 & mutinous\_jj & 0 & -990 \\
    22 & militant\_jj & 0 & -985 \\
    23 & provincial\_jj & 0 & -981 \\
    24 & vocal\_jj & 0 & -965 \\
    25 & tireless\_jj & 0 & -958 \\
    26 & initiative & 0 & -951 \\
    27 & wildcat\_jj & 0 & -950 \\
    28 & temperate\_jj & 0 & -947 \\
    29 & gabby\_jj & 0 & -946 \\
    30 & phlegmatic\_jj & 0 & -939 \\
    1831 & thrifty\_jj & 0 & 917 \\
    1832 & shrewd\_jj & 0 & 957 \\
    1833 & pixy & 0 & 958 \\
    1834 & bearish\_jj & 0 & 961 \\
    1835 & wise\_jj & 0 & 964 \\
    1836 & holier-than-thou\_jj & 0 & 973 \\
    1837 & forward-looking\_jj & 0 & 977 \\
    1838 & savoir-faire & 0 & 982 \\
    1839 & cruel\_jj & 0 & 991 \\
    1840 & merciless\_jj & 0 & 1000 \\
    1841 & mirthless\_jj & 0 & 1000 \\
    1842 & hellion & 0 & 1007 \\
    1843 & observant\_jj & 0 & 1017 \\
    1844 & exclamatory\_jj & 0 & 1068 \\
    1845 & clubby\_jj & 0 & 1075 \\
    1846 & hasty\_jj & 0 & 1077 \\
    1847 & barbarous\_jj & 0 & 1100 \\
    1848 & despotic\_jj & 0 & 1109 \\
    1849 & aboveboard\_jj & 0 & 1131 \\
    1850 & fair-minded\_jj & 0 & 1141 \\
    1851 & autocratic\_jj & 0 & 1154 \\
    1852 & cutthroat\_jj & 0 & 1162 \\
    1853 & unceremonious\_jj & 0 & 1163 \\
    1854 & cautionary\_jj & 0 & 1298 \\
    1855 & censorial\_jj & 0 & 1367 \\
    1856 & speedy\_jj & 0 & 1373 \\
    1857 & upstart & 0 & 1440 \\
    1858 & precipitous\_jj & 0 & 1481 \\
    1859 & supersensitive\_jj & 0 & 1513 \\
    1860 & abrupt\_jj & 0 & 1637 \\
    \hline
    \caption{Scores and rankings for most extreme 30 words in component \#43} \\
\end{longtable}
\begin{longtable}[!htbp]{| rlr@{.}l |}
    \hline
    \textbf{Rank} & \textbf{Word} & \multicolumn{2}{c|}{\textbf{Score}} \\
    \hline
    \endhead
    1 & beneficent\_jj & 0 & -1473 \\
    2 & risqué\_jj & 0 & -1276 \\
    3 & reconciliatory\_jj & 0 & -1218 \\
    4 & meddlesome\_jj & 0 & -1202 \\
    5 & pervious\_jj & 0 & -1176 \\
    6 & independent\_jj & 0 & -1173 \\
    7 & lachrymose\_jj & 0 & -1160 \\
    8 & lethargic\_jj & 0 & -1145 \\
    9 & outdated\_jj & 0 & -1093 \\
    10 & unfeminine\_jj & 0 & -1090 \\
    11 & foul-mouthed\_jj & 0 & -1089 \\
    12 & analytical\_jj & 0 & -1077 \\
    13 & bureaucratic\_jj & 0 & -1057 \\
    14 & oversensitive\_jj & 0 & -1056 \\
    15 & docile\_jj & 0 & -1052 \\
    16 & risque\_jj & 0 & -1043 \\
    17 & lifeless\_jj & 0 & -1036 \\
    18 & undiplomatic\_jj & 0 & -1019 \\
    19 & imitative\_jj & 0 & -1008 \\
    20 & spiritless\_jj & 0 & -1001 \\
    21 & proper\_jj & 0 & -991 \\
    22 & fruit & 0 & -980 \\
    23 & automaton & 0 & -975 \\
    24 & disciplinarian & 0 & -963 \\
    25 & discourteous\_jj & 0 & -957 \\
    26 & brat & 0 & -953 \\
    27 & playboy & 0 & -950 \\
    28 & inimical\_jj & 0 & -940 \\
    29 & ham & 0 & -937 \\
    30 & petulant\_jj & 0 & -936 \\
    1831 & cassandra & 0 & 960 \\
    1832 & affirmative\_jj & 0 & 966 \\
    1833 & vocal\_jj & 0 & 970 \\
    1834 & devotee & 0 & 976 \\
    1835 & hindsight & 0 & 982 \\
    1836 & insidious\_jj & 0 & 1000 \\
    1837 & low-minded\_jj & 0 & 1002 \\
    1838 & retiring\_jj & 0 & 1005 \\
    1839 & balky\_jj & 0 & 1008 \\
    1840 & quaint\_jj & 0 & 1010 \\
    1841 & languorous\_jj & 0 & 1023 \\
    1842 & stay-at-home\_jj & 0 & 1051 \\
    1843 & relentless\_jj & 0 & 1066 \\
    1844 & rugged\_jj & 0 & 1073 \\
    1845 & audacious\_jj & 0 & 1084 \\
    1846 & fancier & 0 & 1097 \\
    1847 & unversed\_jj & 0 & 1104 \\
    1848 & cackler & 0 & 1110 \\
    1849 & itinerant\_jj & 0 & 1132 \\
    1850 & notional\_jj & 0 & 1166 \\
    1851 & law-abiding\_jj & 0 & 1187 \\
    1852 & plain-spoken\_jj & 0 & 1226 \\
    1853 & timid & 0 & 1236 \\
    1854 & improviser & 0 & 1265 \\
    1855 & teetotaler & 0 & 1271 \\
    1856 & driftless\_jj & 0 & 1275 \\
    1857 & hard-shell\_jj & 0 & 1291 \\
    1858 & carper & 0 & 1298 \\
    1859 & crowder & 0 & 1463 \\
    1860 & rivalrous\_jj & 0 & 1668 \\
    \hline
    \caption{Scores and rankings for most extreme 30 words in component \#44} \\
\end{longtable}
\begin{longtable}[!htbp]{| rlr@{.}l |}
    \hline
    \textbf{Rank} & \textbf{Word} & \multicolumn{2}{c|}{\textbf{Score}} \\
    \hline
    \endhead
    1 & efficient\_jj & 0 & -1476 \\
    2 & corrosive\_jj & 0 & -1457 \\
    3 & observant\_jj & 0 & -1328 \\
    4 & solemn\_jj & 0 & -1315 \\
    5 & poisonous\_jj & 0 & -1298 \\
    6 & mirthless\_jj & 0 & -1283 \\
    7 & traditional\_jj & 0 & -1258 \\
    8 & caustic\_jj & 0 & -1245 \\
    9 & flammable\_jj & 0 & -1211 \\
    10 & bovine\_jj & 0 & -1200 \\
    11 & prodigal\_jj & 0 & -1181 \\
    12 & volatile\_jj & 0 & -1164 \\
    13 & merciful\_jj & 0 & -1157 \\
    14 & durable\_jj & 0 & -1129 \\
    15 & acid\_jj & 0 & -1128 \\
    16 & wildcat\_jj & 0 & -1119 \\
    17 & explosive\_jj & 0 & -1108 \\
    18 & abrupt\_jj & 0 & -1071 \\
    19 & controlled\_jj & 0 & -1057 \\
    20 & deliberate\_jj & 0 & -1042 \\
    21 & roughneck & 0 & -1035 \\
    22 & polemical\_jj & 0 & -1028 \\
    23 & expeditious\_jj & 0 & -1013 \\
    24 & abject\_jj & 0 & -1000 \\
    25 & exhaustible\_jj & 0 & -998 \\
    26 & prissy\_jj & 0 & -986 \\
    27 & torpid\_jj & 0 & -980 \\
    28 & unerring\_jj & 0 & -969 \\
    29 & inward\_jj & 0 & -964 \\
    30 & butter-fingered\_jj & 0 & -955 \\
    1831 & bellicose\_jj & 0 & 938 \\
    1832 & disputatious\_jj & 0 & 959 \\
    1833 & unanchored\_jj & 0 & 961 \\
    1834 & card & 0 & 968 \\
    1835 & intransigent\_jj & 0 & 974 \\
    1836 & self-reliant\_jj & 0 & 983 \\
    1837 & impersonal\_jj & 0 & 985 \\
    1838 & trustful\_jj & 0 & 986 \\
    1839 & lewd\_jj & 0 & 987 \\
    1840 & self-defensive\_jj & 0 & 988 \\
    1841 & unyielding\_jj & 0 & 999 \\
    1842 & cooperative\_jj & 0 & 1001 \\
    1843 & animated\_jj & 0 & 1027 \\
    1844 & teaser & 0 & 1057 \\
    1845 & mobile\_jj & 0 & 1059 \\
    1846 & fearless\_jj & 0 & 1078 \\
    1847 & cryptic\_jj & 0 & 1079 \\
    1848 & visionary\_jj & 0 & 1085 \\
    1849 & staunch\_jj & 0 & 1092 \\
    1850 & principled\_jj & 0 & 1116 \\
    1851 & quaint\_jj & 0 & 1168 \\
    1852 & arbitrary\_jj & 0 & 1193 \\
    1853 & adjustable\_jj & 0 & 1250 \\
    1854 & dedicated\_jj & 0 & 1305 \\
    1855 & congratulatory\_jj & 0 & 1311 \\
    1856 & innocuous\_jj & 0 & 1321 \\
    1857 & censorial\_jj & 0 & 1362 \\
    1858 & inexpressive\_jj & 0 & 1371 \\
    1859 & defamatory\_jj & 0 & 1480 \\
    1860 & obscene\_jj & 0 & 1527 \\
    \hline
    \caption{Scores and rankings for most extreme 30 words in component \#45} \\
\end{longtable}
\begin{longtable}[!htbp]{| rlr@{.}l |}
    \hline
    \textbf{Rank} & \textbf{Word} & \multicolumn{2}{c|}{\textbf{Score}} \\
    \hline
    \endhead
    1 & pervious\_jj & 0 & -1487 \\
    2 & prodigal & 0 & -1477 \\
    3 & authoritative\_jj & 0 & -1454 \\
    4 & philanthropic\_jj & 0 & -1278 \\
    5 & high-sounding\_jj & 0 & -1267 \\
    6 & unblushing\_jj & 0 & -1168 \\
    7 & hostile\_jj & 0 & -1157 \\
    8 & undependable\_jj & 0 & -1145 \\
    9 & ambitious\_jj & 0 & -1105 \\
    10 & incontrovertible\_jj & 0 & -1090 \\
    11 & earnest\_jj & 0 & -1089 \\
    12 & manly\_jj & 0 & -1047 \\
    13 & shy\_jj & 0 & -1046 \\
    14 & stringent\_jj & 0 & -1044 \\
    15 & brazen\_jj & 0 & -1039 \\
    16 & secretive\_jj & 0 & -1018 \\
    17 & tyrannical\_jj & 0 & -1015 \\
    18 & careworn\_jj & 0 & -1015 \\
    19 & outgoing\_jj & 0 & -1002 \\
    20 & proven\_jj & 0 & -996 \\
    21 & lazy\_jj & 0 & -965 \\
    22 & fluky\_jj & 0 & -958 \\
    23 & cold\_jj & 0 & -950 \\
    24 & congratulatory\_jj & 0 & -943 \\
    25 & politic\_jj & 0 & -935 \\
    26 & absent-minded\_jj & 0 & -923 \\
    27 & spry\_jj & 0 & -921 \\
    28 & eruptive\_jj & 0 & -912 \\
    29 & reclusive\_jj & 0 & -901 \\
    30 & retiring\_jj & 0 & -890 \\
    1831 & compromiser & 0 & 945 \\
    1832 & self-sufficient\_jj & 0 & 948 \\
    1833 & narrow\_jj & 0 & 956 \\
    1834 & accessible\_jj & 0 & 963 \\
    1835 & undiscriminating\_jj & 0 & 979 \\
    1836 & derogatory\_jj & 0 & 990 \\
    1837 & dignified\_jj & 0 & 990 \\
    1838 & unguarded\_jj & 0 & 991 \\
    1839 & amnesic\_jj & 0 & 992 \\
    1840 & orderly\_jj & 0 & 1005 \\
    1841 & stable\_jj & 0 & 1011 \\
    1842 & sympathetic\_jj & 0 & 1029 \\
    1843 & fair-minded\_jj & 0 & 1043 \\
    1844 & itinerant\_jj & 0 & 1044 \\
    1845 & disorderly\_jj & 0 & 1053 \\
    1846 & incorrupt\_jj & 0 & 1062 \\
    1847 & tireless\_jj & 0 & 1067 \\
    1848 & die-hard\_jj & 0 & 1069 \\
    1849 & docile\_jj & 0 & 1073 \\
    1850 & equitable\_jj & 0 & 1074 \\
    1851 & daring & 0 & 1106 \\
    1852 & hasty\_jj & 0 & 1127 \\
    1853 & particular\_jj & 0 & 1136 \\
    1854 & encyclopedic\_jj & 0 & 1145 \\
    1855 & careless\_jj & 0 & 1185 \\
    1856 & uneconomical\_jj & 0 & 1187 \\
    1857 & vain\_jj & 0 & 1194 \\
    1858 & speedy\_jj & 0 & 1314 \\
    1859 & unrestrained\_jj & 0 & 1403 \\
    1860 & impartial\_jj & 0 & 1712 \\
    \hline
    \caption{Scores and rankings for most extreme 30 words in component \#46} \\
\end{longtable}
\begin{longtable}[!htbp]{| rlr@{.}l |}
    \hline
    \textbf{Rank} & \textbf{Word} & \multicolumn{2}{c|}{\textbf{Score}} \\
    \hline
    \endhead
    1 & derelict\_jj & 0 & -1561 \\
    2 & gadabout & 0 & -1474 \\
    3 & beneficial\_jj & 0 & -1423 \\
    4 & truckling & 0 & -1417 \\
    5 & sophistic\_jj & 0 & -1311 \\
    6 & flammable\_jj & 0 & -1242 \\
    7 & penetrative\_jj & 0 & -1195 \\
    8 & dauber & 0 & -1181 \\
    9 & plastic\_jj & 0 & -1168 \\
    10 & forensic\_jj & 0 & -1155 \\
    11 & hellion & 0 & -1149 \\
    12 & pliable\_jj & 0 & -1138 \\
    13 & unconventional\_jj & 0 & -1089 \\
    14 & tidy & 0 & -1075 \\
    15 & gainful\_jj & 0 & -1061 \\
    16 & august\_jj & 0 & -1056 \\
    17 & coercive\_jj & 0 & -1055 \\
    18 & lascivious\_jj & 0 & -1046 \\
    19 & abrasive\_jj & 0 & -1037 \\
    20 & caustic\_jj & 0 & -1031 \\
    21 & abandoned\_jj & 0 & -1012 \\
    22 & discretionary\_jj & 0 & -999 \\
    23 & protective\_jj & 0 & -965 \\
    24 & verve & 0 & -960 \\
    25 & impassive\_jj & 0 & -959 \\
    26 & retrospective\_jj & 0 & -955 \\
    27 & remiss\_jj & 0 & -935 \\
    28 & inventive\_jj & 0 & -934 \\
    29 & unappreciative\_jj & 0 & -918 \\
    30 & controlled\_jj & 0 & -915 \\
    1831 & nagging\_jj & 0 & 961 \\
    1832 & ultraconservative\_jj & 0 & 970 \\
    1833 & terse\_jj & 0 & 974 \\
    1834 & finicky\_jj & 0 & 979 \\
    1835 & clamorous\_jj & 0 & 980 \\
    1836 & cultivated\_jj & 0 & 988 \\
    1837 & bureaucratic\_jj & 0 & 993 \\
    1838 & sportsmanlike\_jj & 0 & 1002 \\
    1839 & dissident & 0 & 1005 \\
    1840 & pontifical\_jj & 0 & 1014 \\
    1841 & madcap & 0 & 1028 \\
    1842 & smarty & 0 & 1040 \\
    1843 & litigious\_jj & 0 & 1062 \\
    1844 & samaritan & 0 & 1063 \\
    1845 & critical\_jj & 0 & 1065 \\
    1846 & patient\_jj & 0 & 1072 \\
    1847 & clement\_jj & 0 & 1104 \\
    1848 & ruffian & 0 & 1155 \\
    1849 & uncontrolled\_jj & 0 & 1182 \\
    1850 & succinct\_jj & 0 & 1190 \\
    1851 & punctual\_jj & 0 & 1204 \\
    1852 & tractable\_jj & 0 & 1212 \\
    1853 & capricious\_jj & 0 & 1242 \\
    1854 & headlong\_jj & 0 & 1249 \\
    1855 & big-mouthed\_jj & 0 & 1274 \\
    1856 & acute\_jj & 0 & 1366 \\
    1857 & solemn\_jj & 0 & 1375 \\
    1858 & speedy\_jj & 0 & 1471 \\
    1859 & serious\_jj & 0 & 1540 \\
    1860 & alert\_jj & 0 & 1565 \\
    \hline
    \caption{Scores and rankings for most extreme 30 words in component \#47} \\
\end{longtable}
\begin{longtable}[!htbp]{| rlr@{.}l |}
    \hline
    \textbf{Rank} & \textbf{Word} & \multicolumn{2}{c|}{\textbf{Score}} \\
    \hline
    \endhead
    1 & touchy\_jj & 0 & -1429 \\
    2 & munificent\_jj & 0 & -1366 \\
    3 & yellow\_jj & 0 & -1357 \\
    4 & soulful\_jj & 0 & -1284 \\
    5 & affirmative\_jj & 0 & -1255 \\
    6 & litigious\_jj & 0 & -1249 \\
    7 & labile\_jj & 0 & -1220 \\
    8 & perspicuous\_jj & 0 & -1219 \\
    9 & tidy & 0 & -1208 \\
    10 & thorny\_jj & 0 & -1109 \\
    11 & bright\_jj & 0 & -1108 \\
    12 & unsportsmanlike\_jj & 0 & -1084 \\
    13 & lawless\_jj & 0 & -1082 \\
    14 & admonitory\_jj & 0 & -1070 \\
    15 & piercing\_jj & 0 & -1070 \\
    16 & wildcat\_jj & 0 & -1037 \\
    17 & patronizing\_jj & 0 & -1035 \\
    18 & acrid\_jj & 0 & -1018 \\
    19 & prophetic\_jj & 0 & -1014 \\
    20 & fly-by-night\_jj & 0 & -1010 \\
    21 & card & 0 & -1005 \\
    22 & hayseed & 0 & -998 \\
    23 & jaunty\_jj & 0 & -996 \\
    24 & stingy\_jj & 0 & -996 \\
    25 & unmanly\_jj & 0 & -972 \\
    26 & vocal\_jj & 0 & -964 \\
    27 & sloppy\_jj & 0 & -963 \\
    28 & noncompliant\_jj & 0 & -954 \\
    29 & pedantic\_jj & 0 & -939 \\
    30 & meticulous\_jj & 0 & -923 \\
    1831 & outdoor\_jj & 0 & 962 \\
    1832 & sagacious\_jj & 0 & 965 \\
    1833 & unchivalrous\_jj & 0 & 966 \\
    1834 & idealistic\_jj & 0 & 967 \\
    1835 & ultrareligious\_jj & 0 & 970 \\
    1836 & butter-fingered\_jj & 0 & 972 \\
    1837 & cunning & 0 & 972 \\
    1838 & chirpy\_jj & 0 & 989 \\
    1839 & quick-tempered\_jj & 0 & 994 \\
    1840 & complimentary\_jj & 0 & 994 \\
    1841 & speedy\_jj & 0 & 997 \\
    1842 & resistive\_jj & 0 & 1005 \\
    1843 & pugilistic\_jj & 0 & 1008 \\
    1844 & abrupt\_jj & 0 & 1010 \\
    1845 & retiring\_jj & 0 & 1015 \\
    1846 & adjustable\_jj & 0 & 1021 \\
    1847 & fancier & 0 & 1027 \\
    1848 & avid\_jj & 0 & 1036 \\
    1849 & contrary\_jj & 0 & 1039 \\
    1850 & indoor\_jj & 0 & 1050 \\
    1851 & thick-headed\_jj & 0 & 1057 \\
    1852 & eager\_jj & 0 & 1086 \\
    1853 & boneless\_jj & 0 & 1096 \\
    1854 & unrestrained\_jj & 0 & 1116 \\
    1855 & shallow\_jj & 0 & 1129 \\
    1856 & wishful\_jj & 0 & 1185 \\
    1857 & hasty\_jj & 0 & 1250 \\
    1858 & timorous\_jj & 0 & 1296 \\
    1859 & sop & 0 & 1479 \\
    1860 & carper & 0 & 1602 \\
    \hline
    \caption{Scores and rankings for most extreme 30 words in component \#48} \\
\end{longtable}
\begin{longtable}[!htbp]{| rlr@{.}l |}
    \hline
    \textbf{Rank} & \textbf{Word} & \multicolumn{2}{c|}{\textbf{Score}} \\
    \hline
    \endhead
    1 & self-denying\_jj & 0 & -1489 \\
    2 & busybody & 0 & -1335 \\
    3 & insubordinate\_jj & 0 & -1267 \\
    4 & fair-weather\_jj & 0 & -1206 \\
    5 & censorial\_jj & 0 & -1164 \\
    6 & undependable\_jj & 0 & -1131 \\
    7 & bountiful\_jj & 0 & -1080 \\
    8 & unobtrusive\_jj & 0 & -1078 \\
    9 & ultraconservative\_jj & 0 & -1068 \\
    10 & tidy & 0 & -1048 \\
    11 & beneficent\_jj & 0 & -1032 \\
    12 & unfaltering\_jj & 0 & -1028 \\
    13 & militant\_jj & 0 & -1005 \\
    14 & clear-cut\_jj & 0 & -992 \\
    15 & fluky\_jj & 0 & -987 \\
    16 & hit-or-miss\_jj & 0 & -985 \\
    17 & overzealous\_jj & 0 & -974 \\
    18 & coherent\_jj & 0 & -971 \\
    19 & stay-at-home\_jj & 0 & -946 \\
    20 & crowder & 0 & -928 \\
    21 & obsessive & 0 & -918 \\
    22 & particular\_jj & 0 & -916 \\
    23 & biased\_jj & 0 & -913 \\
    24 & spasmodic\_jj & 0 & -912 \\
    25 & coddle & 0 & -904 \\
    26 & unimaginative\_jj & 0 & -900 \\
    27 & careless\_jj & 0 & -898 \\
    28 & predatory\_jj & 0 & -895 \\
    29 & magisterial\_jj & 0 & -886 \\
    30 & remiss\_jj & 0 & -877 \\
    1831 & gourmet & 0 & 897 \\
    1832 & sedentary\_jj & 0 & 909 \\
    1833 & invincible\_jj & 0 & 925 \\
    1834 & muddle-headed\_jj & 0 & 941 \\
    1835 & shiftless\_jj & 0 & 943 \\
    1836 & sour\_jj & 0 & 952 \\
    1837 & discriminative\_jj & 0 & 952 \\
    1838 & nonconforming\_jj & 0 & 962 \\
    1839 & uppity\_jj & 0 & 967 \\
    1840 & derogatory\_jj & 0 & 967 \\
    1841 & august\_jj & 0 & 981 \\
    1842 & clandestine\_jj & 0 & 981 \\
    1843 & spitfire & 0 & 984 \\
    1844 & warlike\_jj & 0 & 985 \\
    1845 & gallant\_jj & 0 & 996 \\
    1846 & illiterate\_jj & 0 & 1007 \\
    1847 & daredevil\_jj & 0 & 1017 \\
    1848 & unconstrained\_jj & 0 & 1031 \\
    1849 & daredevil & 0 & 1039 \\
    1850 & hot-blooded\_jj & 0 & 1041 \\
    1851 & prayerful\_jj & 0 & 1041 \\
    1852 & bellicose\_jj & 0 & 1045 \\
    1853 & torturous\_jj & 0 & 1045 \\
    1854 & belligerent\_jj & 0 & 1105 \\
    1855 & bearish\_jj & 0 & 1170 \\
    1856 & treacherous\_jj & 0 & 1177 \\
    1857 & querulous\_jj & 0 & 1179 \\
    1858 & hostile\_jj & 0 & 1336 \\
    1859 & strenuous\_jj & 0 & 1527 \\
    1860 & notional\_jj & 0 & 2070 \\
    \hline
    \caption{Scores and rankings for most extreme 30 words in component \#49} \\
\end{longtable}
\begin{longtable}[!htbp]{| rlr@{.}l |}
    \hline
    \textbf{Rank} & \textbf{Word} & \multicolumn{2}{c|}{\textbf{Score}} \\
    \hline
    \endhead
    1 & corrective\_jj & 0 & -1440 \\
    2 & sportsmanlike\_jj & 0 & -1368 \\
    3 & sincere\_jj & 0 & -1355 \\
    4 & gluttonous\_jj & 0 & -1291 \\
    5 & fan & 0 & -1278 \\
    6 & torturous\_jj & 0 & -1251 \\
    7 & die-hard\_jj & 0 & -1206 \\
    8 & plastic\_jj & 0 & -1125 \\
    9 & obscene\_jj & 0 & -1114 \\
    10 & obsessive\_jj & 0 & -1107 \\
    11 & defamatory\_jj & 0 & -1085 \\
    12 & effeminate\_jj & 0 & -1047 \\
    13 & compulsive\_jj & 0 & -1046 \\
    14 & immoderate\_jj & 0 & -1026 \\
    15 & funereal\_jj & 0 & -1021 \\
    16 & low-minded\_jj & 0 & -998 \\
    17 & perspicuous\_jj & 0 & -996 \\
    18 & passionate\_jj & 0 & -995 \\
    19 & hedonist & 0 & -989 \\
    20 & card & 0 & -989 \\
    21 & bountiful\_jj & 0 & -983 \\
    22 & polished\_jj & 0 & -981 \\
    23 & dependable\_jj & 0 & -977 \\
    24 & stolid\_jj & 0 & -975 \\
    25 & consolatory\_jj & 0 & -965 \\
    26 & venturesome\_jj & 0 & -955 \\
    27 & sparkling\_jj & 0 & -937 \\
    28 & invariable\_jj & 0 & -934 \\
    29 & autocratic\_jj & 0 & -933 \\
    30 & serious\_jj & 0 & -930 \\
    1831 & bossy\_jj & 0 & 904 \\
    1832 & pontifical\_jj & 0 & 916 \\
    1833 & militant\_jj & 0 & 917 \\
    1834 & minx & 0 & 934 \\
    1835 & neglectful\_jj & 0 & 937 \\
    1836 & loud\_jj & 0 & 939 \\
    1837 & unsophisticated\_jj & 0 & 940 \\
    1838 & adaptive\_jj & 0 & 941 \\
    1839 & upstart & 0 & 965 \\
    1840 & philanthropic\_jj & 0 & 967 \\
    1841 & quiet\_jj & 0 & 968 \\
    1842 & money-grubbing\_jj & 0 & 975 \\
    1843 & clairvoyant\_jj & 0 & 984 \\
    1844 & brazen\_jj & 0 & 986 \\
    1845 & laggard & 0 & 995 \\
    1846 & cognizant\_jj & 0 & 1017 \\
    1847 & indeterminate\_jj & 0 & 1024 \\
    1848 & cool\_jj & 0 & 1027 \\
    1849 & impersonal\_jj & 0 & 1036 \\
    1850 & provident\_jj & 0 & 1049 \\
    1851 & passive\_jj & 0 & 1049 \\
    1852 & insurgent\_jj & 0 & 1073 \\
    1853 & tidy & 0 & 1077 \\
    1854 & clamorous\_jj & 0 & 1086 \\
    1855 & prescient\_jj & 0 & 1091 \\
    1856 & hussy & 0 & 1150 \\
    1857 & humanitarian & 0 & 1150 \\
    1858 & generalist & 0 & 1154 \\
    1859 & intellectual\_jj & 0 & 1209 \\
    1860 & intellectual & 0 & 1444 \\
    \hline
    \caption{Scores and rankings for most extreme 30 words in component \#50} \\
\end{longtable}
\begin{longtable}[!htbp]{| rlr@{.}l |}
    \hline
    \textbf{Rank} & \textbf{Word} & \multicolumn{2}{c|}{\textbf{Score}} \\
    \hline
    \endhead
    1 & nonhostile\_jj & 0 & -1316 \\
    2 & connoisseur & 0 & -1205 \\
    3 & experimentalist & 0 & -1189 \\
    4 & insurgent\_jj & 0 & -1116 \\
    5 & stand-offish\_jj & 0 & -1105 \\
    6 & patient\_jj & 0 & -1092 \\
    7 & double-faced\_jj & 0 & -1066 \\
    8 & indirect\_jj & 0 & -1064 \\
    9 & blunt\_jj & 0 & -1062 \\
    10 & spasmodic\_jj & 0 & -1039 \\
    11 & tame\_jj & 0 & -1027 \\
    12 & literary\_jj & 0 & -1020 \\
    13 & ruminative\_jj & 0 & -1013 \\
    14 & automaton & 0 & -1011 \\
    15 & geezer & 0 & -995 \\
    16 & timid & 0 & -990 \\
    17 & exclamatory\_jj & 0 & -981 \\
    18 & gluttonous\_jj & 0 & -974 \\
    19 & gainful\_jj & 0 & -974 \\
    20 & informal\_jj & 0 & -968 \\
    21 & comradely\_jj & 0 & -967 \\
    22 & abstract\_jj & 0 & -965 \\
    23 & gourmand & 0 & -964 \\
    24 & mannish\_jj & 0 & -955 \\
    25 & elusive\_jj & 0 & -943 \\
    26 & samaritan & 0 & -929 \\
    27 & forensic\_jj & 0 & -928 \\
    28 & persnickety\_jj & 0 & -914 \\
    29 & dissident\_jj & 0 & -912 \\
    30 & savoir-faire & 0 & -906 \\
    1831 & gutsy\_jj & 0 & 938 \\
    1832 & nonvolatile\_jj & 0 & 943 \\
    1833 & salacious\_jj & 0 & 946 \\
    1834 & low-minded\_jj & 0 & 949 \\
    1835 & vibrant\_jj & 0 & 956 \\
    1836 & malcontent & 0 & 957 \\
    1837 & unromantic\_jj & 0 & 961 \\
    1838 & brave\_jj & 0 & 964 \\
    1839 & initiative & 0 & 965 \\
    1840 & barbed\_jj & 0 & 972 \\
    1841 & expansive\_jj & 0 & 1017 \\
    1842 & mimetic\_jj & 0 & 1020 \\
    1843 & meteoric\_jj & 0 & 1021 \\
    1844 & hectic\_jj & 0 & 1024 \\
    1845 & observant\_jj & 0 & 1026 \\
    1846 & theoretic\_jj & 0 & 1028 \\
    1847 & live\_jj & 0 & 1032 \\
    1848 & boastful\_jj & 0 & 1037 \\
    1849 & unchangeable\_jj & 0 & 1037 \\
    1850 & defamatory\_jj & 0 & 1037 \\
    1851 & ambitious\_jj & 0 & 1070 \\
    1852 & unceremonious\_jj & 0 & 1122 \\
    1853 & graceless\_jj & 0 & 1155 \\
    1854 & madcap & 0 & 1262 \\
    1855 & compliant\_jj & 0 & 1262 \\
    1856 & cognizant\_jj & 0 & 1317 \\
    1857 & strict\_jj & 0 & 1321 \\
    1858 & good-tempered\_jj & 0 & 1378 \\
    1859 & proprietary\_jj & 0 & 1408 \\
    1860 & mobile\_jj & 0 & 1706 \\
    \hline
    \caption{Scores and rankings for most extreme 30 words in component \#51} \\
\end{longtable}
\begin{longtable}[!htbp]{| rlr@{.}l |}
    \hline
    \textbf{Rank} & \textbf{Word} & \multicolumn{2}{c|}{\textbf{Score}} \\
    \hline
    \endhead
    1 & charitable\_jj & 0 & -1428 \\
    2 & discretionary\_jj & 0 & -1302 \\
    3 & martial\_jj & 0 & -1208 \\
    4 & quitter & 0 & -1176 \\
    5 & insubordinate\_jj & 0 & -1152 \\
    6 & cunning & 0 & -1088 \\
    7 & proprietary\_jj & 0 & -1074 \\
    8 & eruptive\_jj & 0 & -1060 \\
    9 & nomadic\_jj & 0 & -1010 \\
    10 & reliant\_jj & 0 & -999 \\
    11 & dependent\_jj & 0 & -995 \\
    12 & cocky\_jj & 0 & -993 \\
    13 & live\_jj & 0 & -991 \\
    14 & upstart\_jj & 0 & -986 \\
    15 & self-reliant\_jj & 0 & -974 \\
    16 & incautious\_jj & 0 & -973 \\
    17 & unaccommodating\_jj & 0 & -962 \\
    18 & ornery\_jj & 0 & -958 \\
    19 & staid\_jj & 0 & -943 \\
    20 & thick-skinned\_jj & 0 & -938 \\
    21 & pundit & 0 & -909 \\
    22 & lazy\_jj & 0 & -908 \\
    23 & stoic & 0 & -902 \\
    24 & vinegary\_jj & 0 & -893 \\
    25 & bitter\_jj & 0 & -893 \\
    26 & immoderate\_jj & 0 & -889 \\
    27 & oratorical\_jj & 0 & -888 \\
    28 & nonreligious\_jj & 0 & -860 \\
    29 & philanthropic\_jj & 0 & -857 \\
    30 & farcical\_jj & 0 & -853 \\
    1831 & corrective\_jj & 0 & 910 \\
    1832 & corrosive\_jj & 0 & 911 \\
    1833 & brigandish\_jj & 0 & 916 \\
    1834 & innocuous\_jj & 0 & 919 \\
    1835 & unpredictable\_jj & 0 & 924 \\
    1836 & clamorous\_jj & 0 & 932 \\
    1837 & clairvoyant & 0 & 943 \\
    1838 & hypercritical\_jj & 0 & 959 \\
    1839 & god-fearing\_jj & 0 & 962 \\
    1840 & fraternal\_jj & 0 & 976 \\
    1841 & profound\_jj & 0 & 996 \\
    1842 & inexorable\_jj & 0 & 1007 \\
    1843 & unshakable\_jj & 0 & 1025 \\
    1844 & unfriendly\_jj & 0 & 1037 \\
    1845 & labile\_jj & 0 & 1045 \\
    1846 & undisguised\_jj & 0 & 1052 \\
    1847 & elegant\_jj & 0 & 1055 \\
    1848 & untidy\_jj & 0 & 1060 \\
    1849 & stalwart\_jj & 0 & 1108 \\
    1850 & epicurean\_jj & 0 & 1125 \\
    1851 & feline\_jj & 0 & 1137 \\
    1852 & bohemian\_jj & 0 & 1139 \\
    1853 & brave\_jj & 0 & 1154 \\
    1854 & mutable\_jj & 0 & 1216 \\
    1855 & unhelpful\_jj & 0 & 1225 \\
    1856 & saucy\_jj & 0 & 1279 \\
    1857 & mannerly\_jj & 0 & 1291 \\
    1858 & confirmed\_jj & 0 & 1322 \\
    1859 & suspicious\_jj & 0 & 1362 \\
    1860 & variant\_jj & 0 & 1800 \\
    \hline
    \caption{Scores and rankings for most extreme 30 words in component \#52} \\
\end{longtable}


\section{Combined 101 and 438 word list}
\label{app:rankedwordlists:438and101words}
\subsection{Unnormalized PCA}
\label{app:rankedwordlists:438and101words:unnormalized}
\begin{longtable}[!htbp]{| rlr@{.}l |}
    \hline
    \textbf{Rank} & \textbf{Word} & \multicolumn{2}{c|}{\textbf{Score}} \\
    \hline
    \endhead
    1 & sociable\_jj & -2 & -4721 \\
    2 & vivacious\_jj & -2 & -2242 \\
    3 & considerate\_jj & -2 & -1556 \\
    4 & easygoing\_jj & -2 & -764 \\
    5 & witty\_jj & -1 & -7646 \\
    6 & affectionate\_jj & -1 & -7127 \\
    7 & talkative\_jj & -1 & -6901 \\
    8 & gregarious\_jj & -1 & -6307 \\
    9 & down-to-earth\_jj & -1 & -5782 \\
    10 & courteous\_jj & -1 & -5758 \\
    11 & jovial\_jj & -1 & -4264 \\
    12 & extroverted\_jj & -1 & -4091 \\
    13 & cultured\_jj & -1 & -3854 \\
    14 & introspective\_jj & -1 & -3347 \\
    15 & inquisitive\_jj & -1 & -2925 \\
    16 & genial\_jj & -1 & -2254 \\
    17 & mischievous\_jj & -1 & -2082 \\
    18 & amiable\_jj & -1 & -2069 \\
    19 & high-strung\_jj & -1 & -1965 \\
    20 & happy-go-lucky\_jj & -1 & -1839 \\
    21 & perceptive\_jj & -1 & -1494 \\
    22 & humorous\_jj & -1 & -1450 \\
    23 & folksy\_jj & -1 & -1417 \\
    24 & expressive\_jj & -1 & -1359 \\
    25 & kind\_jj & -1 & -953 \\
    26 & playful\_jj & -1 & -752 \\
    27 & cheerful\_jj & -1 & -746 \\
    28 & impetuous\_jj & -1 & -722 \\
    29 & self-pitying\_jj & -1 & -685 \\
    30 & communicative\_jj & -1 & -563 \\
    401 & independence & 1 & 596 \\
    402 & reckless\_jj & 1 & 667 \\
    403 & charitable\_jj & 1 & 748 \\
    404 & prompt\_jj & 1 & 753 \\
    405 & suspicious\_jj & 1 & 766 \\
    406 & decisive\_jj & 1 & 798 \\
    407 & responsible\_jj & 1 & 823 \\
    408 & optimism & 1 & 945 \\
    409 & organization & 1 & 1044 \\
    410 & assertion & 1 & 1053 \\
    411 & direct\_jj & 1 & 1059 \\
    412 & organized\_jj & 1 & 1212 \\
    413 & consistent\_jj & 1 & 1311 \\
    414 & diplomatic\_jj & 1 & 1334 \\
    415 & negligent\_jj & 1 & 1609 \\
    416 & efficiency & 1 & 1770 \\
    417 & reasonable\_jj & 1 & 1772 \\
    418 & systematic\_jj & 1 & 2076 \\
    419 & volatile\_jj & 1 & 2184 \\
    420 & explosive\_jj & 1 & 2392 \\
    421 & reserve & 1 & 2630 \\
    422 & autonomous\_jj & 1 & 2682 \\
    423 & caution & 1 & 2750 \\
    424 & flexibility & 1 & 2778 \\
    425 & sluggish\_jj & 1 & 2850 \\
    426 & intelligence & 1 & 3494 \\
    427 & volatility & 1 & 4393 \\
    428 & cooperation & 1 & 4472 \\
    429 & instability & 1 & 5221 \\
    430 & negligence & 1 & 5715 \\
    \hline
    \caption{Scores and rankings for most extreme 30 words in component \#1} \\
\end{longtable}
\begin{longtable}[!htbp]{| rlr@{.}l |}
    \hline
    \textbf{Rank} & \textbf{Word} & \multicolumn{2}{c|}{\textbf{Score}} \\
    \hline
    \endhead
    1 & callousness & -2 & -338 \\
    2 & selfishness & -1 & -7979 \\
    3 & stupidity & -1 & -6671 \\
    4 & recklessness & -1 & -6551 \\
    5 & gullibility & -1 & -6541 \\
    6 & rudeness & -1 & -6113 \\
    7 & deceit & -1 & -5641 \\
    8 & belligerence & -1 & -5535 \\
    9 & thoughtless\_jj & -1 & -4730 \\
    10 & lethargy & -1 & -4659 \\
    11 & shallowness & -1 & -4635 \\
    12 & passivity & -1 & -4430 \\
    13 & bigoted\_jj & -1 & -3960 \\
    14 & irritability & -1 & -3925 \\
    15 & deceitful\_jj & -1 & -3816 \\
    16 & self-pitying\_jj & -1 & -3189 \\
    17 & stubbornness & -1 & -3172 \\
    18 & indecisiveness & -1 & -3134 \\
    19 & pomposity & -1 & -2972 \\
    20 & inconsiderate\_jj & -1 & -2325 \\
    21 & disorganization & -1 & -2102 \\
    22 & unreflective\_jj & -1 & -1921 \\
    23 & vindictive\_jj & -1 & -1875 \\
    24 & abusive\_jj & -1 & -1486 \\
    25 & selfish\_jj & -1 & -1436 \\
    26 & aloofness & -1 & -1293 \\
    27 & sloth & -1 & -1087 \\
    28 & ungracious\_jj & -1 & -1082 \\
    29 & unintelligent\_jj & -1 & -906 \\
    30 & forgetfulness & -1 & -539 \\
    401 & thorough\_jj & 0 & 9811 \\
    402 & straightforward\_jj & 0 & 9926 \\
    403 & affectionate\_jj & 0 & 9967 \\
    404 & quiet\_jj & 1 & 119 \\
    405 & economical\_jj & 1 & 140 \\
    406 & cautious\_jj & 1 & 444 \\
    407 & pleasant\_jj & 1 & 751 \\
    408 & generous\_jj & 1 & 770 \\
    409 & energetic\_jj & 1 & 900 \\
    410 & gregarious\_jj & 1 & 1371 \\
    411 & cordial\_jj & 1 & 1478 \\
    412 & confident\_jj & 1 & 1640 \\
    413 & kind\_jj & 1 & 1820 \\
    414 & cultured\_jj & 1 & 2010 \\
    415 & enthusiastic\_jj & 1 & 2391 \\
    416 & intelligent\_jj & 1 & 2423 \\
    417 & courteous\_jj & 1 & 2527 \\
    418 & adventurous\_jj & 1 & 2750 \\
    419 & flexible\_jj & 1 & 2810 \\
    420 & vivacious\_jj & 1 & 2961 \\
    421 & optimistic\_jj & 1 & 2982 \\
    422 & warm\_jj & 1 & 3032 \\
    423 & concise\_jj & 1 & 3089 \\
    424 & dependable\_jj & 1 & 3568 \\
    425 & reliable\_jj & 1 & 4280 \\
    426 & efficient\_jj & 1 & 4345 \\
    427 & easygoing\_jj & 1 & 4387 \\
    428 & friendly\_jj & 1 & 4420 \\
    429 & considerate\_jj & 1 & 7592 \\
    430 & sociable\_jj & 2 & 1710 \\
    \hline
    \caption{Scores and rankings for most extreme 30 words in component \#2} \\
\end{longtable}
\begin{longtable}[!htbp]{| rlr@{.}l |}
    \hline
    \textbf{Rank} & \textbf{Word} & \multicolumn{2}{c|}{\textbf{Score}} \\
    \hline
    \endhead
    1 & abusive\_jj & -1 & -9277 \\
    2 & uncooperative\_jj & -1 & -7213 \\
    3 & inconsiderate\_jj & -1 & -3735 \\
    4 & disrespectful\_jj & -1 & -3733 \\
    5 & insensitive\_jj & -1 & -3559 \\
    6 & dishonest\_jj & -1 & -3284 \\
    7 & selfish\_jj & -1 & -3244 \\
    8 & ignorant\_jj & -1 & -3157 \\
    9 & lenient\_jj & -1 & -3049 \\
    10 & unscrupulous\_jj & -1 & -2934 \\
    11 & bigoted\_jj & -1 & -2340 \\
    12 & unfriendly\_jj & -1 & -1981 \\
    13 & pessimistic\_jj & -1 & -1883 \\
    14 & gullible\_jj & -1 & -1674 \\
    15 & unsympathetic\_jj & -1 & -1520 \\
    16 & lazy\_jj & -1 & -1471 \\
    17 & inefficient\_jj & -1 & -1272 \\
    18 & greedy\_jj & -1 & -1220 \\
    19 & vindictive\_jj & -1 & -899 \\
    20 & intrusive\_jj & -1 & -789 \\
    21 & absent-minded\_jj & -1 & -544 \\
    22 & impolite\_jj & -1 & -499 \\
    23 & naïve\_jj & -1 & -366 \\
    24 & rude\_jj & -1 & -340 \\
    25 & unreliable\_jj & 0 & -9975 \\
    26 & deceitful\_jj & 0 & -9648 \\
    27 & negligent\_jj & 0 & -9601 \\
    28 & prejudiced\_jj & 0 & -9599 \\
    29 & distrustful\_jj & 0 & -9138 \\
    30 & thoughtless\_jj & 0 & -8591 \\
    401 & melancholic\_jj & 0 & 9635 \\
    402 & shyness & 0 & 9744 \\
    403 & inhibition & 0 & 9889 \\
    404 & imperturbable\_jj & 0 & 9914 \\
    405 & courage & 1 & 13 \\
    406 & empathy & 1 & 30 \\
    407 & modesty & 1 & 96 \\
    408 & dependability & 1 & 219 \\
    409 & cunning & 1 & 416 \\
    410 & generosity & 1 & 635 \\
    411 & artistic\_jj & 1 & 815 \\
    412 & depth & 1 & 884 \\
    413 & spirit & 1 & 1254 \\
    414 & irritability & 1 & 1866 \\
    415 & aloofness & 1 & 2126 \\
    416 & persistence & 1 & 2241 \\
    417 & precision & 1 & 2288 \\
    418 & meditative\_jj & 1 & 2413 \\
    419 & decisiveness & 1 & 2565 \\
    420 & humor & 1 & 3009 \\
    421 & creativity & 1 & 3080 \\
    422 & lethargy & 1 & 3402 \\
    423 & candor & 1 & 3843 \\
    424 & warmth & 1 & 4407 \\
    425 & sophistication & 1 & 4900 \\
    426 & earthiness & 1 & 5285 \\
    427 & naturalness & 1 & 5530 \\
    428 & expressiveness & 1 & 5797 \\
    429 & spontaneity & 1 & 6509 \\
    430 & playfulness & 1 & 7820 \\
    \hline
    \caption{Scores and rankings for most extreme 30 words in component \#3} \\
\end{longtable}
\begin{longtable}[!htbp]{| rlr@{.}l |}
    \hline
    \textbf{Rank} & \textbf{Word} & \multicolumn{2}{c|}{\textbf{Score}} \\
    \hline
    \endhead
    1 & sincere\_jj & -1 & -6761 \\
    2 & considerate\_jj & -1 & -6056 \\
    3 & dignity & -1 & -4144 \\
    4 & selfless\_jj & -1 & -3510 \\
    5 & courage & -1 & -3383 \\
    6 & courageous\_jj & -1 & -1384 \\
    7 & honest\_jj & -1 & -1285 \\
    8 & principled\_jj & -1 & -888 \\
    9 & stupidity & -1 & -830 \\
    10 & compassionate\_jj & -1 & -644 \\
    11 & selfish\_jj & -1 & -607 \\
    12 & negligence & -1 & -89 \\
    13 & courteous\_jj & 0 & -9773 \\
    14 & candor & 0 & -9766 \\
    15 & moral\_jj & 0 & -9697 \\
    16 & generosity & 0 & -9631 \\
    17 & recklessness & 0 & -9469 \\
    18 & dishonest\_jj & 0 & -9185 \\
    19 & selfishness & 0 & -9097 \\
    20 & empathy & 0 & -9097 \\
    21 & sociable\_jj & 0 & -9091 \\
    22 & negligent\_jj & 0 & -8838 \\
    23 & truthful\_jj & 0 & -8806 \\
    24 & systematic\_jj & 0 & -8553 \\
    25 & deliberate\_jj & 0 & -8419 \\
    26 & modesty & 0 & -8297 \\
    27 & kind\_jj & 0 & -7994 \\
    28 & respectful\_jj & 0 & -7894 \\
    29 & cruelty & 0 & -7876 \\
    30 & ethical\_jj & 0 & -7826 \\
    401 & insecure\_jj & 0 & 6262 \\
    402 & grumpy\_jj & 0 & 6341 \\
    403 & uncooperative\_jj & 0 & 6392 \\
    404 & nonconforming\_jj & 0 & 6399 \\
    405 & high-strung\_jj & 0 & 6588 \\
    406 & placidity & 0 & 6640 \\
    407 & extroverted\_jj & 0 & 6688 \\
    408 & quarrelsome\_jj & 0 & 6838 \\
    409 & dominant\_jj & 0 & 6871 \\
    410 & meditative\_jj & 0 & 6898 \\
    411 & anxious\_jj & 0 & 6939 \\
    412 & aimless\_jj & 0 & 6983 \\
    413 & cranky\_jj & 0 & 7216 \\
    414 & volatility & 0 & 7515 \\
    415 & morose\_jj & 0 & 7638 \\
    416 & forgetful\_jj & 0 & 7650 \\
    417 & volatile\_jj & 0 & 7810 \\
    418 & nervous\_jj & 0 & 8180 \\
    419 & moody\_jj & 0 & 8295 \\
    420 & surly\_jj & 0 & 8347 \\
    421 & cold\_jj & 0 & 8867 \\
    422 & forgetfulness & 0 & 8891 \\
    423 & fretful\_jj & 0 & 9667 \\
    424 & erratic\_jj & 0 & 9985 \\
    425 & lethargic\_jj & 1 & 2088 \\
    426 & sluggish\_jj & 1 & 2140 \\
    427 & lethargy & 1 & 5049 \\
    428 & irritability & 1 & 8700 \\
    429 & irritable\_jj & 1 & 9344 \\
    430 & absent-minded\_jj & 2 & 5715 \\
    \hline
    \caption{Scores and rankings for most extreme 30 words in component \#4} \\
\end{longtable}
\begin{longtable}[!htbp]{| rlr@{.}l |}
    \hline
    \textbf{Rank} & \textbf{Word} & \multicolumn{2}{c|}{\textbf{Score}} \\
    \hline
    \endhead
    1 & irritability & -1 & -6548 \\
    2 & optimism & -1 & -2821 \\
    3 & lethargy & -1 & -2377 \\
    4 & sociable\_jj & -1 & -2003 \\
    5 & cordial\_jj & -1 & -978 \\
    6 & nervous\_jj & -1 & -955 \\
    7 & irritable\_jj & -1 & -903 \\
    8 & sincere\_jj & 0 & -9981 \\
    9 & silence & 0 & -9798 \\
    10 & kind\_jj & 0 & -9499 \\
    11 & pessimistic\_jj & 0 & -8915 \\
    12 & optimistic\_jj & 0 & -8915 \\
    13 & distrust & 0 & -8905 \\
    14 & pessimism & 0 & -8672 \\
    15 & warm\_jj & 0 & -8498 \\
    16 & anxious\_jj & 0 & -8474 \\
    17 & gregarious\_jj & 0 & -8298 \\
    18 & jovial\_jj & 0 & -8085 \\
    19 & instability & 0 & -7984 \\
    20 & considerate\_jj & 0 & -7976 \\
    21 & talkative\_jj & 0 & -7713 \\
    22 & insecurity & 0 & -7687 \\
    23 & fearful\_jj & 0 & -7665 \\
    24 & self-esteem & 0 & -7616 \\
    25 & cautious\_jj & 0 & -7510 \\
    26 & quiet\_jj & 0 & -7403 \\
    27 & polite\_jj & 0 & -7336 \\
    28 & bitter\_jj & 0 & -7249 \\
    29 & fear & 0 & -7248 \\
    30 & caution & 0 & -6898 \\
    401 & creative\_jj & 0 & 6351 \\
    402 & precision & 0 & 6355 \\
    403 & unsophisticated\_jj & 0 & 6466 \\
    404 & perceptive\_jj & 0 & 6513 \\
    405 & unintelligent\_jj & 0 & 6603 \\
    406 & manipulative\_jj & 0 & 6706 \\
    407 & unconventional\_jj & 0 & 6817 \\
    408 & meticulous\_jj & 0 & 6905 \\
    409 & wordy\_jj & 0 & 7307 \\
    410 & expressiveness & 0 & 7314 \\
    411 & insightful\_jj & 0 & 7336 \\
    412 & inefficient\_jj & 0 & 7376 \\
    413 & expressive\_jj & 0 & 7621 \\
    414 & complex\_jj & 0 & 8082 \\
    415 & unimaginative\_jj & 0 & 8157 \\
    416 & exacting\_jj & 0 & 8712 \\
    417 & devious\_jj & 0 & 8968 \\
    418 & cunning\_jj & 0 & 9134 \\
    419 & imaginative\_jj & 0 & 9486 \\
    420 & analytical\_jj & 0 & 9823 \\
    421 & adaptable\_jj & 0 & 9939 \\
    422 & sophisticated\_jj & 1 & 31 \\
    423 & underhanded\_jj & 1 & 568 \\
    424 & inventive\_jj & 1 & 1650 \\
    425 & innovative\_jj & 1 & 1857 \\
    426 & efficient\_jj & 1 & 2365 \\
    427 & concise\_jj & 1 & 4280 \\
    428 & economical\_jj & 1 & 4726 \\
    429 & refined\_jj & 1 & 5831 \\
    430 & absent-minded\_jj & 2 & 6021 \\
    \hline
    \caption{Scores and rankings for most extreme 30 words in component \#5} \\
\end{longtable}
\begin{longtable}[!htbp]{| rlr@{.}l |}
    \hline
    \textbf{Rank} & \textbf{Word} & \multicolumn{2}{c|}{\textbf{Score}} \\
    \hline
    \endhead
    1 & irritability & -3 & -1985 \\
    2 & lethargy & -1 & -9461 \\
    3 & irritable\_jj & -1 & -8816 \\
    4 & economical\_jj & -1 & -6447 \\
    5 & forgetfulness & -1 & -5925 \\
    6 & absent-minded\_jj & -1 & -4462 \\
    7 & abusive\_jj & -1 & -4155 \\
    8 & considerate\_jj & -1 & -3407 \\
    9 & sociable\_jj & -1 & -2110 \\
    10 & self-esteem & -1 & -2039 \\
    11 & adaptable\_jj & -1 & -888 \\
    12 & inhibition & -1 & -9 \\
    13 & communicative\_jj & 0 & -9456 \\
    14 & compassionate\_jj & 0 & -8780 \\
    15 & intelligent\_jj & 0 & -8580 \\
    16 & analytical\_jj & 0 & -8335 \\
    17 & extroverted\_jj & 0 & -8328 \\
    18 & efficient\_jj & 0 & -7950 \\
    19 & empathy & 0 & -7612 \\
    20 & kind\_jj & 0 & -7247 \\
    21 & disorganization & 0 & -7160 \\
    22 & dependability & 0 & -6965 \\
    23 & suggestible\_jj & 0 & -6480 \\
    24 & refined\_jj & 0 & -6141 \\
    25 & reliable\_jj & 0 & -6072 \\
    26 & expressive\_jj & 0 & -6049 \\
    27 & individualistic\_jj & 0 & -6000 \\
    28 & impersonal\_jj & 0 & -5916 \\
    29 & insecurity & 0 & -5900 \\
    30 & selfishness & 0 & -5756 \\
    401 & unassuming\_jj & 0 & 5360 \\
    402 & assertion & 0 & 5536 \\
    403 & earthiness & 0 & 5555 \\
    404 & courtesy & 0 & 5557 \\
    405 & bullheaded\_jj & 0 & 5576 \\
    406 & bold\_jj & 0 & 5603 \\
    407 & defensive\_jj & 0 & 5633 \\
    408 & skeptical\_jj & 0 & 5695 \\
    409 & neat\_jj & 0 & 5768 \\
    410 & merry\_jj & 0 & 5909 \\
    411 & impudent\_jj & 0 & 5911 \\
    412 & zestful\_jj & 0 & 5937 \\
    413 & quiet\_jj & 0 & 6026 \\
    414 & scornful\_jj & 0 & 6124 \\
    415 & rambunctious\_jj & 0 & 6198 \\
    416 & tempestuous\_jj & 0 & 6271 \\
    417 & vain\_jj & 0 & 6361 \\
    418 & homespun\_jj & 0 & 6499 \\
    419 & spirited\_jj & 0 & 6585 \\
    420 & genial\_jj & 0 & 6831 \\
    421 & curt\_jj & 0 & 7104 \\
    422 & caustic\_jj & 0 & 7113 \\
    423 & sly\_jj & 0 & 7176 \\
    424 & gruff\_jj & 0 & 7273 \\
    425 & flamboyant\_jj & 0 & 7399 \\
    426 & reserve & 0 & 7508 \\
    427 & bitter\_jj & 0 & 7684 \\
    428 & somber\_jj & 0 & 8008 \\
    429 & folksy\_jj & 1 & 986 \\
    430 & imperturbable\_jj & 1 & 4209 \\
    \hline
    \caption{Scores and rankings for most extreme 30 words in component \#6} \\
\end{longtable}
\begin{longtable}[!htbp]{| rlr@{.}l |}
    \hline
    \textbf{Rank} & \textbf{Word} & \multicolumn{2}{c|}{\textbf{Score}} \\
    \hline
    \endhead
    1 & sociable\_jj & 0 & -8860 \\
    2 & autonomous\_jj & 0 & -8391 \\
    3 & adventurous\_jj & 0 & -8229 \\
    4 & lazy\_jj & 0 & -7778 \\
    5 & greedy\_jj & 0 & -7629 \\
    6 & unstable\_jj & 0 & -7483 \\
    7 & nonconforming\_jj & 0 & -7380 \\
    8 & gullible\_jj & 0 & -7374 \\
    9 & unscrupulous\_jj & 0 & -6894 \\
    10 & abusive\_jj & 0 & -6507 \\
    11 & happy-go-lucky\_jj & 0 & -6437 \\
    12 & vivacious\_jj & 0 & -6199 \\
    13 & rebellious\_jj & 0 & -6175 \\
    14 & jealous\_jj & 0 & -6082 \\
    15 & cranky\_jj & 0 & -6036 \\
    16 & neat\_jj & 0 & -5958 \\
    17 & cultured\_jj & 0 & -5871 \\
    18 & natural\_jj & 0 & -5754 \\
    19 & devious\_jj & 0 & -5701 \\
    20 & carefree\_jj & 0 & -5692 \\
    21 & shallow\_jj & 0 & -5584 \\
    22 & insecure\_jj & 0 & -5441 \\
    23 & intelligent\_jj & 0 & -5440 \\
    24 & cunning\_jj & 0 & -5373 \\
    25 & inefficient\_jj & 0 & -5351 \\
    26 & sophisticated\_jj & 0 & -5347 \\
    27 & selfish\_jj & 0 & -5332 \\
    28 & animation & 0 & -5303 \\
    29 & conventional\_jj & 0 & -5296 \\
    30 & creative\_jj & 0 & -5270 \\
    401 & surly\_jj & 0 & 5836 \\
    402 & assertive\_jj & 0 & 5838 \\
    403 & sincere\_jj & 0 & 5871 \\
    404 & touchy\_jj & 0 & 6107 \\
    405 & accommodating\_jj & 0 & 6134 \\
    406 & restrained\_jj & 0 & 6166 \\
    407 & punctual\_jj & 0 & 6207 \\
    408 & courage & 0 & 6239 \\
    409 & lenient\_jj & 0 & 6264 \\
    410 & polite\_jj & 0 & 6414 \\
    411 & imperturbable\_jj & 0 & 6656 \\
    412 & combative\_jj & 0 & 6926 \\
    413 & principled\_jj & 0 & 7027 \\
    414 & careful\_jj & 0 & 7037 \\
    415 & tactful\_jj & 0 & 7129 \\
    416 & unemotional\_jj & 0 & 7340 \\
    417 & truthful\_jj & 0 & 7542 \\
    418 & decisiveness & 0 & 7907 \\
    419 & uncooperative\_jj & 0 & 8478 \\
    420 & forceful\_jj & 0 & 8568 \\
    421 & leniency & 0 & 8833 \\
    422 & curt\_jj & 0 & 8984 \\
    423 & concise\_jj & 0 & 9064 \\
    424 & frank\_jj & 0 & 9704 \\
    425 & respectful\_jj & 1 & 76 \\
    426 & candor & 1 & 704 \\
    427 & prompt\_jj & 1 & 795 \\
    428 & belligerence & 1 & 1768 \\
    429 & cordial\_jj & 1 & 4820 \\
    430 & absent-minded\_jj & 4 & 9876 \\
    \hline
    \caption{Scores and rankings for most extreme 30 words in component \#7} \\
\end{longtable}
\begin{longtable}[!htbp]{| rlr@{.}l |}
    \hline
    \textbf{Rank} & \textbf{Word} & \multicolumn{2}{c|}{\textbf{Score}} \\
    \hline
    \endhead
    1 & absent-minded\_jj & -5 & -4840 \\
    2 & charitable\_jj & -1 & -317 \\
    3 & proud\_jj & 0 & -9306 \\
    4 & kind\_jj & 0 & -8834 \\
    5 & unscrupulous\_jj & 0 & -8302 \\
    6 & sociable\_jj & 0 & -7824 \\
    7 & lazy\_jj & 0 & -7785 \\
    8 & vivacious\_jj & 0 & -7460 \\
    9 & inconsiderate\_jj & 0 & -7401 \\
    10 & brave\_jj & 0 & -7035 \\
    11 & courtesy & 0 & -6718 \\
    12 & gullible\_jj & 0 & -6645 \\
    13 & generosity & 0 & -6445 \\
    14 & greedy\_jj & 0 & -6347 \\
    15 & spirit & 0 & -5956 \\
    16 & shy\_jj & 0 & -5647 \\
    17 & negligence & 0 & -5640 \\
    18 & envious\_jj & 0 & -5627 \\
    19 & suspicious\_jj & 0 & -5598 \\
    20 & insight & 0 & -5577 \\
    21 & intelligent\_jj & 0 & -5453 \\
    22 & cranky\_jj & 0 & -5414 \\
    23 & courage & 0 & -5383 \\
    24 & confident\_jj & 0 & -5242 \\
    25 & stupidity & 0 & -5211 \\
    26 & fear & 0 & -5178 \\
    27 & ignorant\_jj & 0 & -5083 \\
    28 & jealous\_jj & 0 & -4953 \\
    29 & organization & 0 & -4938 \\
    30 & selfless\_jj & 0 & -4804 \\
    401 & accommodating\_jj & 0 & 4979 \\
    402 & self-critical\_jj & 0 & 4981 \\
    403 & erratic\_jj & 0 & 5115 \\
    404 & indecisiveness & 0 & 5189 \\
    405 & argumentative\_jj & 0 & 5203 \\
    406 & insensitive\_jj & 0 & 5330 \\
    407 & inconsistent\_jj & 0 & 5450 \\
    408 & dignified\_jj & 0 & 5484 \\
    409 & somber\_jj & 0 & 5713 \\
    410 & meditative\_jj & 0 & 5983 \\
    411 & predictable\_jj & 0 & 6076 \\
    412 & restrained\_jj & 0 & 6119 \\
    413 & forceful\_jj & 0 & 6244 \\
    414 & verbose\_jj & 0 & 6565 \\
    415 & unemotional\_jj & 0 & 6662 \\
    416 & respectful\_jj & 0 & 6777 \\
    417 & intrusive\_jj & 0 & 6913 \\
    418 & caustic\_jj & 0 & 7022 \\
    419 & combative\_jj & 0 & 7095 \\
    420 & abusive\_jj & 0 & 7183 \\
    421 & belligerence & 0 & 7296 \\
    422 & lenient\_jj & 0 & 7346 \\
    423 & assertive\_jj & 0 & 7642 \\
    424 & concise\_jj & 0 & 7861 \\
    425 & refined\_jj & 0 & 8073 \\
    426 & antagonistic\_jj & 0 & 9870 \\
    427 & cordial\_jj & 1 & 638 \\
    428 & economical\_jj & 1 & 884 \\
    429 & surly\_jj & 1 & 1702 \\
    430 & uncooperative\_jj & 1 & 6890 \\
    \hline
    \caption{Scores and rankings for most extreme 30 words in component \#8} \\
\end{longtable}
\begin{longtable}[!htbp]{| rlr@{.}l |}
    \hline
    \textbf{Rank} & \textbf{Word} & \multicolumn{2}{c|}{\textbf{Score}} \\
    \hline
    \endhead
    1 & distrustful\_jj & -1 & -6725 \\
    2 & individualistic\_jj & -1 & -3041 \\
    3 & belligerence & -1 & -1503 \\
    4 & distrust & -1 & -1426 \\
    5 & aloofness & -1 & -1417 \\
    6 & assertive\_jj & -1 & -1072 \\
    7 & autonomous\_jj & -1 & -925 \\
    8 & antagonistic\_jj & -1 & -61 \\
    9 & dependability & 0 & -8655 \\
    10 & accommodating\_jj & 0 & -8568 \\
    11 & independence & 0 & -8236 \\
    12 & uncritical\_jj & 0 & -8227 \\
    13 & insecure\_jj & 0 & -7289 \\
    14 & obstinate\_jj & 0 & -7232 \\
    15 & conscientious\_jj & 0 & -7029 \\
    16 & cosmopolitan\_jj & 0 & -6965 \\
    17 & independent\_jj & 0 & -6598 \\
    18 & cooperation & 0 & -6562 \\
    19 & insecurity & 0 & -6490 \\
    20 & pessimistic\_jj & 0 & -6310 \\
    21 & worldly\_jj & 0 & -6201 \\
    22 & instability & 0 & -6200 \\
    23 & prejudiced\_jj & 0 & -6177 \\
    24 & adaptable\_jj & 0 & -6119 \\
    25 & unfriendly\_jj & 0 & -6091 \\
    26 & generosity & 0 & -6028 \\
    27 & ambition & 0 & -5981 \\
    28 & extroverted\_jj & 0 & -5850 \\
    29 & stubbornness & 0 & -5834 \\
    30 & dominant\_jj & 0 & -5819 \\
    401 & systematic\_jj & 0 & 6487 \\
    402 & spontaneous\_jj & 0 & 6734 \\
    403 & lazy\_jj & 0 & 6835 \\
    404 & lethargy & 0 & 6851 \\
    405 & humorous\_jj & 0 & 6888 \\
    406 & cruel\_jj & 0 & 6908 \\
    407 & rash\_jj & 0 & 6981 \\
    408 & straightforward\_jj & 0 & 7043 \\
    409 & thorough\_jj & 0 & 7105 \\
    410 & meditative\_jj & 0 & 7130 \\
    411 & cruelty & 0 & 7308 \\
    412 & verbal\_jj & 0 & 7337 \\
    413 & abusive\_jj & 0 & 7419 \\
    414 & caustic\_jj & 0 & 7451 \\
    415 & warm\_jj & 0 & 7544 \\
    416 & playful\_jj & 0 & 7576 \\
    417 & simple\_jj & 0 & 8056 \\
    418 & reckless\_jj & 0 & 8062 \\
    419 & sloppy\_jj & 0 & 8914 \\
    420 & neat\_jj & 0 & 8916 \\
    421 & folksy\_jj & 0 & 9023 \\
    422 & forgetfulness & 0 & 9464 \\
    423 & imperturbable\_jj & 1 & 153 \\
    424 & prompt\_jj & 1 & 1001 \\
    425 & deliberate\_jj & 1 & 1588 \\
    426 & concise\_jj & 1 & 2148 \\
    427 & careless\_jj & 1 & 2543 \\
    428 & negligence & 1 & 2996 \\
    429 & irritability & 1 & 4359 \\
    430 & negligent\_jj & 1 & 4573 \\
    \hline
    \caption{Scores and rankings for most extreme 30 words in component \#9} \\
\end{longtable}
\begin{longtable}[!htbp]{| rlr@{.}l |}
    \hline
    \textbf{Rank} & \textbf{Word} & \multicolumn{2}{c|}{\textbf{Score}} \\
    \hline
    \endhead
    1 & uncooperative\_jj & -2 & -9863 \\
    2 & abusive\_jj & -2 & -740 \\
    3 & surly\_jj & -1 & -9411 \\
    4 & negligent\_jj & -1 & -6512 \\
    5 & negligence & -1 & -1984 \\
    6 & cruelty & -1 & -1299 \\
    7 & unscrupulous\_jj & 0 & -8978 \\
    8 & friendly\_jj & 0 & -7589 \\
    9 & cooperation & 0 & -7314 \\
    10 & informal\_jj & 0 & -7261 \\
    11 & organization & 0 & -6858 \\
    12 & organized\_jj & 0 & -6782 \\
    13 & independent\_jj & 0 & -6756 \\
    14 & absent-minded\_jj & 0 & -6752 \\
    15 & easygoing\_jj & 0 & -6696 \\
    16 & orderly\_jj & 0 & -6572 \\
    17 & systematic\_jj & 0 & -6394 \\
    18 & unrestrained\_jj & 0 & -6122 \\
    19 & cooperative\_jj & 0 & -6069 \\
    20 & autonomous\_jj & 0 & -6032 \\
    21 & exacting\_jj & 0 & -5895 \\
    22 & spirit & 0 & -5885 \\
    23 & sociable\_jj & 0 & -5849 \\
    24 & leniency & 0 & -5749 \\
    25 & earthiness & 0 & -5714 \\
    26 & explosive\_jj & 0 & -5506 \\
    27 & cordial\_jj & 0 & -5445 \\
    28 & charitable\_jj & 0 & -5345 \\
    29 & earthy\_jj & 0 & -5329 \\
    30 & recklessness & 0 & -5090 \\
    401 & irritability & 0 & 5302 \\
    402 & logical\_jj & 0 & 5329 \\
    403 & careful\_jj & 0 & 5345 \\
    404 & optimism & 0 & 5352 \\
    405 & perceptive\_jj & 0 & 5445 \\
    406 & insight & 0 & 5487 \\
    407 & gullibility & 0 & 5510 \\
    408 & irritable\_jj & 0 & 5632 \\
    409 & flippant\_jj & 0 & 5656 \\
    410 & unimaginative\_jj & 0 & 5659 \\
    411 & nervous\_jj & 0 & 5780 \\
    412 & philosophical\_jj & 0 & 5818 \\
    413 & sluggish\_jj & 0 & 5835 \\
    414 & shallowness & 0 & 5910 \\
    415 & pessimism & 0 & 6091 \\
    416 & ignorant\_jj & 0 & 6316 \\
    417 & confident\_jj & 0 & 6413 \\
    418 & insightful\_jj & 0 & 6650 \\
    419 & gullible\_jj & 0 & 7008 \\
    420 & lazy\_jj & 0 & 7025 \\
    421 & lethargy & 0 & 7028 \\
    422 & skeptical\_jj & 0 & 7727 \\
    423 & naïve\_jj & 0 & 7760 \\
    424 & cynical\_jj & 0 & 7914 \\
    425 & wishy-washy\_jj & 0 & 8016 \\
    426 & foolhardy\_jj & 0 & 8084 \\
    427 & optimistic\_jj & 0 & 9086 \\
    428 & cautious\_jj & 0 & 9748 \\
    429 & pessimistic\_jj & 1 & 4010 \\
    430 & concise\_jj & 1 & 5247 \\
    \hline
    \caption{Scores and rankings for most extreme 30 words in component \#10} \\
\end{longtable}
\begin{longtable}[!htbp]{| rlr@{.}l |}
    \hline
    \textbf{Rank} & \textbf{Word} & \multicolumn{2}{c|}{\textbf{Score}} \\
    \hline
    \endhead
    1 & abusive\_jj & -1 & -8538 \\
    2 & nonconforming\_jj & 0 & -9545 \\
    3 & touchy\_jj & 0 & -7406 \\
    4 & insight & 0 & -7274 \\
    5 & understanding & 0 & -6990 \\
    6 & warm\_jj & 0 & -6912 \\
    7 & diplomatic\_jj & 0 & -6608 \\
    8 & formal\_jj & 0 & -6326 \\
    9 & wordy\_jj & 0 & -6102 \\
    10 & unkind\_jj & 0 & -5715 \\
    11 & flippant\_jj & 0 & -5667 \\
    12 & silent\_jj & 0 & -5539 \\
    13 & concise\_jj & 0 & -5526 \\
    14 & philosophical\_jj & 0 & -5267 \\
    15 & morality & 0 & -5219 \\
    16 & nonconformity & 0 & -5137 \\
    17 & curious\_jj & 0 & -4994 \\
    18 & practical\_jj & 0 & -4985 \\
    19 & truthful\_jj & 0 & -4984 \\
    20 & intellectuality & 0 & -4913 \\
    21 & melancholic\_jj & 0 & -4879 \\
    22 & logic & 0 & -4866 \\
    23 & bashful\_jj & 0 & -4855 \\
    24 & curt\_jj & 0 & -4846 \\
    25 & trustful\_jj & 0 & -4758 \\
    26 & traditional\_jj & 0 & -4711 \\
    27 & intrusiveness & 0 & -4672 \\
    28 & prompt\_jj & 0 & -4650 \\
    29 & direct\_jj & 0 & -4644 \\
    30 & helpful\_jj & 0 & -4606 \\
    401 & combative\_jj & 0 & 4977 \\
    402 & courageous\_jj & 0 & 4993 \\
    403 & unscrupulous\_jj & 0 & 5102 \\
    404 & gregarious\_jj & 0 & 5298 \\
    405 & inefficient\_jj & 0 & 5318 \\
    406 & lethargy & 0 & 5330 \\
    407 & indecisiveness & 0 & 5354 \\
    408 & deceit & 0 & 5399 \\
    409 & stubborn\_jj & 0 & 5467 \\
    410 & optimism & 0 & 5652 \\
    411 & deceitful\_jj & 0 & 5703 \\
    412 & unpredictable\_jj & 0 & 5797 \\
    413 & impetuous\_jj & 0 & 5922 \\
    414 & sluggish\_jj & 0 & 6143 \\
    415 & underhanded\_jj & 0 & 6196 \\
    416 & adventurous\_jj & 0 & 6276 \\
    417 & lethargic\_jj & 0 & 6522 \\
    418 & easygoing\_jj & 0 & 6674 \\
    419 & assertive\_jj & 0 & 7046 \\
    420 & energetic\_jj & 0 & 7439 \\
    421 & negligence & 0 & 7957 \\
    422 & erratic\_jj & 0 & 9013 \\
    423 & obstinate\_jj & 0 & 9098 \\
    424 & tenacious\_jj & 0 & 9152 \\
    425 & economical\_jj & 0 & 9990 \\
    426 & negligent\_jj & 1 & 596 \\
    427 & indecisive\_jj & 1 & 648 \\
    428 & recklessness & 1 & 1007 \\
    429 & reckless\_jj & 1 & 1738 \\
    430 & imperturbable\_jj & 4 & 2952 \\
    \hline
    \caption{Scores and rankings for most extreme 30 words in component \#11} \\
\end{longtable}
\begin{longtable}[!htbp]{| rlr@{.}l |}
    \hline
    \textbf{Rank} & \textbf{Word} & \multicolumn{2}{c|}{\textbf{Score}} \\
    \hline
    \endhead
    1 & abusive\_jj & -4 & -1617 \\
    2 & explosive\_jj & -1 & -1979 \\
    3 & expressive\_jj & -1 & -158 \\
    4 & erratic\_jj & 0 & -8488 \\
    5 & unpredictable\_jj & 0 & -8016 \\
    6 & suspicious\_jj & 0 & -7390 \\
    7 & inventive\_jj & 0 & -7262 \\
    8 & insight & 0 & -6688 \\
    9 & concise\_jj & 0 & -6672 \\
    10 & unstable\_jj & 0 & -6374 \\
    11 & perceptive\_jj & 0 & -6182 \\
    12 & forceful\_jj & 0 & -6107 \\
    13 & assertive\_jj & 0 & -6062 \\
    14 & skeptical\_jj & 0 & -5923 \\
    15 & emotional\_jj & 0 & -5884 \\
    16 & insightful\_jj & 0 & -5809 \\
    17 & tenacious\_jj & 0 & -5418 \\
    18 & energetic\_jj & 0 & -5321 \\
    19 & manipulative\_jj & 0 & -5318 \\
    20 & witty\_jj & 0 & -5252 \\
    21 & tempestuous\_jj & 0 & -5221 \\
    22 & verbal\_jj & 0 & -5154 \\
    23 & optimistic\_jj & 0 & -5091 \\
    24 & insecure\_jj & 0 & -5039 \\
    25 & stubbornness & 0 & -4820 \\
    26 & rebellious\_jj & 0 & -4797 \\
    27 & belligerence & 0 & -4757 \\
    28 & earthiness & 0 & -4719 \\
    29 & absent-minded\_jj & 0 & -4717 \\
    30 & introspective\_jj & 0 & -4697 \\
    401 & courteous\_jj & 0 & 4509 \\
    402 & frivolity & 0 & 4542 \\
    403 & thrifty\_jj & 0 & 4558 \\
    404 & careless\_jj & 0 & 4743 \\
    405 & impractical\_jj & 0 & 4780 \\
    406 & unsociable\_jj & 0 & 4846 \\
    407 & patient\_jj & 0 & 4930 \\
    408 & obliging\_jj & 0 & 4997 \\
    409 & meddlesome\_jj & 0 & 5002 \\
    410 & rudeness & 0 & 5349 \\
    411 & nonconforming\_jj & 0 & 5399 \\
    412 & frivolous\_jj & 0 & 5430 \\
    413 & pleasant\_jj & 0 & 5444 \\
    414 & crabby\_jj & 0 & 5577 \\
    415 & leniency & 0 & 6162 \\
    416 & neat\_jj & 0 & 6179 \\
    417 & mannerly\_jj & 0 & 6240 \\
    418 & miserly\_jj & 0 & 6397 \\
    419 & cordial\_jj & 0 & 7007 \\
    420 & bossiness & 0 & 7168 \\
    421 & lenient\_jj & 0 & 7336 \\
    422 & inefficient\_jj & 0 & 7548 \\
    423 & efficient\_jj & 0 & 7700 \\
    424 & punctual\_jj & 0 & 8028 \\
    425 & prompt\_jj & 0 & 8522 \\
    426 & inconsiderate\_jj & 0 & 8869 \\
    427 & efficiency & 0 & 8942 \\
    428 & punctuality & 0 & 9515 \\
    429 & economical\_jj & 1 & 215 \\
    430 & friendly\_jj & 1 & 1130 \\
    \hline
    \caption{Scores and rankings for most extreme 30 words in component \#12} \\
\end{longtable}
\begin{longtable}[!htbp]{| rlr@{.}l |}
    \hline
    \textbf{Rank} & \textbf{Word} & \multicolumn{2}{c|}{\textbf{Score}} \\
    \hline
    \endhead
    1 & economical\_jj & -1 & -1145 \\
    2 & refined\_jj & -1 & -96 \\
    3 & warm\_jj & 0 & -8946 \\
    4 & thrifty\_jj & 0 & -7016 \\
    5 & cold\_jj & 0 & -6921 \\
    6 & stingy\_jj & 0 & -6893 \\
    7 & stubbornness & 0 & -6380 \\
    8 & belligerence & 0 & -6376 \\
    9 & erratic\_jj & 0 & -6232 \\
    10 & sluggish\_jj & 0 & -6224 \\
    11 & absent-minded\_jj & 0 & -6211 \\
    12 & dependable\_jj & 0 & -5933 \\
    13 & cunning & 0 & -5929 \\
    14 & crafty\_jj & 0 & -5830 \\
    15 & sloppy\_jj & 0 & -5552 \\
    16 & homespun\_jj & 0 & -5480 \\
    17 & miserly\_jj & 0 & -5469 \\
    18 & cruel\_jj & 0 & -5460 \\
    19 & courage & 0 & -5418 \\
    20 & warmth & 0 & -5369 \\
    21 & friendly\_jj & 0 & -5269 \\
    22 & ruthless\_jj & 0 & -5106 \\
    23 & rude\_jj & 0 & -5020 \\
    24 & cunning\_jj & 0 & -5013 \\
    25 & down-to-earth\_jj & 0 & -4966 \\
    26 & lethargic\_jj & 0 & -4851 \\
    27 & efficient\_jj & 0 & -4771 \\
    28 & neat\_jj & 0 & -4685 \\
    29 & smart\_jj & 0 & -4662 \\
    30 & decisiveness & 0 & -4661 \\
    401 & irritability & 0 & 4846 \\
    402 & philosophical\_jj & 0 & 4887 \\
    403 & silence & 0 & 4951 \\
    404 & orderly\_jj & 0 & 4963 \\
    405 & wordy\_jj & 0 & 4969 \\
    406 & unreflective\_jj & 0 & 4982 \\
    407 & uncritical\_jj & 0 & 5008 \\
    408 & unscrupulous\_jj & 0 & 5056 \\
    409 & nonconforming\_jj & 0 & 5086 \\
    410 & meditative\_jj & 0 & 5095 \\
    411 & ethical\_jj & 0 & 5124 \\
    412 & organized\_jj & 0 & 5450 \\
    413 & intellectuality & 0 & 5759 \\
    414 & trustful\_jj & 0 & 5812 \\
    415 & informal\_jj & 0 & 5851 \\
    416 & antagonistic\_jj & 0 & 5949 \\
    417 & self-critical\_jj & 0 & 5971 \\
    418 & organization & 0 & 6006 \\
    419 & independence & 0 & 6206 \\
    420 & unkind\_jj & 0 & 6334 \\
    421 & intelligence & 0 & 6591 \\
    422 & suspicious\_jj & 0 & 6679 \\
    423 & impudent\_jj & 0 & 7056 \\
    424 & uncharitable\_jj & 0 & 7423 \\
    425 & understanding\_jj & 0 & 7842 \\
    426 & autonomous\_jj & 0 & 8486 \\
    427 & insight & 0 & 8791 \\
    428 & independent\_jj & 0 & 9834 \\
    429 & concise\_jj & 1 & 2498 \\
    430 & imperturbable\_jj & 3 & 8420 \\
    \hline
    \caption{Scores and rankings for most extreme 30 words in component \#13} \\
\end{longtable}
\begin{longtable}[!htbp]{| rlr@{.}l |}
    \hline
    \textbf{Rank} & \textbf{Word} & \multicolumn{2}{c|}{\textbf{Score}} \\
    \hline
    \endhead
    1 & abusive\_jj & -2 & -4175 \\
    2 & brave\_jj & -1 & -1278 \\
    3 & selfless\_jj & 0 & -8956 \\
    4 & imperturbable\_jj & 0 & -8567 \\
    5 & thrifty\_jj & 0 & -8466 \\
    6 & conscientious\_jj & 0 & -8278 \\
    7 & courage & 0 & -8181 \\
    8 & lenient\_jj & 0 & -7913 \\
    9 & nonconformity & 0 & -7217 \\
    10 & dignity & 0 & -7201 \\
    11 & unreflective\_jj & 0 & -6734 \\
    12 & accommodating\_jj & 0 & -6682 \\
    13 & courageous\_jj & 0 & -6553 \\
    14 & reserve & 0 & -6210 \\
    15 & dignified\_jj & 0 & -6023 \\
    16 & thrift & 0 & -5753 \\
    17 & principled\_jj & 0 & -5591 \\
    18 & economical\_jj & 0 & -5476 \\
    19 & restrained\_jj & 0 & -5365 \\
    20 & expressiveness & 0 & -5256 \\
    21 & individualistic\_jj & 0 & -5167 \\
    22 & stingy\_jj & 0 & -5152 \\
    23 & vain\_jj & 0 & -5140 \\
    24 & quiet & 0 & -5113 \\
    25 & charitable\_jj & 0 & -5050 \\
    26 & selfish\_jj & 0 & -5035 \\
    27 & modesty & 0 & -4949 \\
    28 & meditative\_jj & 0 & -4883 \\
    29 & understanding\_jj & 0 & -4722 \\
    30 & refined\_jj & 0 & -4602 \\
    401 & sociable\_jj & 0 & 5382 \\
    402 & secretive\_jj & 0 & 5463 \\
    403 & insecurity & 0 & 5467 \\
    404 & witty\_jj & 0 & 5507 \\
    405 & deliberate\_jj & 0 & 5525 \\
    406 & friendly\_jj & 0 & 5604 \\
    407 & touchy\_jj & 0 & 5624 \\
    408 & sly\_jj & 0 & 5653 \\
    409 & sophisticated\_jj & 0 & 5681 \\
    410 & caustic\_jj & 0 & 5726 \\
    411 & organized\_jj & 0 & 5731 \\
    412 & unstable\_jj & 0 & 6003 \\
    413 & humorous\_jj & 0 & 6136 \\
    414 & cooperation & 0 & 6163 \\
    415 & unsophisticated\_jj & 0 & 6394 \\
    416 & humor & 0 & 6716 \\
    417 & unreliable\_jj & 0 & 6823 \\
    418 & easygoing\_jj & 0 & 6846 \\
    419 & deceit & 0 & 7201 \\
    420 & folksy\_jj & 0 & 8691 \\
    421 & diplomatic\_jj & 0 & 8786 \\
    422 & belligerence & 0 & 8830 \\
    423 & suspicious\_jj & 0 & 8858 \\
    424 & distrust & 0 & 9526 \\
    425 & cordial\_jj & 1 & 55 \\
    426 & intelligence & 1 & 551 \\
    427 & unscrupulous\_jj & 1 & 848 \\
    428 & sophistication & 1 & 1021 \\
    429 & instability & 1 & 2699 \\
    430 & explosive\_jj & 1 & 4476 \\
    \hline
    \caption{Scores and rankings for most extreme 30 words in component \#14} \\
\end{longtable}
\begin{longtable}[!htbp]{| rlr@{.}l |}
    \hline
    \textbf{Rank} & \textbf{Word} & \multicolumn{2}{c|}{\textbf{Score}} \\
    \hline
    \endhead
    1 & abusive\_jj & -2 & -4727 \\
    2 & imperturbable\_jj & -1 & -4865 \\
    3 & dependability & -1 & -3596 \\
    4 & punctuality & -1 & -2536 \\
    5 & efficiency & -1 & -1917 \\
    6 & insight & 0 & -9725 \\
    7 & reliable\_jj & 0 & -8745 \\
    8 & stingy\_jj & 0 & -7680 \\
    9 & ungracious\_jj & 0 & -7647 \\
    10 & economical\_jj & 0 & -7343 \\
    11 & unreliable\_jj & 0 & -6791 \\
    12 & assured\_jj & 0 & -6588 \\
    13 & envious\_jj & 0 & -6394 \\
    14 & candor & 0 & -6256 \\
    15 & punctual\_jj & 0 & -6092 \\
    16 & courtesy & 0 & -5875 \\
    17 & efficient\_jj & 0 & -5794 \\
    18 & patient\_jj & 0 & -5563 \\
    19 & refined\_jj & 0 & -5500 \\
    20 & defensive\_jj & 0 & -5455 \\
    21 & optimism & 0 & -5389 \\
    22 & courteous\_jj & 0 & -5346 \\
    23 & unkind\_jj & 0 & -5213 \\
    24 & predictability & 0 & -5207 \\
    25 & volatility & 0 & -5150 \\
    26 & rude\_jj & 0 & -4988 \\
    27 & shy\_jj & 0 & -4895 \\
    28 & smart\_jj & 0 & -4871 \\
    29 & dependable\_jj & 0 & -4820 \\
    30 & warmth & 0 & -4804 \\
    401 & lenient\_jj & 0 & 4873 \\
    402 & cosmopolitan\_jj & 0 & 4945 \\
    403 & prejudice & 0 & 5222 \\
    404 & expressiveness & 0 & 5378 \\
    405 & forceful\_jj & 0 & 5405 \\
    406 & distrustful\_jj & 0 & 5412 \\
    407 & assertive\_jj & 0 & 5494 \\
    408 & docile\_jj & 0 & 5573 \\
    409 & unstable\_jj & 0 & 5623 \\
    410 & contemplative\_jj & 0 & 5634 \\
    411 & principled\_jj & 0 & 5681 \\
    412 & lethargy & 0 & 5702 \\
    413 & folksy\_jj & 0 & 5796 \\
    414 & organized\_jj & 0 & 5918 \\
    415 & cruel\_jj & 0 & 5936 \\
    416 & negligent\_jj & 0 & 6110 \\
    417 & expressive\_jj & 0 & 6152 \\
    418 & systematic\_jj & 0 & 6156 \\
    419 & somber\_jj & 0 & 6213 \\
    420 & philosophical\_jj & 0 & 6221 \\
    421 & meditative\_jj & 0 & 6957 \\
    422 & vigorous\_jj & 0 & 7095 \\
    423 & prejudiced\_jj & 0 & 7198 \\
    424 & nonconformity & 0 & 7666 \\
    425 & cruelty & 0 & 8355 \\
    426 & inventive\_jj & 0 & 8630 \\
    427 & individualistic\_jj & 0 & 8786 \\
    428 & deliberate\_jj & 0 & 9902 \\
    429 & rebellious\_jj & 1 & 305 \\
    430 & peaceful\_jj & 1 & 526 \\
    \hline
    \caption{Scores and rankings for most extreme 30 words in component \#15} \\
\end{longtable}
\begin{longtable}[!htbp]{| rlr@{.}l |}
    \hline
    \textbf{Rank} & \textbf{Word} & \multicolumn{2}{c|}{\textbf{Score}} \\
    \hline
    \endhead
    1 & negligent\_jj & -1 & -8284 \\
    2 & surly\_jj & -1 & -1318 \\
    3 & negligence & -1 & -210 \\
    4 & pessimistic\_jj & 0 & -9933 \\
    5 & uncooperative\_jj & 0 & -8930 \\
    6 & optimistic\_jj & 0 & -8874 \\
    7 & leniency & 0 & -8125 \\
    8 & careless\_jj & 0 & -7878 \\
    9 & insight & 0 & -7448 \\
    10 & animation & 0 & -7156 \\
    11 & perceptive\_jj & 0 & -6561 \\
    12 & dependability & 0 & -6111 \\
    13 & sophistication & 0 & -6068 \\
    14 & indecisive\_jj & 0 & -5745 \\
    15 & conscientious\_jj & 0 & -5669 \\
    16 & adventurous\_jj & 0 & -5628 \\
    17 & inquisitive\_jj & 0 & -5600 \\
    18 & truthful\_jj & 0 & -5582 \\
    19 & unemotional\_jj & 0 & -5509 \\
    20 & cautious\_jj & 0 & -5468 \\
    21 & high-strung\_jj & 0 & -5466 \\
    22 & skeptical\_jj & 0 & -5317 \\
    23 & lenient\_jj & 0 & -5174 \\
    24 & forgetful\_jj & 0 & -5172 \\
    25 & inventive\_jj & 0 & -5128 \\
    26 & boastful\_jj & 0 & -5005 \\
    27 & morose\_jj & 0 & -4967 \\
    28 & reckless\_jj & 0 & -4951 \\
    29 & caution & 0 & -4948 \\
    30 & analytical\_jj & 0 & -4933 \\
    401 & volatility & 0 & 4469 \\
    402 & touchy\_jj & 0 & 4491 \\
    403 & passivity & 0 & 4514 \\
    404 & prejudice & 0 & 4656 \\
    405 & quarrelsome\_jj & 0 & 4707 \\
    406 & folksy\_jj & 0 & 4715 \\
    407 & insecurity & 0 & 5065 \\
    408 & cold\_jj & 0 & 5419 \\
    409 & independence & 0 & 5460 \\
    410 & distrust & 0 & 5610 \\
    411 & pleasant\_jj & 0 & 5658 \\
    412 & natural\_jj & 0 & 5718 \\
    413 & unfriendly\_jj & 0 & 5774 \\
    414 & harsh\_jj & 0 & 5794 \\
    415 & orderly\_jj & 0 & 6197 \\
    416 & bitter\_jj & 0 & 6258 \\
    417 & silence & 0 & 6416 \\
    418 & dignified\_jj & 0 & 6793 \\
    419 & bigoted\_jj & 0 & 6931 \\
    420 & friendly\_jj & 0 & 7456 \\
    421 & cruel\_jj & 0 & 7465 \\
    422 & volatile\_jj & 0 & 8001 \\
    423 & economical\_jj & 0 & 8638 \\
    424 & instability & 0 & 8741 \\
    425 & absent-minded\_jj & 0 & 8890 \\
    426 & peaceful\_jj & 0 & 9060 \\
    427 & warm\_jj & 1 & 1656 \\
    428 & refined\_jj & 1 & 4701 \\
    429 & abusive\_jj & 1 & 9150 \\
    430 & imperturbable\_jj & 2 & 430 \\
    \hline
    \caption{Scores and rankings for most extreme 30 words in component \#16} \\
\end{longtable}
\begin{longtable}[!htbp]{| rlr@{.}l |}
    \hline
    \textbf{Rank} & \textbf{Word} & \multicolumn{2}{c|}{\textbf{Score}} \\
    \hline
    \endhead
    1 & uncooperative\_jj & -1 & -6862 \\
    2 & surly\_jj & -1 & -6508 \\
    3 & pessimism & -1 & -3595 \\
    4 & pessimistic\_jj & -1 & -1632 \\
    5 & optimism & 0 & -9353 \\
    6 & optimistic\_jj & 0 & -8904 \\
    7 & imperturbable\_jj & 0 & -8094 \\
    8 & contemplative\_jj & 0 & -7418 \\
    9 & meditative\_jj & 0 & -6557 \\
    10 & frivolity & 0 & -6520 \\
    11 & cruel\_jj & 0 & -6507 \\
    12 & lenient\_jj & 0 & -6494 \\
    13 & sophistication & 0 & -6477 \\
    14 & absent-minded\_jj & 0 & -6350 \\
    15 & insight & 0 & -6292 \\
    16 & cautious\_jj & 0 & -6141 \\
    17 & insecurity & 0 & -5860 \\
    18 & cruelty & 0 & -5697 \\
    19 & playfulness & 0 & -5686 \\
    20 & curiosity & 0 & -5579 \\
    21 & philosophical\_jj & 0 & -5506 \\
    22 & individualistic\_jj & 0 & -5254 \\
    23 & curious\_jj & 0 & -5247 \\
    24 & skeptical\_jj & 0 & -5243 \\
    25 & sluggish\_jj & 0 & -5223 \\
    26 & earthy\_jj & 0 & -5214 \\
    27 & refined\_jj & 0 & -5188 \\
    28 & impersonal\_jj & 0 & -5064 \\
    29 & melancholic\_jj & 0 & -5036 \\
    30 & morose\_jj & 0 & -4971 \\
    401 & forceful\_jj & 0 & 4816 \\
    402 & brave\_jj & 0 & 4876 \\
    403 & cunning & 0 & 4898 \\
    404 & cunning\_jj & 0 & 4917 \\
    405 & nonconforming\_jj & 0 & 5284 \\
    406 & courage & 0 & 5309 \\
    407 & boastful\_jj & 0 & 5331 \\
    408 & ungracious\_jj & 0 & 5448 \\
    409 & reserve & 0 & 5504 \\
    410 & conceited\_jj & 0 & 5523 \\
    411 & prompt\_jj & 0 & 5616 \\
    412 & intelligence & 0 & 5900 \\
    413 & rash\_jj & 0 & 6216 \\
    414 & irritable\_jj & 0 & 6329 \\
    415 & independence & 0 & 6353 \\
    416 & irritability & 0 & 6548 \\
    417 & gregariousness & 0 & 6685 \\
    418 & decisive\_jj & 0 & 6903 \\
    419 & ruthless\_jj & 0 & 6967 \\
    420 & concise\_jj & 0 & 6980 \\
    421 & tenacious\_jj & 0 & 7047 \\
    422 & suspicious\_jj & 0 & 7106 \\
    423 & diplomatic\_jj & 0 & 7179 \\
    424 & courageous\_jj & 0 & 7316 \\
    425 & lethargy & 0 & 7502 \\
    426 & selfless\_jj & 0 & 7569 \\
    427 & withdrawn\_jj & 0 & 7844 \\
    428 & defensive\_jj & 0 & 8796 \\
    429 & belligerence & 0 & 8878 \\
    430 & explosive\_jj & 1 & 1737 \\
    \hline
    \caption{Scores and rankings for most extreme 30 words in component \#17} \\
\end{longtable}
\begin{longtable}[!htbp]{| rlr@{.}l |}
    \hline
    \textbf{Rank} & \textbf{Word} & \multicolumn{2}{c|}{\textbf{Score}} \\
    \hline
    \endhead
    1 & uncooperative\_jj & -1 & -3459 \\
    2 & explosive\_jj & -1 & -155 \\
    3 & surly\_jj & 0 & -9604 \\
    4 & insensitive\_jj & 0 & -8932 \\
    5 & unkind\_jj & 0 & -8862 \\
    6 & disrespectful\_jj & 0 & -8831 \\
    7 & irritability & 0 & -8741 \\
    8 & autonomous\_jj & 0 & -8160 \\
    9 & ungracious\_jj & 0 & -8097 \\
    10 & courage & 0 & -7972 \\
    11 & irritable\_jj & 0 & -7447 \\
    12 & economical\_jj & 0 & -7310 \\
    13 & self-esteem & 0 & -6844 \\
    14 & imperturbable\_jj & 0 & -6806 \\
    15 & independence & 0 & -6605 \\
    16 & unstable\_jj & 0 & -6450 \\
    17 & charitable\_jj & 0 & -6134 \\
    18 & emotional\_jj & 0 & -5948 \\
    19 & foolhardy\_jj & 0 & -5825 \\
    20 & conceited\_jj & 0 & -5810 \\
    21 & verbose\_jj & 0 & -5772 \\
    22 & rebellious\_jj & 0 & -5723 \\
    23 & courtesy & 0 & -5688 \\
    24 & rude\_jj & 0 & -5661 \\
    25 & naïve\_jj & 0 & -5563 \\
    26 & innovative\_jj & 0 & -5459 \\
    27 & daring & 0 & -5437 \\
    28 & brave\_jj & 0 & -5420 \\
    29 & egotistical\_jj & 0 & -5355 \\
    30 & impractical\_jj & 0 & -5333 \\
    401 & distrustful\_jj & 0 & 4925 \\
    402 & conscientious\_jj & 0 & 4946 \\
    403 & surliness & 0 & 4973 \\
    404 & distrust & 0 & 5025 \\
    405 & haphazard\_jj & 0 & 5083 \\
    406 & predictability & 0 & 5249 \\
    407 & thorough\_jj & 0 & 5286 \\
    408 & nosey\_jj & 0 & 5445 \\
    409 & miserly\_jj & 0 & 5499 \\
    410 & steady\_jj & 0 & 5626 \\
    411 & gullibility & 0 & 5643 \\
    412 & passivity & 0 & 5674 \\
    413 & sociable\_jj & 0 & 5709 \\
    414 & selfishness & 0 & 5725 \\
    415 & thrift & 0 & 5774 \\
    416 & courteous\_jj & 0 & 5786 \\
    417 & stinginess & 0 & 6039 \\
    418 & systematic\_jj & 0 & 6111 \\
    419 & recklessness & 0 & 6281 \\
    420 & slothful\_jj & 0 & 6655 \\
    421 & negligence & 0 & 6733 \\
    422 & docile\_jj & 0 & 7500 \\
    423 & considerate\_jj & 0 & 7532 \\
    424 & quarrelsome\_jj & 0 & 7610 \\
    425 & thrifty\_jj & 0 & 7753 \\
    426 & prompt\_jj & 0 & 7811 \\
    427 & sloth & 0 & 9372 \\
    428 & unscrupulous\_jj & 1 & 1131 \\
    429 & abusive\_jj & 1 & 3725 \\
    430 & concise\_jj & 1 & 7121 \\
    \hline
    \caption{Scores and rankings for most extreme 30 words in component \#18} \\
\end{longtable}
\begin{longtable}[!htbp]{| rlr@{.}l |}
    \hline
    \textbf{Rank} & \textbf{Word} & \multicolumn{2}{c|}{\textbf{Score}} \\
    \hline
    \endhead
    1 & prompt\_jj & -1 & -6721 \\
    2 & unscrupulous\_jj & -1 & -1976 \\
    3 & leniency & -1 & -1050 \\
    4 & lenient\_jj & -1 & -949 \\
    5 & impolite\_jj & 0 & -9986 \\
    6 & refined\_jj & 0 & -9903 \\
    7 & belligerence & 0 & -8830 \\
    8 & sophistication & 0 & -8018 \\
    9 & economical\_jj & 0 & -7687 \\
    10 & adventurous\_jj & 0 & -7431 \\
    11 & condescending\_jj & 0 & -6667 \\
    12 & intrusive\_jj & 0 & -6488 \\
    13 & extravagant\_jj & 0 & -5881 \\
    14 & scornful\_jj & 0 & -5822 \\
    15 & generous\_jj & 0 & -5583 \\
    16 & opportunistic\_jj & 0 & -5200 \\
    17 & frivolous\_jj & 0 & -5177 \\
    18 & generosity & 0 & -5174 \\
    19 & accommodating\_jj & 0 & -5102 \\
    20 & callousness & 0 & -4831 \\
    21 & earthiness & 0 & -4687 \\
    22 & skeptical\_jj & 0 & -4540 \\
    23 & exacting\_jj & 0 & -4528 \\
    24 & enthusiastic\_jj & 0 & -4511 \\
    25 & unsympathetic\_jj & 0 & -4476 \\
    26 & inventive\_jj & 0 & -4447 \\
    27 & candor & 0 & -4419 \\
    28 & meddlesome\_jj & 0 & -4400 \\
    29 & restrained\_jj & 0 & -4367 \\
    30 & assertive\_jj & 0 & -4362 \\
    401 & rebellious\_jj & 0 & 4723 \\
    402 & sloppy\_jj & 0 & 4730 \\
    403 & unemotional\_jj & 0 & 4741 \\
    404 & self-critical\_jj & 0 & 4744 \\
    405 & steady\_jj & 0 & 4819 \\
    406 & depth & 0 & 4843 \\
    407 & firm\_jj & 0 & 4985 \\
    408 & compassionate\_jj & 0 & 5156 \\
    409 & egocentric\_jj & 0 & 5243 \\
    410 & intelligent\_jj & 0 & 5293 \\
    411 & inconsistent\_jj & 0 & 5414 \\
    412 & stubborn\_jj & 0 & 5496 \\
    413 & understanding & 0 & 5710 \\
    414 & dependable\_jj & 0 & 5829 \\
    415 & defensive\_jj & 0 & 5872 \\
    416 & morality & 0 & 5993 \\
    417 & dominant\_jj & 0 & 6033 \\
    418 & moral\_jj & 0 & 6239 \\
    419 & tempestuous\_jj & 0 & 6375 \\
    420 & selfish\_jj & 0 & 6387 \\
    421 & warm\_jj & 0 & 6402 \\
    422 & reliable\_jj & 0 & 6641 \\
    423 & detached\_jj & 0 & 6677 \\
    424 & consistent\_jj & 0 & 7222 \\
    425 & shallow\_jj & 0 & 7664 \\
    426 & neat\_jj & 0 & 7829 \\
    427 & erratic\_jj & 0 & 8002 \\
    428 & concise\_jj & 1 & 2291 \\
    429 & uncooperative\_jj & 1 & 2981 \\
    430 & surly\_jj & 1 & 6087 \\
    \hline
    \caption{Scores and rankings for most extreme 30 words in component \#19} \\
\end{longtable}
\begin{longtable}[!htbp]{| rlr@{.}l |}
    \hline
    \textbf{Rank} & \textbf{Word} & \multicolumn{2}{c|}{\textbf{Score}} \\
    \hline
    \endhead
    1 & thrift & -1 & -6509 \\
    2 & lethargy & -1 & -1350 \\
    3 & disorganization & -1 & -393 \\
    4 & uncooperative\_jj & 0 & -9059 \\
    5 & irritability & 0 & -8569 \\
    6 & charitable\_jj & 0 & -7510 \\
    7 & surly\_jj & 0 & -7373 \\
    8 & inventive\_jj & 0 & -7064 \\
    9 & stingy\_jj & 0 & -7005 \\
    10 & abusive\_jj & 0 & -6641 \\
    11 & sophistication & 0 & -6359 \\
    12 & belligerence & 0 & -6324 \\
    13 & courtesy & 0 & -6235 \\
    14 & tenacious\_jj & 0 & -6163 \\
    15 & aimless\_jj & 0 & -5973 \\
    16 & haphazard\_jj & 0 & -5952 \\
    17 & bold\_jj & 0 & -5587 \\
    18 & forgetfulness & 0 & -5435 \\
    19 & secretive\_jj & 0 & -5324 \\
    20 & neat\_jj & 0 & -5016 \\
    21 & unscrupulous\_jj & 0 & -4851 \\
    22 & unassuming\_jj & 0 & -4749 \\
    23 & persistence & 0 & -4521 \\
    24 & folksy\_jj & 0 & -4507 \\
    25 & inefficient\_jj & 0 & -4502 \\
    26 & creative\_jj & 0 & -4452 \\
    27 & underhanded\_jj & 0 & -4405 \\
    28 & meddlesome\_jj & 0 & -4400 \\
    29 & meticulous\_jj & 0 & -4376 \\
    30 & jovial\_jj & 0 & -4289 \\
    401 & prejudiced\_jj & 0 & 4669 \\
    402 & silence & 0 & 4773 \\
    403 & expressive\_jj & 0 & 4815 \\
    404 & self-pitying\_jj & 0 & 4819 \\
    405 & unstable\_jj & 0 & 5166 \\
    406 & communicative\_jj & 0 & 5602 \\
    407 & natural\_jj & 0 & 5923 \\
    408 & expressiveness & 0 & 5996 \\
    409 & self-critical\_jj & 0 & 6013 \\
    410 & pessimistic\_jj & 0 & 6096 \\
    411 & selfish\_jj & 0 & 6252 \\
    412 & reckless\_jj & 0 & 6305 \\
    413 & friendly\_jj & 0 & 6427 \\
    414 & volatility & 0 & 6469 \\
    415 & economical\_jj & 0 & 6604 \\
    416 & pessimism & 0 & 6795 \\
    417 & optimistic\_jj & 0 & 6826 \\
    418 & cruelty & 0 & 7179 \\
    419 & negligence & 0 & 7577 \\
    420 & careless\_jj & 0 & 7591 \\
    421 & volatile\_jj & 0 & 7692 \\
    422 & nonconforming\_jj & 0 & 8840 \\
    423 & dependability & 0 & 8994 \\
    424 & mannerly\_jj & 0 & 9027 \\
    425 & earthiness & 0 & 9095 \\
    426 & suspicious\_jj & 0 & 9150 \\
    427 & caustic\_jj & 0 & 9287 \\
    428 & negligent\_jj & 1 & 587 \\
    429 & explosive\_jj & 1 & 3693 \\
    430 & refined\_jj & 1 & 3901 \\
    \hline
    \caption{Scores and rankings for most extreme 30 words in component \#20} \\
\end{longtable}
\begin{longtable}[!htbp]{| rlr@{.}l |}
    \hline
    \textbf{Rank} & \textbf{Word} & \multicolumn{2}{c|}{\textbf{Score}} \\
    \hline
    \endhead
    1 & prompt\_jj & -1 & -1496 \\
    2 & unscrupulous\_jj & -1 & -14 \\
    3 & insight & 0 & -9581 \\
    4 & warm\_jj & 0 & -9163 \\
    5 & silence & 0 & -8777 \\
    6 & brave\_jj & 0 & -8613 \\
    7 & surly\_jj & 0 & -8372 \\
    8 & perceptive\_jj & 0 & -8068 \\
    9 & belligerence & 0 & -7787 \\
    10 & insightful\_jj & 0 & -7732 \\
    11 & uncooperative\_jj & 0 & -7648 \\
    12 & cold\_jj & 0 & -6883 \\
    13 & vain\_jj & 0 & -6463 \\
    14 & instability & 0 & -6347 \\
    15 & quiet & 0 & -6119 \\
    16 & orderly\_jj & 0 & -6114 \\
    17 & obstinate\_jj & 0 & -5906 \\
    18 & somber\_jj & 0 & -5869 \\
    19 & courageous\_jj & 0 & -5823 \\
    20 & adaptable\_jj & 0 & -5701 \\
    21 & aimlessness & 0 & -5690 \\
    22 & dignified\_jj & 0 & -5649 \\
    23 & concise\_jj & 0 & -5557 \\
    24 & cruel\_jj & 0 & -5159 \\
    25 & peaceful\_jj & 0 & -5078 \\
    26 & lethargic\_jj & 0 & -5019 \\
    27 & selfless\_jj & 0 & -4893 \\
    28 & self-pitying\_jj & 0 & -4863 \\
    29 & greedy\_jj & 0 & -4790 \\
    30 & silent\_jj & 0 & -4630 \\
    401 & flexibility & 0 & 4248 \\
    402 & cosmopolitan\_jj & 0 & 4357 \\
    403 & cautious\_jj & 0 & 4446 \\
    404 & inhibition & 0 & 4529 \\
    405 & carefree\_jj & 0 & 4550 \\
    406 & rebellious\_jj & 0 & 4563 \\
    407 & easygoing\_jj & 0 & 4604 \\
    408 & antagonistic\_jj & 0 & 4691 \\
    409 & sociable\_jj & 0 & 4736 \\
    410 & punctuality & 0 & 4740 \\
    411 & condescending\_jj & 0 & 4745 \\
    412 & talkative\_jj & 0 & 4959 \\
    413 & combative\_jj & 0 & 5115 \\
    414 & homespun\_jj & 0 & 5187 \\
    415 & unconventional\_jj & 0 & 5352 \\
    416 & lenient\_jj & 0 & 5402 \\
    417 & argumentative\_jj & 0 & 5567 \\
    418 & conventional\_jj & 0 & 6091 \\
    419 & nonconformity & 0 & 6103 \\
    420 & thrift & 0 & 6131 \\
    421 & happy-go-lucky\_jj & 0 & 6137 \\
    422 & casual\_jj & 0 & 6150 \\
    423 & assertive\_jj & 0 & 6227 \\
    424 & nonconforming\_jj & 0 & 6691 \\
    425 & down-to-earth\_jj & 0 & 7295 \\
    426 & traditional\_jj & 0 & 7939 \\
    427 & deliberate\_jj & 0 & 8988 \\
    428 & abusive\_jj & 1 & 1651 \\
    429 & friendly\_jj & 1 & 2786 \\
    430 & folksy\_jj & 2 & 2851 \\
    \hline
    \caption{Scores and rankings for most extreme 30 words in component \#21} \\
\end{longtable}
\begin{longtable}[!htbp]{| rlr@{.}l |}
    \hline
    \textbf{Rank} & \textbf{Word} & \multicolumn{2}{c|}{\textbf{Score}} \\
    \hline
    \endhead
    1 & insight & -1 & -3948 \\
    2 & belligerence & -1 & -152 \\
    3 & erratic\_jj & -1 & -92 \\
    4 & cordial\_jj & 0 & -8909 \\
    5 & lenient\_jj & 0 & -8708 \\
    6 & negligent\_jj & 0 & -8300 \\
    7 & tempestuous\_jj & 0 & -8138 \\
    8 & touchy\_jj & 0 & -7905 \\
    9 & inconsiderate\_jj & 0 & -7554 \\
    10 & warm\_jj & 0 & -6924 \\
    11 & careless\_jj & 0 & -6002 \\
    12 & antagonistic\_jj & 0 & -5974 \\
    13 & cold\_jj & 0 & -5914 \\
    14 & artistic\_jj & 0 & -5911 \\
    15 & cruelty & 0 & -5870 \\
    16 & indecisive\_jj & 0 & -5675 \\
    17 & diplomatic\_jj & 0 & -5619 \\
    18 & dependability & 0 & -5213 \\
    19 & concise\_jj & 0 & -5185 \\
    20 & leniency & 0 & -5153 \\
    21 & animation & 0 & -5046 \\
    22 & punctuality & 0 & -4959 \\
    23 & unpredictable\_jj & 0 & -4802 \\
    24 & expressive\_jj & 0 & -4802 \\
    25 & impetuous\_jj & 0 & -4694 \\
    26 & inventive\_jj & 0 & -4614 \\
    27 & charitable\_jj & 0 & -4541 \\
    28 & neat\_jj & 0 & -4529 \\
    29 & cruel\_jj & 0 & -4527 \\
    30 & creative\_jj & 0 & -4429 \\
    401 & surliness & 0 & 4330 \\
    402 & irritable\_jj & 0 & 4404 \\
    403 & unscrupulous\_jj & 0 & 4404 \\
    404 & conventional\_jj & 0 & 4465 \\
    405 & reasonable\_jj & 0 & 4484 \\
    406 & sophisticated\_jj & 0 & 4573 \\
    407 & deliberate\_jj & 0 & 4864 \\
    408 & sophistication & 0 & 4889 \\
    409 & distrustful\_jj & 0 & 4932 \\
    410 & unintelligent\_jj & 0 & 4981 \\
    411 & vigorous\_jj & 0 & 4995 \\
    412 & confident\_jj & 0 & 5192 \\
    413 & crabby\_jj & 0 & 5211 \\
    414 & skeptical\_jj & 0 & 5303 \\
    415 & fearful\_jj & 0 & 5450 \\
    416 & earthiness & 0 & 5518 \\
    417 & steady\_jj & 0 & 5518 \\
    418 & inhibition & 0 & 5549 \\
    419 & prejudice & 0 & 5635 \\
    420 & self-pitying\_jj & 0 & 5679 \\
    421 & nervous\_jj & 0 & 5707 \\
    422 & gullible\_jj & 0 & 6067 \\
    423 & folksy\_jj & 0 & 6337 \\
    424 & cultured\_jj & 0 & 6345 \\
    425 & suspicious\_jj & 0 & 8288 \\
    426 & thorough\_jj & 0 & 9108 \\
    427 & uncooperative\_jj & 1 & 1867 \\
    428 & surly\_jj & 1 & 2497 \\
    429 & refined\_jj & 1 & 2930 \\
    430 & prompt\_jj & 1 & 5311 \\
    \hline
    \caption{Scores and rankings for most extreme 30 words in component \#22} \\
\end{longtable}
\begin{longtable}[!htbp]{| rlr@{.}l |}
    \hline
    \textbf{Rank} & \textbf{Word} & \multicolumn{2}{c|}{\textbf{Score}} \\
    \hline
    \endhead
    1 & explosive\_jj & -1 & -4893 \\
    2 & volatility & 0 & -9692 \\
    3 & orderly\_jj & 0 & -9596 \\
    4 & frivolous\_jj & 0 & -8030 \\
    5 & instability & 0 & -7081 \\
    6 & unsophisticated\_jj & 0 & -7046 \\
    7 & concise\_jj & 0 & -6692 \\
    8 & unrestrained\_jj & 0 & -6640 \\
    9 & spontaneous\_jj & 0 & -5907 \\
    10 & abusive\_jj & 0 & -5869 \\
    11 & foolhardy\_jj & 0 & -5840 \\
    12 & haphazard\_jj & 0 & -5670 \\
    13 & peaceful\_jj & 0 & -5647 \\
    14 & aimless\_jj & 0 & -5591 \\
    15 & suspicious\_jj & 0 & -5435 \\
    16 & rash\_jj & 0 & -5283 \\
    17 & reckless\_jj & 0 & -5270 \\
    18 & humorous\_jj & 0 & -5269 \\
    19 & frivolity & 0 & -5222 \\
    20 & flippant\_jj & 0 & -5115 \\
    21 & optimism & 0 & -5068 \\
    22 & quiet & 0 & -4823 \\
    23 & uninhibited\_jj & 0 & -4685 \\
    24 & absent-minded\_jj & 0 & -4631 \\
    25 & pessimism & 0 & -4517 \\
    26 & extravagant\_jj & 0 & -4508 \\
    27 & volatile\_jj & 0 & -4414 \\
    28 & reserve & 0 & -4388 \\
    29 & homespun\_jj & 0 & -4288 \\
    30 & happy-go-lucky\_jj & 0 & -4190 \\
    401 & secretive\_jj & 0 & 4214 \\
    402 & compassionate\_jj & 0 & 4232 \\
    403 & assertive\_jj & 0 & 4310 \\
    404 & aloofness & 0 & 4355 \\
    405 & deep\_jj & 0 & 4523 \\
    406 & negligence & 0 & 4716 \\
    407 & leniency & 0 & 4736 \\
    408 & expressive\_jj & 0 & 4790 \\
    409 & verbal\_jj & 0 & 4812 \\
    410 & shallow\_jj & 0 & 4827 \\
    411 & sloth & 0 & 4844 \\
    412 & cunning & 0 & 5019 \\
    413 & earthy\_jj & 0 & 5040 \\
    414 & earthiness & 0 & 5614 \\
    415 & lazy\_jj & 0 & 5666 \\
    416 & ruthless\_jj & 0 & 5704 \\
    417 & depth & 0 & 6141 \\
    418 & submissive\_jj & 0 & 6579 \\
    419 & analytical\_jj & 0 & 6593 \\
    420 & irritable\_jj & 0 & 6883 \\
    421 & perceptive\_jj & 0 & 7586 \\
    422 & charitable\_jj & 0 & 7982 \\
    423 & imperturbable\_jj & 0 & 7993 \\
    424 & harsh\_jj & 0 & 8008 \\
    425 & cruelty & 0 & 8806 \\
    426 & insight & 1 & 146 \\
    427 & cold\_jj & 1 & 325 \\
    428 & lenient\_jj & 1 & 1798 \\
    429 & warm\_jj & 1 & 2303 \\
    430 & refined\_jj & 1 & 6842 \\
    \hline
    \caption{Scores and rankings for most extreme 30 words in component \#23} \\
\end{longtable}
\begin{longtable}[!htbp]{| rlr@{.}l |}
    \hline
    \textbf{Rank} & \textbf{Word} & \multicolumn{2}{c|}{\textbf{Score}} \\
    \hline
    \endhead
    1 & belligerence & -1 & -3810 \\
    2 & naturalness & 0 & -7869 \\
    3 & silence & 0 & -7638 \\
    4 & dependability & 0 & -7376 \\
    5 & adaptable\_jj & 0 & -7173 \\
    6 & negligent\_jj & 0 & -7158 \\
    7 & docile\_jj & 0 & -6470 \\
    8 & efficient\_jj & 0 & -6151 \\
    9 & precise\_jj & 0 & -6111 \\
    10 & unemotional\_jj & 0 & -6016 \\
    11 & rude\_jj & 0 & -5838 \\
    12 & rambunctious\_jj & 0 & -5834 \\
    13 & intelligence & 0 & -5603 \\
    14 & decisiveness & 0 & -5561 \\
    15 & proud\_jj & 0 & -5267 \\
    16 & imperturbable\_jj & 0 & -5223 \\
    17 & thorough\_jj & 0 & -5095 \\
    18 & meditative\_jj & 0 & -4939 \\
    19 & obliging\_jj & 0 & -4905 \\
    20 & deliberate\_jj & 0 & -4837 \\
    21 & sophistication & 0 & -4746 \\
    22 & intrusiveness & 0 & -4732 \\
    23 & understanding\_jj & 0 & -4722 \\
    24 & emotionality & 0 & -4543 \\
    25 & respectful\_jj & 0 & -4352 \\
    26 & peaceful\_jj & 0 & -4322 \\
    27 & bullheaded\_jj & 0 & -4306 \\
    28 & cunning\_jj & 0 & -4275 \\
    29 & passive\_jj & 0 & -4269 \\
    30 & spontaneity & 0 & -4156 \\
    401 & verbose\_jj & 0 & 4800 \\
    402 & earthy\_jj & 0 & 4812 \\
    403 & spirited\_jj & 0 & 4862 \\
    404 & miserly\_jj & 0 & 4875 \\
    405 & pessimism & 0 & 4958 \\
    406 & selfishness & 0 & 4996 \\
    407 & self-critical\_jj & 0 & 5066 \\
    408 & ethical\_jj & 0 & 5382 \\
    409 & bitter\_jj & 0 & 5403 \\
    410 & individualistic\_jj & 0 & 5490 \\
    411 & earthiness & 0 & 5513 \\
    412 & prejudice & 0 & 5619 \\
    413 & touchy\_jj & 0 & 5623 \\
    414 & lenient\_jj & 0 & 5654 \\
    415 & principled\_jj & 0 & 5717 \\
    416 & caustic\_jj & 0 & 5899 \\
    417 & suspicious\_jj & 0 & 5969 \\
    418 & prompt\_jj & 0 & 6044 \\
    419 & frivolous\_jj & 0 & 6211 \\
    420 & stingy\_jj & 0 & 6644 \\
    421 & crabby\_jj & 0 & 6943 \\
    422 & refined\_jj & 0 & 7269 \\
    423 & philosophical\_jj & 0 & 7437 \\
    424 & moral\_jj & 0 & 7935 \\
    425 & lethargy & 0 & 8107 \\
    426 & charitable\_jj & 0 & 9510 \\
    427 & compassionate\_jj & 0 & 9641 \\
    428 & explosive\_jj & 1 & 1446 \\
    429 & concise\_jj & 1 & 1762 \\
    430 & unscrupulous\_jj & 1 & 2353 \\
    \hline
    \caption{Scores and rankings for most extreme 30 words in component \#24} \\
\end{longtable}
\begin{longtable}[!htbp]{| rlr@{.}l |}
    \hline
    \textbf{Rank} & \textbf{Word} & \multicolumn{2}{c|}{\textbf{Score}} \\
    \hline
    \endhead
    1 & friendly\_jj & -1 & -1660 \\
    2 & cordial\_jj & 0 & -8857 \\
    3 & verbal\_jj & 0 & -8181 \\
    4 & selfish\_jj & 0 & -8092 \\
    5 & sloth & 0 & -7916 \\
    6 & understanding\_jj & 0 & -7794 \\
    7 & egocentric\_jj & 0 & -7740 \\
    8 & crafty\_jj & 0 & -6654 \\
    9 & erratic\_jj & 0 & -6450 \\
    10 & pessimism & 0 & -6316 \\
    11 & antagonistic\_jj & 0 & -6269 \\
    12 & adaptable\_jj & 0 & -5934 \\
    13 & envy & 0 & -5741 \\
    14 & animation & 0 & -5559 \\
    15 & surly\_jj & 0 & -5331 \\
    16 & tempestuous\_jj & 0 & -5254 \\
    17 & spirited\_jj & 0 & -5239 \\
    18 & unscrupulous\_jj & 0 & -5190 \\
    19 & miserly\_jj & 0 & -5189 \\
    20 & inventive\_jj & 0 & -5002 \\
    21 & gullibility & 0 & -4935 \\
    22 & devious\_jj & 0 & -4853 \\
    23 & brave\_jj & 0 & -4822 \\
    24 & imperturbable\_jj & 0 & -4796 \\
    25 & touchy\_jj & 0 & -4752 \\
    26 & opportunistic\_jj & 0 & -4691 \\
    27 & playful\_jj & 0 & -4662 \\
    28 & curiosity & 0 & -4554 \\
    29 & vigorous\_jj & 0 & -4424 \\
    30 & innovative\_jj & 0 & -4156 \\
    401 & modesty & 0 & 4671 \\
    402 & dignity & 0 & 4887 \\
    403 & unstable\_jj & 0 & 4925 \\
    404 & truthful\_jj & 0 & 5017 \\
    405 & independent\_jj & 0 & 5067 \\
    406 & impractical\_jj & 0 & 5133 \\
    407 & concise\_jj & 0 & 5234 \\
    408 & dependability & 0 & 5239 \\
    409 & insecure\_jj & 0 & 5384 \\
    410 & inefficient\_jj & 0 & 5610 \\
    411 & warm\_jj & 0 & 5646 \\
    412 & deliberate\_jj & 0 & 5751 \\
    413 & haphazard\_jj & 0 & 5756 \\
    414 & disorganization & 0 & 5938 \\
    415 & expressiveness & 0 & 5979 \\
    416 & wishy-washy\_jj & 0 & 6141 \\
    417 & melancholic\_jj & 0 & 6377 \\
    418 & thrift & 0 & 6605 \\
    419 & earthy\_jj & 0 & 6634 \\
    420 & forgetful\_jj & 0 & 6796 \\
    421 & unemotional\_jj & 0 & 6996 \\
    422 & orderly\_jj & 0 & 7083 \\
    423 & unreliable\_jj & 0 & 7125 \\
    424 & distrustful\_jj & 0 & 7825 \\
    425 & compassionate\_jj & 0 & 8521 \\
    426 & indecisive\_jj & 0 & 8528 \\
    427 & folksy\_jj & 0 & 9886 \\
    428 & earthiness & 1 & 208 \\
    429 & independence & 1 & 2044 \\
    430 & negligent\_jj & 1 & 5378 \\
    \hline
    \caption{Scores and rankings for most extreme 30 words in component \#25} \\
\end{longtable}
\begin{longtable}[!htbp]{| rlr@{.}l |}
    \hline
    \textbf{Rank} & \textbf{Word} & \multicolumn{2}{c|}{\textbf{Score}} \\
    \hline
    \endhead
    1 & lenient\_jj & -1 & -999 \\
    2 & explosive\_jj & -1 & -221 \\
    3 & warm\_jj & 0 & -8281 \\
    4 & intrusive\_jj & 0 & -7424 \\
    5 & inefficient\_jj & 0 & -6910 \\
    6 & efficient\_jj & 0 & -6808 \\
    7 & expressiveness & 0 & -6583 \\
    8 & warmth & 0 & -6552 \\
    9 & suspicious\_jj & 0 & -5910 \\
    10 & distrust & 0 & -5906 \\
    11 & concise\_jj & 0 & -5667 \\
    12 & callousness & 0 & -5515 \\
    13 & cold\_jj & 0 & -5370 \\
    14 & organized\_jj & 0 & -5090 \\
    15 & impersonal\_jj & 0 & -4989 \\
    16 & generosity & 0 & -4873 \\
    17 & harsh\_jj & 0 & -4820 \\
    18 & curt\_jj & 0 & -4773 \\
    19 & obliging\_jj & 0 & -4701 \\
    20 & playfulness & 0 & -4651 \\
    21 & unreliable\_jj & 0 & -4547 \\
    22 & decisiveness & 0 & -4542 \\
    23 & spirit & 0 & -4496 \\
    24 & courteous\_jj & 0 & -4483 \\
    25 & orderly\_jj & 0 & -4350 \\
    26 & passionless\_jj & 0 & -4317 \\
    27 & generous\_jj & 0 & -4266 \\
    28 & impractical\_jj & 0 & -4095 \\
    29 & irritable\_jj & 0 & -4060 \\
    30 & gullibility & 0 & -4056 \\
    401 & economical\_jj & 0 & 3988 \\
    402 & impolite\_jj & 0 & 4023 \\
    403 & gullible\_jj & 0 & 4049 \\
    404 & spirited\_jj & 0 & 4098 \\
    405 & snobbish\_jj & 0 & 4203 \\
    406 & organization & 0 & 4467 \\
    407 & verbose\_jj & 0 & 4483 \\
    408 & sociable\_jj & 0 & 4706 \\
    409 & witty\_jj & 0 & 4755 \\
    410 & vivacious\_jj & 0 & 4808 \\
    411 & steady\_jj & 0 & 4826 \\
    412 & wishy-washy\_jj & 0 & 4981 \\
    413 & deliberate\_jj & 0 & 5166 \\
    414 & instability & 0 & 5231 \\
    415 & dependability & 0 & 5456 \\
    416 & self-esteem & 0 & 5533 \\
    417 & reserve & 0 & 5641 \\
    418 & uncharitable\_jj & 0 & 6083 \\
    419 & insight & 0 & 6203 \\
    420 & cultured\_jj & 0 & 6604 \\
    421 & diplomatic\_jj & 0 & 6827 \\
    422 & pessimism & 0 & 7021 \\
    423 & perceptive\_jj & 0 & 7927 \\
    424 & sluggish\_jj & 0 & 8412 \\
    425 & volatility & 0 & 8530 \\
    426 & thrift & 0 & 8838 \\
    427 & negligent\_jj & 1 & 443 \\
    428 & charitable\_jj & 1 & 1966 \\
    429 & refined\_jj & 1 & 6911 \\
    430 & belligerence & 2 & 334 \\
    \hline
    \caption{Scores and rankings for most extreme 30 words in component \#26} \\
\end{longtable}
\begin{longtable}[!htbp]{| rlr@{.}l |}
    \hline
    \textbf{Rank} & \textbf{Word} & \multicolumn{2}{c|}{\textbf{Score}} \\
    \hline
    \endhead
    1 & prompt\_jj & -1 & -6998 \\
    2 & charitable\_jj & -1 & -3662 \\
    3 & selfish\_jj & 0 & -9625 \\
    4 & manipulative\_jj & 0 & -9182 \\
    5 & naturalness & 0 & -8989 \\
    6 & erratic\_jj & 0 & -8130 \\
    7 & volatility & 0 & -7630 \\
    8 & frivolous\_jj & 0 & -6660 \\
    9 & melancholic\_jj & 0 & -5958 \\
    10 & volatile\_jj & 0 & -5753 \\
    11 & self-esteem & 0 & -5484 \\
    12 & high-strung\_jj & 0 & -5063 \\
    13 & tempestuous\_jj & 0 & -5043 \\
    14 & aloofness & 0 & -4871 \\
    15 & unemotional\_jj & 0 & -4821 \\
    16 & argumentative\_jj & 0 & -4756 \\
    17 & withdrawn\_jj & 0 & -4732 \\
    18 & earthiness & 0 & -4700 \\
    19 & underhanded\_jj & 0 & -4463 \\
    20 & expressiveness & 0 & -4411 \\
    21 & devious\_jj & 0 & -4344 \\
    22 & shyness & 0 & -4272 \\
    23 & greedy\_jj & 0 & -4270 \\
    24 & moody\_jj & 0 & -4129 \\
    25 & insight & 0 & -4075 \\
    26 & deceitful\_jj & 0 & -4033 \\
    27 & insecure\_jj & 0 & -3990 \\
    28 & compassionate\_jj & 0 & -3873 \\
    29 & boastful\_jj & 0 & -3855 \\
    30 & emotionality & 0 & -3753 \\
    401 & punctual\_jj & 0 & 4330 \\
    402 & accommodating\_jj & 0 & 4378 \\
    403 & rude\_jj & 0 & 4412 \\
    404 & cold\_jj & 0 & 4435 \\
    405 & envy & 0 & 4482 \\
    406 & enterprising\_jj & 0 & 4484 \\
    407 & lethargy & 0 & 4494 \\
    408 & adaptable\_jj & 0 & 4504 \\
    409 & scornful\_jj & 0 & 4532 \\
    410 & insensitive\_jj & 0 & 4978 \\
    411 & surly\_jj & 0 & 5226 \\
    412 & uncooperative\_jj & 0 & 5396 \\
    413 & cultured\_jj & 0 & 5504 \\
    414 & individualistic\_jj & 0 & 5512 \\
    415 & rudeness & 0 & 5680 \\
    416 & punctuality & 0 & 5817 \\
    417 & intelligence & 0 & 6012 \\
    418 & negligent\_jj & 0 & 6018 \\
    419 & brave\_jj & 0 & 6315 \\
    420 & mannerly\_jj & 0 & 6563 \\
    421 & folksy\_jj & 0 & 7553 \\
    422 & cosmopolitan\_jj & 0 & 7726 \\
    423 & cruelty & 0 & 7932 \\
    424 & refined\_jj & 0 & 8742 \\
    425 & suspicious\_jj & 0 & 9269 \\
    426 & sophistication & 0 & 9431 \\
    427 & belligerence & 0 & 9919 \\
    428 & thrift & 1 & 564 \\
    429 & concise\_jj & 1 & 1300 \\
    430 & explosive\_jj & 1 & 2582 \\
    \hline
    \caption{Scores and rankings for most extreme 30 words in component \#27} \\
\end{longtable}
\begin{longtable}[!htbp]{| rlr@{.}l |}
    \hline
    \textbf{Rank} & \textbf{Word} & \multicolumn{2}{c|}{\textbf{Score}} \\
    \hline
    \endhead
    1 & insight & 0 & -9999 \\
    2 & neat\_jj & 0 & -8505 \\
    3 & brave\_jj & 0 & -7491 \\
    4 & sociable\_jj & 0 & -7242 \\
    5 & foolhardy\_jj & 0 & -6316 \\
    6 & nonconforming\_jj & 0 & -6157 \\
    7 & compassionate\_jj & 0 & -6059 \\
    8 & bitter\_jj & 0 & -6023 \\
    9 & rash\_jj & 0 & -5768 \\
    10 & caution & 0 & -5745 \\
    11 & impudent\_jj & 0 & -5628 \\
    12 & tactful\_jj & 0 & -5306 \\
    13 & independence & 0 & -5267 \\
    14 & harsh\_jj & 0 & -5135 \\
    15 & wishy-washy\_jj & 0 & -5129 \\
    16 & bossy\_jj & 0 & -4947 \\
    17 & daring & 0 & -4817 \\
    18 & vivacious\_jj & 0 & -4787 \\
    19 & tenacious\_jj & 0 & -4771 \\
    20 & leniency & 0 & -4753 \\
    21 & abusive\_jj & 0 & -4746 \\
    22 & decisive\_jj & 0 & -4505 \\
    23 & perceptive\_jj & 0 & -4249 \\
    24 & crabby\_jj & 0 & -4241 \\
    25 & meddlesome\_jj & 0 & -4206 \\
    26 & shyness & 0 & -4060 \\
    27 & lenient\_jj & 0 & -4053 \\
    28 & unimaginative\_jj & 0 & -4050 \\
    29 & conventional\_jj & 0 & -4044 \\
    30 & intrusiveness & 0 & -3948 \\
    401 & stupidity & 0 & 4201 \\
    402 & sincere\_jj & 0 & 4322 \\
    403 & polite\_jj & 0 & 4378 \\
    404 & warm\_jj & 0 & 4407 \\
    405 & cosmopolitan\_jj & 0 & 4412 \\
    406 & recklessness & 0 & 4509 \\
    407 & refined\_jj & 0 & 4592 \\
    408 & belligerence & 0 & 4598 \\
    409 & assured\_jj & 0 & 4683 \\
    410 & generous\_jj & 0 & 4701 \\
    411 & worldly\_jj & 0 & 4744 \\
    412 & condescending\_jj & 0 & 4786 \\
    413 & autonomous\_jj & 0 & 4798 \\
    414 & selfless\_jj & 0 & 4844 \\
    415 & greedy\_jj & 0 & 5149 \\
    416 & egocentric\_jj & 0 & 5323 \\
    417 & benevolent\_jj & 0 & 5366 \\
    418 & wordy\_jj & 0 & 5694 \\
    419 & extravagant\_jj & 0 & 5707 \\
    420 & naturalness & 0 & 5920 \\
    421 & uncharitable\_jj & 0 & 6155 \\
    422 & generosity & 0 & 6196 \\
    423 & selfish\_jj & 0 & 7692 \\
    424 & egotistical\_jj & 0 & 7842 \\
    425 & proud\_jj & 0 & 8064 \\
    426 & suspicious\_jj & 0 & 8916 \\
    427 & erratic\_jj & 0 & 9522 \\
    428 & expressive\_jj & 0 & 9710 \\
    429 & concise\_jj & 1 & 5287 \\
    430 & charitable\_jj & 2 & 40 \\
    \hline
    \caption{Scores and rankings for most extreme 30 words in component \#28} \\
\end{longtable}
\begin{longtable}[!htbp]{| rlr@{.}l |}
    \hline
    \textbf{Rank} & \textbf{Word} & \multicolumn{2}{c|}{\textbf{Score}} \\
    \hline
    \endhead
    1 & negligent\_jj & 0 & -8965 \\
    2 & independence & 0 & -7923 \\
    3 & reliable\_jj & 0 & -7313 \\
    4 & concise\_jj & 0 & -6829 \\
    5 & unscrupulous\_jj & 0 & -6810 \\
    6 & perceptive\_jj & 0 & -6776 \\
    7 & touchy\_jj & 0 & -6648 \\
    8 & rebellious\_jj & 0 & -6265 \\
    9 & earthiness & 0 & -6119 \\
    10 & expressive\_jj & 0 & -6093 \\
    11 & punctual\_jj & 0 & -5846 \\
    12 & negligence & 0 & -5828 \\
    13 & flippant\_jj & 0 & -5522 \\
    14 & bitter\_jj & 0 & -5410 \\
    15 & patient\_jj & 0 & -5162 \\
    16 & demanding\_jj & 0 & -5155 \\
    17 & instability & 0 & -5154 \\
    18 & silence & 0 & -5085 \\
    19 & egocentric\_jj & 0 & -5036 \\
    20 & wishy-washy\_jj & 0 & -4999 \\
    21 & fastidious\_jj & 0 & -4688 \\
    22 & mannerly\_jj & 0 & -4671 \\
    23 & dependable\_jj & 0 & -4611 \\
    24 & inconsiderate\_jj & 0 & -4348 \\
    25 & conceited\_jj & 0 & -4341 \\
    26 & bossiness & 0 & -4289 \\
    27 & passivity & 0 & -4269 \\
    28 & decisive\_jj & 0 & -4215 \\
    29 & tenacious\_jj & 0 & -4121 \\
    30 & nonconformity & 0 & -4086 \\
    401 & sloth & 0 & 4518 \\
    402 & unsophisticated\_jj & 0 & 4744 \\
    403 & compassionate\_jj & 0 & 4774 \\
    404 & haphazard\_jj & 0 & 4833 \\
    405 & withdrawn\_jj & 0 & 4833 \\
    406 & zestful\_jj & 0 & 4959 \\
    407 & quarrelsome\_jj & 0 & 5040 \\
    408 & inefficient\_jj & 0 & 5202 \\
    409 & recklessness & 0 & 5230 \\
    410 & selfish\_jj & 0 & 5265 \\
    411 & unemotional\_jj & 0 & 5316 \\
    412 & unintelligent\_jj & 0 & 5358 \\
    413 & intelligence & 0 & 5468 \\
    414 & aimlessness & 0 & 5526 \\
    415 & thorough\_jj & 0 & 5625 \\
    416 & explosive\_jj & 0 & 5838 \\
    417 & morose\_jj & 0 & 5874 \\
    418 & thoughtless\_jj & 0 & 5926 \\
    419 & impersonal\_jj & 0 & 6126 \\
    420 & orderly\_jj & 0 & 6532 \\
    421 & happy-go-lucky\_jj & 0 & 6784 \\
    422 & nonconforming\_jj & 0 & 7820 \\
    423 & aimless\_jj & 0 & 7992 \\
    424 & shallow\_jj & 0 & 8091 \\
    425 & refined\_jj & 0 & 8312 \\
    426 & reserve & 0 & 8786 \\
    427 & cordial\_jj & 0 & 9849 \\
    428 & suspicious\_jj & 1 & 1279 \\
    429 & insight & 1 & 1707 \\
    430 & thrift & 1 & 8158 \\
    \hline
    \caption{Scores and rankings for most extreme 30 words in component \#29} \\
\end{longtable}
\begin{longtable}[!htbp]{| rlr@{.}l |}
    \hline
    \textbf{Rank} & \textbf{Word} & \multicolumn{2}{c|}{\textbf{Score}} \\
    \hline
    \endhead
    1 & belligerence & -1 & -2447 \\
    2 & assured\_jj & -1 & -1762 \\
    3 & negligence & 0 & -7320 \\
    4 & assertion & 0 & -6639 \\
    5 & optimistic\_jj & 0 & -6594 \\
    6 & deliberate\_jj & 0 & -6167 \\
    7 & impudent\_jj & 0 & -6057 \\
    8 & inventive\_jj & 0 & -5806 \\
    9 & irritability & 0 & -5675 \\
    10 & conceited\_jj & 0 & -5663 \\
    11 & neat\_jj & 0 & -5641 \\
    12 & friendly\_jj & 0 & -5401 \\
    13 & pessimism & 0 & -5236 \\
    14 & reliable\_jj & 0 & -5183 \\
    15 & earthiness & 0 & -5146 \\
    16 & courtesy & 0 & -5090 \\
    17 & forgetfulness & 0 & -5035 \\
    18 & abusive\_jj & 0 & -4923 \\
    19 & sincere\_jj & 0 & -4832 \\
    20 & inhibition & 0 & -4772 \\
    21 & dependable\_jj & 0 & -4768 \\
    22 & curt\_jj & 0 & -4756 \\
    23 & innovative\_jj & 0 & -4679 \\
    24 & warm\_jj & 0 & -4677 \\
    25 & animation & 0 & -4631 \\
    26 & cultured\_jj & 0 & -4604 \\
    27 & boastful\_jj & 0 & -4407 \\
    28 & optimism & 0 & -4378 \\
    29 & unreflective\_jj & 0 & -4355 \\
    30 & rash\_jj & 0 & -4312 \\
    401 & prompt\_jj & 0 & 4435 \\
    402 & high-strung\_jj & 0 & 4504 \\
    403 & dependability & 0 & 4543 \\
    404 & quiet\_jj & 0 & 4549 \\
    405 & unstable\_jj & 0 & 4573 \\
    406 & impolite\_jj & 0 & 4587 \\
    407 & mannerly\_jj & 0 & 4593 \\
    408 & rambunctious\_jj & 0 & 4642 \\
    409 & secretive\_jj & 0 & 4670 \\
    410 & intrusiveness & 0 & 4690 \\
    411 & explosive\_jj & 0 & 4702 \\
    412 & curiosity & 0 & 4740 \\
    413 & worldly\_jj & 0 & 4850 \\
    414 & ethical\_jj & 0 & 4986 \\
    415 & intrusive\_jj & 0 & 4996 \\
    416 & imperturbable\_jj & 0 & 5065 \\
    417 & insecure\_jj & 0 & 5497 \\
    418 & reserve & 0 & 5517 \\
    419 & verbose\_jj & 0 & 5542 \\
    420 & shy\_jj & 0 & 5712 \\
    421 & lazy\_jj & 0 & 5828 \\
    422 & volatile\_jj & 0 & 6119 \\
    423 & courage & 0 & 6180 \\
    424 & punctuality & 0 & 6647 \\
    425 & rude\_jj & 0 & 7157 \\
    426 & demanding\_jj & 0 & 7843 \\
    427 & sophistication & 0 & 7986 \\
    428 & thrift & 0 & 8952 \\
    429 & touchy\_jj & 1 & 226 \\
    430 & independence & 1 & 377 \\
    \hline
    \caption{Scores and rankings for most extreme 30 words in component \#30} \\
\end{longtable}
\begin{longtable}[!htbp]{| rlr@{.}l |}
    \hline
    \textbf{Rank} & \textbf{Word} & \multicolumn{2}{c|}{\textbf{Score}} \\
    \hline
    \endhead
    1 & insight & -1 & -5852 \\
    2 & concise\_jj & -1 & -3226 \\
    3 & prompt\_jj & -1 & -1967 \\
    4 & independence & -1 & -583 \\
    5 & autonomous\_jj & -1 & -551 \\
    6 & withdrawn\_jj & 0 & -9377 \\
    7 & happy-go-lucky\_jj & 0 & -7424 \\
    8 & unemotional\_jj & 0 & -6939 \\
    9 & economical\_jj & 0 & -6331 \\
    10 & rebellious\_jj & 0 & -6231 \\
    11 & caustic\_jj & 0 & -5937 \\
    12 & scornful\_jj & 0 & -5840 \\
    13 & docile\_jj & 0 & -5748 \\
    14 & rude\_jj & 0 & -5741 \\
    15 & pomposity & 0 & -5720 \\
    16 & rambunctious\_jj & 0 & -5674 \\
    17 & independent\_jj & 0 & -5322 \\
    18 & lenient\_jj & 0 & -5156 \\
    19 & excitable\_jj & 0 & -4948 \\
    20 & humor & 0 & -4759 \\
    21 & unreliable\_jj & 0 & -4586 \\
    22 & insightful\_jj & 0 & -4351 \\
    23 & dependability & 0 & -4286 \\
    24 & reserve & 0 & -4073 \\
    25 & unfriendly\_jj & 0 & -4066 \\
    26 & perceptive\_jj & 0 & -4045 \\
    27 & easygoing\_jj & 0 & -4025 \\
    28 & distrustful\_jj & 0 & -3950 \\
    29 & folksy\_jj & 0 & -3919 \\
    30 & impetuous\_jj & 0 & -3909 \\
    401 & wishy-washy\_jj & 0 & 4110 \\
    402 & demanding\_jj & 0 & 4129 \\
    403 & unscrupulous\_jj & 0 & 4137 \\
    404 & contemplative\_jj & 0 & 4297 \\
    405 & expressiveness & 0 & 4360 \\
    406 & meditative\_jj & 0 & 4411 \\
    407 & warm\_jj & 0 & 4414 \\
    408 & unimaginative\_jj & 0 & 4514 \\
    409 & suggestible\_jj & 0 & 4530 \\
    410 & pleasant\_jj & 0 & 4553 \\
    411 & stinginess & 0 & 4614 \\
    412 & diplomatic\_jj & 0 & 4655 \\
    413 & touchy\_jj & 0 & 4754 \\
    414 & frank\_jj & 0 & 4758 \\
    415 & tactful\_jj & 0 & 4841 \\
    416 & fastidious\_jj & 0 & 5137 \\
    417 & punctuality & 0 & 5274 \\
    418 & meticulous\_jj & 0 & 5342 \\
    419 & imperturbable\_jj & 0 & 5444 \\
    420 & nonconforming\_jj & 0 & 5462 \\
    421 & compassionate\_jj & 0 & 5474 \\
    422 & careful\_jj & 0 & 5569 \\
    423 & patient\_jj & 0 & 5643 \\
    424 & charitable\_jj & 0 & 5673 \\
    425 & impolite\_jj & 0 & 6212 \\
    426 & sophistication & 0 & 6218 \\
    427 & conscientious\_jj & 0 & 6608 \\
    428 & exacting\_jj & 0 & 7614 \\
    429 & punctual\_jj & 0 & 7651 \\
    430 & explosive\_jj & 0 & 8822 \\
    \hline
    \caption{Scores and rankings for most extreme 30 words in component \#31} \\
\end{longtable}
\begin{longtable}[!htbp]{| rlr@{.}l |}
    \hline
    \textbf{Rank} & \textbf{Word} & \multicolumn{2}{c|}{\textbf{Score}} \\
    \hline
    \endhead
    1 & quarrelsome\_jj & -1 & -1169 \\
    2 & mannerly\_jj & 0 & -9750 \\
    3 & economical\_jj & 0 & -8444 \\
    4 & surliness & 0 & -7519 \\
    5 & silence & 0 & -6256 \\
    6 & organization & 0 & -6212 \\
    7 & negligent\_jj & 0 & -6165 \\
    8 & compassionate\_jj & 0 & -5831 \\
    9 & animation & 0 & -5644 \\
    10 & thorough\_jj & 0 & -5601 \\
    11 & inefficient\_jj & 0 & -5559 \\
    12 & sluggish\_jj & 0 & -5543 \\
    13 & abusive\_jj & 0 & -5469 \\
    14 & punctuality & 0 & -5398 \\
    15 & vindictive\_jj & 0 & -5313 \\
    16 & efficient\_jj & 0 & -5271 \\
    17 & self-indulgent\_jj & 0 & -5270 \\
    18 & lethargy & 0 & -5224 \\
    19 & absent-minded\_jj & 0 & -5212 \\
    20 & intelligence & 0 & -5019 \\
    21 & insensitive\_jj & 0 & -4813 \\
    22 & somber\_jj & 0 & -4607 \\
    23 & reliable\_jj & 0 & -4438 \\
    24 & friendly\_jj & 0 & -4392 \\
    25 & optimistic\_jj & 0 & -4390 \\
    26 & cooperative\_jj & 0 & -4323 \\
    27 & melancholic\_jj & 0 & -4319 \\
    28 & humorous\_jj & 0 & -4314 \\
    29 & intrusive\_jj & 0 & -4285 \\
    30 & negligence & 0 & -4172 \\
    401 & gullible\_jj & 0 & 4581 \\
    402 & cultured\_jj & 0 & 4654 \\
    403 & conceited\_jj & 0 & 4661 \\
    404 & unemotional\_jj & 0 & 4736 \\
    405 & orderly\_jj & 0 & 4773 \\
    406 & adaptable\_jj & 0 & 4838 \\
    407 & argumentative\_jj & 0 & 4847 \\
    408 & impolite\_jj & 0 & 5055 \\
    409 & submissive\_jj & 0 & 5148 \\
    410 & caution & 0 & 5232 \\
    411 & unconventional\_jj & 0 & 5275 \\
    412 & withdrawn\_jj & 0 & 5443 \\
    413 & insight & 0 & 5473 \\
    414 & unscrupulous\_jj & 0 & 5595 \\
    415 & obliging\_jj & 0 & 5819 \\
    416 & concise\_jj & 0 & 5824 \\
    417 & docile\_jj & 0 & 6040 \\
    418 & distrustful\_jj & 0 & 6067 \\
    419 & uncooperative\_jj & 0 & 6114 \\
    420 & cold\_jj & 0 & 6194 \\
    421 & imperturbable\_jj & 0 & 6300 \\
    422 & opportunistic\_jj & 0 & 6395 \\
    423 & volatility & 0 & 6634 \\
    424 & shyness & 0 & 6714 \\
    425 & prompt\_jj & 0 & 6856 \\
    426 & inconsiderate\_jj & 0 & 7777 \\
    427 & inhibition & 0 & 8546 \\
    428 & unsophisticated\_jj & 0 & 9201 \\
    429 & belligerence & 0 & 9409 \\
    430 & warm\_jj & 1 & 1711 \\
    \hline
    \caption{Scores and rankings for most extreme 30 words in component \#32} \\
\end{longtable}
\begin{longtable}[!htbp]{| rlr@{.}l |}
    \hline
    \textbf{Rank} & \textbf{Word} & \multicolumn{2}{c|}{\textbf{Score}} \\
    \hline
    \endhead
    1 & reserve & 0 & -9474 \\
    2 & stingy\_jj & 0 & -9024 \\
    3 & scornful\_jj & 0 & -8468 \\
    4 & sluggish\_jj & 0 & -7909 \\
    5 & steady\_jj & 0 & -7398 \\
    6 & intrusiveness & 0 & -7177 \\
    7 & undemanding\_jj & 0 & -7068 \\
    8 & earthiness & 0 & -6993 \\
    9 & insight & 0 & -6572 \\
    10 & expressive\_jj & 0 & -6491 \\
    11 & silence & 0 & -6238 \\
    12 & uncritical\_jj & 0 & -6237 \\
    13 & systematic\_jj & 0 & -6049 \\
    14 & unsophisticated\_jj & 0 & -5679 \\
    15 & sophistication & 0 & -5671 \\
    16 & nonconforming\_jj & 0 & -5662 \\
    17 & organized\_jj & 0 & -5495 \\
    18 & sloppy\_jj & 0 & -5102 \\
    19 & inhibition & 0 & -5020 \\
    20 & insensitive\_jj & 0 & -5017 \\
    21 & thrifty\_jj & 0 & -4987 \\
    22 & depth & 0 & -4881 \\
    23 & kind\_jj & 0 & -4879 \\
    24 & active\_jj & 0 & -4832 \\
    25 & inconsistent\_jj & 0 & -4825 \\
    26 & shallowness & 0 & -4674 \\
    27 & dominant\_jj & 0 & -4656 \\
    28 & defensive\_jj & 0 & -4597 \\
    29 & miserly\_jj & 0 & -4531 \\
    30 & passivity & 0 & -4407 \\
    401 & morose\_jj & 0 & 4016 \\
    402 & lenient\_jj & 0 & 4036 \\
    403 & predictability & 0 & 4076 \\
    404 & exacting\_jj & 0 & 4145 \\
    405 & spirit & 0 & 4162 \\
    406 & agreeable\_jj & 0 & 4262 \\
    407 & rash\_jj & 0 & 4351 \\
    408 & forgetful\_jj & 0 & 4386 \\
    409 & imperturbable\_jj & 0 & 4548 \\
    410 & suspicious\_jj & 0 & 4555 \\
    411 & happy-go-lucky\_jj & 0 & 4561 \\
    412 & homespun\_jj & 0 & 4565 \\
    413 & lethargy & 0 & 4696 \\
    414 & naturalness & 0 & 4725 \\
    415 & decisiveness & 0 & 4767 \\
    416 & logical\_jj & 0 & 5016 \\
    417 & surly\_jj & 0 & 5023 \\
    418 & quarrelsome\_jj & 0 & 5791 \\
    419 & pessimism & 0 & 6090 \\
    420 & leniency & 0 & 6140 \\
    421 & prompt\_jj & 0 & 6288 \\
    422 & compassionate\_jj & 0 & 6300 \\
    423 & cunning\_jj & 0 & 6433 \\
    424 & instability & 0 & 7574 \\
    425 & cooperation & 0 & 7835 \\
    426 & belligerence & 0 & 8143 \\
    427 & cranky\_jj & 0 & 8851 \\
    428 & refined\_jj & 0 & 9161 \\
    429 & concise\_jj & 0 & 9247 \\
    430 & grumpy\_jj & 0 & 9810 \\
    \hline
    \caption{Scores and rankings for most extreme 30 words in component \#33} \\
\end{longtable}
\begin{longtable}[!htbp]{| rlr@{.}l |}
    \hline
    \textbf{Rank} & \textbf{Word} & \multicolumn{2}{c|}{\textbf{Score}} \\
    \hline
    \endhead
    1 & compassionate\_jj & -1 & -2602 \\
    2 & leniency & -1 & -426 \\
    3 & erratic\_jj & 0 & -8872 \\
    4 & lenient\_jj & 0 & -8598 \\
    5 & mannerly\_jj & 0 & -7434 \\
    6 & courtesy & 0 & -7312 \\
    7 & cosmopolitan\_jj & 0 & -7100 \\
    8 & nonconforming\_jj & 0 & -6802 \\
    9 & belligerence & 0 & -6605 \\
    10 & rude\_jj & 0 & -6593 \\
    11 & aloofness & 0 & -5935 \\
    12 & orderly\_jj & 0 & -5841 \\
    13 & detached\_jj & 0 & -5803 \\
    14 & rudeness & 0 & -5390 \\
    15 & intelligence & 0 & -5389 \\
    16 & silence & 0 & -5364 \\
    17 & callousness & 0 & -5246 \\
    18 & withdrawn\_jj & 0 & -4811 \\
    19 & sophistication & 0 & -4795 \\
    20 & self-pitying\_jj & 0 & -4788 \\
    21 & wordy\_jj & 0 & -4687 \\
    22 & sophisticated\_jj & 0 & -4607 \\
    23 & conceited\_jj & 0 & -4596 \\
    24 & somber\_jj & 0 & -4552 \\
    25 & bigoted\_jj & 0 & -4444 \\
    26 & volatility & 0 & -4419 \\
    27 & perceptive\_jj & 0 & -4399 \\
    28 & punctuality & 0 & -4323 \\
    29 & moody\_jj & 0 & -4291 \\
    30 & dignified\_jj & 0 & -4286 \\
    401 & cordial\_jj & 0 & 4270 \\
    402 & philosophical\_jj & 0 & 4276 \\
    403 & rash\_jj & 0 & 4311 \\
    404 & reserve & 0 & 4368 \\
    405 & playful\_jj & 0 & 4380 \\
    406 & recklessness & 0 & 4423 \\
    407 & cranky\_jj & 0 & 4424 \\
    408 & sociable\_jj & 0 & 4583 \\
    409 & independence & 0 & 4648 \\
    410 & refined\_jj & 0 & 4686 \\
    411 & candor & 0 & 4719 \\
    412 & careless\_jj & 0 & 4794 \\
    413 & disorganization & 0 & 4862 \\
    414 & quarrelsome\_jj & 0 & 4978 \\
    415 & inquisitive\_jj & 0 & 5128 \\
    416 & touchy\_jj & 0 & 5140 \\
    417 & natural\_jj & 0 & 5315 \\
    418 & unkind\_jj & 0 & 5528 \\
    419 & argumentative\_jj & 0 & 5602 \\
    420 & sloth & 0 & 5635 \\
    421 & caustic\_jj & 0 & 5695 \\
    422 & warm\_jj & 0 & 6138 \\
    423 & decisiveness & 0 & 6476 \\
    424 & autonomous\_jj & 0 & 6485 \\
    425 & stingy\_jj & 0 & 6982 \\
    426 & rambunctious\_jj & 0 & 6986 \\
    427 & negligent\_jj & 0 & 7014 \\
    428 & naturalness & 0 & 8225 \\
    429 & meddlesome\_jj & 0 & 8305 \\
    430 & economical\_jj & 1 & 6956 \\
    \hline
    \caption{Scores and rankings for most extreme 30 words in component \#34} \\
\end{longtable}
\begin{longtable}[!htbp]{| rlr@{.}l |}
    \hline
    \textbf{Rank} & \textbf{Word} & \multicolumn{2}{c|}{\textbf{Score}} \\
    \hline
    \endhead
    1 & prompt\_jj & -1 & -7155 \\
    2 & warm\_jj & -1 & -506 \\
    3 & suspicious\_jj & 0 & -9979 \\
    4 & selfish\_jj & 0 & -8614 \\
    5 & belligerence & 0 & -6754 \\
    6 & folksy\_jj & 0 & -6559 \\
    7 & distrustful\_jj & 0 & -6363 \\
    8 & disorganization & 0 & -6324 \\
    9 & explosive\_jj & 0 & -6123 \\
    10 & detached\_jj & 0 & -6050 \\
    11 & inefficient\_jj & 0 & -6046 \\
    12 & punctual\_jj & 0 & -5604 \\
    13 & verbose\_jj & 0 & -5562 \\
    14 & self-disciplined\_jj & 0 & -5554 \\
    15 & unemotional\_jj & 0 & -5388 \\
    16 & uncritical\_jj & 0 & -5280 \\
    17 & charitable\_jj & 0 & -4964 \\
    18 & scornful\_jj & 0 & -4940 \\
    19 & erratic\_jj & 0 & -4897 \\
    20 & sloth & 0 & -4884 \\
    21 & melancholic\_jj & 0 & -4860 \\
    22 & lazy\_jj & 0 & -4810 \\
    23 & indecisiveness & 0 & -4747 \\
    24 & courtesy & 0 & -4607 \\
    25 & rudeness & 0 & -4602 \\
    26 & secretive\_jj & 0 & -4536 \\
    27 & cultured\_jj & 0 & -4478 \\
    28 & compassionate\_jj & 0 & -4455 \\
    29 & individualistic\_jj & 0 & -4395 \\
    30 & bossiness & 0 & -4309 \\
    401 & depth & 0 & 3935 \\
    402 & honest\_jj & 0 & 3945 \\
    403 & sophistication & 0 & 4082 \\
    404 & cooperative\_jj & 0 & 4088 \\
    405 & respectful\_jj & 0 & 4121 \\
    406 & irritable\_jj & 0 & 4175 \\
    407 & truthful\_jj & 0 & 4380 \\
    408 & cunning\_jj & 0 & 4543 \\
    409 & shallowness & 0 & 4591 \\
    410 & rude\_jj & 0 & 4601 \\
    411 & cordial\_jj & 0 & 4716 \\
    412 & independence & 0 & 4969 \\
    413 & ungracious\_jj & 0 & 4992 \\
    414 & understanding\_jj & 0 & 5124 \\
    415 & refined\_jj & 0 & 5401 \\
    416 & leniency & 0 & 5429 \\
    417 & underhanded\_jj & 0 & 5511 \\
    418 & deceitful\_jj & 0 & 6007 \\
    419 & cooperation & 0 & 6210 \\
    420 & submissive\_jj & 0 & 6250 \\
    421 & devious\_jj & 0 & 6656 \\
    422 & naturalness & 0 & 6856 \\
    423 & rebellious\_jj & 0 & 7054 \\
    424 & flexibility & 0 & 7150 \\
    425 & conceited\_jj & 0 & 7232 \\
    426 & manipulative\_jj & 0 & 7369 \\
    427 & self-esteem & 0 & 8173 \\
    428 & unsophisticated\_jj & 0 & 9068 \\
    429 & concise\_jj & 1 & 795 \\
    430 & unscrupulous\_jj & 1 & 1589 \\
    \hline
    \caption{Scores and rankings for most extreme 30 words in component \#35} \\
\end{longtable}
\begin{longtable}[!htbp]{| rlr@{.}l |}
    \hline
    \textbf{Rank} & \textbf{Word} & \multicolumn{2}{c|}{\textbf{Score}} \\
    \hline
    \endhead
    1 & mannerly\_jj & -1 & -2289 \\
    2 & nonconforming\_jj & -1 & -2186 \\
    3 & concise\_jj & -1 & -664 \\
    4 & earthiness & 0 & -7987 \\
    5 & careless\_jj & 0 & -7324 \\
    6 & reckless\_jj & 0 & -7261 \\
    7 & rude\_jj & 0 & -6851 \\
    8 & unintelligent\_jj & 0 & -6474 \\
    9 & autonomous\_jj & 0 & -6446 \\
    10 & expressive\_jj & 0 & -6224 \\
    11 & reserve & 0 & -6147 \\
    12 & bold\_jj & 0 & -6047 \\
    13 & brave\_jj & 0 & -5877 \\
    14 & belligerence & 0 & -5824 \\
    15 & uncooperative\_jj & 0 & -5528 \\
    16 & diplomatic\_jj & 0 & -5291 \\
    17 & unreflective\_jj & 0 & -5004 \\
    18 & earthy\_jj & 0 & -4933 \\
    19 & prompt\_jj & 0 & -4842 \\
    20 & gullibility & 0 & -4786 \\
    21 & independence & 0 & -4729 \\
    22 & assertive\_jj & 0 & -4595 \\
    23 & expressiveness & 0 & -4568 \\
    24 & wordy\_jj & 0 & -4549 \\
    25 & frank\_jj & 0 & -4306 \\
    26 & deep\_jj & 0 & -4204 \\
    27 & bossiness & 0 & -4129 \\
    28 & shallowness & 0 & -4099 \\
    29 & considerate\_jj & 0 & -4083 \\
    30 & vivacious\_jj & 0 & -4064 \\
    401 & envious\_jj & 0 & 4110 \\
    402 & unsympathetic\_jj & 0 & 4149 \\
    403 & antagonistic\_jj & 0 & 4225 \\
    404 & flippant\_jj & 0 & 4326 \\
    405 & shyness & 0 & 4331 \\
    406 & thrifty\_jj & 0 & 4364 \\
    407 & miserly\_jj & 0 & 4425 \\
    408 & modesty & 0 & 4430 \\
    409 & haphazard\_jj & 0 & 4494 \\
    410 & worldly\_jj & 0 & 4536 \\
    411 & exacting\_jj & 0 & 4581 \\
    412 & predictability & 0 & 4589 \\
    413 & secretive\_jj & 0 & 4727 \\
    414 & argumentative\_jj & 0 & 4796 \\
    415 & suspicious\_jj & 0 & 4862 \\
    416 & meticulous\_jj & 0 & 4863 \\
    417 & grumpy\_jj & 0 & 5120 \\
    418 & sentimental\_jj & 0 & 5139 \\
    419 & respectful\_jj & 0 & 5277 \\
    420 & orderly\_jj & 0 & 5583 \\
    421 & tempestuous\_jj & 0 & 5593 \\
    422 & explosive\_jj & 0 & 5649 \\
    423 & forgetful\_jj & 0 & 5717 \\
    424 & quiet & 0 & 6034 \\
    425 & happy-go-lucky\_jj & 0 & 7048 \\
    426 & insight & 0 & 7093 \\
    427 & unscrupulous\_jj & 0 & 7097 \\
    428 & silent\_jj & 0 & 7188 \\
    429 & refined\_jj & 0 & 8808 \\
    430 & silence & 1 & 3590 \\
    \hline
    \caption{Scores and rankings for most extreme 30 words in component \#36} \\
\end{longtable}
\begin{longtable}[!htbp]{| rlr@{.}l |}
    \hline
    \textbf{Rank} & \textbf{Word} & \multicolumn{2}{c|}{\textbf{Score}} \\
    \hline
    \endhead
    1 & warm\_jj & -1 & -3059 \\
    2 & somber\_jj & 0 & -8677 \\
    3 & refined\_jj & 0 & -8087 \\
    4 & punctual\_jj & 0 & -7981 \\
    5 & argumentative\_jj & 0 & -7103 \\
    6 & analytical\_jj & 0 & -6774 \\
    7 & unemotional\_jj & 0 & -6684 \\
    8 & jovial\_jj & 0 & -6540 \\
    9 & punctuality & 0 & -6535 \\
    10 & inconsiderate\_jj & 0 & -6350 \\
    11 & intrusiveness & 0 & -6257 \\
    12 & nonconforming\_jj & 0 & -6026 \\
    13 & irritability & 0 & -5975 \\
    14 & shallowness & 0 & -5495 \\
    15 & boastful\_jj & 0 & -5272 \\
    16 & courteous\_jj & 0 & -5236 \\
    17 & inventive\_jj & 0 & -5019 \\
    18 & unreflective\_jj & 0 & -4999 \\
    19 & cold\_jj & 0 & -4913 \\
    20 & pessimistic\_jj & 0 & -4872 \\
    21 & wishy-washy\_jj & 0 & -4871 \\
    22 & deceitful\_jj & 0 & -4806 \\
    23 & thrift & 0 & -4664 \\
    24 & scornful\_jj & 0 & -4570 \\
    25 & underhanded\_jj & 0 & -4548 \\
    26 & animation & 0 & -4425 \\
    27 & gruff\_jj & 0 & -4415 \\
    28 & cautious\_jj & 0 & -4410 \\
    29 & volatility & 0 & -4244 \\
    30 & rambunctious\_jj & 0 & -4185 \\
    401 & folksy\_jj & 0 & 4030 \\
    402 & neat\_jj & 0 & 4037 \\
    403 & lazy\_jj & 0 & 4145 \\
    404 & unimaginative\_jj & 0 & 4361 \\
    405 & inhibition & 0 & 4465 \\
    406 & bashful\_jj & 0 & 4475 \\
    407 & extroverted\_jj & 0 & 4692 \\
    408 & negligence & 0 & 4744 \\
    409 & vindictive\_jj & 0 & 4892 \\
    410 & sloth & 0 & 4900 \\
    411 & unadventurous\_jj & 0 & 4920 \\
    412 & wordy\_jj & 0 & 4940 \\
    413 & unrestrained\_jj & 0 & 4957 \\
    414 & jealous\_jj & 0 & 5059 \\
    415 & vain\_jj & 0 & 5101 \\
    416 & cruel\_jj & 0 & 5140 \\
    417 & obliging\_jj & 0 & 5382 \\
    418 & earthiness & 0 & 5558 \\
    419 & thorough\_jj & 0 & 5827 \\
    420 & naturalness & 0 & 6279 \\
    421 & insecure\_jj & 0 & 6351 \\
    422 & dignity & 0 & 6591 \\
    423 & compassionate\_jj & 0 & 6620 \\
    424 & nosey\_jj & 0 & 6654 \\
    425 & courtesy & 0 & 6681 \\
    426 & lethargic\_jj & 0 & 7013 \\
    427 & friendly\_jj & 0 & 7321 \\
    428 & irritable\_jj & 0 & 7445 \\
    429 & belligerence & 0 & 9173 \\
    430 & suspicious\_jj & 1 & 447 \\
    \hline
    \caption{Scores and rankings for most extreme 30 words in component \#37} \\
\end{longtable}
\begin{longtable}[!htbp]{| rlr@{.}l |}
    \hline
    \textbf{Rank} & \textbf{Word} & \multicolumn{2}{c|}{\textbf{Score}} \\
    \hline
    \endhead
    1 & nonconforming\_jj & 0 & -9302 \\
    2 & erratic\_jj & 0 & -9126 \\
    3 & distrustful\_jj & 0 & -8966 \\
    4 & homespun\_jj & 0 & -8601 \\
    5 & self-pitying\_jj & 0 & -8102 \\
    6 & negligent\_jj & 0 & -7521 \\
    7 & insight & 0 & -7133 \\
    8 & economical\_jj & 0 & -6980 \\
    9 & brave\_jj & 0 & -6232 \\
    10 & thrift & 0 & -6025 \\
    11 & skeptical\_jj & 0 & -5830 \\
    12 & careless\_jj & 0 & -5617 \\
    13 & recklessness & 0 & -5300 \\
    14 & earthiness & 0 & -5261 \\
    15 & touchy\_jj & 0 & -5219 \\
    16 & emotional\_jj & 0 & -5089 \\
    17 & reckless\_jj & 0 & -4962 \\
    18 & scornful\_jj & 0 & -4868 \\
    19 & helpful\_jj & 0 & -4631 \\
    20 & unscrupulous\_jj & 0 & -4562 \\
    21 & smart\_jj & 0 & -4522 \\
    22 & joyless\_jj & 0 & -4457 \\
    23 & self-esteem & 0 & -4430 \\
    24 & zestful\_jj & 0 & -4351 \\
    25 & fastidious\_jj & 0 & -4336 \\
    26 & impolite\_jj & 0 & -4171 \\
    27 & candor & 0 & -4075 \\
    28 & inefficient\_jj & 0 & -4071 \\
    29 & frank\_jj & 0 & -4004 \\
    30 & irritability & 0 & -3993 \\
    401 & thoughtless\_jj & 0 & 4037 \\
    402 & belligerence & 0 & 4267 \\
    403 & impudent\_jj & 0 & 4299 \\
    404 & courtesy & 0 & 4334 \\
    405 & vigorous\_jj & 0 & 4482 \\
    406 & depth & 0 & 4538 \\
    407 & lethargic\_jj & 0 & 4638 \\
    408 & meddlesome\_jj & 0 & 4810 \\
    409 & egotistical\_jj & 0 & 4813 \\
    410 & docile\_jj & 0 & 4858 \\
    411 & flexibility & 0 & 5060 \\
    412 & unrestrained\_jj & 0 & 5146 \\
    413 & shallow\_jj & 0 & 5180 \\
    414 & naïve\_jj & 0 & 5195 \\
    415 & bigoted\_jj & 0 & 5390 \\
    416 & condescending\_jj & 0 & 5400 \\
    417 & sluggish\_jj & 0 & 5557 \\
    418 & insensitive\_jj & 0 & 5708 \\
    419 & gullibility & 0 & 6100 \\
    420 & easygoing\_jj & 0 & 6107 \\
    421 & prompt\_jj & 0 & 6259 \\
    422 & punctuality & 0 & 6486 \\
    423 & shallowness & 0 & 6873 \\
    424 & autonomous\_jj & 0 & 6951 \\
    425 & sloth & 0 & 6977 \\
    426 & argumentative\_jj & 0 & 7381 \\
    427 & compassionate\_jj & 0 & 7431 \\
    428 & lazy\_jj & 0 & 8050 \\
    429 & explosive\_jj & 0 & 8133 \\
    430 & sophistication & 0 & 9828 \\
    \hline
    \caption{Scores and rankings for most extreme 30 words in component \#38} \\
\end{longtable}
\begin{longtable}[!htbp]{| rlr@{.}l |}
    \hline
    \textbf{Rank} & \textbf{Word} & \multicolumn{2}{c|}{\textbf{Score}} \\
    \hline
    \endhead
    1 & inconsiderate\_jj & 0 & -9080 \\
    2 & mannerly\_jj & 0 & -7816 \\
    3 & aimless\_jj & 0 & -6931 \\
    4 & impudent\_jj & 0 & -6506 \\
    5 & shallowness & 0 & -6243 \\
    6 & lethargy & 0 & -6214 \\
    7 & slothful\_jj & 0 & -6209 \\
    8 & autonomous\_jj & 0 & -6149 \\
    9 & haphazard\_jj & 0 & -6146 \\
    10 & sincere\_jj & 0 & -6036 \\
    11 & dependability & 0 & -5862 \\
    12 & industrious\_jj & 0 & -5702 \\
    13 & joyless\_jj & 0 & -5599 \\
    14 & neat\_jj & 0 & -5511 \\
    15 & exacting\_jj & 0 & -5237 \\
    16 & punctuality & 0 & -5234 \\
    17 & rudeness & 0 & -5049 \\
    18 & pessimism & 0 & -5042 \\
    19 & unemotional\_jj & 0 & -4939 \\
    20 & stubborn\_jj & 0 & -4632 \\
    21 & adventurous\_jj & 0 & -4547 \\
    22 & detached\_jj & 0 & -4527 \\
    23 & rash\_jj & 0 & -4415 \\
    24 & careless\_jj & 0 & -4311 \\
    25 & opportunistic\_jj & 0 & -4307 \\
    26 & pessimistic\_jj & 0 & -4291 \\
    27 & fastidious\_jj & 0 & -4239 \\
    28 & undemanding\_jj & 0 & -4189 \\
    29 & precise\_jj & 0 & -4155 \\
    30 & high-strung\_jj & 0 & -4019 \\
    401 & friendly\_jj & 0 & 4045 \\
    402 & humorous\_jj & 0 & 4257 \\
    403 & gullible\_jj & 0 & 4356 \\
    404 & surliness & 0 & 4465 \\
    405 & uncooperative\_jj & 0 & 4880 \\
    406 & devious\_jj & 0 & 4920 \\
    407 & aloofness & 0 & 5057 \\
    408 & thrift & 0 & 5078 \\
    409 & independence & 0 & 5146 \\
    410 & organized\_jj & 0 & 5175 \\
    411 & moody\_jj & 0 & 5181 \\
    412 & wishy-washy\_jj & 0 & 5310 \\
    413 & indecisiveness & 0 & 5529 \\
    414 & talkative\_jj & 0 & 5598 \\
    415 & curt\_jj & 0 & 5643 \\
    416 & forgetful\_jj & 0 & 5886 \\
    417 & conscientious\_jj & 0 & 5894 \\
    418 & boastful\_jj & 0 & 6099 \\
    419 & smart\_jj & 0 & 6122 \\
    420 & prejudiced\_jj & 0 & 6395 \\
    421 & irritable\_jj & 0 & 6419 \\
    422 & happy-go-lucky\_jj & 0 & 6646 \\
    423 & indecisive\_jj & 0 & 6998 \\
    424 & recklessness & 0 & 7069 \\
    425 & animation & 0 & 7921 \\
    426 & instability & 0 & 8016 \\
    427 & volatility & 0 & 8148 \\
    428 & warm\_jj & 0 & 8282 \\
    429 & reserve & 0 & 8401 \\
    430 & compassionate\_jj & 1 & 2819 \\
    \hline
    \caption{Scores and rankings for most extreme 30 words in component \#39} \\
\end{longtable}
\begin{longtable}[!htbp]{| rlr@{.}l |}
    \hline
    \textbf{Rank} & \textbf{Word} & \multicolumn{2}{c|}{\textbf{Score}} \\
    \hline
    \endhead
    1 & prompt\_jj & -1 & -4890 \\
    2 & cordial\_jj & -1 & -2165 \\
    3 & negligent\_jj & -1 & -978 \\
    4 & warm\_jj & 0 & -9226 \\
    5 & intellectual\_jj & 0 & -7088 \\
    6 & adventurous\_jj & 0 & -5847 \\
    7 & artistic\_jj & 0 & -5667 \\
    8 & silence & 0 & -5621 \\
    9 & ungracious\_jj & 0 & -5445 \\
    10 & unscrupulous\_jj & 0 & -5070 \\
    11 & friendly\_jj & 0 & -4923 \\
    12 & folksy\_jj & 0 & -4897 \\
    13 & firm\_jj & 0 & -4838 \\
    14 & frank\_jj & 0 & -4739 \\
    15 & egotistical\_jj & 0 & -4644 \\
    16 & assertive\_jj & 0 & -4617 \\
    17 & opportunistic\_jj & 0 & -4500 \\
    18 & selfish\_jj & 0 & -4433 \\
    19 & condescending\_jj & 0 & -4366 \\
    20 & emotional\_jj & 0 & -4328 \\
    21 & lazy\_jj & 0 & -4320 \\
    22 & self-critical\_jj & 0 & -4319 \\
    23 & warmth & 0 & -4222 \\
    24 & creative\_jj & 0 & -4172 \\
    25 & passivity & 0 & -4139 \\
    26 & innovative\_jj & 0 & -4019 \\
    27 & smug\_jj & 0 & -4019 \\
    28 & impersonal\_jj & 0 & -3979 \\
    29 & self-esteem & 0 & -3952 \\
    30 & detached\_jj & 0 & -3923 \\
    401 & underhanded\_jj & 0 & 3992 \\
    402 & cruelty & 0 & 4048 \\
    403 & foolhardy\_jj & 0 & 4112 \\
    404 & secretive\_jj & 0 & 4181 \\
    405 & callousness & 0 & 4296 \\
    406 & gregarious\_jj & 0 & 4324 \\
    407 & disorganization & 0 & 4327 \\
    408 & cunning & 0 & 4354 \\
    409 & organization & 0 & 4433 \\
    410 & generosity & 0 & 4437 \\
    411 & belligerence & 0 & 4458 \\
    412 & jovial\_jj & 0 & 4488 \\
    413 & courteous\_jj & 0 & 4516 \\
    414 & surliness & 0 & 4566 \\
    415 & pomposity & 0 & 4578 \\
    416 & careless\_jj & 0 & 4645 \\
    417 & miserly\_jj & 0 & 4783 \\
    418 & rash\_jj & 0 & 4899 \\
    419 & cruel\_jj & 0 & 5008 \\
    420 & autonomous\_jj & 0 & 5111 \\
    421 & impudent\_jj & 0 & 5141 \\
    422 & understanding\_jj & 0 & 5347 \\
    423 & volatile\_jj & 0 & 5519 \\
    424 & touchy\_jj & 0 & 5536 \\
    425 & suspicious\_jj & 0 & 6278 \\
    426 & compassionate\_jj & 0 & 6293 \\
    427 & refined\_jj & 0 & 6779 \\
    428 & courtesy & 0 & 7102 \\
    429 & lenient\_jj & 0 & 9286 \\
    430 & charitable\_jj & 0 & 9743 \\
    \hline
    \caption{Scores and rankings for most extreme 30 words in component \#40} \\
\end{longtable}
\begin{longtable}[!htbp]{| rlr@{.}l |}
    \hline
    \textbf{Rank} & \textbf{Word} & \multicolumn{2}{c|}{\textbf{Score}} \\
    \hline
    \endhead
    1 & economical\_jj & -1 & -561 \\
    2 & assured\_jj & -1 & -399 \\
    3 & thorough\_jj & 0 & -7824 \\
    4 & rebellious\_jj & 0 & -7657 \\
    5 & verbal\_jj & 0 & -7217 \\
    6 & nonconforming\_jj & 0 & -6115 \\
    7 & impolite\_jj & 0 & -6003 \\
    8 & animation & 0 & -5773 \\
    9 & insecure\_jj & 0 & -5636 \\
    10 & negligence & 0 & -5627 \\
    11 & informal\_jj & 0 & -5401 \\
    12 & unrestrained\_jj & 0 & -5173 \\
    13 & leniency & 0 & -5050 \\
    14 & lenient\_jj & 0 & -4964 \\
    15 & independent\_jj & 0 & -4927 \\
    16 & unassuming\_jj & 0 & -4882 \\
    17 & inconsiderate\_jj & 0 & -4780 \\
    18 & punctuality & 0 & -4722 \\
    19 & autonomous\_jj & 0 & -4712 \\
    20 & bashful\_jj & 0 & -4557 \\
    21 & moral\_jj & 0 & -4421 \\
    22 & instability & 0 & -4408 \\
    23 & unambitious\_jj & 0 & -4405 \\
    24 & aloofness & 0 & -4339 \\
    25 & introspective\_jj & 0 & -4338 \\
    26 & bossiness & 0 & -4216 \\
    27 & conceited\_jj & 0 & -4210 \\
    28 & shyness & 0 & -4208 \\
    29 & frank\_jj & 0 & -4164 \\
    30 & careless\_jj & 0 & -4047 \\
    401 & courteous\_jj & 0 & 4133 \\
    402 & deliberate\_jj & 0 & 4139 \\
    403 & pessimistic\_jj & 0 & 4226 \\
    404 & manipulative\_jj & 0 & 4228 \\
    405 & pessimism & 0 & 4369 \\
    406 & helpful\_jj & 0 & 4434 \\
    407 & lethargy & 0 & 4478 \\
    408 & independence & 0 & 4613 \\
    409 & prompt\_jj & 0 & 4628 \\
    410 & modesty & 0 & 4685 \\
    411 & playfulness & 0 & 4869 \\
    412 & insight & 0 & 4977 \\
    413 & volatility & 0 & 4991 \\
    414 & explosive\_jj & 0 & 5140 \\
    415 & irritability & 0 & 5168 \\
    416 & disrespectful\_jj & 0 & 5378 \\
    417 & thrifty\_jj & 0 & 5398 \\
    418 & accommodating\_jj & 0 & 5492 \\
    419 & suspicious\_jj & 0 & 5515 \\
    420 & deceitful\_jj & 0 & 5573 \\
    421 & shallow\_jj & 0 & 5942 \\
    422 & devious\_jj & 0 & 6037 \\
    423 & earthiness & 0 & 6070 \\
    424 & surly\_jj & 0 & 6213 \\
    425 & stinginess & 0 & 6252 \\
    426 & individualistic\_jj & 0 & 6688 \\
    427 & foolhardy\_jj & 0 & 7278 \\
    428 & intelligence & 0 & 7391 \\
    429 & negligent\_jj & 0 & 9860 \\
    430 & mannerly\_jj & 1 & 338 \\
    \hline
    \caption{Scores and rankings for most extreme 30 words in component \#41} \\
\end{longtable}
\begin{longtable}[!htbp]{| rlr@{.}l |}
    \hline
    \textbf{Rank} & \textbf{Word} & \multicolumn{2}{c|}{\textbf{Score}} \\
    \hline
    \endhead
    1 & orderly\_jj & -1 & -1820 \\
    2 & sophistication & 0 & -8326 \\
    3 & sloth & 0 & -7845 \\
    4 & pomposity & 0 & -7083 \\
    5 & thorough\_jj & 0 & -6829 \\
    6 & refined\_jj & 0 & -6362 \\
    7 & flippant\_jj & 0 & -5776 \\
    8 & earthiness & 0 & -5661 \\
    9 & easygoing\_jj & 0 & -5538 \\
    10 & erratic\_jj & 0 & -5337 \\
    11 & naturalness & 0 & -5320 \\
    12 & logical\_jj & 0 & -5261 \\
    13 & agreeable\_jj & 0 & -5216 \\
    14 & haphazard\_jj & 0 & -5121 \\
    15 & artistic\_jj & 0 & -5043 \\
    16 & negligent\_jj & 0 & -4895 \\
    17 & cosmopolitan\_jj & 0 & -4727 \\
    18 & insensitive\_jj & 0 & -4678 \\
    19 & volatility & 0 & -4489 \\
    20 & naïve\_jj & 0 & -4475 \\
    21 & detached\_jj & 0 & -4409 \\
    22 & impolite\_jj & 0 & -4264 \\
    23 & selfish\_jj & 0 & -4208 \\
    24 & envious\_jj & 0 & -4178 \\
    25 & extravagant\_jj & 0 & -4037 \\
    26 & humble\_jj & 0 & -3944 \\
    27 & careful\_jj & 0 & -3931 \\
    28 & inhibition & 0 & -3859 \\
    29 & condescending\_jj & 0 & -3831 \\
    30 & fastidious\_jj & 0 & -3816 \\
    401 & stinginess & 0 & 4191 \\
    402 & explosive\_jj & 0 & 4295 \\
    403 & punctual\_jj & 0 & 4319 \\
    404 & cruelty & 0 & 4347 \\
    405 & introspective\_jj & 0 & 4387 \\
    406 & wordy\_jj & 0 & 4387 \\
    407 & shallowness & 0 & 4502 \\
    408 & merry\_jj & 0 & 4540 \\
    409 & argumentative\_jj & 0 & 4597 \\
    410 & self-pitying\_jj & 0 & 4598 \\
    411 & unsophisticated\_jj & 0 & 4695 \\
    412 & insight & 0 & 4941 \\
    413 & rebellious\_jj & 0 & 4944 \\
    414 & impudent\_jj & 0 & 5197 \\
    415 & expressive\_jj & 0 & 5922 \\
    416 & morose\_jj & 0 & 6102 \\
    417 & unscrupulous\_jj & 0 & 6263 \\
    418 & aimless\_jj & 0 & 6305 \\
    419 & contemplative\_jj & 0 & 6412 \\
    420 & deliberate\_jj & 0 & 6467 \\
    421 & efficiency & 0 & 6798 \\
    422 & expressiveness & 0 & 6805 \\
    423 & dependability & 0 & 6935 \\
    424 & gullible\_jj & 0 & 7062 \\
    425 & punctuality & 0 & 7071 \\
    426 & meditative\_jj & 0 & 7253 \\
    427 & defensive\_jj & 0 & 7294 \\
    428 & friendly\_jj & 0 & 8779 \\
    429 & economical\_jj & 0 & 9357 \\
    430 & folksy\_jj & 1 & 3384 \\
    \hline
    \caption{Scores and rankings for most extreme 30 words in component \#42} \\
\end{longtable}
\begin{longtable}[!htbp]{| rlr@{.}l |}
    \hline
    \textbf{Rank} & \textbf{Word} & \multicolumn{2}{c|}{\textbf{Score}} \\
    \hline
    \endhead
    1 & refined\_jj & -1 & -900 \\
    2 & negligence & 0 & -7647 \\
    3 & disorganization & 0 & -7566 \\
    4 & distrustful\_jj & 0 & -7066 \\
    5 & cordial\_jj & 0 & -6627 \\
    6 & vain\_jj & 0 & -6571 \\
    7 & selfish\_jj & 0 & -6434 \\
    8 & irritable\_jj & 0 & -5996 \\
    9 & cosmopolitan\_jj & 0 & -5728 \\
    10 & unsophisticated\_jj & 0 & -5665 \\
    11 & perceptive\_jj & 0 & -5605 \\
    12 & natural\_jj & 0 & -5557 \\
    13 & aimless\_jj & 0 & -5503 \\
    14 & punctuality & 0 & -5490 \\
    15 & impudent\_jj & 0 & -5415 \\
    16 & shallowness & 0 & -5369 \\
    17 & leniency & 0 & -5251 \\
    18 & punctual\_jj & 0 & -4991 \\
    19 & explosive\_jj & 0 & -4956 \\
    20 & intrusiveness & 0 & -4852 \\
    21 & responsible\_jj & 0 & -4600 \\
    22 & cultured\_jj & 0 & -4479 \\
    23 & unimaginative\_jj & 0 & -4454 \\
    24 & disrespectful\_jj & 0 & -4447 \\
    25 & gregarious\_jj & 0 & -4387 \\
    26 & diplomatic\_jj & 0 & -4334 \\
    27 & defensive\_jj & 0 & -4309 \\
    28 & rude\_jj & 0 & -4270 \\
    29 & earthiness & 0 & -4220 \\
    30 & frivolous\_jj & 0 & -4091 \\
    401 & honest\_jj & 0 & 3783 \\
    402 & harsh\_jj & 0 & 3836 \\
    403 & fastidious\_jj & 0 & 4109 \\
    404 & bossiness & 0 & 4125 \\
    405 & cautious\_jj & 0 & 4148 \\
    406 & caustic\_jj & 0 & 4216 \\
    407 & cruelty & 0 & 4247 \\
    408 & homespun\_jj & 0 & 4301 \\
    409 & communicative\_jj & 0 & 4310 \\
    410 & withdrawn\_jj & 0 & 4310 \\
    411 & nonconformity & 0 & 4422 \\
    412 & prejudiced\_jj & 0 & 4528 \\
    413 & thrift & 0 & 4986 \\
    414 & conscientious\_jj & 0 & 5126 \\
    415 & steady\_jj & 0 & 5282 \\
    416 & obliging\_jj & 0 & 5432 \\
    417 & careful\_jj & 0 & 5900 \\
    418 & charitable\_jj & 0 & 5935 \\
    419 & compassionate\_jj & 0 & 6291 \\
    420 & unscrupulous\_jj & 0 & 6353 \\
    421 & underhanded\_jj & 0 & 6437 \\
    422 & pomposity & 0 & 6451 \\
    423 & deliberate\_jj & 0 & 6534 \\
    424 & caution & 0 & 6687 \\
    425 & independent\_jj & 0 & 6886 \\
    426 & warm\_jj & 0 & 7001 \\
    427 & unkind\_jj & 0 & 7557 \\
    428 & thorough\_jj & 0 & 7622 \\
    429 & folksy\_jj & 0 & 9326 \\
    430 & mannerly\_jj & 1 & 6412 \\
    \hline
    \caption{Scores and rankings for most extreme 30 words in component \#43} \\
\end{longtable}
\begin{longtable}[!htbp]{| rlr@{.}l |}
    \hline
    \textbf{Rank} & \textbf{Word} & \multicolumn{2}{c|}{\textbf{Score}} \\
    \hline
    \endhead
    1 & suspicious\_jj & -1 & -2870 \\
    2 & negligence & 0 & -7393 \\
    3 & detached\_jj & 0 & -7375 \\
    4 & obstinate\_jj & 0 & -7201 \\
    5 & high-strung\_jj & 0 & -7078 \\
    6 & meddlesome\_jj & 0 & -6605 \\
    7 & unreflective\_jj & 0 & -6605 \\
    8 & cunning\_jj & 0 & -6311 \\
    9 & extroverted\_jj & 0 & -6216 \\
    10 & insight & 0 & -5791 \\
    11 & self-pitying\_jj & 0 & -5576 \\
    12 & economical\_jj & 0 & -5273 \\
    13 & belligerence & 0 & -5187 \\
    14 & inconsistency & 0 & -5013 \\
    15 & unassuming\_jj & 0 & -4942 \\
    16 & uncritical\_jj & 0 & -4771 \\
    17 & expressive\_jj & 0 & -4685 \\
    18 & prejudice & 0 & -4623 \\
    19 & bossy\_jj & 0 & -4616 \\
    20 & unsympathetic\_jj & 0 & -4515 \\
    21 & bashful\_jj & 0 & -4498 \\
    22 & mannerly\_jj & 0 & -4497 \\
    23 & pessimism & 0 & -4383 \\
    24 & neat\_jj & 0 & -4325 \\
    25 & animation & 0 & -4302 \\
    26 & principled\_jj & 0 & -4245 \\
    27 & rebellious\_jj & 0 & -4066 \\
    28 & reserved\_jj & 0 & -3981 \\
    29 & contemplative\_jj & 0 & -3922 \\
    30 & forgetful\_jj & 0 & -3894 \\
    401 & envious\_jj & 0 & 3776 \\
    402 & naïve\_jj & 0 & 3783 \\
    403 & cultured\_jj & 0 & 3790 \\
    404 & fearful\_jj & 0 & 3794 \\
    405 & impolite\_jj & 0 & 3815 \\
    406 & nervous\_jj & 0 & 3819 \\
    407 & sociable\_jj & 0 & 3917 \\
    408 & zestful\_jj & 0 & 3949 \\
    409 & quarrelsome\_jj & 0 & 3954 \\
    410 & reckless\_jj & 0 & 3990 \\
    411 & cosmopolitan\_jj & 0 & 4366 \\
    412 & cruelty & 0 & 4430 \\
    413 & rash\_jj & 0 & 4934 \\
    414 & volatile\_jj & 0 & 5019 \\
    415 & jealous\_jj & 0 & 5065 \\
    416 & lazy\_jj & 0 & 5282 \\
    417 & naturalness & 0 & 5380 \\
    418 & impudent\_jj & 0 & 5490 \\
    419 & exacting\_jj & 0 & 5554 \\
    420 & ethical\_jj & 0 & 5669 \\
    421 & silence & 0 & 5987 \\
    422 & surliness & 0 & 6051 \\
    423 & dependability & 0 & 6065 \\
    424 & surly\_jj & 0 & 6398 \\
    425 & assured\_jj & 0 & 6788 \\
    426 & compassionate\_jj & 0 & 7675 \\
    427 & shallowness & 0 & 7939 \\
    428 & uncharitable\_jj & 0 & 8126 \\
    429 & tempestuous\_jj & 0 & 9391 \\
    430 & caustic\_jj & 1 & 55 \\
    \hline
    \caption{Scores and rankings for most extreme 30 words in component \#44} \\
\end{longtable}
\begin{longtable}[!htbp]{| rlr@{.}l |}
    \hline
    \textbf{Rank} & \textbf{Word} & \multicolumn{2}{c|}{\textbf{Score}} \\
    \hline
    \endhead
    1 & mannerly\_jj & 0 & -8774 \\
    2 & leniency & 0 & -8541 \\
    3 & self-pitying\_jj & 0 & -7870 \\
    4 & selfish\_jj & 0 & -7760 \\
    5 & recklessness & 0 & -7467 \\
    6 & reserve & 0 & -7222 \\
    7 & reckless\_jj & 0 & -6099 \\
    8 & abusive\_jj & 0 & -5996 \\
    9 & deliberate\_jj & 0 & -5816 \\
    10 & depth & 0 & -5718 \\
    11 & independence & 0 & -5481 \\
    12 & shallow\_jj & 0 & -5443 \\
    13 & underhanded\_jj & 0 & -5174 \\
    14 & economical\_jj & 0 & -4898 \\
    15 & autonomous\_jj & 0 & -4810 \\
    16 & curiosity & 0 & -4797 \\
    17 & insight & 0 & -4729 \\
    18 & gruff\_jj & 0 & -4675 \\
    19 & vigorous\_jj & 0 & -4565 \\
    20 & crabby\_jj & 0 & -4498 \\
    21 & logical\_jj & 0 & -4385 \\
    22 & boastful\_jj & 0 & -4339 \\
    23 & adventurous\_jj & 0 & -4324 \\
    24 & genial\_jj & 0 & -4291 \\
    25 & punctual\_jj & 0 & -4255 \\
    26 & meditative\_jj & 0 & -4150 \\
    27 & rash\_jj & 0 & -4104 \\
    28 & excitable\_jj & 0 & -4100 \\
    29 & uncritical\_jj & 0 & -3988 \\
    30 & inhibition & 0 & -3869 \\
    401 & dignity & 0 & 3753 \\
    402 & joyless\_jj & 0 & 3792 \\
    403 & insecure\_jj & 0 & 3812 \\
    404 & unkind\_jj & 0 & 3821 \\
    405 & charitable\_jj & 0 & 3830 \\
    406 & snobbish\_jj & 0 & 3864 \\
    407 & spirit & 0 & 4038 \\
    408 & refined\_jj & 0 & 4073 \\
    409 & firm\_jj & 0 & 4074 \\
    410 & expressive\_jj & 0 & 4092 \\
    411 & stingy\_jj & 0 & 4397 \\
    412 & suspicious\_jj & 0 & 4430 \\
    413 & kind\_jj & 0 & 4450 \\
    414 & inconsistent\_jj & 0 & 4469 \\
    415 & zestful\_jj & 0 & 4559 \\
    416 & surliness & 0 & 4625 \\
    417 & unreliable\_jj & 0 & 4654 \\
    418 & expressiveness & 0 & 4751 \\
    419 & cruel\_jj & 0 & 4875 \\
    420 & passivity & 0 & 5222 \\
    421 & proud\_jj & 0 & 5260 \\
    422 & nonconforming\_jj & 0 & 6267 \\
    423 & negligent\_jj & 0 & 6494 \\
    424 & friendly\_jj & 0 & 6813 \\
    425 & disrespectful\_jj & 0 & 7680 \\
    426 & rude\_jj & 0 & 8517 \\
    427 & prejudice & 0 & 8648 \\
    428 & animation & 0 & 9218 \\
    429 & quarrelsome\_jj & 0 & 9933 \\
    430 & prompt\_jj & 1 & 1784 \\
    \hline
    \caption{Scores and rankings for most extreme 30 words in component \#45} \\
\end{longtable}
\begin{longtable}[!htbp]{| rlr@{.}l |}
    \hline
    \textbf{Rank} & \textbf{Word} & \multicolumn{2}{c|}{\textbf{Score}} \\
    \hline
    \endhead
    1 & inhibition & -1 & -69 \\
    2 & lazy\_jj & 0 & -9828 \\
    3 & neat\_jj & 0 & -9292 \\
    4 & ungracious\_jj & 0 & -7690 \\
    5 & compassionate\_jj & 0 & -7631 \\
    6 & refined\_jj & 0 & -6467 \\
    7 & defensive\_jj & 0 & -6177 \\
    8 & slothful\_jj & 0 & -5845 \\
    9 & concise\_jj & 0 & -5808 \\
    10 & inconsiderate\_jj & 0 & -5656 \\
    11 & peaceful\_jj & 0 & -5481 \\
    12 & self-disciplined\_jj & 0 & -5442 \\
    13 & creative\_jj & 0 & -5368 \\
    14 & curious\_jj & 0 & -5271 \\
    15 & wishy-washy\_jj & 0 & -5251 \\
    16 & earthiness & 0 & -5231 \\
    17 & orderly\_jj & 0 & -5188 \\
    18 & diplomatic\_jj & 0 & -5078 \\
    19 & artistic\_jj & 0 & -5025 \\
    20 & cunning\_jj & 0 & -4962 \\
    21 & argumentative\_jj & 0 & -4829 \\
    22 & indecisive\_jj & 0 & -4744 \\
    23 & distrustful\_jj & 0 & -4477 \\
    24 & energetic\_jj & 0 & -4374 \\
    25 & economical\_jj & 0 & -4339 \\
    26 & aimless\_jj & 0 & -4300 \\
    27 & silent\_jj & 0 & -4275 \\
    28 & unsociable\_jj & 0 & -4253 \\
    29 & aloofness & 0 & -4157 \\
    30 & surliness & 0 & -4127 \\
    401 & accommodating\_jj & 0 & 3903 \\
    402 & thrift & 0 & 3986 \\
    403 & negligent\_jj & 0 & 4017 \\
    404 & modesty & 0 & 4140 \\
    405 & unsophisticated\_jj & 0 & 4174 \\
    406 & tactless\_jj & 0 & 4176 \\
    407 & predictability & 0 & 4207 \\
    408 & truthful\_jj & 0 & 4269 \\
    409 & impetuous\_jj & 0 & 4366 \\
    410 & humorous\_jj & 0 & 4368 \\
    411 & unemotional\_jj & 0 & 4372 \\
    412 & unscrupulous\_jj & 0 & 4431 \\
    413 & vivacious\_jj & 0 & 4523 \\
    414 & logic & 0 & 4534 \\
    415 & consistent\_jj & 0 & 4642 \\
    416 & autonomous\_jj & 0 & 4654 \\
    417 & logical\_jj & 0 & 4700 \\
    418 & nonconformity & 0 & 4788 \\
    419 & tempestuous\_jj & 0 & 4864 \\
    420 & demanding\_jj & 0 & 4971 \\
    421 & cold\_jj & 0 & 5066 \\
    422 & high-strung\_jj & 0 & 5096 \\
    423 & intrusive\_jj & 0 & 5110 \\
    424 & irritability & 0 & 5310 \\
    425 & forgetful\_jj & 0 & 5316 \\
    426 & reasonable\_jj & 0 & 6027 \\
    427 & foolhardy\_jj & 0 & 6056 \\
    428 & uncharitable\_jj & 0 & 6179 \\
    429 & perceptive\_jj & 0 & 6255 \\
    430 & unreflective\_jj & 1 & 1179 \\
    \hline
    \caption{Scores and rankings for most extreme 30 words in component \#46} \\
\end{longtable}
\begin{longtable}[!htbp]{| rlr@{.}l |}
    \hline
    \textbf{Rank} & \textbf{Word} & \multicolumn{2}{c|}{\textbf{Score}} \\
    \hline
    \endhead
    1 & refined\_jj & -1 & -3982 \\
    2 & independence & 0 & -8114 \\
    3 & peaceful\_jj & 0 & -8102 \\
    4 & tempestuous\_jj & 0 & -6706 \\
    5 & intelligence & 0 & -6638 \\
    6 & cordial\_jj & 0 & -6361 \\
    7 & unscrupulous\_jj & 0 & -6106 \\
    8 & lazy\_jj & 0 & -5834 \\
    9 & selfish\_jj & 0 & -5747 \\
    10 & lethargic\_jj & 0 & -5739 \\
    11 & wordy\_jj & 0 & -5660 \\
    12 & inconsistent\_jj & 0 & -5633 \\
    13 & erratic\_jj & 0 & -5488 \\
    14 & melancholic\_jj & 0 & -5428 \\
    15 & underhanded\_jj & 0 & -5222 \\
    16 & communicative\_jj & 0 & -5018 \\
    17 & vain\_jj & 0 & -4836 \\
    18 & surliness & 0 & -4758 \\
    19 & candor & 0 & -4695 \\
    20 & nonconforming\_jj & 0 & -4582 \\
    21 & mannerly\_jj & 0 & -4580 \\
    22 & steady\_jj & 0 & -4547 \\
    23 & stubbornness & 0 & -4520 \\
    24 & uncritical\_jj & 0 & -4454 \\
    25 & irritable\_jj & 0 & -4170 \\
    26 & intrusiveness & 0 & -4124 \\
    27 & truthful\_jj & 0 & -4063 \\
    28 & unkind\_jj & 0 & -4035 \\
    29 & creativity & 0 & -3965 \\
    30 & unreliable\_jj & 0 & -3943 \\
    401 & distrust & 0 & 3891 \\
    402 & crabby\_jj & 0 & 3907 \\
    403 & humble\_jj & 0 & 3914 \\
    404 & explosive\_jj & 0 & 3959 \\
    405 & unassuming\_jj & 0 & 4031 \\
    406 & generous\_jj & 0 & 4071 \\
    407 & efficient\_jj & 0 & 4172 \\
    408 & combative\_jj & 0 & 4177 \\
    409 & jovial\_jj & 0 & 4267 \\
    410 & negligent\_jj & 0 & 4345 \\
    411 & dignified\_jj & 0 & 4355 \\
    412 & natural\_jj & 0 & 4401 \\
    413 & detached\_jj & 0 & 4529 \\
    414 & charitable\_jj & 0 & 4599 \\
    415 & rude\_jj & 0 & 4753 \\
    416 & egotistical\_jj & 0 & 4813 \\
    417 & fretful\_jj & 0 & 5004 \\
    418 & concise\_jj & 0 & 5523 \\
    419 & deep\_jj & 0 & 5630 \\
    420 & somber\_jj & 0 & 5819 \\
    421 & shallow\_jj & 0 & 5941 \\
    422 & prejudiced\_jj & 0 & 5947 \\
    423 & ungracious\_jj & 0 & 6449 \\
    424 & impolite\_jj & 0 & 6508 \\
    425 & negligence & 0 & 6710 \\
    426 & sloth & 0 & 6946 \\
    427 & economical\_jj & 0 & 7296 \\
    428 & thorough\_jj & 0 & 8379 \\
    429 & shallowness & 0 & 8816 \\
    430 & insight & 1 & 183 \\
    \hline
    \caption{Scores and rankings for most extreme 30 words in component \#47} \\
\end{longtable}
\begin{longtable}[!htbp]{| rlr@{.}l |}
    \hline
    \textbf{Rank} & \textbf{Word} & \multicolumn{2}{c|}{\textbf{Score}} \\
    \hline
    \endhead
    1 & autonomous\_jj & -1 & -903 \\
    2 & selfish\_jj & 0 & -9429 \\
    3 & irritability & 0 & -8727 \\
    4 & independent\_jj & 0 & -8484 \\
    5 & independence & 0 & -7729 \\
    6 & animation & 0 & -7332 \\
    7 & modesty & 0 & -6848 \\
    8 & belligerence & 0 & -5973 \\
    9 & miserly\_jj & 0 & -5926 \\
    10 & unimaginative\_jj & 0 & -5391 \\
    11 & morose\_jj & 0 & -5322 \\
    12 & erratic\_jj & 0 & -4935 \\
    13 & earthy\_jj & 0 & -4871 \\
    14 & fastidious\_jj & 0 & -4863 \\
    15 & lethargy & 0 & -4841 \\
    16 & deceitful\_jj & 0 & -4836 \\
    17 & cordial\_jj & 0 & -4790 \\
    18 & meddlesome\_jj & 0 & -4756 \\
    19 & earthiness & 0 & -4687 \\
    20 & wordy\_jj & 0 & -4686 \\
    21 & thrifty\_jj & 0 & -4566 \\
    22 & scornful\_jj & 0 & -4537 \\
    23 & frank\_jj & 0 & -4398 \\
    24 & aimlessness & 0 & -4388 \\
    25 & argumentative\_jj & 0 & -4383 \\
    26 & greedy\_jj & 0 & -4367 \\
    27 & dependability & 0 & -4332 \\
    28 & dishonest\_jj & 0 & -4071 \\
    29 & caution & 0 & -3971 \\
    30 & honest\_jj & 0 & -3931 \\
    401 & natural\_jj & 0 & 3700 \\
    402 & skeptical\_jj & 0 & 3735 \\
    403 & defensive\_jj & 0 & 3736 \\
    404 & self-disciplined\_jj & 0 & 3744 \\
    405 & excitable\_jj & 0 & 3747 \\
    406 & anxious\_jj & 0 & 3799 \\
    407 & passivity & 0 & 3810 \\
    408 & happy-go-lucky\_jj & 0 & 3844 \\
    409 & concise\_jj & 0 & 3867 \\
    410 & organized\_jj & 0 & 3917 \\
    411 & steady\_jj & 0 & 4080 \\
    412 & envy & 0 & 4083 \\
    413 & detached\_jj & 0 & 4095 \\
    414 & talkative\_jj & 0 & 4196 \\
    415 & shallow\_jj & 0 & 4239 \\
    416 & warmth & 0 & 4381 \\
    417 & wishy-washy\_jj & 0 & 4476 \\
    418 & inventive\_jj & 0 & 4496 \\
    419 & instability & 0 & 4618 \\
    420 & touchy\_jj & 0 & 4971 \\
    421 & uncooperative\_jj & 0 & 5052 \\
    422 & uncharitable\_jj & 0 & 5189 \\
    423 & aimless\_jj & 0 & 5340 \\
    424 & underhanded\_jj & 0 & 5532 \\
    425 & generosity & 0 & 5728 \\
    426 & gullibility & 0 & 5916 \\
    427 & zestful\_jj & 0 & 6131 \\
    428 & mannerly\_jj & 0 & 8281 \\
    429 & unemotional\_jj & 0 & 8403 \\
    430 & unintelligent\_jj & 0 & 9396 \\
    \hline
    \caption{Scores and rankings for most extreme 30 words in component \#48} \\
\end{longtable}
\begin{longtable}[!htbp]{| rlr@{.}l |}
    \hline
    \textbf{Rank} & \textbf{Word} & \multicolumn{2}{c|}{\textbf{Score}} \\
    \hline
    \endhead
    1 & surliness & 0 & -9669 \\
    2 & erratic\_jj & 0 & -7376 \\
    3 & friendly\_jj & 0 & -7015 \\
    4 & insight & 0 & -6997 \\
    5 & thorough\_jj & 0 & -6855 \\
    6 & adventurous\_jj & 0 & -6651 \\
    7 & rude\_jj & 0 & -6465 \\
    8 & mannerly\_jj & 0 & -6339 \\
    9 & forgetful\_jj & 0 & -5797 \\
    10 & disorganization & 0 & -5756 \\
    11 & independent\_jj & 0 & -5678 \\
    12 & argumentative\_jj & 0 & -5643 \\
    13 & inconsiderate\_jj & 0 & -5563 \\
    14 & playfulness & 0 & -5305 \\
    15 & expressive\_jj & 0 & -5123 \\
    16 & quarrelsome\_jj & 0 & -5062 \\
    17 & intelligence & 0 & -4974 \\
    18 & uncharitable\_jj & 0 & -4780 \\
    19 & docile\_jj & 0 & -4703 \\
    20 & suspicious\_jj & 0 & -4670 \\
    21 & detached\_jj & 0 & -4628 \\
    22 & impudent\_jj & 0 & -4519 \\
    23 & impolite\_jj & 0 & -4478 \\
    24 & playful\_jj & 0 & -4223 \\
    25 & meddlesome\_jj & 0 & -4090 \\
    26 & slothful\_jj & 0 & -4023 \\
    27 & assertive\_jj & 0 & -3927 \\
    28 & bigoted\_jj & 0 & -3887 \\
    29 & bullheaded\_jj & 0 & -3864 \\
    30 & dominant\_jj & 0 & -3832 \\
    401 & flexible\_jj & 0 & 3971 \\
    402 & unassuming\_jj & 0 & 3976 \\
    403 & pessimism & 0 & 3981 \\
    404 & morose\_jj & 0 & 3993 \\
    405 & vivacious\_jj & 0 & 3998 \\
    406 & generous\_jj & 0 & 4021 \\
    407 & animation & 0 & 4182 \\
    408 & innovative\_jj & 0 & 4199 \\
    409 & cruelty & 0 & 4221 \\
    410 & concise\_jj & 0 & 4388 \\
    411 & timid\_jj & 0 & 4596 \\
    412 & individualistic\_jj & 0 & 4647 \\
    413 & talkative\_jj & 0 & 4751 \\
    414 & excitable\_jj & 0 & 4807 \\
    415 & lethargy & 0 & 4893 \\
    416 & crafty\_jj & 0 & 5029 \\
    417 & gregarious\_jj & 0 & 5044 \\
    418 & charitable\_jj & 0 & 5860 \\
    419 & touchy\_jj & 0 & 5876 \\
    420 & predictability & 0 & 5927 \\
    421 & orderly\_jj & 0 & 6025 \\
    422 & volatility & 0 & 6270 \\
    423 & verbose\_jj & 0 & 6291 \\
    424 & scornful\_jj & 0 & 6300 \\
    425 & silence & 0 & 6835 \\
    426 & efficiency & 0 & 6887 \\
    427 & deceit & 0 & 7502 \\
    428 & inefficient\_jj & 0 & 7914 \\
    429 & inhibition & 0 & 8550 \\
    430 & understanding\_jj & 1 & 233 \\
    \hline
    \caption{Scores and rankings for most extreme 30 words in component \#49} \\
\end{longtable}
\begin{longtable}[!htbp]{| rlr@{.}l |}
    \hline
    \textbf{Rank} & \textbf{Word} & \multicolumn{2}{c|}{\textbf{Score}} \\
    \hline
    \endhead
    1 & intelligence & 0 & -9338 \\
    2 & ethical\_jj & 0 & -6923 \\
    3 & intrusive\_jj & 0 & -6741 \\
    4 & shallowness & 0 & -6548 \\
    5 & aimless\_jj & 0 & -6187 \\
    6 & concise\_jj & 0 & -5936 \\
    7 & indecisive\_jj & 0 & -5806 \\
    8 & frivolity & 0 & -5746 \\
    9 & underhanded\_jj & 0 & -5662 \\
    10 & frivolous\_jj & 0 & -5369 \\
    11 & obstinate\_jj & 0 & -5288 \\
    12 & adventurous\_jj & 0 & -5274 \\
    13 & unfriendly\_jj & 0 & -5104 \\
    14 & intrusiveness & 0 & -4951 \\
    15 & perceptive\_jj & 0 & -4866 \\
    16 & bright\_jj & 0 & -4766 \\
    17 & submissive\_jj & 0 & -4749 \\
    18 & curious\_jj & 0 & -4696 \\
    19 & warm\_jj & 0 & -4628 \\
    20 & recklessness & 0 & -4605 \\
    21 & spontaneity & 0 & -4529 \\
    22 & neat\_jj & 0 & -4472 \\
    23 & nonconforming\_jj & 0 & -4374 \\
    24 & direct\_jj & 0 & -4298 \\
    25 & thrift & 0 & -4279 \\
    26 & frank\_jj & 0 & -4242 \\
    27 & irritable\_jj & 0 & -4062 \\
    28 & smug\_jj & 0 & -4045 \\
    29 & pomposity & 0 & -3960 \\
    30 & uncritical\_jj & 0 & -3844 \\
    401 & volatility & 0 & 3916 \\
    402 & prejudiced\_jj & 0 & 3919 \\
    403 & gullible\_jj & 0 & 3937 \\
    404 & talkative\_jj & 0 & 3964 \\
    405 & flexibility & 0 & 4064 \\
    406 & stinginess & 0 & 4102 \\
    407 & proud\_jj & 0 & 4103 \\
    408 & humorous\_jj & 0 & 4215 \\
    409 & careful\_jj & 0 & 4278 \\
    410 & bitter\_jj & 0 & 4333 \\
    411 & caustic\_jj & 0 & 4343 \\
    412 & insightful\_jj & 0 & 4600 \\
    413 & inventive\_jj & 0 & 4609 \\
    414 & wordy\_jj & 0 & 4815 \\
    415 & animation & 0 & 4910 \\
    416 & intellectuality & 0 & 5049 \\
    417 & crafty\_jj & 0 & 5148 \\
    418 & tactful\_jj & 0 & 5205 \\
    419 & distrustful\_jj & 0 & 5537 \\
    420 & thrifty\_jj & 0 & 5564 \\
    421 & expressiveness & 0 & 5846 \\
    422 & instability & 0 & 5882 \\
    423 & quarrelsome\_jj & 0 & 5990 \\
    424 & negligent\_jj & 0 & 6344 \\
    425 & autonomous\_jj & 0 & 6388 \\
    426 & merry\_jj & 0 & 7363 \\
    427 & leniency & 0 & 7424 \\
    428 & reserve & 0 & 7509 \\
    429 & self-critical\_jj & 0 & 7682 \\
    430 & unemotional\_jj & 0 & 9598 \\
    \hline
    \caption{Scores and rankings for most extreme 30 words in component \#50} \\
\end{longtable}
\begin{longtable}[!htbp]{| rlr@{.}l |}
    \hline
    \textbf{Rank} & \textbf{Word} & \multicolumn{2}{c|}{\textbf{Score}} \\
    \hline
    \endhead
    1 & belligerence & 0 & -9114 \\
    2 & selfish\_jj & 0 & -8181 \\
    3 & unscrupulous\_jj & 0 & -7077 \\
    4 & concise\_jj & 0 & -6895 \\
    5 & irritable\_jj & 0 & -6520 \\
    6 & surliness & 0 & -6186 \\
    7 & miserly\_jj & 0 & -6130 \\
    8 & patient\_jj & 0 & -6000 \\
    9 & harsh\_jj & 0 & -5991 \\
    10 & easygoing\_jj & 0 & -5918 \\
    11 & uncooperative\_jj & 0 & -5755 \\
    12 & adventurous\_jj & 0 & -5687 \\
    13 & thrifty\_jj & 0 & -5685 \\
    14 & happy-go-lucky\_jj & 0 & -5249 \\
    15 & bullheaded\_jj & 0 & -5113 \\
    16 & grumpy\_jj & 0 & -5066 \\
    17 & ethical\_jj & 0 & -4868 \\
    18 & careful\_jj & 0 & -4708 \\
    19 & nonconforming\_jj & 0 & -4671 \\
    20 & bitter\_jj & 0 & -4432 \\
    21 & diplomatic\_jj & 0 & -4400 \\
    22 & passivity & 0 & -4238 \\
    23 & thoughtless\_jj & 0 & -4215 \\
    24 & animation & 0 & -4102 \\
    25 & insight & 0 & -4071 \\
    26 & aloofness & 0 & -3977 \\
    27 & withdrawn\_jj & 0 & -3736 \\
    28 & moral\_jj & 0 & -3655 \\
    29 & wishy-washy\_jj & 0 & -3643 \\
    30 & unambitious\_jj & 0 & -3636 \\
    401 & courtesy & 0 & 3891 \\
    402 & individualistic\_jj & 0 & 4023 \\
    403 & industrious\_jj & 0 & 4146 \\
    404 & careless\_jj & 0 & 4271 \\
    405 & aimless\_jj & 0 & 4347 \\
    406 & impudent\_jj & 0 & 4350 \\
    407 & volatile\_jj & 0 & 4359 \\
    408 & cordial\_jj & 0 & 4411 \\
    409 & reserve & 0 & 4484 \\
    410 & communicative\_jj & 0 & 4637 \\
    411 & curt\_jj & 0 & 4781 \\
    412 & snobbish\_jj & 0 & 4839 \\
    413 & gullible\_jj & 0 & 4964 \\
    414 & conscientious\_jj & 0 & 4971 \\
    415 & prejudice & 0 & 5007 \\
    416 & lenient\_jj & 0 & 5024 \\
    417 & dominant\_jj & 0 & 5025 \\
    418 & explosive\_jj & 0 & 5363 \\
    419 & assured\_jj & 0 & 5369 \\
    420 & thrift & 0 & 5418 \\
    421 & courteous\_jj & 0 & 5434 \\
    422 & argumentative\_jj & 0 & 5601 \\
    423 & cosmopolitan\_jj & 0 & 5745 \\
    424 & enterprising\_jj & 0 & 6066 \\
    425 & leniency & 0 & 6510 \\
    426 & prompt\_jj & 0 & 6513 \\
    427 & perceptive\_jj & 0 & 6912 \\
    428 & neat\_jj & 0 & 7252 \\
    429 & punctual\_jj & 0 & 8359 \\
    430 & unsophisticated\_jj & 1 & 329 \\
    \hline
    \caption{Scores and rankings for most extreme 30 words in component \#51} \\
\end{longtable}
\begin{longtable}[!htbp]{| rlr@{.}l |}
    \hline
    \textbf{Rank} & \textbf{Word} & \multicolumn{2}{c|}{\textbf{Score}} \\
    \hline
    \endhead
    1 & shallowness & 0 & -9535 \\
    2 & modesty & 0 & -8321 \\
    3 & silent\_jj & 0 & -7146 \\
    4 & pomposity & 0 & -6061 \\
    5 & quarrelsome\_jj & 0 & -6049 \\
    6 & reserve & 0 & -5761 \\
    7 & inquisitive\_jj & 0 & -5671 \\
    8 & cultured\_jj & 0 & -5220 \\
    9 & thrift & 0 & -5129 \\
    10 & quiet\_jj & 0 & -5006 \\
    11 & friendly\_jj & 0 & -4877 \\
    12 & cunning & 0 & -4869 \\
    13 & caution & 0 & -4406 \\
    14 & forgetfulness & 0 & -4390 \\
    15 & silence & 0 & -4383 \\
    16 & timid\_jj & 0 & -4377 \\
    17 & lazy\_jj & 0 & -4343 \\
    18 & moral\_jj & 0 & -4304 \\
    19 & caustic\_jj & 0 & -4273 \\
    20 & decisiveness & 0 & -4113 \\
    21 & negligence & 0 & -4071 \\
    22 & pompous\_jj & 0 & -4051 \\
    23 & nonconformity & 0 & -3973 \\
    24 & unintelligent\_jj & 0 & -3957 \\
    25 & secretive\_jj & 0 & -3859 \\
    26 & intellectual\_jj & 0 & -3854 \\
    27 & courage & 0 & -3842 \\
    28 & careless\_jj & 0 & -3823 \\
    29 & conceit & 0 & -3811 \\
    30 & argumentative\_jj & 0 & -3806 \\
    401 & callousness & 0 & 3783 \\
    402 & bitter\_jj & 0 & 3791 \\
    403 & curious\_jj & 0 & 3901 \\
    404 & haphazard\_jj & 0 & 3942 \\
    405 & deep\_jj & 0 & 3995 \\
    406 & insensitive\_jj & 0 & 4121 \\
    407 & systematic\_jj & 0 & 4150 \\
    408 & crabby\_jj & 0 & 4152 \\
    409 & spontaneous\_jj & 0 & 4211 \\
    410 & unreflective\_jj & 0 & 4294 \\
    411 & rebellious\_jj & 0 & 4410 \\
    412 & compassionate\_jj & 0 & 4516 \\
    413 & vindictive\_jj & 0 & 4566 \\
    414 & carefree\_jj & 0 & 4570 \\
    415 & casual\_jj & 0 & 4698 \\
    416 & generosity & 0 & 4913 \\
    417 & forgetful\_jj & 0 & 4923 \\
    418 & independence & 0 & 4929 \\
    419 & sincere\_jj & 0 & 5284 \\
    420 & distrustful\_jj & 0 & 5287 \\
    421 & thorough\_jj & 0 & 5596 \\
    422 & spirited\_jj & 0 & 5687 \\
    423 & scornful\_jj & 0 & 5806 \\
    424 & organized\_jj & 0 & 5912 \\
    425 & depth & 0 & 6365 \\
    426 & withdrawn\_jj & 0 & 6403 \\
    427 & warm\_jj & 0 & 7012 \\
    428 & neat\_jj & 0 & 7568 \\
    429 & sophistication & 0 & 9481 \\
    430 & dependability & 1 & 1548 \\
    \hline
    \caption{Scores and rankings for most extreme 30 words in component \#52} \\
\end{longtable}
\begin{longtable}[!htbp]{| rlr@{.}l |}
    \hline
    \textbf{Rank} & \textbf{Word} & \multicolumn{2}{c|}{\textbf{Score}} \\
    \hline
    \endhead
    1 & tactless\_jj & 0 & -9620 \\
    2 & autonomous\_jj & 0 & -8120 \\
    3 & folksy\_jj & 0 & -7527 \\
    4 & wishy-washy\_jj & 0 & -7035 \\
    5 & thoughtless\_jj & 0 & -6614 \\
    6 & flippant\_jj & 0 & -6385 \\
    7 & slothful\_jj & 0 & -6250 \\
    8 & insight & 0 & -6204 \\
    9 & unambitious\_jj & 0 & -6009 \\
    10 & indecisive\_jj & 0 & -5836 \\
    11 & cultured\_jj & 0 & -5643 \\
    12 & erratic\_jj & 0 & -5444 \\
    13 & understanding\_jj & 0 & -5266 \\
    14 & instability & 0 & -5239 \\
    15 & haphazard\_jj & 0 & -4771 \\
    16 & ambitious\_jj & 0 & -4765 \\
    17 & intelligence & 0 & -4757 \\
    18 & prompt\_jj & 0 & -4723 \\
    19 & earthiness & 0 & -4651 \\
    20 & self-critical\_jj & 0 & -4403 \\
    21 & bright\_jj & 0 & -4294 \\
    22 & unsophisticated\_jj & 0 & -4268 \\
    23 & passivity & 0 & -4255 \\
    24 & refined\_jj & 0 & -4201 \\
    25 & decisiveness & 0 & -3922 \\
    26 & cunning\_jj & 0 & -3766 \\
    27 & intrusiveness & 0 & -3743 \\
    28 & bossiness & 0 & -3678 \\
    29 & systematic\_jj & 0 & -3674 \\
    30 & insecurity & 0 & -3516 \\
    401 & cranky\_jj & 0 & 3505 \\
    402 & temperamental\_jj & 0 & 3521 \\
    403 & caustic\_jj & 0 & 3618 \\
    404 & prejudice & 0 & 3694 \\
    405 & neat\_jj & 0 & 3729 \\
    406 & reasonable\_jj & 0 & 3832 \\
    407 & adventurous\_jj & 0 & 3839 \\
    408 & exacting\_jj & 0 & 3886 \\
    409 & demanding\_jj & 0 & 3952 \\
    410 & artistic\_jj & 0 & 4129 \\
    411 & intellectual\_jj & 0 & 4169 \\
    412 & demonstrative\_jj & 0 & 4182 \\
    413 & lazy\_jj & 0 & 4228 \\
    414 & crabby\_jj & 0 & 4365 \\
    415 & predictable\_jj & 0 & 4481 \\
    416 & shallow\_jj & 0 & 4540 \\
    417 & dependable\_jj & 0 & 4676 \\
    418 & negligent\_jj & 0 & 4766 \\
    419 & punctual\_jj & 0 & 4854 \\
    420 & defensive\_jj & 0 & 5144 \\
    421 & envy & 0 & 5240 \\
    422 & verbose\_jj & 0 & 5373 \\
    423 & unscrupulous\_jj & 0 & 5373 \\
    424 & predictability & 0 & 5750 \\
    425 & reserve & 0 & 6684 \\
    426 & logical\_jj & 0 & 7098 \\
    427 & economical\_jj & 0 & 7816 \\
    428 & argumentative\_jj & 0 & 9665 \\
    429 & belligerence & 1 & 649 \\
    430 & nonconforming\_jj & 1 & 731 \\
    \hline
    \caption{Scores and rankings for most extreme 30 words in component \#53} \\
\end{longtable}
\begin{longtable}[!htbp]{| rlr@{.}l |}
    \hline
    \textbf{Rank} & \textbf{Word} & \multicolumn{2}{c|}{\textbf{Score}} \\
    \hline
    \endhead
    1 & shallowness & -1 & -817 \\
    2 & insight & 0 & -9696 \\
    3 & uncritical\_jj & 0 & -7783 \\
    4 & thorough\_jj & 0 & -6260 \\
    5 & demanding\_jj & 0 & -6093 \\
    6 & unsociable\_jj & 0 & -5873 \\
    7 & joyless\_jj & 0 & -5426 \\
    8 & unrestrained\_jj & 0 & -5406 \\
    9 & naïve\_jj & 0 & -5229 \\
    10 & recklessness & 0 & -5136 \\
    11 & peaceful\_jj & 0 & -5135 \\
    12 & cranky\_jj & 0 & -4892 \\
    13 & conscientious\_jj & 0 & -4864 \\
    14 & spirit & 0 & -4805 \\
    15 & understanding\_jj & 0 & -4610 \\
    16 & merry\_jj & 0 & -4397 \\
    17 & rebellious\_jj & 0 & -4361 \\
    18 & spontaneity & 0 & -4303 \\
    19 & crabby\_jj & 0 & -4282 \\
    20 & negligent\_jj & 0 & -4247 \\
    21 & quiet\_jj & 0 & -4241 \\
    22 & stingy\_jj & 0 & -4182 \\
    23 & demonstrative\_jj & 0 & -4064 \\
    24 & prompt\_jj & 0 & -4050 \\
    25 & verbal\_jj & 0 & -4029 \\
    26 & pleasant\_jj & 0 & -3961 \\
    27 & animation & 0 & -3935 \\
    28 & self-pitying\_jj & 0 & -3866 \\
    29 & diplomatic\_jj & 0 & -3726 \\
    30 & aimless\_jj & 0 & -3640 \\
    401 & curious\_jj & 0 & 3611 \\
    402 & concise\_jj & 0 & 3657 \\
    403 & organization & 0 & 3734 \\
    404 & scornful\_jj & 0 & 3767 \\
    405 & unimaginative\_jj & 0 & 3774 \\
    406 & submissive\_jj & 0 & 3844 \\
    407 & leniency & 0 & 3859 \\
    408 & dependability & 0 & 4016 \\
    409 & opportunistic\_jj & 0 & 4169 \\
    410 & talkative\_jj & 0 & 4258 \\
    411 & thrift & 0 & 4297 \\
    412 & bullheaded\_jj & 0 & 4388 \\
    413 & callousness & 0 & 4388 \\
    414 & folksy\_jj & 0 & 4390 \\
    415 & ethical\_jj & 0 & 4396 \\
    416 & quarrelsome\_jj & 0 & 4473 \\
    417 & temperamental\_jj & 0 & 4579 \\
    418 & playful\_jj & 0 & 4580 \\
    419 & unreflective\_jj & 0 & 4601 \\
    420 & cruelty & 0 & 4622 \\
    421 & detached\_jj & 0 & 5688 \\
    422 & expressive\_jj & 0 & 6001 \\
    423 & disorganization & 0 & 6882 \\
    424 & selfish\_jj & 0 & 7011 \\
    425 & intrusiveness & 0 & 7080 \\
    426 & reserve & 0 & 8113 \\
    427 & neat\_jj & 0 & 8165 \\
    428 & volatility & 0 & 8204 \\
    429 & unscrupulous\_jj & 0 & 8951 \\
    430 & silence & 0 & 9763 \\
    \hline
    \caption{Scores and rankings for most extreme 30 words in component \#54} \\
\end{longtable}
\begin{longtable}[!htbp]{| rlr@{.}l |}
    \hline
    \textbf{Rank} & \textbf{Word} & \multicolumn{2}{c|}{\textbf{Score}} \\
    \hline
    \endhead
    1 & intelligence & 0 & -9212 \\
    2 & economical\_jj & 0 & -8390 \\
    3 & unconventional\_jj & 0 & -7785 \\
    4 & impudent\_jj & 0 & -6838 \\
    5 & nonconformity & 0 & -6721 \\
    6 & aloofness & 0 & -6655 \\
    7 & shyness & 0 & -6348 \\
    8 & negligent\_jj & 0 & -6312 \\
    9 & shallow\_jj & 0 & -5993 \\
    10 & self-esteem & 0 & -5818 \\
    11 & sophistication & 0 & -5775 \\
    12 & sincere\_jj & 0 & -5492 \\
    13 & condescending\_jj & 0 & -5453 \\
    14 & compassionate\_jj & 0 & -5420 \\
    15 & flippant\_jj & 0 & -5265 \\
    16 & irritable\_jj & 0 & -5115 \\
    17 & explosive\_jj & 0 & -5018 \\
    18 & disorganization & 0 & -5003 \\
    19 & secretive\_jj & 0 & -4541 \\
    20 & unsophisticated\_jj & 0 & -4453 \\
    21 & unreflective\_jj & 0 & -4407 \\
    22 & concise\_jj & 0 & -4341 \\
    23 & insecure\_jj & 0 & -4235 \\
    24 & silent\_jj & 0 & -4227 \\
    25 & lazy\_jj & 0 & -4099 \\
    26 & perceptive\_jj & 0 & -4085 \\
    27 & careless\_jj & 0 & -4054 \\
    28 & leniency & 0 & -4050 \\
    29 & egotistical\_jj & 0 & -4006 \\
    30 & imperturbable\_jj & 0 & -4006 \\
    401 & aimless\_jj & 0 & 3561 \\
    402 & energetic\_jj & 0 & 3671 \\
    403 & reliable\_jj & 0 & 3691 \\
    404 & bigoted\_jj & 0 & 3747 \\
    405 & irritability & 0 & 3755 \\
    406 & intrusive\_jj & 0 & 3891 \\
    407 & folksy\_jj & 0 & 3933 \\
    408 & undependable\_jj & 0 & 4126 \\
    409 & underhanded\_jj & 0 & 4176 \\
    410 & frivolous\_jj & 0 & 4238 \\
    411 & dependable\_jj & 0 & 4295 \\
    412 & wishy-washy\_jj & 0 & 4327 \\
    413 & naïve\_jj & 0 & 4388 \\
    414 & punctuality & 0 & 4418 \\
    415 & demanding\_jj & 0 & 4441 \\
    416 & placidity & 0 & 4454 \\
    417 & combative\_jj & 0 & 4527 \\
    418 & earthiness & 0 & 4649 \\
    419 & rudeness & 0 & 4655 \\
    420 & vivacious\_jj & 0 & 4778 \\
    421 & cruelty & 0 & 4940 \\
    422 & surliness & 0 & 5051 \\
    423 & expressiveness & 0 & 5463 \\
    424 & recklessness & 0 & 5480 \\
    425 & quarrelsome\_jj & 0 & 5934 \\
    426 & dependability & 0 & 6015 \\
    427 & ungracious\_jj & 0 & 7485 \\
    428 & thrift & 0 & 7784 \\
    429 & verbose\_jj & 0 & 8007 \\
    430 & punctual\_jj & 0 & 8632 \\
    \hline
    \caption{Scores and rankings for most extreme 30 words in component \#55} \\
\end{longtable}
\begin{longtable}[!htbp]{| rlr@{.}l |}
    \hline
    \textbf{Rank} & \textbf{Word} & \multicolumn{2}{c|}{\textbf{Score}} \\
    \hline
    \endhead
    1 & irritability & 0 & -9224 \\
    2 & economical\_jj & 0 & -8534 \\
    3 & cordial\_jj & 0 & -8165 \\
    4 & punctuality & 0 & -8123 \\
    5 & compassionate\_jj & 0 & -7719 \\
    6 & intelligence & 0 & -7341 \\
    7 & docile\_jj & 0 & -6567 \\
    8 & sophistication & 0 & -6422 \\
    9 & unreliable\_jj & 0 & -6217 \\
    10 & distrustful\_jj & 0 & -5853 \\
    11 & rash\_jj & 0 & -5680 \\
    12 & earthiness & 0 & -5608 \\
    13 & nonconformity & 0 & -5516 \\
    14 & moody\_jj & 0 & -5358 \\
    15 & stingy\_jj & 0 & -5291 \\
    16 & neat\_jj & 0 & -5071 \\
    17 & nonconforming\_jj & 0 & -4905 \\
    18 & artistic\_jj & 0 & -4782 \\
    19 & selfish\_jj & 0 & -4753 \\
    20 & volatility & 0 & -4696 \\
    21 & cooperation & 0 & -4272 \\
    22 & reserve & 0 & -4080 \\
    23 & brave\_jj & 0 & -4033 \\
    24 & crabby\_jj & 0 & -3980 \\
    25 & rebellious\_jj & 0 & -3958 \\
    26 & adventurous\_jj & 0 & -3833 \\
    27 & volatile\_jj & 0 & -3813 \\
    28 & proud\_jj & 0 & -3812 \\
    29 & tempestuous\_jj & 0 & -3686 \\
    30 & jealous\_jj & 0 & -3549 \\
    401 & frivolity & 0 & 3738 \\
    402 & smart\_jj & 0 & 3753 \\
    403 & polite\_jj & 0 & 3756 \\
    404 & submissive\_jj & 0 & 3825 \\
    405 & wishy-washy\_jj & 0 & 3855 \\
    406 & selfishness & 0 & 3872 \\
    407 & conceited\_jj & 0 & 3882 \\
    408 & cultured\_jj & 0 & 4046 \\
    409 & self-pitying\_jj & 0 & 4048 \\
    410 & condescending\_jj & 0 & 4155 \\
    411 & talkative\_jj & 0 & 4202 \\
    412 & courage & 0 & 4510 \\
    413 & refined\_jj & 0 & 4686 \\
    414 & casual\_jj & 0 & 4696 \\
    415 & mannerly\_jj & 0 & 4801 \\
    416 & rudeness & 0 & 5016 \\
    417 & merry\_jj & 0 & 5069 \\
    418 & boastful\_jj & 0 & 5182 \\
    419 & indecisiveness & 0 & 5533 \\
    420 & verbose\_jj & 0 & 5894 \\
    421 & flexible\_jj & 0 & 6128 \\
    422 & aimlessness & 0 & 6174 \\
    423 & insecurity & 0 & 6215 \\
    424 & friendly\_jj & 0 & 6427 \\
    425 & lazy\_jj & 0 & 6548 \\
    426 & aimless\_jj & 0 & 6685 \\
    427 & erratic\_jj & 0 & 6693 \\
    428 & organized\_jj & 0 & 7063 \\
    429 & self-disciplined\_jj & 0 & 8481 \\
    430 & independence & 0 & 8728 \\
    \hline
    \caption{Scores and rankings for most extreme 30 words in component \#56} \\
\end{longtable}
\begin{longtable}[!htbp]{| rlr@{.}l |}
    \hline
    \textbf{Rank} & \textbf{Word} & \multicolumn{2}{c|}{\textbf{Score}} \\
    \hline
    \endhead
    1 & cruelty & -1 & -1313 \\
    2 & intelligence & 0 & -8132 \\
    3 & reserve & 0 & -8063 \\
    4 & punctual\_jj & 0 & -6313 \\
    5 & harsh\_jj & 0 & -5844 \\
    6 & defensive\_jj & 0 & -5695 \\
    7 & cosmopolitan\_jj & 0 & -5526 \\
    8 & surliness & 0 & -5506 \\
    9 & disorganization & 0 & -5368 \\
    10 & intellectual\_jj & 0 & -5262 \\
    11 & docile\_jj & 0 & -5224 \\
    12 & flamboyant\_jj & 0 & -5219 \\
    13 & self-disciplined\_jj & 0 & -5207 \\
    14 & individualistic\_jj & 0 & -5057 \\
    15 & talkative\_jj & 0 & -4962 \\
    16 & amiability & 0 & -4950 \\
    17 & shallow\_jj & 0 & -4943 \\
    18 & self-esteem & 0 & -4921 \\
    19 & naturalness & 0 & -4905 \\
    20 & punctuality & 0 & -4830 \\
    21 & courteous\_jj & 0 & -4737 \\
    22 & erratic\_jj & 0 & -4512 \\
    23 & folksy\_jj & 0 & -4427 \\
    24 & pessimistic\_jj & 0 & -4407 \\
    25 & pessimism & 0 & -4393 \\
    26 & sloth & 0 & -4322 \\
    27 & assertion & 0 & -4237 \\
    28 & persistent\_jj & 0 & -4181 \\
    29 & prompt\_jj & 0 & -4127 \\
    30 & diplomatic\_jj & 0 & -4088 \\
    401 & assured\_jj & 0 & 3427 \\
    402 & gullible\_jj & 0 & 3721 \\
    403 & high-strung\_jj & 0 & 3767 \\
    404 & considerate\_jj & 0 & 3867 \\
    405 & decisive\_jj & 0 & 3976 \\
    406 & friendly\_jj & 0 & 3999 \\
    407 & thorough\_jj & 0 & 4031 \\
    408 & bossy\_jj & 0 & 4073 \\
    409 & neat\_jj & 0 & 4103 \\
    410 & assertive\_jj & 0 & 4255 \\
    411 & sloppy\_jj & 0 & 4259 \\
    412 & leniency & 0 & 4284 \\
    413 & explosive\_jj & 0 & 4299 \\
    414 & passive\_jj & 0 & 4569 \\
    415 & fretful\_jj & 0 & 4597 \\
    416 & courtesy & 0 & 4604 \\
    417 & withdrawn\_jj & 0 & 4891 \\
    418 & moody\_jj & 0 & 4931 \\
    419 & communicative\_jj & 0 & 5081 \\
    420 & suspicious\_jj & 0 & 5298 \\
    421 & nonconformity & 0 & 5335 \\
    422 & surly\_jj & 0 & 5362 \\
    423 & detached\_jj & 0 & 5586 \\
    424 & distrustful\_jj & 0 & 5657 \\
    425 & passivity & 0 & 6043 \\
    426 & bossiness & 0 & 6356 \\
    427 & quarrelsome\_jj & 0 & 6463 \\
    428 & silence & 0 & 6500 \\
    429 & thrift & 0 & 8106 \\
    430 & mannerly\_jj & 0 & 8392 \\
    \hline
    \caption{Scores and rankings for most extreme 30 words in component \#57} \\
\end{longtable}
\begin{longtable}[!htbp]{| rlr@{.}l |}
    \hline
    \textbf{Rank} & \textbf{Word} & \multicolumn{2}{c|}{\textbf{Score}} \\
    \hline
    \endhead
    1 & sophistication & -1 & -404 \\
    2 & extroverted\_jj & 0 & -7253 \\
    3 & mannerly\_jj & 0 & -6170 \\
    4 & intelligence & 0 & -5844 \\
    5 & zestful\_jj & 0 & -5255 \\
    6 & unsophisticated\_jj & 0 & -5207 \\
    7 & individualistic\_jj & 0 & -5087 \\
    8 & reckless\_jj & 0 & -4994 \\
    9 & jovial\_jj & 0 & -4958 \\
    10 & assured\_jj & 0 & -4891 \\
    11 & unrestrained\_jj & 0 & -4842 \\
    12 & compassionate\_jj & 0 & -4632 \\
    13 & defensive\_jj & 0 & -4560 \\
    14 & folksy\_jj & 0 & -4446 \\
    15 & rude\_jj & 0 & -4423 \\
    16 & aimlessness & 0 & -4408 \\
    17 & instability & 0 & -4371 \\
    18 & peaceful\_jj & 0 & -4329 \\
    19 & stupidity & 0 & -4310 \\
    20 & gruff\_jj & 0 & -4305 \\
    21 & erratic\_jj & 0 & -4252 \\
    22 & talkative\_jj & 0 & -4170 \\
    23 & passivity & 0 & -4143 \\
    24 & obliging\_jj & 0 & -4139 \\
    25 & boastful\_jj & 0 & -3846 \\
    26 & quiet & 0 & -3833 \\
    27 & impetuous\_jj & 0 & -3700 \\
    28 & fastidious\_jj & 0 & -3695 \\
    29 & patient\_jj & 0 & -3649 \\
    30 & nonconforming\_jj & 0 & -3648 \\
    401 & dependability & 0 & 3657 \\
    402 & punctual\_jj & 0 & 3664 \\
    403 & manipulative\_jj & 0 & 3765 \\
    404 & devious\_jj & 0 & 3837 \\
    405 & prejudiced\_jj & 0 & 3888 \\
    406 & unkind\_jj & 0 & 3970 \\
    407 & affectionate\_jj & 0 & 4269 \\
    408 & volatile\_jj & 0 & 4297 \\
    409 & joyless\_jj & 0 & 4337 \\
    410 & pomposity & 0 & 4682 \\
    411 & deceit & 0 & 4685 \\
    412 & thrift & 0 & 4710 \\
    413 & systematic\_jj & 0 & 4820 \\
    414 & cruelty & 0 & 4828 \\
    415 & pessimistic\_jj & 0 & 4886 \\
    416 & undemanding\_jj & 0 & 5152 \\
    417 & merry\_jj & 0 & 5348 \\
    418 & adventurous\_jj & 0 & 5406 \\
    419 & sociable\_jj & 0 & 5457 \\
    420 & surly\_jj & 0 & 5643 \\
    421 & explosive\_jj & 0 & 5779 \\
    422 & exacting\_jj & 0 & 5829 \\
    423 & lenient\_jj & 0 & 6046 \\
    424 & cunning\_jj & 0 & 6061 \\
    425 & indecisiveness & 0 & 6324 \\
    426 & self-esteem & 0 & 6815 \\
    427 & rebellious\_jj & 0 & 7127 \\
    428 & irritable\_jj & 0 & 7160 \\
    429 & reserve & 0 & 7439 \\
    430 & aloofness & 1 & 1931 \\
    \hline
    \caption{Scores and rankings for most extreme 30 words in component \#58} \\
\end{longtable}
\begin{longtable}[!htbp]{| rlr@{.}l |}
    \hline
    \textbf{Rank} & \textbf{Word} & \multicolumn{2}{c|}{\textbf{Score}} \\
    \hline
    \endhead
    1 & perceptive\_jj & 0 & -8353 \\
    2 & charitable\_jj & 0 & -6997 \\
    3 & unscrupulous\_jj & 0 & -6816 \\
    4 & caustic\_jj & 0 & -5863 \\
    5 & punctual\_jj & 0 & -5830 \\
    6 & folksy\_jj & 0 & -5710 \\
    7 & curt\_jj & 0 & -5441 \\
    8 & principled\_jj & 0 & -5231 \\
    9 & uncharitable\_jj & 0 & -5092 \\
    10 & candor & 0 & -5002 \\
    11 & impersonal\_jj & 0 & -4986 \\
    12 & punctuality & 0 & -4796 \\
    13 & ungracious\_jj & 0 & -4693 \\
    14 & inhibition & 0 & -4685 \\
    15 & devious\_jj & 0 & -4663 \\
    16 & unemotional\_jj & 0 & -4501 \\
    17 & communicative\_jj & 0 & -4304 \\
    18 & earthiness & 0 & -4095 \\
    19 & morose\_jj & 0 & -4094 \\
    20 & sloth & 0 & -4080 \\
    21 & spirited\_jj & 0 & -4027 \\
    22 & organization & 0 & -4026 \\
    23 & thrifty\_jj & 0 & -3953 \\
    24 & sophistication & 0 & -3863 \\
    25 & docile\_jj & 0 & -3844 \\
    26 & cunning & 0 & -3771 \\
    27 & unambitious\_jj & 0 & -3696 \\
    28 & empathy & 0 & -3677 \\
    29 & autonomous\_jj & 0 & -3642 \\
    30 & nosey\_jj & 0 & -3641 \\
    401 & predictability & 0 & 4022 \\
    402 & erratic\_jj & 0 & 4076 \\
    403 & unreflective\_jj & 0 & 4081 \\
    404 & depth & 0 & 4084 \\
    405 & gregarious\_jj & 0 & 4156 \\
    406 & precise\_jj & 0 & 4231 \\
    407 & contemplative\_jj & 0 & 4240 \\
    408 & mannerly\_jj & 0 & 4602 \\
    409 & impudent\_jj & 0 & 4621 \\
    410 & touchy\_jj & 0 & 4642 \\
    411 & accommodating\_jj & 0 & 4669 \\
    412 & prompt\_jj & 0 & 4794 \\
    413 & flamboyant\_jj & 0 & 4889 \\
    414 & careful\_jj & 0 & 5158 \\
    415 & economical\_jj & 0 & 5163 \\
    416 & nonconformity & 0 & 5243 \\
    417 & proud\_jj & 0 & 5250 \\
    418 & traditional\_jj & 0 & 5282 \\
    419 & gullibility & 0 & 5499 \\
    420 & volatility & 0 & 5592 \\
    421 & merry\_jj & 0 & 5698 \\
    422 & wordy\_jj & 0 & 5716 \\
    423 & argumentative\_jj & 0 & 5726 \\
    424 & vivacious\_jj & 0 & 5850 \\
    425 & modesty & 0 & 5907 \\
    426 & self-critical\_jj & 0 & 5920 \\
    427 & defensive\_jj & 0 & 6172 \\
    428 & negligence & 0 & 7368 \\
    429 & recklessness & 0 & 8036 \\
    430 & expressiveness & 0 & 8553 \\
    \hline
    \caption{Scores and rankings for most extreme 30 words in component \#59} \\
\end{longtable}
\begin{longtable}[!htbp]{| rlr@{.}l |}
    \hline
    \textbf{Rank} & \textbf{Word} & \multicolumn{2}{c|}{\textbf{Score}} \\
    \hline
    \endhead
    1 & inconsiderate\_jj & -1 & -571 \\
    2 & surliness & 0 & -8653 \\
    3 & thrift & 0 & -7834 \\
    4 & cultured\_jj & 0 & -6285 \\
    5 & autonomous\_jj & 0 & -5809 \\
    6 & shallow\_jj & 0 & -5355 \\
    7 & silence & 0 & -5226 \\
    8 & selfish\_jj & 0 & -5207 \\
    9 & animation & 0 & -5023 \\
    10 & inconsistency & 0 & -4983 \\
    11 & sincere\_jj & 0 & -4917 \\
    12 & passionless\_jj & 0 & -4907 \\
    13 & lenient\_jj & 0 & -4867 \\
    14 & cranky\_jj & 0 & -4641 \\
    15 & underhanded\_jj & 0 & -4560 \\
    16 & adaptable\_jj & 0 & -4558 \\
    17 & analytical\_jj & 0 & -4543 \\
    18 & demanding\_jj & 0 & -4499 \\
    19 & creative\_jj & 0 & -4430 \\
    20 & disrespectful\_jj & 0 & -4186 \\
    21 & vindictive\_jj & 0 & -4032 \\
    22 & self-pitying\_jj & 0 & -3996 \\
    23 & belligerence & 0 & -3976 \\
    24 & zestful\_jj & 0 & -3974 \\
    25 & down-to-earth\_jj & 0 & -3868 \\
    26 & careless\_jj & 0 & -3786 \\
    27 & bright\_jj & 0 & -3785 \\
    28 & intrusiveness & 0 & -3702 \\
    29 & reserved\_jj & 0 & -3607 \\
    30 & touchy\_jj & 0 & -3425 \\
    401 & surly\_jj & 0 & 3588 \\
    402 & lazy\_jj & 0 & 3626 \\
    403 & sophistication & 0 & 3666 \\
    404 & perceptive\_jj & 0 & 3697 \\
    405 & expressiveness & 0 & 3727 \\
    406 & uncritical\_jj & 0 & 3769 \\
    407 & volatile\_jj & 0 & 3908 \\
    408 & courage & 0 & 4058 \\
    409 & aimlessness & 0 & 4141 \\
    410 & sloth & 0 & 4154 \\
    411 & reasonable\_jj & 0 & 4157 \\
    412 & humor & 0 & 4160 \\
    413 & forgetful\_jj & 0 & 4191 \\
    414 & punctuality & 0 & 4297 \\
    415 & understanding\_jj & 0 & 4775 \\
    416 & decisiveness & 0 & 5097 \\
    417 & disorganization & 0 & 5153 \\
    418 & ambitious\_jj & 0 & 5232 \\
    419 & quarrelsome\_jj & 0 & 5307 \\
    420 & neat\_jj & 0 & 5361 \\
    421 & timid\_jj & 0 & 5502 \\
    422 & friendly\_jj & 0 & 5516 \\
    423 & self-critical\_jj & 0 & 5789 \\
    424 & volatility & 0 & 6324 \\
    425 & organized\_jj & 0 & 6457 \\
    426 & generosity & 0 & 7386 \\
    427 & argumentative\_jj & 0 & 7389 \\
    428 & wordy\_jj & 0 & 7793 \\
    429 & independence & 0 & 7814 \\
    430 & nonconforming\_jj & 0 & 9786 \\
    \hline
    \caption{Scores and rankings for most extreme 30 words in component \#60} \\
\end{longtable}
\begin{longtable}[!htbp]{| rlr@{.}l |}
    \hline
    \textbf{Rank} & \textbf{Word} & \multicolumn{2}{c|}{\textbf{Score}} \\
    \hline
    \endhead
    1 & efficient\_jj & 0 & -8326 \\
    2 & suspicious\_jj & 0 & -6893 \\
    3 & self-disciplined\_jj & 0 & -6852 \\
    4 & verbal\_jj & 0 & -6752 \\
    5 & lenient\_jj & 0 & -6705 \\
    6 & shallow\_jj & 0 & -5752 \\
    7 & caustic\_jj & 0 & -5655 \\
    8 & warm\_jj & 0 & -5522 \\
    9 & efficiency & 0 & -5441 \\
    10 & passionless\_jj & 0 & -5195 \\
    11 & forgetfulness & 0 & -5088 \\
    12 & vivacious\_jj & 0 & -5082 \\
    13 & slothful\_jj & 0 & -4932 \\
    14 & intellectual\_jj & 0 & -4747 \\
    15 & somber\_jj & 0 & -4474 \\
    16 & rebellious\_jj & 0 & -4447 \\
    17 & high-strung\_jj & 0 & -4427 \\
    18 & self-esteem & 0 & -4305 \\
    19 & spontaneity & 0 & -4260 \\
    20 & nonconforming\_jj & 0 & -4258 \\
    21 & belligerence & 0 & -4163 \\
    22 & charitable\_jj & 0 & -4096 \\
    23 & sophistication & 0 & -4095 \\
    24 & leniency & 0 & -4092 \\
    25 & practical\_jj & 0 & -4057 \\
    26 & generous\_jj & 0 & -4054 \\
    27 & economical\_jj & 0 & -3905 \\
    28 & artistic\_jj & 0 & -3865 \\
    29 & unscrupulous\_jj & 0 & -3859 \\
    30 & thoughtless\_jj & 0 & -3801 \\
    401 & insight & 0 & 3557 \\
    402 & inconsistency & 0 & 3603 \\
    403 & careless\_jj & 0 & 3612 \\
    404 & dependability & 0 & 3705 \\
    405 & impolite\_jj & 0 & 3857 \\
    406 & exacting\_jj & 0 & 3887 \\
    407 & adaptable\_jj & 0 & 3928 \\
    408 & meditative\_jj & 0 & 3930 \\
    409 & cruel\_jj & 0 & 3983 \\
    410 & meticulous\_jj & 0 & 4104 \\
    411 & unassuming\_jj & 0 & 4138 \\
    412 & morose\_jj & 0 & 4162 \\
    413 & fastidious\_jj & 0 & 4188 \\
    414 & aloofness & 0 & 4189 \\
    415 & mannerly\_jj & 0 & 4291 \\
    416 & inquisitive\_jj & 0 & 4304 \\
    417 & inventive\_jj & 0 & 4321 \\
    418 & bullheaded\_jj & 0 & 4638 \\
    419 & compassionate\_jj & 0 & 4673 \\
    420 & gruff\_jj & 0 & 4795 \\
    421 & reserve & 0 & 4825 \\
    422 & folksy\_jj & 0 & 4921 \\
    423 & neat\_jj & 0 & 5607 \\
    424 & animation & 0 & 5914 \\
    425 & antagonistic\_jj & 0 & 5945 \\
    426 & prompt\_jj & 0 & 6727 \\
    427 & steady\_jj & 0 & 7804 \\
    428 & inconsiderate\_jj & 0 & 8338 \\
    429 & understanding\_jj & 0 & 8906 \\
    430 & irritable\_jj & 0 & 9567 \\
    \hline
    \caption{Scores and rankings for most extreme 30 words in component \#61} \\
\end{longtable}
\begin{longtable}[!htbp]{| rlr@{.}l |}
    \hline
    \textbf{Rank} & \textbf{Word} & \multicolumn{2}{c|}{\textbf{Score}} \\
    \hline
    \endhead
    1 & disorganization & 0 & -7110 \\
    2 & understanding\_jj & 0 & -6696 \\
    3 & sloppy\_jj & 0 & -6389 \\
    4 & ethical\_jj & 0 & -6124 \\
    5 & dependability & 0 & -5896 \\
    6 & cultured\_jj & 0 & -5759 \\
    7 & impersonal\_jj & 0 & -5625 \\
    8 & somber\_jj & 0 & -5617 \\
    9 & friendly\_jj & 0 & -5416 \\
    10 & tactless\_jj & 0 & -5233 \\
    11 & warm\_jj & 0 & -4826 \\
    12 & impudent\_jj & 0 & -4713 \\
    13 & unconventional\_jj & 0 & -4674 \\
    14 & concise\_jj & 0 & -4660 \\
    15 & mannerly\_jj & 0 & -4578 \\
    16 & cunning\_jj & 0 & -4540 \\
    17 & forgetful\_jj & 0 & -4521 \\
    18 & unreflective\_jj & 0 & -4512 \\
    19 & earthiness & 0 & -4370 \\
    20 & recklessness & 0 & -4354 \\
    21 & reserve & 0 & -4303 \\
    22 & unfriendly\_jj & 0 & -4272 \\
    23 & analytical\_jj & 0 & -4219 \\
    24 & merry\_jj & 0 & -4162 \\
    25 & self-pitying\_jj & 0 & -4113 \\
    26 & deceit & 0 & -4076 \\
    27 & opportunistic\_jj & 0 & -4065 \\
    28 & verbose\_jj & 0 & -4035 \\
    29 & moody\_jj & 0 & -4011 \\
    30 & joyless\_jj & 0 & -3939 \\
    401 & steady\_jj & 0 & 3843 \\
    402 & brave\_jj & 0 & 3909 \\
    403 & charitable\_jj & 0 & 3957 \\
    404 & happy-go-lucky\_jj & 0 & 3991 \\
    405 & pomposity & 0 & 4080 \\
    406 & shallow\_jj & 0 & 4137 \\
    407 & explosive\_jj & 0 & 4239 \\
    408 & inefficient\_jj & 0 & 4309 \\
    409 & flexible\_jj & 0 & 4391 \\
    410 & neat\_jj & 0 & 4449 \\
    411 & frivolous\_jj & 0 & 4526 \\
    412 & caustic\_jj & 0 & 4539 \\
    413 & self-critical\_jj & 0 & 4544 \\
    414 & refined\_jj & 0 & 4605 \\
    415 & generous\_jj & 0 & 4674 \\
    416 & rudeness & 0 & 4777 \\
    417 & wordy\_jj & 0 & 4783 \\
    418 & surliness & 0 & 4798 \\
    419 & surly\_jj & 0 & 4934 \\
    420 & insight & 0 & 4973 \\
    421 & timid\_jj & 0 & 5076 \\
    422 & expressiveness & 0 & 5451 \\
    423 & belligerence & 0 & 5471 \\
    424 & fretful\_jj & 0 & 6192 \\
    425 & modest\_jj & 0 & 6366 \\
    426 & courteous\_jj & 0 & 6415 \\
    427 & expressive\_jj & 0 & 6438 \\
    428 & conscientious\_jj & 0 & 6962 \\
    429 & zestful\_jj & 0 & 7776 \\
    430 & detached\_jj & 0 & 9342 \\
    \hline
    \caption{Scores and rankings for most extreme 30 words in component \#62} \\
\end{longtable}
\begin{longtable}[!htbp]{| rlr@{.}l |}
    \hline
    \textbf{Rank} & \textbf{Word} & \multicolumn{2}{c|}{\textbf{Score}} \\
    \hline
    \endhead
    1 & reserve & 0 & -9619 \\
    2 & self-critical\_jj & 0 & -9542 \\
    3 & thrift & 0 & -7701 \\
    4 & inhibition & 0 & -6394 \\
    5 & naturalness & 0 & -5731 \\
    6 & conscientious\_jj & 0 & -5707 \\
    7 & cold\_jj & 0 & -5354 \\
    8 & rebellious\_jj & 0 & -5096 \\
    9 & defensive\_jj & 0 & -5062 \\
    10 & harsh\_jj & 0 & -4969 \\
    11 & distrustful\_jj & 0 & -4765 \\
    12 & sincere\_jj & 0 & -4692 \\
    13 & prompt\_jj & 0 & -4552 \\
    14 & volatile\_jj & 0 & -4297 \\
    15 & recklessness & 0 & -4207 \\
    16 & obliging\_jj & 0 & -4162 \\
    17 & negligence & 0 & -4080 \\
    18 & flippant\_jj & 0 & -4015 \\
    19 & punctuality & 0 & -3987 \\
    20 & intelligence & 0 & -3983 \\
    21 & anxious\_jj & 0 & -3980 \\
    22 & inventive\_jj & 0 & -3976 \\
    23 & devious\_jj & 0 & -3897 \\
    24 & thrifty\_jj & 0 & -3866 \\
    25 & crafty\_jj & 0 & -3865 \\
    26 & forgetful\_jj & 0 & -3730 \\
    27 & industrious\_jj & 0 & -3716 \\
    28 & placidity & 0 & -3703 \\
    29 & sophistication & 0 & -3688 \\
    30 & undemanding\_jj & 0 & -3544 \\
    401 & worldly\_jj & 0 & 3563 \\
    402 & communicative\_jj & 0 & 3712 \\
    403 & rambunctious\_jj & 0 & 3746 \\
    404 & independence & 0 & 3757 \\
    405 & courteous\_jj & 0 & 3770 \\
    406 & caution & 0 & 3801 \\
    407 & high-strung\_jj & 0 & 3827 \\
    408 & cruelty & 0 & 3854 \\
    409 & somber\_jj & 0 & 3929 \\
    410 & gullibility & 0 & 4002 \\
    411 & envy & 0 & 4078 \\
    412 & unsophisticated\_jj & 0 & 4100 \\
    413 & disorganization & 0 & 4163 \\
    414 & vigorous\_jj & 0 & 4195 \\
    415 & compassionate\_jj & 0 & 4265 \\
    416 & insensitive\_jj & 0 & 4328 \\
    417 & impractical\_jj & 0 & 4460 \\
    418 & inquisitive\_jj & 0 & 4552 \\
    419 & understanding\_jj & 0 & 4769 \\
    420 & crabby\_jj & 0 & 4918 \\
    421 & insight & 0 & 5664 \\
    422 & leniency & 0 & 5774 \\
    423 & shallowness & 0 & 5990 \\
    424 & unintelligent\_jj & 0 & 6115 \\
    425 & indecisive\_jj & 0 & 6189 \\
    426 & self-disciplined\_jj & 0 & 6199 \\
    427 & bullheaded\_jj & 0 & 6227 \\
    428 & steady\_jj & 0 & 6251 \\
    429 & touchy\_jj & 0 & 6291 \\
    430 & bright\_jj & 0 & 9845 \\
    \hline
    \caption{Scores and rankings for most extreme 30 words in component \#63} \\
\end{longtable}
\begin{longtable}[!htbp]{| rlr@{.}l |}
    \hline
    \textbf{Rank} & \textbf{Word} & \multicolumn{2}{c|}{\textbf{Score}} \\
    \hline
    \endhead
    1 & compassionate\_jj & 0 & -9003 \\
    2 & thrift & 0 & -8913 \\
    3 & disorganization & 0 & -8588 \\
    4 & shallowness & 0 & -7873 \\
    5 & excitable\_jj & 0 & -6696 \\
    6 & crabby\_jj & 0 & -6293 \\
    7 & adventurous\_jj & 0 & -5799 \\
    8 & suspicious\_jj & 0 & -5659 \\
    9 & conceited\_jj & 0 & -5597 \\
    10 & friendly\_jj & 0 & -5566 \\
    11 & economical\_jj & 0 & -5487 \\
    12 & zestful\_jj & 0 & -5357 \\
    13 & warm\_jj & 0 & -5275 \\
    14 & perceptive\_jj & 0 & -5062 \\
    15 & assertive\_jj & 0 & -4977 \\
    16 & self-critical\_jj & 0 & -4810 \\
    17 & reserved\_jj & 0 & -4787 \\
    18 & pomposity & 0 & -4626 \\
    19 & wordy\_jj & 0 & -4514 \\
    20 & analytical\_jj & 0 & -4333 \\
    21 & rash\_jj & 0 & -4280 \\
    22 & uncritical\_jj & 0 & -4182 \\
    23 & vain\_jj & 0 & -4160 \\
    24 & rambunctious\_jj & 0 & -4148 \\
    25 & rude\_jj & 0 & -4118 \\
    26 & amiability & 0 & -4032 \\
    27 & underhanded\_jj & 0 & -3836 \\
    28 & rebellious\_jj & 0 & -3675 \\
    29 & negligent\_jj & 0 & -3599 \\
    30 & conceit & 0 & -3595 \\
    401 & concise\_jj & 0 & 3452 \\
    402 & inefficient\_jj & 0 & 3492 \\
    403 & impolite\_jj & 0 & 3492 \\
    404 & folksy\_jj & 0 & 3525 \\
    405 & intrusive\_jj & 0 & 3539 \\
    406 & scornful\_jj & 0 & 3548 \\
    407 & merry\_jj & 0 & 3552 \\
    408 & verbose\_jj & 0 & 3633 \\
    409 & efficiency & 0 & 3669 \\
    410 & naturalness & 0 & 3708 \\
    411 & decisive\_jj & 0 & 3870 \\
    412 & shy\_jj & 0 & 3934 \\
    413 & inconsiderate\_jj & 0 & 4102 \\
    414 & forgetful\_jj & 0 & 4128 \\
    415 & nonconformity & 0 & 4159 \\
    416 & gregarious\_jj & 0 & 4193 \\
    417 & aloofness & 0 & 4338 \\
    418 & formal\_jj & 0 & 4338 \\
    419 & unreliable\_jj & 0 & 4584 \\
    420 & irritability & 0 & 4704 \\
    421 & punctual\_jj & 0 & 4900 \\
    422 & modesty & 0 & 5039 \\
    423 & explosive\_jj & 0 & 5151 \\
    424 & traditional\_jj & 0 & 5850 \\
    425 & careless\_jj & 0 & 6230 \\
    426 & worldly\_jj & 0 & 6235 \\
    427 & inhibition & 0 & 6732 \\
    428 & unreflective\_jj & 0 & 7665 \\
    429 & indecisive\_jj & 0 & 8624 \\
    430 & insight & 0 & 9169 \\
    \hline
    \caption{Scores and rankings for most extreme 30 words in component \#64} \\
\end{longtable}
\begin{longtable}[!htbp]{| rlr@{.}l |}
    \hline
    \textbf{Rank} & \textbf{Word} & \multicolumn{2}{c|}{\textbf{Score}} \\
    \hline
    \endhead
    1 & detached\_jj & -1 & -1506 \\
    2 & independence & -1 & -204 \\
    3 & verbose\_jj & 0 & -7469 \\
    4 & rebellious\_jj & 0 & -6482 \\
    5 & friendly\_jj & 0 & -6261 \\
    6 & cruelty & 0 & -6194 \\
    7 & economical\_jj & 0 & -5929 \\
    8 & scornful\_jj & 0 & -5829 \\
    9 & surliness & 0 & -5718 \\
    10 & cosmopolitan\_jj & 0 & -5673 \\
    11 & autonomous\_jj & 0 & -5657 \\
    12 & truthful\_jj & 0 & -5565 \\
    13 & sophistication & 0 & -5481 \\
    14 & zestful\_jj & 0 & -5366 \\
    15 & exacting\_jj & 0 & -4953 \\
    16 & inquisitive\_jj & 0 & -4756 \\
    17 & skeptical\_jj & 0 & -4680 \\
    18 & deceit & 0 & -4517 \\
    19 & moody\_jj & 0 & -4472 \\
    20 & meticulous\_jj & 0 & -4386 \\
    21 & deceitful\_jj & 0 & -4296 \\
    22 & inventive\_jj & 0 & -4240 \\
    23 & ethical\_jj & 0 & -4152 \\
    24 & boastful\_jj & 0 & -4032 \\
    25 & tempestuous\_jj & 0 & -3981 \\
    26 & impetuous\_jj & 0 & -3942 \\
    27 & unassuming\_jj & 0 & -3863 \\
    28 & playfulness & 0 & -3792 \\
    29 & forgetfulness & 0 & -3751 \\
    30 & fastidious\_jj & 0 & -3673 \\
    401 & bossiness & 0 & 3389 \\
    402 & direct\_jj & 0 & 3428 \\
    403 & nonconformity & 0 & 3482 \\
    404 & indecisiveness & 0 & 3628 \\
    405 & rash\_jj & 0 & 3629 \\
    406 & generous\_jj & 0 & 3640 \\
    407 & proud\_jj & 0 & 3648 \\
    408 & uncooperative\_jj & 0 & 3714 \\
    409 & casual\_jj & 0 & 3743 \\
    410 & humor & 0 & 3768 \\
    411 & impractical\_jj & 0 & 3808 \\
    412 & intrusive\_jj & 0 & 3814 \\
    413 & intelligence & 0 & 3835 \\
    414 & candor & 0 & 3938 \\
    415 & unrestrained\_jj & 0 & 4044 \\
    416 & obstinate\_jj & 0 & 4119 \\
    417 & predictability & 0 & 4267 \\
    418 & folksy\_jj & 0 & 4311 \\
    419 & courage & 0 & 4459 \\
    420 & mannerly\_jj & 0 & 4655 \\
    421 & wishy-washy\_jj & 0 & 4662 \\
    422 & worldly\_jj & 0 & 5048 \\
    423 & verbal\_jj & 0 & 5080 \\
    424 & vain\_jj & 0 & 5113 \\
    425 & traditional\_jj & 0 & 5170 \\
    426 & unscrupulous\_jj & 0 & 5277 \\
    427 & perceptive\_jj & 0 & 5874 \\
    428 & dependability & 0 & 5882 \\
    429 & frivolous\_jj & 0 & 5941 \\
    430 & animation & 0 & 7292 \\
    \hline
    \caption{Scores and rankings for most extreme 30 words in component \#65} \\
\end{longtable}
\begin{longtable}[!htbp]{| rlr@{.}l |}
    \hline
    \textbf{Rank} & \textbf{Word} & \multicolumn{2}{c|}{\textbf{Score}} \\
    \hline
    \endhead
    1 & volatility & 0 & -8330 \\
    2 & thorough\_jj & 0 & -6929 \\
    3 & spirit & 0 & -6116 \\
    4 & detached\_jj & 0 & -5493 \\
    5 & naturalness & 0 & -5483 \\
    6 & pessimism & 0 & -5472 \\
    7 & inconsiderate\_jj & 0 & -5300 \\
    8 & unintelligent\_jj & 0 & -5112 \\
    9 & volatile\_jj & 0 & -5076 \\
    10 & candor & 0 & -5051 \\
    11 & adventurous\_jj & 0 & -5038 \\
    12 & quiet & 0 & -4967 \\
    13 & vivacious\_jj & 0 & -4836 \\
    14 & extroverted\_jj & 0 & -4832 \\
    15 & aimlessness & 0 & -4680 \\
    16 & uncooperative\_jj & 0 & -4638 \\
    17 & depth & 0 & -4534 \\
    18 & tenacious\_jj & 0 & -4525 \\
    19 & cunning & 0 & -4376 \\
    20 & courtesy & 0 & -4241 \\
    21 & lethargy & 0 & -4197 \\
    22 & expressiveness & 0 & -4173 \\
    23 & shallow\_jj & 0 & -4139 \\
    24 & docile\_jj & 0 & -4127 \\
    25 & self-critical\_jj & 0 & -4089 \\
    26 & rudeness & 0 & -4064 \\
    27 & shy\_jj & 0 & -3971 \\
    28 & reasonable\_jj & 0 & -3844 \\
    29 & frivolous\_jj & 0 & -3757 \\
    30 & unambitious\_jj & 0 & -3682 \\
    401 & negligent\_jj & 0 & 3492 \\
    402 & nervous\_jj & 0 & 3496 \\
    403 & temperamental\_jj & 0 & 3502 \\
    404 & unscrupulous\_jj & 0 & 3582 \\
    405 & forgetfulness & 0 & 3585 \\
    406 & amiable\_jj & 0 & 3748 \\
    407 & passive\_jj & 0 & 3777 \\
    408 & emotional\_jj & 0 & 3812 \\
    409 & unassuming\_jj & 0 & 3910 \\
    410 & prompt\_jj & 0 & 3938 \\
    411 & surliness & 0 & 3939 \\
    412 & gullible\_jj & 0 & 4112 \\
    413 & industrious\_jj & 0 & 4189 \\
    414 & wordy\_jj & 0 & 4355 \\
    415 & inefficient\_jj & 0 & 4447 \\
    416 & naïve\_jj & 0 & 4508 \\
    417 & friendly\_jj & 0 & 4535 \\
    418 & reckless\_jj & 0 & 4627 \\
    419 & rash\_jj & 0 & 4832 \\
    420 & conscientious\_jj & 0 & 4925 \\
    421 & merry\_jj & 0 & 5106 \\
    422 & obliging\_jj & 0 & 5336 \\
    423 & verbal\_jj & 0 & 5473 \\
    424 & self-esteem & 0 & 5669 \\
    425 & neat\_jj & 0 & 5809 \\
    426 & moral\_jj & 0 & 5870 \\
    427 & sluggish\_jj & 0 & 5872 \\
    428 & unstable\_jj & 0 & 6244 \\
    429 & unreflective\_jj & 1 & 248 \\
    430 & pomposity & 1 & 444 \\
    \hline
    \caption{Scores and rankings for most extreme 30 words in component \#66} \\
\end{longtable}
\begin{longtable}[!htbp]{| rlr@{.}l |}
    \hline
    \textbf{Rank} & \textbf{Word} & \multicolumn{2}{c|}{\textbf{Score}} \\
    \hline
    \endhead
    1 & careless\_jj & 0 & -6559 \\
    2 & indecisiveness & 0 & -6526 \\
    3 & intelligence & 0 & -6433 \\
    4 & morose\_jj & 0 & -6281 \\
    5 & somber\_jj & 0 & -6252 \\
    6 & caustic\_jj & 0 & -5901 \\
    7 & submissive\_jj & 0 & -5900 \\
    8 & nonconforming\_jj & 0 & -5538 \\
    9 & modesty & 0 & -5305 \\
    10 & unimaginative\_jj & 0 & -5269 \\
    11 & crabby\_jj & 0 & -5215 \\
    12 & organization & 0 & -5177 \\
    13 & surly\_jj & 0 & -4946 \\
    14 & thrifty\_jj & 0 & -4569 \\
    15 & rudeness & 0 & -4445 \\
    16 & boastful\_jj & 0 & -4440 \\
    17 & charitable\_jj & 0 & -4378 \\
    18 & adventurous\_jj & 0 & -4374 \\
    19 & systematic\_jj & 0 & -4252 \\
    20 & melancholic\_jj & 0 & -4208 \\
    21 & impolite\_jj & 0 & -4145 \\
    22 & enterprising\_jj & 0 & -4075 \\
    23 & humorous\_jj & 0 & -4070 \\
    24 & instability & 0 & -4043 \\
    25 & forgetful\_jj & 0 & -4002 \\
    26 & truthful\_jj & 0 & -3930 \\
    27 & daring & 0 & -3927 \\
    28 & withdrawn\_jj & 0 & -3916 \\
    29 & intellectuality & 0 & -3883 \\
    30 & insecurity & 0 & -3836 \\
    401 & vigorous\_jj & 0 & 3379 \\
    402 & aloofness & 0 & 3529 \\
    403 & mannerly\_jj & 0 & 3654 \\
    404 & polite\_jj & 0 & 3674 \\
    405 & greedy\_jj & 0 & 3684 \\
    406 & aimless\_jj & 0 & 3745 \\
    407 & conscientious\_jj & 0 & 3764 \\
    408 & punctual\_jj & 0 & 3850 \\
    409 & decisiveness & 0 & 3858 \\
    410 & obstinate\_jj & 0 & 3877 \\
    411 & joyless\_jj & 0 & 4105 \\
    412 & meddlesome\_jj & 0 & 4186 \\
    413 & frivolous\_jj & 0 & 4429 \\
    414 & curt\_jj & 0 & 4444 \\
    415 & rebellious\_jj & 0 & 4515 \\
    416 & vindictive\_jj & 0 & 4670 \\
    417 & frivolity & 0 & 5016 \\
    418 & insight & 0 & 5149 \\
    419 & selfish\_jj & 0 & 5235 \\
    420 & verbal\_jj & 0 & 5323 \\
    421 & jealous\_jj & 0 & 5478 \\
    422 & wordy\_jj & 0 & 5533 \\
    423 & impractical\_jj & 0 & 5675 \\
    424 & reserve & 0 & 5852 \\
    425 & sophistication & 0 & 6093 \\
    426 & argumentative\_jj & 0 & 6418 \\
    427 & unreflective\_jj & 0 & 7348 \\
    428 & aimlessness & 0 & 7998 \\
    429 & negligent\_jj & 0 & 8323 \\
    430 & compassionate\_jj & 0 & 9875 \\
    \hline
    \caption{Scores and rankings for most extreme 30 words in component \#67} \\
\end{longtable}
\begin{longtable}[!htbp]{| rlr@{.}l |}
    \hline
    \textbf{Rank} & \textbf{Word} & \multicolumn{2}{c|}{\textbf{Score}} \\
    \hline
    \endhead
    1 & self-disciplined\_jj & 0 & -9114 \\
    2 & self-critical\_jj & 0 & -8176 \\
    3 & unreflective\_jj & 0 & -6918 \\
    4 & shallowness & 0 & -6353 \\
    5 & bigoted\_jj & 0 & -5508 \\
    6 & proud\_jj & 0 & -5448 \\
    7 & friendly\_jj & 0 & -5428 \\
    8 & uncooperative\_jj & 0 & -5179 \\
    9 & rambunctious\_jj & 0 & -4941 \\
    10 & spirit & 0 & -4875 \\
    11 & sophistication & 0 & -4779 \\
    12 & meditative\_jj & 0 & -4585 \\
    13 & verbose\_jj & 0 & -4357 \\
    14 & cosmopolitan\_jj & 0 & -4347 \\
    15 & wordy\_jj & 0 & -4332 \\
    16 & gregarious\_jj & 0 & -4309 \\
    17 & curious\_jj & 0 & -4155 \\
    18 & peaceful\_jj & 0 & -4036 \\
    19 & aimlessness & 0 & -4015 \\
    20 & intrusiveness & 0 & -3944 \\
    21 & earthiness & 0 & -3892 \\
    22 & helpful\_jj & 0 & -3860 \\
    23 & cruel\_jj & 0 & -3808 \\
    24 & meddlesome\_jj & 0 & -3786 \\
    25 & stinginess & 0 & -3771 \\
    26 & unconventional\_jj & 0 & -3763 \\
    27 & reliable\_jj & 0 & -3756 \\
    28 & persistence & 0 & -3726 \\
    29 & instability & 0 & -3670 \\
    30 & inconsiderate\_jj & 0 & -3624 \\
    401 & modest\_jj & 0 & 3569 \\
    402 & conceit & 0 & 3678 \\
    403 & decisiveness & 0 & 3684 \\
    404 & verbal\_jj & 0 & 3822 \\
    405 & negligence & 0 & 3835 \\
    406 & frivolity & 0 & 3851 \\
    407 & cold\_jj & 0 & 3914 \\
    408 & ungracious\_jj & 0 & 3927 \\
    409 & warm\_jj & 0 & 3986 \\
    410 & disorganization & 0 & 4023 \\
    411 & detached\_jj & 0 & 4038 \\
    412 & self-pitying\_jj & 0 & 4041 \\
    413 & unsympathetic\_jj & 0 & 4132 \\
    414 & animation & 0 & 4196 \\
    415 & unemotional\_jj & 0 & 4227 \\
    416 & curt\_jj & 0 & 4398 \\
    417 & defensive\_jj & 0 & 4454 \\
    418 & callousness & 0 & 4788 \\
    419 & pessimism & 0 & 4878 \\
    420 & expressiveness & 0 & 5668 \\
    421 & understanding\_jj & 0 & 5803 \\
    422 & stingy\_jj & 0 & 5836 \\
    423 & rude\_jj & 0 & 5927 \\
    424 & modesty & 0 & 6019 \\
    425 & autonomous\_jj & 0 & 6385 \\
    426 & naturalness & 0 & 6807 \\
    427 & organized\_jj & 0 & 7145 \\
    428 & demonstrative\_jj & 0 & 7456 \\
    429 & surliness & 0 & 8021 \\
    430 & mannerly\_jj & 1 & 963 \\
    \hline
    \caption{Scores and rankings for most extreme 30 words in component \#68} \\
\end{longtable}
\begin{longtable}[!htbp]{| rlr@{.}l |}
    \hline
    \textbf{Rank} & \textbf{Word} & \multicolumn{2}{c|}{\textbf{Score}} \\
    \hline
    \endhead
    1 & economical\_jj & 0 & -8142 \\
    2 & inhibition & 0 & -8048 \\
    3 & perceptive\_jj & 0 & -8031 \\
    4 & slothful\_jj & 0 & -7974 \\
    5 & erratic\_jj & 0 & -6909 \\
    6 & suspicious\_jj & 0 & -6460 \\
    7 & ungracious\_jj & 0 & -6213 \\
    8 & independence & 0 & -6089 \\
    9 & easygoing\_jj & 0 & -5564 \\
    10 & instability & 0 & -5507 \\
    11 & respectful\_jj & 0 & -5284 \\
    12 & nonconforming\_jj & 0 & -5214 \\
    13 & folksy\_jj & 0 & -5130 \\
    14 & conventionality & 0 & -5092 \\
    15 & unscrupulous\_jj & 0 & -5048 \\
    16 & understanding\_jj & 0 & -4965 \\
    17 & orderly\_jj & 0 & -4826 \\
    18 & cunning & 0 & -4757 \\
    19 & unreflective\_jj & 0 & -4685 \\
    20 & sloppy\_jj & 0 & -4678 \\
    21 & shallow\_jj & 0 & -4310 \\
    22 & lenient\_jj & 0 & -4245 \\
    23 & gregariousness & 0 & -4179 \\
    24 & spontaneous\_jj & 0 & -4044 \\
    25 & depth & 0 & -3955 \\
    26 & intellectual\_jj & 0 & -3936 \\
    27 & conceited\_jj & 0 & -3921 \\
    28 & demonstrative\_jj & 0 & -3804 \\
    29 & thorough\_jj & 0 & -3803 \\
    30 & friendly\_jj & 0 & -3793 \\
    401 & modest\_jj & 0 & 3140 \\
    402 & belligerence & 0 & 3175 \\
    403 & unreliable\_jj & 0 & 3194 \\
    404 & conventional\_jj & 0 & 3264 \\
    405 & happy-go-lucky\_jj & 0 & 3355 \\
    406 & earthiness & 0 & 3459 \\
    407 & insecure\_jj & 0 & 3519 \\
    408 & autonomous\_jj & 0 & 3541 \\
    409 & fastidious\_jj & 0 & 3803 \\
    410 & conceit & 0 & 3811 \\
    411 & dependability & 0 & 3817 \\
    412 & worldly\_jj & 0 & 3898 \\
    413 & patient\_jj & 0 & 3907 \\
    414 & self-esteem & 0 & 4078 \\
    415 & nosey\_jj & 0 & 4136 \\
    416 & obliging\_jj & 0 & 4185 \\
    417 & irritable\_jj & 0 & 4275 \\
    418 & pleasant\_jj & 0 & 4423 \\
    419 & deceitful\_jj & 0 & 4479 \\
    420 & sophisticated\_jj & 0 & 4511 \\
    421 & cunning\_jj & 0 & 4718 \\
    422 & reserved\_jj & 0 & 4745 \\
    423 & caustic\_jj & 0 & 5077 \\
    424 & mannerly\_jj & 0 & 5159 \\
    425 & manipulative\_jj & 0 & 5274 \\
    426 & forgetfulness & 0 & 5395 \\
    427 & organized\_jj & 0 & 5411 \\
    428 & self-critical\_jj & 0 & 6006 \\
    429 & cordial\_jj & 0 & 6878 \\
    430 & conscientious\_jj & 0 & 7160 \\
    \hline
    \caption{Scores and rankings for most extreme 30 words in component \#69} \\
\end{longtable}
\begin{longtable}[!htbp]{| rlr@{.}l |}
    \hline
    \textbf{Rank} & \textbf{Word} & \multicolumn{2}{c|}{\textbf{Score}} \\
    \hline
    \endhead
    1 & rude\_jj & 0 & -9553 \\
    2 & stinginess & 0 & -7768 \\
    3 & surliness & 0 & -7585 \\
    4 & tactful\_jj & 0 & -6043 \\
    5 & natural\_jj & 0 & -5868 \\
    6 & indecisiveness & 0 & -5668 \\
    7 & earthiness & 0 & -5619 \\
    8 & thorough\_jj & 0 & -5400 \\
    9 & recklessness & 0 & -5306 \\
    10 & leniency & 0 & -5089 \\
    11 & sincere\_jj & 0 & -4946 \\
    12 & understanding\_jj & 0 & -4937 \\
    13 & indecisive\_jj & 0 & -4854 \\
    14 & extroverted\_jj & 0 & -4557 \\
    15 & high-strung\_jj & 0 & -4553 \\
    16 & inquisitive\_jj & 0 & -4533 \\
    17 & scornful\_jj & 0 & -4467 \\
    18 & punctuality & 0 & -4390 \\
    19 & lenient\_jj & 0 & -4309 \\
    20 & folksy\_jj & 0 & -4292 \\
    21 & impolite\_jj & 0 & -4270 \\
    22 & envious\_jj & 0 & -4069 \\
    23 & generosity & 0 & -3982 \\
    24 & sloth & 0 & -3970 \\
    25 & deliberate\_jj & 0 & -3900 \\
    26 & defensive\_jj & 0 & -3806 \\
    27 & worldly\_jj & 0 & -3766 \\
    28 & vain\_jj & 0 & -3674 \\
    29 & self-esteem & 0 & -3530 \\
    30 & orderly\_jj & 0 & -3514 \\
    401 & easygoing\_jj & 0 & 3466 \\
    402 & courtesy & 0 & 3468 \\
    403 & joyless\_jj & 0 & 3623 \\
    404 & cooperation & 0 & 3693 \\
    405 & down-to-earth\_jj & 0 & 3694 \\
    406 & nervous\_jj & 0 & 3709 \\
    407 & naturalness & 0 & 3731 \\
    408 & docile\_jj & 0 & 3781 \\
    409 & caustic\_jj & 0 & 3824 \\
    410 & emotionality & 0 & 3838 \\
    411 & selfless\_jj & 0 & 3881 \\
    412 & efficiency & 0 & 3984 \\
    413 & thrift & 0 & 4038 \\
    414 & expressiveness & 0 & 4047 \\
    415 & economical\_jj & 0 & 4148 \\
    416 & thoughtless\_jj & 0 & 4240 \\
    417 & callousness & 0 & 4297 \\
    418 & wordy\_jj & 0 & 4328 \\
    419 & modesty & 0 & 4406 \\
    420 & restrained\_jj & 0 & 4584 \\
    421 & detached\_jj & 0 & 4661 \\
    422 & perceptive\_jj & 0 & 4744 \\
    423 & uncharitable\_jj & 0 & 5037 \\
    424 & sophistication & 0 & 5107 \\
    425 & cruelty & 0 & 5176 \\
    426 & somber\_jj & 0 & 5240 \\
    427 & inconsiderate\_jj & 0 & 5996 \\
    428 & careless\_jj & 0 & 6244 \\
    429 & inhibition & 0 & 6531 \\
    430 & disorganization & 0 & 9831 \\
    \hline
    \caption{Scores and rankings for most extreme 30 words in component \#70} \\
\end{longtable}

\subsection{Normalized PCA}
\label{app:rankedwordlists:438and101words:normalized}
\begin{table}[tbp]
    \begin{tabular}{| rlr@{.}l | rlr@{.}l |}
    \hline
    \textbf{Rank} & \textbf{Word} & \multicolumn{2}{c|}{\textbf{Score}} & \textbf{Rank} & \textbf{Word} & \multicolumn{2}{c|}{\textbf{Score}} \\
    \hline
    1 & sociable\_jj & -17 & 1613    &    430 & negligence & 11 & 358 \\
    2 & vivacious\_jj & -15 & 4833    &    429 & instability & 10 & 5123 \\
    3 & considerate\_jj & -15 & 1377    &    428 & volatility & 10 & 4257 \\
    4 & easygoing\_jj & -14 & 3729    &    427 & cooperation & 10 & 1822 \\
    5 & witty\_jj & -12 & 1624    &    426 & intelligence & 9 & 6017 \\
    6 & talkative\_jj & -11 & 8973    &    425 & autonomous\_jj & 9 & 2545 \\
    7 & affectionate\_jj & -11 & 8415    &    424 & caution & 9 & 1946 \\
    8 & gregarious\_jj & -11 & 5638    &    423 & flexibility & 9 & 1753 \\
    9 & courteous\_jj & -11 & 1458    &    422 & sluggish\_jj & 9 & 1560 \\
    10 & down-to-earth\_jj & -11 & 1402    &    421 & reserve & 8 & 9164 \\
    11 & jovial\_jj & -10 & 2155    &    420 & explosive\_jj & 8 & 7703 \\
    12 & extroverted\_jj & -10 & 4    &    419 & volatile\_jj & 8 & 6939 \\
    13 & cultured\_jj & -9 & 6777    &    418 & reasonable\_jj & 8 & 5220 \\
    14 & introspective\_jj & -9 & 2876    &    417 & systematic\_jj & 8 & 4964 \\
    15 & inquisitive\_jj & -9 & 2141    &    416 & efficiency & 8 & 4162 \\
    16 & genial\_jj & -8 & 6657    &    415 & consistent\_jj & 8 & 1601 \\
    17 & high-strung\_jj & -8 & 4199    &    414 & negligent\_jj & 8 & 211 \\
    18 & amiable\_jj & -8 & 4169    &    413 & organized\_jj & 7 & 9904 \\
    19 & happy-go-lucky\_jj & -8 & 3458    &    412 & diplomatic\_jj & 7 & 9533 \\
    20 & mischievous\_jj & -8 & 3407    &    411 & optimism & 7 & 9249 \\
    21 & folksy\_jj & -8 & 2166    &    410 & assertion & 7 & 8374 \\
    22 & perceptive\_jj & -8 & 802    &    409 & organization & 7 & 8366 \\
    23 & humorous\_jj & -7 & 8014    &    408 & direct\_jj & 7 & 7916 \\
    24 & expressive\_jj & -7 & 7564    &    407 & decisive\_jj & 7 & 7831 \\
    25 & surly\_jj & -7 & 7240    &    406 & suspicious\_jj & 7 & 6996 \\
    26 & self-pitying\_jj & -7 & 6608    &    405 & prompt\_jj & 7 & 6805 \\
    27 & cheerful\_jj & -7 & 5674    &    404 & independence & 7 & 6521 \\
    28 & impetuous\_jj & -7 & 4934    &    403 & responsible\_jj & 7 & 5634 \\
    29 & kind\_jj & -7 & 4427    &    402 & charitable\_jj & 7 & 5233 \\
    30 & gruff\_jj & -7 & 3585    &    401 & understanding & 7 & 4039 \\
    \hline
    \end{tabular}
    \caption{Scores and rankings for most extreme 30 words in component \#1} 
\end{table}
\clearpage
\begin{table}[tbp]
    \begin{tabular}{| rlr@{.}l | rlr@{.}l |}
    \hline
    \textbf{Rank} & \textbf{Word} & \multicolumn{2}{c|}{\textbf{Score}} & \textbf{Rank} & \textbf{Word} & \multicolumn{2}{c|}{\textbf{Score}} \\
    \hline
    1 & callousness & -14 & 5355    &    430 & sociable\_jj & 15 & 7771 \\
    2 & selfishness & -12 & 7445    &    429 & considerate\_jj & 12 & 9264 \\
    3 & recklessness & -11 & 7556    &    428 & easygoing\_jj & 10 & 4169 \\
    4 & stupidity & -11 & 6951    &    427 & efficient\_jj & 10 & 1380 \\
    5 & gullibility & -11 & 6803    &    426 & friendly\_jj & 10 & 1034 \\
    6 & rudeness & -11 & 3170    &    425 & reliable\_jj & 10 & 580 \\
    7 & deceit & -11 & 586    &    424 & dependable\_jj & 9 & 7572 \\
    8 & belligerence & -10 & 8199    &    423 & concise\_jj & 9 & 5130 \\
    9 & thoughtless\_jj & -10 & 5322    &    422 & vivacious\_jj & 9 & 4622 \\
    10 & shallowness & -10 & 4444    &    421 & warm\_jj & 9 & 2307 \\
    11 & lethargy & -10 & 3020    &    420 & adventurous\_jj & 9 & 2176 \\
    12 & passivity & -10 & 1815    &    419 & courteous\_jj & 9 & 1602 \\
    13 & irritability & -10 & 318    &    418 & flexible\_jj & 9 & 1298 \\
    14 & bigoted\_jj & -9 & 8330    &    417 & intelligent\_jj & 9 & 1041 \\
    15 & deceitful\_jj & -9 & 7722    &    416 & optimistic\_jj & 8 & 8454 \\
    16 & stubbornness & -9 & 4340    &    415 & enthusiastic\_jj & 8 & 7242 \\
    17 & indecisiveness & -9 & 3560    &    414 & cultured\_jj & 8 & 6880 \\
    18 & pomposity & -9 & 2595    &    413 & gregarious\_jj & 8 & 4580 \\
    19 & self-pitying\_jj & -9 & 1424    &    412 & kind\_jj & 8 & 4360 \\
    20 & inconsiderate\_jj & -8 & 5614    &    411 & confident\_jj & 8 & 1581 \\
    21 & disorganization & -8 & 5495    &    410 & cordial\_jj & 8 & 86 \\
    22 & unreflective\_jj & -8 & 4043    &    409 & energetic\_jj & 7 & 7475 \\
    23 & vindictive\_jj & -8 & 2883    &    408 & generous\_jj & 7 & 6945 \\
    24 & abusive\_jj & -8 & 1403    &    407 & pleasant\_jj & 7 & 6004 \\
    25 & selfish\_jj & -8 & 1271    &    406 & economical\_jj & 7 & 3185 \\
    26 & aloofness & -8 & 74    &    405 & quiet\_jj & 7 & 3055 \\
    27 & ungracious\_jj & -7 & 9145    &    404 & down-to-earth\_jj & 7 & 1378 \\
    28 & sloth & -7 & 7208    &    403 & affectionate\_jj & 7 & 584 \\
    29 & insensitive\_jj & -7 & 6950    &    402 & thorough\_jj & 7 & 70 \\
    30 & forgetfulness & -7 & 6227    &    401 & cautious\_jj & 6 & 9835 \\
    \hline
    \end{tabular}
    \caption{Scores and rankings for most extreme 30 words in component \#2} 
\end{table}
\clearpage
\begin{table}[tbp]
    \begin{tabular}{| rlr@{.}l | rlr@{.}l |}
    \hline
    \textbf{Rank} & \textbf{Word} & \multicolumn{2}{c|}{\textbf{Score}} & \textbf{Rank} & \textbf{Word} & \multicolumn{2}{c|}{\textbf{Score}} \\
    \hline
    1 & abusive\_jj & -13 & 5654    &    430 & playfulness & 12 & 7866 \\
    2 & uncooperative\_jj & -12 & 3510    &    429 & spontaneity & 11 & 8384 \\
    3 & insensitive\_jj & -9 & 7018    &    428 & expressiveness & 11 & 2601 \\
    4 & lenient\_jj & -9 & 5512    &    427 & naturalness & 11 & 1958 \\
    5 & disrespectful\_jj & -9 & 4975    &    426 & earthiness & 10 & 8151 \\
    6 & unscrupulous\_jj & -9 & 4711    &    425 & sophistication & 10 & 7080 \\
    7 & dishonest\_jj & -9 & 4037    &    424 & warmth & 10 & 3225 \\
    8 & inconsiderate\_jj & -9 & 3947    &    423 & candor & 10 & 160 \\
    9 & selfish\_jj & -9 & 3064    &    422 & lethargy & 9 & 6683 \\
    10 & ignorant\_jj & -9 & 2907    &    421 & creativity & 9 & 3767 \\
    11 & pessimistic\_jj & -8 & 9860    &    420 & humor & 9 & 1895 \\
    12 & unfriendly\_jj & -8 & 7117    &    419 & meditative\_jj & 9 & 1217 \\
    13 & bigoted\_jj & -8 & 6336    &    418 & decisiveness & 8 & 8714 \\
    14 & gullible\_jj & -8 & 3374    &    417 & precision & 8 & 8353 \\
    15 & unsympathetic\_jj & -8 & 2715    &    416 & irritability & 8 & 7991 \\
    16 & lazy\_jj & -8 & 878    &    415 & aloofness & 8 & 7614 \\
    17 & inefficient\_jj & -7 & 9970    &    414 & persistence & 8 & 7045 \\
    18 & greedy\_jj & -7 & 9563    &    413 & spirit & 8 & 2739 \\
    19 & intrusive\_jj & -7 & 7127    &    412 & depth & 7 & 7403 \\
    20 & vindictive\_jj & -7 & 6480    &    411 & cunning & 7 & 6491 \\
    21 & naïve\_jj & -7 & 4663    &    410 & artistic\_jj & 7 & 5735 \\
    22 & impolite\_jj & -7 & 4053    &    409 & generosity & 7 & 4449 \\
    23 & absent-minded\_jj & -7 & 2373    &    408 & modesty & 7 & 2877 \\
    24 & rude\_jj & -7 & 1750    &    407 & dependability & 7 & 2298 \\
    25 & negligent\_jj & -7 & 1652    &    406 & inhibition & 7 & 1542 \\
    26 & unreliable\_jj & -7 & 28    &    405 & imperturbable\_jj & 7 & 968 \\
    27 & deceitful\_jj & -6 & 8734    &    404 & courage & 7 & 457 \\
    28 & prejudiced\_jj & -6 & 8269    &    403 & empathy & 7 & 165 \\
    29 & distrustful\_jj & -6 & 7572    &    402 & melancholic\_jj & 6 & 9767 \\
    30 & optimistic\_jj & -6 & 1851    &    401 & shyness & 6 & 8676 \\
    \hline
    \end{tabular}
    \caption{Scores and rankings for most extreme 30 words in component \#3} 
\end{table}
\clearpage
\begin{table}[tbp]
    \begin{tabular}{| rlr@{.}l | rlr@{.}l |}
    \hline
    \textbf{Rank} & \textbf{Word} & \multicolumn{2}{c|}{\textbf{Score}} & \textbf{Rank} & \textbf{Word} & \multicolumn{2}{c|}{\textbf{Score}} \\
    \hline
    1 & sincere\_jj & -11 & 8576    &    430 & absent-minded\_jj & 18 & 14 \\
    2 & considerate\_jj & -11 & 2168    &    429 & irritable\_jj & 13 & 9176 \\
    3 & dignity & -10 & 1889    &    428 & irritability & 13 & 6244 \\
    4 & courage & -9 & 6272    &    427 & lethargy & 10 & 4878 \\
    5 & selfless\_jj & -9 & 6008    &    426 & sluggish\_jj & 8 & 8454 \\
    6 & courageous\_jj & -8 & 2696    &    425 & lethargic\_jj & 8 & 6119 \\
    7 & honest\_jj & -7 & 9983    &    424 & erratic\_jj & 7 & 1742 \\
    8 & stupidity & -7 & 8935    &    423 & fretful\_jj & 6 & 8370 \\
    9 & compassionate\_jj & -7 & 8601    &    422 & forgetfulness & 6 & 4514 \\
    10 & principled\_jj & -7 & 8243    &    421 & cold\_jj & 6 & 1796 \\
    11 & selfish\_jj & -7 & 4394    &    420 & surly\_jj & 6 & 1217 \\
    12 & negligence & -7 & 3909    &    419 & nervous\_jj & 5 & 8901 \\
    13 & moral\_jj & -7 & 1544    &    418 & moody\_jj & 5 & 8318 \\
    14 & generosity & -7 & 411    &    417 & volatile\_jj & 5 & 6710 \\
    15 & recklessness & -6 & 8252    &    416 & volatility & 5 & 6576 \\
    16 & candor & -6 & 7387    &    415 & morose\_jj & 5 & 3507 \\
    17 & selfishness & -6 & 6885    &    414 & dominant\_jj & 5 & 1938 \\
    18 & dishonest\_jj & -6 & 6743    &    413 & extroverted\_jj & 5 & 1679 \\
    19 & negligent\_jj & -6 & 5776    &    412 & meditative\_jj & 5 & 1439 \\
    20 & courteous\_jj & -6 & 5426    &    411 & forgetful\_jj & 5 & 843 \\
    21 & empathy & -6 & 4901    &    410 & anxious\_jj & 5 & 404 \\
    22 & deliberate\_jj & -6 & 2926    &    409 & cranky\_jj & 4 & 9999 \\
    23 & sociable\_jj & -6 & 2919    &    408 & placidity & 4 & 9579 \\
    24 & truthful\_jj & -6 & 2286    &    407 & quarrelsome\_jj & 4 & 9115 \\
    25 & systematic\_jj & -5 & 9391    &    406 & aimless\_jj & 4 & 9089 \\
    26 & modesty & -5 & 8543    &    405 & high-strung\_jj & 4 & 8267 \\
    27 & cruelty & -5 & 7621    &    404 & tempestuous\_jj & 4 & 7255 \\
    28 & kind\_jj & -5 & 6363    &    403 & uncooperative\_jj & 4 & 6852 \\
    29 & ethical\_jj & -5 & 6091    &    402 & grumpy\_jj & 4 & 5697 \\
    30 & morality & -5 & 5741    &    401 & insecure\_jj & 4 & 4483 \\
    \hline
    \end{tabular}
    \caption{Scores and rankings for most extreme 30 words in component \#4} 
\end{table}
\clearpage
\begin{table}[tbp]
    \begin{tabular}{| rlr@{.}l | rlr@{.}l |}
    \hline
    \textbf{Rank} & \textbf{Word} & \multicolumn{2}{c|}{\textbf{Score}} & \textbf{Rank} & \textbf{Word} & \multicolumn{2}{c|}{\textbf{Score}} \\
    \hline
    1 & irritability & -10 & 5343    &    430 & absent-minded\_jj & 20 & 4679 \\
    2 & optimism & -9 & 6835    &    429 & refined\_jj & 11 & 698 \\
    3 & sociable\_jj & -8 & 6509    &    428 & economical\_jj & 10 & 9302 \\
    4 & cordial\_jj & -8 & 3924    &    427 & concise\_jj & 10 & 4710 \\
    5 & lethargy & -8 & 338    &    426 & efficient\_jj & 8 & 8897 \\
    6 & nervous\_jj & -7 & 6554    &    425 & innovative\_jj & 8 & 4795 \\
    7 & sincere\_jj & -7 & 2783    &    424 & inventive\_jj & 8 & 3284 \\
    8 & irritable\_jj & -7 & 1382    &    423 & underhanded\_jj & 7 & 6807 \\
    9 & silence & -7 & 793    &    422 & adaptable\_jj & 7 & 2359 \\
    10 & pessimistic\_jj & -6 & 8264    &    421 & sophisticated\_jj & 7 & 1427 \\
    11 & kind\_jj & -6 & 7786    &    420 & analytical\_jj & 6 & 9271 \\
    12 & optimistic\_jj & -6 & 6942    &    419 & imaginative\_jj & 6 & 8964 \\
    13 & distrust & -6 & 6551    &    418 & cunning\_jj & 6 & 6330 \\
    14 & pessimism & -6 & 5530    &    417 & devious\_jj & 6 & 4777 \\
    15 & gregarious\_jj & -6 & 714    &    416 & exacting\_jj & 6 & 3446 \\
    16 & anxious\_jj & -5 & 9699    &    415 & expressive\_jj & 5 & 8383 \\
    17 & warm\_jj & -5 & 9172    &    414 & complex\_jj & 5 & 7636 \\
    18 & instability & -5 & 8465    &    413 & unimaginative\_jj & 5 & 6550 \\
    19 & jovial\_jj & -5 & 8087    &    412 & insightful\_jj & 5 & 3485 \\
    20 & cautious\_jj & -5 & 7266    &    411 & expressiveness & 5 & 2639 \\
    21 & considerate\_jj & -5 & 6282    &    410 & inefficient\_jj & 5 & 2264 \\
    22 & insecurity & -5 & 5942    &    409 & manipulative\_jj & 5 & 1208 \\
    23 & quiet\_jj & -5 & 5155    &    408 & meticulous\_jj & 4 & 9511 \\
    24 & bitter\_jj & -5 & 5026    &    407 & wordy\_jj & 4 & 9280 \\
    25 & fearful\_jj & -5 & 4689    &    406 & perceptive\_jj & 4 & 9066 \\
    26 & talkative\_jj & -5 & 3597    &    405 & unintelligent\_jj & 4 & 7430 \\
    27 & polite\_jj & -5 & 2398    &    404 & unconventional\_jj & 4 & 7407 \\
    28 & fear & -5 & 1586    &    403 & precision & 4 & 7262 \\
    29 & self-esteem & -5 & 894    &    402 & cunning & 4 & 5315 \\
    30 & cooperation & -4 & 9830    &    401 & creative\_jj & 4 & 4918 \\
    \hline
    \end{tabular}
    \caption{Scores and rankings for most extreme 30 words in component \#5} 
\end{table}
\clearpage
\begin{table}[tbp]
    \begin{tabular}{| rlr@{.}l | rlr@{.}l |}
    \hline
    \textbf{Rank} & \textbf{Word} & \multicolumn{2}{c|}{\textbf{Score}} & \textbf{Rank} & \textbf{Word} & \multicolumn{2}{c|}{\textbf{Score}} \\
    \hline
    1 & irritability & -22 & 9939    &    430 & imperturbable\_jj & 9 & 8095 \\
    2 & lethargy & -13 & 7557    &    429 & folksy\_jj & 8 & 1013 \\
    3 & irritable\_jj & -13 & 5068    &    428 & reserve & 5 & 7288 \\
    4 & economical\_jj & -12 & 3289    &    427 & bitter\_jj & 5 & 6255 \\
    5 & forgetfulness & -11 & 2333    &    426 & flamboyant\_jj & 5 & 4877 \\
    6 & considerate\_jj & -9 & 7069    &    425 & sly\_jj & 5 & 4765 \\
    7 & abusive\_jj & -9 & 3837    &    424 & curt\_jj & 5 & 2208 \\
    8 & self-esteem & -8 & 6799    &    423 & gruff\_jj & 5 & 2201 \\
    9 & sociable\_jj & -8 & 4748    &    422 & homespun\_jj & 5 & 1066 \\
    10 & adaptable\_jj & -7 & 8270    &    421 & vain\_jj & 4 & 9826 \\
    11 & inhibition & -7 & 2363    &    420 & somber\_jj & 4 & 9084 \\
    12 & communicative\_jj & -7 & 293    &    419 & spirited\_jj & 4 & 8503 \\
    13 & extroverted\_jj & -6 & 6658    &    418 & caustic\_jj & 4 & 7628 \\
    14 & compassionate\_jj & -6 & 2052    &    417 & genial\_jj & 4 & 7516 \\
    15 & analytical\_jj & -6 & 676    &    416 & zestful\_jj & 4 & 6995 \\
    16 & efficient\_jj & -6 & 339    &    415 & neat\_jj & 4 & 6378 \\
    17 & intelligent\_jj & -5 & 6736    &    414 & courtesy & 4 & 6179 \\
    18 & empathy & -5 & 6391    &    413 & rambunctious\_jj & 4 & 5723 \\
    19 & disorganization & -5 & 4838    &    412 & tempestuous\_jj & 4 & 5101 \\
    20 & dependability & -5 & 241    &    411 & merry\_jj & 4 & 4965 \\
    21 & suggestible\_jj & -4 & 9085    &    410 & bold\_jj & 4 & 4609 \\
    22 & absent-minded\_jj & -4 & 7432    &    409 & defensive\_jj & 4 & 4489 \\
    23 & individualistic\_jj & -4 & 7353    &    408 & quiet\_jj & 4 & 4235 \\
    24 & kind\_jj & -4 & 5891    &    407 & bullheaded\_jj & 4 & 3257 \\
    25 & impersonal\_jj & -4 & 5452    &    406 & skeptical\_jj & 4 & 2439 \\
    26 & insecurity & -4 & 5057    &    405 & impudent\_jj & 4 & 2267 \\
    27 & refined\_jj & -4 & 5056    &    404 & crafty\_jj & 4 & 2258 \\
    28 & selfishness & -4 & 4718    &    403 & brave\_jj & 4 & 1171 \\
    29 & talkative\_jj & -4 & 4365    &    402 & assertion & 4 & 1124 \\
    30 & rudeness & -4 & 3264    &    401 & scornful\_jj & 4 & 1084 \\
    \hline
    \end{tabular}
    \caption{Scores and rankings for most extreme 30 words in component \#6} 
\end{table}
\clearpage
\begin{table}[tbp]
    \begin{tabular}{| rlr@{.}l | rlr@{.}l |}
    \hline
    \textbf{Rank} & \textbf{Word} & \multicolumn{2}{c|}{\textbf{Score}} & \textbf{Rank} & \textbf{Word} & \multicolumn{2}{c|}{\textbf{Score}} \\
    \hline
    1 & absent-minded\_jj & -44 & 7348    &    430 & abusive\_jj & 6 & 8228 \\
    2 & cordial\_jj & -8 & 4444    &    429 & autonomous\_jj & 6 & 2009 \\
    3 & prompt\_jj & -6 & 9199    &    428 & unstable\_jj & 5 & 5619 \\
    4 & candor & -6 & 8290    &    427 & adventurous\_jj & 5 & 2441 \\
    5 & belligerence & -6 & 7079    &    426 & nonconforming\_jj & 5 & 1214 \\
    6 & respectful\_jj & -6 & 264    &    425 & unimaginative\_jj & 4 & 5247 \\
    7 & frank\_jj & -6 & 1    &    424 & sociable\_jj & 4 & 3483 \\
    8 & leniency & -5 & 9930    &    423 & inefficient\_jj & 4 & 2415 \\
    9 & curt\_jj & -5 & 7292    &    422 & devious\_jj & 4 & 1498 \\
    10 & courage & -5 & 5486    &    421 & conventional\_jj & 4 & 146 \\
    11 & concise\_jj & -5 & 4540    &    420 & greedy\_jj & 3 & 9250 \\
    12 & decisiveness & -5 & 3566    &    419 & manipulative\_jj & 3 & 8381 \\
    13 & principled\_jj & -5 & 328    &    418 & shallow\_jj & 3 & 8265 \\
    14 & tactful\_jj & -4 & 8864    &    417 & lazy\_jj & 3 & 8115 \\
    15 & careful\_jj & -4 & 7700    &    416 & cunning\_jj & 3 & 7447 \\
    16 & truthful\_jj & -4 & 6790    &    415 & sophisticated\_jj & 3 & 7435 \\
    17 & forceful\_jj & -4 & 5947    &    414 & rebellious\_jj & 3 & 7388 \\
    18 & sincere\_jj & -4 & 5395    &    413 & refined\_jj & 3 & 7362 \\
    19 & stubbornness & -4 & 2127    &    412 & happy-go-lucky\_jj & 3 & 7214 \\
    20 & pessimism & -4 & 1240    &    411 & egocentric\_jj & 3 & 7214 \\
    21 & imperturbable\_jj & -4 & 681    &    410 & carefree\_jj & 3 & 6918 \\
    22 & polite\_jj & -4 & 530    &    409 & cultured\_jj & 3 & 6866 \\
    23 & compassionate\_jj & -4 & 76    &    408 & gullible\_jj & 3 & 6517 \\
    24 & unemotional\_jj & -4 & 14    &    407 & neat\_jj & 3 & 6396 \\
    25 & touchy\_jj & -3 & 9464    &    406 & artistic\_jj & 3 & 5430 \\
    26 & courteous\_jj & -3 & 9312    &    405 & instability & 3 & 4956 \\
    27 & cautious\_jj & -3 & 8239    &    404 & aimlessness & 3 & 4813 \\
    28 & punctual\_jj & -3 & 7770    &    403 & unscrupulous\_jj & 3 & 4577 \\
    29 & courageous\_jj & -3 & 7344    &    402 & cranky\_jj & 3 & 4177 \\
    30 & combative\_jj & -3 & 5960    &    401 & natural\_jj & 3 & 3615 \\
    \hline
    \end{tabular}
    \caption{Scores and rankings for most extreme 30 words in component \#7} 
\end{table}
\clearpage
\begin{table}[tbp]
    \begin{tabular}{| rlr@{.}l | rlr@{.}l |}
    \hline
    \textbf{Rank} & \textbf{Word} & \multicolumn{2}{c|}{\textbf{Score}} & \textbf{Rank} & \textbf{Word} & \multicolumn{2}{c|}{\textbf{Score}} \\
    \hline
    1 & uncooperative\_jj & -10 & 9391    &    430 & absent-minded\_jj & 30 & 1267 \\
    2 & cordial\_jj & -10 & 2951    &    429 & sociable\_jj & 8 & 104 \\
    3 & surly\_jj & -7 & 8708    &    428 & charitable\_jj & 7 & 7975 \\
    4 & concise\_jj & -7 & 8698    &    427 & unscrupulous\_jj & 7 & 7352 \\
    5 & antagonistic\_jj & -7 & 6271    &    426 & kind\_jj & 7 & 1500 \\
    6 & economical\_jj & -7 & 261    &    425 & vivacious\_jj & 6 & 9036 \\
    7 & belligerence & -6 & 9918    &    424 & proud\_jj & 6 & 6242 \\
    8 & lenient\_jj & -6 & 5493    &    423 & lazy\_jj & 6 & 3302 \\
    9 & combative\_jj & -6 & 5008    &    422 & inconsiderate\_jj & 6 & 1793 \\
    10 & respectful\_jj & -6 & 3089    &    421 & greedy\_jj & 5 & 7293 \\
    11 & forceful\_jj & -6 & 1486    &    420 & gullible\_jj & 5 & 6196 \\
    12 & assertive\_jj & -6 & 1040    &    419 & intelligent\_jj & 5 & 3895 \\
    13 & caustic\_jj & -6 & 414    &    418 & negligence & 4 & 7915 \\
    14 & unemotional\_jj & -5 & 9846    &    417 & shy\_jj & 4 & 7014 \\
    15 & verbose\_jj & -5 & 8102    &    416 & cranky\_jj & 4 & 6793 \\
    16 & restrained\_jj & -5 & 6238    &    415 & brave\_jj & 4 & 6520 \\
    17 & somber\_jj & -5 & 2446    &    414 & courtesy & 4 & 5666 \\
    18 & frank\_jj & -4 & 9528    &    413 & fear & 4 & 4779 \\
    19 & inconsistent\_jj & -4 & 9496    &    412 & forgetful\_jj & 4 & 4739 \\
    20 & dignified\_jj & -4 & 7780    &    411 & considerate\_jj & 4 & 4423 \\
    21 & scornful\_jj & -4 & 6537    &    410 & ignorant\_jj & 4 & 4204 \\
    22 & candor & -4 & 5783    &    409 & organization & 4 & 3809 \\
    23 & predictable\_jj & -4 & 4373    &    408 & jealous\_jj & 4 & 2679 \\
    24 & accommodating\_jj & -4 & 4201    &    407 & adventurous\_jj & 4 & 2427 \\
    25 & refined\_jj & -4 & 4093    &    406 & envious\_jj & 4 & 1751 \\
    26 & intrusive\_jj & -4 & 3263    &    405 & generosity & 4 & 1599 \\
    27 & truthful\_jj & -4 & 2859    &    404 & spirit & 4 & 1356 \\
    28 & insensitive\_jj & -4 & 2299    &    403 & cruelty & 4 & 239 \\
    29 & self-critical\_jj & -4 & 1256    &    402 & creative\_jj & 4 & 203 \\
    30 & folksy\_jj & -4 & 1150    &    401 & gregarious\_jj & 4 & 187 \\
    \hline
    \end{tabular}
    \caption{Scores and rankings for most extreme 30 words in component \#8} 
\end{table}
\clearpage
\begin{table}[tbp]
    \begin{tabular}{| rlr@{.}l | rlr@{.}l |}
    \hline
    \textbf{Rank} & \textbf{Word} & \multicolumn{2}{c|}{\textbf{Score}} & \textbf{Rank} & \textbf{Word} & \multicolumn{2}{c|}{\textbf{Score}} \\
    \hline
    1 & distrustful\_jj & -12 & 1777    &    430 & irritability & 11 & 5403 \\
    2 & individualistic\_jj & -9 & 4769    &    429 & negligent\_jj & 10 & 3508 \\
    3 & aloofness & -8 & 2521    &    428 & negligence & 9 & 5026 \\
    4 & assertive\_jj & -8 & 390    &    427 & careless\_jj & 9 & 1045 \\
    5 & autonomous\_jj & -7 & 9504    &    426 & deliberate\_jj & 8 & 4961 \\
    6 & belligerence & -7 & 8626    &    425 & concise\_jj & 8 & 2906 \\
    7 & distrust & -7 & 6464    &    424 & prompt\_jj & 7 & 9541 \\
    8 & antagonistic\_jj & -7 & 3143    &    423 & forgetfulness & 7 & 2044 \\
    9 & accommodating\_jj & -6 & 3523    &    422 & sloppy\_jj & 6 & 5511 \\
    10 & dependability & -6 & 958    &    421 & folksy\_jj & 6 & 4942 \\
    11 & independence & -6 & 425    &    420 & neat\_jj & 6 & 4019 \\
    12 & uncritical\_jj & -5 & 9943    &    419 & imperturbable\_jj & 6 & 813 \\
    13 & obstinate\_jj & -5 & 3525    &    418 & warm\_jj & 5 & 7520 \\
    14 & conscientious\_jj & -4 & 9854    &    417 & lethargy & 5 & 6900 \\
    15 & cosmopolitan\_jj & -4 & 9092    &    416 & simple\_jj & 5 & 6856 \\
    16 & prejudiced\_jj & -4 & 9053    &    415 & reckless\_jj & 5 & 5995 \\
    17 & insecure\_jj & -4 & 8452    &    414 & caustic\_jj & 5 & 5651 \\
    18 & pessimistic\_jj & -4 & 8112    &    413 & playful\_jj & 5 & 3908 \\
    19 & independent\_jj & -4 & 7273    &    412 & thorough\_jj & 5 & 3238 \\
    20 & cooperation & -4 & 5904    &    411 & verbal\_jj & 5 & 2775 \\
    21 & generosity & -4 & 5213    &    410 & lazy\_jj & 5 & 1467 \\
    22 & adaptable\_jj & -4 & 4632    &    409 & straightforward\_jj & 5 & 940 \\
    23 & worldly\_jj & -4 & 4251    &    408 & humorous\_jj & 5 & 141 \\
    24 & unfriendly\_jj & -4 & 3850    &    407 & rash\_jj & 5 & 57 \\
    25 & ambition & -4 & 3343    &    406 & cruel\_jj & 4 & 9855 \\
    26 & dominant\_jj & -4 & 3273    &    405 & spontaneous\_jj & 4 & 9463 \\
    27 & stubbornness & -4 & 1400    &    404 & cruelty & 4 & 8901 \\
    28 & insecurity & -4 & 1390    &    403 & meditative\_jj & 4 & 7012 \\
    29 & active\_jj & -4 & 1301    &    402 & systematic\_jj & 4 & 6811 \\
    30 & benevolent\_jj & -3 & 9460    &    401 & abusive\_jj & 4 & 6804 \\
    \hline
    \end{tabular}
    \caption{Scores and rankings for most extreme 30 words in component \#9} 
\end{table}
\clearpage
\begin{table}[tbp]
    \begin{tabular}{| rlr@{.}l | rlr@{.}l |}
    \hline
    \textbf{Rank} & \textbf{Word} & \multicolumn{2}{c|}{\textbf{Score}} & \textbf{Rank} & \textbf{Word} & \multicolumn{2}{c|}{\textbf{Score}} \\
    \hline
    1 & uncooperative\_jj & -22 & 3396    &    430 & concise\_jj & 10 & 1755 \\
    2 & abusive\_jj & -15 & 9532    &    429 & pessimistic\_jj & 9 & 8800 \\
    3 & surly\_jj & -14 & 5083    &    428 & cautious\_jj & 6 & 3025 \\
    4 & negligent\_jj & -11 & 8929    &    427 & optimistic\_jj & 6 & 1928 \\
    5 & negligence & -8 & 8749    &    426 & gullible\_jj & 5 & 8612 \\
    6 & cruelty & -7 & 7505    &    425 & lazy\_jj & 5 & 8422 \\
    7 & unscrupulous\_jj & -5 & 7466    &    424 & cynical\_jj & 5 & 7949 \\
    8 & friendly\_jj & -5 & 5212    &    423 & foolhardy\_jj & 5 & 7347 \\
    9 & cooperation & -5 & 3316    &    422 & naïve\_jj & 5 & 6949 \\
    10 & informal\_jj & -5 & 1749    &    421 & skeptical\_jj & 5 & 5648 \\
    11 & orderly\_jj & -5 & 870    &    420 & wishy-washy\_jj & 5 & 4455 \\
    12 & absent-minded\_jj & -5 & 266    &    419 & lethargy & 5 & 396 \\
    13 & easygoing\_jj & -4 & 8979    &    418 & ignorant\_jj & 4 & 8767 \\
    14 & independent\_jj & -4 & 8245    &    417 & insightful\_jj & 4 & 5778 \\
    15 & organized\_jj & -4 & 8067    &    416 & confident\_jj & 4 & 5534 \\
    16 & organization & -4 & 7265    &    415 & gullibility & 4 & 3868 \\
    17 & unrestrained\_jj & -4 & 7124    &    414 & irritable\_jj & 4 & 2144 \\
    18 & systematic\_jj & -4 & 5669    &    413 & nervous\_jj & 4 & 1913 \\
    19 & cooperative\_jj & -4 & 5296    &    412 & neat\_jj & 4 & 1715 \\
    20 & exacting\_jj & -4 & 4680    &    411 & logical\_jj & 4 & 1515 \\
    21 & earthiness & -4 & 3957    &    410 & dull\_jj & 4 & 1515 \\
    22 & recklessness & -4 & 3464    &    409 & shallowness & 4 & 1497 \\
    23 & cordial\_jj & -4 & 3355    &    408 & insight & 4 & 1168 \\
    24 & spirit & -4 & 2388    &    407 & philosophical\_jj & 4 & 781 \\
    25 & leniency & -4 & 1740    &    406 & uncharitable\_jj & 4 & 503 \\
    26 & autonomous\_jj & -4 & 1117    &    405 & envious\_jj & 4 & 401 \\
    27 & explosive\_jj & -3 & 8946    &    404 & smug\_jj & 3 & 9715 \\
    28 & reckless\_jj & -3 & 7827    &    403 & unimaginative\_jj & 3 & 9711 \\
    29 & gruff\_jj & -3 & 7733    &    402 & irritability & 3 & 9689 \\
    30 & earthy\_jj & -3 & 7410    &    401 & brave\_jj & 3 & 9471 \\
    \hline
    \end{tabular}
    \caption{Scores and rankings for most extreme 30 words in component \#10} 
\end{table}
\clearpage
\begin{table}[tbp]
    \begin{tabular}{| rlr@{.}l | rlr@{.}l |}
    \hline
    \textbf{Rank} & \textbf{Word} & \multicolumn{2}{c|}{\textbf{Score}} & \textbf{Rank} & \textbf{Word} & \multicolumn{2}{c|}{\textbf{Score}} \\
    \hline
    1 & abusive\_jj & -17 & 2074    &    430 & imperturbable\_jj & 33 & 2338 \\
    2 & nonconforming\_jj & -6 & 4581    &    429 & economical\_jj & 9 & 516 \\
    3 & understanding & -5 & 2099    &    428 & reckless\_jj & 7 & 9265 \\
    4 & insight & -5 & 885    &    427 & recklessness & 7 & 4432 \\
    5 & touchy\_jj & -5 & 654    &    426 & negligent\_jj & 7 & 3527 \\
    6 & diplomatic\_jj & -4 & 7912    &    425 & indecisive\_jj & 6 & 9242 \\
    7 & warm\_jj & -4 & 7150    &    424 & obstinate\_jj & 6 & 1544 \\
    8 & formal\_jj & -4 & 2685    &    423 & tenacious\_jj & 5 & 9874 \\
    9 & silent\_jj & -4 & 2593    &    422 & negligence & 5 & 5422 \\
    10 & concise\_jj & -4 & 365    &    421 & erratic\_jj & 5 & 374 \\
    11 & flippant\_jj & -3 & 8470    &    420 & underhanded\_jj & 4 & 9733 \\
    12 & wordy\_jj & -3 & 8407    &    419 & lethargic\_jj & 4 & 8577 \\
    13 & curious\_jj & -3 & 7310    &    418 & sluggish\_jj & 4 & 7013 \\
    14 & melancholic\_jj & -3 & 7100    &    417 & energetic\_jj & 4 & 6937 \\
    15 & philosophical\_jj & -3 & 6424    &    416 & easygoing\_jj & 4 & 6508 \\
    16 & logic & -3 & 6188    &    415 & inefficient\_jj & 4 & 3747 \\
    17 & morality & -3 & 4800    &    414 & assertive\_jj & 4 & 1266 \\
    18 & jealous\_jj & -3 & 4792    &    413 & careless\_jj & 4 & 633 \\
    19 & unkind\_jj & -3 & 4708    &    412 & adventurous\_jj & 4 & 249 \\
    20 & intrusiveness & -3 & 4170    &    411 & optimism & 3 & 9153 \\
    21 & uncooperative\_jj & -3 & 3756    &    410 & efficient\_jj & 3 & 8381 \\
    22 & direct\_jj & -3 & 3536    &    409 & pessimism & 3 & 8024 \\
    23 & respectful\_jj & -3 & 3182    &    408 & impetuous\_jj & 3 & 7852 \\
    24 & truthful\_jj & -3 & 3141    &    407 & stubborn\_jj & 3 & 7789 \\
    25 & bashful\_jj & -3 & 3139    &    406 & gregarious\_jj & 3 & 7238 \\
    26 & curt\_jj & -3 & 2820    &    405 & lethargy & 3 & 7115 \\
    27 & intellectuality & -3 & 2296    &    404 & indecisiveness & 3 & 6586 \\
    28 & suspicious\_jj & -3 & 1618    &    403 & deceit & 3 & 5559 \\
    29 & curiosity & -3 & 1390    &    402 & unscrupulous\_jj & 3 & 5129 \\
    30 & nonconformity & -3 & 1052    &    401 & considerate\_jj & 3 & 4264 \\
    \hline
    \end{tabular}
    \caption{Scores and rankings for most extreme 30 words in component \#11} 
\end{table}
\clearpage
\begin{table}[tbp]
    \begin{tabular}{| rlr@{.}l | rlr@{.}l |}
    \hline
    \textbf{Rank} & \textbf{Word} & \multicolumn{2}{c|}{\textbf{Score}} & \textbf{Rank} & \textbf{Word} & \multicolumn{2}{c|}{\textbf{Score}} \\
    \hline
    1 & abusive\_jj & -27 & 6034    &    430 & friendly\_jj & 8 & 1823 \\
    2 & explosive\_jj & -7 & 7951    &    429 & punctuality & 6 & 8692 \\
    3 & expressive\_jj & -7 & 5071    &    428 & efficiency & 6 & 8356 \\
    4 & erratic\_jj & -6 & 6751    &    427 & economical\_jj & 6 & 5448 \\
    5 & unpredictable\_jj & -6 & 2747    &    426 & prompt\_jj & 6 & 2457 \\
    6 & assertive\_jj & -5 & 4022    &    425 & inconsiderate\_jj & 6 & 1945 \\
    7 & inventive\_jj & -5 & 3457    &    424 & punctual\_jj & 5 & 9841 \\
    8 & concise\_jj & -5 & 1180    &    423 & bossiness & 5 & 7880 \\
    9 & unstable\_jj & -5 & 330    &    422 & cordial\_jj & 5 & 5406 \\
    10 & forceful\_jj & -4 & 9645    &    421 & inefficient\_jj & 5 & 3578 \\
    11 & tenacious\_jj & -4 & 7233    &    420 & efficient\_jj & 5 & 2216 \\
    12 & suspicious\_jj & -4 & 6768    &    419 & neat\_jj & 5 & 2205 \\
    13 & perceptive\_jj & -4 & 6144    &    418 & nonconforming\_jj & 5 & 293 \\
    14 & indecisive\_jj & -4 & 6018    &    417 & mannerly\_jj & 4 & 7802 \\
    15 & energetic\_jj & -4 & 3732    &    416 & crabby\_jj & 4 & 6059 \\
    16 & manipulative\_jj & -4 & 3233    &    415 & miserly\_jj & 4 & 5273 \\
    17 & emotional\_jj & -4 & 1995    &    414 & lenient\_jj & 4 & 3725 \\
    18 & optimistic\_jj & -4 & 1509    &    413 & surly\_jj & 4 & 3089 \\
    19 & insight & -4 & 843    &    412 & leniency & 4 & 2566 \\
    20 & skeptical\_jj & -3 & 8676    &    411 & pleasant\_jj & 4 & 959 \\
    21 & rebellious\_jj & -3 & 7622    &    410 & warm\_jj & 4 & 339 \\
    22 & adventurous\_jj & -3 & 6958    &    409 & patient\_jj & 4 & 120 \\
    23 & combative\_jj & -3 & 6837    &    408 & rudeness & 3 & 9834 \\
    24 & insightful\_jj & -3 & 6576    &    407 & obliging\_jj & 3 & 9242 \\
    25 & insecure\_jj & -3 & 6347    &    406 & conceited\_jj & 3 & 8276 \\
    26 & obstinate\_jj & -3 & 5970    &    405 & nosey\_jj & 3 & 6790 \\
    27 & stubbornness & -3 & 5867    &    404 & frivolous\_jj & 3 & 6342 \\
    28 & verbal\_jj & -3 & 5864    &    403 & envy & 3 & 5972 \\
    29 & ruthless\_jj & -3 & 5800    &    402 & trustful\_jj & 3 & 5734 \\
    30 & impetuous\_jj & -3 & 5413    &    401 & traditional\_jj & 3 & 5236 \\
    \hline
    \end{tabular}
    \caption{Scores and rankings for most extreme 30 words in component \#12} 
\end{table}
\clearpage
\begin{table}[tbp]
    \begin{tabular}{| rlr@{.}l | rlr@{.}l |}
    \hline
    \textbf{Rank} & \textbf{Word} & \multicolumn{2}{c|}{\textbf{Score}} & \textbf{Rank} & \textbf{Word} & \multicolumn{2}{c|}{\textbf{Score}} \\
    \hline
    1 & belligerence & -8 & 5113    &    430 & imperturbable\_jj & 24 & 8043 \\
    2 & explosive\_jj & -7 & 4134    &    429 & abusive\_jj & 12 & 7503 \\
    3 & cordial\_jj & -6 & 7910    &    428 & concise\_jj & 6 & 2637 \\
    4 & sophistication & -6 & 5626    &    427 & understanding\_jj & 6 & 2566 \\
    5 & instability & -6 & 5247    &    426 & unreflective\_jj & 6 & 327 \\
    6 & erratic\_jj & -5 & 5485    &    425 & meditative\_jj & 5 & 4432 \\
    7 & unsophisticated\_jj & -5 & 2996    &    424 & silence & 5 & 1792 \\
    8 & absent-minded\_jj & -5 & 2833    &    423 & independent\_jj & 5 & 746 \\
    9 & friendly\_jj & -5 & 1745    &    422 & nonconformity & 5 & 691 \\
    10 & unstable\_jj & -5 & 1662    &    421 & ethical\_jj & 4 & 7681 \\
    11 & warm\_jj & -5 & 917    &    420 & independence & 4 & 7634 \\
    12 & humor & -4 & 7998    &    419 & impudent\_jj & 4 & 6990 \\
    13 & folksy\_jj & -4 & 7813    &    418 & charitable\_jj & 4 & 6789 \\
    14 & unreliable\_jj & -4 & 6686    &    417 & quiet & 4 & 5929 \\
    15 & deceit & -4 & 6535    &    416 & autonomous\_jj & 4 & 5331 \\
    16 & sophisticated\_jj & -4 & 6516    &    415 & principled\_jj & 4 & 4877 \\
    17 & sloppy\_jj & -4 & 5587    &    414 & nonconforming\_jj & 4 & 2785 \\
    18 & easygoing\_jj & -4 & 5191    &    413 & organization & 4 & 2527 \\
    19 & insecure\_jj & -4 & 4784    &    412 & uncharitable\_jj & 3 & 9787 \\
    20 & aloofness & -4 & 3046    &    411 & aimlessness & 3 & 9437 \\
    21 & crafty\_jj & -4 & 2296    &    410 & respectful\_jj & 3 & 8287 \\
    22 & deliberate\_jj & -4 & 785    &    409 & unkind\_jj & 3 & 8101 \\
    23 & volatile\_jj & -3 & 9933    &    408 & conscientious\_jj & 3 & 5503 \\
    24 & cunning & -3 & 9339    &    407 & thrift & 3 & 4718 \\
    25 & insecurity & -3 & 9215    &    406 & morality & 3 & 4683 \\
    26 & ruthless\_jj & -3 & 8258    &    405 & fretful\_jj & 3 & 4493 \\
    27 & dependable\_jj & -3 & 8215    &    404 & brave\_jj & 3 & 4289 \\
    28 & unpredictable\_jj & -3 & 8039    &    403 & forgetfulness & 3 & 3487 \\
    29 & warmth & -3 & 7354    &    402 & artistic\_jj & 3 & 3461 \\
    30 & cunning\_jj & -3 & 7144    &    401 & prompt\_jj & 3 & 2336 \\
    \hline
    \end{tabular}
    \caption{Scores and rankings for most extreme 30 words in component \#13} 
\end{table}
\clearpage
\begin{table}[tbp]
    \begin{tabular}{| rlr@{.}l | rlr@{.}l |}
    \hline
    \textbf{Rank} & \textbf{Word} & \multicolumn{2}{c|}{\textbf{Score}} & \textbf{Rank} & \textbf{Word} & \multicolumn{2}{c|}{\textbf{Score}} \\
    \hline
    1 & abusive\_jj & -16 & 2180    &    430 & imperturbable\_jj & 11 & 4333 \\
    2 & economical\_jj & -9 & 6739    &    429 & intelligence & 8 & 2835 \\
    3 & thrifty\_jj & -8 & 5737    &    428 & unscrupulous\_jj & 7 & 6574 \\
    4 & refined\_jj & -8 & 1262    &    427 & suspicious\_jj & 7 & 4177 \\
    5 & brave\_jj & -7 & 8849    &    426 & concise\_jj & 7 & 3334 \\
    6 & selfless\_jj & -6 & 9235    &    425 & distrust & 7 & 3232 \\
    7 & courage & -6 & 8368    &    424 & instability & 7 & 2943 \\
    8 & stingy\_jj & -6 & 6408    &    423 & explosive\_jj & 7 & 1920 \\
    9 & dignity & -5 & 1503    &    422 & organized\_jj & 6 & 2629 \\
    10 & generosity & -4 & 9471    &    421 & diplomatic\_jj & 5 & 9307 \\
    11 & restrained\_jj & -4 & 9015    &    420 & touchy\_jj & 5 & 1729 \\
    12 & courageous\_jj & -4 & 7124    &    419 & insight & 5 & 1722 \\
    13 & lenient\_jj & -4 & 6127    &    418 & antagonistic\_jj & 4 & 7170 \\
    14 & efficient\_jj & -4 & 4406    &    417 & independent\_jj & 4 & 7003 \\
    15 & lethargic\_jj & -4 & 4331    &    416 & wordy\_jj & 4 & 6500 \\
    16 & leniency & -4 & 2775    &    415 & informal\_jj & 4 & 4658 \\
    17 & accommodating\_jj & -4 & 2051    &    414 & autonomous\_jj & 4 & 4210 \\
    18 & dignified\_jj & -4 & 1967    &    413 & cooperation & 4 & 4204 \\
    19 & rude\_jj & -4 & 1716    &    412 & folksy\_jj & 4 & 3808 \\
    20 & modest\_jj & -4 & 440    &    411 & disorganization & 4 & 3581 \\
    21 & sluggish\_jj & -4 & 266    &    410 & quarrelsome\_jj & 4 & 2405 \\
    22 & modesty & -3 & 9912    &    409 & sophistication & 4 & 1764 \\
    23 & conscientious\_jj & -3 & 9001    &    408 & insightful\_jj & 4 & 1580 \\
    24 & pleasant\_jj & -3 & 8655    &    407 & systematic\_jj & 4 & 1359 \\
    25 & reserve & -3 & 8264    &    406 & secretive\_jj & 4 & 756 \\
    26 & impractical\_jj & -3 & 7451    &    405 & understanding & 4 & 549 \\
    27 & flexibility & -3 & 6816    &    404 & philosophical\_jj & 4 & 243 \\
    28 & rash\_jj & -3 & 6514    &    403 & humorous\_jj & 3 & 9507 \\
    29 & thrift & -3 & 5826    &    402 & analytical\_jj & 3 & 9177 \\
    30 & miserly\_jj & -3 & 5734    &    401 & cordial\_jj & 3 & 6471 \\
    \hline
    \end{tabular}
    \caption{Scores and rankings for most extreme 30 words in component \#14} 
\end{table}
\clearpage
\begin{table}[tbp]
    \begin{tabular}{| rlr@{.}l | rlr@{.}l |}
    \hline
    \textbf{Rank} & \textbf{Word} & \multicolumn{2}{c|}{\textbf{Score}} & \textbf{Rank} & \textbf{Word} & \multicolumn{2}{c|}{\textbf{Score}} \\
    \hline
    1 & abusive\_jj & -14 & 6956    &    430 & peaceful\_jj & 8 & 7739 \\
    2 & imperturbable\_jj & -11 & 1114    &    429 & rebellious\_jj & 7 & 5401 \\
    3 & dependability & -9 & 6327    &    428 & deliberate\_jj & 7 & 791 \\
    4 & insight & -9 & 1665    &    427 & individualistic\_jj & 7 & 607 \\
    5 & punctuality & -8 & 4690    &    426 & cruelty & 6 & 6129 \\
    6 & efficiency & -7 & 8670    &    425 & inventive\_jj & 5 & 8976 \\
    7 & reliable\_jj & -5 & 9355    &    424 & nonconformity & 5 & 8888 \\
    8 & candor & -5 & 6014    &    423 & vigorous\_jj & 5 & 1425 \\
    9 & unreliable\_jj & -5 & 3087    &    422 & prejudiced\_jj & 5 & 1117 \\
    10 & ungracious\_jj & -5 & 2361    &    421 & somber\_jj & 5 & 765 \\
    11 & envious\_jj & -5 & 1782    &    420 & meditative\_jj & 4 & 8248 \\
    12 & assured\_jj & -5 & 1393    &    419 & lethargy & 4 & 8026 \\
    13 & intelligence & -4 & 4875    &    418 & cruel\_jj & 4 & 6992 \\
    14 & patient\_jj & -4 & 3243    &    417 & contemplative\_jj & 4 & 5325 \\
    15 & unkind\_jj & -4 & 2795    &    416 & absent-minded\_jj & 4 & 4650 \\
    16 & punctual\_jj & -4 & 2389    &    415 & expressive\_jj & 4 & 4605 \\
    17 & stingy\_jj & -4 & 2315    &    414 & docile\_jj & 4 & 4172 \\
    18 & shy\_jj & -4 & 201    &    413 & folksy\_jj & 4 & 3976 \\
    19 & courtesy & -3 & 9104    &    412 & cosmopolitan\_jj & 4 & 3504 \\
    20 & optimism & -3 & 9055    &    411 & principled\_jj & 4 & 2889 \\
    21 & depth & -3 & 8179    &    410 & expressiveness & 4 & 2816 \\
    22 & sophistication & -3 & 7565    &    409 & assertive\_jj & 4 & 2267 \\
    23 & predictability & -3 & 7448    &    408 & dignified\_jj & 4 & 1174 \\
    24 & defensive\_jj & -3 & 4564    &    407 & systematic\_jj & 4 & 701 \\
    25 & courteous\_jj & -3 & 4085    &    406 & organized\_jj & 3 & 9659 \\
    26 & volatility & -3 & 3800    &    405 & traditional\_jj & 3 & 9308 \\
    27 & underhanded\_jj & -3 & 3393    &    404 & lenient\_jj & 3 & 9218 \\
    28 & humor & -3 & 3379    &    403 & prejudice & 3 & 8982 \\
    29 & perceptive\_jj & -3 & 3281    &    402 & negligent\_jj & 3 & 8973 \\
    30 & reserve & -3 & 2388    &    401 & forceful\_jj & 3 & 8506 \\
    \hline
    \end{tabular}
    \caption{Scores and rankings for most extreme 30 words in component \#15} 
\end{table}
\clearpage

\subsection{MDS}
\label{app:rankedwordlists:438and101words:mds}
\begin{longtable}[!htbp]{| rlr@{.}l |}
    \hline
    \textbf{Rank} & \textbf{Word} & \multicolumn{2}{c|}{\textbf{Score}} \\
    \hline
    \endhead
    1 & self-pitying\_jj & 0 & -2945 \\
    2 & scatterbrained\_jj & 0 & -2882 \\
    3 & pomposity & 0 & -2841 \\
    4 & pompous\_jj & 0 & -2762 \\
    5 & mischievous\_jj & 0 & -2698 \\
    6 & sly\_jj & 0 & -2687 \\
    7 & snobbish\_jj & 0 & -2682 \\
    8 & condescending\_jj & 0 & -2507 \\
    9 & genial\_jj & 0 & -2486 \\
    10 & shyness & 0 & -2465 \\
    11 & self-indulgent\_jj & 0 & -2419 \\
    12 & exhibitionistic\_jj & 0 & -2397 \\
    13 & egotistical\_jj & 0 & -2364 \\
    14 & introspective\_jj & 0 & -2356 \\
    15 & bossy\_jj & 0 & -2340 \\
    16 & passionless\_jj & 0 & -2301 \\
    17 & playfulness & 0 & -2292 \\
    18 & witty\_jj & 0 & -2260 \\
    19 & gruff\_jj & 0 & -2220 \\
    20 & amiability & 0 & -2193 \\
    21 & egocentric\_jj & 0 & -2186 \\
    22 & bashful\_jj & 0 & -2185 \\
    23 & flippant\_jj & 0 & -2147 \\
    24 & happy-go-lucky\_jj & 0 & -2138 \\
    25 & amiable\_jj & 0 & -2129 \\
    26 & aloofness & 0 & -2107 \\
    27 & forgetful\_jj & 0 & -2103 \\
    28 & boastful\_jj & 0 & -2095 \\
    29 & vivacious\_jj & 0 & -2072 \\
    30 & crabby\_jj & 0 & -2068 \\
    401 & informal\_jj & 0 & 2695 \\
    402 & sluggish\_jj & 0 & 2756 \\
    403 & innovative\_jj & 0 & 2769 \\
    404 & patient\_jj & 0 & 2772 \\
    405 & active\_jj & 0 & 2779 \\
    406 & confident\_jj & 0 & 2781 \\
    407 & diplomatic\_jj & 0 & 2793 \\
    408 & efficiency & 0 & 2833 \\
    409 & responsible\_jj & 0 & 2850 \\
    410 & thorough\_jj & 0 & 2887 \\
    411 & careful\_jj & 0 & 2910 \\
    412 & firm\_jj & 0 & 2920 \\
    413 & decisive\_jj & 0 & 2928 \\
    414 & caution & 0 & 2960 \\
    415 & cautious\_jj & 0 & 2961 \\
    416 & autonomous\_jj & 0 & 2988 \\
    417 & optimistic\_jj & 0 & 2997 \\
    418 & organization & 0 & 3044 \\
    419 & helpful\_jj & 0 & 3118 \\
    420 & modest\_jj & 0 & 3192 \\
    421 & efficient\_jj & 0 & 3201 \\
    422 & volatile\_jj & 0 & 3230 \\
    423 & flexible\_jj & 0 & 3239 \\
    424 & formal\_jj & 0 & 3279 \\
    425 & reasonable\_jj & 0 & 3361 \\
    426 & flexibility & 0 & 3397 \\
    427 & reliable\_jj & 0 & 3425 \\
    428 & direct\_jj & 0 & 3457 \\
    429 & cooperation & 0 & 3682 \\
    430 & consistent\_jj & 0 & 3926 \\
    \hline
    \caption{Scores and rankings for most extreme 30 words in component \#1} \\
\end{longtable}
\begin{longtable}[!htbp]{| rlr@{.}l |}
    \hline
    \textbf{Rank} & \textbf{Word} & \multicolumn{2}{c|}{\textbf{Score}} \\
    \hline
    \endhead
    1 & sociable\_jj & 0 & -3994 \\
    2 & cheerful\_jj & 0 & -3527 \\
    3 & vivacious\_jj & 0 & -3487 \\
    4 & easygoing\_jj & 0 & -3419 \\
    5 & considerate\_jj & 0 & -3408 \\
    6 & enthusiastic\_jj & 0 & -3264 \\
    7 & dependable\_jj & 0 & -3187 \\
    8 & gregarious\_jj & 0 & -3183 \\
    9 & affectionate\_jj & 0 & -3168 \\
    10 & energetic\_jj & 0 & -3086 \\
    11 & down-to-earth\_jj & 0 & -3069 \\
    12 & talkative\_jj & 0 & -3012 \\
    13 & courteous\_jj & 0 & -2982 \\
    14 & cultured\_jj & 0 & -2954 \\
    15 & intelligent\_jj & 0 & -2940 \\
    16 & adventurous\_jj & 0 & -2911 \\
    17 & amiable\_jj & 0 & -2844 \\
    18 & witty\_jj & 0 & -2825 \\
    19 & kind\_jj & 0 & -2781 \\
    20 & unassuming\_jj & 0 & -2771 \\
    21 & pleasant\_jj & 0 & -2752 \\
    22 & jovial\_jj & 0 & -2585 \\
    23 & inquisitive\_jj & 0 & -2523 \\
    24 & humble\_jj & 0 & -2460 \\
    25 & quiet\_jj & 0 & -2369 \\
    26 & polite\_jj & 0 & -2345 \\
    27 & playful\_jj & 0 & -2283 \\
    28 & confident\_jj & 0 & -2282 \\
    29 & introspective\_jj & 0 & -2270 \\
    30 & genial\_jj & 0 & -2262 \\
    401 & morality & 0 & 2206 \\
    402 & intrusiveness & 0 & 2230 \\
    403 & cruelty & 0 & 2234 \\
    404 & unrestrained\_jj & 0 & 2241 \\
    405 & sloth & 0 & 2252 \\
    406 & moral\_jj & 0 & 2297 \\
    407 & thoughtless\_jj & 0 & 2305 \\
    408 & shallowness & 0 & 2377 \\
    409 & belligerence & 0 & 2385 \\
    410 & instability & 0 & 2540 \\
    411 & reckless\_jj & 0 & 2544 \\
    412 & lethargy & 0 & 2622 \\
    413 & inconsistency & 0 & 2722 \\
    414 & rudeness & 0 & 2726 \\
    415 & stubbornness & 0 & 2816 \\
    416 & disorganization & 0 & 2836 \\
    417 & indecisiveness & 0 & 2839 \\
    418 & prejudice & 0 & 2860 \\
    419 & thoughtlessness & 0 & 2860 \\
    420 & insecurity & 0 & 2885 \\
    421 & negligence & 0 & 2911 \\
    422 & passivity & 0 & 2926 \\
    423 & distrust & 0 & 2930 \\
    424 & fear & 0 & 2941 \\
    425 & deceit & 0 & 3365 \\
    426 & recklessness & 0 & 3601 \\
    427 & gullibility & 0 & 3611 \\
    428 & selfishness & 0 & 3630 \\
    429 & callousness & 0 & 3733 \\
    430 & stupidity & 0 & 4071 \\
    \hline
    \caption{Scores and rankings for most extreme 30 words in component \#2} \\
\end{longtable}
\begin{longtable}[!htbp]{| rlr@{.}l |}
    \hline
    \textbf{Rank} & \textbf{Word} & \multicolumn{2}{c|}{\textbf{Score}} \\
    \hline
    \endhead
    1 & ignorant\_jj & 0 & -3365 \\
    2 & dishonest\_jj & 0 & -3196 \\
    3 & insensitive\_jj & 0 & -3084 \\
    4 & greedy\_jj & 0 & -2947 \\
    5 & unfriendly\_jj & 0 & -2934 \\
    6 & disrespectful\_jj & 0 & -2926 \\
    7 & unsympathetic\_jj & 0 & -2834 \\
    8 & bigoted\_jj & 0 & -2831 \\
    9 & inconsiderate\_jj & 0 & -2730 \\
    10 & inefficient\_jj & 0 & -2618 \\
    11 & gullible\_jj & 0 & -2604 \\
    12 & vindictive\_jj & 0 & -2579 \\
    13 & pessimistic\_jj & 0 & -2495 \\
    14 & lenient\_jj & 0 & -2466 \\
    15 & intrusive\_jj & 0 & -2428 \\
    16 & naïve\_jj & 0 & -2370 \\
    17 & unreliable\_jj & 0 & -2349 \\
    18 & lazy\_jj & 0 & -2293 \\
    19 & uncooperative\_jj & 0 & -2227 \\
    20 & deceitful\_jj & 0 & -2183 \\
    21 & unscrupulous\_jj & 0 & -2166 \\
    22 & cynical\_jj & 0 & -2142 \\
    23 & selfish\_jj & 0 & -2139 \\
    24 & timid\_jj & 0 & -2138 \\
    25 & impolite\_jj & 0 & -2115 \\
    26 & prejudiced\_jj & 0 & -2109 \\
    27 & distrustful\_jj & 0 & -1992 \\
    28 & rude\_jj & 0 & -1933 \\
    29 & thoughtless\_jj & 0 & -1931 \\
    30 & stingy\_jj & 0 & -1924 \\
    401 & dependability & 0 & 2011 \\
    402 & earthy\_jj & 0 & 2046 \\
    403 & gregariousness & 0 & 2053 \\
    404 & homespun\_jj & 0 & 2053 \\
    405 & predictability & 0 & 2058 \\
    406 & understanding & 0 & 2068 \\
    407 & curiosity & 0 & 2091 \\
    408 & empathy & 0 & 2241 \\
    409 & optimism & 0 & 2242 \\
    410 & ambition & 0 & 2256 \\
    411 & generosity & 0 & 2315 \\
    412 & courage & 0 & 2350 \\
    413 & decisiveness & 0 & 2518 \\
    414 & meditative\_jj & 0 & 2536 \\
    415 & flexibility & 0 & 2554 \\
    416 & earthiness & 0 & 2576 \\
    417 & depth & 0 & 2637 \\
    418 & sophistication & 0 & 2645 \\
    419 & candor & 0 & 2750 \\
    420 & naturalness & 0 & 2772 \\
    421 & expressiveness & 0 & 2820 \\
    422 & artistic\_jj & 0 & 2847 \\
    423 & humor & 0 & 2850 \\
    424 & spirit & 0 & 2954 \\
    425 & persistence & 0 & 3069 \\
    426 & playfulness & 0 & 3388 \\
    427 & spontaneity & 0 & 3476 \\
    428 & precision & 0 & 3573 \\
    429 & warmth & 0 & 3584 \\
    430 & creativity & 0 & 3739 \\
    \hline
    \caption{Scores and rankings for most extreme 30 words in component \#3} \\
\end{longtable}
\begin{longtable}[!htbp]{| rlr@{.}l |}
    \hline
    \textbf{Rank} & \textbf{Word} & \multicolumn{2}{c|}{\textbf{Score}} \\
    \hline
    \endhead
    1 & innovative\_jj & 0 & -2731 \\
    2 & dishonest\_jj & 0 & -2458 \\
    3 & intelligent\_jj & 0 & -2327 \\
    4 & practical\_jj & 0 & -2310 \\
    5 & ethical\_jj & 0 & -2305 \\
    6 & imaginative\_jj & 0 & -2281 \\
    7 & truthful\_jj & 0 & -2202 \\
    8 & principled\_jj & 0 & -2200 \\
    9 & courageous\_jj & 0 & -2126 \\
    10 & honest\_jj & 0 & -2110 \\
    11 & insightful\_jj & 0 & -2071 \\
    12 & cunning\_jj & 0 & -2068 \\
    13 & analytical\_jj & 0 & -2050 \\
    14 & logic & 0 & -2016 \\
    15 & creative\_jj & 0 & -1998 \\
    16 & moral\_jj & 0 & -1906 \\
    17 & adaptable\_jj & 0 & -1819 \\
    18 & logical\_jj & 0 & -1790 \\
    19 & morality & 0 & -1776 \\
    20 & straightforward\_jj & 0 & -1748 \\
    21 & devious\_jj & 0 & -1744 \\
    22 & sophisticated\_jj & 0 & -1743 \\
    23 & selfless\_jj & 0 & -1738 \\
    24 & systematic\_jj & 0 & -1719 \\
    25 & unintelligent\_jj & 0 & -1710 \\
    26 & creativity & 0 & -1693 \\
    27 & efficient\_jj & 0 & -1693 \\
    28 & intellectual\_jj & 0 & -1677 \\
    29 & deceitful\_jj & 0 & -1642 \\
    30 & dignity & 0 & -1628 \\
    401 & cranky\_jj & 0 & 1615 \\
    402 & aimlessness & 0 & 1643 \\
    403 & pessimism & 0 & 1681 \\
    404 & quiet\_jj & 0 & 1713 \\
    405 & cautious\_jj & 0 & 1719 \\
    406 & insecurity & 0 & 1730 \\
    407 & pessimistic\_jj & 0 & 1753 \\
    408 & dominant\_jj & 0 & 1759 \\
    409 & warm\_jj & 0 & 1825 \\
    410 & somber\_jj & 0 & 1855 \\
    411 & reserve & 0 & 1903 \\
    412 & quiet & 0 & 1934 \\
    413 & optimistic\_jj & 0 & 1954 \\
    414 & volatile\_jj & 0 & 2036 \\
    415 & persistent\_jj & 0 & 2039 \\
    416 & volatility & 0 & 2092 \\
    417 & instability & 0 & 2151 \\
    418 & steady\_jj & 0 & 2193 \\
    419 & fretful\_jj & 0 & 2370 \\
    420 & irritability & 0 & 2422 \\
    421 & bitter\_jj & 0 & 2465 \\
    422 & fearful\_jj & 0 & 2481 \\
    423 & optimism & 0 & 2505 \\
    424 & lethargy & 0 & 2533 \\
    425 & irritable\_jj & 0 & 2688 \\
    426 & cold\_jj & 0 & 2769 \\
    427 & anxious\_jj & 0 & 2905 \\
    428 & lethargic\_jj & 0 & 2981 \\
    429 & sluggish\_jj & 0 & 3316 \\
    430 & nervous\_jj & 0 & 3341 \\
    \hline
    \caption{Scores and rankings for most extreme 30 words in component \#4} \\
\end{longtable}
\begin{longtable}[!htbp]{| rlr@{.}l |}
    \hline
    \textbf{Rank} & \textbf{Word} & \multicolumn{2}{c|}{\textbf{Score}} \\
    \hline
    \endhead
    1 & complex\_jj & 0 & -2574 \\
    2 & innovative\_jj & 0 & -2541 \\
    3 & efficient\_jj & 0 & -2341 \\
    4 & inefficient\_jj & 0 & -2313 \\
    5 & conventional\_jj & 0 & -2273 \\
    6 & extravagant\_jj & 0 & -2172 \\
    7 & refined\_jj & 0 & -2156 \\
    8 & sophisticated\_jj & 0 & -2098 \\
    9 & undemanding\_jj & 0 & -2050 \\
    10 & traditional\_jj & 0 & -2043 \\
    11 & unimaginative\_jj & 0 & -1928 \\
    12 & unconventional\_jj & 0 & -1875 \\
    13 & inventive\_jj & 0 & -1855 \\
    14 & economical\_jj & 0 & -1804 \\
    15 & animation & 0 & -1759 \\
    16 & undependable\_jj & 0 & -1747 \\
    17 & flexible\_jj & 0 & -1640 \\
    18 & impractical\_jj & 0 & -1623 \\
    19 & nonconforming\_jj & 0 & -1616 \\
    20 & conventionality & 0 & -1616 \\
    21 & slothful\_jj & 0 & -1591 \\
    22 & unaggressive\_jj & 0 & -1573 \\
    23 & impersonal\_jj & 0 & -1534 \\
    24 & imaginative\_jj & 0 & -1497 \\
    25 & unsophisticated\_jj & 0 & -1455 \\
    26 & unsociable\_jj & 0 & -1441 \\
    27 & autonomous\_jj & 0 & -1437 \\
    28 & adaptable\_jj & 0 & -1437 \\
    29 & ambitious\_jj & 0 & -1400 \\
    30 & unreliable\_jj & 0 & -1381 \\
    401 & cordial\_jj & 0 & 1588 \\
    402 & skeptical\_jj & 0 & 1632 \\
    403 & leniency & 0 & 1632 \\
    404 & quiet\_jj & 0 & 1648 \\
    405 & prejudice & 0 & 1678 \\
    406 & modesty & 0 & 1679 \\
    407 & silence & 0 & 1708 \\
    408 & disrespectful\_jj & 0 & 1709 \\
    409 & brave\_jj & 0 & 1714 \\
    410 & courteous\_jj & 0 & 1752 \\
    411 & fear & 0 & 1754 \\
    412 & confident\_jj & 0 & 1781 \\
    413 & considerate\_jj & 0 & 1840 \\
    414 & principled\_jj & 0 & 1883 \\
    415 & empathy & 0 & 2038 \\
    416 & spirit & 0 & 2089 \\
    417 & polite\_jj & 0 & 2118 \\
    418 & generosity & 0 & 2165 \\
    419 & candor & 0 & 2185 \\
    420 & respectful\_jj & 0 & 2188 \\
    421 & selfless\_jj & 0 & 2203 \\
    422 & kind\_jj & 0 & 2253 \\
    423 & courageous\_jj & 0 & 2354 \\
    424 & frank\_jj & 0 & 2437 \\
    425 & proud\_jj & 0 & 2467 \\
    426 & honest\_jj & 0 & 2504 \\
    427 & optimism & 0 & 2544 \\
    428 & dignity & 0 & 2911 \\
    429 & courage & 0 & 3063 \\
    430 & sincere\_jj & 0 & 3636 \\
    \hline
    \caption{Scores and rankings for most extreme 30 words in component \#5} \\
\end{longtable}
\begin{longtable}[!htbp]{| rlr@{.}l |}
    \hline
    \textbf{Rank} & \textbf{Word} & \multicolumn{2}{c|}{\textbf{Score}} \\
    \hline
    \endhead
    1 & organization & 0 & -2561 \\
    2 & active\_jj & 0 & -2410 \\
    3 & independent\_jj & 0 & -2408 \\
    4 & autonomous\_jj & 0 & -2378 \\
    5 & sociable\_jj & 0 & -2246 \\
    6 & proud\_jj & 0 & -2065 \\
    7 & fear & 0 & -2031 \\
    8 & intelligent\_jj & 0 & -1895 \\
    9 & kind\_jj & 0 & -1834 \\
    10 & adventurous\_jj & 0 & -1828 \\
    11 & considerate\_jj & 0 & -1818 \\
    12 & envious\_jj & 0 & -1747 \\
    13 & insecurity & 0 & -1726 \\
    14 & insecure\_jj & 0 & -1705 \\
    15 & enterprising\_jj & 0 & -1681 \\
    16 & self-esteem & 0 & -1621 \\
    17 & envy & 0 & -1615 \\
    18 & curiosity & 0 & -1570 \\
    19 & cultured\_jj & 0 & -1566 \\
    20 & fearful\_jj & 0 & -1558 \\
    21 & distrustful\_jj & 0 & -1547 \\
    22 & creative\_jj & 0 & -1522 \\
    23 & distrust & 0 & -1520 \\
    24 & inquisitive\_jj & 0 & -1518 \\
    25 & understanding & 0 & -1486 \\
    26 & individualistic\_jj & 0 & -1466 \\
    27 & innovative\_jj & 0 & -1463 \\
    28 & vivacious\_jj & 0 & -1442 \\
    29 & independence & 0 & -1417 \\
    30 & worldly\_jj & 0 & -1382 \\
    401 & frank\_jj & 0 & 1392 \\
    402 & simple\_jj & 0 & 1427 \\
    403 & sly\_jj & 0 & 1455 \\
    404 & unemotional\_jj & 0 & 1493 \\
    405 & daring\_jj & 0 & 1502 \\
    406 & folksy\_jj & 0 & 1556 \\
    407 & precise\_jj & 0 & 1581 \\
    408 & verbose\_jj & 0 & 1588 \\
    409 & haphazard\_jj & 0 & 1665 \\
    410 & rash\_jj & 0 & 1691 \\
    411 & predictable\_jj & 0 & 1717 \\
    412 & harsh\_jj & 0 & 1730 \\
    413 & tempestuous\_jj & 0 & 1736 \\
    414 & dull\_jj & 0 & 1749 \\
    415 & spirited\_jj & 0 & 1821 \\
    416 & consistent\_jj & 0 & 1863 \\
    417 & verbal\_jj & 0 & 1877 \\
    418 & curt\_jj & 0 & 1889 \\
    419 & decisive\_jj & 0 & 1920 \\
    420 & caustic\_jj & 0 & 1957 \\
    421 & inconsistent\_jj & 0 & 1969 \\
    422 & deliberate\_jj & 0 & 2012 \\
    423 & combative\_jj & 0 & 2031 \\
    424 & vigorous\_jj & 0 & 2055 \\
    425 & bold\_jj & 0 & 2080 \\
    426 & somber\_jj & 0 & 2117 \\
    427 & restrained\_jj & 0 & 2289 \\
    428 & straightforward\_jj & 0 & 2406 \\
    429 & forceful\_jj & 0 & 2631 \\
    430 & sloppy\_jj & 0 & 2794 \\
    \hline
    \caption{Scores and rankings for most extreme 30 words in component \#6} \\
\end{longtable}
\begin{longtable}[!htbp]{| rlr@{.}l |}
    \hline
    \textbf{Rank} & \textbf{Word} & \multicolumn{2}{c|}{\textbf{Score}} \\
    \hline
    \endhead
    1 & brave\_jj & 0 & -2208 \\
    2 & cunning\_jj & 0 & -2023 \\
    3 & shy\_jj & 0 & -2020 \\
    4 & bold\_jj & 0 & -1976 \\
    5 & tenacious\_jj & 0 & -1910 \\
    6 & crafty\_jj & 0 & -1908 \\
    7 & daring & 0 & -1825 \\
    8 & proud\_jj & 0 & -1822 \\
    9 & bright\_jj & 0 & -1820 \\
    10 & flamboyant\_jj & 0 & -1799 \\
    11 & smart\_jj & 0 & -1790 \\
    12 & ruthless\_jj & 0 & -1766 \\
    13 & courtesy & 0 & -1752 \\
    14 & greedy\_jj & 0 & -1707 \\
    15 & lazy\_jj & 0 & -1691 \\
    16 & ambition & 0 & -1646 \\
    17 & extravagant\_jj & 0 & -1644 \\
    18 & neat\_jj & 0 & -1629 \\
    19 & ambitious\_jj & 0 & -1619 \\
    20 & gullible\_jj & 0 & -1613 \\
    21 & daring\_jj & 0 & -1518 \\
    22 & vain\_jj & 0 & -1502 \\
    23 & cranky\_jj & 0 & -1444 \\
    24 & defensive\_jj & 0 & -1434 \\
    25 & stupidity & 0 & -1434 \\
    26 & cunning & 0 & -1384 \\
    27 & jealous\_jj & 0 & -1381 \\
    28 & sloppy\_jj & 0 & -1369 \\
    29 & creative\_jj & 0 & -1350 \\
    30 & dominant\_jj & 0 & -1343 \\
    401 & tactful\_jj & 0 & 1382 \\
    402 & candor & 0 & 1394 \\
    403 & truthful\_jj & 0 & 1395 \\
    404 & self-critical\_jj & 0 & 1402 \\
    405 & nonconforming\_jj & 0 & 1416 \\
    406 & rudeness & 0 & 1431 \\
    407 & self-disciplined\_jj & 0 & 1436 \\
    408 & understanding\_jj & 0 & 1439 \\
    409 & touchy\_jj & 0 & 1457 \\
    410 & frank\_jj & 0 & 1471 \\
    411 & diplomatic\_jj & 0 & 1477 \\
    412 & inhibition & 0 & 1498 \\
    413 & agreeable\_jj & 0 & 1506 \\
    414 & cooperation & 0 & 1608 \\
    415 & patient\_jj & 0 & 1615 \\
    416 & communicative\_jj & 0 & 1672 \\
    417 & forgetfulness & 0 & 1680 \\
    418 & punctual\_jj & 0 & 1758 \\
    419 & irritable\_jj & 0 & 1805 \\
    420 & talkativeness & 0 & 1877 \\
    421 & suggestible\_jj & 0 & 1911 \\
    422 & extroverted\_jj & 0 & 1986 \\
    423 & irritability & 0 & 2009 \\
    424 & cooperative\_jj & 0 & 2031 \\
    425 & uncooperative\_jj & 0 & 2048 \\
    426 & cordial\_jj & 0 & 2125 \\
    427 & unsystematic\_jj & 0 & 2251 \\
    428 & respectful\_jj & 0 & 2338 \\
    429 & antagonistic\_jj & 0 & 2377 \\
    430 & trustful\_jj & 0 & 2951 \\
    \hline
    \caption{Scores and rankings for most extreme 30 words in component \#7} \\
\end{longtable}
\begin{longtable}[!htbp]{| rlr@{.}l |}
    \hline
    \textbf{Rank} & \textbf{Word} & \multicolumn{2}{c|}{\textbf{Score}} \\
    \hline
    \endhead
    1 & simple\_jj & 0 & -2770 \\
    2 & neat\_jj & 0 & -2337 \\
    3 & pleasant\_jj & 0 & -2104 \\
    4 & lazy\_jj & 0 & -2076 \\
    5 & patient\_jj & 0 & -2074 \\
    6 & careless\_jj & 0 & -2021 \\
    7 & irritability & 0 & -2008 \\
    8 & warm\_jj & 0 & -1850 \\
    9 & cold\_jj & 0 & -1825 \\
    10 & shallow\_jj & 0 & -1787 \\
    11 & casual\_jj & 0 & -1753 \\
    12 & forgetfulness & 0 & -1750 \\
    13 & helpful\_jj & 0 & -1614 \\
    14 & unsociable\_jj & 0 & -1576 \\
    15 & practical\_jj & 0 & -1548 \\
    16 & inconsiderate\_jj & 0 & -1509 \\
    17 & nosey\_jj & 0 & -1478 \\
    18 & rude\_jj & 0 & -1470 \\
    19 & dull\_jj & 0 & -1448 \\
    20 & smart\_jj & 0 & -1447 \\
    21 & spontaneous\_jj & 0 & -1443 \\
    22 & negligence & 0 & -1440 \\
    23 & reasonable\_jj & 0 & -1434 \\
    24 & courtesy & 0 & -1368 \\
    25 & quiet\_jj & 0 & -1359 \\
    26 & bright\_jj & 0 & -1358 \\
    27 & scatterbrained\_jj & 0 & -1322 \\
    28 & crabby\_jj & 0 & -1309 \\
    29 & frivolous\_jj & 0 & -1298 \\
    30 & curious\_jj & 0 & -1290 \\
    401 & demonstrative\_jj & 0 & 1291 \\
    402 & amiability & 0 & 1316 \\
    403 & distrust & 0 & 1323 \\
    404 & impetuous\_jj & 0 & 1331 \\
    405 & cooperation & 0 & 1364 \\
    406 & decisive\_jj & 0 & 1372 \\
    407 & dominant\_jj & 0 & 1399 \\
    408 & principled\_jj & 0 & 1461 \\
    409 & aloofness & 0 & 1474 \\
    410 & secretive\_jj & 0 & 1475 \\
    411 & unsympathetic\_jj & 0 & 1510 \\
    412 & courageous\_jj & 0 & 1539 \\
    413 & belligerence & 0 & 1705 \\
    414 & indecisive\_jj & 0 & 1722 \\
    415 & exacting\_jj & 0 & 1740 \\
    416 & decisiveness & 0 & 1773 \\
    417 & forceful\_jj & 0 & 1838 \\
    418 & ambition & 0 & 1877 \\
    419 & ambitious\_jj & 0 & 1928 \\
    420 & accommodating\_jj & 0 & 1951 \\
    421 & stubborn\_jj & 0 & 1958 \\
    422 & distrustful\_jj & 0 & 1961 \\
    423 & stubbornness & 0 & 2001 \\
    424 & ruthless\_jj & 0 & 2002 \\
    425 & tenacious\_jj & 0 & 2005 \\
    426 & individualistic\_jj & 0 & 2092 \\
    427 & obstinate\_jj & 0 & 2193 \\
    428 & combative\_jj & 0 & 2379 \\
    429 & antagonistic\_jj & 0 & 2380 \\
    430 & assertive\_jj & 0 & 3498 \\
    \hline
    \caption{Scores and rankings for most extreme 30 words in component \#8} \\
\end{longtable}
\begin{longtable}[!htbp]{| rlr@{.}l |}
    \hline
    \textbf{Rank} & \textbf{Word} & \multicolumn{2}{c|}{\textbf{Score}} \\
    \hline
    \endhead
    1 & nervous\_jj & 0 & -2088 \\
    2 & fearful\_jj & 0 & -2078 \\
    3 & insecurity & 0 & -2050 \\
    4 & pessimistic\_jj & 0 & -2020 \\
    5 & irritability & 0 & -1982 \\
    6 & insecure\_jj & 0 & -1872 \\
    7 & unpredictable\_jj & 0 & -1858 \\
    8 & irritable\_jj & 0 & -1780 \\
    9 & cautious\_jj & 0 & -1740 \\
    10 & lethargy & 0 & -1739 \\
    11 & emotional\_jj & 0 & -1711 \\
    12 & adaptable\_jj & 0 & -1697 \\
    13 & suggestible\_jj & 0 & -1692 \\
    14 & predictable\_jj & 0 & -1665 \\
    15 & forgetfulness & 0 & -1643 \\
    16 & cynical\_jj & 0 & -1642 \\
    17 & optimistic\_jj & 0 & -1607 \\
    18 & pessimism & 0 & -1585 \\
    19 & imaginative\_jj & 0 & -1534 \\
    20 & philosophical\_jj & 0 & -1474 \\
    21 & confident\_jj & 0 & -1435 \\
    22 & logic & 0 & -1432 \\
    23 & unstable\_jj & 0 & -1426 \\
    24 & anxious\_jj & 0 & -1407 \\
    25 & persistent\_jj & 0 & -1404 \\
    26 & dull\_jj & 0 & -1402 \\
    27 & practical\_jj & 0 & -1369 \\
    28 & self-esteem & 0 & -1367 \\
    29 & insightful\_jj & 0 & -1360 \\
    30 & complex\_jj & 0 & -1356 \\
    401 & silence & 0 & 1311 \\
    402 & prompt\_jj & 0 & 1313 \\
    403 & unexcitable\_jj & 0 & 1323 \\
    404 & merry\_jj & 0 & 1359 \\
    405 & courtesy & 0 & 1369 \\
    406 & unsystematic\_jj & 0 & 1447 \\
    407 & crabby\_jj & 0 & 1451 \\
    408 & unassuming\_jj & 0 & 1494 \\
    409 & intelligence & 0 & 1507 \\
    410 & gruff\_jj & 0 & 1528 \\
    411 & reserve & 0 & 1531 \\
    412 & obliging\_jj & 0 & 1559 \\
    413 & uncooperative\_jj & 0 & 1569 \\
    414 & foresighted\_jj & 0 & 1590 \\
    415 & cordial\_jj & 0 & 1597 \\
    416 & curt\_jj & 0 & 1598 \\
    417 & negligence & 0 & 1606 \\
    418 & direct\_jj & 0 & 1636 \\
    419 & leniency & 0 & 1673 \\
    420 & informal\_jj & 0 & 1691 \\
    421 & negligent\_jj & 0 & 1742 \\
    422 & friendly\_jj & 0 & 1762 \\
    423 & intellectuality & 0 & 1826 \\
    424 & independence & 0 & 1844 \\
    425 & independent\_jj & 0 & 1860 \\
    426 & charitable\_jj & 0 & 1883 \\
    427 & cooperation & 0 & 2046 \\
    428 & formal\_jj & 0 & 2318 \\
    429 & diplomatic\_jj & 0 & 2528 \\
    430 & organization & 0 & 2623 \\
    \hline
    \caption{Scores and rankings for most extreme 30 words in component \#9} \\
\end{longtable}
\begin{longtable}[!htbp]{| rlr@{.}l |}
    \hline
    \textbf{Rank} & \textbf{Word} & \multicolumn{2}{c|}{\textbf{Score}} \\
    \hline
    \endhead
    1 & philosophical\_jj & 0 & -2256 \\
    2 & skeptical\_jj & 0 & -2007 \\
    3 & distrustful\_jj & 0 & -1915 \\
    4 & logic & 0 & -1881 \\
    5 & traditional\_jj & 0 & -1877 \\
    6 & touchy\_jj & 0 & -1778 \\
    7 & uncritical\_jj & 0 & -1772 \\
    8 & pessimistic\_jj & 0 & -1709 \\
    9 & morality & 0 & -1632 \\
    10 & cosmopolitan\_jj & 0 & -1571 \\
    11 & wishy-washy\_jj & 0 & -1514 \\
    12 & logical\_jj & 0 & -1510 \\
    13 & intellectuality & 0 & -1504 \\
    14 & unobservant\_jj & 0 & -1486 \\
    15 & distrust & 0 & -1478 \\
    16 & foresighted\_jj & 0 & -1471 \\
    17 & sentimental\_jj & 0 & -1470 \\
    18 & unadventurous\_jj & 0 & -1425 \\
    19 & nonconformity & 0 & -1400 \\
    20 & formal\_jj & 0 & -1349 \\
    21 & trustful\_jj & 0 & -1321 \\
    22 & gullible\_jj & 0 & -1309 \\
    23 & envious\_jj & 0 & -1308 \\
    24 & frivolity & 0 & -1277 \\
    25 & naïve\_jj & 0 & -1271 \\
    26 & envy & 0 & -1262 \\
    27 & prejudiced\_jj & 0 & -1239 \\
    28 & cynical\_jj & 0 & -1236 \\
    29 & snobbish\_jj & 0 & -1230 \\
    30 & stinginess & 0 & -1215 \\
    401 & sociable\_jj & 0 & 1144 \\
    402 & talkative\_jj & 0 & 1148 \\
    403 & unscrupulous\_jj & 0 & 1170 \\
    404 & patient\_jj & 0 & 1172 \\
    405 & impetuous\_jj & 0 & 1204 \\
    406 & explosive\_jj & 0 & 1222 \\
    407 & aimless\_jj & 0 & 1265 \\
    408 & ruthless\_jj & 0 & 1269 \\
    409 & rash\_jj & 0 & 1280 \\
    410 & persistent\_jj & 0 & 1296 \\
    411 & selfless\_jj & 0 & 1313 \\
    412 & withdrawn\_jj & 0 & 1332 \\
    413 & spontaneous\_jj & 0 & 1350 \\
    414 & unstable\_jj & 0 & 1358 \\
    415 & manipulative\_jj & 0 & 1406 \\
    416 & unpredictable\_jj & 0 & 1501 \\
    417 & forgetfulness & 0 & 1525 \\
    418 & irritable\_jj & 0 & 1596 \\
    419 & lethargic\_jj & 0 & 1690 \\
    420 & lethargy & 0 & 1709 \\
    421 & cruelty & 0 & 1762 \\
    422 & energetic\_jj & 0 & 1781 \\
    423 & tenacious\_jj & 0 & 1830 \\
    424 & careless\_jj & 0 & 1919 \\
    425 & recklessness & 0 & 1924 \\
    426 & erratic\_jj & 0 & 2110 \\
    427 & reckless\_jj & 0 & 2178 \\
    428 & negligence & 0 & 2183 \\
    429 & irritability & 0 & 2267 \\
    430 & negligent\_jj & 0 & 2408 \\
    \hline
    \caption{Scores and rankings for most extreme 30 words in component \#10} \\
\end{longtable}
\begin{longtable}[!htbp]{| rlr@{.}l |}
    \hline
    \textbf{Rank} & \textbf{Word} & \multicolumn{2}{c|}{\textbf{Score}} \\
    \hline
    \endhead
    1 & unstable\_jj & 0 & -2105 \\
    2 & systematic\_jj & 0 & -2055 \\
    3 & cruel\_jj & 0 & -2052 \\
    4 & organized\_jj & 0 & -2004 \\
    5 & emotional\_jj & 0 & -1990 \\
    6 & complex\_jj & 0 & -1915 \\
    7 & cruelty & 0 & -1792 \\
    8 & unpredictable\_jj & 0 & -1761 \\
    9 & philosophical\_jj & 0 & -1667 \\
    10 & peaceful\_jj & 0 & -1653 \\
    11 & spontaneous\_jj & 0 & -1617 \\
    12 & rebellious\_jj & 0 & -1594 \\
    13 & silent\_jj & 0 & -1586 \\
    14 & bitter\_jj & 0 & -1571 \\
    15 & formal\_jj & 0 & -1559 \\
    16 & instability & 0 & -1546 \\
    17 & inventive\_jj & 0 & -1527 \\
    18 & unaggressive\_jj & 0 & -1513 \\
    19 & harsh\_jj & 0 & -1474 \\
    20 & informal\_jj & 0 & -1474 \\
    21 & fear & 0 & -1470 \\
    22 & deliberate\_jj & 0 & -1389 \\
    23 & secretive\_jj & 0 & -1383 \\
    24 & meditative\_jj & 0 & -1362 \\
    25 & autonomous\_jj & 0 & -1334 \\
    26 & silence & 0 & -1304 \\
    27 & prejudice & 0 & -1257 \\
    28 & volatile\_jj & 0 & -1217 \\
    29 & melancholic\_jj & 0 & -1215 \\
    30 & devious\_jj & 0 & -1205 \\
    401 & smart\_jj & 0 & 1180 \\
    402 & pessimistic\_jj & 0 & 1192 \\
    403 & assured\_jj & 0 & 1226 \\
    404 & confident\_jj & 0 & 1247 \\
    405 & generosity & 0 & 1272 \\
    406 & stinginess & 0 & 1277 \\
    407 & smug\_jj & 0 & 1313 \\
    408 & persistence & 0 & 1316 \\
    409 & economical\_jj & 0 & 1348 \\
    410 & optimism & 0 & 1366 \\
    411 & envious\_jj & 0 & 1395 \\
    412 & defensive\_jj & 0 & 1409 \\
    413 & unambitious\_jj & 0 & 1426 \\
    414 & patient\_jj & 0 & 1432 \\
    415 & efficient\_jj & 0 & 1438 \\
    416 & modest\_jj & 0 & 1452 \\
    417 & predictability & 0 & 1467 \\
    418 & lethargic\_jj & 0 & 1537 \\
    419 & decisiveness & 0 & 1548 \\
    420 & punctual\_jj & 0 & 1575 \\
    421 & consistent\_jj & 0 & 1632 \\
    422 & flexibility & 0 & 1667 \\
    423 & dependable\_jj & 0 & 1816 \\
    424 & reliable\_jj & 0 & 1816 \\
    425 & miserly\_jj & 0 & 1925 \\
    426 & thrifty\_jj & 0 & 2072 \\
    427 & stingy\_jj & 0 & 2309 \\
    428 & dependability & 0 & 2636 \\
    429 & efficiency & 0 & 2895 \\
    430 & punctuality & 0 & 2930 \\
    \hline
    \caption{Scores and rankings for most extreme 30 words in component \#11} \\
\end{longtable}
\begin{longtable}[!htbp]{| rlr@{.}l |}
    \hline
    \textbf{Rank} & \textbf{Word} & \multicolumn{2}{c|}{\textbf{Score}} \\
    \hline
    \endhead
    1 & assertion & 0 & -2214 \\
    2 & insight & 0 & -2122 \\
    3 & intelligence & 0 & -2107 \\
    4 & skeptical\_jj & 0 & -2057 \\
    5 & suspicious\_jj & 0 & -1881 \\
    6 & explosive\_jj & 0 & -1862 \\
    7 & insightful\_jj & 0 & -1845 \\
    8 & analytical\_jj & 0 & -1747 \\
    9 & enthusiastic\_jj & 0 & -1693 \\
    10 & nervous\_jj & 0 & -1639 \\
    11 & jealous\_jj & 0 & -1624 \\
    12 & verbal\_jj & 0 & -1596 \\
    13 & unkind\_jj & 0 & -1549 \\
    14 & unreliable\_jj & 0 & -1511 \\
    15 & depth & 0 & -1494 \\
    16 & anxious\_jj & 0 & -1454 \\
    17 & perceptive\_jj & 0 & -1424 \\
    18 & precision & 0 & -1422 \\
    19 & passionless\_jj & 0 & -1400 \\
    20 & envious\_jj & 0 & -1366 \\
    21 & precise\_jj & 0 & -1350 \\
    22 & sly\_jj & 0 & -1345 \\
    23 & shy\_jj & 0 & -1338 \\
    24 & unexcitable\_jj & 0 & -1330 \\
    25 & verbose\_jj & 0 & -1283 \\
    26 & sophistication & 0 & -1257 \\
    27 & careful\_jj & 0 & -1217 \\
    28 & frank\_jj & 0 & -1216 \\
    29 & animation & 0 & -1214 \\
    30 & intellectuality & 0 & -1214 \\
    401 & generous\_jj & 0 & 1134 \\
    402 & humble\_jj & 0 & 1135 \\
    403 & stubborn\_jj & 0 & 1155 \\
    404 & principled\_jj & 0 & 1167 \\
    405 & restrained\_jj & 0 & 1207 \\
    406 & miserly\_jj & 0 & 1228 \\
    407 & brave\_jj & 0 & 1248 \\
    408 & carefree\_jj & 0 & 1268 \\
    409 & reserved\_jj & 0 & 1302 \\
    410 & aimlessness & 0 & 1344 \\
    411 & cosmopolitan\_jj & 0 & 1347 \\
    412 & foolhardy\_jj & 0 & 1350 \\
    413 & impractical\_jj & 0 & 1449 \\
    414 & docile\_jj & 0 & 1467 \\
    415 & flexible\_jj & 0 & 1475 \\
    416 & nonconformity & 0 & 1481 \\
    417 & traditional\_jj & 0 & 1504 \\
    418 & selfless\_jj & 0 & 1512 \\
    419 & accommodating\_jj & 0 & 1540 \\
    420 & benevolent\_jj & 0 & 1567 \\
    421 & dignity & 0 & 1581 \\
    422 & unrestrained\_jj & 0 & 1586 \\
    423 & selfishness & 0 & 1619 \\
    424 & quiet\_jj & 0 & 1636 \\
    425 & modest\_jj & 0 & 1641 \\
    426 & pleasant\_jj & 0 & 1676 \\
    427 & individualistic\_jj & 0 & 1931 \\
    428 & thrifty\_jj & 0 & 2161 \\
    429 & peaceful\_jj & 0 & 2203 \\
    430 & dignified\_jj & 0 & 2275 \\
    \hline
    \caption{Scores and rankings for most extreme 30 words in component \#12} \\
\end{longtable}
\begin{longtable}[!htbp]{| rlr@{.}l |}
    \hline
    \textbf{Rank} & \textbf{Word} & \multicolumn{2}{c|}{\textbf{Score}} \\
    \hline
    \endhead
    1 & understanding\_jj & 0 & -1824 \\
    2 & brave\_jj & 0 & -1751 \\
    3 & vigorous\_jj & 0 & -1721 \\
    4 & reserve & 0 & -1634 \\
    5 & animation & 0 & -1625 \\
    6 & nonconformity & 0 & -1568 \\
    7 & fretful\_jj & 0 & -1501 \\
    8 & quiet & 0 & -1475 \\
    9 & artistic\_jj & 0 & -1426 \\
    10 & talkativeness & 0 & -1402 \\
    11 & meditative\_jj & 0 & -1347 \\
    12 & negligent\_jj & 0 & -1325 \\
    13 & suggestible\_jj & 0 & -1325 \\
    14 & daring & 0 & -1324 \\
    15 & unsociable\_jj & 0 & -1316 \\
    16 & inventive\_jj & 0 & -1277 \\
    17 & inhibition & 0 & -1267 \\
    18 & optimistic\_jj & 0 & -1265 \\
    19 & rash\_jj & 0 & -1262 \\
    20 & bold\_jj & 0 & -1236 \\
    21 & unaggressive\_jj & 0 & -1233 \\
    22 & absent-minded\_jj & 0 & -1228 \\
    23 & adventurous\_jj & 0 & -1220 \\
    24 & bullheaded\_jj & 0 & -1189 \\
    25 & rambunctious\_jj & 0 & -1189 \\
    26 & charitable\_jj & 0 & -1185 \\
    27 & conscientious\_jj & 0 & -1165 \\
    28 & morose\_jj & 0 & -1133 \\
    29 & vain\_jj & 0 & -1129 \\
    30 & skeptical\_jj & 0 & -1121 \\
    401 & inconsistent\_jj & 0 & 1137 \\
    402 & unpredictable\_jj & 0 & 1152 \\
    403 & earthy\_jj & 0 & 1159 \\
    404 & insecure\_jj & 0 & 1161 \\
    405 & generous\_jj & 0 & 1188 \\
    406 & shallow\_jj & 0 & 1197 \\
    407 & predictable\_jj & 0 & 1275 \\
    408 & easygoing\_jj & 0 & 1309 \\
    409 & deceit & 0 & 1334 \\
    410 & diplomatic\_jj & 0 & 1364 \\
    411 & unstable\_jj & 0 & 1402 \\
    412 & aloofness & 0 & 1420 \\
    413 & predictability & 0 & 1423 \\
    414 & bitter\_jj & 0 & 1426 \\
    415 & warm\_jj & 0 & 1430 \\
    416 & undependable\_jj & 0 & 1434 \\
    417 & envy & 0 & 1435 \\
    418 & humor & 0 & 1449 \\
    419 & unfriendliness & 0 & 1515 \\
    420 & friendly\_jj & 0 & 1578 \\
    421 & volatile\_jj & 0 & 1696 \\
    422 & cordial\_jj & 0 & 1729 \\
    423 & inefficient\_jj & 0 & 1738 \\
    424 & insecurity & 0 & 1767 \\
    425 & unreliable\_jj & 0 & 1811 \\
    426 & warmth & 0 & 1850 \\
    427 & reliable\_jj & 0 & 2020 \\
    428 & dependable\_jj & 0 & 2034 \\
    429 & instability & 0 & 2066 \\
    430 & distrust & 0 & 2206 \\
    \hline
    \caption{Scores and rankings for most extreme 30 words in component \#13} \\
\end{longtable}
\begin{longtable}[!htbp]{| rlr@{.}l |}
    \hline
    \textbf{Rank} & \textbf{Word} & \multicolumn{2}{c|}{\textbf{Score}} \\
    \hline
    \endhead
    1 & dominant\_jj & 0 & -1943 \\
    2 & defensive\_jj & 0 & -1728 \\
    3 & steady\_jj & 0 & -1700 \\
    4 & stubborn\_jj & 0 & -1658 \\
    5 & deep\_jj & 0 & -1598 \\
    6 & gregariousness & 0 & -1547 \\
    7 & tenacious\_jj & 0 & -1530 \\
    8 & decisive\_jj & 0 & -1440 \\
    9 & brave\_jj & 0 & -1386 \\
    10 & principled\_jj & 0 & -1361 \\
    11 & prejudiced\_jj & 0 & -1345 \\
    12 & reliable\_jj & 0 & -1343 \\
    13 & shallow\_jj & 0 & -1311 \\
    14 & nonconforming\_jj & 0 & -1286 \\
    15 & ruthless\_jj & 0 & -1282 \\
    16 & vigorous\_jj & 0 & -1281 \\
    17 & submissive\_jj & 0 & -1248 \\
    18 & depth & 0 & -1241 \\
    19 & moral\_jj & 0 & -1236 \\
    20 & natural\_jj & 0 & -1226 \\
    21 & courageous\_jj & 0 & -1212 \\
    22 & consistent\_jj & 0 & -1206 \\
    23 & foresighted\_jj & 0 & -1179 \\
    24 & active\_jj & 0 & -1154 \\
    25 & vain\_jj & 0 & -1146 \\
    26 & dependable\_jj & 0 & -1122 \\
    27 & demonstrative\_jj & 0 & -1097 \\
    28 & perceptive\_jj & 0 & -1092 \\
    29 & unintelligent\_jj & 0 & -1086 \\
    30 & concise\_jj & 0 & -1085 \\
    401 & generosity & 0 & 1031 \\
    402 & uncooperative\_jj & 0 & 1042 \\
    403 & reckless\_jj & 0 & 1058 \\
    404 & playfulness & 0 & 1079 \\
    405 & spontaneous\_jj & 0 & 1082 \\
    406 & ambitious\_jj & 0 & 1119 \\
    407 & curiosity & 0 & 1120 \\
    408 & unrestrained\_jj & 0 & 1147 \\
    409 & restrained\_jj & 0 & 1169 \\
    410 & inventive\_jj & 0 & 1195 \\
    411 & pessimistic\_jj & 0 & 1213 \\
    412 & somber\_jj & 0 & 1217 \\
    413 & optimistic\_jj & 0 & 1234 \\
    414 & enterprising\_jj & 0 & 1237 \\
    415 & imaginative\_jj & 0 & 1257 \\
    416 & adventurous\_jj & 0 & 1269 \\
    417 & spontaneity & 0 & 1285 \\
    418 & carefree\_jj & 0 & 1291 \\
    419 & pessimism & 0 & 1351 \\
    420 & unfriendly\_jj & 0 & 1390 \\
    421 & optimism & 0 & 1450 \\
    422 & unscrupulous\_jj & 0 & 1469 \\
    423 & frivolity & 0 & 1665 \\
    424 & leniency & 0 & 1671 \\
    425 & casual\_jj & 0 & 1717 \\
    426 & impractical\_jj & 0 & 1876 \\
    427 & lenient\_jj & 0 & 1923 \\
    428 & frivolous\_jj & 0 & 2149 \\
    429 & informal\_jj & 0 & 2329 \\
    430 & extravagant\_jj & 0 & 3025 \\
    \hline
    \caption{Scores and rankings for most extreme 30 words in component \#14} \\
\end{longtable}
\begin{longtable}[!htbp]{| rlr@{.}l |}
    \hline
    \textbf{Rank} & \textbf{Word} & \multicolumn{2}{c|}{\textbf{Score}} \\
    \hline
    \endhead
    1 & formal\_jj & 0 & -1823 \\
    2 & ruthless\_jj & 0 & -1634 \\
    3 & sophisticated\_jj & 0 & -1547 \\
    4 & sloth & 0 & -1522 \\
    5 & direct\_jj & 0 & -1432 \\
    6 & casual\_jj & 0 & -1430 \\
    7 & meticulous\_jj & 0 & -1363 \\
    8 & cooperative\_jj & 0 & -1340 \\
    9 & flexibility & 0 & -1284 \\
    10 & systematic\_jj & 0 & -1282 \\
    11 & distrust & 0 & -1234 \\
    12 & flexible\_jj & 0 & -1202 \\
    13 & passive\_jj & 0 & -1168 \\
    14 & fearful\_jj & 0 & -1148 \\
    15 & traditional\_jj & 0 & -1146 \\
    16 & rambunctious\_jj & 0 & -1125 \\
    17 & mischievous\_jj & 0 & -1120 \\
    18 & cynical\_jj & 0 & -1115 \\
    19 & intrusive\_jj & 0 & -1111 \\
    20 & fastidious\_jj & 0 & -1101 \\
    21 & thorough\_jj & 0 & -1099 \\
    22 & crafty\_jj & 0 & -1087 \\
    23 & crabby\_jj & 0 & -1075 \\
    24 & cooperation & 0 & -1064 \\
    25 & shy\_jj & 0 & -1053 \\
    26 & cautious\_jj & 0 & -1049 \\
    27 & nervous\_jj & 0 & -1035 \\
    28 & jovial\_jj & 0 & -1016 \\
    29 & docile\_jj & 0 & -1008 \\
    30 & envy & 0 & -1007 \\
    401 & brave\_jj & 0 & 1051 \\
    402 & cruel\_jj & 0 & 1060 \\
    403 & earthy\_jj & 0 & 1069 \\
    404 & dignity & 0 & 1075 \\
    405 & imperturbable\_jj & 0 & 1083 \\
    406 & assertion & 0 & 1109 \\
    407 & ungracious\_jj & 0 & 1120 \\
    408 & organization & 0 & 1154 \\
    409 & warm\_jj & 0 & 1173 \\
    410 & rash\_jj & 0 & 1177 \\
    411 & economical\_jj & 0 & 1204 \\
    412 & impudent\_jj & 0 & 1211 \\
    413 & emotional\_jj & 0 & 1224 \\
    414 & insight & 0 & 1226 \\
    415 & explosive\_jj & 0 & 1275 \\
    416 & unstable\_jj & 0 & 1294 \\
    417 & naïve\_jj & 0 & 1301 \\
    418 & quiet & 0 & 1349 \\
    419 & insensitive\_jj & 0 & 1354 \\
    420 & unreliable\_jj & 0 & 1377 \\
    421 & autonomous\_jj & 0 & 1414 \\
    422 & foolhardy\_jj & 0 & 1440 \\
    423 & cold\_jj & 0 & 1532 \\
    424 & orderly\_jj & 0 & 1561 \\
    425 & independence & 0 & 1610 \\
    426 & undemanding\_jj & 0 & 1614 \\
    427 & silence & 0 & 1637 \\
    428 & independent\_jj & 0 & 1669 \\
    429 & insightful\_jj & 0 & 1911 \\
    430 & unkind\_jj & 0 & 2073 \\
    \hline
    \caption{Scores and rankings for most extreme 30 words in component \#15} \\
\end{longtable}


\section{Combined 101 and 438 and 2797 word list}
\label{app:rankedwordlists:2797and438and101words}
\subsection{Unnormalized PCA}
\label{app:rankedwordlists:2797and438and101words:unnormalized}
\begin{longtable}[!htbp]{| rlr@{.}l |}
    \hline
    \textbf{Rank} & \textbf{Word} & \multicolumn{2}{c|}{\textbf{Score}} \\
    \hline
    \endhead
    1 & stringent\_jj & -1 & -6371 \\
    2 & indirect\_jj & -1 & -5523 \\
    3 & beneficial\_jj & -1 & -5440 \\
    4 & contentious\_jj & -1 & -4951 \\
    5 & dependent\_jj & -1 & -4820 \\
    6 & discretionary\_jj & -1 & -4573 \\
    7 & prudent\_jj & -1 & -4560 \\
    8 & volatile\_jj & -1 & -4485 \\
    9 & lax\_jj & -1 & -4454 \\
    10 & reasonable\_jj & -1 & -4444 \\
    11 & corrective\_jj & -1 & -4439 \\
    12 & autonomous\_jj & -1 & -4426 \\
    13 & affected\_jj & -1 & -4258 \\
    14 & exclusive\_jj & -1 & -4250 \\
    15 & instability & -1 & -4246 \\
    16 & drastic\_jj & -1 & -4208 \\
    17 & confidential\_jj & -1 & -4110 \\
    18 & fraudulent\_jj & -1 & -4054 \\
    19 & indefinite\_jj & -1 & -3889 \\
    20 & systematic\_jj & -1 & -3847 \\
    21 & stable\_jj & -1 & -3519 \\
    22 & unfair\_jj & -1 & -3476 \\
    23 & negligence & -1 & -3456 \\
    24 & critical\_jj & -1 & -3391 \\
    25 & cooperation & -1 & -3299 \\
    26 & rigorous\_jj & -1 & -3265 \\
    27 & forward-looking\_jj & -1 & -3230 \\
    28 & severe\_jj & -1 & -3200 \\
    29 & volatility & -1 & -3198 \\
    30 & inaccurate\_jj & -1 & -3130 \\
    1949 & suave\_jj & 1 & 2103 \\
    1950 & poised\_jj & 1 & 2108 \\
    1951 & bumpkin & 1 & 2167 \\
    1952 & pert\_jj & 1 & 2173 \\
    1953 & diffident\_jj & 1 & 2191 \\
    1954 & roguish\_jj & 1 & 2237 \\
    1955 & improviser & 1 & 2239 \\
    1956 & shrewish\_jj & 1 & 2288 \\
    1957 & easygoing\_jj & 1 & 2321 \\
    1958 & beatific\_jj & 1 & 2513 \\
    1959 & misanthropic\_jj & 1 & 2558 \\
    1960 & flirtatious\_jj & 1 & 2688 \\
    1961 & debonair\_jj & 1 & 2729 \\
    1962 & puckish\_jj & 1 & 2798 \\
    1963 & bubbly\_jj & 1 & 2827 \\
    1964 & boyish\_jj & 1 & 2927 \\
    1965 & witty\_jj & 1 & 3075 \\
    1966 & stoic & 1 & 3369 \\
    1967 & coquettish\_jj & 1 & 3413 \\
    1968 & coquette & 1 & 3438 \\
    1969 & tomboy & 1 & 3474 \\
    1970 & impish\_jj & 1 & 3521 \\
    1971 & sassy\_jj & 1 & 3632 \\
    1972 & sardonic\_jj & 1 & 3986 \\
    1973 & guileless\_jj & 1 & 4009 \\
    1974 & girlish\_jj & 1 & 4112 \\
    1975 & go-getter & 1 & 4133 \\
    1976 & self-possessed\_jj & 1 & 4325 \\
    1977 & vivacious\_jj & 1 & 5587 \\
    1978 & mousy\_jj & 1 & 6154 \\
    \hline
    \caption{Scores and rankings for most extreme 30 words in component \#1} \\
\end{longtable}
\begin{longtable}[!htbp]{| rlr@{.}l |}
    \hline
    \textbf{Rank} & \textbf{Word} & \multicolumn{2}{c|}{\textbf{Score}} \\
    \hline
    \endhead
    1 & defamatory\_jj & -1 & -9427 \\
    2 & bigoted\_jj & -1 & -8599 \\
    3 & untruthful\_jj & -1 & -7668 \\
    4 & deceitful\_jj & -1 & -7521 \\
    5 & cowardly\_jj & -1 & -7378 \\
    6 & inhuman\_jj & -1 & -7159 \\
    7 & selfish\_jj & -1 & -6974 \\
    8 & slanderous\_jj & -1 & -6913 \\
    9 & irresponsible\_jj & -1 & -6111 \\
    10 & hypocritical\_jj & -1 & -5953 \\
    11 & unethical\_jj & -1 & -5574 \\
    12 & thoughtless\_jj & -1 & -5427 \\
    13 & vindictive\_jj & -1 & -5298 \\
    14 & dishonest\_jj & -1 & -5243 \\
    15 & disrespectful\_jj & -1 & -4964 \\
    16 & mendacious\_jj & -1 & -4843 \\
    17 & callous\_jj & -1 & -4659 \\
    18 & inconsiderate\_jj & -1 & -4640 \\
    19 & godless\_jj & -1 & -4606 \\
    20 & polemic\_jj & -1 & -4555 \\
    21 & insensitive\_jj & -1 & -4409 \\
    22 & callousness & -1 & -4404 \\
    23 & stupidity & -1 & -4269 \\
    24 & unfeeling\_jj & -1 & -4213 \\
    25 & selfishness & -1 & -4155 \\
    26 & narrow-minded\_jj & -1 & -4142 \\
    27 & inhumane\_jj & -1 & -4120 \\
    28 & intolerant\_jj & -1 & -4095 \\
    29 & ignorant\_jj & -1 & -3814 \\
    30 & unintelligent\_jj & -1 & -3754 \\
    1949 & breezy\_jj & 1 & 464 \\
    1950 & serene\_jj & 1 & 541 \\
    1951 & pleasant\_jj & 1 & 657 \\
    1952 & ethereal\_jj & 1 & 660 \\
    1953 & bubbly\_jj & 1 & 682 \\
    1954 & dainty\_jj & 1 & 706 \\
    1955 & cultured\_jj & 1 & 728 \\
    1956 & brisk\_jj & 1 & 835 \\
    1957 & calm\_jj & 1 & 892 \\
    1958 & rugged\_jj & 1 & 893 \\
    1959 & bright\_jj & 1 & 973 \\
    1960 & sensual\_jj & 1 & 989 \\
    1961 & intuitive\_jj & 1 & 1415 \\
    1962 & easygoing\_jj & 1 & 1649 \\
    1963 & sultry\_jj & 1 & 1707 \\
    1964 & lively\_jj & 1 & 1726 \\
    1965 & versatile\_jj & 1 & 1816 \\
    1966 & sunny\_jj & 1 & 2079 \\
    1967 & buttery\_jj & 1 & 2087 \\
    1968 & polished\_jj & 1 & 2122 \\
    1969 & airy\_jj & 1 & 2549 \\
    1970 & luxurious\_jj & 1 & 2671 \\
    1971 & vivacious\_jj & 1 & 2705 \\
    1972 & graceful\_jj & 1 & 2742 \\
    1973 & sociable\_jj & 1 & 2784 \\
    1974 & sparkling\_jj & 1 & 2934 \\
    1975 & savant & 1 & 3155 \\
    1976 & vibrant\_jj & 1 & 3458 \\
    1977 & elegant\_jj & 1 & 4204 \\
    1978 & warm\_jj & 1 & 4993 \\
    \hline
    \caption{Scores and rankings for most extreme 30 words in component \#2} \\
\end{longtable}
\begin{longtable}[!htbp]{| rlr@{.}l |}
    \hline
    \textbf{Rank} & \textbf{Word} & \multicolumn{2}{c|}{\textbf{Score}} \\
    \hline
    \endhead
    1 & considerate\_jj & -2 & -5278 \\
    2 & sociable\_jj & -2 & -1848 \\
    3 & courteous\_jj & -2 & -159 \\
    4 & trustworthy\_jj & -1 & -9283 \\
    5 & articulate\_jj & -1 & -8506 \\
    6 & open-minded\_jj & -1 & -7273 \\
    7 & kind\_jj & -1 & -6687 \\
    8 & approachable\_jj & -1 & -6024 \\
    9 & easygoing\_jj & -1 & -5805 \\
    10 & respectful\_jj & -1 & -5740 \\
    11 & thoughtful\_jj & -1 & -5673 \\
    12 & self-confident\_jj & -1 & -5450 \\
    13 & easy-going\_jj & -1 & -5111 \\
    14 & honest\_jj & -1 & -4972 \\
    15 & level-headed\_jj & -1 & -4828 \\
    16 & circumspect\_jj & -1 & -4381 \\
    17 & talkative\_jj & -1 & -4233 \\
    18 & down-to-earth\_jj & -1 & -4034 \\
    19 & intelligent\_jj & -1 & -3699 \\
    20 & sincere\_jj & -1 & -3532 \\
    21 & cordial\_jj & -1 & -3515 \\
    22 & accommodating\_jj & -1 & -3365 \\
    23 & pragmatic\_jj & -1 & -3359 \\
    24 & opinionated\_jj & -1 & -3256 \\
    25 & deferential\_jj & -1 & -3168 \\
    26 & knowledgeable\_jj & -1 & -3158 \\
    27 & perceptive\_jj & -1 & -3001 \\
    28 & fair-minded\_jj & -1 & -2766 \\
    29 & gregarious\_jj & -1 & -2752 \\
    30 & affectionate\_jj & -1 & -2617 \\
    1949 & fossil & 0 & 9073 \\
    1950 & ductile\_jj & 0 & 9093 \\
    1951 & double-faced\_jj & 0 & 9144 \\
    1952 & unmanly\_jj & 0 & 9163 \\
    1953 & driftless\_jj & 0 & 9212 \\
    1954 & gooey\_jj & 0 & 9334 \\
    1955 & magnetic\_jj & 0 & 9414 \\
    1956 & high-hat & 0 & 9441 \\
    1957 & bristly\_jj & 0 & 9491 \\
    1958 & gourmand & 0 & 9631 \\
    1959 & lethargy & 0 & 9637 \\
    1960 & wildcat\_jj & 0 & 9712 \\
    1961 & catlike\_jj & 0 & 9777 \\
    1962 & croaking & 0 & 9783 \\
    1963 & clam & 0 & 9847 \\
    1964 & yellow\_jj & 0 & 9924 \\
    1965 & acrid\_jj & 1 & 227 \\
    1966 & aplastic\_jj & 1 & 246 \\
    1967 & hard-shell\_jj & 1 & 392 \\
    1968 & baneful\_jj & 1 & 650 \\
    1969 & vinegary\_jj & 1 & 976 \\
    1970 & low-pressure\_jj & 1 & 1047 \\
    1971 & boneless\_jj & 1 & 1354 \\
    1972 & broiler & 1 & 1513 \\
    1973 & soft-shelled\_jj & 1 & 1693 \\
    1974 & bendable\_jj & 1 & 1726 \\
    1975 & pixy & 1 & 2788 \\
    1976 & butterfly & 1 & 2801 \\
    1977 & comforter & 1 & 3416 \\
    1978 & gingery\_jj & 1 & 3593 \\
    \hline
    \caption{Scores and rankings for most extreme 30 words in component \#3} \\
\end{longtable}
\begin{longtable}[!htbp]{| rlr@{.}l |}
    \hline
    \textbf{Rank} & \textbf{Word} & \multicolumn{2}{c|}{\textbf{Score}} \\
    \hline
    \endhead
    1 & sociable\_jj & -1 & -2300 \\
    2 & has-been & -1 & -2081 \\
    3 & stay-at-home\_jj & -1 & -1910 \\
    4 & lucky\_jj & -1 & -1897 \\
    5 & cheater & -1 & -1634 \\
    6 & gambler & -1 & -1445 \\
    7 & brat & -1 & -1144 \\
    8 & alcoholic & -1 & -955 \\
    9 & clown & -1 & -873 \\
    10 & fan & -1 & -637 \\
    11 & loyal\_jj & -1 & -584 \\
    12 & bulldog & -1 & -391 \\
    13 & geezer & -1 & -295 \\
    14 & go-getter & -1 & -124 \\
    15 & quitter & -1 & -124 \\
    16 & angler & -1 & -115 \\
    17 & intrepid\_jj & 0 & -9839 \\
    18 & teetotaler & 0 & -9765 \\
    19 & samaritan & 0 & -9595 \\
    20 & avid\_jj & 0 & -9587 \\
    21 & bitch & 0 & -9255 \\
    22 & gregarious\_jj & 0 & -9204 \\
    23 & addicted\_jj & 0 & -9202 \\
    24 & cranky\_jj & 0 & -9168 \\
    25 & objector & 0 & -9132 \\
    26 & devout\_jj & 0 & -9121 \\
    27 & sissy & 0 & -8956 \\
    28 & illiterate\_jj & 0 & -8942 \\
    29 & jealous\_jj & 0 & -8931 \\
    30 & kind-hearted\_jj & 0 & -8806 \\
    1949 & mechanistic\_jj & 0 & 9790 \\
    1950 & intuitive\_jj & 0 & 9811 \\
    1951 & abstract\_jj & 0 & 9860 \\
    1952 & expressive\_jj & 0 & 9869 \\
    1953 & oratorical\_jj & 0 & 9971 \\
    1954 & subjective\_jj & 1 & 29 \\
    1955 & callousness & 1 & 56 \\
    1956 & shallowness & 1 & 134 \\
    1957 & didactic\_jj & 1 & 157 \\
    1958 & unflagging\_jj & 1 & 202 \\
    1959 & dissonant\_jj & 1 & 210 \\
    1960 & declamatory\_jj & 1 & 337 \\
    1961 & incisive\_jj & 1 & 439 \\
    1962 & oblique\_jj & 1 & 512 \\
    1963 & deterministic\_jj & 1 & 648 \\
    1964 & lyrical\_jj & 1 & 761 \\
    1965 & earthiness & 1 & 817 \\
    1966 & abstruse\_jj & 1 & 867 \\
    1967 & discursive\_jj & 1 & 1196 \\
    1968 & unfailing\_jj & 1 & 1219 \\
    1969 & decisiveness & 1 & 1396 \\
    1970 & naturalness & 1 & 1785 \\
    1971 & playfulness & 1 & 2116 \\
    1972 & meditative\_jj & 1 & 2155 \\
    1973 & poetic\_jj & 1 & 2295 \\
    1974 & limpid\_jj & 1 & 2481 \\
    1975 & spontaneity & 1 & 2578 \\
    1976 & candor & 1 & 2671 \\
    1977 & unvarying\_jj & 1 & 3889 \\
    1978 & expressiveness & 1 & 4830 \\
    \hline
    \caption{Scores and rankings for most extreme 30 words in component \#4} \\
\end{longtable}
\begin{longtable}[!htbp]{| rlr@{.}l |}
    \hline
    \textbf{Rank} & \textbf{Word} & \multicolumn{2}{c|}{\textbf{Score}} \\
    \hline
    \endhead
    1 & unswerving\_jj & -1 & -5249 \\
    2 & unshakable\_jj & -1 & -4240 \\
    3 & unbending\_jj & -1 & -3826 \\
    4 & ultraconservative\_jj & -1 & -3378 \\
    5 & steadfast\_jj & -1 & -3045 \\
    6 & staunch\_jj & -1 & -2962 \\
    7 & unflagging\_jj & -1 & -2945 \\
    8 & untiring\_jj & -1 & -2726 \\
    9 & unfailing\_jj & -1 & -2393 \\
    10 & objector & -1 & -2222 \\
    11 & courage & -1 & -2025 \\
    12 & secular\_jj & -1 & -1971 \\
    13 & unquestioning\_jj & -1 & -1963 \\
    14 & firebrand & -1 & -1958 \\
    15 & incorruptible\_jj & -1 & -1803 \\
    16 & tireless\_jj & -1 & -1653 \\
    17 & dissident & -1 & -1605 \\
    18 & fervent\_jj & -1 & -910 \\
    19 & visionary & -1 & -724 \\
    20 & intellectual & -1 & -646 \\
    21 & theistic\_jj & -1 & -551 \\
    22 & devout\_jj & -1 & -480 \\
    23 & militant & -1 & -405 \\
    24 & stalwart & -1 & -396 \\
    25 & generosity & -1 & -225 \\
    26 & selfless\_jj & -1 & -163 \\
    27 & religious\_jj & -1 & -135 \\
    28 & extremist & -1 & -100 \\
    29 & indefatigable\_jj & -1 & -24 \\
    30 & outspoken\_jj & 0 & -9877 \\
    1949 & tasteful\_jj & 0 & 8831 \\
    1950 & inaccurate\_jj & 0 & 8971 \\
    1951 & sexy\_jj & 0 & 9029 \\
    1952 & unrefined\_jj & 0 & 9049 \\
    1953 & contrived\_jj & 0 & 9152 \\
    1954 & flammable\_jj & 0 & 9225 \\
    1955 & libidinous\_jj & 0 & 9333 \\
    1956 & oily\_jj & 0 & 9354 \\
    1957 & raunchy\_jj & 0 & 9469 \\
    1958 & inconsiderate\_jj & 0 & 9619 \\
    1959 & loud\_jj & 0 & 9741 \\
    1960 & clingy\_jj & 0 & 9764 \\
    1961 & frisky\_jj & 0 & 9820 \\
    1962 & obtrusive\_jj & 0 & 9962 \\
    1963 & irritability & 1 & 403 \\
    1964 & picky\_jj & 1 & 492 \\
    1965 & gooey\_jj & 1 & 590 \\
    1966 & buttery\_jj & 1 & 652 \\
    1967 & lazy\_jj & 1 & 693 \\
    1968 & disorderly\_jj & 1 & 863 \\
    1969 & obscene\_jj & 1 & 903 \\
    1970 & fussy\_jj & 1 & 933 \\
    1971 & rude\_jj & 1 & 944 \\
    1972 & risqué\_jj & 1 & 1111 \\
    1973 & abusive\_jj & 1 & 1377 \\
    1974 & tasteless\_jj & 1 & 2634 \\
    1975 & lascivious\_jj & 1 & 2990 \\
    1976 & irritable\_jj & 1 & 3685 \\
    1977 & lewd\_jj & 1 & 4413 \\
    1978 & defamatory\_jj & 1 & 6287 \\
    \hline
    \caption{Scores and rankings for most extreme 30 words in component \#5} \\
\end{longtable}
\begin{longtable}[!htbp]{| rlr@{.}l |}
    \hline
    \textbf{Rank} & \textbf{Word} & \multicolumn{2}{c|}{\textbf{Score}} \\
    \hline
    \endhead
    1 & deterministic\_jj & -1 & -3063 \\
    2 & self-reliant\_jj & -1 & -2900 \\
    3 & sociable\_jj & -1 & -2892 \\
    4 & educated\_jj & -1 & -2583 \\
    5 & adaptable\_jj & -1 & -2474 \\
    6 & considerate\_jj & -1 & -2130 \\
    7 & modifiable\_jj & -1 & -1233 \\
    8 & intelligent\_jj & -1 & -1225 \\
    9 & open-minded\_jj & -1 & -1022 \\
    10 & self-sufficient\_jj & -1 & -994 \\
    11 & materialistic\_jj & -1 & -690 \\
    12 & mutable\_jj & -1 & -533 \\
    13 & closed-minded\_jj & -1 & -442 \\
    14 & sedentary\_jj & -1 & -299 \\
    15 & distractible\_jj & 0 & -9948 \\
    16 & unchangeable\_jj & 0 & -9840 \\
    17 & choosy\_jj & 0 & -9772 \\
    18 & equitable\_jj & 0 & -9698 \\
    19 & conformist\_jj & 0 & -9590 \\
    20 & theistic\_jj & 0 & -9529 \\
    21 & efficient\_jj & 0 & -9418 \\
    22 & perspicuous\_jj & 0 & -9395 \\
    23 & literate\_jj & 0 & -9343 \\
    24 & intuitive\_jj & 0 & -9335 \\
    25 & individualistic\_jj & 0 & -9260 \\
    26 & celibate & 0 & -9240 \\
    27 & altruistic\_jj & 0 & -9226 \\
    28 & variant\_jj & 0 & -9181 \\
    29 & cultured\_jj & 0 & -9097 \\
    30 & trustworthy\_jj & 0 & -9044 \\
    1949 & curt\_jj & 0 & 9647 \\
    1950 & disorderly\_jj & 0 & 9801 \\
    1951 & belligerent\_jj & 0 & 9806 \\
    1952 & lukewarm\_jj & 0 & 9820 \\
    1953 & belligerence & 0 & 9885 \\
    1954 & bitter\_jj & 0 & 9931 \\
    1955 & strident\_jj & 0 & 9950 \\
    1956 & combative\_jj & 0 & 9959 \\
    1957 & raucous\_jj & 0 & 9994 \\
    1958 & fiery\_jj & 1 & 72 \\
    1959 & rowdy\_jj & 1 & 76 \\
    1960 & torrid\_jj & 1 & 83 \\
    1961 & fierce\_jj & 1 & 222 \\
    1962 & ferocious\_jj & 1 & 424 \\
    1963 & verbal\_jj & 1 & 448 \\
    1964 & stormy\_jj & 1 & 714 \\
    1965 & congratulatory\_jj & 1 & 1067 \\
    1966 & bellicose\_jj & 1 & 1083 \\
    1967 & caustic\_jj & 1 & 1171 \\
    1968 & lewd\_jj & 1 & 1350 \\
    1969 & rancorous\_jj & 1 & 1580 \\
    1970 & stern\_jj & 1 & 1745 \\
    1971 & derisive\_jj & 1 & 2173 \\
    1972 & unsportsmanlike\_jj & 1 & 2222 \\
    1973 & acrimonious\_jj & 1 & 2358 \\
    1974 & thunderous\_jj & 1 & 2569 \\
    1975 & vitriolic\_jj & 1 & 2774 \\
    1976 & defiant\_jj & 1 & 3105 \\
    1977 & testy\_jj & 1 & 5421 \\
    1978 & conciliatory\_jj & 1 & 5459 \\
    \hline
    \caption{Scores and rankings for most extreme 30 words in component \#6} \\
\end{longtable}
\begin{longtable}[!htbp]{| rlr@{.}l |}
    \hline
    \textbf{Rank} & \textbf{Word} & \multicolumn{2}{c|}{\textbf{Score}} \\
    \hline
    \endhead
    1 & assertive\_jj & -1 & -2430 \\
    2 & inhospitable\_jj & -1 & -2039 \\
    3 & distrustful\_jj & -1 & -1841 \\
    4 & conformist\_jj & -1 & -1280 \\
    5 & impervious\_jj & -1 & -1227 \\
    6 & stagnant\_jj & -1 & -1223 \\
    7 & mistrustful\_jj & -1 & -844 \\
    8 & autocratic\_jj & -1 & -588 \\
    9 & fractious\_jj & -1 & -458 \\
    10 & apathetic\_jj & -1 & -366 \\
    11 & laggard\_jj & -1 & -256 \\
    12 & sanguine\_jj & 0 & -9889 \\
    13 & reliant\_jj & 0 & -9709 \\
    14 & intransigent\_jj & 0 & -9624 \\
    15 & blase\_jj & 0 & -9578 \\
    16 & pliable\_jj & 0 & -9392 \\
    17 & spendthrift\_jj & 0 & -9265 \\
    18 & choosy\_jj & 0 & -9006 \\
    19 & hospitable\_jj & 0 & -8913 \\
    20 & docile\_jj & 0 & -8874 \\
    21 & bellicose\_jj & 0 & -8740 \\
    22 & accommodating\_jj & 0 & -8557 \\
    23 & inclement\_jj & 0 & -8545 \\
    24 & frosty\_jj & 0 & -8425 \\
    25 & indecisive\_jj & 0 & -8422 \\
    26 & fretful\_jj & 0 & -8381 \\
    27 & hidebound\_jj & 0 & -8369 \\
    28 & clement\_jj & 0 & -8360 \\
    29 & wary\_jj & 0 & -8319 \\
    30 & ungovernable\_jj & 0 & -8282 \\
    1949 & cruelty & 0 & 8960 \\
    1950 & bitch & 0 & 9067 \\
    1951 & genius & 0 & 9097 \\
    1952 & loving\_jj & 0 & 9120 \\
    1953 & systematic\_jj & 0 & 9322 \\
    1954 & unreserved\_jj & 0 & 9414 \\
    1955 & derogatory\_jj & 0 & 9416 \\
    1956 & autistic\_jj & 0 & 9430 \\
    1957 & unbiased\_jj & 0 & 9509 \\
    1958 & forcible\_jj & 0 & 9535 \\
    1959 & cry-baby & 0 & 9590 \\
    1960 & teachable\_jj & 0 & 9652 \\
    1961 & considerate\_jj & 0 & 9709 \\
    1962 & fraudulent\_jj & 0 & 9796 \\
    1963 & musical\_jj & 0 & 9894 \\
    1964 & humorous\_jj & 1 & 187 \\
    1965 & comedian & 1 & 215 \\
    1966 & quitter & 1 & 455 \\
    1967 & disorderly\_jj & 1 & 455 \\
    1968 & malicious\_jj & 1 & 790 \\
    1969 & obscene\_jj & 1 & 858 \\
    1970 & sincere\_jj & 1 & 927 \\
    1971 & negligent\_jj & 1 & 1017 \\
    1972 & liar & 1 & 1151 \\
    1973 & negligence & 1 & 1209 \\
    1974 & slanderous\_jj & 1 & 1977 \\
    1975 & defamatory\_jj & 1 & 4250 \\
    1976 & lascivious\_jj & 1 & 4920 \\
    1977 & lewd\_jj & 1 & 6329 \\
    1978 & savant & 2 & 1705 \\
    \hline
    \caption{Scores and rankings for most extreme 30 words in component \#7} \\
\end{longtable}
\begin{longtable}[!htbp]{| rlr@{.}l |}
    \hline
    \textbf{Rank} & \textbf{Word} & \multicolumn{2}{c|}{\textbf{Score}} \\
    \hline
    \endhead
    1 & irritability & -2 & -4937 \\
    2 & irritable\_jj & -1 & -7585 \\
    3 & overactive\_jj & -1 & -7086 \\
    4 & obstructive\_jj & -1 & -6798 \\
    5 & lethargy & -1 & -6542 \\
    6 & autistic\_jj & -1 & -6043 \\
    7 & impulsive\_jj & -1 & -5492 \\
    8 & aplastic\_jj & -1 & -5130 \\
    9 & forgetfulness & -1 & -3064 \\
    10 & lascivious\_jj & -1 & -3005 \\
    11 & cognitive\_jj & -1 & -2942 \\
    12 & disorderly\_jj & -1 & -2925 \\
    13 & acute\_jj & -1 & -2773 \\
    14 & antisocial\_jj & -1 & -2529 \\
    15 & compulsive\_jj & -1 & -2498 \\
    16 & cerebral\_jj & -1 & -2472 \\
    17 & savant & -1 & -2265 \\
    18 & refractory\_jj & -1 & -1864 \\
    19 & forcible\_jj & -1 & -1742 \\
    20 & abusive\_jj & -1 & -1627 \\
    21 & affective\_jj & -1 & -1277 \\
    22 & inhuman\_jj & -1 & -1191 \\
    23 & cruelty & -1 & -944 \\
    24 & erratic\_jj & -1 & -884 \\
    25 & severe\_jj & -1 & -872 \\
    26 & uncontrolled\_jj & -1 & -817 \\
    27 & possessive\_jj & -1 & -309 \\
    28 & inhibition & -1 & -160 \\
    29 & alcoholic & 0 & -9783 \\
    30 & self-esteem & 0 & -9594 \\
    1949 & buzzy\_jj & 0 & 7203 \\
    1950 & prophetic\_jj & 0 & 7220 \\
    1951 & squeamish\_jj & 0 & 7295 \\
    1952 & gourmet & 0 & 7300 \\
    1953 & grandiloquent\_jj & 0 & 7394 \\
    1954 & tightwad & 0 & 7450 \\
    1955 & sop & 0 & 7494 \\
    1956 & coy\_jj & 0 & 7567 \\
    1957 & pithy\_jj & 0 & 7647 \\
    1958 & fanciful\_jj & 0 & 7648 \\
    1959 & cursory\_jj & 0 & 7658 \\
    1960 & heretical\_jj & 0 & 7730 \\
    1961 & highbrow\_jj & 0 & 7795 \\
    1962 & lukewarm\_jj & 0 & 7853 \\
    1963 & wishful\_jj & 0 & 8102 \\
    1964 & fuddy-duddy & 0 & 8179 \\
    1965 & cogent\_jj & 0 & 8205 \\
    1966 & churlish\_jj & 0 & 8420 \\
    1967 & risque\_jj & 0 & 8526 \\
    1968 & unread\_jj & 0 & 8712 \\
    1969 & facetious\_jj & 0 & 8811 \\
    1970 & do-nothing & 0 & 9081 \\
    1971 & quitter & 0 & 9266 \\
    1972 & wishy-washy\_jj & 0 & 9283 \\
    1973 & sophistic\_jj & 0 & 9452 \\
    1974 & remiss\_jj & 0 & 9724 \\
    1975 & cautionary\_jj & 0 & 9980 \\
    1976 & concise\_jj & 1 & 77 \\
    1977 & highfalutin\_jj & 1 & 83 \\
    1978 & spendthrift & 1 & 1351 \\
    \hline
    \caption{Scores and rankings for most extreme 30 words in component \#8} \\
\end{longtable}
\begin{longtable}[!htbp]{| rlr@{.}l |}
    \hline
    \textbf{Rank} & \textbf{Word} & \multicolumn{2}{c|}{\textbf{Score}} \\
    \hline
    \endhead
    1 & irritability & -1 & -8033 \\
    2 & vegetative\_jj & -1 & -7198 \\
    3 & abstinent\_jj & -1 & -5738 \\
    4 & irritable\_jj & -1 & -3820 \\
    5 & cognitive\_jj & -1 & -2743 \\
    6 & forgetfulness & -1 & -2557 \\
    7 & affective\_jj & -1 & -2036 \\
    8 & lethargy & -1 & -1944 \\
    9 & aplastic\_jj & -1 & -1723 \\
    10 & remiss\_jj & -1 & -1570 \\
    11 & overactive\_jj & -1 & -1551 \\
    12 & malignant\_jj & -1 & -635 \\
    13 & modifiable\_jj & -1 & -60 \\
    14 & obsessive & 0 & -9560 \\
    15 & refractory\_jj & 0 & -9527 \\
    16 & self-esteem & 0 & -9515 \\
    17 & mild\_jj & 0 & -9363 \\
    18 & spendthrift & 0 & -9343 \\
    19 & nervous\_jj & 0 & -9189 \\
    20 & autistic\_jj & 0 & -9138 \\
    21 & nagging\_jj & 0 & -9078 \\
    22 & obstructive\_jj & 0 & -8936 \\
    23 & prayerful\_jj & 0 & -8814 \\
    24 & maternal\_jj & 0 & -8798 \\
    25 & tight-lipped\_jj & 0 & -8771 \\
    26 & testy\_jj & 0 & -8666 \\
    27 & compulsive & 0 & -8524 \\
    28 & candid\_jj & 0 & -8400 \\
    29 & hypersensitive\_jj & 0 & -8317 \\
    30 & shyness & 0 & -8167 \\
    1949 & rugged\_jj & 0 & 7896 \\
    1950 & archaic\_jj & 0 & 7925 \\
    1951 & despotic\_jj & 0 & 7970 \\
    1952 & manipulative\_jj & 0 & 7986 \\
    1953 & deceitful\_jj & 0 & 8087 \\
    1954 & sophisticated\_jj & 0 & 8150 \\
    1955 & rapacious\_jj & 0 & 8276 \\
    1956 & ingenious\_jj & 0 & 8281 \\
    1957 & monopolistic\_jj & 0 & 8312 \\
    1958 & dictatorial\_jj & 0 & 8339 \\
    1959 & cruel\_jj & 0 & 8443 \\
    1960 & chic\_jj & 0 & 8514 \\
    1961 & venal\_jj & 0 & 8628 \\
    1962 & eclectic\_jj & 0 & 8651 \\
    1963 & inefficient\_jj & 0 & 8735 \\
    1964 & inventive\_jj & 0 & 8769 \\
    1965 & wanton\_jj & 0 & 8973 \\
    1966 & unscrupulous\_jj & 0 & 9027 \\
    1967 & devious\_jj & 0 & 9272 \\
    1968 & lawless\_jj & 0 & 9363 \\
    1969 & ruthless\_jj & 0 & 9363 \\
    1970 & brazen\_jj & 0 & 9516 \\
    1971 & cunning\_jj & 0 & 9772 \\
    1972 & intricate\_jj & 0 & 9827 \\
    1973 & barbarous\_jj & 1 & 906 \\
    1974 & inhumane\_jj & 1 & 924 \\
    1975 & elegant\_jj & 1 & 1219 \\
    1976 & undemocratic\_jj & 1 & 1464 \\
    1977 & cowardly\_jj & 1 & 2106 \\
    1978 & inhuman\_jj & 1 & 2277 \\
    \hline
    \caption{Scores and rankings for most extreme 30 words in component \#9} \\
\end{longtable}
\begin{longtable}[!htbp]{| rlr@{.}l |}
    \hline
    \textbf{Rank} & \textbf{Word} & \multicolumn{2}{c|}{\textbf{Score}} \\
    \hline
    \endhead
    1 & lascivious\_jj & -1 & -5622 \\
    2 & secular\_jj & -1 & -5444 \\
    3 & celibate & -1 & -4354 \\
    4 & religious\_jj & -1 & -4269 \\
    5 & defamatory\_jj & -1 & -3157 \\
    6 & devout\_jj & -1 & -3134 \\
    7 & nonreligious\_jj & -1 & -2509 \\
    8 & blasphemous\_jj & -1 & -2375 \\
    9 & forcible\_jj & -1 & -2217 \\
    10 & peaceful\_jj & -1 & -1869 \\
    11 & ultraconservative\_jj & -1 & -1195 \\
    12 & prayerful\_jj & -1 & -1132 \\
    13 & heretical\_jj & -1 & -1034 \\
    14 & lewd\_jj & -1 & -1029 \\
    15 & solemn\_jj & -1 & -1000 \\
    16 & inhuman\_jj & -1 & -702 \\
    17 & satanic\_jj & -1 & -681 \\
    18 & tolerant\_jj & -1 & -431 \\
    19 & respectful\_jj & -1 & -256 \\
    20 & monastic\_jj & -1 & -169 \\
    21 & loving\_jj & -1 & -83 \\
    22 & pious\_jj & 0 & -9863 \\
    23 & cordial\_jj & 0 & -9665 \\
    24 & irreligious\_jj & 0 & -9570 \\
    25 & derogatory\_jj & 0 & -9537 \\
    26 & warm\_jj & 0 & -9452 \\
    27 & godless\_jj & 0 & -9437 \\
    28 & puritanical\_jj & 0 & -9394 \\
    29 & sincere\_jj & 0 & -9325 \\
    30 & civilized\_jj & 0 & -9129 \\
    1949 & analytical\_jj & 0 & 7004 \\
    1950 & undisciplined\_jj & 0 & 7008 \\
    1951 & improviser & 0 & 7123 \\
    1952 & irascible\_jj & 0 & 7296 \\
    1953 & accurate\_jj & 0 & 7333 \\
    1954 & overconfident\_jj & 0 & 7377 \\
    1955 & assured\_jj & 0 & 7507 \\
    1956 & defensive\_jj & 0 & 7579 \\
    1957 & opportunist & 0 & 7627 \\
    1958 & egghead & 0 & 7639 \\
    1959 & astute\_jj & 0 & 7673 \\
    1960 & unerring\_jj & 0 & 7705 \\
    1961 & jack-of-all-trades & 0 & 7798 \\
    1962 & erratic\_jj & 0 & 7801 \\
    1963 & dependability & 0 & 7904 \\
    1964 & canny\_jj & 0 & 7911 \\
    1965 & absent-minded\_jj & 0 & 7914 \\
    1966 & unguarded\_jj & 0 & 7968 \\
    1967 & inexperienced\_jj & 0 & 8152 \\
    1968 & unreliable\_jj & 0 & 8254 \\
    1969 & hard-nosed\_jj & 0 & 8336 \\
    1970 & balky\_jj & 0 & 8883 \\
    1971 & accomplished\_jj & 0 & 9086 \\
    1972 & cunning & 0 & 9120 \\
    1973 & indestructible\_jj & 0 & 9312 \\
    1974 & imperturbable\_jj & 0 & 9331 \\
    1975 & mathematical\_jj & 0 & 9371 \\
    1976 & adroit\_jj & 0 & 9430 \\
    1977 & inexact\_jj & 1 & 59 \\
    1978 & savant & 1 & 840 \\
    \hline
    \caption{Scores and rankings for most extreme 30 words in component \#10} \\
\end{longtable}
\begin{longtable}[!htbp]{| rlr@{.}l |}
    \hline
    \textbf{Rank} & \textbf{Word} & \multicolumn{2}{c|}{\textbf{Score}} \\
    \hline
    \endhead
    1 & ultraconservative\_jj & -1 & -4972 \\
    2 & defamatory\_jj & -1 & -4833 \\
    3 & uncooperative\_jj & -1 & -3063 \\
    4 & savant & -1 & -2749 \\
    5 & exclusive\_jj & -1 & -2429 \\
    6 & outspoken\_jj & -1 & -2277 \\
    7 & unguarded\_jj & -1 & -2102 \\
    8 & acrimonious\_jj & -1 & -1349 \\
    9 & eclectic\_jj & -1 & -965 \\
    10 & oblique\_jj & -1 & -919 \\
    11 & itinerant & -1 & -784 \\
    12 & avid\_jj & -1 & -463 \\
    13 & obscene\_jj & 0 & -9903 \\
    14 & explicit\_jj & 0 & -9782 \\
    15 & autistic\_jj & 0 & -9569 \\
    16 & amicable\_jj & 0 & -9147 \\
    17 & antagonistic\_jj & 0 & -9039 \\
    18 & accomplished\_jj & 0 & -8939 \\
    19 & independent\_jj & 0 & -8880 \\
    20 & offhand\_jj & 0 & -8877 \\
    21 & encyclopedic\_jj & 0 & -8829 \\
    22 & exhaustive\_jj & 0 & -8425 \\
    23 & blasphemous\_jj & 0 & -8402 \\
    24 & derogatory\_jj & 0 & -8371 \\
    25 & indeterminate\_jj & 0 & -8333 \\
    26 & alcoholic & 0 & -8266 \\
    27 & abusive\_jj & 0 & -8221 \\
    28 & animated\_jj & 0 & -8136 \\
    29 & irreverent\_jj & 0 & -8021 \\
    30 & authoritative\_jj & 0 & -8008 \\
    1949 & fearless\_jj & 0 & 7023 \\
    1950 & single-minded\_jj & 0 & 7045 \\
    1951 & valiant\_jj & 0 & 7049 \\
    1952 & sloth & 0 & 7120 \\
    1953 & steady\_jj & 0 & 7137 \\
    1954 & merciful\_jj & 0 & 7174 \\
    1955 & relentless\_jj & 0 & 7264 \\
    1956 & cold-blooded\_jj & 0 & 7271 \\
    1957 & precipitous\_jj & 0 & 7361 \\
    1958 & brute\_jj & 0 & 7465 \\
    1959 & empathy & 0 & 7528 \\
    1960 & callous\_jj & 0 & 7602 \\
    1961 & methodical\_jj & 0 & 7743 \\
    1962 & selfishness & 0 & 7917 \\
    1963 & quitter & 0 & 7951 \\
    1964 & cunning & 0 & 7993 \\
    1965 & doer & 0 & 8421 \\
    1966 & persistence & 0 & 8566 \\
    1967 & optimism & 0 & 8619 \\
    1968 & deliberate\_jj & 0 & 8644 \\
    1969 & decisiveness & 0 & 8763 \\
    1970 & brave\_jj & 0 & 9085 \\
    1971 & generosity & 0 & 9379 \\
    1972 & selfless\_jj & 0 & 9566 \\
    1973 & recklessness & 0 & 9950 \\
    1974 & dignity & 0 & 9950 \\
    1975 & stupidity & 1 & 377 \\
    1976 & courage & 1 & 894 \\
    1977 & cowardly\_jj & 1 & 1581 \\
    1978 & cheater & 1 & 4946 \\
    \hline
    \caption{Scores and rankings for most extreme 30 words in component \#11} \\
\end{longtable}
\begin{longtable}[!htbp]{| rlr@{.}l |}
    \hline
    \textbf{Rank} & \textbf{Word} & \multicolumn{2}{c|}{\textbf{Score}} \\
    \hline
    \endhead
    1 & morbid\_jj & -1 & -3189 \\
    2 & macabre\_jj & 0 & -9844 \\
    3 & mystical\_jj & 0 & -9661 \\
    4 & rebellious\_jj & 0 & -9512 \\
    5 & downright\_jj & 0 & -8888 \\
    6 & bleak\_jj & 0 & -8534 \\
    7 & dispiriting\_jj & 0 & -8494 \\
    8 & risqué\_jj & 0 & -8379 \\
    9 & cry-baby & 0 & -8328 \\
    10 & risque\_jj & 0 & -8319 \\
    11 & raunchy\_jj & 0 & -8314 \\
    12 & lawless\_jj & 0 & -8305 \\
    13 & materialistic\_jj & 0 & -8156 \\
    14 & curious\_jj & 0 & -8092 \\
    15 & musical\_jj & 0 & -7924 \\
    16 & chaste\_jj & 0 & -7780 \\
    17 & mundane\_jj & 0 & -7759 \\
    18 & instability & 0 & -7713 \\
    19 & savant & 0 & -7681 \\
    20 & nomadic\_jj & 0 & -7672 \\
    21 & self-conscious\_jj & 0 & -7547 \\
    22 & violent\_jj & 0 & -7513 \\
    23 & die-hard\_jj & 0 & -7341 \\
    24 & rancorous\_jj & 0 & -7312 \\
    25 & satanic\_jj & 0 & -7310 \\
    26 & otherworldly\_jj & 0 & -7290 \\
    27 & utopian\_jj & 0 & -7170 \\
    28 & fatalistic\_jj & 0 & -7045 \\
    29 & madcap\_jj & 0 & -7006 \\
    30 & cautionary\_jj & 0 & -7006 \\
    1949 & punctual\_jj & 0 & 7677 \\
    1950 & incorruptible\_jj & 0 & 7690 \\
    1951 & earthy\_jj & 0 & 7751 \\
    1952 & uncooperative\_jj & 0 & 7757 \\
    1953 & unbending\_jj & 0 & 7824 \\
    1954 & pert\_jj & 0 & 7858 \\
    1955 & transparent\_jj & 0 & 7917 \\
    1956 & tender\_jj & 0 & 8181 \\
    1957 & hard-shell\_jj & 0 & 8249 \\
    1958 & hearty\_jj & 0 & 8526 \\
    1959 & absent-minded\_jj & 0 & 8585 \\
    1960 & earthiness & 0 & 8601 \\
    1961 & comforter & 0 & 8731 \\
    1962 & pungent\_jj & 0 & 8919 \\
    1963 & fruit & 0 & 9220 \\
    1964 & gooey\_jj & 0 & 9464 \\
    1965 & clam & 0 & 9509 \\
    1966 & unrefined\_jj & 0 & 9740 \\
    1967 & prompt\_jj & 0 & 9877 \\
    1968 & expeditious\_jj & 1 & 664 \\
    1969 & impartial\_jj & 1 & 745 \\
    1970 & unctuous\_jj & 1 & 2371 \\
    1971 & boneless\_jj & 1 & 2819 \\
    1972 & unsportsmanlike\_jj & 1 & 3057 \\
    1973 & oily\_jj & 1 & 3760 \\
    1974 & broiler & 1 & 4724 \\
    1975 & buttery\_jj & 1 & 4724 \\
    1976 & peppery\_jj & 1 & 4940 \\
    1977 & gingery\_jj & 1 & 5128 \\
    1978 & vinegary\_jj & 1 & 7869 \\
    \hline
    \caption{Scores and rankings for most extreme 30 words in component \#12} \\
\end{longtable}
\begin{longtable}[!htbp]{| rlr@{.}l |}
    \hline
    \textbf{Rank} & \textbf{Word} & \multicolumn{2}{c|}{\textbf{Score}} \\
    \hline
    \endhead
    1 & coercive\_jj & -1 & -625 \\
    2 & deliberative\_jj & 0 & -9841 \\
    3 & humane\_jj & 0 & -9638 \\
    4 & expeditious\_jj & 0 & -9432 \\
    5 & vegetative\_jj & 0 & -9159 \\
    6 & bloodless\_jj & 0 & -8987 \\
    7 & corrective\_jj & 0 & -8845 \\
    8 & thorough\_jj & 0 & -8803 \\
    9 & disciplinarian & 0 & -8714 \\
    10 & forcible\_jj & 0 & -8626 \\
    11 & mechanistic\_jj & 0 & -8602 \\
    12 & methodical\_jj & 0 & -8565 \\
    13 & rigorous\_jj & 0 & -8354 \\
    14 & low-pressure\_jj & 0 & -8349 \\
    15 & high-strung\_jj & 0 & -8269 \\
    16 & stringent\_jj & 0 & -7974 \\
    17 & lascivious\_jj & 0 & -7828 \\
    18 & gentle-hearted\_jj & 0 & -7771 \\
    19 & libidinous\_jj & 0 & -7628 \\
    20 & painstaking\_jj & 0 & -7614 \\
    21 & concise\_jj & 0 & -7577 \\
    22 & barbarous\_jj & 0 & -7355 \\
    23 & mousy\_jj & 0 & -7170 \\
    24 & objector & 0 & -7156 \\
    25 & inhuman\_jj & 0 & -7132 \\
    26 & cold-blooded\_jj & 0 & -7067 \\
    27 & coquette & 0 & -7043 \\
    28 & lenient\_jj & 0 & -7029 \\
    29 & warlike\_jj & 0 & -6980 \\
    30 & dilatory\_jj & 0 & -6954 \\
    1949 & humor & 0 & 7470 \\
    1950 & rudeness & 0 & 7671 \\
    1951 & unbridled\_jj & 0 & 7682 \\
    1952 & kind\_jj & 0 & 7692 \\
    1953 & ignorant\_jj & 0 & 7882 \\
    1954 & die-hard\_jj & 0 & 7888 \\
    1955 & pessimism & 0 & 8017 \\
    1956 & broiler & 0 & 8032 \\
    1957 & sugary\_jj & 0 & 8114 \\
    1958 & gingery\_jj & 0 & 8135 \\
    1959 & fruit & 0 & 8366 \\
    1960 & warmth & 0 & 8445 \\
    1961 & irritability & 0 & 8639 \\
    1962 & earthy\_jj & 0 & 8716 \\
    1963 & gooey\_jj & 0 & 8722 \\
    1964 & eclectic\_jj & 0 & 8743 \\
    1965 & generosity & 0 & 9089 \\
    1966 & downright\_jj & 0 & 9123 \\
    1967 & mushy\_jj & 0 & 9216 \\
    1968 & acrid\_jj & 0 & 9237 \\
    1969 & buttery\_jj & 0 & 9242 \\
    1970 & pungent\_jj & 0 & 9246 \\
    1971 & avid\_jj & 0 & 9303 \\
    1972 & vinegary\_jj & 0 & 9470 \\
    1973 & optimism & 1 & 174 \\
    1974 & sour\_jj & 1 & 580 \\
    1975 & ham & 1 & 691 \\
    1976 & stupidity & 1 & 1037 \\
    1977 & savant & 1 & 1710 \\
    1978 & peppery\_jj & 1 & 2480 \\
    \hline
    \caption{Scores and rankings for most extreme 30 words in component \#13} \\
\end{longtable}
\begin{longtable}[!htbp]{| rlr@{.}l |}
    \hline
    \textbf{Rank} & \textbf{Word} & \multicolumn{2}{c|}{\textbf{Score}} \\
    \hline
    \endhead
    1 & imprudent\_jj & -1 & -1635 \\
    2 & risqué\_jj & -1 & -280 \\
    3 & ostentatious\_jj & -1 & -1 \\
    4 & chic\_jj & 0 & -9444 \\
    5 & unsportsmanlike\_jj & 0 & -9073 \\
    6 & luxurious\_jj & 0 & -8879 \\
    7 & ladylike\_jj & 0 & -8796 \\
    8 & immodest\_jj & 0 & -8550 \\
    9 & unfailing\_jj & 0 & -8451 \\
    10 & abusive\_jj & 0 & -8417 \\
    11 & modesty & 0 & -8384 \\
    12 & lenient\_jj & 0 & -8337 \\
    13 & lavish\_jj & 0 & -8319 \\
    14 & lewd\_jj & 0 & -8307 \\
    15 & unreasonable\_jj & 0 & -8115 \\
    16 & lax\_jj & 0 & -8074 \\
    17 & disorderly\_jj & 0 & -8033 \\
    18 & womanly\_jj & 0 & -7839 \\
    19 & unrestrained\_jj & 0 & -7558 \\
    20 & dignity & 0 & -7495 \\
    21 & lascivious\_jj & 0 & -7491 \\
    22 & extravagant\_jj & 0 & -7489 \\
    23 & generosity & 0 & -7441 \\
    24 & demure\_jj & 0 & -7437 \\
    25 & unshakable\_jj & 0 & -7424 \\
    26 & flashy\_jj & 0 & -7360 \\
    27 & leniency & 0 & -7322 \\
    28 & coquette & 0 & -7246 \\
    29 & remiss\_jj & 0 & -7193 \\
    30 & unreserved\_jj & 0 & -7140 \\
    1949 & divisive\_jj & 0 & 6980 \\
    1950 & insightful\_jj & 0 & 6999 \\
    1951 & mathematical\_jj & 0 & 7061 \\
    1952 & unctuous\_jj & 0 & 7200 \\
    1953 & pundit & 0 & 7238 \\
    1954 & cheater & 0 & 7244 \\
    1955 & flammable\_jj & 0 & 7329 \\
    1956 & cold-blooded\_jj & 0 & 7511 \\
    1957 & refractory\_jj & 0 & 7586 \\
    1958 & crusty\_jj & 0 & 7589 \\
    1959 & volcanic\_jj & 0 & 7618 \\
    1960 & fiery\_jj & 0 & 7853 \\
    1961 & dissident\_jj & 0 & 7950 \\
    1962 & boneless\_jj & 0 & 8074 \\
    1963 & vicious\_jj & 0 & 8118 \\
    1964 & oily\_jj & 0 & 8201 \\
    1965 & migratory\_jj & 0 & 8269 \\
    1966 & gingery\_jj & 0 & 8372 \\
    1967 & caustic\_jj & 0 & 8685 \\
    1968 & aplastic\_jj & 0 & 8714 \\
    1969 & tender\_jj & 0 & 8931 \\
    1970 & pungent\_jj & 0 & 9046 \\
    1971 & hard-boiled\_jj & 0 & 9119 \\
    1972 & liar & 0 & 9189 \\
    1973 & variant\_jj & 0 & 9429 \\
    1974 & malignant\_jj & 1 & 114 \\
    1975 & broiler & 1 & 138 \\
    1976 & peppery\_jj & 1 & 600 \\
    1977 & venomous\_jj & 1 & 767 \\
    1978 & poisonous\_jj & 1 & 1582 \\
    \hline
    \caption{Scores and rankings for most extreme 30 words in component \#14} \\
\end{longtable}
\begin{longtable}[!htbp]{| rlr@{.}l |}
    \hline
    \textbf{Rank} & \textbf{Word} & \multicolumn{2}{c|}{\textbf{Score}} \\
    \hline
    \endhead
    1 & unguarded\_jj & -1 & -2558 \\
    2 & objector & -1 & -1845 \\
    3 & intrepid\_jj & -1 & -1630 \\
    4 & indeterminate\_jj & -1 & -1335 \\
    5 & unfaithful\_jj & -1 & -309 \\
    6 & inhuman\_jj & 0 & -9932 \\
    7 & indefinite\_jj & 0 & -9519 \\
    8 & amicable\_jj & 0 & -9504 \\
    9 & unreserved\_jj & 0 & -8893 \\
    10 & heroic\_jj & 0 & -8621 \\
    11 & unwary\_jj & 0 & -8293 \\
    12 & angler & 0 & -8278 \\
    13 & samaritan & 0 & -8268 \\
    14 & immutable\_jj & 0 & -8170 \\
    15 & lifeless\_jj & 0 & -8120 \\
    16 & inhospitable\_jj & 0 & -8113 \\
    17 & otherworldly\_jj & 0 & -8083 \\
    18 & wanton & 0 & -7916 \\
    19 & inhumane\_jj & 0 & -7900 \\
    20 & abandoned\_jj & 0 & -7674 \\
    21 & immovable\_jj & 0 & -7619 \\
    22 & indomitable\_jj & 0 & -7574 \\
    23 & migratory\_jj & 0 & -7445 \\
    24 & cautionary\_jj & 0 & -7364 \\
    25 & butterfly & 0 & -7102 \\
    26 & torturous\_jj & 0 & -7051 \\
    27 & unkind\_jj & 0 & -6993 \\
    28 & prophetic\_jj & 0 & -6989 \\
    29 & tender\_jj & 0 & -6981 \\
    30 & exhaustive\_jj & 0 & -6908 \\
    1949 & flexible\_jj & 0 & 6847 \\
    1950 & sophistication & 0 & 6853 \\
    1951 & rigorous\_jj & 0 & 6865 \\
    1952 & feminine\_jj & 0 & 6889 \\
    1953 & traditional\_jj & 0 & 6950 \\
    1954 & mannish\_jj & 0 & 7018 \\
    1955 & pundit & 0 & 7031 \\
    1956 & stringent\_jj & 0 & 7196 \\
    1957 & irritable\_jj & 0 & 7262 \\
    1958 & rigid\_jj & 0 & 7296 \\
    1959 & aplastic\_jj & 0 & 7404 \\
    1960 & clownish\_jj & 0 & 7665 \\
    1961 & chic\_jj & 0 & 7787 \\
    1962 & pragmatic\_jj & 0 & 7807 \\
    1963 & masculine\_jj & 0 & 7899 \\
    1964 & progressive\_jj & 0 & 7903 \\
    1965 & resistive\_jj & 0 & 8095 \\
    1966 & secular\_jj & 0 & 8522 \\
    1967 & staunch\_jj & 0 & 8564 \\
    1968 & refractory\_jj & 0 & 8685 \\
    1969 & compulsive\_jj & 0 & 8718 \\
    1970 & folksy\_jj & 0 & 8744 \\
    1971 & cerebral\_jj & 0 & 9268 \\
    1972 & firebrand & 0 & 9702 \\
    1973 & lethargy & 1 & 0 \\
    1974 & affective\_jj & 1 & 14 \\
    1975 & cognitive\_jj & 1 & 221 \\
    1976 & moderate\_jj & 1 & 807 \\
    1977 & obstructive\_jj & 1 & 1071 \\
    1978 & irritability & 1 & 4301 \\
    \hline
    \caption{Scores and rankings for most extreme 30 words in component \#15} \\
\end{longtable}

\subsection{Normalized PCA}
\label{app:rankedwordlists:2797and438and101words:normalized}
\begin{table}[tbp]
    \begin{tabular}{| rlr@{.}l | rlr@{.}l |}
    \hline
    \textbf{Rank} & \textbf{Word} & \multicolumn{2}{c|}{\textbf{Score}} & \textbf{Rank} & \textbf{Word} & \multicolumn{2}{c|}{\textbf{Score}} \\
    \hline
    1 & mousy\_jj & -11 & 3047    &    1978 & stringent\_jj & 11 & 5947 \\
    2 & vivacious\_jj & -10 & 8877    &    1977 & indirect\_jj & 10 & 9176 \\
    3 & self-possessed\_jj & -10 & 815    &    1976 & beneficial\_jj & 10 & 9016 \\
    4 & go-getter & -9 & 9885    &    1975 & contentious\_jj & 10 & 5380 \\
    5 & girlish\_jj & -9 & 9223    &    1974 & dependent\_jj & 10 & 4100 \\
    6 & sardonic\_jj & -9 & 8553    &    1973 & reasonable\_jj & 10 & 3263 \\
    7 & guileless\_jj & -9 & 8385    &    1972 & autonomous\_jj & 10 & 2758 \\
    8 & sassy\_jj & -9 & 5201    &    1971 & lax\_jj & 10 & 2406 \\
    9 & impish\_jj & -9 & 5172    &    1970 & prudent\_jj & 10 & 2283 \\
    10 & coquette & -9 & 5115    &    1969 & corrective\_jj & 10 & 1941 \\
    11 & tomboy & -9 & 4868    &    1968 & volatile\_jj & 10 & 1895 \\
    12 & coquettish\_jj & -9 & 4435    &    1967 & discretionary\_jj & 10 & 1521 \\
    13 & stoic & -9 & 3931    &    1966 & affected\_jj & 10 & 89 \\
    14 & boyish\_jj & -9 & 586    &    1965 & fraudulent\_jj & 9 & 8758 \\
    15 & debonair\_jj & -9 & 77    &    1964 & exclusive\_jj & 9 & 8588 \\
    16 & witty\_jj & -9 & 5    &    1963 & drastic\_jj & 9 & 8415 \\
    17 & bubbly\_jj & -8 & 9992    &    1962 & confidential\_jj & 9 & 7812 \\
    18 & puckish\_jj & -8 & 9828    &    1961 & instability & 9 & 7792 \\
    19 & flirtatious\_jj & -8 & 9665    &    1960 & systematic\_jj & 9 & 7649 \\
    20 & misanthropic\_jj & -8 & 8838    &    1959 & indefinite\_jj & 9 & 6326 \\
    21 & beatific\_jj & -8 & 7844    &    1958 & stable\_jj & 9 & 5247 \\
    22 & shrewish\_jj & -8 & 7550    &    1957 & forward-looking\_jj & 9 & 4600 \\
    23 & bumpkin & -8 & 6756    &    1956 & unfair\_jj & 9 & 4347 \\
    24 & suave\_jj & -8 & 6514    &    1955 & negligence & 9 & 4343 \\
    25 & improviser & -8 & 6483    &    1954 & critical\_jj & 9 & 4059 \\
    26 & diffident\_jj & -8 & 6211    &    1953 & rigorous\_jj & 9 & 3645 \\
    27 & roguish\_jj & -8 & 5920    &    1952 & cooperation & 9 & 3374 \\
    28 & easygoing\_jj & -8 & 5080    &    1951 & volatility & 9 & 2724 \\
    29 & brassy\_jj & -8 & 4645    &    1950 & consistent\_jj & 9 & 2508 \\
    30 & priggish\_jj & -8 & 4409    &    1949 & inaccurate\_jj & 9 & 1656 \\
    \hline
    \end{tabular}
    \caption{Scores and rankings for most extreme 30 words in component \#1} 
\end{table}
\clearpage
\begin{table}[tbp]
    \begin{tabular}{| rlr@{.}l | rlr@{.}l |}
    \hline
    \textbf{Rank} & \textbf{Word} & \multicolumn{2}{c|}{\textbf{Score}} & \textbf{Rank} & \textbf{Word} & \multicolumn{2}{c|}{\textbf{Score}} \\
    \hline
    1 & defamatory\_jj & -13 & 4116    &    1978 & warm\_jj & 10 & 5553 \\
    2 & bigoted\_jj & -12 & 9998    &    1977 & elegant\_jj & 10 & 96 \\
    3 & cowardly\_jj & -12 & 4866    &    1976 & vibrant\_jj & 9 & 5522 \\
    4 & untruthful\_jj & -12 & 4505    &    1975 & savant & 9 & 5071 \\
    5 & deceitful\_jj & -12 & 2820    &    1974 & sociable\_jj & 9 & 4002 \\
    6 & inhuman\_jj & -12 & 1651    &    1973 & luxurious\_jj & 9 & 3554 \\
    7 & slanderous\_jj & -11 & 9676    &    1972 & vivacious\_jj & 9 & 2449 \\
    8 & selfish\_jj & -11 & 7386    &    1971 & graceful\_jj & 8 & 9938 \\
    9 & irresponsible\_jj & -11 & 3608    &    1970 & sparkling\_jj & 8 & 9666 \\
    10 & hypocritical\_jj & -11 & 2452    &    1969 & airy\_jj & 8 & 6906 \\
    11 & unethical\_jj & -10 & 8945    &    1968 & polished\_jj & 8 & 6718 \\
    12 & vindictive\_jj & -10 & 7508    &    1967 & buttery\_jj & 8 & 5388 \\
    13 & dishonest\_jj & -10 & 7492    &    1966 & sunny\_jj & 8 & 5160 \\
    14 & thoughtless\_jj & -10 & 7133    &    1965 & versatile\_jj & 8 & 4905 \\
    15 & polemic\_jj & -10 & 3803    &    1964 & easygoing\_jj & 8 & 4166 \\
    16 & mendacious\_jj & -10 & 3638    &    1963 & lively\_jj & 8 & 3405 \\
    17 & godless\_jj & -10 & 3198    &    1962 & intuitive\_jj & 8 & 2589 \\
    18 & callous\_jj & -10 & 3035    &    1961 & sultry\_jj & 8 & 1329 \\
    19 & disrespectful\_jj & -10 & 2872    &    1960 & sensual\_jj & 7 & 8292 \\
    20 & insensitive\_jj & -10 & 2814    &    1959 & bubbly\_jj & 7 & 7994 \\
    21 & inconsiderate\_jj & -10 & 1649    &    1958 & cultured\_jj & 7 & 7730 \\
    22 & unfeeling\_jj & -10 & 1640    &    1957 & calm\_jj & 7 & 7627 \\
    23 & callousness & -10 & 1246    &    1956 & brisk\_jj & 7 & 6862 \\
    24 & intolerant\_jj & -10 & 1205    &    1955 & bright\_jj & 7 & 6468 \\
    25 & inhumane\_jj & -10 & 335    &    1954 & rugged\_jj & 7 & 5943 \\
    26 & selfishness & -10 & 257    &    1953 & dainty\_jj & 7 & 5869 \\
    27 & stupidity & -9 & 9130    &    1952 & pleasant\_jj & 7 & 5442 \\
    28 & narrow-minded\_jj & -9 & 8678    &    1951 & ethereal\_jj & 7 & 5017 \\
    29 & unprincipled\_jj & -9 & 6858    &    1950 & serene\_jj & 7 & 4608 \\
    30 & ignorant\_jj & -9 & 6420    &    1949 & musical\_jj & 7 & 3736 \\
    \hline
    \end{tabular}
    \caption{Scores and rankings for most extreme 30 words in component \#2} 
\end{table}
\clearpage
\begin{table}[tbp]
    \begin{tabular}{| rlr@{.}l | rlr@{.}l |}
    \hline
    \textbf{Rank} & \textbf{Word} & \multicolumn{2}{c|}{\textbf{Score}} & \textbf{Rank} & \textbf{Word} & \multicolumn{2}{c|}{\textbf{Score}} \\
    \hline
    1 & considerate\_jj & -17 & 8337    &    1978 & gingery\_jj & 9 & 6631 \\
    2 & sociable\_jj & -15 & 4233    &    1977 & comforter & 9 & 3914 \\
    3 & courteous\_jj & -14 & 2601    &    1976 & pixy & 9 & 1247 \\
    4 & trustworthy\_jj & -13 & 6870    &    1975 & butterfly & 8 & 9837 \\
    5 & articulate\_jj & -13 & 372    &    1974 & soft-shelled\_jj & 8 & 3354 \\
    6 & open-minded\_jj & -12 & 1706    &    1973 & broiler & 8 & 3273 \\
    7 & kind\_jj & -11 & 7469    &    1972 & bendable\_jj & 8 & 2755 \\
    8 & approachable\_jj & -11 & 4373    &    1971 & boneless\_jj & 8 & 215 \\
    9 & respectful\_jj & -11 & 2963    &    1970 & low-pressure\_jj & 7 & 8824 \\
    10 & easygoing\_jj & -11 & 1302    &    1969 & vinegary\_jj & 7 & 7934 \\
    11 & thoughtful\_jj & -11 & 498    &    1968 & baneful\_jj & 7 & 6523 \\
    12 & self-confident\_jj & -10 & 9259    &    1967 & aplastic\_jj & 7 & 4337 \\
    13 & honest\_jj & -10 & 6089    &    1966 & hard-shell\_jj & 7 & 2866 \\
    14 & easy-going\_jj & -10 & 5864    &    1965 & acrid\_jj & 7 & 2057 \\
    15 & level-headed\_jj & -10 & 4849    &    1964 & clam & 7 & 415 \\
    16 & circumspect\_jj & -10 & 2173    &    1963 & yellow\_jj & 6 & 9932 \\
    17 & talkative\_jj & -10 & 904    &    1962 & lethargy & 6 & 9822 \\
    18 & down-to-earth\_jj & -9 & 8776    &    1961 & wildcat\_jj & 6 & 9296 \\
    19 & cordial\_jj & -9 & 7024    &    1960 & croaking & 6 & 9227 \\
    20 & sincere\_jj & -9 & 6711    &    1959 & catlike\_jj & 6 & 8634 \\
    21 & intelligent\_jj & -9 & 6610    &    1958 & magnetic\_jj & 6 & 8060 \\
    22 & accommodating\_jj & -9 & 5648    &    1957 & gourmand & 6 & 7768 \\
    23 & deferential\_jj & -9 & 4221    &    1956 & bristly\_jj & 6 & 7638 \\
    24 & pragmatic\_jj & -9 & 3846    &    1955 & high-hat & 6 & 7376 \\
    25 & opinionated\_jj & -9 & 3055    &    1954 & driftless\_jj & 6 & 6364 \\
    26 & perceptive\_jj & -9 & 3004    &    1953 & gooey\_jj & 6 & 6241 \\
    27 & knowledgeable\_jj & -9 & 2546    &    1952 & fossil & 6 & 4963 \\
    28 & fair-minded\_jj & -9 & 471    &    1951 & unmanly\_jj & 6 & 4686 \\
    29 & educated\_jj & -8 & 9806    &    1950 & ductile\_jj & 6 & 4343 \\
    30 & gregarious\_jj & -8 & 9701    &    1949 & dare-devil & 6 & 4313 \\
    \hline
    \end{tabular}
    \caption{Scores and rankings for most extreme 30 words in component \#3} 
\end{table}
\clearpage
\begin{table}[tbp]
    \begin{tabular}{| rlr@{.}l | rlr@{.}l |}
    \hline
    \textbf{Rank} & \textbf{Word} & \multicolumn{2}{c|}{\textbf{Score}} & \textbf{Rank} & \textbf{Word} & \multicolumn{2}{c|}{\textbf{Score}} \\
    \hline
    1 & sociable\_jj & -8 & 5892    &    1978 & expressiveness & 10 & 4366 \\
    2 & stay-at-home\_jj & -8 & 5319    &    1977 & unvarying\_jj & 9 & 9055 \\
    3 & has-been & -8 & 5025    &    1976 & limpid\_jj & 9 & 1007 \\
    4 & cheater & -8 & 4129    &    1975 & spontaneity & 8 & 9198 \\
    5 & lucky\_jj & -8 & 3987    &    1974 & meditative\_jj & 8 & 9073 \\
    6 & gambler & -8 & 1982    &    1973 & candor & 8 & 8202 \\
    7 & brat & -8 & 817    &    1972 & poetic\_jj & 8 & 7965 \\
    8 & clown & -7 & 8625    &    1971 & playfulness & 8 & 6317 \\
    9 & fan & -7 & 6801    &    1970 & naturalness & 8 & 3673 \\
    10 & loyal\_jj & -7 & 6756    &    1969 & decisiveness & 7 & 9163 \\
    11 & alcoholic & -7 & 5721    &    1968 & discursive\_jj & 7 & 8548 \\
    12 & quitter & -7 & 4596    &    1967 & earthiness & 7 & 7903 \\
    13 & angler & -7 & 3760    &    1966 & lyrical\_jj & 7 & 7115 \\
    14 & bulldog & -7 & 2605    &    1965 & oblique\_jj & 7 & 7081 \\
    15 & geezer & -7 & 2274    &    1964 & unfailing\_jj & 7 & 6090 \\
    16 & go-getter & -7 & 2019    &    1963 & abstruse\_jj & 7 & 5731 \\
    17 & samaritan & -7 & 1955    &    1962 & dissonant\_jj & 7 & 5685 \\
    18 & avid\_jj & -7 & 1651    &    1961 & deterministic\_jj & 7 & 5046 \\
    19 & intrepid\_jj & -7 & 953    &    1960 & irritability & 7 & 4777 \\
    20 & teetotaler & -7 & 84    &    1959 & incisive\_jj & 7 & 3292 \\
    21 & objector & -6 & 9462    &    1958 & callousness & 7 & 3045 \\
    22 & devout\_jj & -6 & 7581    &    1957 & subjective\_jj & 7 & 2828 \\
    23 & gregarious\_jj & -6 & 6568    &    1956 & shallowness & 7 & 2412 \\
    24 & bitch & -6 & 5143    &    1955 & didactic\_jj & 7 & 1870 \\
    25 & cranky\_jj & -6 & 5013    &    1954 & declamatory\_jj & 7 & 1635 \\
    26 & illiterate\_jj & -6 & 4873    &    1953 & expressive\_jj & 7 & 1081 \\
    27 & kind-hearted\_jj & -6 & 4347    &    1952 & abstract\_jj & 7 & 933 \\
    28 & staunch\_jj & -6 & 4310    &    1951 & unflagging\_jj & 7 & 230 \\
    29 & addicted\_jj & -6 & 4209    &    1950 & oratorical\_jj & 6 & 9591 \\
    30 & sissy & -6 & 3948    &    1949 & mechanistic\_jj & 6 & 9166 \\
    \hline
    \end{tabular}
    \caption{Scores and rankings for most extreme 30 words in component \#4} 
\end{table}
\clearpage
\begin{table}[tbp]
    \begin{tabular}{| rlr@{.}l | rlr@{.}l |}
    \hline
    \textbf{Rank} & \textbf{Word} & \multicolumn{2}{c|}{\textbf{Score}} & \textbf{Rank} & \textbf{Word} & \multicolumn{2}{c|}{\textbf{Score}} \\
    \hline
    1 & unswerving\_jj & -11 & 1557    &    1978 & defamatory\_jj & 11 & 6166 \\
    2 & unshakable\_jj & -10 & 4912    &    1977 & lewd\_jj & 10 & 7897 \\
    3 & unbending\_jj & -9 & 8090    &    1976 & lascivious\_jj & 9 & 5283 \\
    4 & ultraconservative\_jj & -9 & 4919    &    1975 & irritable\_jj & 9 & 3800 \\
    5 & unflagging\_jj & -9 & 4661    &    1974 & tasteless\_jj & 8 & 9795 \\
    6 & steadfast\_jj & -9 & 3132    &    1973 & risqué\_jj & 8 & 6063 \\
    7 & untiring\_jj & -9 & 2291    &    1972 & abusive\_jj & 8 & 3078 \\
    8 & unfailing\_jj & -9 & 1972    &    1971 & obscene\_jj & 8 & 17 \\
    9 & staunch\_jj & -8 & 8809    &    1970 & disorderly\_jj & 7 & 9202 \\
    10 & unquestioning\_jj & -8 & 7340    &    1969 & rude\_jj & 7 & 7320 \\
    11 & courage & -8 & 6379    &    1968 & lazy\_jj & 7 & 6929 \\
    12 & secular\_jj & -8 & 5413    &    1967 & gooey\_jj & 7 & 6363 \\
    13 & incorruptible\_jj & -8 & 4974    &    1966 & fussy\_jj & 7 & 6327 \\
    14 & objector & -8 & 3448    &    1965 & picky\_jj & 7 & 4075 \\
    15 & firebrand & -8 & 2922    &    1964 & buttery\_jj & 7 & 3077 \\
    16 & tireless\_jj & -8 & 1613    &    1963 & frisky\_jj & 7 & 2835 \\
    17 & theistic\_jj & -7 & 9626    &    1962 & loud\_jj & 7 & 2767 \\
    18 & intellectual & -7 & 7523    &    1961 & raunchy\_jj & 6 & 9365 \\
    19 & fervent\_jj & -7 & 7326    &    1960 & irritability & 6 & 9137 \\
    20 & dissident & -7 & 6919    &    1959 & clingy\_jj & 6 & 8689 \\
    21 & selfless\_jj & -7 & 5712    &    1958 & obtrusive\_jj & 6 & 8489 \\
    22 & generosity & -7 & 5460    &    1957 & libidinous\_jj & 6 & 8183 \\
    23 & visionary & -7 & 5148    &    1956 & inconsiderate\_jj & 6 & 7991 \\
    24 & devout\_jj & -7 & 4065    &    1955 & flammable\_jj & 6 & 6118 \\
    25 & religious\_jj & -7 & 3665    &    1954 & contrived\_jj & 6 & 5978 \\
    26 & militant & -7 & 1884    &    1953 & risque\_jj & 6 & 5646 \\
    27 & stalwart & -7 & 1377    &    1952 & inaccurate\_jj & 6 & 5149 \\
    28 & nonreligious\_jj & -7 & 1308    &    1951 & sexy\_jj & 6 & 4697 \\
    29 & indefatigable\_jj & -7 & 1287    &    1950 & oily\_jj & 6 & 3683 \\
    30 & spirit & -6 & 9057    &    1949 & unrefined\_jj & 6 & 1943 \\
    \hline
    \end{tabular}
    \caption{Scores and rankings for most extreme 30 words in component \#5} 
\end{table}
\clearpage
\begin{table}[tbp]
    \begin{tabular}{| rlr@{.}l | rlr@{.}l |}
    \hline
    \textbf{Rank} & \textbf{Word} & \multicolumn{2}{c|}{\textbf{Score}} & \textbf{Rank} & \textbf{Word} & \multicolumn{2}{c|}{\textbf{Score}} \\
    \hline
    1 & deterministic\_jj & -9 & 5721    &    1978 & conciliatory\_jj & 10 & 8749 \\
    2 & self-reliant\_jj & -9 & 1782    &    1977 & testy\_jj & 10 & 7551 \\
    3 & sociable\_jj & -9 & 951    &    1976 & defiant\_jj & 9 & 4731 \\
    4 & adaptable\_jj & -9 & 177    &    1975 & vitriolic\_jj & 8 & 9708 \\
    5 & educated\_jj & -8 & 7080    &    1974 & thunderous\_jj & 8 & 8784 \\
    6 & considerate\_jj & -8 & 4955    &    1973 & acrimonious\_jj & 8 & 8564 \\
    7 & modifiable\_jj & -8 & 3710    &    1972 & derisive\_jj & 8 & 5722 \\
    8 & intelligent\_jj & -7 & 8819    &    1971 & unsportsmanlike\_jj & 8 & 5255 \\
    9 & self-sufficient\_jj & -7 & 7833    &    1970 & stern\_jj & 8 & 4432 \\
    10 & open-minded\_jj & -7 & 7548    &    1969 & rancorous\_jj & 8 & 814 \\
    11 & sedentary\_jj & -7 & 7391    &    1968 & bellicose\_jj & 7 & 8619 \\
    12 & materialistic\_jj & -7 & 7286    &    1967 & caustic\_jj & 7 & 8505 \\
    13 & mutable\_jj & -7 & 5726    &    1966 & congratulatory\_jj & 7 & 8014 \\
    14 & choosy\_jj & -7 & 5591    &    1965 & lewd\_jj & 7 & 6123 \\
    15 & distractible\_jj & -7 & 4603    &    1964 & stormy\_jj & 7 & 5452 \\
    16 & closed-minded\_jj & -7 & 3416    &    1963 & ferocious\_jj & 7 & 5444 \\
    17 & unchangeable\_jj & -7 & 48    &    1962 & fierce\_jj & 7 & 5189 \\
    18 & conformist\_jj & -6 & 9977    &    1961 & verbal\_jj & 7 & 4903 \\
    19 & equitable\_jj & -6 & 8655    &    1960 & fiery\_jj & 7 & 2812 \\
    20 & efficient\_jj & -6 & 8026    &    1959 & torrid\_jj & 7 & 2216 \\
    21 & perspicuous\_jj & -6 & 7600    &    1958 & belligerence & 7 & 2120 \\
    22 & intuitive\_jj & -6 & 6945    &    1957 & bitter\_jj & 7 & 1720 \\
    23 & individualistic\_jj & -6 & 6538    &    1956 & raucous\_jj & 7 & 1058 \\
    24 & literate\_jj & -6 & 6375    &    1955 & combative\_jj & 7 & 942 \\
    25 & altruistic\_jj & -6 & 6272    &    1954 & rowdy\_jj & 6 & 9682 \\
    26 & variant\_jj & -6 & 6002    &    1953 & strident\_jj & 6 & 9131 \\
    27 & abstinent\_jj & -6 & 5940    &    1952 & belligerent\_jj & 6 & 8508 \\
    28 & cultured\_jj & -6 & 5149    &    1951 & lukewarm\_jj & 6 & 8395 \\
    29 & theistic\_jj & -6 & 4909    &    1950 & curt\_jj & 6 & 7263 \\
    30 & generalist & -6 & 4344    &    1949 & spirited\_jj & 6 & 6531 \\
    \hline
    \end{tabular}
    \caption{Scores and rankings for most extreme 30 words in component \#6} 
\end{table}
\clearpage
\begin{table}[tbp]
    \begin{tabular}{| rlr@{.}l | rlr@{.}l |}
    \hline
    \textbf{Rank} & \textbf{Word} & \multicolumn{2}{c|}{\textbf{Score}} & \textbf{Rank} & \textbf{Word} & \multicolumn{2}{c|}{\textbf{Score}} \\
    \hline
    1 & assertive\_jj & -8 & 9687    &    1978 & savant & 15 & 2659 \\
    2 & inhospitable\_jj & -8 & 6012    &    1977 & lewd\_jj & 11 & 1727 \\
    3 & distrustful\_jj & -8 & 5243    &    1976 & lascivious\_jj & 10 & 731 \\
    4 & stagnant\_jj & -8 & 157    &    1975 & defamatory\_jj & 9 & 9780 \\
    5 & conformist\_jj & -7 & 9455    &    1974 & slanderous\_jj & 8 & 3611 \\
    6 & impervious\_jj & -7 & 8269    &    1973 & negligence & 7 & 8789 \\
    7 & mistrustful\_jj & -7 & 6402    &    1972 & liar & 7 & 8540 \\
    8 & autocratic\_jj & -7 & 6053    &    1971 & quitter & 7 & 8139 \\
    9 & fractious\_jj & -7 & 3496    &    1970 & obscene\_jj & 7 & 6730 \\
    10 & apathetic\_jj & -7 & 3098    &    1969 & negligent\_jj & 7 & 5774 \\
    11 & sanguine\_jj & -7 & 3051    &    1968 & sincere\_jj & 7 & 5061 \\
    12 & laggard\_jj & -7 & 1472    &    1967 & malicious\_jj & 7 & 4217 \\
    13 & intransigent\_jj & -6 & 8559    &    1966 & disorderly\_jj & 7 & 3129 \\
    14 & blase\_jj & -6 & 8110    &    1965 & humorous\_jj & 7 & 2885 \\
    15 & reliant\_jj & -6 & 7964    &    1964 & comedian & 7 & 1149 \\
    16 & spendthrift\_jj & -6 & 5516    &    1963 & musical\_jj & 7 & 889 \\
    17 & bellicose\_jj & -6 & 4762    &    1962 & fraudulent\_jj & 7 & 257 \\
    18 & pliable\_jj & -6 & 4660    &    1961 & unbiased\_jj & 7 & 245 \\
    19 & docile\_jj & -6 & 4285    &    1960 & cry-baby & 6 & 9510 \\
    20 & choosy\_jj & -6 & 3633    &    1959 & teachable\_jj & 6 & 8243 \\
    21 & hospitable\_jj & -6 & 3465    &    1958 & considerate\_jj & 6 & 8002 \\
    22 & frosty\_jj & -6 & 2276    &    1957 & genius & 6 & 5379 \\
    23 & wary\_jj & -6 & 2163    &    1956 & systematic\_jj & 6 & 5341 \\
    24 & accommodating\_jj & -6 & 656    &    1955 & thorough\_jj & 6 & 4934 \\
    25 & sluggish\_jj & -6 & 337    &    1954 & autistic\_jj & 6 & 4798 \\
    26 & fretful\_jj & -5 & 9726    &    1953 & unreserved\_jj & 6 & 4325 \\
    27 & inclement\_jj & -5 & 9445    &    1952 & derogatory\_jj & 6 & 4302 \\
    28 & indecisive\_jj & -5 & 9328    &    1951 & bitch & 6 & 4285 \\
    29 & aloofness & -5 & 8720    &    1950 & loving\_jj & 6 & 3704 \\
    30 & hidebound\_jj & -5 & 8680    &    1949 & forcible\_jj & 6 & 3487 \\
    \hline
    \end{tabular}
    \caption{Scores and rankings for most extreme 30 words in component \#7} 
\end{table}
\clearpage
\begin{table}[tbp]
    \begin{tabular}{| rlr@{.}l | rlr@{.}l |}
    \hline
    \textbf{Rank} & \textbf{Word} & \multicolumn{2}{c|}{\textbf{Score}} & \textbf{Rank} & \textbf{Word} & \multicolumn{2}{c|}{\textbf{Score}} \\
    \hline
    1 & irritability & -17 & 9215    &    1978 & spendthrift & 7 & 8869 \\
    2 & irritable\_jj & -12 & 7878    &    1977 & concise\_jj & 7 & 3225 \\
    3 & overactive\_jj & -12 & 4779    &    1976 & cautionary\_jj & 7 & 1441 \\
    4 & obstructive\_jj & -12 & 317    &    1975 & highfalutin\_jj & 7 & 809 \\
    5 & autistic\_jj & -11 & 8969    &    1974 & quitter & 6 & 7692 \\
    6 & lethargy & -11 & 6709    &    1973 & remiss\_jj & 6 & 6631 \\
    7 & impulsive\_jj & -11 & 3158    &    1972 & sophistic\_jj & 6 & 5807 \\
    8 & aplastic\_jj & -10 & 8300    &    1971 & wishy-washy\_jj & 6 & 5511 \\
    9 & cognitive\_jj & -9 & 6558    &    1970 & do-nothing & 6 & 2886 \\
    10 & forgetfulness & -9 & 4533    &    1969 & unread\_jj & 6 & 2139 \\
    11 & lascivious\_jj & -9 & 4342    &    1968 & risque\_jj & 6 & 1836 \\
    12 & antisocial\_jj & -9 & 3542    &    1967 & facetious\_jj & 6 & 1641 \\
    13 & acute\_jj & -9 & 3088    &    1966 & churlish\_jj & 6 & 827 \\
    14 & disorderly\_jj & -9 & 3017    &    1965 & cogent\_jj & 5 & 9437 \\
    15 & cerebral\_jj & -9 & 1377    &    1964 & wishful\_jj & 5 & 7940 \\
    16 & compulsive\_jj & -9 & 1365    &    1963 & fuddy-duddy & 5 & 7885 \\
    17 & savant & -9 & 852    &    1962 & fanciful\_jj & 5 & 6719 \\
    18 & refractory\_jj & -8 & 6784    &    1961 & sop & 5 & 5782 \\
    19 & forcible\_jj & -8 & 4583    &    1960 & coy\_jj & 5 & 5133 \\
    20 & affective\_jj & -8 & 2700    &    1959 & lukewarm\_jj & 5 & 5062 \\
    21 & abusive\_jj & -8 & 2539    &    1958 & heretical\_jj & 5 & 4804 \\
    22 & severe\_jj & -7 & 9166    &    1957 & cursory\_jj & 5 & 4625 \\
    23 & erratic\_jj & -7 & 7999    &    1956 & highbrow\_jj & 5 & 4278 \\
    24 & inhuman\_jj & -7 & 7980    &    1955 & squeamish\_jj & 5 & 3574 \\
    25 & cruelty & -7 & 7790    &    1954 & pithy\_jj & 5 & 2290 \\
    26 & uncontrolled\_jj & -7 & 7017    &    1953 & grandiloquent\_jj & 5 & 2256 \\
    27 & alcoholic & -7 & 4398    &    1952 & backhanded\_jj & 5 & 1850 \\
    28 & possessive\_jj & -7 & 3568    &    1951 & tightwad & 5 & 1803 \\
    29 & inhibition & -7 & 2398    &    1950 & immodest\_jj & 5 & 1550 \\
    30 & sociable\_jj & -7 & 1226    &    1949 & prophetic\_jj & 5 & 1185 \\
    \hline
    \end{tabular}
    \caption{Scores and rankings for most extreme 30 words in component \#8} 
\end{table}
\clearpage
\begin{table}[tbp]
    \begin{tabular}{| rlr@{.}l | rlr@{.}l |}
    \hline
    \textbf{Rank} & \textbf{Word} & \multicolumn{2}{c|}{\textbf{Score}} & \textbf{Rank} & \textbf{Word} & \multicolumn{2}{c|}{\textbf{Score}} \\
    \hline
    1 & irritability & -12 & 4744    &    1978 & inhuman\_jj & 8 & 7606 \\
    2 & vegetative\_jj & -12 & 2325    &    1977 & cowardly\_jj & 8 & 7009 \\
    3 & abstinent\_jj & -11 & 1106    &    1976 & undemocratic\_jj & 8 & 4995 \\
    4 & irritable\_jj & -9 & 4270    &    1975 & elegant\_jj & 7 & 9955 \\
    5 & cognitive\_jj & -8 & 8898    &    1974 & barbarous\_jj & 7 & 7693 \\
    6 & forgetfulness & -8 & 6241    &    1973 & inhumane\_jj & 7 & 7090 \\
    7 & aplastic\_jj & -8 & 2952    &    1972 & intricate\_jj & 7 & 1189 \\
    8 & remiss\_jj & -8 & 2841    &    1971 & cunning\_jj & 6 & 8503 \\
    9 & affective\_jj & -8 & 2458    &    1970 & brazen\_jj & 6 & 6754 \\
    10 & lethargy & -7 & 9524    &    1969 & ruthless\_jj & 6 & 6077 \\
    11 & overactive\_jj & -7 & 8680    &    1968 & lawless\_jj & 6 & 5923 \\
    12 & malignant\_jj & -7 & 6397    &    1967 & devious\_jj & 6 & 4911 \\
    13 & modifiable\_jj & -6 & 9552    &    1966 & wanton\_jj & 6 & 4644 \\
    14 & obsessive & -6 & 7226    &    1965 & eclectic\_jj & 6 & 3953 \\
    15 & spendthrift & -6 & 5662    &    1964 & inefficient\_jj & 6 & 3352 \\
    16 & mild\_jj & -6 & 5275    &    1963 & inventive\_jj & 6 & 3036 \\
    17 & refractory\_jj & -6 & 5087    &    1962 & chic\_jj & 6 & 2260 \\
    18 & self-esteem & -6 & 4868    &    1961 & venal\_jj & 6 & 2120 \\
    19 & nervous\_jj & -6 & 4643    &    1960 & unscrupulous\_jj & 6 & 2009 \\
    20 & prayerful\_jj & -6 & 4502    &    1959 & dictatorial\_jj & 6 & 1050 \\
    21 & maternal\_jj & -6 & 4174    &    1958 & rapacious\_jj & 5 & 9346 \\
    22 & tight-lipped\_jj & -6 & 4138    &    1957 & monopolistic\_jj & 5 & 9248 \\
    23 & nagging\_jj & -6 & 3442    &    1956 & ingenious\_jj & 5 & 8860 \\
    24 & autistic\_jj & -6 & 1993    &    1955 & cruel\_jj & 5 & 8751 \\
    25 & testy\_jj & -6 & 1819    &    1954 & despotic\_jj & 5 & 7921 \\
    26 & candid\_jj & -6 & 1215    &    1953 & rugged\_jj & 5 & 7360 \\
    27 & curt\_jj & -6 & 479    &    1952 & archaic\_jj & 5 & 6443 \\
    28 & compulsive & -5 & 9994    &    1951 & sophisticated\_jj & 5 & 6242 \\
    29 & obstructive\_jj & -5 & 8267    &    1950 & manipulative\_jj & 5 & 5951 \\
    30 & circumspect\_jj & -5 & 7969    &    1949 & extravagant\_jj & 5 & 5860 \\
    \hline
    \end{tabular}
    \caption{Scores and rankings for most extreme 30 words in component \#9} 
\end{table}
\clearpage
\begin{table}[tbp]
    \begin{tabular}{| rlr@{.}l | rlr@{.}l |}
    \hline
    \textbf{Rank} & \textbf{Word} & \multicolumn{2}{c|}{\textbf{Score}} & \textbf{Rank} & \textbf{Word} & \multicolumn{2}{c|}{\textbf{Score}} \\
    \hline
    1 & lascivious\_jj & -11 & 1519    &    1978 & savant & 7 & 5758 \\
    2 & secular\_jj & -11 & 1193    &    1977 & inexact\_jj & 7 & 3410 \\
    3 & celibate & -10 & 6100    &    1976 & imperturbable\_jj & 7 & 1725 \\
    4 & religious\_jj & -10 & 2339    &    1975 & adroit\_jj & 7 & 1044 \\
    5 & devout\_jj & -9 & 2543    &    1974 & accomplished\_jj & 6 & 9586 \\
    6 & forcible\_jj & -8 & 8655    &    1973 & absent-minded\_jj & 6 & 7682 \\
    7 & nonreligious\_jj & -8 & 6910    &    1972 & balky\_jj & 6 & 6495 \\
    8 & peaceful\_jj & -8 & 6313    &    1971 & indestructible\_jj & 6 & 4526 \\
    9 & defamatory\_jj & -8 & 5780    &    1970 & mathematical\_jj & 6 & 4288 \\
    10 & blasphemous\_jj & -8 & 4830    &    1969 & cunning & 6 & 4052 \\
    11 & solemn\_jj & -8 & 1109    &    1968 & unguarded\_jj & 6 & 1870 \\
    12 & satanic\_jj & -7 & 8382    &    1967 & unreliable\_jj & 5 & 8925 \\
    13 & heretical\_jj & -7 & 8379    &    1966 & inexperienced\_jj & 5 & 8879 \\
    14 & prayerful\_jj & -7 & 8147    &    1965 & hard-nosed\_jj & 5 & 8707 \\
    15 & inhuman\_jj & -7 & 7754    &    1964 & improviser & 5 & 6998 \\
    16 & lewd\_jj & -7 & 7469    &    1963 & astute\_jj & 5 & 6943 \\
    17 & loving\_jj & -7 & 5225    &    1962 & unerring\_jj & 5 & 6623 \\
    18 & tolerant\_jj & -7 & 4155    &    1961 & canny\_jj & 5 & 6622 \\
    19 & respectful\_jj & -7 & 4133    &    1960 & dependability & 5 & 6229 \\
    20 & monastic\_jj & -7 & 3256    &    1959 & jack-of-all-trades & 5 & 5844 \\
    21 & ultraconservative\_jj & -7 & 1802    &    1958 & opportunist & 5 & 5642 \\
    22 & pious\_jj & -7 & 1005    &    1957 & irascible\_jj & 5 & 5340 \\
    23 & civilized\_jj & -6 & 8244    &    1956 & egghead & 5 & 5024 \\
    24 & warm\_jj & -6 & 7988    &    1955 & assured\_jj & 5 & 4888 \\
    25 & godless\_jj & -6 & 7866    &    1954 & erratic\_jj & 5 & 4430 \\
    26 & puritanical\_jj & -6 & 7661    &    1953 & overactive\_jj & 5 & 2679 \\
    27 & irreligious\_jj & -6 & 7406    &    1952 & accurate\_jj & 5 & 1995 \\
    28 & sincere\_jj & -6 & 7372    &    1951 & defensive\_jj & 5 & 1943 \\
    29 & carnal\_jj & -6 & 5572    &    1950 & innovative\_jj & 5 & 1654 \\
    30 & sociable\_jj & -6 & 4617    &    1949 & analytical\_jj & 5 & 1261 \\
    \hline
    \end{tabular}
    \caption{Scores and rankings for most extreme 30 words in component \#10} 
\end{table}
\clearpage
\begin{table}[tbp]
    \begin{tabular}{| rlr@{.}l | rlr@{.}l |}
    \hline
    \textbf{Rank} & \textbf{Word} & \multicolumn{2}{c|}{\textbf{Score}} & \textbf{Rank} & \textbf{Word} & \multicolumn{2}{c|}{\textbf{Score}} \\
    \hline
    1 & ultraconservative\_jj & -10 & 9000    &    1978 & cheater & 9 & 4191 \\
    2 & defamatory\_jj & -9 & 8064    &    1977 & courage & 8 & 1123 \\
    3 & savant & -9 & 5952    &    1976 & cowardly\_jj & 7 & 8223 \\
    4 & exclusive\_jj & -8 & 8551    &    1975 & recklessness & 7 & 7769 \\
    5 & acrimonious\_jj & -8 & 1393    &    1974 & decisiveness & 7 & 6991 \\
    6 & eclectic\_jj & -8 & 1011    &    1973 & dignity & 7 & 3777 \\
    7 & oblique\_jj & -7 & 8436    &    1972 & stupidity & 7 & 2380 \\
    8 & uncooperative\_jj & -7 & 7209    &    1971 & selfless\_jj & 7 & 1259 \\
    9 & autistic\_jj & -7 & 7202    &    1970 & deliberate\_jj & 6 & 5797 \\
    10 & outspoken\_jj & -7 & 6633    &    1969 & generosity & 6 & 3817 \\
    11 & unguarded\_jj & -7 & 5924    &    1968 & doer & 6 & 2456 \\
    12 & explicit\_jj & -7 & 5296    &    1967 & persistence & 6 & 1869 \\
    13 & itinerant & -7 & 4751    &    1966 & brave\_jj & 6 & 1547 \\
    14 & avid\_jj & -7 & 2383    &    1965 & cunning & 6 & 1432 \\
    15 & blasphemous\_jj & -6 & 8140    &    1964 & methodical\_jj & 5 & 9765 \\
    16 & animated\_jj & -6 & 7800    &    1963 & quitter & 5 & 8558 \\
    17 & antagonistic\_jj & -6 & 5575    &    1962 & prudent\_jj & 5 & 5580 \\
    18 & obscene\_jj & -6 & 5161    &    1961 & optimism & 5 & 4461 \\
    19 & encyclopedic\_jj & -6 & 3545    &    1960 & merciful\_jj & 5 & 3519 \\
    20 & amicable\_jj & -6 & 3171    &    1959 & speedy\_jj & 5 & 3456 \\
    21 & irreverent\_jj & -6 & 2352    &    1958 & single-minded\_jj & 5 & 3319 \\
    22 & derogatory\_jj & -6 & 1899    &    1957 & callous\_jj & 5 & 3233 \\
    23 & raunchy\_jj & -6 & 1442    &    1956 & courageous\_jj & 5 & 2812 \\
    24 & lewd\_jj & -6 & 18    &    1955 & selfishness & 5 & 2647 \\
    25 & risque\_jj & -5 & 9852    &    1954 & vinegary\_jj & 5 & 892 \\
    26 & extremist & -5 & 9251    &    1953 & reckless\_jj & 5 & 781 \\
    27 & itinerant\_jj & -5 & 8649    &    1952 & fearless\_jj & 5 & 378 \\
    28 & adulterous\_jj & -5 & 8347    &    1951 & merciless\_jj & 5 & 121 \\
    29 & alcoholic & -5 & 7478    &    1950 & cunning\_jj & 4 & 9399 \\
    30 & independent\_jj & -5 & 6582    &    1949 & brute\_jj & 4 & 9322 \\
    \hline
    \end{tabular}
    \caption{Scores and rankings for most extreme 30 words in component \#11} 
\end{table}
\clearpage
\begin{table}[tbp]
    \begin{tabular}{| rlr@{.}l | rlr@{.}l |}
    \hline
    \textbf{Rank} & \textbf{Word} & \multicolumn{2}{c|}{\textbf{Score}} & \textbf{Rank} & \textbf{Word} & \multicolumn{2}{c|}{\textbf{Score}} \\
    \hline
    1 & morbid\_jj & -9 & 4601    &    1978 & vinegary\_jj & 12 & 1394 \\
    2 & macabre\_jj & -6 & 9469    &    1977 & gingery\_jj & 10 & 6876 \\
    3 & dispiriting\_jj & -6 & 6877    &    1976 & broiler & 10 & 5296 \\
    4 & rebellious\_jj & -6 & 5657    &    1975 & peppery\_jj & 10 & 2953 \\
    5 & bleak\_jj & -6 & 4079    &    1974 & buttery\_jj & 10 & 2315 \\
    6 & mystical\_jj & -6 & 3862    &    1973 & unsportsmanlike\_jj & 9 & 8529 \\
    7 & downright\_jj & -5 & 9983    &    1972 & oily\_jj & 9 & 7581 \\
    8 & materialistic\_jj & -5 & 9757    &    1971 & boneless\_jj & 8 & 4178 \\
    9 & cry-baby & -5 & 7355    &    1970 & unctuous\_jj & 8 & 3637 \\
    10 & curious\_jj & -5 & 7296    &    1969 & impartial\_jj & 7 & 9544 \\
    11 & insecurity & -5 & 6821    &    1968 & expeditious\_jj & 7 & 8876 \\
    12 & nagging\_jj & -5 & 6104    &    1967 & uncooperative\_jj & 7 & 7013 \\
    13 & lawless\_jj & -5 & 5419    &    1966 & unrefined\_jj & 6 & 9898 \\
    14 & hectic\_jj & -5 & 5152    &    1965 & prompt\_jj & 6 & 6347 \\
    15 & fickle\_jj & -5 & 4608    &    1964 & pungent\_jj & 6 & 3876 \\
    16 & self-conscious\_jj & -5 & 3926    &    1963 & defamatory\_jj & 6 & 3633 \\
    17 & cheater & -5 & 3925    &    1962 & clam & 6 & 2641 \\
    18 & whirlwind & -5 & 2967    &    1961 & noncompliant\_jj & 6 & 2609 \\
    19 & fatalistic\_jj & -5 & 2842    &    1960 & gooey\_jj & 6 & 2253 \\
    20 & vicious\_jj & -5 & 2498    &    1959 & earthy\_jj & 6 & 1570 \\
    21 & conceit & -5 & 2189    &    1958 & fruit & 6 & 982 \\
    22 & instability & -5 & 1950    &    1957 & earthiness & 6 & 888 \\
    23 & mundane\_jj & -5 & 1428    &    1956 & peremptory\_jj & 6 & 776 \\
    24 & musical\_jj & -5 & 1257    &    1955 & absent-minded\_jj & 6 & 563 \\
    25 & madcap\_jj & -5 & 1086    &    1954 & unreserved\_jj & 6 & 197 \\
    26 & shrinking & -5 & 801    &    1953 & tender\_jj & 5 & 7162 \\
    27 & nomadic\_jj & -5 & 528    &    1952 & astringent\_jj & 5 & 7131 \\
    28 & otherworldly\_jj & -4 & 9596    &    1951 & slanderous\_jj & 5 & 6474 \\
    29 & lethargy & -4 & 9374    &    1950 & unbending\_jj & 5 & 6210 \\
    30 & profound\_jj & -4 & 9358    &    1949 & objector & 5 & 5956 \\
    \hline
    \end{tabular}
    \caption{Scores and rankings for most extreme 30 words in component \#12} 
\end{table}
\clearpage
\begin{table}[tbp]
    \begin{tabular}{| rlr@{.}l | rlr@{.}l |}
    \hline
    \textbf{Rank} & \textbf{Word} & \multicolumn{2}{c|}{\textbf{Score}} & \textbf{Rank} & \textbf{Word} & \multicolumn{2}{c|}{\textbf{Score}} \\
    \hline
    1 & coercive\_jj & -7 & 4229    &    1978 & peppery\_jj & 9 & 7738 \\
    2 & humane\_jj & -6 & 8194    &    1977 & savant & 8 & 2934 \\
    3 & deliberative\_jj & -6 & 5842    &    1976 & sour\_jj & 8 & 1301 \\
    4 & thorough\_jj & -6 & 4707    &    1975 & ham & 7 & 8978 \\
    5 & expeditious\_jj & -6 & 3679    &    1974 & stupidity & 7 & 6015 \\
    6 & vegetative\_jj & -6 & 3636    &    1973 & vinegary\_jj & 7 & 5225 \\
    7 & bloodless\_jj & -6 & 3337    &    1972 & buttery\_jj & 7 & 3265 \\
    8 & forcible\_jj & -6 & 2497    &    1971 & mushy\_jj & 7 & 2387 \\
    9 & corrective\_jj & -6 & 2420    &    1970 & pungent\_jj & 7 & 1814 \\
    10 & disciplinarian & -6 & 2017    &    1969 & gooey\_jj & 6 & 9006 \\
    11 & high-strung\_jj & -6 & 941    &    1968 & optimism & 6 & 8428 \\
    12 & methodical\_jj & -6 & 742    &    1967 & acrid\_jj & 6 & 7335 \\
    13 & mechanistic\_jj & -5 & 9827    &    1966 & avid\_jj & 6 & 6630 \\
    14 & rigorous\_jj & -5 & 9392    &    1965 & earthy\_jj & 6 & 5757 \\
    15 & gentle-hearted\_jj & -5 & 8862    &    1964 & gingery\_jj & 6 & 5423 \\
    16 & low-pressure\_jj & -5 & 8425    &    1963 & fruit & 6 & 5228 \\
    17 & stringent\_jj & -5 & 7130    &    1962 & broiler & 6 & 5208 \\
    18 & lascivious\_jj & -5 & 6073    &    1961 & sugary\_jj & 6 & 5132 \\
    19 & libidinous\_jj & -5 & 5370    &    1960 & downright\_jj & 6 & 4422 \\
    20 & inhuman\_jj & -5 & 4595    &    1959 & eclectic\_jj & 6 & 3984 \\
    21 & barbarous\_jj & -5 & 4481    &    1958 & generosity & 6 & 1652 \\
    22 & painstaking\_jj & -5 & 3260    &    1957 & irritability & 6 & 1246 \\
    23 & coquette & -5 & 2044    &    1956 & warmth & 5 & 8727 \\
    24 & mousy\_jj & -5 & 1943    &    1955 & rudeness & 5 & 6467 \\
    25 & sadistic\_jj & -5 & 1331    &    1954 & hearty\_jj & 5 & 6193 \\
    26 & cold-blooded\_jj & -5 & 1255    &    1953 & earthiness & 5 & 5946 \\
    27 & concise\_jj & -5 & 956    &    1952 & oily\_jj & 5 & 5741 \\
    28 & warlike\_jj & -5 & 106    &    1951 & die-hard\_jj & 5 & 5460 \\
    29 & objector & -4 & 9946    &    1950 & exclusive\_jj & 5 & 5016 \\
    30 & lenient\_jj & -4 & 9454    &    1949 & kind\_jj & 5 & 4859 \\
    \hline
    \end{tabular}
    \caption{Scores and rankings for most extreme 30 words in component \#13} 
\end{table}
\clearpage
\begin{table}[tbp]
    \begin{tabular}{| rlr@{.}l | rlr@{.}l |}
    \hline
    \textbf{Rank} & \textbf{Word} & \multicolumn{2}{c|}{\textbf{Score}} & \textbf{Rank} & \textbf{Word} & \multicolumn{2}{c|}{\textbf{Score}} \\
    \hline
    1 & imprudent\_jj & -8 & 5449    &    1978 & poisonous\_jj & 8 & 3019 \\
    2 & ostentatious\_jj & -7 & 4563    &    1977 & venomous\_jj & 7 & 8738 \\
    3 & risqué\_jj & -7 & 4407    &    1976 & peppery\_jj & 7 & 450 \\
    4 & chic\_jj & -6 & 9243    &    1975 & broiler & 7 & 147 \\
    5 & unsportsmanlike\_jj & -6 & 8242    &    1974 & malignant\_jj & 6 & 8629 \\
    6 & ladylike\_jj & -6 & 3133    &    1973 & variant\_jj & 6 & 7033 \\
    7 & unreasonable\_jj & -6 & 2734    &    1972 & liar & 6 & 5705 \\
    8 & abusive\_jj & -6 & 2100    &    1971 & hard-boiled\_jj & 6 & 5227 \\
    9 & luxurious\_jj & -6 & 1552    &    1970 & pungent\_jj & 6 & 3955 \\
    10 & modesty & -6 & 1473    &    1969 & tender\_jj & 6 & 2299 \\
    11 & unfailing\_jj & -6 & 203    &    1968 & caustic\_jj & 6 & 1657 \\
    12 & lenient\_jj & -5 & 9901    &    1967 & vicious\_jj & 5 & 9239 \\
    13 & lavish\_jj & -5 & 9577    &    1966 & aplastic\_jj & 5 & 9236 \\
    14 & lax\_jj & -5 & 9318    &    1965 & dissident\_jj & 5 & 9074 \\
    15 & immodest\_jj & -5 & 8688    &    1964 & migratory\_jj & 5 & 8961 \\
    16 & disorderly\_jj & -5 & 7779    &    1963 & cold-blooded\_jj & 5 & 8710 \\
    17 & womanly\_jj & -5 & 5616    &    1962 & cheater & 5 & 6353 \\
    18 & lewd\_jj & -5 & 5492    &    1961 & volcanic\_jj & 5 & 5819 \\
    19 & generosity & -5 & 4516    &    1960 & mathematical\_jj & 5 & 5281 \\
    20 & flashy\_jj & -5 & 4499    &    1959 & oily\_jj & 5 & 4915 \\
    21 & extravagant\_jj & -5 & 4202    &    1958 & boneless\_jj & 5 & 4391 \\
    22 & thrift & -5 & 3984    &    1957 & fiery\_jj & 5 & 4081 \\
    23 & unshakable\_jj & -5 & 3428    &    1956 & gingery\_jj & 5 & 3861 \\
    24 & unrestrained\_jj & -5 & 3343    &    1955 & flammable\_jj & 5 & 2268 \\
    25 & demure\_jj & -5 & 2702    &    1954 & dissident & 5 & 1721 \\
    26 & unreserved\_jj & -5 & 2377    &    1953 & perceptive\_jj & 5 & 1526 \\
    27 & remiss\_jj & -5 & 2306    &    1952 & divisive\_jj & 5 & 1259 \\
    28 & dignity & -5 & 1940    &    1951 & concise\_jj & 5 & 429 \\
    29 & leniency & -5 & 1253    &    1950 & crusty\_jj & 4 & 9951 \\
    30 & flexibility & -5 & 831    &    1949 & refractory\_jj & 4 & 9909 \\
    \hline
    \end{tabular}
    \caption{Scores and rankings for most extreme 30 words in component \#14} 
\end{table}
\clearpage
\begin{table}[tbp]
    \begin{tabular}{| rlr@{.}l | rlr@{.}l |}
    \hline
    \textbf{Rank} & \textbf{Word} & \multicolumn{2}{c|}{\textbf{Score}} & \textbf{Rank} & \textbf{Word} & \multicolumn{2}{c|}{\textbf{Score}} \\
    \hline
    1 & unguarded\_jj & -9 & 6324    &    1978 & irritability & 10 & 5842 \\
    2 & indeterminate\_jj & -8 & 2664    &    1977 & cognitive\_jj & 7 & 8927 \\
    3 & objector & -8 & 1489    &    1976 & lethargy & 7 & 5675 \\
    4 & lifeless\_jj & -7 & 5760    &    1975 & obstructive\_jj & 7 & 5618 \\
    5 & inhuman\_jj & -7 & 3569    &    1974 & affective\_jj & 7 & 3190 \\
    6 & intrepid\_jj & -7 & 3041    &    1973 & moderate\_jj & 7 & 1220 \\
    7 & indefinite\_jj & -7 & 2915    &    1972 & firebrand & 6 & 9706 \\
    8 & unfaithful\_jj & -7 & 2825    &    1971 & compulsive\_jj & 6 & 8533 \\
    9 & impassive\_jj & -6 & 4889    &    1970 & risqué\_jj & 6 & 2206 \\
    10 & immovable\_jj & -6 & 4623    &    1969 & gourmet & 5 & 9767 \\
    11 & amicable\_jj & -6 & 3631    &    1968 & resistive\_jj & 5 & 8711 \\
    12 & otherworldly\_jj & -6 & 2837    &    1967 & cerebral\_jj & 5 & 7805 \\
    13 & immutable\_jj & -6 & 2117    &    1966 & refractory\_jj & 5 & 7184 \\
    14 & cautionary\_jj & -5 & 8305    &    1965 & staunch\_jj & 5 & 7074 \\
    15 & butterfly & -5 & 8027    &    1964 & folksy\_jj & 5 & 6819 \\
    16 & abandoned\_jj & -5 & 7885    &    1963 & pragmatic\_jj & 5 & 5873 \\
    17 & heroic\_jj & -5 & 7644    &    1962 & refined\_jj & 5 & 5002 \\
    18 & bleak\_jj & -5 & 6737    &    1961 & forgetfulness & 5 & 3841 \\
    19 & indomitable\_jj & -5 & 6627    &    1960 & sedentary\_jj & 5 & 2862 \\
    20 & wanton & -5 & 5577    &    1959 & clownish\_jj & 5 & 2834 \\
    21 & unreserved\_jj & -5 & 5496    &    1958 & progressive\_jj & 5 & 2794 \\
    22 & prophetic\_jj & -5 & 5160    &    1957 & secular\_jj & 5 & 2091 \\
    23 & unwary\_jj & -5 & 4809    &    1956 & rigorous\_jj & 5 & 1034 \\
    24 & emotionless\_jj & -5 & 3850    &    1955 & chic\_jj & 4 & 9542 \\
    25 & inhumane\_jj & -5 & 3517    &    1954 & stringent\_jj & 4 & 9265 \\
    26 & unkind\_jj & -5 & 3354    &    1953 & risque\_jj & 4 & 9099 \\
    27 & samaritan & -5 & 2655    &    1952 & disorganization & 4 & 8982 \\
    28 & angelic\_jj & -5 & 2564    &    1951 & aplastic\_jj & 4 & 8873 \\
    29 & inexhaustible\_jj & -5 & 9    &    1950 & pundit & 4 & 8710 \\
    30 & elusive\_jj & -4 & 9242    &    1949 & traditional\_jj & 4 & 7903 \\
    \hline
    \end{tabular}
    \caption{Scores and rankings for most extreme 30 words in component \#15} 
\end{table}
\clearpage

\subsection{MDS}
\label{app:rankedwordlists:2797and438and101words:mds}
\begin{table}[tbp]
    \begin{tabular}{| rlr@{.}l | rlr@{.}l |}
    \hline
    \textbf{Rank} & \textbf{Word} & \multicolumn{2}{c|}{\textbf{Score}} & \textbf{Rank} & \textbf{Word} & \multicolumn{2}{c|}{\textbf{Score}} \\
    \hline
    1 & gamin & 0 & 3363    &    1978 & beneficial\_jj & 0 & 4092 \\
    2 & coquettish\_jj & 0 & 3028    &    1977 & stringent\_jj & 0 & 3968 \\
    3 & tender-hearted\_jj & 0 & 2972    &    1976 & prudent\_jj & 0 & 3909 \\
    4 & kittenish\_jj & 0 & 2937    &    1975 & consistent\_jj & 0 & 3843 \\
    5 & donnish\_jj & 0 & 2909    &    1974 & contentious\_jj & 0 & 3831 \\
    6 & shrewish\_jj & 0 & 2898    &    1973 & reasonable\_jj & 0 & 3795 \\
    7 & slangy\_jj & 0 & 2804    &    1972 & direct\_jj & 0 & 3656 \\
    8 & stuck-up\_jj & 0 & 2757    &    1971 & volatile\_jj & 0 & 3599 \\
    9 & sassy\_jj & 0 & 2745    &    1970 & stable\_jj & 0 & 3565 \\
    10 & melancholic & 0 & 2732    &    1969 & dependent\_jj & 0 & 3562 \\
    11 & sly\_jj & 0 & 2727    &    1968 & productive\_jj & 0 & 3519 \\
    12 & ingenue & 0 & 2668    &    1967 & helpful\_jj & 0 & 3500 \\
    13 & brassy\_jj & 0 & 2654    &    1966 & drastic\_jj & 0 & 3496 \\
    14 & tomboy & 0 & 2621    &    1965 & confidential\_jj & 0 & 3443 \\
    15 & sardonic\_jj & 0 & 2608    &    1964 & indirect\_jj & 0 & 3437 \\
    16 & girlish\_jj & 0 & 2566    &    1963 & rigorous\_jj & 0 & 3430 \\
    17 & impish\_jj & 0 & 2557    &    1962 & affected\_jj & 0 & 3427 \\
    18 & untamable\_jj & 0 & 2527    &    1961 & systematic\_jj & 0 & 3418 \\
    19 & go-getter & 0 & 2509    &    1960 & unfair\_jj & 0 & 3380 \\
    20 & suave\_jj & 0 & 2503    &    1959 & responsible\_jj & 0 & 3359 \\
    21 & growly\_jj & 0 & 2486    &    1958 & accurate\_jj & 0 & 3349 \\
    22 & boyish\_jj & 0 & 2464    &    1957 & critical\_jj & 0 & 3343 \\
    23 & virginal\_jj & 0 & 2457    &    1956 & lax\_jj & 0 & 3341 \\
    24 & duffer & 0 & 2453    &    1955 & wary\_jj & 0 & 3287 \\
    25 & genial\_jj & 0 & 2449    &    1954 & severe\_jj & 0 & 3253 \\
    26 & coltish\_jj & 0 & 2444    &    1953 & cooperation & 0 & 3252 \\
    27 & mannish\_jj & 0 & 2441    &    1952 & inconsistent\_jj & 0 & 3243 \\
    28 & stoic & 0 & 2437    &    1951 & initiative & 0 & 3241 \\
    29 & scatterbrained\_jj & 0 & 2430    &    1950 & negative\_jj & 0 & 3233 \\
    30 & falstaffian\_jj & 0 & 2427    &    1949 & disruptive\_jj & 0 & 3231 \\
    \hline
    \end{tabular}
    \caption{Scores and rankings for most extreme 30 words in component \#1} 
\end{table}
\clearpage
\begin{table}[tbp]
    \begin{tabular}{| rlr@{.}l | rlr@{.}l |}
    \hline
    \textbf{Rank} & \textbf{Word} & \multicolumn{2}{c|}{\textbf{Score}} & \textbf{Rank} & \textbf{Word} & \multicolumn{2}{c|}{\textbf{Score}} \\
    \hline
    1 & bigoted\_jj & 0 & 3489    &    1978 & lively\_jj & 0 & 3649 \\
    2 & unethical\_jj & 0 & 3304    &    1977 & graceful\_jj & 0 & 3379 \\
    3 & stupidity & 0 & 3204    &    1976 & vibrant\_jj & 0 & 3309 \\
    4 & hypocritical\_jj & 0 & 3157    &    1975 & calm\_jj & 0 & 3277 \\
    5 & untruthful\_jj & 0 & 3105    &    1974 & elegant\_jj & 0 & 3128 \\
    6 & untransparent\_jj & 0 & 3044    &    1973 & sunny\_jj & 0 & 3093 \\
    7 & godless\_jj & 0 & 3043    &    1972 & polished\_jj & 0 & 3053 \\
    8 & dishonest\_jj & 0 & 3024    &    1971 & gentle\_jj & 0 & 3003 \\
    9 & deceitful\_jj & 0 & 3021    &    1970 & dependable\_jj & 0 & 2954 \\
    10 & gullibility & 0 & 3001    &    1969 & pleasant\_jj & 0 & 2940 \\
    11 & selfishness & 0 & 2987    &    1968 & versatile\_jj & 0 & 2899 \\
    12 & unreasoning\_jj & 0 & 2960    &    1967 & sparkling\_jj & 0 & 2874 \\
    13 & mendacious\_jj & 0 & 2946    &    1966 & serene\_jj & 0 & 2855 \\
    14 & oversensitive\_jj & 0 & 2895    &    1965 & quiet\_jj & 0 & 2853 \\
    15 & ignorant\_jj & 0 & 2872    &    1964 & brisk\_jj & 0 & 2841 \\
    16 & irresponsible\_jj & 0 & 2834    &    1963 & bright\_jj & 0 & 2827 \\
    17 & callousness & 0 & 2815    &    1962 & vivacious\_jj & 0 & 2817 \\
    18 & irrational\_jj & 0 & 2802    &    1961 & warm\_jj & 0 & 2801 \\
    19 & unfeeling\_jj & 0 & 2792    &    1960 & energetic\_jj & 0 & 2789 \\
    20 & greedy\_jj & 0 & 2792    &    1959 & buoyant\_jj & 0 & 2785 \\
    21 & gutless\_jj & 0 & 2780    &    1958 & cool\_jj & 0 & 2763 \\
    22 & misguided\_jj & 0 & 2764    &    1957 & airy\_jj & 0 & 2755 \\
    23 & muddle-headed\_jj & 0 & 2763    &    1956 & cheerful\_jj & 0 & 2684 \\
    24 & uncivilized\_jj & 0 & 2755    &    1955 & sociable\_jj & 0 & 2681 \\
    25 & thoughtless\_jj & 0 & 2755    &    1954 & easygoing\_jj & 0 & 2671 \\
    26 & unprincipled\_jj & 0 & 2747    &    1953 & flexible\_jj & 0 & 2640 \\
    27 & pig-headed\_jj & 0 & 2743    &    1952 & sultry\_jj & 0 & 2628 \\
    28 & vindictive\_jj & 0 & 2727    &    1951 & breezy\_jj & 0 & 2624 \\
    29 & egoistic\_jj & 0 & 2714    &    1950 & somber\_jj & 0 & 2604 \\
    30 & spineless\_jj & 0 & 2672    &    1949 & bubbly\_jj & 0 & 2596 \\
    \hline
    \end{tabular}
    \caption{Scores and rankings for most extreme 30 words in component \#2} 
\end{table}
\clearpage
\begin{table}[tbp]
    \begin{tabular}{| rlr@{.}l | rlr@{.}l |}
    \hline
    \textbf{Rank} & \textbf{Word} & \multicolumn{2}{c|}{\textbf{Score}} & \textbf{Rank} & \textbf{Word} & \multicolumn{2}{c|}{\textbf{Score}} \\
    \hline
    1 & thoughtful\_jj & 0 & 3389    &    1978 & soft-shelled\_jj & 0 & 3419 \\
    2 & honest\_jj & 0 & 3323    &    1977 & pixy & 0 & 3394 \\
    3 & trustworthy\_jj & 0 & 3156    &    1976 & gingery\_jj & 0 & 3111 \\
    4 & pragmatic\_jj & 0 & 3107    &    1975 & bendable\_jj & 0 & 3071 \\
    5 & open-minded\_jj & 0 & 3077    &    1974 & broiler & 0 & 3061 \\
    6 & articulate\_jj & 0 & 3050    &    1973 & butterfly & 0 & 3032 \\
    7 & cynical\_jj & 0 & 3048    &    1972 & high-hat & 0 & 2832 \\
    8 & forthright\_jj & 0 & 3028    &    1971 & low-pressure\_jj & 0 & 2811 \\
    9 & naive\_jj & 0 & 3012    &    1970 & yellow\_jj & 0 & 2796 \\
    10 & arrogant\_jj & 0 & 3011    &    1969 & magnetic\_jj & 0 & 2744 \\
    11 & considerate\_jj & 0 & 3009    &    1968 & ductile\_jj & 0 & 2713 \\
    12 & level-headed\_jj & 0 & 2964    &    1967 & dare-devil & 0 & 2710 \\
    13 & deferential\_jj & 0 & 2920    &    1966 & driftless\_jj & 0 & 2700 \\
    14 & circumspect\_jj & 0 & 2908    &    1965 & plastic\_jj & 0 & 2665 \\
    15 & courageous\_jj & 0 & 2888    &    1964 & bubbler & 0 & 2658 \\
    16 & courteous\_jj & 0 & 2869    &    1963 & outdoor\_jj & 0 & 2644 \\
    17 & combative\_jj & 0 & 2768    &    1962 & boneless\_jj & 0 & 2631 \\
    18 & self-confident\_jj & 0 & 2745    &    1961 & hard-shelled\_jj & 0 & 2631 \\
    19 & aloof\_jj & 0 & 2698    &    1960 & nonvolatile\_jj & 0 & 2620 \\
    20 & approachable\_jj & 0 & 2679    &    1959 & aplastic\_jj & 0 & 2574 \\
    21 & sympathetic\_jj & 0 & 2671    &    1958 & fruit & 0 & 2571 \\
    22 & kind\_jj & 0 & 2634    &    1957 & epicurean & 0 & 2552 \\
    23 & enthusiastic\_jj & 0 & 2620    &    1956 & comforter & 0 & 2545 \\
    24 & respectful\_jj & 0 & 2608    &    1955 & dauber & 0 & 2539 \\
    25 & accommodating\_jj & 0 & 2586    &    1954 & eruptive\_jj & 0 & 2512 \\
    26 & opinionated\_jj & 0 & 2550    &    1953 & oscillatory\_jj & 0 & 2495 \\
    27 & principled\_jj & 0 & 2549    &    1952 & fossil & 0 & 2493 \\
    28 & self-assured\_jj & 0 & 2519    &    1951 & indoor\_jj & 0 & 2478 \\
    29 & astute\_jj & 0 & 2515    &    1950 & clam & 0 & 2464 \\
    30 & passionate\_jj & 0 & 2510    &    1949 & exclusive\_jj & 0 & 2446 \\
    \hline
    \end{tabular}
    \caption{Scores and rankings for most extreme 30 words in component \#3} 
\end{table}
\clearpage
\begin{table}[tbp]
    \begin{tabular}{| rlr@{.}l | rlr@{.}l |}
    \hline
    \textbf{Rank} & \textbf{Word} & \multicolumn{2}{c|}{\textbf{Score}} & \textbf{Rank} & \textbf{Word} & \multicolumn{2}{c|}{\textbf{Score}} \\
    \hline
    1 & lucky\_jj & 0 & 2824    &    1978 & undogmatic\_jj & 0 & 2937 \\
    2 & fan & 0 & 2705    &    1977 & abstract\_jj & 0 & 2932 \\
    3 & bulldog & 0 & 2388    &    1976 & creativity & 0 & 2828 \\
    4 & clown & 0 & 2359    &    1975 & poetic\_jj & 0 & 2795 \\
    5 & rowdy\_jj & 0 & 2347    &    1974 & imaginative\_jj & 0 & 2716 \\
    6 & hapless\_jj & 0 & 2332    &    1973 & assimilative\_jj & 0 & 2665 \\
    7 & jealous\_jj & 0 & 2308    &    1972 & participative\_jj & 0 & 2644 \\
    8 & has-been & 0 & 2209    &    1971 & spontaneity & 0 & 2537 \\
    9 & bitch & 0 & 2202    &    1970 & recondite\_jj & 0 & 2523 \\
    10 & loyal\_jj & 0 & 2188    &    1969 & expressiveness & 0 & 2515 \\
    11 & gambler & 0 & 2155    &    1968 & subjective\_jj & 0 & 2488 \\
    12 & geezer & 0 & 2137    &    1967 & unconstrained\_jj & 0 & 2454 \\
    13 & crabby\_jj & 0 & 2097    &    1966 & unvarying\_jj & 0 & 2433 \\
    14 & brat & 0 & 2086    &    1965 & precision & 0 & 2388 \\
    15 & barker & 0 & 2086    &    1964 & meditative\_jj & 0 & 2331 \\
    16 & cranky\_jj & 0 & 2063    &    1963 & intricate\_jj & 0 & 2330 \\
    17 & stay-at-home\_jj & 0 & 2054    &    1962 & analytical\_jj & 0 & 2269 \\
    18 & driftless\_jj & 0 & 2052    &    1961 & mechanistic\_jj & 0 & 2255 \\
    19 & angler & 0 & 2051    &    1960 & intuitive\_jj & 0 & 2222 \\
    20 & stalwart & 0 & 2027    &    1959 & lyrical\_jj & 0 & 2214 \\
    21 & cackler & 0 & 2027    &    1958 & complex\_jj & 0 & 2209 \\
    22 & congratulatory\_jj & 0 & 2023    &    1957 & decisiveness & 0 & 2184 \\
    23 & short-tempered\_jj & 0 & 2004    &    1956 & logic & 0 & 2170 \\
    24 & inexperienced\_jj & 0 & 1997    &    1955 & understanding & 0 & 2151 \\
    25 & shy\_jj & 0 & 1935    &    1954 & mutable\_jj & 0 & 2151 \\
    26 & alcoholic & 0 & 1929    &    1953 & inquisitorial\_jj & 0 & 2143 \\
    27 & staunch\_jj & 0 & 1927    &    1952 & conversational\_jj & 0 & 2097 \\
    28 & long-suffering\_jj & 0 & 1908    &    1951 & philosophical\_jj & 0 & 2097 \\
    29 & grumpy\_jj & 0 & 1901    &    1950 & deterministic\_jj & 0 & 2085 \\
    30 & stalwart\_jj & 0 & 1885    &    1949 & incisive\_jj & 0 & 2076 \\
    \hline
    \end{tabular}
    \caption{Scores and rankings for most extreme 30 words in component \#4} 
\end{table}
\clearpage
\begin{table}[tbp]
    \begin{tabular}{| rlr@{.}l | rlr@{.}l |}
    \hline
    \textbf{Rank} & \textbf{Word} & \multicolumn{2}{c|}{\textbf{Score}} & \textbf{Rank} & \textbf{Word} & \multicolumn{2}{c|}{\textbf{Score}} \\
    \hline
    1 & intelligent\_jj & 0 & 2672    &    1978 & fierce\_jj & 0 & 2843 \\
    2 & self-sufficient\_jj & 0 & 2560    &    1977 & ferocious\_jj & 0 & 2815 \\
    3 & sociable\_jj & 0 & 2554    &    1976 & thunderous\_jj & 0 & 2803 \\
    4 & educated\_jj & 0 & 2508    &    1975 & vitriolic\_jj & 0 & 2656 \\
    5 & choosy\_jj & 0 & 2426    &    1974 & defiant\_jj & 0 & 2636 \\
    6 & picky\_jj & 0 & 2403    &    1973 & unrelenting\_jj & 0 & 2578 \\
    7 & adaptable\_jj & 0 & 2361    &    1972 & dogged\_jj & 0 & 2486 \\
    8 & beneficial\_jj & 0 & 2295    &    1971 & relentless\_jj & 0 & 2484 \\
    9 & considerate\_jj & 0 & 2280    &    1970 & acrimonious\_jj & 0 & 2430 \\
    10 & trustworthy\_jj & 0 & 2264    &    1969 & decisive\_jj & 0 & 2412 \\
    11 & sedentary\_jj & 0 & 2253    &    1968 & conciliatory\_jj & 0 & 2406 \\
    12 & efficient\_jj & 0 & 2237    &    1967 & verbal\_jj & 0 & 2382 \\
    13 & open-minded\_jj & 0 & 2227    &    1966 & fiery\_jj & 0 & 2362 \\
    14 & flexible\_jj & 0 & 2223    &    1965 & stern\_jj & 0 & 2343 \\
    15 & self-reliant\_jj & 0 & 2212    &    1964 & torrid\_jj & 0 & 2314 \\
    16 & literate\_jj & 0 & 2196    &    1963 & brutal\_jj & 0 & 2296 \\
    17 & generalist & 0 & 2172    &    1962 & bitter\_jj & 0 & 2256 \\
    18 & penny-wise\_jj & 0 & 2164    &    1961 & brief\_jj & 0 & 2247 \\
    19 & fogy & 0 & 2139    &    1960 & spirited\_jj & 0 & 2241 \\
    20 & abstinent\_jj & 0 & 2101    &    1959 & rancorous\_jj & 0 & 2236 \\
    21 & stay-at-home & 0 & 2054    &    1958 & unremitting\_jj & 0 & 2186 \\
    22 & proscriptive\_jj & 0 & 2053    &    1957 & rhetorical\_jj & 0 & 2160 \\
    23 & trustful\_jj & 0 & 2031    &    1956 & persistence & 0 & 2151 \\
    24 & knowledgeable\_jj & 0 & 2017    &    1955 & testy\_jj & 0 & 2127 \\
    25 & expensive\_jj & 0 & 2003    &    1954 & bellicose\_jj & 0 & 2118 \\
    26 & suggestible\_jj & 0 & 1994    &    1953 & strident\_jj & 0 & 2105 \\
    27 & informed\_jj & 0 & 1981    &    1952 & stubbornness & 0 & 2101 \\
    28 & fancier & 0 & 1975    &    1951 & silence & 0 & 2073 \\
    29 & modifiable\_jj & 0 & 1969    &    1950 & quick-fire\_jj & 0 & 2069 \\
    30 & reliable\_jj & 0 & 1951    &    1949 & undisguised\_jj & 0 & 2047 \\
    \hline
    \end{tabular}
    \caption{Scores and rankings for most extreme 30 words in component \#5} 
\end{table}
\clearpage
\begin{table}[tbp]
    \begin{tabular}{| rlr@{.}l | rlr@{.}l |}
    \hline
    \textbf{Rank} & \textbf{Word} & \multicolumn{2}{c|}{\textbf{Score}} & \textbf{Rank} & \textbf{Word} & \multicolumn{2}{c|}{\textbf{Score}} \\
    \hline
    1 & tasteless\_jj & 0 & 2886    &    1978 & organization & 0 & 3018 \\
    2 & dull\_jj & 0 & 2621    &    1977 & untiring\_jj & 0 & 3007 \\
    3 & contrived\_jj & 0 & 2326    &    1976 & unswerving\_jj & 0 & 2949 \\
    4 & nonsensical\_jj & 0 & 2275    &    1975 & staunch\_jj & 0 & 2864 \\
    5 & inaccurate\_jj & 0 & 2265    &    1974 & indefatigable\_jj & 0 & 2687 \\
    6 & bland\_jj & 0 & 2261    &    1973 & devout\_jj & 0 & 2683 \\
    7 & ludicrous\_jj & 0 & 2207    &    1972 & steadfast\_jj & 0 & 2664 \\
    8 & messy\_jj & 0 & 2172    &    1971 & stalwart & 0 & 2642 \\
    9 & sloppy\_jj & 0 & 2146    &    1970 & unshakable\_jj & 0 & 2605 \\
    10 & expensive\_jj & 0 & 2070    &    1969 & fervent\_jj & 0 & 2599 \\
    11 & fussy\_jj & 0 & 2025    &    1968 & tireless\_jj & 0 & 2573 \\
    12 & farcical\_jj & 0 & 2024    &    1967 & loyal\_jj & 0 & 2570 \\
    13 & innocuous\_jj & 0 & 2024    &    1966 & religious\_jj & 0 & 2460 \\
    14 & clumsy\_jj & 0 & 2015    &    1965 & courage & 0 & 2450 \\
    15 & vulgar\_jj & 0 & 2011    &    1964 & unbending\_jj & 0 & 2434 \\
    16 & self-conscious\_jj & 0 & 1983    &    1963 & incorruptible\_jj & 0 & 2402 \\
    17 & predictable\_jj & 0 & 1979    &    1962 & militant & 0 & 2378 \\
    18 & loud\_jj & 0 & 1977    &    1961 & democratic\_jj & 0 & 2351 \\
    19 & stilted\_jj & 0 & 1943    &    1960 & selfless\_jj & 0 & 2336 \\
    20 & mundane\_jj & 0 & 1937    &    1959 & intellectual & 0 & 2312 \\
    21 & lazy\_jj & 0 & 1911    &    1958 & firebrand & 0 & 2304 \\
    22 & risque\_jj & 0 & 1904    &    1957 & evangelistic\_jj & 0 & 2303 \\
    23 & dodgy\_jj & 0 & 1903    &    1956 & ultraconservative\_jj & 0 & 2289 \\
    24 & gooey\_jj & 0 & 1865    &    1955 & outspoken\_jj & 0 & 2281 \\
    25 & salacious\_jj & 0 & 1856    &    1954 & god-fearing\_jj & 0 & 2258 \\
    26 & one-sided\_jj & 0 & 1842    &    1953 & extremist & 0 & 2257 \\
    27 & imprecise\_jj & 0 & 1840    &    1952 & stout-hearted\_jj & 0 & 2243 \\
    28 & thoughtless\_jj & 0 & 1840    &    1951 & cooperation & 0 & 2229 \\
    29 & funereal\_jj & 0 & 1833    &    1950 & spirit & 0 & 2216 \\
    30 & rude\_jj & 0 & 1831    &    1949 & philanthropic\_jj & 0 & 2209 \\
    \hline
    \end{tabular}
    \caption{Scores and rankings for most extreme 30 words in component \#6} 
\end{table}
\clearpage
\begin{table}[tbp]
    \begin{tabular}{| rlr@{.}l | rlr@{.}l |}
    \hline
    \textbf{Rank} & \textbf{Word} & \multicolumn{2}{c|}{\textbf{Score}} & \textbf{Rank} & \textbf{Word} & \multicolumn{2}{c|}{\textbf{Score}} \\
    \hline
    1 & genius & 0 & 2615    &    1978 & stagnant\_jj & 0 & 2916 \\
    2 & retrospective\_jj & 0 & 2334    &    1977 & inhospitable\_jj & 0 & 2857 \\
    3 & humorous\_jj & 0 & 2310    &    1976 & exhaustible\_jj & 0 & 2529 \\
    4 & bitch & 0 & 2230    &    1975 & assertive\_jj & 0 & 2491 \\
    5 & sincere\_jj & 0 & 2165    &    1974 & fractious\_jj & 0 & 2452 \\
    6 & liar & 0 & 2140    &    1973 & impervious\_jj & 0 & 2390 \\
    7 & unbiased\_jj & 0 & 2129    &    1972 & clement\_jj & 0 & 2241 \\
    8 & malicious\_jj & 0 & 2101    &    1971 & autocratic\_jj & 0 & 2227 \\
    9 & original\_jj & 0 & 2091    &    1970 & distrustful\_jj & 0 & 2180 \\
    10 & confidential\_jj & 0 & 2089    &    1969 & mistrustful\_jj & 0 & 2165 \\
    11 & risque\_jj & 0 & 2082    &    1968 & hospitable\_jj & 0 & 2155 \\
    12 & teaser & 0 & 2059    &    1967 & reliant\_jj & 0 & 2137 \\
    13 & remiss\_jj & 0 & 2024    &    1966 & apathetic\_jj & 0 & 2083 \\
    14 & derogatory\_jj & 0 & 2016    &    1965 & conformist\_jj & 0 & 2075 \\
    15 & fraudulent\_jj & 0 & 2015    &    1964 & docile\_jj & 0 & 2053 \\
    16 & truthful\_jj & 0 & 2000    &    1963 & indifferent\_jj & 0 & 2029 \\
    17 & facetious\_jj & 0 & 1997    &    1962 & fickle\_jj & 0 & 2007 \\
    18 & giving\_jj & 0 & 1991    &    1961 & fretful\_jj & 0 & 2003 \\
    19 & objective\_jj & 0 & 1986    &    1960 & buoyant\_jj & 0 & 2003 \\
    20 & slanderous\_jj & 0 & 1985    &    1959 & intransigent\_jj & 0 & 2001 \\
    21 & insightful\_jj & 0 & 1936    &    1958 & volatile\_jj & 0 & 1971 \\
    22 & thorough\_jj & 0 & 1923    &    1957 & dominant\_jj & 0 & 1966 \\
    23 & unreserved\_jj & 0 & 1909    &    1956 & icy\_jj & 0 & 1965 \\
    24 & teachable\_jj & 0 & 1887    &    1955 & rapacious\_jj & 0 & 1955 \\
    25 & playboy & 0 & 1885    &    1954 & hurly-burly\_jj & 0 & 1922 \\
    26 & comedian & 0 & 1872    &    1953 & fearful\_jj & 0 & 1917 \\
    27 & humor & 0 & 1862    &    1952 & assimilative\_jj & 0 & 1898 \\
    28 & honest\_jj & 0 & 1849    &    1951 & temperate\_jj & 0 & 1896 \\
    29 & negligence & 0 & 1842    &    1950 & nomadic\_jj & 0 & 1889 \\
    30 & patient\_jj & 0 & 1830    &    1949 & impatient\_jj & 0 & 1876 \\
    \hline
    \end{tabular}
    \caption{Scores and rankings for most extreme 30 words in component \#7} 
\end{table}
\clearpage
\begin{table}[tbp]
    \begin{tabular}{| rlr@{.}l | rlr@{.}l |}
    \hline
    \textbf{Rank} & \textbf{Word} & \multicolumn{2}{c|}{\textbf{Score}} & \textbf{Rank} & \textbf{Word} & \multicolumn{2}{c|}{\textbf{Score}} \\
    \hline
    1 & abstinent\_jj & 0 & 2585    &    1978 & intricate\_jj & 0 & 2733 \\
    2 & lukewarm\_jj & 0 & 2372    &    1977 & ruthless\_jj & 0 & 2573 \\
    3 & remiss\_jj & 0 & 2371    &    1976 & innovative\_jj & 0 & 2380 \\
    4 & circumspect\_jj & 0 & 2180    &    1975 & audacious\_jj & 0 & 2376 \\
    5 & frosty\_jj & 0 & 2158    &    1974 & cunning\_jj & 0 & 2206 \\
    6 & congratulatory\_jj & 0 & 2129    &    1973 & devious\_jj & 0 & 2187 \\
    7 & curt\_jj & 0 & 2107    &    1972 & sophisticated\_jj & 0 & 2185 \\
    8 & prayerful\_jj & 0 & 2097    &    1971 & extravagant\_jj & 0 & 2180 \\
    9 & tight-lipped\_jj & 0 & 2088    &    1970 & inventive\_jj & 0 & 2146 \\
    10 & unenthusiastic\_jj & 0 & 2080    &    1969 & manipulative\_jj & 0 & 2122 \\
    11 & negative\_jj & 0 & 2010    &    1968 & ingenious\_jj & 0 & 2102 \\
    12 & sanguine\_jj & 0 & 1992    &    1967 & clever\_jj & 0 & 1989 \\
    13 & trustful\_jj & 0 & 1985    &    1966 & unpredictable\_jj & 0 & 1954 \\
    14 & cautious\_jj & 0 & 1975    &    1965 & ambitious\_jj & 0 & 1950 \\
    15 & adulatory\_jj & 0 & 1958    &    1964 & accomplished\_jj & 0 & 1946 \\
    16 & respectful\_jj & 0 & 1955    &    1963 & imaginative\_jj & 0 & 1928 \\
    17 & conciliatory\_jj & 0 & 1945    &    1962 & archaic\_jj & 0 & 1926 \\
    18 & nagging\_jj & 0 & 1913    &    1961 & antiquated\_jj & 0 & 1925 \\
    19 & testy\_jj & 0 & 1905    &    1960 & brutal\_jj & 0 & 1873 \\
    20 & touchy\_jj & 0 & 1887    &    1959 & predatory\_jj & 0 & 1853 \\
    21 & terse\_jj & 0 & 1860    &    1958 & eclectic\_jj & 0 & 1825 \\
    22 & pessimistic\_jj & 0 & 1830    &    1957 & inhumane\_jj & 0 & 1816 \\
    23 & nervous\_jj & 0 & 1819    &    1956 & capricious\_jj & 0 & 1815 \\
    24 & candid\_jj & 0 & 1779    &    1955 & murderous\_jj & 0 & 1809 \\
    25 & conformist & 0 & 1768    &    1954 & rugged\_jj & 0 & 1784 \\
    26 & vegetative\_jj & 0 & 1762    &    1953 & undemocratic\_jj & 0 & 1780 \\
    27 & well-disposed\_jj & 0 & 1756    &    1952 & opportunist & 0 & 1775 \\
    28 & brief\_jj & 0 & 1731    &    1951 & greedy\_jj & 0 & 1771 \\
    29 & frank\_jj & 0 & 1722    &    1950 & brazen\_jj & 0 & 1761 \\
    30 & coy\_jj & 0 & 1701    &    1949 & adaptive\_jj & 0 & 1760 \\
    \hline
    \end{tabular}
    \caption{Scores and rankings for most extreme 30 words in component \#8} 
\end{table}
\clearpage
\begin{table}[tbp]
    \begin{tabular}{| rlr@{.}l | rlr@{.}l |}
    \hline
    \textbf{Rank} & \textbf{Word} & \multicolumn{2}{c|}{\textbf{Score}} & \textbf{Rank} & \textbf{Word} & \multicolumn{2}{c|}{\textbf{Score}} \\
    \hline
    1 & grandiose\_jj & 0 & 2388    &    1978 & irritability & 0 & 3722 \\
    2 & expensive\_jj & 0 & 2255    &    1977 & forgetfulness & 0 & 3082 \\
    3 & original\_jj & 0 & 2250    &    1976 & cognitive\_jj & 0 & 2963 \\
    4 & fanciful\_jj & 0 & 2191    &    1975 & irritable\_jj & 0 & 2949 \\
    5 & lavish\_jj & 0 & 2190    &    1974 & aplastic\_jj & 0 & 2938 \\
    6 & staid\_jj & 0 & 2133    &    1973 & impulsive\_jj & 0 & 2904 \\
    7 & traditional\_jj & 0 & 1949    &    1972 & autistic\_jj & 0 & 2814 \\
    8 & fancier & 0 & 1862    &    1971 & emotional\_jj & 0 & 2765 \\
    9 & highfalutin\_jj & 0 & 1819    &    1970 & self-esteem & 0 & 2693 \\
    10 & bold\_jj & 0 & 1817    &    1969 & acute\_jj & 0 & 2690 \\
    11 & utopian\_jj & 0 & 1768    &    1968 & lethargy & 0 & 2685 \\
    12 & highbrow\_jj & 0 & 1758    &    1967 & empathy & 0 & 2662 \\
    13 & heretical\_jj & 0 & 1744    &    1966 & overactive\_jj & 0 & 2616 \\
    14 & stodgy\_jj & 0 & 1740    &    1965 & severe\_jj & 0 & 2562 \\
    15 & do-nothing & 0 & 1728    &    1964 & inhibition & 0 & 2389 \\
    16 & militaristic\_jj & 0 & 1716    &    1963 & abstinent\_jj & 0 & 2370 \\
    17 & extravagant\_jj & 0 & 1698    &    1962 & talkativeness & 0 & 2315 \\
    18 & fuddy-duddy & 0 & 1692    &    1961 & compulsive\_jj & 0 & 2312 \\
    19 & chic\_jj & 0 & 1661    &    1960 & antisocial\_jj & 0 & 2278 \\
    20 & quaint\_jj & 0 & 1640    &    1959 & shyness & 0 & 2146 \\
    21 & clerkish\_jj & 0 & 1632    &    1958 & uncontrolled\_jj & 0 & 2134 \\
    22 & risque\_jj & 0 & 1627    &    1957 & cerebral\_jj & 0 & 2106 \\
    23 & laggard\_jj & 0 & 1587    &    1956 & mild\_jj & 0 & 2093 \\
    24 & glamorous\_jj & 0 & 1587    &    1955 & insecurity & 0 & 2082 \\
    25 & anachronistic\_jj & 0 & 1583    &    1954 & cruelty & 0 & 2071 \\
    26 & doctrinaire\_jj & 0 & 1576    &    1953 & maternal\_jj & 0 & 2065 \\
    27 & bland\_jj & 0 & 1575    &    1952 & loving\_jj & 0 & 2053 \\
    28 & yes-man & 0 & 1573    &    1951 & obstructive\_jj & 0 & 2018 \\
    29 & truckling & 0 & 1567    &    1950 & persistent\_jj & 0 & 2003 \\
    30 & high-minded\_jj & 0 & 1536    &    1949 & callousness & 0 & 1965 \\
    \hline
    \end{tabular}
    \caption{Scores and rankings for most extreme 30 words in component \#9} 
\end{table}
\clearpage
\begin{table}[tbp]
    \begin{tabular}{| rlr@{.}l | rlr@{.}l |}
    \hline
    \textbf{Rank} & \textbf{Word} & \multicolumn{2}{c|}{\textbf{Score}} & \textbf{Rank} & \textbf{Word} & \multicolumn{2}{c|}{\textbf{Score}} \\
    \hline
    1 & accomplished\_jj & 0 & 2146    &    1978 & quiet\_jj & 0 & 2271 \\
    2 & adroit\_jj & 0 & 2140    &    1977 & fear & 0 & 2195 \\
    3 & outspoken\_jj & 0 & 2106    &    1976 & morality & 0 & 2185 \\
    4 & noncompliant\_jj & 0 & 2072    &    1975 & religious\_jj & 0 & 2129 \\
    5 & uncooperative\_jj & 0 & 2061    &    1974 & mystical\_jj & 0 & 2017 \\
    6 & assiduous\_jj & 0 & 2014    &    1973 & spirit & 0 & 1995 \\
    7 & improviser & 0 & 1933    &    1972 & insecurity & 0 & 1993 \\
    8 & acrimonious\_jj & 0 & 1888    &    1971 & stupidity & 0 & 1983 \\
    9 & jack-of-all-trades & 0 & 1855    &    1970 & civilized\_jj & 0 & 1963 \\
    10 & irascible\_jj & 0 & 1851    &    1969 & dignity & 0 & 1954 \\
    11 & abrasive\_jj & 0 & 1841    &    1968 & optimism & 0 & 1918 \\
    12 & cackler & 0 & 1838    &    1967 & prejudice & 0 & 1917 \\
    13 & unsportsmanlike\_jj & 0 & 1833    &    1966 & murderous\_jj & 0 & 1897 \\
    14 & offhand\_jj & 0 & 1820    &    1965 & loving\_jj & 0 & 1881 \\
    15 & nonvolatile\_jj & 0 & 1803    &    1964 & vibrant\_jj & 0 & 1810 \\
    16 & unresponsive\_jj & 0 & 1775    &    1963 & profound\_jj & 0 & 1799 \\
    17 & incisive\_jj & 0 & 1744    &    1962 & vicious\_jj & 0 & 1796 \\
    18 & expeditious\_jj & 0 & 1738    &    1961 & generosity & 0 & 1789 \\
    19 & unguarded\_jj & 0 & 1726    &    1960 & moral\_jj & 0 & 1777 \\
    20 & emphatic\_jj & 0 & 1725    &    1959 & law-abiding\_jj & 0 & 1772 \\
    21 & exhaustive\_jj & 0 & 1712    &    1958 & pleasant\_jj & 0 & 1769 \\
    22 & abrupt\_jj & 0 & 1708    &    1957 & bleak\_jj & 0 & 1748 \\
    23 & evasive\_jj & 0 & 1682    &    1956 & peaceful\_jj & 0 & 1705 \\
    24 & refractory\_jj & 0 & 1670    &    1955 & cruel\_jj & 0 & 1684 \\
    25 & unbiased\_jj & 0 & 1656    &    1954 & courage & 0 & 1669 \\
    26 & unreliable\_jj & 0 & 1642    &    1953 & devout\_jj & 0 & 1664 \\
    27 & unceremonious\_jj & 0 & 1642    &    1952 & curiosity & 0 & 1660 \\
    28 & name-dropper & 0 & 1635    &    1951 & proud\_jj & 0 & 1657 \\
    29 & inexperienced\_jj & 0 & 1632    &    1950 & ghoulish\_jj & 0 & 1653 \\
    30 & obdurate\_jj & 0 & 1621    &    1949 & wild\_jj & 0 & 1645 \\
    \hline
    \end{tabular}
    \caption{Scores and rankings for most extreme 30 words in component \#10} 
\end{table}
\clearpage
\begin{table}[tbp]
    \begin{tabular}{| rlr@{.}l | rlr@{.}l |}
    \hline
    \textbf{Rank} & \textbf{Word} & \multicolumn{2}{c|}{\textbf{Score}} & \textbf{Rank} & \textbf{Word} & \multicolumn{2}{c|}{\textbf{Score}} \\
    \hline
    1 & decisiveness & 0 & 2570    &    1978 & religious\_jj & 0 & 3017 \\
    2 & hard-nosed\_jj & 0 & 2241    &    1977 & violent\_jj & 0 & 2666 \\
    3 & persistence & 0 & 2193    &    1976 & explicit\_jj & 0 & 2538 \\
    4 & cunning & 0 & 2179    &    1975 & blasphemous\_jj & 0 & 2442 \\
    5 & sloppy\_jj & 0 & 2119    &    1974 & ultraconservative\_jj & 0 & 2297 \\
    6 & dogged\_jj & 0 & 2086    &    1973 & intimate\_jj & 0 & 2294 \\
    7 & proven\_jj & 0 & 2074    &    1972 & satanic\_jj & 0 & 2192 \\
    8 & speedy\_jj & 0 & 2014    &    1971 & informal\_jj & 0 & 2077 \\
    9 & shrewd\_jj & 0 & 2012    &    1970 & ultraconservative & 0 & 2035 \\
    10 & prudent\_jj & 0 & 1976    &    1969 & nonreligious\_jj & 0 & 2031 \\
    11 & defensive\_jj & 0 & 1960    &    1968 & eclectic\_jj & 0 & 1972 \\
    12 & gutsy\_jj & 0 & 1912    &    1967 & secular\_jj & 0 & 1961 \\
    13 & dependability & 0 & 1907    &    1966 & extremist & 0 & 1949 \\
    14 & methodical\_jj & 0 & 1871    &    1965 & derogatory\_jj & 0 & 1929 \\
    15 & efficiency & 0 & 1842    &    1964 & devout\_jj & 0 & 1902 \\
    16 & stinginess & 0 & 1824    &    1963 & observant\_jj & 0 & 1902 \\
    17 & bloody-minded\_jj & 0 & 1805    &    1962 & irreverent\_jj & 0 & 1862 \\
    18 & single-minded\_jj & 0 & 1774    &    1961 & amicable\_jj & 0 & 1861 \\
    19 & flexibility & 0 & 1759    &    1960 & austere\_jj & 0 & 1851 \\
    20 & vim & 0 & 1754    &    1959 & noisy\_jj & 0 & 1841 \\
    21 & lethargic\_jj & 0 & 1719    &    1958 & raucous\_jj & 0 & 1829 \\
    22 & consistent\_jj & 0 & 1687    &    1957 & acrimonious\_jj & 0 & 1825 \\
    23 & tenacious\_jj & 0 & 1681    &    1956 & irreligious\_jj & 0 & 1822 \\
    24 & quick-fire\_jj & 0 & 1678    &    1955 & militant & 0 & 1744 \\
    25 & overconfident\_jj & 0 & 1671    &    1954 & prudish\_jj & 0 & 1741 \\
    26 & stand-offish\_jj & 0 & 1659    &    1953 & militant\_jj & 0 & 1724 \\
    27 & decisive\_jj & 0 & 1638    &    1952 & unruly\_jj & 0 & 1710 \\
    28 & verve & 0 & 1624    &    1951 & vulgar\_jj & 0 & 1688 \\
    29 & resilient\_jj & 0 & 1615    &    1950 & licentious\_jj & 0 & 1665 \\
    30 & jack-of-all-trades & 0 & 1612    &    1949 & nonhostile\_jj & 0 & 1661 \\
    \hline
    \end{tabular}
    \caption{Scores and rankings for most extreme 30 words in component \#11} 
\end{table}
\clearpage
\begin{table}[tbp]
    \begin{tabular}{| rlr@{.}l | rlr@{.}l |}
    \hline
    \textbf{Rank} & \textbf{Word} & \multicolumn{2}{c|}{\textbf{Score}} & \textbf{Rank} & \textbf{Word} & \multicolumn{2}{c|}{\textbf{Score}} \\
    \hline
    1 & die-hard\_jj & 0 & 2065    &    1978 & buttery\_jj & 0 & 2327 \\
    2 & avid\_jj & 0 & 1925    &    1977 & dignified\_jj & 0 & 2231 \\
    3 & nagging\_jj & 0 & 1906    &    1976 & reserved\_jj & 0 & 2159 \\
    4 & upstart\_jj & 0 & 1871    &    1975 & transparent\_jj & 0 & 2143 \\
    5 & obsessive\_jj & 0 & 1814    &    1974 & gingery\_jj & 0 & 1992 \\
    6 & emotional\_jj & 0 & 1802    &    1973 & vinegary\_jj & 0 & 1970 \\
    7 & elusive\_jj & 0 & 1745    &    1972 & humane\_jj & 0 & 1903 \\
    8 & morbid\_jj & 0 & 1733    &    1971 & bristly\_jj & 0 & 1855 \\
    9 & upstart & 0 & 1729    &    1970 & expeditious\_jj & 0 & 1825 \\
    10 & conceit & 0 & 1726    &    1969 & incorruptible\_jj & 0 & 1817 \\
    11 & audacious\_jj & 0 & 1710    &    1968 & hearty\_jj & 0 & 1800 \\
    12 & acute\_jj & 0 & 1676    &    1967 & peppery\_jj & 0 & 1791 \\
    13 & anxious\_jj & 0 & 1668    &    1966 & oily\_jj & 0 & 1784 \\
    14 & fretful\_jj & 0 & 1666    &    1965 & yellow\_jj & 0 & 1773 \\
    15 & gun-shy\_jj & 0 & 1664    &    1964 & inhumane\_jj & 0 & 1767 \\
    16 & instructive\_jj & 0 & 1663    &    1963 & flowery\_jj & 0 & 1739 \\
    17 & pessimistic\_jj & 0 & 1663    &    1962 & coarse\_jj & 0 & 1735 \\
    18 & musical\_jj & 0 & 1657    &    1961 & incorrupt\_jj & 0 & 1705 \\
    19 & accomplished\_jj & 0 & 1653    &    1960 & discreet\_jj & 0 & 1672 \\
    20 & optimistic\_jj & 0 & 1639    &    1959 & peaceful\_jj & 0 & 1656 \\
    21 & ambition & 0 & 1599    &    1958 & dainty\_jj & 0 & 1654 \\
    22 & curious\_jj & 0 & 1586    &    1957 & inhuman\_jj & 0 & 1652 \\
    23 & nervous\_jj & 0 & 1583    &    1956 & warm\_jj & 0 & 1624 \\
    24 & literary\_jj & 0 & 1579    &    1955 & coercive\_jj & 0 & 1623 \\
    25 & shrinking & 0 & 1573    &    1954 & courteous\_jj & 0 & 1601 \\
    26 & highbrow\_jj & 0 & 1568    &    1953 & dignity & 0 & 1593 \\
    27 & inexact\_jj & 0 & 1554    &    1952 & protective\_jj & 0 & 1590 \\
    28 & egghead & 0 & 1544    &    1951 & fruit & 0 & 1588 \\
    29 & ambitious\_jj & 0 & 1525    &    1950 & forcible\_jj & 0 & 1587 \\
    30 & dispiriting\_jj & 0 & 1517    &    1949 & noncompliant\_jj & 0 & 1569 \\
    \hline
    \end{tabular}
    \caption{Scores and rankings for most extreme 30 words in component \#12} 
\end{table}
\clearpage
\begin{table}[tbp]
    \begin{tabular}{| rlr@{.}l | rlr@{.}l |}
    \hline
    \textbf{Rank} & \textbf{Word} & \multicolumn{2}{c|}{\textbf{Score}} & \textbf{Rank} & \textbf{Word} & \multicolumn{2}{c|}{\textbf{Score}} \\
    \hline
    1 & complicated\_jj & 0 & 2090    &    1978 & optimism & 0 & 2424 \\
    2 & complex\_jj & 0 & 2077    &    1977 & airy\_jj & 0 & 2193 \\
    3 & brutal\_jj & 0 & 2043    &    1976 & indulgent\_jj & 0 & 2171 \\
    4 & painstaking\_jj & 0 & 2042    &    1975 & generosity & 0 & 2150 \\
    5 & murderous\_jj & 0 & 1915    &    1974 & warmth & 0 & 2063 \\
    6 & methodical\_jj & 0 & 1906    &    1973 & ostentatious\_jj & 0 & 2032 \\
    7 & cold-blooded\_jj & 0 & 1862    &    1972 & unreserved\_jj & 0 & 2028 \\
    8 & bloodthirsty\_jj & 0 & 1845    &    1971 & earthy\_jj & 0 & 2017 \\
    9 & devious\_jj & 0 & 1808    &    1970 & extravagant\_jj & 0 & 1960 \\
    10 & demanding\_jj & 0 & 1796    &    1969 & stupidity & 0 & 1917 \\
    11 & mechanistic\_jj & 0 & 1789    &    1968 & unbridled\_jj & 0 & 1845 \\
    12 & ruthless\_jj & 0 & 1734    &    1967 & ambition & 0 & 1822 \\
    13 & sadistic\_jj & 0 & 1693    &    1966 & unfailing\_jj & 0 & 1792 \\
    14 & noncoercive\_jj & 0 & 1668    &    1965 & vinegary\_jj & 0 & 1781 \\
    15 & clandestine\_jj & 0 & 1663    &    1964 & peppery\_jj & 0 & 1764 \\
    16 & wily\_jj & 0 & 1630    &    1963 & giving\_jj & 0 & 1742 \\
    17 & chaotic\_jj & 0 & 1627    &    1962 & envy & 0 & 1730 \\
    18 & humane\_jj & 0 & 1616    &    1961 & ham & 0 & 1722 \\
    19 & high-strung\_jj & 0 & 1606    &    1960 & eclectic\_jj & 0 & 1720 \\
    20 & warlike\_jj & 0 & 1582    &    1959 & buttery\_jj & 0 & 1718 \\
    21 & tempestuous\_jj & 0 & 1536    &    1958 & modesty & 0 & 1686 \\
    22 & rigorous\_jj & 0 & 1518    &    1957 & unshakable\_jj & 0 & 1653 \\
    23 & thorough\_jj & 0 & 1504    &    1956 & candor & 0 & 1615 \\
    24 & cut-and-dried\_jj & 0 & 1490    &    1955 & gooey\_jj & 0 & 1614 \\
    25 & nomadic\_jj & 0 & 1462    &    1954 & excessive\_jj & 0 & 1606 \\
    26 & fiendish\_jj & 0 & 1462    &    1953 & courage & 0 & 1601 \\
    27 & dilatory\_jj & 0 & 1452    &    1952 & outspoken\_jj & 0 & 1599 \\
    28 & mathematical\_jj & 0 & 1442    &    1951 & enthusiastic\_jj & 0 & 1588 \\
    29 & blunderbuss & 0 & 1437    &    1950 & unfair\_jj & 0 & 1586 \\
    30 & homicidal\_jj & 0 & 1421    &    1949 & creativity & 0 & 1556 \\
    \hline
    \end{tabular}
    \caption{Scores and rankings for most extreme 30 words in component \#13} 
\end{table}
\clearpage
\begin{table}[tbp]
    \begin{tabular}{| rlr@{.}l | rlr@{.}l |}
    \hline
    \textbf{Rank} & \textbf{Word} & \multicolumn{2}{c|}{\textbf{Score}} & \textbf{Rank} & \textbf{Word} & \multicolumn{2}{c|}{\textbf{Score}} \\
    \hline
    1 & vocal\_jj & 0 & 2261    &    1978 & leisurely\_jj & 0 & 2859 \\
    2 & moderate\_jj & 0 & 2013    &    1977 & hectic\_jj & 0 & 2251 \\
    3 & double-faced\_jj & 0 & 1757    &    1976 & expeditious\_jj & 0 & 2094 \\
    4 & wary\_jj & 0 & 1723    &    1975 & amicable\_jj & 0 & 2058 \\
    5 & strident\_jj & 0 & 1722    &    1974 & unreserved\_jj & 0 & 2024 \\
    6 & smart\_jj & 0 & 1701    &    1973 & angler & 0 & 1916 \\
    7 & piercing\_jj & 0 & 1683    &    1972 & outdoor\_jj & 0 & 1860 \\
    8 & nervous\_jj & 0 & 1677    &    1971 & do-or-die\_jj & 0 & 1789 \\
    9 & unauthoritative\_jj & 0 & 1660    &    1970 & torturous\_jj & 0 & 1763 \\
    10 & sophisticated\_jj & 0 & 1634    &    1969 & convivial\_jj & 0 & 1729 \\
    11 & dominant\_jj & 0 & 1609    &    1968 & bacchanalian\_jj & 0 & 1720 \\
    12 & mannish\_jj & 0 & 1586    &    1967 & inhospitable\_jj & 0 & 1690 \\
    13 & fearful\_jj & 0 & 1530    &    1966 & epicurean\_jj & 0 & 1690 \\
    14 & political\_jj & 0 & 1522    &    1965 & unhurried\_jj & 0 & 1684 \\
    15 & sexy\_jj & 0 & 1504    &    1964 & orderly\_jj & 0 & 1682 \\
    16 & masculine\_jj & 0 & 1489    &    1963 & sportsmanlike\_jj & 0 & 1651 \\
    17 & staunch\_jj & 0 & 1480    &    1962 & clement\_jj & 0 & 1643 \\
    18 & conventional\_jj & 0 & 1478    &    1961 & abject\_jj & 0 & 1628 \\
    19 & skeptical\_jj & 0 & 1460    &    1960 & enterprising\_jj & 0 & 1606 \\
    20 & protective\_jj & 0 & 1403    &    1959 & dispiriting\_jj & 0 & 1603 \\
    21 & nonvolatile\_jj & 0 & 1382    &    1958 & agreeable\_jj & 0 & 1598 \\
    22 & particular\_jj & 0 & 1372    &    1957 & unsociable\_jj & 0 & 1592 \\
    23 & brash\_jj & 0 & 1362    &    1956 & indulgent\_jj & 0 & 1581 \\
    24 & precision & 0 & 1359    &    1955 & unsportsmanlike\_jj & 0 & 1556 \\
    25 & firebrand & 0 & 1351    &    1954 & pleasant\_jj & 0 & 1554 \\
    26 & fluttery\_jj & 0 & 1350    &    1953 & informal\_jj & 0 & 1548 \\
    27 & sympathetic\_jj & 0 & 1340    &    1952 & whirlwind & 0 & 1538 \\
    28 & unsure\_jj & 0 & 1337    &    1951 & derelict\_jj & 0 & 1536 \\
    29 & sensitive\_jj & 0 & 1331    &    1950 & exhaustive\_jj & 0 & 1525 \\
    30 & reflective\_jj & 0 & 1324    &    1949 & buccaneer & 0 & 1522 \\
    \hline
    \end{tabular}
    \caption{Scores and rankings for most extreme 30 words in component \#14} 
\end{table}
\clearpage
\begin{table}[tbp]
    \begin{tabular}{| rlr@{.}l | rlr@{.}l |}
    \hline
    \textbf{Rank} & \textbf{Word} & \multicolumn{2}{c|}{\textbf{Score}} & \textbf{Rank} & \textbf{Word} & \multicolumn{2}{c|}{\textbf{Score}} \\
    \hline
    1 & flexibility & 0 & 2169    &    1978 & peppery\_jj & 0 & 2557 \\
    2 & stringent\_jj & 0 & 2163    &    1977 & broiler & 0 & 2405 \\
    3 & lax\_jj & 0 & 2021    &    1976 & tender\_jj & 0 & 2265 \\
    4 & strict\_jj & 0 & 2004    &    1975 & pungent\_jj & 0 & 2200 \\
    5 & formal\_jj & 0 & 1780    &    1974 & fruit & 0 & 2133 \\
    6 & demure\_jj & 0 & 1747    &    1973 & gingery\_jj & 0 & 2118 \\
    7 & playboy & 0 & 1742    &    1972 & venomous\_jj & 0 & 2094 \\
    8 & prim\_jj & 0 & 1738    &    1971 & oily\_jj & 0 & 2080 \\
    9 & lavish\_jj & 0 & 1721    &    1970 & amnesic\_jj & 0 & 2040 \\
    10 & rigid\_jj & 0 & 1718    &    1969 & hard-shelled\_jj & 0 & 2016 \\
    11 & staid\_jj & 0 & 1717    &    1968 & fiery\_jj & 0 & 1942 \\
    12 & rigorous\_jj & 0 & 1716    &    1967 & insightful\_jj & 0 & 1885 \\
    13 & chic\_jj & 0 & 1685    &    1966 & backboned\_jj & 0 & 1872 \\
    14 & intrusive\_jj & 0 & 1655    &    1965 & migratory\_jj & 0 & 1809 \\
    15 & discreet\_jj & 0 & 1631    &    1964 & clam & 0 & 1807 \\
    16 & casual\_jj & 0 & 1629    &    1963 & cultivated\_jj & 0 & 1789 \\
    17 & lenient\_jj & 0 & 1625    &    1962 & hard-boiled\_jj & 0 & 1786 \\
    18 & coquette & 0 & 1620    &    1961 & wild\_jj & 0 & 1749 \\
    19 & unrestrained\_jj & 0 & 1610    &    1960 & tasteless\_jj & 0 & 1723 \\
    20 & corrective\_jj & 0 & 1574    &    1959 & eloquent\_jj & 0 & 1720 \\
    21 & ladylike\_jj & 0 & 1555    &    1958 & poisonous\_jj & 0 & 1711 \\
    22 & flirtatious\_jj & 0 & 1547    &    1957 & buttery\_jj & 0 & 1707 \\
    23 & ostentatious\_jj & 0 & 1533    &    1956 & boneless\_jj & 0 & 1702 \\
    24 & masculine\_jj & 0 & 1522    &    1955 & ignorant\_jj & 0 & 1688 \\
    25 & tomboy & 0 & 1497    &    1954 & temperate\_jj & 0 & 1686 \\
    26 & inflexible\_jj & 0 & 1479    &    1953 & crusty\_jj & 0 & 1683 \\
    27 & chaste\_jj & 0 & 1473    &    1952 & hearty\_jj & 0 & 1678 \\
    28 & rakish\_jj & 0 & 1459    &    1951 & mushy\_jj & 0 & 1626 \\
    29 & feminine\_jj & 0 & 1427    &    1950 & shallow\_jj & 0 & 1614 \\
    30 & traditional\_jj & 0 & 1421    &    1949 & incisive\_jj & 0 & 1599 \\
    \hline
    \end{tabular}
    \caption{Scores and rankings for most extreme 30 words in component \#15} 
\end{table}
\clearpage


