\documentclass[10pt,letterpaper]{article}
\usepackage[utf8]{inputenc}
\usepackage{amsmath}
\usepackage{amsfonts}
\usepackage{amssymb}

\author{Eric Moyer, Michael Raymer}
\title{Improving local deconvolution using summit-focused starting points}
\begin{document}

\maketitle

\begin{abstract}
There will be an abstract here
\end{abstract}

\section{Introduction}
There will be an introduction here
\section{Related work}
Most nonlinear optimization methods work by using a current model and improving the model by making small changes.
The optimization stops when no more small changes will improve the model. Because the problems being solved in
NMR curve fitting are not convex, there are many local optima where small changes will not reduce the error but 
large changes can still bring about an improvement. If the algorithm finds one of these before it finds the best
model, it will halt and return a sub-optimum model. The closer the initial model is to the global optimum, 
the less likely the optimization algorithm is to run into an inferior local optimum. Further, if good constraints
are chosen, many local optima can be eliminated from the search \textit{a-priori}, increasing the likelihood
of finding the best model first. Thus, deconvolution performance can be greatly improved by choosing a model 
closer to the global optimum and constraining the search to eliminate local optima.

Surprisingly, many authors do not mention their starting points \cite{Martin1994,Romano2002,Vanhamme2000}. 
Gipson\cite{Gipson2006} makes his fitting convex by using binning to avoid peak shift and not altering peak
width.

\subsection{Fitting using metabolite database}

Most of the rest of the authors doing curve fitting start from a database of known metabolites and
rely on this for their starting points. Poullet \textit{et. al.} start with all the metabolites unshifted at 
some low initial intensity.\cite{Poullet2007} 

Constrained total-line-shape fitting \cite{Laatikainen1996,Soininen2005} represents the metabolite database as 
a list of peaks with location and other constraints. It uses a two pass approach. In the first pass, only a few
baseline terms are used, all peaks are pure Lorentzians with equal width and initial values come directly from
the database. The second phase uses the results of the first phase as its starting point, but allows the 
line-shape parameters and widths to vary and includes more terms to approximate the baseline.

Possibly the most complex starting point among those using a metabolite database was reported by Mercier 
\textit{et. al.} \cite{Mercier2011}

\subsection{Fitting without metabolite database}

\section{Methods}
Here I will discuss my new method and my way of testing it
\section{Results}
Here I will present the results of my tests without much elaboration
\section{Discussion}
Here I will draw conclusions from my tests
\section{Conclusion}
Here I will summarize what I've said

\bibliographystyle{plain}
\bibliography{/home/eric/Bibtex/Deconvolution}

\end{document}
