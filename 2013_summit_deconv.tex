\documentclass[10pt,letterpaper]{article}
\usepackage[utf8]{inputenc}
\usepackage{amsmath}
\usepackage{amsfonts}
\usepackage{amssymb}
\author{Eric Moyer, Michael Raymer}
\title{Improving local deconvolution using summit-focused starting points}
\begin{document}

\maketitle

\begin{abstract}
There will be an abstract here
\end{abstract}

\section{Introduction}
There will be an introduction here
\section{Related work}
Spectrum analysis methods can be divided along several axes. The input data can be lightly preprocessed sensor readings in the time-domain or those readings can be passed through a Fourier transform yielding frequency domain data (usually expressed in ppm). The results can be modeled using Bayesian statistics or conventional regression. Baseline roll can be corrected in preprocessing or along with deconvolution. They can require human interaction or be fully automated.

One important distinction has to do with what level of semantics are inserted at each level of the analysis. A Weljie et. al. distinguish between chemometric and targeted profiling approaches\cite{Weljie2006}. 
\section{Methods}
Here I will discuss my new method and my way of testing it
\section{Results}
Here I will present the results of my tests without much elaboration
\section{Discussion}
Here I will draw conclusions from my tests
\section{Conclusion}
Here I will summarize what I've said

\bibliographystyle{plain}
\bibliography{/home/eric/Bibtex/Deconvolution.bib}

\end{document}
